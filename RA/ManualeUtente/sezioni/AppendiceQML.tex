\newpage
\section{QML - Quiz Markup Language}
Uno dei punti fondamentali dell'applicazione è permettere agli utenti di poter creare domande da proporre negli allenamenti e nei questionari. E' stato realizzato un nuovo linguaggio di \textit{markup\ped{G}} che permette la definizione di domande non ordinarie denominato QML, acronimo di Quiz Markup Language.

\subsection{Definizione della grammatica}
Per mantenere la semplicità di realizzazione di una domanda tramite QML è stato deciso di mantenere la sintassi \textit{JSON}, con l'aggiunta delle sole parole chiave necessarie per mantenere coerente la grammatica con lo scopo del QML. Alla grammatica è stata data possibilità di inserire commenti.

\subsubsection{Sintassi domanda Vero/Falso}
\begin{lstlisting}[language=json,firstnumber=1]
{	
"type": "veroFalso",
"image": "/img/veroFalso/example.png", //immagine possibile nel testo della domanda vero e falso
"answer":
	[{
	"text": "In Italia la guida e' a destra",
	"isItRight": true
	}],
"keywords":
	[{
	"guida",
	"legge",
	"italia"
	}] 
}
\end{lstlisting}

\subsubsection{Sintassi domanda a Risposta Multipla}
\begin{lstlisting}[language=json,firstnumber=1]
{	
"type" : "rispostaMultipla",
"questionText": "Quali di questi numeri e' pari?",
"url": "/img/rispostaMultipla/D0_3.png", //immagine possibile nel testo della domanda risposta multipla
"answer":
	[{
	"text": "1",
	"url": "/img/rispostaMultipla/D0_1.png", //url dell'immagine, possibile campo facoltativo
	"isItRight": false
	 },{
	"text": "2",
	"url": "/img/rispostaMultipla/D0_2.png", //url dell'immagine, possibile campo facoltativo
	"isItRight": true
	 },{
	"text": "7",
	"url": "/img/rispostaMultipla/D0_3.png", //url dell'immagine, possibile campo facoltativo
	"isItRight": false
	 },{
	"text": "9",
	"url": "/img/rispostaMultipla/D0_4.png", //url dell'immagine, possibile campo facoltativo
	"isItRight": false
	 }],
"keywords":
	[{
	"numeri",
	"matematica",
	"scuola"
	}]
}
\end{lstlisting}

\subsubsection{Sintassi domanda a Ordinamento di Stringhe}
\begin{lstlisting}[language=json,firstnumber=1]
{
"type": "ordinamentoStringhe",
"questionText": "Ordina questi numeri in modo decrescente.",
"answer":
	[{
	"text": "1",
	"position": 4
	},{
	"text": "2",
	"position": 3
	},{
	"text": "7",
	"position": 2
	},{
	"text": "9",
	"position": 1
	}],
"keywords":
	[{
	"ordinamento",
	"numeri",
	}]
}
\end{lstlisting}

\subsubsection{Sintassi domanda a Ordinamento Immagini}
\begin{lstlisting}[language=json,firstnumber=1]
{
"type": "ordinamentoImmagini",
"questionText": "Ordina queste immagini in modo da ottenere la parola "CIAO".",
"url": "/img/ordinamentoImmagini/D0_1.png", //immagine possibile nel testo della domanda ad ordianamento di immagini
"answer":
	[{
	"url": "/img/domandeOrdinamentoImmagini/I.png", //url dell'immagine
	"position": 2
	},{
	"url": "/img/domandeOrdinamentoImmagini/A.png",
	"position": 3
	},{
	"url": "/img/domandeOrdinamentoImmagini/O.png",
	"position": 4
	},{
	"url": "/img/domandeOrdinamentoImmagini/C.png",
	"position": 1
	}],
"keywords":
	[{
	"parole",
	"grammatica"
	}]
}
\end{lstlisting}

\subsubsection{Sintassi domanda a Collegamento di Elementi}
\begin{lstlisting}[language=json,firstnumber=1]
{
"type": "collegamentoElementi",
"questionText": "Collega queste coppie di nemici classici.",
"answer":
	[{
	"text_1_A": "cane",
	"text_1_B": "gatto"
	},{
	"url_2_A": "/img/collegamento/uncino.png",
	"text_2_B": "peter pan"
	},{
	"url_3_A": "/img/collegamento/D0_1.png",
	"url_3_B": "/img/collegamento/D0_5.png"
	}],
"keywords":
	[{
	"nemici"
	}]
}
\end{lstlisting}

\subsubsection{Sintassi domanda ad Area Cliccabile}
\begin{lstlisting}[language=json,firstnumber=1]
{
"type": "areaCliccabile",
"questionText": "Clicca quale tra le seguenti scelte e' il bicipide.",
"image": "/img/areaCliccabile/D0_1.png", //sfondo dell'area cliccabile
"resolution": { "x":400, "y":500 },
"answer":
	[{
	"x": "200",
	"y": "50",
	"text": "testo facoltativo di arricchimento"
	},{
	"x": "300",
	"y": "120",
	"text": "testo facoltativo di arricchimento"
	},{
	"x": "200",
	"y": "200",
	"text": "testo facoltativo di arricchimento"
	}],
"keywords":
	[{
	"corpo umano",
	"medicina",
	"scienze" ,
	"muscoli"
	}]
}
\end{lstlisting}

\subsubsection{Sintassi domanda a Riempimento spazi vuoti}
\begin{lstlisting}[language=json,firstnumber=1]
{
"type": "spaziVuoti",
"questionText": "Giulio Cesare era un console romano.",
"answer":
	[{
	"parolaNumero": 2 //sta ad indicare quale parola e' da oscurare. In questo caso la numero 2
	},{
	"parolaNumero": 5
	}],
"keywords":
	[{
	"storia",
	}] 
}
\end{lstlisting}

\subsection{Generazione del JSON}
Se il \textit{parser\ped{G}} non trova errori o ambiguità viene creato un \textit{JSON\ped{G}} contente la domanda e altre informazioni necessarie per aggiungerla al sistema. Ogni tipologia di domanda crea un tipo di \textit{JSON\ped{G}} diverso, questo accada sia per la generazione tramite \textit{wizard\ped{G}} sia tramite QML.

\subsubsection{JSON domanda Vero/Falso}
\begin{lstlisting}[language=json,firstnumber=1]
{
"author" : "__userId",
"makeWith" : "QML" || "wizzard" ,
"language" : "it || en",
"question" :
	[{
	"type": "veroFalso",
	"image": "/img/veroFalso/example.png", //immagine possibile nel testo della domanda vero e falso
	"answers":
		[{
		"text": "In Italia la guida e' a destra",
		"isItRight": true
		}]
	}],
"keywords" : 
	[{
	"guida",
	"legge",
	"italia"
	}],
"level" : 500,
"totalAnswers" : 0,
"correctAnswers" : 0
}
\end{lstlisting}

\subsubsection{JSON domanda a Risposta Multipla}
\begin{lstlisting}[language=json,firstnumber=1]
{	
"author" : "_UserId",
"makeWith" : "QML" || "wizzard" ,
"language" : "it || en" ,
"type": "rispostaMultipla",
"question" [{
	"questionText": "Quali di questi numeri e' pari?",
	"url": "/img/rispostaMultipla/D0_3.png", //immagine possibile nel testo della domanda risposta multipla
	"answers":
		[{
		"text": "1",
		"url": "/img/rispostaMultipla/D0_1.png", //url dell'immagine, possibile campo facoltativo
		"isItRight": false
		},{
		"text": "2",
		"url": "/img/rispostaMultipla/D0_2.png", //url dell'immagine, possibile campo facoltativo
		"isItRight": true
		},{
		"text": "7",
		"url": "/img/rispostaMultipla/D0_3.png", //url dell'immagine, possibile campo facoltativo
		"isItRight": false
		},{
		"text": "9",
		"url": "/img/rispostaMultipla/D0_4.png", //url dell'immagine, possibile campo facoltativo
		"isItRight": false
		}]
}],
"keywords":
	[{
	"numeri",
	"matematica",
	"scuola"
	}],
"level" : 500,
"totalAnswers" : 0,
"correctAnswers" : 0
}
\end{lstlisting}

\subsubsection{JSON domanda a Ordinamento di Stringhe}
\begin{lstlisting}[language=json,firstnumber=1]
{
"author" : "_UserId", 
"makeWith" : "QML" || "wizzard" ,
"language" : "it || en",
"question" : [{
	"type": "ordinamentoStringhe",
	"questionText": "Ordina questi numeri in modo decrescente.",
	"answers":
		[{
		"text": "1",
		"position": 4
		},{
		"text": "2",
		"position": 3
		},{
		"text": "7",
		"position": 2
		},{
		"text": "9",
		"position": 1
		}]
}],
"keywords":
	[{
	"ordinamento",
	"numeri",
	}],
"level" : 500,
"totalAnswers" : 0,
"correctAnswers" : 0
}
\end{lstlisting}

\subsubsection{JSON domanda a Ordinamento Immagini}
\begin{lstlisting}[language=json,firstnumber=1]
{
"author" : "_UserId",
"makeWith" : "QML" || "wizzard" ,
"language" : "it || en",
"question" : [{
	"type": "ordinamentoImmagini",
	"questionText": "Ordina queste immagini in modo da ottenere la parola "CIAO".",
	"url": "/img/ordinamentoImmagini/D0_1.png", //immagine possibile nel testo della domanda ad ordianamento di immagini
	"answers":
		[{
		"url": "/img/domandeOrdinamentoImmagini/I.png", //url dell'immagine
		"position": 2
		},{
		"url": "/img/domandeOrdinamentoImmagini/A.png",
		"position": 3
		},{
		"url": "/img/domandeOrdinamentoImmagini/O.png",
		"position": 4
		},{
		"url": "/img/domandeOrdinamentoImmagini/C.png",
		"position": 1
		}]
}],
"keywords":
	[{
	"parole",
	"grammatica"
	}],
"level" : 500,
"totalAnswers" : 0,
"correctAnswers" : 0
}
\end{lstlisting}

\subsubsection{JSON domanda a Collegamento di Elementi}
\begin{lstlisting}[language=json,firstnumber=1]
{
"author" : "_UserId",
"makeWith" : "QML" || "wizzard" ,
"language" : "it || en",
"question" : [{
	"type": "collegamento",
	"questionText": "Collega queste coppie di nemici classici.",
	"answers":
		[{
		"text_1_A": "cane",
		"text_1_B": "gatto"
		},{
		"url_2_A": "/img/collegamento/uncino.png",
		"text_2_B": "peter pan"
		},{
		"url_3_A": "/img/collegamento/D0_1.png",
		"url_3_B": "/img/collegamento/D0_5.png"
		}]
}],
"keywords":
	[{
	"nemici"
	}],
"level" : 500,
"totalAnswers" : 0,
"correctAnswers" : 0
}
\end{lstlisting}

\subsubsection{JSON domanda ad Area Cliccabile}
\begin{lstlisting}[language=json,firstnumber=1]
{
"author" : "_UserId" ,
"makeWith" : "QML" || "wizzard" ,
"language" : "it || en", 
"question" : [{
	"type": "areaCliccabile",
	"questionText": "Clicca quale tra le seguenti scelte e' il bicipide.",
	"image": "/img/areaCliccabile/D0_1.png", //sfondo dell'area cliccabile
	"resolution": { "x":400, "y":500 },
	"answers":
		[{
		"x": "200",
		"y": "50",
		"text": "testo facoltativo di arricchimento"
		},{
		"x": "300",
		"y": "120",
		"text": "testo facoltativo di arricchimento"
		},{
		"x": "200",
		"y": "200",
		"text": "testo facoltativo di arricchimento"
		}]
}],
"keywords":
	[{
	"corpo umano",
	"medicina",
	"scienze" ,
	"muscoli"
	}],
"level" : 500,
"totalAnswers" : 0,
"correctAnswers" : 0
}
\end{lstlisting}

\subsubsection{JSON domanda a Riempimento spazi vuoti}
\begin{lstlisting}[language=json,firstnumber=1]
{
"author" : "_UserId",
"makeWith" : "QML" || "wizzard",
"language" : "it || en",
"question" : [{
	"type": "spaziVuoti",
	"questionText": "Giulio Cesare era un console romano.",
	"answers":
		[{
		"parolaNumero": 2 //sta ad indicare quale parola e' da oscurare. In questo caso la numero 2
		},{
		"parolaNumero": 5
		}]
}],
"keywords":
	[{
	"storia",
	}],
"level" : 500,
"totalAnswers" : 0,
"correctAnswers" : 0
}
\end{lstlisting}
