\item \texttt{/:lang/user/training/question}
\begin{itemize}
	\item \textbf{Method}: POST;
	\item \textbf{Livello di Accesso}: Utente;
	\item \textbf{Descrizione}: restituisce una domanda in base al livello di abilità raggiunto dall'utente;
	\item \textbf{Request}: la richiesta deve contenere i seguenti campi:
	\begin{lstlisting}[language=json,firstnumber=1]
	{
	"topic" : [indica l'argomento in cui si vuoi esercitare l'utente]
	"keywords" : [indica le keywords per effettuare la ricerca della domanda]
	"level" : [indica il livello dell'utente inerente all'argomento scelto]
	}
	\end{lstlisting}
	\item \textbf{Response}: la risposta deve contenere i seguenti campi:
	\begin{lstlisting}[language=json,firstnumber=1]
	{
	"_questionId" : [identificativo della domanda]
	"makeWith" : [identifica lo strumento utilizzato per creare la domanda]
	"language" : [identifica la lingua della domanda]
	"question" : [identifica l'array di JSON contentente gli attributi che formano una domanda]
	}
	\end{lstlisting}
\end{itemize}

\item \texttt{/:lang/user/:userId/training/userstatistics}
\begin{itemize}
	\item \textbf{Method}: PUT;
	\item \textbf{Livello di Accesso}: Utente autenticato;
	\item \textbf{Descrizione}: aggiorna le statistiche dell'utente dopo che ha risposto ad una domanda;
	\item \textbf{Request}: la richiesta deve contenere i seguenti campi:
	\begin{lstlisting}[language=json,firstnumber=1]
	{
	"statistics" : [statistiche relative alla domanda]
	"level" : [livello dell'utente]
	}
	\end{lstlisting}	
\end{itemize}

\item \texttt{/:lang/user/:userId/training/questionstatistics}
\begin{itemize}
	\item \textbf{Method}: PUT;
	\item \textbf{Livello di Accesso}: Utente autenticato;
	\item \textbf{Descrizione}: aggiorna le statistiche della domanda;
	\item \textbf{Request}: la richiesta deve contenere i seguenti campi:
	\begin{lstlisting}[language=json,firstnumber=1]
	{
	"level" : [livello della domanda]
	"totalAnswers" : [risposte totali date alla domanda]
	"correctAnswers" : [risposte corrette totali date alla domanda]
	}
	\end{lstlisting}
\end{itemize}

\item \texttt{/:lang/user/training/userlevelupdate}
\begin{itemize}
	\item \textbf{Method}: PUT;
	\item \textbf{Livello di Accesso}: Utente non autenticato;
	\item \textbf{Descrizione}: aggiorna il livello dell'utente non autenticato;
	\item \textbf{Request}: la richiesta deve contenere i seguenti campi:
	\begin{lstlisting}[language=json,firstnumber=1]
	{
	"level" : [livello della domanda]
	}
	\end{lstlisting}
\end{itemize}