\newpage
\section{Tecnologie utilizzate}
\subsection{AngularJS}
Per lo sviluppo della parte Front-End dell'applicazione si è scelto di usare il \textit{framework\ped{G}} JavaScript AngularJS, che permette lo sviluppo di applicazioni in singola pagina. Esso consente di utilizzare \textit{HTML\ped{G}} come linguaggio di template e permette di estenderne la sintassi per esprimere i componenti dell'applicazione in modo chiaro e conciso. Il \textit{data binding\ped{G}} e il \textit{dependency injection\ped{G}} di AngularJS eliminano gran parte del codice che altrimenti si sarebbe dovuto scrivere. Tutto ciò avviene all'interno del browser, che lo rende un partner ideale con qualsiasi tecnologia server.
\subsection{Node.js}
Per lo sviluppo della parte Back-End dell'applicazione si è deciso di utilizzare la piattaforma \textit{event-driven\ped{G}} Node.js, basata sul motore JavaScript V8. Esso permette di realizzare applicazioni Web, come appunto \progetto, utilizzando il linguaggio JavaScript, tipicamente client-side, per la scrittura server-side. La caratteristica principale di Node.js risiede nella possibilità che offre di accedere alle risorse del sistema operativo in modalità \textit{event-driven\ped{G}} non sfruttando il classico modello basato su processi o thread concorrenti, utilizzato dai classici web server. Il modello \textit{event-driven\ped{G}}, o programmazione ad eventi, si basa su un concetto piuttosto semplice: si lancia una azione quando accade qualcosa. Ogni azione quindi risulta asincrona a differenza dei pattern di programmazione più comune in cui una azione succede ad un’altra solo dopo che essa è stata completata. Ciò garantisce una certa efficienza delle applicazioni grazie ad un sistema di \textit{callback\ped{G}} gestito a basso livello a runtime.
\subsection{MongoDB}
Come \textit{DBMS\ped{G}}, per l'applicativo \progetto, si è utilizzato MongoDB. E' un \textit{DBMS\ped{G}} non relazionale, orientato ai documenti. Classificato come un database di tipo NoSQL, si allontana dalla struttura tradizionale basata su tabelle dei database relazionali in favore di documenti in stile JSON con schema dinamico, rendendo l'integrazione di dati delle applicazioni web più facile e veloce. 
\subsection{MaterialAngular.js}
Per la gestione delle componenti grafiche si è scelto il \textit{framework\ped{G}} MaterialAngular.js per il suo nativo legame con AngularJS. Esso è sia un \textit{framework\ped{G}} per lo sviluppo di componenti UI, che un riferimento di implementazione per il \textit{Material Design\ped{G}} di Google. Espone un insieme di componenti dell'interfaccia utente riutilizzabili, ben collaudati, e accessibili basati sul \textit{Material Design\ped{G}}.
\subsection{Charts.js}
Per rappresentare a video i grafici che raccolgono le statistiche di ogni utente, si è utilizzata la libreria JavaScript Charts.js. Essa permette di visualizzate otto differenti tipi di grafici, animati, personalizzabili e \textit{responsive\ped{G}} 
\subsection{Angles.js}
Per utilizzare la libreria Charts.js all'interno dell'ambiente AngularJS, è necessario l'utilizzo della libreria Angles.js.
\subsection{Ace}
Per permettere all'utente dell'applicazione \progetto{} di comporre la propria domanda in codice QML (si veda Appendice B), si è scelto di sfruttare l'editor Ace. Esso è un editor di codice scritto in JavaScript e può essere facilmente integrato in qualsiasi pagina web e applicazioni JavaScript.
\subsection{Jison}
Per controllare la sintassi del codice QML scritto dall'utente, viene utilizzato il generatore di parser in codice JavaScript Jison. Esso prende una \textit{grammatica context-free\ped{G}} come input e restituisce un file JavaScript in grado di analizzare il linguaggio descritto da quella grammatica. È quindi possibile utilizzare lo script generato per analizzare gli ingressi e di accettare, rifiutare o eseguire azioni in base all'ingresso.
\subsection{AngularCSS}
AngularCSS è una libreria che permette di ottimizzare il livello di presentazione delle applicazioni in singola pagina con fogli di stile che si iniettano in modo dinamico in base alle esigenze. Resta in ascolto di eventi di modifica, aggiunge il CSS definito sul percorso attuale e rimuove il CSS dal percorso precedente.
\subsection{AngularUI}
AngularUI è una libreria per la gestione dei componenti grafici utilizzata per soddisfare alcune esigenze grafiche che MaterialAngular.js non permetteva di implementare nativamente.
\subsection{Drag and Drop for AngularJS}
Libreria che permette l'operazione di drag and drop all'interno dell'ambiente AngularJS. Utilizzata per implementare graficamente le domande a ordinamento, a collegamento e a riempimento spazi vuoti.
\subsection{angular-number-picker}
Direttiva che permette la raccolta di numeri tramite pulsante su / giù, invece di digitarli. Utilizzata per dare la possibilità all'utente di scegliere il numero di domande nella modalità allenamento.
\subsection{Express}
Express è un web application \textit{framework\ped{G}} per Node.js. Mette a disposizione numerosi metodi HTTP, il supporto per connect \textit{middleware\ped{G}} e consente di realizzare rapidamente interfacce per la programmazione; richiede moduli Node di terze parti per applicazioni che prevedono l'interazione con le basi di dati;
\subsection{Passport}
Passport è un \textit{middleware\ped{G}} per Node.js che permette l'autenticazione, in modo sicuro, ad una applicazione web, tramite username e password. 