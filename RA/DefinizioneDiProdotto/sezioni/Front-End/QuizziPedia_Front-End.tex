\subsection{QuizziPedia::Front-End}
\subsubsection{Informazioni generali}
\label{QuizziPedia::Front-End}
\begin{figure}[ht]
	\centering
	\includegraphics[scale=0.35]{UML/Package/QuizziPedia_Front-end.png}
	\caption{QuizziPedia::Front-End}
\end{figure}
\FloatBarrier
	\begin{itemize}
		\item \textbf{Descrizione}: package\ped{G} contenente le componenti front-end dell'applicazione;
		\item \textbf{Package contenuti}:
		\begin{itemize}
			\item \texttt{Controllers}: package\ped{G} contenente i \textit{controllers\ped{G}} front-end dell'applicazione;
			\item \texttt{Directives}: package\ped{G} contenente le \textit{directives\ped{G}} front-end dell'applicazione;
			\item \texttt{Models}: package\ped{G} contenente le classi che definiscono la business logic dell'applicazione;
			\item \texttt{ModelViews}: package contenente le classi che saranno presenti nella variabile d'ambiente \texttt{\$scope} di \textit{Angular\ped{G}} che permettono il \textit{Two-Way Data-Binding\ped{G}} tra le views e i \textit{controllers\ped{G}};
			\item \texttt{Services}: package\ped{G} contenente i \textit{services\ped{G}} front-end dell'applicazione;
			\item \texttt{Views}: package\ped{G} contenente le \textit{views\ped{G}} front-end dell'applicazione;
			\item \texttt{Filters}: package\ped{G} contenente i \textit{filters\ped{G}} front-end dell'applicazione;
			\item \texttt{QML}: package\ped{G} contenente le classi che compongono il parser QML.
		\end{itemize}
	\end{itemize}

\subsubsection{Classi}
		
	\paragraph{QuizziPedia::Front-End::Index}
	\begin{figure} [ht]
		\centering
		\includegraphics[scale=0.80]{UML/Classi/Front-End/QuizziPedia_Front-end_Views_Index.png}
		\caption{QuizziPedia::Front-End::Views::Index}
	\end{figure} \FloatBarrier
	\begin{itemize}
		\item \textbf{Descrizione}: view generale dell'applicazione;
		\item \textbf{Utilizzo}: contiene gli elementi che saranno presenti in ogni pagina dell'applicazione;
		\item \textbf{Relazioni con altre classi}:
		\begin{itemize}
			\item \textit{IN} \texttt{AppRun}: classe che verifica se l'utente sia autenticato e che abbia le giuste autorizzazioni per la pagina in cui si trova;
			\item \textit{IN} \texttt{MenuBarDirective}: rappresenta il menù, presente in ogni pagina dell'applicazione, generato in base agli oggetti passati nello \$scope isolato. Fornisce un pulsante per ogni oggetto ricevuto come parametro, ogni pulsante viene rappresentato con un’icona e con un testo. Al click di un pulsante viene invocata la funzione ad esso associata;
			\item \textit{IN} \texttt{FooterDirective}: directive che mostra il footer dell'applicazione che sarà presente in ogni pagina;
			\item \textit{IN} \texttt{ClickableAreaQuestionsView}: view contenente i campi e le direttive per creare una domanda ad area cliccabile;
			\item \textit{IN} \texttt{ConnectionQuestionsView}: view contenente i campi e le direttive per creare una domanda a collegamento;
			\item \textit{IN} \texttt{CreateQuestionnaireView}: view per la creazione del questionario. In questo componente viene permesso anche all'utente di:
			\begin{itemize}
				\item Effettuare delle ricerche sul database di domande;
				\item Selezionare le domande da inserire nel questionario;
				\item Mostrare le domande già inserite e permettere all'utente di eliminarle da tale lista.
			\end{itemize}
			\item \textit{IN} \texttt{EditorQMLView}: view contenente l'editor QML per la creazione di domande personalizzate;
			\item \textit{IN} \texttt{FillingQuestionnaireView}: view principale per la compilazione del questionario; conterrà i vari templates di ogni domanda appartenente al questionario;
			\item \textit{IN} \texttt{FilligQuestionsView}: view contenente i campi e le direttive per creare una domanda a riempimento testo;
			\item \textit{IN} \texttt{HomeView}: view contenente la direttiva per barra di ricerca degli utenti e questionari e il bottone che porterà l'utente nella modalità allenamento;
			\item \textit{IN} \texttt{ImagesSortingQuestionsView}: view contenente i campi e le direttive per creare una domanda a ordinamento immagini;
			\item \textit{IN} \texttt{LoginView}: view contenente le form necessarie per effettuare il login. Contiene inoltre un link alla pagina di registrazione e uno alla pagina per il recupero della password;
			\item \textit{IN} \texttt{MultipleQuestionsView}: view contenente le direttive per creare una domanda a risposta multipla;
			\item \textit{IN} \texttt{OtherUserView}: view contenente le direttive dei dati personali e delle statistiche di un utente ricercato;
			\item \textit{IN} \texttt{PasswordForgotView}: view contenente le form necessarie per il recupero della password dimenticata; 
			\item \textit{IN} \texttt{ProfileManagementView}: view contenente i dati personali che un utente può modificare dopo essersi registrato al sistema;
			\item \textit{IN} \texttt{QuestionnaireManagementView}:  view principale per la gestione dei questionari;
			\item \textit{IN} \texttt{QuestionsManagementView}: view contenente l'elenco delle domande create; 
			\item \textit{IN} \texttt{RegistrationManagementView}: view che permette di visualizzare gli utenti iscritti ad un questionario;
			\item \textit{IN} \texttt{ResultsQuestionnaireView}: view contenente i risultati conseguiti dagli utenti che hanno compilato il proprio questionario;
			\item \textit{IN} \texttt{ResultsView}: view contenente i risultati della ricerca effettuata. Vengono visualizzati sia gli utenti che i questionari trovati;
			\item \textit{IN} \texttt{SignUp}:  view contenente le form dedicate alla registrazione utente. Contiene inoltre un link alla pagina di login;
			\item \textit{IN} \texttt{StringSortingQuestionsView}:  view contenente i campi e le direttive per creare una domanda a ordinamento stringhe; 
			\item \textit{IN} \texttt{TrainingView}: view principale della modalità allenamento; conterrà i vari templates di ogni domanda dell'allenamento;
			\item \textit{IN} \texttt{TrueFalseQuestionsView}: view contenente le direttive per creare una domanda vero/falso;
			\item \textit{IN} \texttt{UserView}:  view contenente le direttive dei dati personali dell'utente, delle sue statistiche relative ai questionari e agli allenamenti effettuati e dei questionari a cui è iscritto;
		\end{itemize}
	\end{itemize}

		\paragraph{QuizziPedia::Front-End::AppRun}
		
		\label{QuizziPedia::Front-End::AppRun}
		
		\begin{figure}[ht]
			\centering
			\includegraphics[scale=0.45,keepaspectratio]{UML/Classi/Front-End/QuizziPedia_Front-end_AppRun.png}
			\caption{QuizziPedia::Front-End::AppRun}
		\end{figure} \FloatBarrier
		
		\begin{itemize}

			\item \textbf{Descrizione}: classe che istanza l'applicazione;
			\item \textbf{Utilizzo}: viene utilizzata per indicare le dipendenze tra l'applicazione con i \textit{packages\ped{G}} esterni;

			\item \textbf{Attributi}: 
			\begin{itemize}
				\item \texttt{-} \texttt{ngRoute: ngRoute} \\
				Campo dati contenente un riferimento all'oggetto ngRoute creato da \textit{Angular\ped{G}};
				\item \texttt{-} \texttt{ngAnimate: ngAnimate} \\
				Campo dati contenente un riferimento all'oggetto ngAnimate creato da \textit{Angular\ped{G}};
				\item \texttt{-} \texttt{ngMaterial: ngMaterial} \\
				Campo dati contenente un riferimento all'oggetto ngMaterial creato dal \textit{package\ped{G}} \textit{Material Angular\ped{G}};
				\item \texttt{-} \texttt{ngMessages: ngMessages} \\
				Campo dati contenente un riferimento all'oggetto ngMessages creato da \textit{Angular\ped{G}};
				\item \texttt{-} \texttt{ngCookies: ngCookies} \\
				Campo dati contenente un riferimento all'oggetto ngCookies creato da \textit{Angular\ped{G}};
				\item \texttt{-} \texttt{ngFileUpload: ngFileUpload} \\
				Campo dati contenente un riferimento all'oggetto ngFileUpload creato dal \textit{package\ped{G}} \textit{ng-file-upload\ped{G}};
				\item \texttt{-} \texttt{angularCSS: angularCSS} \\
				Campo dati contenente un riferimento all'oggetto angularCSS creato dal \textit{package\ped{G}} \textit{AngularCSS\ped{G}};
				\item \texttt{-} \texttt{ui.bootstrap: ui.bootstrap} \\
				Campo dati contenente un riferimento all'oggetto ui.bootstrap creato dal \textit{package\ped{G}} \textit{AngularUI\ped{G}};
				\item \texttt{-} \texttt{ngDragDrop: ngDragDrop} \\
				Campo dati contenente un riferimento all'oggetto ngDragDrop creato dal \textit{package\ped{G}} \textit{Drag and Drop for AngularJS\ped{G}};
				\item \texttt{-} \texttt{angularNumberPicker: angularNumberPicker} \\
				Campo dati contenente un riferimento all'oggetto angularNumberPicker creato dal \textit{package\ped{G}} \textit{angular-number-picker\ped{G}};
				\item \texttt{-} \texttt{angles: angles} \\
				Campo dati contenente un riferimento all'oggetto angles creato dal \textit{package\ped{G}} \textit{Angles.js\ped{G}}.
			\end{itemize}
			\item \textbf{Metodi}: 
			\begin{itemize}
				\item \texttt{+} \texttt{InitialSetting(\$mdThemingProvider: \$mdThemingProvider): void} \\
				Metodo che imposta il tema dell'applicazione. \\
				\textbf{Parametri}:
				\begin{itemize}
					\item \texttt{\$mdThemingProvider: \$mdThemingProvider}\\ Parametro contenente un riferimento al servizio di \textit{Angular Material\ped{G}} che si occupa di configurare il tema dell'applicazione.
				\end{itemize}			
			\end{itemize}
		\end{itemize}
		
	
	\paragraph{QuizziPedia::Front-End::AppRouter}
	
	\label{QuizziPedia::Front-End::AppRouter}
	
	\begin{figure}[ht]
		\centering
		\includegraphics[scale=0.5,keepaspectratio]{UML/Classi/Front-End/QuizziPedia_Front-end_AppRouter.png}
		\caption{QuizziPedia::Front-End::AppRouter}
	\end{figure} \FloatBarrier
	
	\begin{itemize}
		\item \textbf{Descrizione}: classe che gestisce i routes dell’applicazione, utilizza il servizio \$routeProvider per associare ad ogni route un controller e una view;
		\item \textbf{Utilizzo}: viene utilizzata per associare un URL alle varie view dell’applicazione;
		\item \textbf{Metodi}: 
		\begin{itemize}
			\item \texttt{-} \texttt{appRouter(\$routeProvider: \$routeProvider, \$locationProvider: \$locationProvider)}: metodo che gestisce i routes dell’applicazione. Utilizza il servizio \$routeProvider per associare ad ogni route un controller e una view; e \$locationProvider per configurare come i paths dell'applicazione vengono salvati. Questa funzione viene utilizzata come parametro nel metodo texttt{config} di \textit{Angular.js}. Il metoto \texttt{config} permette di impostare l'esecuzione di una funzione al caricamento del \textit{modulo\ped{G}} principale di \textit{Angular.js\ped{G}};
			\textbf{Metodi}:
			\begin{itemize}
				\item \texttt{\$routeProvider}: campo dati contenente un riferimento al servizio di \textit{Angular.js\ped{G}} che si occupa di definire le route dell’applicazione;
				\item \texttt{\$locationProvider}: campo dati contenente un riferimento al servizio di \textit{Angular.js\ped{G}} che si occupa di configuare come i paths vengono memorizzati;
			\end{itemize}
		\end{itemize}
	\end{itemize}
	
