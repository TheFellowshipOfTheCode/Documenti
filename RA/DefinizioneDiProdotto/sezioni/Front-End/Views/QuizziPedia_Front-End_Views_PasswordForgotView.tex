\paragraph{QuizziPedia::Front-End::Views::PasswordForgotView}

\label{QuizziPedia::Front-End::View::PasswordForgotView}
\begin{figure} [ht]
	\centering
	\includegraphics[scale=0.80]{UML/Classi/Front-End/QuizziPedia_Front-end_Views_PasswordForgotView.png}
	\caption{QuizziPedia::Front-End::Views::PasswordForgotView}
\end{figure} \FloatBarrier
\begin{itemize}
	\item \textbf{Descrizione}: \textit{view\ped{G}} contenente le form necessarie per il recupero della password dimenticata;
	\item \textbf{Utilizzo}: permette all'utente di recuperare la password dimenticata inserendo i campi dati necessari;
	\item \textbf{Relazioni con altre classi}:
	\begin{itemize}
			\item \textbf{IN \texttt{PasswordForgotModelView}}: classe di tipo \textit{modelview\ped{G}} la cui istanziazione è contenuta all'interno della variabile di ambiente \$scope di \textit{Angular\ped{G}}. All'interno di essa sono presenti le variabili e i metodi necessari per il \textit{Two-Way Data-Binding\ped{G}} tra la \textit{view\ped{G}} \texttt{PasswordForgotView} e il \textit{controller\ped{G}} \texttt{PasswordForgotController};
			\item \textbf{IN \texttt{LangModel}}: rappresenta il modello delle informazioni per la giusta traduzione dell'applicazione.
	\end{itemize}
	\item \textbf{Attributi}:
	\begin{itemize}
		\item \texttt{+ email: String} \\ Campo dati contenente l'email per il recupero password;
		\item \texttt{+ titleLangPasswordForgot: String} \\ Attributo che viene utilizzato per visualizzare la giusta traduzione del titolo della pagina, in italiano o in inglese;
		\item \texttt{+ emailLangPasswordForgot: String} \\ Attributo che viene utilizzato per visualizzare la giusta traduzione della \textit{label\ped{G}} per l'inserimento della posta elettronica, in italiano o in inglese;
		\item \texttt{+ recoveryButtonLangPasswordForgot: String} \\ Attributo che viene utilizzato per visualizzare la giusta traduzione della \textit{label\ped{G}} per il bottone di invio della nuova password all'indirizzo specificato, in italiano o in inglese;
		\item \texttt{+ loginButtonLangPasswordForgot: String} \\ Attributo che viene utilizzato per visualizzare la giusta traduzione della \textit{label\ped{G}} per il bottone di link all'autenticazione, in italiano o in inglese;
		\item \texttt{+ successRecovery: String} \\ Attributo che visualizza un messaggio di avvenuto invio della nuova password;
		\item \texttt{+ errorEmail: String} \\ Attributo che visualizza un eventuale messaggio di errore nell'inserimento della email.
	\end{itemize}
\end{itemize}