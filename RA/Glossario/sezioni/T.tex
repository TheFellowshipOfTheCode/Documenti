\section{T}
\begin{itemize}
	\item
	\textbf{Task}: lavoro, compito, incarico.
	\item
	\textbf{Team}: gruppo di persone che collabora nello svolgimento di un'attività.
	\item
	\textbf{TexMaker}: è un editor gratuito, moderno e multi-piattaforma per Linux, sistemi Mac OS e Microsoft Windows che integra molti strumenti utili per sviluppare documenti in \LaTeX.
	TexMaker include il supporto Unicode, il controllo ortografico, il completamento automatico, il raggruppamento del codice e un visore incorporato PDF con il supporto Synctex e modalità di visualizzazione continua.
	TexMaker è uno strumento facile da usare e da configurare.
	\item
	\textbf{Thread}:  è una suddivisione di un processo in due o più filoni o sottoprocessi, che vengono eseguiti concorrentemente da un sistema di elaborazione monoprocessore, multiprocessore o multicore.
	\item
	\textbf{Top-down}: nel modello top-down si formula inizialmente una visione generale del sistema ovvero se ne descrive la finalità principale senza scendere nel dettaglio delle sue parti. Ogni parte del sistema è successivamente rifinita (decomposizione, specializzazione e specificazione o identificazione) aggiungendo maggiori dettagli della progettazione. Ogni nuova parte così ottenuta può quindi essere nuovamente rifinita, specificando ulteriori dettagli, finché la specifica completa è sufficientemente dettagliata da validare il modello. Il modello top-down è spesso progettato con l'ausilio di scatole nere che semplificano il riempimento ma non consentono di capirne il meccanismo elementare.
	\item
	\textbf{Two-Way Data-Binding}: il data binding è il meccanismo di sincronizzazione automatica dei dati tra il modello e la view. Il data binding di AngularJS è bidirezionale, cioè ogni modifica al modello dei dati si riflette automaticamente sulla view e ogni modifica alla view viene riportata sul modello dei dati
\end{itemize}
