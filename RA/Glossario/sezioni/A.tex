\section{A}
\begin{itemize}
	\item
	\textbf{Amazon Web Service}: è una collezione di servizi di cloud computing, chiamato anche i "servizi web", che costituiscono una piattaforma di cloud computing offerto da Amazon.com.
	\item
	\textbf{Angular}: è un framework web open source principalmente sviluppato da Google e dalla comunità di sviluppatori individuali che ruotano intorno al framework nato per affrontare le molte difficoltà incontrate nello sviluppo di applicazioni singola pagina. Ha l'obiettivo di semplificare lo sviluppo e il test di questa tipologia di applicazioni fornendo un framework lato client con architettura MVC (Model View Controller) e Model–view–viewmodel (MVVM).
	\item
	\textbf{angular-number-picker}: direttiva usata per la raccolta dei numeri tramite pulsante su / giù, invece di digitarli.
	\item
	\textbf{AngularCSS}: libreria che ottimizza il livello di presentazione delle single page applications iniettando i fogli di stile in modo dinamico in base alle esigenze.
	\item 
	\textbf{AngularUI}: libreria che permette di utilizzare direttive Bootstrap all'interno dell'ambiente Angular.
	\item
	\textbf{API}: si indica ogni insieme di procedure disponibili al programmatore, di solito raggruppate a formare un set di strumenti specifici per l'espletamento di un determinato compito all'interno di un certo programma. Spesso con tale termine si intendono le librerie software disponibili in un certo linguaggio di programmazione.
	\item 
	\textbf{Astah}: precedentemente noto come JUDE (Java e ambiente UML Developers), è uno strumento di modellazione UML creato dall'azienda giapponese Change Vision.
\end{itemize}