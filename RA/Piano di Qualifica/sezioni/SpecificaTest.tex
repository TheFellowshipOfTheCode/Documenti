\section{Specifica dei test}
Il \textit{team\ped{G}} \gruppo, al fine di implementare del software che sia di qualità, ha strutturato dei test atti a verificare che il software prodotto rispecchi le funzionalità a fronte di risultati attesi.
Questi test sono il frutto dell'applicazione delle tecniche di verifica dinamica che sono state introdotte nel documento \NdP.
Tutte le attività di testing prodotte devono poter essere ripetibili e devono essere deterministiche, al fine di poter fornire delle informazioni utili a intraprendere azioni di correzione, nel caso in cui i risultati ottenuti siano diversi da quelli attesi.
Per avere un tracciamento dei test prodotti e dei risultati ottenuti si è scelto di classificare il tutto producendo dei log che siano di facile consultazione e che possano fornire una precisa indicazione di quelli che sono stati gli output di queste attività di verifica, eventuali errori o eventuali risultati che siano non coerenti con quanto in precedenza fissato.
\subsection{Tipi di test}
Sono stati individuati quattro livelli di testing e sono rispettivamente:
\begin{itemize}
\item \textbf{Test di unità [TU]:} con questa tipologia di test si cerca di verificare la più piccola parte di lavoro prodotta da un programmatore. Questo si traduce tendenzialmente a verificare i metodi e le funzioni scritte;
\item \textbf{Test di integrazione [TI]:} con questa tipologia di test si cerca di verificare le componenti di sistema. Più precisamente, l'obiettivo è quello di testare il funzionamento dei vari package prodotti, sia singolarmente che nel loro insieme;
\item \textbf{Test di sistema [TS]:} con questa tipologia di test si cerca di verificare che il comportamento e il funzionamento dell'architettura siano corretti;
\item \textbf{Test di validazione [TV]:} con questa tipologia di test si vuole verificare che il lavoro prodotto soddisfi quanto richiesto dal proponente.
\end{itemize}

% INSERIMENTO DEI TEST
\subsubsection{Test di Validazione}
I test di validazione hanno lo scopo di accertare che tutte le funzionalità richieste dal proponente siano state soddisfatte. Per questo motivo, attraverso delle macro azioni, si andrà a simulare il comportamento generale dell'applicativo e dell'utente che interagisce con esso.
I test di validazione saranno organizzati nel modo seguente:
\begin{center}
\textbf{TV}[\textit{TipoRequisito}][\textit{ImportanzaRequisito}][\textit{IdRequisito}]
\end{center}
dove:
\begin{itemize}
\item
\textbf{TipoRequisito} può assumere valori tra:
	\begin{itemize}
	\item
	\textit{F} per i requisiti funzionali;
	\item
	\textit{Q} per i requisiti di qualità;
	\item
	\textit{V} per i requisiti di vincolo;
	\item
	\textit{P} per i requisiti prestazionali.
	\end{itemize}
\item 
\textbf{ImportanzaRequisito} può assumere valori tra:
	\begin{itemize}
	\item
	\textit{D} per i requisiti desiderabili;
	\item
	\textit{O} per i requisiti di obbligatori;
	\item
	\textit{F} per i requisiti di facoltativi.
	\end{itemize}
\item
\textbf{IdRequisito} assume un valore gerarchico che identifica il singolo requisito.
\end{itemize}

% TABELLA
\normalsize
\begin{longtable}[ht]{|c|>{}m{8cm}|c|}
\hline 
\textbf{Id Test} & \textbf{Descrizione} & \textbf{Stato}\\
\hline
\endhead
\hypertarget{TVFO1}{TVFO1} & L’utente intende registrarsi al sistema. All’utente è richiesto di:
\begin{itemize}
\item
Trovarsi nella sezione apposita;
\item
Compilare il form di registrazione;
\item
Premere il pulsante di conferma.
\item
Verificare attraverso l’autenticazione che la registrazione sia effettivamente avvenuta.
\end{itemize}
 & \textit{Non Implementato}\\ \hline
\hypertarget{TVFO2}{TVFO2} & L’utente intende autenticarsi al sistema. All’utente è richiesto di:
\begin{itemize}
\item
Trovarsi nella sezione apposita;
\item
Inserire le credenziali nell’apposito form;
\item
Premere il pulsante di autenticazione;
\item
Verificare che l’autenticazione sia effettivamente avvenuta.
\end{itemize}
 & \textit{Non Implementato}\\ \hline
\hypertarget{TVFO3}{TVFO3} & L’utente intende disconnettersi dal sistema. All’utente è richiesto di:
\begin{itemize}
\item
Essere Autenticato;
\item
Trovarsi nella sezione apposita;
\item
Premere il pulsante di logout;
\item
Verificare che la disconnessione sia effettivamente avvenuta. 
\end{itemize}
& \textit{Non Implementato}\\ \hline
\hypertarget{TVFD4}{TVFD4} & L’utente autenticato  intende gestire i propri dati. All’utente è richiesto di:
\begin{itemize}
\item
Essere autenticato;
\item
Trovarsi nella sezione apposita;
\item
Modificare i campi dati consentiti;
\item
Premere il tasto conferma modifica;
\item
Visualizzare il profilo dell'utente modificato.
\end{itemize}
 & \textit{Non Implementato}\\ \hline
\hypertarget{TVFD4.3}{TVFD4.3} & L’utente autenticato  intende impostare la propria foto profilo . All’utente è richiesto di:
\begin{itemize}
\item
Essere autenticato;
\item
Trovarsi nella sezione apposita;
\item
Impostare la foto da inserire;
\item
Visualizzare il profilo dell’utente modificato.
\end{itemize}
 & \textit{Non Implementato}\\ \hline
\hypertarget{TVFD4.8}{TVFD4.8} & L’utente autenticato  intende modificare la tipologia di utenza. All’utente è richiesto di:
\begin{itemize}
\item Essere autenticato;
\item Trovarsi nella sezione apposita;
\item Cambiare la tipologia di utenza;
\item Verificare la modifica effettuata. 
\end{itemize}& \textit{Non Implementato}\\ \hline
\hypertarget{TVFD4.9}{TVFD4.9} & L’utente autenticato  intende eliminare il proprio account. All’utente è richiesto di:
\begin{itemize}
\item Essere autenticato;
\item Trovarsi nella sezione apposita;
\item Premere il tasto di elimanazione;
\item Verificare la disconnessione della sessione sia avvenuta;
\item Verificare che l’autenticazione con le credenziali dell’utente eliminato non avvenga.
\end{itemize} & \textit{Non Implementato}\\ \hline
\hypertarget{TVFD5}{TVFD5} & L’utente autenticato  intende ricercare un questionario per iscriversi. All'utente è richiesto di:
\begin{itemize}
\item Essere autenticato;
\item Trovarsi nella sezione apposita;
\item Ricercare un questionario;
\item Iscrizione ad un questionario trovato mediante l’apposito tasto;
\item Verificare l'operazione appena effettuata.
\end{itemize}
 & \textit{Non Implementato}\\ \hline
\hypertarget{TVFO6}{TVFO6} & L’utente autenticato  intende compilare un questionario a cui si è  iscritto. All’utente è richiesto di:
\begin{itemize}
\item Essere autenticato;
\item Trovarsi nella sezione apposita;
\item Selezionare il questionario da compilare;
\item Rispondere alle domande previste;
\item Confermare le risposte date al questionario;
\item Verificare il risultato del questionario.
\end{itemize}
 & \textit{Non Implementato}\\ \hline
\hypertarget{TVFD7.1}{TVFD7.1} & L’utente autenticato intende creare una nuova
domanda tramite procedura guidata. All’utente è richiesto di:
\begin{itemize}
\item Essere autenticato;
\item Trovarsi nella sezione apposita;
\item Premere il pulsante Crea nuova domanda;
\item Inserire i dati necessari alla creazione di una nuova domanda;
\item Premere il pulsante di conferma creazione
nuova domanda;
\item Verificare che sia stato creata la nuova domanda.
\end{itemize}

 & \textit{Non Implementato}\\ \hline
\hypertarget{TVFD7.2}{TVFD7.2} & L’utente autenticato intende modificare una
domanda esistente tramite procedura guidata. All’utente è richiesto di:
\begin{itemize}
\item Essere autenticato;
\item Trovarsi nella sezione apposita;
\item Premere il pulsante  modifica domanda;
\item Modificare i dati della domanda selezionata;
\item Premere il pulsante di conferma modifica;
\item Verificare che sia stata modificata la domanda.
\end{itemize}
 & \textit{Non Implementato}\\ \hline
\hypertarget{TVFO7.3}{TVFO7.3} & L’utente autenticato intende creare una nuova
domanda tramite QML. All’utente è richiesto di:
\begin{itemize}
\item Essere autenticato;
\item Trovarsi nella sezione apposita;
\item Premere il pulsante Crea nuova domanda;
\item Inserire i dati necessari alla creazione di una nuova domanda;
\item Premere il pulsante di conferma creazione
nuova domanda;
\item Verificare che sia stato creata la nuova domanda.
\end{itemize}
 & \textit{Non Implementato}\\ \hline
\hypertarget{TVFD7.4}{TVFD7.4} & L’utente autenticato intende modificare una
domanda esistente tramite QML. All’utente è richiesto di:
\begin{itemize}
\item Essere autenticato;
\item Trovarsi nella sezione apposita;
\item Premere il pulsante  modifica domanda;
\item Modificare i dati della domanda selezionata;
\item Premere il pulsante di conferma modifica;
\item Verificare che sia stata modifcata la domanda.
\end{itemize}
 & \textit{Non Implementato}\\ \hline
\hypertarget{TVFD8.1}{TVFD8.1} & L’utente autenticato intende visualizzare i questionari creati. All’utente è richiesto di:
\begin{itemize}
\item
Essere autenticato;
\item
Trovarsi nella sezione apposita;
\item
Verificare che vengano visualizzati tutti i questionari creati
\end{itemize}
 & \textit{Non Implementato}\\ \hline
\hypertarget{TVFD8.2}{TVFD8.2} & 
L’utente autenticato pro  intende modificare un questionario esistente. All’utente è richiesto di:
\begin{itemize}
\item Essere autenticato pro;
\item Trovarsi nella sezione apposita;
\item Selezionare il questionario da modificare tra quelli creati;
\item Dalla sezione di modifica effettuare tutte le modifiche consentite dal sistema 
\item Premere il pulsante di conferma modifiche;
\item Verificare che sia stato modificato il questionario correttamente.
\end{itemize}
 & \textit{Non Implementato}\\ \hline
\hypertarget{TVFD8.3}{TVFD8.3} & L’utente autenticato pro  intende eliminare un questionario esistente. All’utente è richiesto di:
\begin{itemize}
\item Essere autenticato pro;
\item Trovarsi nella sezione apposita;
\item Selezionare un questionario esistente da eliminare;
\item Premere il tasto di conferma eliminazione;
\item Verificare l’operazione appena effettuata. 
\end{itemize}
& \textit{Non Implementato}\\ \hline
\hypertarget{TVFD8.4}{TVFD8.4} & L’utente autenticato pro  intende controllare i risultati degli esaminandi di un suo questionario. All’utente è richiesto di:
\begin{itemize}
\item Essere autenticato pro;
\item Trovarsi nella sezione apposita;
\item Selezionare il questionario da controllare gli esiti;
\item Verificare che si ci siano gli esiti di tutti gli utenti iscritti.
\end{itemize}
 & \textit{Non Implementato}\\ \hline
\hypertarget{TVFD8.5}{TVFD8.5} & L’utente autenticato pro  intende attivare la modalità esame di un suo questionario. All’utente è richiesto di:
\begin{itemize}
\item Essere autenticato pro;
\item Trovarsi nella sezione apposita;
\item Selezionare il questionario da attivare;
\item Verificare l'operazione effettuata.
\end{itemize} & \textit{Non Implementato}\\ \hline
\hypertarget{TVFO8.6}{TVFO8.6} & L’utente autenticato pro  intende creare  un nuovo questionario. All’utente è richiesto di:
\begin{itemize}
\item Essere autenticato pro;
\item Trovarsi nella sezione apposita;
\item Premere il pulsante Crea nuovo questionario;
\item Inserire il titolo del nuovo questionario;
\item Inserire l’argomento del nuovo questionario;
\item Ricercare e selezionare le domande che andranno a comporre il questionario;
\item Premere il pulsante di conferma creazione nuovo questionario;
\item Verificare che sia stato creato il nuovo questionario.
\end{itemize} & \textit{Non Implementato}\\ \hline
\hypertarget{TVFO8.7}{TVFO8.7} & L’utente autenticato pro  intende gestire le domande del questionario. All’utente è richiesto di:
\begin{itemize}
\item Essere autenticato pro;
\item Trovarsi nella sezione apposita;
\item Premere il pulsante per gestire il questionario;
\item Effettuare le operazioni consentite;
\item Verificare che le operazioni siano effettivamente avvenute.
\end{itemize} & \textit{Non Implementato}\\ \hline
\hypertarget{TVFD8.8}{TVFD8.8} & L’utente autenticato pro  intende gestire le iscrizione di un suo questionario. All’utente è richiesto di:
\begin{itemize}
\item Essere autenticato pro;
\item Trovarsi nella sezione apposita;
\item Selezionare il questionario da gestire le iscrizioni;
\item Effettuare le operazioni consentite;
\item Verificare le operazioni apportate.
\end{itemize} & \textit{Non Implementato}\\ \hline
\hypertarget{TVFO9}{TVFO9} & L’utente intende esercitarsi effettuando un allenamento. All’utente è richiesto di:
\begin{itemize}
\item Trovarsi nella sezione apposita;
\item Selezionare gli appositi filtri per focalizzare l’allenamento che si vuole effettuare;
\item Premere il pulsante per iniziare l’allenamento.
\item Rispondere alle domande proposte e controllare l’esito delle risposte date.
\end{itemize} & \textit{Non Implementato}\\ \hline
\hypertarget{TVFD10}{TVFD10} & L’utente autenticato  intende visualizzare il proprio profilo. All’utente è richiesto di:
\begin{itemize}
\item Essere autenticato;
\item Trovarsi nella sezione apposita;
\item Visualizzare il proprio profilo.
\end{itemize}
 & \textit{Non Implementato}\\ \hline
\hypertarget{TVFD10.4}{TVFD10.4} & L’utente autenticato  intende visualizzare la cronologia dei questionari svolti. All’utente è richiesto di:
\begin{itemize}
\item Essere autenticato;
\item Trovarsi nella sezione apposita;
\item Visualizzare la cronologia dei questionari svolti.
\end{itemize}
 & \textit{Non Implementato}\\ \hline
\hypertarget{TVFD10.4.1}{TVFD10.4.1} & L’utente autenticato  intende visualizzare il riepilogo di un questionario svolto. All’utente è richiesto di:
\begin{itemize}
\item Essere autenticato;
\item Trovarsi nella sezione apposita;
\item Selezionare il questionario svolto da visualizzare il riepilogo. 
\end{itemize}
& \textit{Non Implementato}\\ \hline
\hypertarget{TVFO11}{TVFO11} & L’utente  intende rispondere alle domande. All’utente è richiesto di:
\begin{itemize}
\item Essere autenticato;
\item Trovarsi nella sezione apposita;
\item Rispondere alle domande;
\item Verificare la funzionalità dell'operazione.
\end{itemize}
 & \textit{Non Implementato}\\ \hline
\hypertarget{TVFO12}{TVFO12} & L’utente autenticato  intende ricercare un utente. All’utente è richiesto di:
\begin{itemize}
\item Essere autenticato;
\item Trovarsi nella sezione apposita;
\item Inserire le keywords nella barra di ricerca;
\item Premere l'apposito tasto per effettuare la ricerca;
\item Verificare l’operazione appena effettuata.
\end{itemize}
 & \textit{Non Implementato}\\ \hline
\caption[Test di Validazione]{Test di Validazione}
\label{tabella:test0}
\end{longtable}
\clearpage

\subsubsection{Test di Sistema}
test sistema
\subsubsection{Test di Integrazione}
Con questa tipologia di test si vuole determinare il corretto funzionamento delle componenti progettate durante la definizione dell'architettura ad alto livello.

I test di integrazione saranno descritti nel modo seguente:
\begin{center}
\textbf{TI}[\textit{IdComponente}]
\end{center}
dove:
\begin{itemize}
\item
\textbf{IdComponente} rappresenta il codice identificativo crescente del componente considerato.
\end{itemize}
È stato scelto di utilizzare un approccio top-down nel determinare i test di integrazione. Di seguito viene riportato un diagramma informale per rendere chiara la struttura dei test identificati.
\begin{figure}[ht]
	\centering
	\includegraphics[scale=0.45]{AlberoDiIntegrazione.png}
	\caption{Albero di integrazione delle componenti}
\end{figure}
\FloatBarrier

Nell'approccio top-down dei test di integrazione i moduli di livello più alto vengono sottoposti a test e integrati per primi. Così facendo anche la logica di alto livello e il flusso di dati vengono sottoposti a test fin da subito; sarà perciò necessario simulare le componenti di livello più basso con degli stub. Una volta codificate, le componenti di più basso livello dovranno a loro volta essere integrate e testate. L'approccio top-down rientra tra le strategie di integrazione incrementali, che conferiscono il vantaggio di poter determinare in modo più immediato quale componente causa problemi: i difetti rilevati dai test, infatti, nella maggioranza dei casi saranno da attribuirsi all'ultima componente aggiunta.
\begin{figure}[ht]
	\centering
	\includegraphics[scale=0.45]{DiagrammaDiAttivita.png}
	\caption{Diagramma di attività dei test di integrazione}
\end{figure}
\FloatBarrier

% TABELLA
\normalsize
\begin{longtable}[ht]{|c|>{}m{10cm}|c|}
\hline 
\textbf{Id Test} & \textbf{Descrizione} & \textbf{Stato}\\
\hline
\endhead
\hypertarget{TI1}{TI1} & Viene verificato che l’applicazione Web gestisca
correttamente il Front-End del prodotto e le sue interazioni
con il Back-End. & \textcolor{Green}{\textit{Superato}}\\ \hline
\hypertarget{TI2}{TI2} & Viene verificato che i Controllers del Front-End si integrino
correttamente nell’applicazione Web. & \textcolor{Green}{\textit{Superato}}\\ \hline
\hypertarget{TI3}{TI3} & Viene verificato che i Services permettano di interagire
correttamente con il Back-End. & \textcolor{Green}{\textit{Superato}}\\ \hline
\hypertarget{TI4}{TI4} & Viene verificato che il Model si integri correttamente con i
Services e con le componenti dell’applicazione che lo
utilizzano. & \textcolor{Green}{\textit{Superato}}\\ \hline
\hypertarget{TI5}{TI5} & Viene verificato che le Directives si integrino correttamente
con le Views. & \textcolor{Green}{\textit{Superato}}\\ \hline
\hypertarget{TI6}{TI6} & Viene verificato che le Views si integrino correttamente con i
Controllers e che visualizzino in modo corretto i dati da essi
ricevuti. & \textcolor{Green}{\textit{Superato}}\\ \hline
\hypertarget{TI7}{TI7} & Viene verificato che il server si avvii correttamente,
utilizzando Config per effettuare le configurazioni
dell’applicazione, e che l’applicazione Web gestisca
correttamente il Back-End del prodotto in modo tale da
fornire al Front-End tutte le informazioni richieste. & \textcolor{Green}{\textit{Superato}}\\ \hline
\hypertarget{TI8}{TI8} & Viene verificato che App si integri correttamente con le
librerie di Node.js utilizzate. & \textcolor{Green}{\textit{Superato}}\\ \hline
\hypertarget{TI9}{TI9} & Viene verificato che Config si integri con Server, carichi
correttamente tutte le librerie per Node.js che utilizzerà e
che istanzi le classi del package App in modo corretto. & \textcolor{Green}{\textit{Superato}}\\ \hline
\hypertarget{TI10}{TI10} & Viene verificato che Config si integri con Server, carichi
correttamente tutte le librerie per Node.js che utilizzerà e
che istanzi le classi del package App in modo corretto. & \textcolor{Green}{\textit{Superato}}\\ \hline
\hypertarget{TI11}{TI11} & Viene verificato che i Controllers si integrino correttamente
e gestiscano le richieste inoltrate dai Routers. & \textcolor{Green}{\textit{Superato}}\\ \hline
\hypertarget{TI12}{TI12} & Viene verificato che il Model si integri correttamente con i
Controllers per la gestione dell’inserimento, della modifica e
dell’eliminazione dei dati. & \textcolor{Green}{\textit{Superato}}\\ \hline
\caption[Test di Integrazione]{Test di Integrazione}
\label{tabella:test2}
\end{longtable}
\clearpage
\subsubsection{Test di Unità}
Con questa tipologia di test si vuole verificare il corretto funzionamento delle unità individuate durante la definizione dell'architettura ad alto livello.
I test di unità saranno descritti nel modo seguente:
\begin{center}
\textbf{TU}[IdTest]
\end{center}
dove:
\begin{itemize}
\item IdTest rappresenta il codice identificativo crescente dell'unità considerata.
\end{itemize}