\newpage

\section{Architettura del software}
\label{Architettura}
\begin{figure}[ht]
	\centering
	\includegraphics[scale=0.30]{UML/Package/QuizziPedia.png}
	\caption{Architettura}
\end{figure}
\FloatBarrier
\begin{itemize}
	\item \textbf{Descrizione}: architettura ad alto livello dell'applicazione \progetto;
	\item \textbf{Packages contenuti}:
	\begin{itemize}
		\item \texttt{QuizziPedia::Front-End}: \textit{package\ped{G}} contenente i \textit{packages\ped{G}} che compongono il Front-End;
		\item \texttt{QuizziPedia::Back-End}: \textit{package\ped{G}} contenente i \textit{packages\ped{G}} che compongono il Back-End;
		\item \texttt{MaterialAngular.js}: framework per lo sviluppo dei componenti grafici dell'applicazione;
		\item \texttt{Charts.js}: libreria per lo sviluppo dei grafici per la visualizzazione delle statistiche utente;
		\item \texttt{Angles.js}: libreria necessaria per integrare la libreria Charts.js all'interno dell'ambiente \textit{Angular\ped{G}};
		\item \texttt{Ace}: editor di codice incorporabile scritto in JavaScript;
		\item
		\texttt{angular-number-picker}: direttiva usata per la raccolta dei numeri tramite pulsante su / giù, invece di digitarli;
		\item
		\texttt{AngularCSS}: libreria che ottimizza il livello di presentazione delle single page applications iniettando i fogli di stile in modo dinamico in base alle esigenze;
		\item 
		\texttt{AngularUI}: libreria che permette di utilizzare direttive Bootstrap all'interno dell'ambiente Angular;
		\item
		\texttt{Drag and Drop for AngularJS}: libreria che implementa la funzionalità jQuery UI Drag and Drop nell'ambiente Angular;
		\item \texttt{Jison}: è un generatore di parser JavaScript;
		\item \texttt{Express}: framework Web di routing e middleware, con funzionalità sua propria minima: un’applicazione Express è essenzialmente una serie di chiamate a funzioni middleware;
		\item \texttt{Passport}: \textit{middleware\ped{G}} per l'autenticazione in ambiente \textit{Node.js\ped{G}}.
	\end{itemize}
\end{itemize}
