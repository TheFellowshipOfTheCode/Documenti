\newpage
\section{Estensione delle funzionalità}
Per ragioni di tempo e di competenze, alcune funzionalità del software \progetto{} non sono state implementate. In questo capitolo vengono elencati tutti i punti di possibile estensione delle funzionalità e un loro possibile sviluppo.
\subsection{Autenticazione tramite social network}
Uno degli obbiettivi principali dell'applicazione \progetto{} è quello della condivisione di domande e questionari. E' stata perciò progettata e sviluppata per essere un software social e conforme alle ultime tendenze grafiche. Per questo, uno dei requisiti desiderabili è permettere all'utente di non perdere tempo durante la fase di registrazione, ma di poterlo fare appoggiandosi alle ultime piattaforme social come Facebook e Twitter. Il \textit{framework\ped{G}} Passport utilizzato del team per le funzionalità di registrazione e autenticazione, permette nativamente di implementare la registrazione tramite i più famosi social network attuali. La guida per implementare queste funzionalità si trova alla pagina \url{http://passportjs.org/docs}. 
\subsection{Cambio tipologia account}
Le tipologie di utenza dell'applicazione \progetto{} sono principalmente tre:
\begin{itemize}
	\item Utente non autenticato;
	\item Utente autenticato;
	\item Utente autenticato pro.
\end{itemize}
La tipologia di utente autenticato pro è stata pensata principalmente per i professori universitari e delle scuole medie/superiori che vogliono appoggiarsi alla piattaforma \progetto{} per creare questionari finalizzati alle verifiche e agli esami. Allo stato attuale, l'applicazione permette il passaggio da utente normale a pro semplicemente cliccando l'apposito pulsante presente all'interno della pagina \textit{Gestione profilo}. In base alle esigenze strategiche ed economiche dell'azienda acquirente di \progetto{}, sarà opportuno sviluppare ulteriormente questa funzionalità di cambio tipologia account, ad esempio associando un metodo di pagamento oppure l'immissione di un particolare codice che solo i professori associati a scuole e università possono avere.  
\subsection{Creazione delle domande tramite wizard}
La creazione delle domande avviene tramite il linguaggio di markup QML (si veda l'Appendice A). Esso è stato creato per essere semplice e comprensibile ma allo stesso tempo molto potente. Permette infatti di creare non solo domande classiche come vero/falso, a scelta multipla e ordinamento, ma anche di combinare questi generi tra loro e creare così nuove tipologie di domande. Questo ha causato un piccolo aumento della difficoltà del linguaggio che potrebbe risultare poco attraente ad un utente con poca famigliarità con i linguaggi di programmazione. Per ovviare questo problema è opportuno sviluppare una nuova modalità di creazione delle domande tramite wizard guidati. Le tipologie base delle domande sviluppate allo stato attuale sono sei:
\begin{itemize}
	\item Vero/Falso;
	\item Risposta multipla;
	\item Ordinamento stringhe;
	\item Ordinamento immagini;
	\item Riempimento spazi vuoti;
	\item Collegamento stringhe.
\end{itemize}
Lo sviluppo di wizard per la creazione di ogni tipologia di domanda base, aiuterebbe l'utente medio, che non ha famigliarità con i linguaggi di markup, nell'utilizzo del servizio \progetto.   
\subsection{Nuove tipologie di domande}
Il linguaggio QML (si veda l'Appendice A) prevede sei tipologie di domande base:
\begin{itemize}
	\item Vero/Falso;
	\item Risposta multipla;
	\item Ordinamento stringhe;
	\item Ordinamento immagini;
	\item Riempimento spazi vuoti;
	\item Collegamento stringhe.
\end{itemize}
E' possibile estendere il codice per creare nuove tipologie di domande base. Questo prevede l'aggiunta di una nuova classe con il relativo codice QML all'interno del package \texttt{questionCheck} e aggiornare la classe \texttt{CheckQML} con la nuova tipologia di domanda. Infine è necessario aggiornare la classe \texttt{QuestionsController} all'interno del package \texttt{QuizziPedia/Front-End/Controllers} per il recupero della nuova tipologia di domanda e creare una nuova direttiva per la presentazione della stessa all'interno del package \texttt{QuizziPedia/Front-End/Directives}.
\subsection{Aggiunta nuove lingue}
Il servizio \progetto{} è stato sviluppato per poter essere tradotto in qualsiasi lingua desiderata. Ogni parola chiave, ovvero non di contenuto inserita dall'utente ma nativa del servizio, è in realtà una variabile che prende il suo valore dalla collection \texttt{Variables} all'interno del database MongoDB. Per inserire una nuova lingua è necessario dunque creare un nuovo documento all'interno di \texttt{Variables}, traducendo tutte le keywords dell'applicazione nella lingua desiderata, con la seguente intestazione:
\begin{lstlisting}[language=Java,firstnumber=1]
	"lang": "abbreviazione lingua desiderata",
	"correctWord": "lingua desiderata",
\end{lstlisting}
