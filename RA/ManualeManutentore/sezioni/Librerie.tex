\newpage
\section{Librerie e frameworks utilizzati}
\subsection{Front-End}
Nella parte Front-End dell'applicazione sono state utilizzate le seguenti librerie e \textit{frameworks\ped{G}}:
\begin{itemize}
	\item \texttt{MaterialAngular.js}: \textit{framework\ped{G}} per lo sviluppo dei componenti grafici dell'applicazione;
	\item \texttt{Charts.js}: libreria per lo sviluppo dei grafici per la visualizzazione delle statistiche
	utente;
	\item \texttt{Angles.js}: libreria necessaria per integrare la libreria Charts.js all'interno
	dell'ambiente \textit{Angular\ped{G}};
	\item \texttt{Ace}: editor di codice incorporabile scritto in JavaScript.
	\item
	\texttt{angular-number-picker}: direttiva usata per la raccolta dei numeri tramite pulsante su / giù, invece di digitarli;
	\item
	\texttt{AngularCSS}: libreria che ottimizza il livello di presentazione delle single page application iniettando i fogli di stile in modo dinamico in base alle esigenze;
	\item 
	\texttt{AngularUI}: libreria che permette di utilizzare direttive Bootstrap all'interno dell'ambiente Angular;
	\item
	\texttt{Drag and Drop for AngularJS}: libreria che implementa la funzionalità jQuery UI Drag and Drop nell'ambiente Angular;
	\item \texttt{Jison}: generatore di parser JavaScript.
\end{itemize}
\subsection{Back-End}
Nella parte Back-End dell'applicazione sono state utilizzate le seguenti librerie e \textit{frameworks\ped{G}}:
\begin{itemize}
	\item \texttt{Express}: \textit{framework\ped{G}} Web di routing e \textit{middleware\ped{G}}, con funzionalità sua propria minima: un'applicazione Express è essenzialmente una serie di chiamate a funzioni \textit{middleware\ped{G}};
	\item \texttt{Passport}: \textit{middleware\ped{G}} per l'autenticazione in ambiente \textit{Node.js\ped{G}}.
\end{itemize}
