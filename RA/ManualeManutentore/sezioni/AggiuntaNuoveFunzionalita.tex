\newpage
\section{Aggiunta di nuove funzionalità}
Il software QuizziPedia è stato progettato e sviluppato seguendo il principio di estendibilità del codice. Pertanto è possibile aggiungere nuove classi, sia nella parte Front-End che Back-End, per lo sviluppo di nuove funzionalità.

\subsection{Front-End}

\subsubsection{Inserimento nuova view}
Per inserire una nuova \textit{view\ped{G}} è necessario creare un file $nomeview.html$ all'interno della cartella Views del path: QuizziPedia/Front-End/Views. Se si decide di utilizzare la stessa metodologia dell'intero progetto per la visualizzazione della giusta traduzione delle parole chiave è indispensabile che il controller associato alla view sia \texttt{AppController} e che le keywords che saranno visualizzate a video siano del tipo: \texttt{$listOfKey.nomevariabile$}. Infine è necessario aggiornare il file \texttt{AppRouter} inserendo il codice:

\begin{lstlisting}[language=Java,firstnumber=1]
.when('/:lang/nomeview', {
templateUrl: '/Views/NomeView.html',
controller:"NuovoController",
css: [
{
href: 'css/auth-main.css'
},
{
href: 'css/auth-medium.css',
media: 'handheld, screen and (max-width:960px), only screen and (max-device-width:960px)'
},
{
href: 'css/auth-small.css',
media: 'handheld, screen and (max-width:480px), only screen and (max-device-width:480px)'
}
]
})
\end{lstlisting}

\subsubsection{Inserimento nuova directive}
Per inserire una nuova \textit{directive\ped{G}} custom è necessario creare un file $nomedirective.html$ e $nomedirective.json$ all'interno della cartella Directives del path QuizziPedia/Front-End/Directives. Se si decide di utilizzare la stessa metodologia dell'intero progetto per la visualizzazione della giusta traduzione delle parole chiave è indispensabile che il controller associato alla view sia \texttt{AppController} e che le keywords che saranno visualizzate a video siano del tipo: \texttt{$listOfKey.nomevariabile$}. Infine è necessario inserire all'interno del file \texttt{Index}, presente nella cartella QuizziPedia/Front-End, la seguente linea di codice nella sezione Directives:
\begin{lstlisting}[language=Java,firstnumber=1]
<script src="Directives/NomeDirective.js"></script>
\end{lstlisting}


\subsubsection{Inserimento nuovo controller}  
Per inserire un nuovo \textit{controller\ped{G}} è necessario creare un file $nomecontroller.js$ all'interno della cartella Controllers del path: QuizziPedia/Front-End/Controllers. Infine si deve aggiungere all'interno del file \texttt{Index}, presente nella cartella QuizziPedia/Front-End, la seguente linea di codice nella sezione Controllers:
\begin{lstlisting}[language=Java,firstnumber=1]
	<script src="Controllers/NomeController.js"></script>
\end{lstlisting}

\subsubsection{Inserimento nuovo service}
Per inserire un nuovo \textit{service\ped{G}} è necessario creare un file $nomeservice.js$ all'interno della cartella Services del path QuizziPedia/Front-End/Services. Infine si deve aggiungere all'interno del file \texttt{Index}, presente nella cartella QuizziPedia/Front-End, la seguente linea di codice nella sezione Services:
\begin{lstlisting}[language=Java,firstnumber=1]
<script src="Services/NomeService.js"></script>
\end{lstlisting}

\subsubsection{Inserimento nuovo model}
Per inserire un nuovo \textit{model\ped{G}} è necessario creare un file $nomemodel.js$ all'interno della cartella Models del path QuizziPedia/Front-End/Models. Infine si deve aggiungere all'interno del file \texttt{Index}, presente nella cartella QuizziPedia/Front-End, la seguente linea di codice nella sezione Models:
\begin{lstlisting}[language=Java,firstnumber=1]
<script src="Models/NomeModel.js"></script>
\end{lstlisting}


\subsection{Back-End}
\subsubsection{Inserimento nuovo controller}
Per inserire un nuovo \textit{controller\ped{G}} è necessario creare un file $nomecontroller.js$ all'interno della cartella Controllers del path: QuizziPedia/Back-End/App/Controllers. E' necessario poi associare una nuova classe route creando un nuovo file $nomerouter.js$ all'interno della cartella Routes del path: QuizziPedia/Back-End/App/Routes. All'interno di quest'ultimo inserire poi la seguente riga di codice per associarlo al \textit{controller\ped{G}} appena creato:
\begin{lstlisting}[language=Java,firstnumber=1]
var NomeController = require('../Controller/NomeController.js');
\end{lstlisting}

\subsubsection{Inserimento nuovo model}
Per inserire un nuovo \textit{model\ped{G}} è necessario creare un file $nomemodel.js$ all'interno della cartella Models del path QuizziPedia/Back-End/App/Models. Infine si deve associare il nuovo model al relativo \textit{controller\ped{G}} aggiungendo la seguente riga di codice in quest'ultimo:
\begin{lstlisting}[language=Java,firstnumber=1]
var user = require('../Model/NuovoModel.js');
\end{lstlisting}