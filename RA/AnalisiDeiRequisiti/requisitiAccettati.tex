\begin{itemize}
\item \hyperlink{RFD4}{RFD4}: L’utente autenticato e l’utente
autenticato pro possono gestire il proprio
profilo;
\item \hyperlink{RFD4.1}{RFD4.1}: L’utente autenticato e l’utente
autenticato pro possono modificare il
proprio nome;
\item \hyperlink{RFD4.2}{RFD4.2}: L’utente autenticato e l’utente
autenticato pro possono modificare il
proprio cognome;
\item \hyperlink{RFD4.3}{RFD4.3}: L’utente autenticato e l’utente
autenticato pro possono inserire una
foto/immagine;
\item \hyperlink{RFD4.4}{RFD4.4}: L’utente autenticato e l’utente
autenticato pro possono modificare la
propria e-mail;
\item \hyperlink{RFD4.5}{RFD4.5}: L’utente autenticato e l’utente
autenticato pro possono modificare la
propria password;
\item \hyperlink{RFD4.5.1}{RFD4.5.1}: L’utente autenticato e l’utente
autenticato pro possono inserire la vecchia password;
\item \hyperlink{RFD4.5.2}{RFD4.5.2}: L’utente autenticato e l’utente
autenticato pro possono inserire la nuova password;
\item \hyperlink{RFD4.5.3}{RFD4.5.3}: L’utente autenticato e l’utente
autenticato pro possono inserire nuovamente la nuova password;
\item \hyperlink{RFD4.6}{RFD4.6}: L’utente autenticato e l’utente
autenticato pro possono confermare le
modifiche al proprio profilo utente;
\item \hyperlink{RFD4.7}{RFD4.7}: Il sistema deve visualizzare un messaggio
di errore nel caso l’utente autenticato o
l’utente autenticato pro abbiano effettuato modifiche
non permesse al proprio profilo utente;
\item \hyperlink{RFD4.8}{RFD4.8}: L’utente autenticato e l’utente
autenticato pro possono cambiare la
propria tipologia di account;
\item \hyperlink{RFD4.8.1}{RFD4.8.1}: L’utente autenticato e l’utente
autenticato pro possono selezionare la nuova tipologia di account;
\item \hyperlink{RFD4.8.2}{RFD4.8.2}: L’utente autenticato e l’utente autenticato pro possono inviare la richiesta di passaggio alla nuova tipologia di account;
\item \hyperlink{RFD4.9}{RFD4.9}: L’utente autenticato e l’utente autenticato pro possono eliminare il proprio account;
\item \hyperlink{RFD4.9.1}{RFD4.9.1}: L’utente autenticato e l’utente
autenticato pro possono confermare
l’eliminazione del proprio account;
\item \hyperlink{RFD5}{RFD5}: L’utente autenticato e l’utente
autenticato pro possono cercare un
questionario tramite la barra di ricerca;
\item \hyperlink{RFD6.1}{RFD6.1}: L’utente autenticato e l’utente
autenticato pro a partire da una
domanda possono scegliere di spostarsi
alla domanda successiva del questionario;
\item \hyperlink{RFD6.4}{RFD6.4}: L’utente autenticato e l’utente
autenticato pro possono concludere il
questionario confermando le risposte date
alle domande che lo compongono;
\item \hyperlink{RFD7.4}{RFD7.4}: L’utente autenticato e l’utente autenticato pro possono scegliere di modificare una domanda tramite editor di testo \textit{QML\ped{G}};
\item \hyperlink{RFD7.4.1}{RFD7.4.1}: L’utente autenticato e l’utente autenticato pro possono confermare la modifica di una domanda tramite editor di testo \textit{QML\ped{G}};
\item \hyperlink{RFD7.5}{RFD7.5}: L’utente autenticato e l’utente
autenticato pro possono scegliere un
argomento da assegnare alla nuova
domanda;
\item \hyperlink{RFD7.6}{RFD7.6}: L’utente autenticato e l’utente
autenticato pro possono inserire delle
parole chiave relative alla nuova domanda;
\item \hyperlink{RFD8.1}{RFD8.1}: L’utente autenticato pro può visualizzare
i questionari creati;
\item \hyperlink{RFD8.5}{RFD8.5}: L’utente autenticato pro può rendere il
questionario compilabile da parte degli
esaminandi;
\item \hyperlink{RFD8.6.3}{RFD8.6.3}: L’utente autenticato pro può inserire il
nome del questionario ;
\item \hyperlink{RFD8.6.4.1}{RFD8.6.4.1}: L’utente autenticato pro può consultare il
resoconto del questionario dopo aver
deciso di concluderlo;
\item \hyperlink{RFD8.8}{RFD8.8}: L’utente autenticato pro può gestire la
iscrizione degli esaminandi ai questionari;
\item \hyperlink{RFD8.8.1}{RFD8.8.1}: L’utente autenticato pro può selezionare
il questionario del quale gestire le
iscrizioni;
\item \hyperlink{RFD8.8.2}{RFD8.8.2}: L’utente autenticato pro può accettare le
iscrizioni degli esaminandi ai questionari;
\item \hyperlink{RFD9.1}{RFD9.1}: L’utente non autenticato, l’utente
autenticato e l’utente autenticato pro
possono decidere un argomento per fare
un allenamento;
\item \hyperlink{RFD9.2}{RFD9.2}: L’utente non autenticato, l’utente
autenticato e l’utente autenticato pro
possono decidere delle parole chiave per
filtrare maggiormente le domande poste
durante l’allenamento;
\item \hyperlink{RFD9.3}{RFD9.3}: L’utente non autenticato, l’utente
autenticato e l’utente autenticato pro
possono scegliere il numero di domande
che comporranno l’allenamento
(potenzialmente anche infinite domande);
\item \hyperlink{RFD9.4}{RFD9.4}: L’utente non autenticato, l’utente
autenticato e l’utente autenticato pro
possono rispondere alle domande
proposte iniziando l’allenamento;
\item \hyperlink{RFD9.4.1}{RFD9.4.1}: L’utente non autenticato, l’utente
autenticato e l’utente autenticato pro
possono confermare una risposta durante
un allenamento;
\item \hyperlink{RFD9.4.6}{RFD9.4.6}: L’utente non autenticato, l’utente
autenticato e l’utente autenticato pro
possono decidere di terminare
l’allenamento in qualunque momento;
\item \hyperlink{RFD9.5}{RFD9.5}: Il sistema sceglie una domanda in base
all’abilità dell’avversario sull’argomento
scelto;
\item \hyperlink{RFD9.6}{RFD9.6}: Il sistema aggiorna automaticamente i
dati sull’abilità dell’utente ad ogni
risposta;
\item \hyperlink{RFD9.7}{RFD9.7}: Il sistema aggiorna automaticamente i
dati sulla difficoltà di una domanda
quando un utente risponde alla medesima;
\item \hyperlink{RFD10}{RFD10}: L’utente autenticato e l’utente
autenticato pro possono visualizzare il
proprio profilo;
\item \hyperlink{RFD10.1}{RFD10.1}: L’utente autenticato e l’utente
autenticato pro possono andare alla
pagina di gestione del profilo mediante
l’apposito link;
\item \hyperlink{RFD10.2}{RFD10.2}: L’utente autenticato e l’utente
autenticato pro possono andare alla
pagina di gestione delle domande
mediante l’apposito link;
\item \hyperlink{RFD10.3}{RFD10.3}: L’utente autenticato pro può andare alla
pagina di gestione dei questionari
mediante l’apposito link;
\item \hyperlink{RFD10.4}{RFD10.4}: L’utente autenticato e l’utente
autenticato pro possono visualizzare la
cronologia di tutti i questionari che
hanno svolto;
\item \hyperlink{RFD10.4.1}{RFD10.4.1}: L’utente autenticato e l’utente
autenticato pro possono selezionare e
visualizzare le statistiche di un
questionario scelto dalla cronologia;
\item \hyperlink{RFD10.5}{RFD10.5}: L’utente autenticato e l’utente
autenticato pro possono visualizzare la
lista dei questionari abilitati;
\item \hyperlink{RFD10.5.1}{RFD10.5.1}: L’utente autenticato e l’utente
autenticato pro possono selezionare un
questionario abilitato;
\item \hyperlink{RFD10.6}{RFD10.6}: L’utente autenticato e l’utente
autenticato pro possono tornare alla
home page mediante l’apposito link;
\item \hyperlink{RFD10.7}{RFD10.7}: L’utente autenticato e l’utente
autenticato pro possono visualizzare il proprio username;
\item \hyperlink{RFD10.8}{RFD10.8}: L’utente autenticato e l’utente
autenticato pro possono visualizzare la propria immagine profilo;
\item \hyperlink{RFD10.9}{RFD10.9}: L’utente autenticato e l’utente
autenticato pro possono visualizzare il proprio livello attuale;
\item \hyperlink{RFD10.10}{RFD10.10}: L’utente autenticato e l’utente
autenticato pro possono visualizzare il numero di domande risposte in
modo esatto;
\item \hyperlink{RFD10.11}{RFD10.11}: L’utente autenticato e l’utente autenticato pro possono visualizzare il numero di domande risposte in totale;
\item \hyperlink{RFD11.4}{RFD11.4}: L’utente non autenticato, l’utente
autenticato e l’utente autenticato pro
possono rispondere ad una domanda di
collegamento;
\item \hyperlink{RFD11.4.1}{RFD11.4.1}: L’utente non autenticato, l’utente
autenticato e l’utente autenticato pro
possono collegare le voci;
\item \hyperlink{RFD11.5}{RFD11.5}: L’utente non autenticato, l’utente
autenticato e l’utente autenticato pro
possono ordinare delle immagini;
\item \hyperlink{RFD11.5.1}{RFD11.5.1}: L’utente non autenticato, l’utente
autenticato e l’utente autenticato pro
possono inserire un’immagine in uno
spazio già occupato oppure no;
\item \hyperlink{RFD11.6}{RFD11.6}: L’utente non autenticato, l’utente
autenticato e l’utente autenticato pro
possono ordinare delle stringhe;
\item \hyperlink{RFD11.6.1}{RFD11.6.1}: L’utente non autenticato, l’utente
autenticato e l’utente autenticato pro
possono inserire una stringa in uno spazio
già occupato oppure no;
\item \hyperlink{RFD12.1}{RFD12.1}: L’utente autenticato e l’utente
autenticato pro possono inserire nome e
cognome oppure lo username nella barra
di ricerca per ricercare un utente;
\item \hyperlink{RFD12.2}{RFD12.2}: L’utente autenticato e l’utente
autenticato pro possono selezionare
l’utente ricercato;
\item \hyperlink{RFD12.3}{RFD12.3}: L’utente autenticato e l’utente
autenticato pro possono visualizzare il profilo dell'utente ricercato;
\item \hyperlink{RFD12.3.1}{RFD12.3.1}: L’utente autenticato e l’utente autenticato pro possono visualizzare l'username dell'utente ricercato;
\item \hyperlink{RFD12.3.2}{RFD12.3.2}: L’utente autenticato e l’utente autenticato pro possono visualizzare l'immagine profilo dell'utente ricercato;
\item \hyperlink{RFD12.3.3}{RFD12.3.3}: L’utente autenticato e l’utente autenticato pro possono visualizzare il livello attuale dell'utente ricercato;
\item \hyperlink{RFD12.3.4}{RFD12.3.4}: L’utente autenticato e l’utente autenticato pro possono visualizzare il numero di domande risposte in modo corretto dall'utente ricercato;
\item \hyperlink{RFD12.3.5}{RFD12.3.5}: L’utente autenticato e l’utente autenticato pro possono visualizzare il numero di domande risposte in totale dall'utente ricercato;
\item \hyperlink{RFD12.3.6}{RFD12.3.6}: Il sistema deve mostrare un messaggio di errore in caso la ricerca degli utenti non sia andata a buon fine;
\item \hyperlink{RFD17}{RFD17}: Il sistema deve mostrare un messaggio di errore in caso la ricerca dei questionari non sia andata a buon fine;
\item \hyperlink{RFD18}{RFD18}: L’utente autenticato e l’utente
autenticato pro possono iscriversi ad un
questionario;
\item \hyperlink{RFD18.1}{RFD18.1}: L’utente autenticato e l’utente
autenticato pro possono confermare
l’iscrizione ad un questionario;
\item \hyperlink{RFD23.5}{RFD23.5}: Il sistema attraverso il linguaggio \textit{QML\ped{G}} deve gestire esercizi di ordinamento di
scelte;
\item \hyperlink{RFD23.6}{RFD23.6}: Il sistema attraverso il linguaggio \textit{QML\ped{G}} deve gestire esercizi a corrispondenza di
scelte;
\item \hyperlink{RFD28}{RFD28}: Il sistema deve archiviare i risultati dei
questionari;
\item \hyperlink{RFD29}{RFD29}: Il sistema deve archiviare le statistiche
delle risposte date ad ogni domanda;
\item \hyperlink{RFD31}{RFD31}: Il sistema deve creare questionari
dinamicamente (modalità allenamento)
per un argomento scegliendo le domande
in modo casuale;
\item \hyperlink{RFD32}{RFD32}: Il sistema deve creare questionari
dinamicamente (modalità allenamento)
scegliendo le domande in base alle
risposte date dai questionari precedenti;
\item \hyperlink{RFD33}{RFD33}: Il sistema deve creare questionari
dinamicamente (modalità allenamento)
scegliendo tra le domande più difficili;
\item \hyperlink{RFD34}{RFD34}: Il sistema deve creare questionari
dinamicamente (modalità allenamento)
scegliendo tra le lacune dei partecipanti;
\item \hyperlink{RFD35}{RFD35}: Il sistema deve permettere agli
utilizzatori di proporre nuove domande;
\item \hyperlink{RFF38}{RFF38}: Il sistema deve permettere agli
utilizzatori di rispondere più volte ad una
domanda;
\item \hyperlink{RFF40}{RFF40}: Il sistema deve permettere di visualizzare l'intera applicazione in lingue divese;
\item \hyperlink{RFD41}{RFD41}: Il sistema deve permettere all'utente di far conoscere chi ha sviluppato il sistema e di far capire che cos'è QuizziPedia;
\item \hyperlink{RVD4}{RVD4}: L’applicazione deve utilizzare \textit{fogli di stile\ped{G}} in \textit{CSS3\ped{G}};
\item \hyperlink{RVD13}{RVD13}: L’applicazione deve funzionare su \textit{Google Chrome per iOS\ped{G}} versione 39 o superiore per le funzionalità che riguardano la compilazione dei questionari e delle domande;
\item \hyperlink{RVF15}{RVF15}: L’applicazione deve funzionare su \textit{Mozilla Firefox per Android\ped{G}} versione 33 o superiore per le funzionalità che riguardano la compilazione dei questionari e delle domande;
\item \hyperlink{RVD18}{RVD18}: L’applicazione deve funzionare su \textit{Google Chrome per Android\ped{G}} versione 39 o superiore per le funzionalità che riguardano la creazione dei questionari e delle domande;
\item \hyperlink{RVF19}{RVF19}: L’applicazione deve funzionare su \textit{Mozilla Firefox per Android\ped{G}} versione 33 o superiore per le funzionalità che riguardano la creazione dei questionari e delle domande;
\item \hyperlink{RVF21}{RVF21}: L’applicazione deve funzionare su \textit{Browser Opera per Android\ped{G}} versione 34 o superiore per le funzionalità che riguardano la creazione dei questionari e delle domande;
\end{itemize}