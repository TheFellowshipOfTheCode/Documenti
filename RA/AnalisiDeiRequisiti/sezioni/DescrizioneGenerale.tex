\newpage
\section{Descrizione generale}
\subsection{Contesto d'uso del prodotto}
Il prodotto dovrà essere utilizzabile su un qualsiasi \textit{browser\ped{G}}\footnote[1]{Vedi la \hyperref[TBS1]{"Tabella Browser Supportati"} in Appendice.} compatibile con la tecnologia \textit{HTML5\ped{G}}, \textit{CSS3\ped{G}} e \textit{JavaScript\ped{G}} sia da desktop che da mobile, senza alcuna limitazione sul sistema operativo.

\subsection{Funzioni del prodotto}
Il prodotto consiste in un applicativo web per la creazione e compilazione di questionari. Ogni utente iscritto avrà la possibilità di creare domande singole, che potranno essere inserite da un utente pro nel proprio questionario. Questo verrà poi inserito nell'archivio interno e diventerà accessibile, agli utenti che vi si iscriveranno, in seguito all'approvazione dell'autore. In fase di compilazione di un questionario l'utente sarà libero di rispondere alle domande nell'ordine che preferisce, avendo inoltre la possibilità di cambiare le risposte già date. Dopo aver dato una conferma per le risposte date, l'utente visualizzerà una schermata contente il riepilogo di queste e un voto. Sarà infine possibile, anche per gli utenti che non sono autenticati, fare pratica nella modalità allenamento, potendo scegliere sia l'argomento che il numero delle domande a cui si vuole rispondere.

\subsection{Caratteristiche degli utenti}
Non sono richieste competenze particolari per poter utilizzare questo prodotto, che deve risultare quindi accessibile ad un ampia categoria di utenti. L'interfaccia dovrà quindi essere il più semplice e intuitiva possibile, senza però limitare le funzionalità offerte dal software stesso. Per questo motivo verrà fornito anche un \textit{\MU} con tutte le indicazioni necessarie per consentire un utilizzo corretto ed efficace del prodotto.
\subsection{Vincoli generali}
Tutti i questionari dovranno essere creati e compilati sempre e solo all'interno del servizio offerto.
\subsection{Assunzione dipendenze}
Per il corretto funzionamento dell'applicazione sarà necessario l'utilizzo di un \textit{browser\ped{G}}\textsuperscript{1} che sia compatibile con gli standard \textit{HTML5\ped{G}}, \textit{CSS3\ped{G}} e \textit{JavaSript\ped{G}}.
