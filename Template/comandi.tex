%Generali
\newcommand{\progetto}{QuizziPedia}
\newcommand{\gruppo}{TheFellowshipOfTheCode}
\newcommand{\gruppoLink}{\href{http://thefellowshipofthecode.github.io/}{TheFellowshipOfTheCode}}
\newcommand{\email}{\href{mailto:thefellowshipofthecode@gmail.com}{thefellowshipofthecode@gmail.com}}

%Documenti
\newcommand{\AdR}{Analisi dei Requisiti}
\newcommand{\AdRvDue}{AnalisiDeiRequisiti\_v\_2\_0\_0}
\newcommand{\NdP}{Norme di Progetto}
\newcommand{\PdP}{Piano di Progetto}
\newcommand{\SdF}{Studio di Fattibilità}
\newcommand{\PdQ}{Piano di Qualifica}
\newcommand{\VI}{Verbale Interno}
\newcommand{\VE}{Verbale Esterno}
\newcommand{\ST}{Specifica Tecnica}
\newcommand{\DDP}{Definizione di Prodotto}
\newcommand{\MU}{Manuale Utente}
\newcommand{\G}{Glossario}
\newcommand{\LdP}{Lettera di Presentazione}
\newcommand{\NdPv}{NormeDiProgetto\_v\_4\_0\_0}
\newcommand{\PdPv}{PianoDiProgetto\_v\_5\_0\_0}
\newcommand{\PdQv}{PianoDiQualifica\_v\_5\_0\_0}
\newcommand{\SdFv}{StudioDiFattibilità\_v\_1\_0\_0}
\newcommand{\DdPv}{DefinizioneDiprodotto\_v\_3\_0\_0}

%Componenti del gruppo
\newcommand{\AF}{Alberto Ferrara}
\newcommand{\SM}{Simone Magagna}
\newcommand{\FB}{Franco Berton}
\newcommand{\MP}{Marco Prelaz}
\newcommand{\MV}{Mattia Varotto}
\newcommand{\GN}{Matteo Gnoato}
\newcommand{\GR}{Matteo Granzotto}

%Ruoli
\newcommand{\RdP}{Responsabile di Progetto}
\newcommand{\Res}{Responsabile}
\newcommand{\Amm}{Amministratore}
\newcommand{\Ver}{Verificatore}
\newcommand{\Prog}{Progettista}
\newcommand{\Progr}{Programmatore}
\newcommand{\Ana}{Analista}
\newcommand{\RdPs}{Responsabili di Progetto}
\newcommand{\Ress}{Responsabile}
\newcommand{\Amms}{Amministratori}
\newcommand{\Vers}{Verificatori}
\newcommand{\Progs}{Progettisti}
\newcommand{\Progrs}{Programmatori}
\newcommand{\Anas}{Analisti}

%Professori e proponente
\newcommand{\TV}{Prof. Tullio Vardanega}
\newcommand{\RC}{Prof. Riccardo Cardin}
\newcommand{\ZU}{Zucchetti S.P.A.}
\newcommand{\proponente}{Zucchetti S.P.A.}

\newcommand{\diaryEntry}[5]{#2 & \emph{#4} & #3 & #5 & #1\\ \hline}

%comando per una nuova riga nella tabella del diario delle modifiche
\newcommand{\specialcell}[2][c]{%
	\begin{tabular}[#1]{@{}c@{}}#2\end{tabular}}

\renewcommand*\sectionmark[1]{\markboth{#1}{}}
\renewcommand*\subsectionmark[1]{\markright{#1}}

%Pediodi di lavoro 
\newcommand{\AR}{Analisi dei Requisiti}
\newcommand{\AD}{Analisi dei Requisiti in Dettaglio}
\newcommand{\PA}{Progettazione Architetturale}
\newcommand{\PD}{Progettazione di Dettaglio}
\newcommand{\CO}{Codifica}
\newcommand{\VV}{Validazione}

% Revisioni
\newcommand{\RR}{Revisione dei Requisiti}
\newcommand{\RP}{Revisione di Progettazione}
\newcommand{\RQ}{Revisione di Qualifica}
\newcommand{\RA}{Revisione di Accettazione}

% Comandi analisi dei requisiti
\newcommand{\uau}{utente autenticato}
\newcommand{\uaus}{utenti autenticati}
\newcommand{\uaupro}{utente autenticato pro}
\newcommand{\uauspro}{utenti autenticati pro}

\newcommand{\myincludegraphics}[2][]{%
	\setbox0=\hbox{\phantom{X}}%
	\vtop{
		\hbox{\phantom{X}}
		\vskip-\ht0
		\hbox{\includegraphics[#1]{#2}}}}

\renewcommand\footnoterule{\rule{\linewidth}{1pt}}

\newcommand{\nogloxy}[1]{#1} % comando da usare per evitare di metttere il mark del glossario
\newcommand{\gloxy}[1]{\emph{#1}$_G$}

\colorlet{punct}{red!60!black}
\definecolor{background}{HTML}{EEEEEE}
\definecolor{delim}{RGB}{20,105,176}
\colorlet{numb}{magenta!60!black}
\lstdefinelanguage{json}{
	basicstyle=\small\ttfamily,
	numbers=left,
	numberstyle=\scriptsize,
	stepnumber=1,
	numbersep=8pt,
	showstringspaces=false,
	breaklines=true,
	frame=lines,
	backgroundcolor=\color{background},
	literate=
	*{0}{{{\color{numb}0}}}{1}
	{1}{{{\color{numb}1}}}{1}
	{2}{{{\color{numb}2}}}{1}
	{3}{{{\color{numb}3}}}{1}
	{4}{{{\color{numb}4}}}{1}
	{5}{{{\color{numb}5}}}{1}
	{6}{{{\color{numb}6}}}{1}
	{7}{{{\color{numb}7}}}{1}
	{8}{{{\color{numb}8}}}{1}
	{9}{{{\color{numb}9}}}{1}
	{:}{{{\color{punct}{:}}}}{1}
	{,}{{{\color{punct}{,}}}}{1}
	{\{}{{{\color{delim}{\{}}}}{1}
	{\}}{{{\color{delim}{\}}}}}{1}
	{[}{{{\color{delim}{[}}}}{1}
	{]}{{{\color{delim}{]}}}}{1},
}
\lstset{language=json}
\lstset{literate=%
	{Ö}{{\"O}}1
	{Ä}{{\"A}}1
	{Ü}{{\"U}}1
	{é}{{\"s}}1
	{è}{{\"e}}1
	{à}{{\"a}}1
	{ö}{{\"o}}1
}