\section{Riassunto della riunione}
\subsection{Descrizione}

Riunione tenuta per discutere sulla natura del linguaggio di mark-up QML e sulle sue funzionalità all'interno del sistema.
Dalla discussione interna e su suggerimento del proponente (attraverso lo scambio di posta elettronica), è emersa la necessità di aggiungere all'\AdR {} nuovi casi d'uso riguardanti la creazione delle domande tramite QML.

\subsection{Decisioni prese}
\begin{itemize}
\item VI6.1: il linguaggio di mark-up QML sarà creato a partire dal metalinguaggio XML;
\item VI6.2: aggiunta del Caso d'uso UC8.3: Creazione nuova domanda tramite editor di testo QML, all'interno dell'\AdR;
\item VI6.3: aggiunta del Caso d'uso UC8.3.1: Conferma creazione, all'interno dell'\AdR;
\item VI6.4: aggiunta del Caso d'uso UC8.4: Modifica domanda esistente tramite editor di testo QML, all'interno dell'\AdR;
\item VI6.5: aggiunta del Caso d'uso UC8.4.1: Conferma modifica, all'interno dell'\AdR.
\end{itemize}
