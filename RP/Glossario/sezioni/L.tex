\section{L}
\begin{itemize}
	\item
	\textbf{Label}: è un controllo grafico (widget) che mostra informazioni testuali all'interno di una form. È solitamente un controllo statico, che non prevede alcun tipo di interazione con l'utente, ed è usato per identificare (etichettare) un altro controllo grafico o gruppi di controlli grafici. 
	\item
	\textbf{\LaTeX}: è un linguaggio di markup utilizzato per la produzione di documentazione tecnica e scientifica. \LaTeX è lo standard de facto per la comunicazione e la pubblicazione di documenti scientifici. \LaTeX è disponibile come software libero.
	\item
	\textbf{Lato client}: nell'ambito delle reti di calcolatori, il termine lato client indica le operazioni di elaborazione effettuate da un client in un'architettura client-server.
	\item
	\textbf{Lato server}: nell'ambito delle reti di calcolatori, il termine lato server indica le operazioni di elaborazione effettuate dal server in un'architettura client-server.
	\item
	\textbf{Lexxer}: è un programma, o una parte di un programma (vedi compilatori e parser), che si occupa di analizzare lessicalmente un dato input, genericamente il codice sorgente di un programma. Quindi il compito di un analizzatore lessicale è di analizzare uno stream di caratteri in input e produrre in uscita uno stream di token.
	\item
	\textbf{Linguaggio di markup}: è un insieme di regole che descrive i meccanismi di rappresentazione (strutturali, semantici o presentazionali) di un testo che, utilizzando convenzioni standardizzate, è utilizzabile su più supporti.
	\item
	\textbf{Linguaggio di scripting}: un linguaggio di scripting, in informatica, è un linguaggio di programmazione interpretato destinato in genere a compiti di automazione del sistema operativo o delle applicazioni, o a essere usato all'interno delle pagine web.
	I programmi sviluppati con questi linguaggi sono detti script, termine della lingua inglese utilizzato in ambito teatrale per indicare il testo (anche detto canovaccio) in cui sono tracciate le parti che devono essere interpretate dagli attori.
	\item
	\textbf{Linux}: è una famiglia di sistemi operativi di tipo Unix-like, rilasciati sotto varie possibili distribuzioni, aventi la caratteristica comune di utilizzare come nucleo il kernel Linux.
	\item
	\textbf{Logger}: framework utilizzato durante l'esecuzione di un programma per registrare e riportare informazioni sul sistema, messaggi di errore e tracciamento dell'output.
\end{itemize}