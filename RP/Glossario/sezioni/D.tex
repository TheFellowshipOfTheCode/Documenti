\section{D}
\begin{itemize} 
	\item
	\textbf{Debugger}: è un programma/software specificatamente progettato per l'analisi e l'eliminazione dei bug (debugging), ovvero errori di programmazione interni al codice di altri programmi. Assieme al compilatore è fra i più importanti strumenti di sviluppo a disposizione di un programmatore, spesso compreso all'interno di un ambiente integrato di sviluppo (IDE), in quanto in grado di aiutare il programmatore ad individuare errori di semantica all'interno del codice sorgente del programma, altrimenti di difficile individuazione in fase di runtime.
	\item
	\textbf{Design Pattern}: è un concetto che può essere definito "una soluzione progettuale generale ad un problema ricorrente". Si tratta di una descrizione o modello logico da applicare per la risoluzione di un problema che può presentarsi in diverse situazioni durante le fasi di progettazione e sviluppo del software, ancor prima della definizione dell'algoritmo risolutivo della parte computazionale. È un approccio spesso efficace nel contenere o ridurre il debito tecnico.
	I design pattern orientati agli oggetti tipicamente mostrano relazioni ed interazioni tra classi o oggetti, senza specificare le classi applicative finali coinvolte, risiedendo quindi nel dominio dei moduli e delle interconnessioni. Ad un livello più alto sono invece i pattern architetturali che hanno un ambito ben più ampio, descrivendo un pattern complessivo adottato dall'intero sistema, la cui implementazione logica dà vita ad un framework.
	\item
	\textbf{Device}: può indicare degli elementi concernenti un sistema operativo:
	 \begin{itemize} 
	\item Device: astrazione di un dispositivo hardware al quale il sistema operativo fornisce accesso mediante un driver; nei sistemi Unix storicamente si sono distinti i block device (i dispositivi che accedono ai dati in gruppi di byte e non necessariamente in sequenza) dai character device (ai quali l'astrazione in oggetto fornisce primitive per accedere ai byte strettamente in sequenza);
 	\item Device file: un'interfaccia ad un driver che appare nel file system come se fosse un file ordinario; anch'essi sono divisi storicamente in block e character.
	\end{itemize}
	\item
	\textbf{Diagramma di Gantt}: il diagramma di Gantt è usato principalmente nelle attività di project management. \'E costruito partendo da un asse orizzontale, a rappresentazione dell'arco temporale totale del progetto, suddiviso in fasi incrementali; e da un asse verticale, a rappresentazione delle mansioni o attività che costituiscono il progetto.
	Un diagramma di Gantt permette la rappresentazione grafica di un calendario di attività, utile al fine di pianificare, coordinare e tracciare specifiche attività in un progetto dando una chiara illustrazione dello stato d'avanzamento del progetto rappresentato;
	\item
	\textbf{Directive}: esse rappresentano un modo di estendere il linguaggio HTML con elementi e attributi personalizzati. Questo si riallaccia inoltre all’ottica per cui ogni componente Angular ha un suo ruolo specifico. In particolare il ruolo delle direttive consiste nell’estendere le potenzialità dell’approccio dichiarativo dell’HTML nella costruzione di interfacce grafiche.
	\item
	\textbf{Driver}: l'insieme di procedure, spesso scritte in assembly, che permette ad un sistema operativo di dialogare con un dispositivo hardware attraverso un'interfaccia che astrae dall'implementazione fisica e che ne considera soltanto il funzionamento logico.
	\item
	\textbf{DSL}: In telecomunicazioni il termine DSL (sigla dell'inglese Digital Subscriber Line) è una famiglia di tecnologie che fornisce trasmissione digitale di dati attraverso l'ultimo miglio della rete telefonica fissa, ovvero su doppino telefonico dalla prima centrale di commutazione fino all'utente finale e viceversa.
	Si tratta dunque di una tecnologia di accesso tramite la rispettiva rete di accesso telefonica a servizi di trasferimento dati comunemente utilizzata nella connessione ad Internet da utenza domestica nella sua specifica più diffusa come l'ADSL. 
\end{itemize}