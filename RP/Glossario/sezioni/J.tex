\section{J}
\begin{itemize} 
	\item
	\textbf{JavaScript}: è un linguaggio di scripting orientato agli oggetti e agli eventi, comunemente utilizzato nella programmazione Web lato client per la creazione, in siti web e applicazioni web, di effetti dinamici interattivi tramite funzioni di script invocate da eventi innescati a loro volta in vari modi dall'utente sulla pagina web in uso (mouse, tastiera eccetera).
	\item
	\textbf{Jison}: è un generatore di parser JavaScript;
	\item
	\textbf{JQuery}: è una libreria JavaScript per applicazioni web. Nasce con l'obiettivo di semplificare la selezione, la manipolazione, la gestione degli eventi e l'animazione di elementi DOM in pagine HTML, nonché implementare funzionalità AJAX.
	\item
	\textbf{JSHint}: JSHint è uno strumento di community-driven per rilevare gli errori e potenziali problemi di codice JavaScript e per rispettare le convenzioni di codifica. E' molto flessibile, in modo da poter regolare facilmente particolari linee guida di codifica. JSHint è open source.
	\item
	\textbf{JSON}: JSON, acronimo di JavaScript Object Notation, è un formato adatto all'interscambio di dati fra applicazioni client-server.
\end{itemize}