\section{G}
\begin{itemize}
	\item
	\textbf{Getter}: è un tipo di metodo pubblico utilizzato per esporre il valore di un membro privato di una classe, che non sarebbe accessibile direttamente. 
	\item
	\textbf{Git}: è un sistema software di controllo di versione distribuito, creato da Linus Torvalds nel 2005.
	La progettazione di Git è stata ispirata da BitKeeper e da Monotone. \'E stato pensato inizialmente solamente come motore a basso livello che altri potevano sviluppare applicazioni avanzate. In seguito è però diventato un sistema di controllo di versione direttamente utilizzabile da riga di comando. Vari progetti software adesso usano Git per tale attività, principalmente il kernel Linux.
	\item
	\textbf{GitHub}: è un servizio web di hosting per lo sviluppo di progetti software, che usa il sistema di controllo di versione Git. Può essere utilizzato anche per la condivisione e la modifica di file di testo e documenti revisionabili. 
	\item
	\textbf{Google Chrome}: è un browser sviluppato da Google, basato sul motore di rendering Blink. \'E disponibile per sistemi operativi Windows, Linux, Mac OS X, Android e iOS. 
	\item
	\textbf{Google Chrome DevTools}: gli strumenti di Chrome Developer, sono un insieme di web authoring e strumenti di debug integrati in Google Chrome. Questi strumenti offrono agli sviluppatori web la possibilità di visualizzare le parti interne di un sito internet o di una applicazione web. Utilizzare i DevTools permette di rintracciare in modo efficace problemi di layout, impostare punti di interruzione per gli script scritti in JavaScript e ottenere spunti per ottimizzare il codice.
	\item
	\textbf{Google Drive}: è un servizio, in ambiente cloud computing, di memorizzazione e sincronizzazione online introdotto da Google il 24 aprile 2012. Il servizio comprende il file hosting, il file sharing e la modifica collaborativa di documenti fino a 15 GB gratuiti estendibili fino a 30 TB in totale. Il servizio può essere usato via Web, caricando e visualizzando i file tramite il web browser, oppure tramite l'applicazione installata su computer che sincronizza automaticamente una cartella locale del file system con quella condivisa. Su Google Drive sono presenti anche i documenti creati con Google Documenti.
\end{itemize}