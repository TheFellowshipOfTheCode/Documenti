\section{M}
\begin{itemize} 
	\item
	\textbf{Mac OS}: è il sistema operativo di Apple dedicato ai computer Macintosh; il nome è l'acronimo di Macintosh Operating System.
	\item
	\textbf{Material for Angular}: per gli sviluppatori Angular, è sia un framework per le componenti UI che un'implementazione di riferimento di Google's Material Design Specification. Questo progetto fornisce un insieme di componenti dell'interfaccia utente riutilizzabili, ben collaudato, e un insieme di componenti UI basati sul Material Design.
	\item
	\textbf{Media}: è un dispositivo di memorizzazione su cui si registrano informazioni (dati). Il termine è utilizzato soprattutto in riferimento a file, audio e video, ma teoricamente la registrazione può avvenire con qualunque grandezza fisica, anche un foglio di carta per scrivere.
	\item
	\textbf{Mean}: è un framework javascript fullstack presuntuoso - che semplifica e accelera lo sviluppo di web application. MEAN mette insieme quattro tra le più utilizzate (e apprezzate) tecnologie per lo sviluppo JavaScript, gettando le fondamenta per semplificare lo sviluppo di web application complesse.
	
	\item
	\textbf{Microsoft Edge}: è un browser web sviluppato da Microsoft e incluso in Windows 10. Ufficialmente presentato il 21 gennaio 2015, ha sostituito Internet Explorer come browser predefinito di Windows. Sarà anche il browser predefinito per Windows 10 Mobile, la versione per smartphone e tablet.
	\item
	\textbf{Microsoft Windows}: è una famiglia di ambienti operativi e sistemi operativi dedicati ai personal computer, alle workstation, ai server e agli smartphone. Il sistema operativo si chiama così per via della sua interfaccia a finestre.
	\'E software proprietario della Microsoft Corporation che lo rende disponibile esclusivamente a pagamento.
	\item
	\textbf{Middleware}: si intende un insieme di programmi informatici che fungono da intermediari tra diverse applicazioni e componenti software. Sono spesso utilizzati come supporto per sistemi distribuiti complessi.
	\item
	\textbf{Milestone}: indica importanti traguardi intermedi nello svolgimento del progetto. Molto spesso sono rappresentate da eventi, cioè da attività con durata zero o di un giorno, e vengono evidenziate in maniera diversa dalle altre attività nell'ambito dei documenti di progetto. Esempi di milestone sono: la fine dei collaudi di un impianto, la firma di un contratto, il varo di una nave, la fine delle opere di muratura di un edificio, eccetera.
	\item
	\textbf{Model}: il model fornisce i metodi per accedere ai dati utili all'applicazione;
	\item
	\textbf{Modello ad attori}: il modello ad attori in informatica è un modello matematico di calcolo concorrente che tratta "attori" come i primitivi universali di computazione concorrente: in risposta ad un messaggio che riceve, un attore può prendere decisioni locali, creare più attori, inviare più messaggi, e capire come rispondere al messaggio successivo ricevuto. 
	\item
	\textbf{Modelview}: è il componente intermediario tra la vista e il modello nel pattern MVVM, ed è responsabile per la gestione della logica della vista. In genere, il view model interagisce con il modello invocando metodi nelle classi del modello. Il view model fornisce quindi i dati dal modello in una forma che la vista può usare facilmente.
	\item
	\textbf{MongoDB}: è un DBMS non relazionale, orientato ai documenti. Classificato come un database di tipo NoSQL, MongoDB si allontana dalla struttura tradizionale basata su tabelle dei database relazionali in favore di documenti in stile JSON con schema dinamico (BSON), rendendo l'integrazione di dati di alcuni tipi di applicazioni più facile e veloce. 
	\item
	\textbf{Mongoose}: è un web server multipiattaforma. Mongoose viene distribuito sotto licenza commerciale e GPLv2. 
	\item 
	\textbf{Mozilla Firefox}: è un web browser opensource multipiattaforma prodotto da Mozilla Foundation. \'E disponibile per sistemi operativi Windows, Linux, Mac OS X, Android e Firefox OS.
	\item
	\textbf{MVC}: è un pattern architetturale molto diffuso nello sviluppo di sistemi software, in particolare nell'ambito della programmazione orientata agli oggetti, in grado di separare la logica di presentazione dei dati dalla logica di business.
	\item
	\textbf{MVVM}:  è un pattern architetturale molto diffuso nello sviluppo di sistemi software. Consiste nella separazione degli aspetti dell'applicazione in tre componenti: Model,
	View, e	ViewModel, permettendoci di evitare di mescolare il codice che fornisce la logica a quello che gestisce la UI.
	Il Model rappresenta il punto di accesso ai dati. Trattasi di una o più classi che leggono dati dal DB, oppure da un servizio Web di qualsivoglia natura.
	La View rappresenta la vista dell’applicazione, l’interfaccia grafica che mostrerà i dati.
	Il ViewModel è il punto di incontro tra la View e il Model: i dati ricevuti da quest’ultimo sono elaborati per essere presentati e passati alla View.
	Il fulcro del funzionamento di questo pattern è la creazione di un componente, il ViewModel appunto, che rappresenta tutte le informazioni e i comportamenti della corrispondente View. La View si limita infatti, a visualizzare graficamente quanto esposto dal ViewModel, a riflettere in esso i suoi cambi di stato oppure ad attivarne dei comportamenti. 
\end{itemize}
