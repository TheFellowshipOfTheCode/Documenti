\section{M}
\begin{itemize} 
	\item
	\textbf{Mac OS}: è il sistema operativo di Apple dedicato ai computer Macintosh; il nome è l'acronimo di Macintosh Operating System.
	\item
	\textbf{Media}: è un dispositivo di memorizzazione su cui si registrano informazioni (dati). Il termine è utilizzato soprattutto in riferimento a file, audio e video, ma teoricamente la registrazione può avvenire con qualunque grandezza fisica, anche un foglio di carta per scrivere.
	\item
	\textbf{Mean}: è un framework javascript fullstack presuntuoso - che semplifica e accelera lo sviluppo di web application. MEAN mette insieme quattro tra le più utilizzate (e apprezzate) tecnologie per lo sviluppo JavaScript, gettando le fondamenta per semplificare lo sviluppo di web application complesse.
	
	\item
	\textbf{Microsoft Edge}: è un browser web sviluppato da Microsoft e incluso in Windows 10. Ufficialmente presentato il 21 gennaio 2015, ha sostituito Internet Explorer come browser predefinito di Windows. Sarà anche il browser predefinito per Windows 10 Mobile, la versione per smartphone e tablet.
	\item
	\textbf{Microsoft Windows}: è una famiglia di ambienti operativi e sistemi operativi dedicati ai personal computer, alle workstation, ai server e agli smartphone. Il sistema operativo si chiama così per via della sua interfaccia a finestre.
	\'E software proprietario della Microsoft Corporation che lo rende disponibile esclusivamente a pagamento.
	\item
	\textbf{Middleware}: si intende un insieme di programmi informatici che fungono da intermediari tra diverse applicazioni e componenti software. Sono spesso utilizzati come supporto per sistemi distribuiti complessi.
	\item
	\textbf{Milestone}: indica importanti traguardi intermedi nello svolgimento del progetto. Molto spesso sono rappresentate da eventi, cioè da attività con durata zero o di un giorno, e vengono evidenziate in maniera diversa dalle altre attività nell'ambito dei documenti di progetto. Esempi di milestone sono: la fine dei collaudi di un impianto, la firma di un contratto, il varo di una nave, la fine delle opere di muratura di un edificio, eccetera.
	\item
	\textbf{Modello ad attori}: il modello ad attori in informatica è un modello matematico di calcolo concorrente che tratta "attori" come i primitivi universali di computazione concorrente: in risposta ad un messaggio che riceve, un attore può prendere decisioni locali, creare più attori, inviare più messaggi, e capire come rispondere al messaggio successivo ricevuto. 
	\item
	\textbf{MongoDB}: è un DBMS non relazionale, orientato ai documenti. Classificato come un database di tipo NoSQL, MongoDB si allontana dalla struttura tradizionale basata su tabelle dei database relazionali in favore di documenti in stile JSON con schema dinamico (BSON), rendendo l'integrazione di dati di alcuni tipi di applicazioni più facile e veloce. 
	\item
	\textbf{Mongoose}: è un web server multipiattaforma. Mongoose viene distribuito sotto licenza commerciale e GPLv2. 
	\item 
	\textbf{Mozilla Firefox}: è un web browser opensource multipiattaforma prodotto da Mozilla Foundation. \'E disponibile per sistemi operativi Windows, Linux, Mac OS X, Android e Firefox OS. 
\end{itemize}
