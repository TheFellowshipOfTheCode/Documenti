\newpage
\section{Resoconto attività di verifica}
In questa sezione del documento vengono descritti e analizzati gli esiti delle attività di verifica svolte su tutti i documenti che vengono consegnati nelle varie revisioni di avanzamento del progetto.

\subsection{Revisione dei Requisiti}

\subsubsection{Tracciamento}
Il \textit{team\ped{G}} ha deciso di utilizzare il software interno \textit{DocumentsDB} in modo da facilitare il tracciamento sia delle relazioni fra casi d'uso e requisiti, sia delle relazioni fra requisiti e fonti.

\subsubsection{Analisi statica dei documenti}
L'analisi dei documenti mediante \textit{Walkthrough\ped{G}} ha portato all'individuazione di alcuni errori frequenti a partire dai quali è stata stilata una lista di controllo che è stata inserita all'interno dei Processi Organizzativi nelle \textit{\NdP}. Grazie a questa sarà possibile applicare l'\textit{Inspection\ped{G}} per le future attività di verifica.

\subsubsection{Esiti verifiche automatizzate}
Vengono qui riportati gli esiti delle verifiche automatizzate per il calcolo dell'\textit{indice Gulpease\ped{G}}, alle quali sono stati sottoposti tutti i documenti.
\begin{table}[h]
\begin{center}
\begin{tabular}{|c|c|c|c|}
\hline Documento & Indice Gulpease & Esito\\
\hline
\emph{Norme di Progetto} & 68 & Superato \\
\emph{Studio di Fattibilità} & 56 & Superato \\
\emph{Piano di Progetto} & 60 & Superato \\
\emph{Piano di Qualifica} & 57  & Superato \\
\emph{Analisi dei Requisiti} & 72 & Superato \\
\emph{Glossario} & 48 & Superato \\
\emph{Verbale Interno 2015-12-03} & 75 & Superato \\
\emph{Verbale Interno 2015-12-11} & 74 & Superato \\
\emph{Verbale Interno 2015-12-29} & 75 & Superato \\
\emph{Verbale Interno 2016-01-12} & 75 & Superato \\
\emph{Verbale Esterno 2016-01-11} & 76 & Superato \\
\hline
\end{tabular}
\caption{Resoconto verifiche automatizzate - Revisione dei Requisiti}
\end{center}
\end{table}

\subsection{Revisione di Progettazione}

\subsubsection{Tracciamento}
Il \textit{team\ped{G}} attraverso il software interno \textit{DocumentsDB} è riuscito a effettuare il tracciamento sia delle relazioni fra requisiti e componenti che fra requisiti e classi. Questo software è stato utilizzato inoltre per generare le tabelle dei vari tipi di test e dei relativi tracciamenti con requisiti, componenti, classi e metodi.

\subsubsection{Analisi statica dei documenti}
L'analisi dei documenti mediante \textit{Walkthrough\ped{G}} ha portato all'individuazione di alcuni errori frequenti a partire dai quali è stata stilata una lista di controllo che è stata inserita all'interno dei Processi Organizzativi nelle \textit{\NdP}. Grazie a questa sarà possibile applicare l'\textit{Inspection\ped{G}} per le future attività di verifica.

\subsubsection{Esiti verifiche automatizzate}
Vengono qui riportati gli esiti delle verifiche automatizzate per il calcolo dell'\textit{indice Gulpease\ped{G}}, alle quali sono stati sottoposti tutti i documenti.
\begin{table}[h]
\begin{center}
\begin{tabular}{|c|c|c|c|}
\hline Documento & Indice Gulpease & Esito\\
\hline
\emph{Definizione di Prodotto} & 61 & Superato \\
\emph{Norme di Progetto} &  & Superato \\
\emph{Piano di Progetto} &  & Superato \\
\emph{Piano di Qualifica} &   & Superato \\
\emph{Analisi dei Requisiti} &  & Superato \\
\emph{Glossario} &  & Superato \\
\emph{Verbale} &  & Superato \\
\hline
\end{tabular}
\caption{Resoconto verifiche automatizzate - Revisione di Progettazione}
\end{center}
\end{table}

