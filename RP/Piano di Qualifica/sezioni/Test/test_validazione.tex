\subsubsection{Test di Validazione}
I test di validazione hanno lo scopo di accertare che tutte le funzionalità richieste dal proponente siano state soddisfatte. Per questo motivo, attraverso delle macro azioni, si andrà a simulare il comportamento generale dell'applicativo e dell'utente che interagisce con esso.
I test di validazione saranno organizzati nel modo seguente:
\begin{center}
\textbf{TV}[\textit{TipoRequisito}][\textit{ImportanzaRequisito}][\textit{IdRequisito}]
\end{center}
dove:
\begin{itemize}
\item
\textbf{TipoRequisito} può assumere valori tra:
	\begin{itemize}
	\item
	\textit{F} per i requisiti funzionali;
	\item
	\textit{Q} per i requisiti di qualità;
	\item
	\textit{V} per i requisiti di vincolo;
	\item
	\textit{P} per i requisiti prestazionali.
	\end{itemize}
\item 
\textbf{ImportanzaRequisito} può assumere valori tra:
	\begin{itemize}
	\item
	\textit{D} per i requisiti desiderabili;
	\item
	\textit{O} per i requisiti di obbligatori;
	\item
	\textit{F} per i requisiti di facoltativi.
	\end{itemize}
\item
\textbf{IdRequisito} assume un valore gerarchico che identifica il singolo requisito.
\end{itemize}

% TABELLA
\normalsize
\begin{longtable}[ht]{|c|>{}m{8cm}|c|}
\hline 
\textbf{Id Test} & \textbf{Descrizione} & \textbf{Stato}\\
\hline
\endhead
\hypertarget{TVFO1}{TVFO1} & L’utente intende registrarsi al sistema. All’utente è richiesto di:
\begin{itemize}
\item
Trovarsi nella sezione apposita;
\item
Compilare il form di registrazione;
\item
Premere il pulsante di conferma.
\item
Verificare attraverso l’autenticazione che la registrazione sia effettivamente avvenuta.
\end{itemize}
 & \textit{Non Implementato}\\ \hline
\hypertarget{TVFO2}{TVFO2} & L’utente intende autenticarsi al sistema. All’utente è richiesto di:
\begin{itemize}
\item
Trovarsi nella sezione apposita;
\item
Inserire le credenziali nell’apposito form;
\item
Premere il pulsante di autenticazione;
\item
Verificare che l’autenticazione sia effettivamente avvenuta.
\end{itemize}
 & \textit{Non Implementato}\\ \hline
\hypertarget{TVFO3}{TVFO3} & L’utente intende disconnettersi dal sistema. All’utente è richiesto di:
\begin{itemize}
\item
Essere Autenticato;
\item
Trovarsi nella sezione apposita;
\item
Premere il pulsante di logout;
\item
Verificare che la disconnessione sia effettivamente avvenuta. 
\end{itemize}
& \textit{Non Implementato}\\ \hline
\hypertarget{TVFD4}{TVFD4} & L’utente autenticato  intende gestire i propri dati. All’utente è richiesto di:
\begin{itemize}
\item
Essere autenticato;
\item
Trovarsi nella sezione apposita;
\item
Modificare i campi dati consentiti;
\item
Premere il tasto conferma modifica;
\item
Visualizzare il profilo dell'utente modificato.
\end{itemize}
 & \textit{Non Implementato}\\ \hline
\hypertarget{TVFD4.3}{TVFD4.3} & L’utente autenticato  intende impostare la propria foto profilo . All’utente è richiesto di:
\begin{itemize}
\item
Essere autenticato;
\item
Trovarsi nella sezione apposita;
\item
Impostare la foto da inserire;
\item
Visualizzare il profilo dell’utente modificato.
\end{itemize}
 & \textit{Non Implementato}\\ \hline
\hypertarget{TVFD4.8}{TVFD4.8} & L’utente autenticato  intende modificare la tipologia di utenza. All’utente è richiesto di:
\begin{itemize}
\item Essere autenticato;
\item Trovarsi nella sezione apposita;
\item Cambiare la tipologia di utenza;
\item Verificare la modifica effettuata. 
\end{itemize}& \textit{Non Implementato}\\ \hline
\hypertarget{TVFD4.9}{TVFD4.9} & L’utente autenticato  intende eliminare il proprio account. All’utente è richiesto di:
\begin{itemize}
\item Essere autenticato;
\item Trovarsi nella sezione apposita;
\item Premere il tasto di elimanazione;
\item Verificare la disconnessione della sessione sia avvenuta;
\item Verificare che l’autenticazione con le credenziali dell’utente eliminato non avvenga.
\end{itemize} & \textit{Non Implementato}\\ \hline
\hypertarget{TVFD5}{TVFD5} & L’utente autenticato  intende ricercare un questionario per iscriversi. All'utente è richiesto di:
\begin{itemize}
\item Essere autenticato;
\item Trovarsi nella sezione apposita;
\item Ricercare un questionario;
\item Iscrizione ad un questionario trovato mediante l’apposito tasto;
\item Verificare l'operazione appena effettuata.
\end{itemize}
 & \textit{Non Implementato}\\ \hline
\hypertarget{TVFO6}{TVFO6} & L’utente autenticato  intende compilare un questionario a cui si è  iscritto. All’utente è richiesto di:
\begin{itemize}
\item Essere autenticato;
\item Trovarsi nella sezione apposita;
\item Selezionare il questionario da compilare;
\item Rispondere alle domande previste;
\item Confermare le risposte date al questionario;
\item Verificare il risultato del questionario.
\end{itemize}
 & \textit{Non Implementato}\\ \hline
\hypertarget{TVFD7.1}{TVFD7.1} & L’utente autenticato intende creare una nuova
domanda tramite procedura guidata. All’utente è richiesto di:
\begin{itemize}
\item Essere autenticato;
\item Trovarsi nella sezione apposita;
\item Premere il pulsante Crea nuova domanda;
\item Inserire i dati necessari alla creazione di una nuova domanda;
\item Premere il pulsante di conferma creazione
nuova domanda;
\item Verificare che sia stato creata la nuova domanda.
\end{itemize}

 & \textit{Non Implementato}\\ \hline
\hypertarget{TVFD7.2}{TVFD7.2} & L’utente autenticato intende modificare una
domanda esistente tramite procedura guidata. All’utente è richiesto di:
\begin{itemize}
\item Essere autenticato;
\item Trovarsi nella sezione apposita;
\item Premere il pulsante  modifica domanda;
\item Modificare i dati della domanda selezionata;
\item Premere il pulsante di conferma modifica;
\item Verificare che sia stata modificata la domanda.
\end{itemize}
 & \textit{Non Implementato}\\ \hline
\hypertarget{TVFO7.3}{TVFO7.3} & L’utente autenticato intende creare una nuova
domanda tramite QML. All’utente è richiesto di:
\begin{itemize}
\item Essere autenticato;
\item Trovarsi nella sezione apposita;
\item Premere il pulsante Crea nuova domanda;
\item Inserire i dati necessari alla creazione di una nuova domanda;
\item Premere il pulsante di conferma creazione
nuova domanda;
\item Verificare che sia stato creata la nuova domanda.
\end{itemize}
 & \textit{Non Implementato}\\ \hline
\hypertarget{TVFD7.4}{TVFD7.4} & L’utente autenticato intende modificare una
domanda esistente tramite QML. All’utente è richiesto di:
\begin{itemize}
\item Essere autenticato;
\item Trovarsi nella sezione apposita;
\item Premere il pulsante  modifica domanda;
\item Modificare i dati della domanda selezionata;
\item Premere il pulsante di conferma modifica;
\item Verificare che sia stata modifcata la domanda.
\end{itemize}
 & \textit{Non Implementato}\\ \hline
\hypertarget{TVFD8.1}{TVFD8.1} & L’utente autenticato intende visualizzare i questionari creati. All’utente è richiesto di:
\begin{itemize}
\item
Essere autenticato;
\item
Trovarsi nella sezione apposita;
\item
Verificare che vengano visualizzati tutti i questionari creati
\end{itemize}
 & \textit{Non Implementato}\\ \hline
\hypertarget{TVFD8.2}{TVFD8.2} & 
L’utente autenticato pro  intende modificare un questionario esistente. All’utente è richiesto di:
\begin{itemize}
\item Essere autenticato pro;
\item Trovarsi nella sezione apposita;
\item Selezionare il questionario da modificare tra quelli creati;
\item Dalla sezione di modifica effettuare tutte le modifiche consentite dal sistema 
\item Premere il pulsante di conferma modifiche;
\item Verificare che sia stato modificato il questionario correttamente.
\end{itemize}
 & \textit{Non Implementato}\\ \hline
\hypertarget{TVFD8.3}{TVFD8.3} & L’utente autenticato pro  intende eliminare un questionario esistente. All’utente è richiesto di:
\begin{itemize}
\item Essere autenticato pro;
\item Trovarsi nella sezione apposita;
\item Selezionare un questionario esistente da eliminare;
\item Premere il tasto di conferma eliminazione;
\item Verificare l’operazione appena effettuata. 
\end{itemize}
& \textit{Non Implementato}\\ \hline
\hypertarget{TVFD8.4}{TVFD8.4} & L’utente autenticato pro  intende controllare i risultati degli esaminandi di un suo questionario. All’utente è richiesto di:
\begin{itemize}
\item Essere autenticato pro;
\item Trovarsi nella sezione apposita;
\item Selezionare il questionario da controllare gli esiti;
\item Verificare che si ci siano gli esiti di tutti gli utenti iscritti.
\end{itemize}
 & \textit{Non Implementato}\\ \hline
\hypertarget{TVFD8.5}{TVFD8.5} & L’utente autenticato pro  intende attivare la modalità esame di un suo questionario. All’utente è richiesto di:
\begin{itemize}
\item Essere autenticato pro;
\item Trovarsi nella sezione apposita;
\item Selezionare il questionario da attivare;
\item Verificare l'operazione effettuata.
\end{itemize} & \textit{Non Implementato}\\ \hline
\hypertarget{TVFO8.6}{TVFO8.6} & L’utente autenticato pro  intende creare  un nuovo questionario. All’utente è richiesto di:
\begin{itemize}
\item Essere autenticato pro;
\item Trovarsi nella sezione apposita;
\item Premere il pulsante Crea nuovo questionario;
\item Inserire il titolo del nuovo questionario;
\item Inserire l’argomento del nuovo questionario;
\item Ricercare e selezionare le domande che andranno a comporre il questionario;
\item Premere il pulsante di conferma creazione nuovo questionario;
\item Verificare che sia stato creato il nuovo questionario.
\end{itemize} & \textit{Non Implementato}\\ \hline
\hypertarget{TVFO8.7}{TVFO8.7} & L’utente autenticato pro  intende gestire le domande del questionario. All’utente è richiesto di:
\begin{itemize}
\item Essere autenticato pro;
\item Trovarsi nella sezione apposita;
\item Premere il pulsante per gestire il questionario;
\item Effettuare le operazioni consentite;
\item Verificare che le operazioni siano effettivamente avvenute.
\end{itemize} & \textit{Non Implementato}\\ \hline
\hypertarget{TVFD8.8}{TVFD8.8} & L’utente autenticato pro  intende gestire le iscrizione di un suo questionario. All’utente è richiesto di:
\begin{itemize}
\item Essere autenticato pro;
\item Trovarsi nella sezione apposita;
\item Selezionare il questionario da gestire le iscrizioni;
\item Effettuare le operazioni consentite;
\item Verificare le operazioni apportate.
\end{itemize} & \textit{Non Implementato}\\ \hline
\hypertarget{TVFO9}{TVFO9} & L’utente intende esercitarsi effettuando un allenamento. All’utente è richiesto di:
\begin{itemize}
\item Trovarsi nella sezione apposita;
\item Selezionare gli appositi filtri per focalizzare l’allenamento che si vuole effettuare;
\item Premere il pulsante per iniziare l’allenamento.
\item Rispondere alle domande proposte e controllare l’esito delle risposte date.
\end{itemize} & \textit{Non Implementato}\\ \hline
\hypertarget{TVFD10}{TVFD10} & L’utente autenticato  intende visualizzare il proprio profilo. All’utente è richiesto di:
\begin{itemize}
\item Essere autenticato;
\item Trovarsi nella sezione apposita;
\item Visualizzare il proprio profilo.
\end{itemize}
 & \textit{Non Implementato}\\ \hline
\hypertarget{TVFD10.4}{TVFD10.4} & L’utente autenticato  intende visualizzare la cronologia dei questionari svolti. All’utente è richiesto di:
\begin{itemize}
\item Essere autenticato;
\item Trovarsi nella sezione apposita;
\item Visualizzare la cronologia dei questionari svolti.
\end{itemize}
 & \textit{Non Implementato}\\ \hline
\hypertarget{TVFD10.4.1}{TVFD10.4.1} & L’utente autenticato  intende visualizzare il riepilogo di un questionario svolto. All’utente è richiesto di:
\begin{itemize}
\item Essere autenticato;
\item Trovarsi nella sezione apposita;
\item Selezionare il questionario svolto da visualizzare il riepilogo. 
\end{itemize}
& \textit{Non Implementato}\\ \hline
\hypertarget{TVFO11}{TVFO11} & L’utente  intende rispondere alle domande. All’utente è richiesto di:
\begin{itemize}
\item Essere autenticato;
\item Trovarsi nella sezione apposita;
\item Rispondere alle domande;
\item Verificare la funzionalità dell'operazione.
\end{itemize}
 & \textit{Non Implementato}\\ \hline
\hypertarget{TVFO12}{TVFO12} & L’utente autenticato  intende ricercare un utente. All’utente è richiesto di:
\begin{itemize}
\item Essere autenticato;
\item Trovarsi nella sezione apposita;
\item Inserire le keywords nella barra di ricerca;
\item Premere l'apposito tasto per effettuare la ricerca;
\item Verificare l’operazione appena effettuata.
\end{itemize}
 & \textit{Non Implementato}\\ \hline
\caption[Test di Validazione]{Test di Validazione}
\label{tabella:test0}
\end{longtable}
\clearpage
