\newpage
\section{Introduzione}

\subsection{Scopo del documento}
Il presente documento ha lo scopo di definire in dettaglio la struttura e il funzionamento delle componenti del progetto \progetto. Questo documento servirà come guida per i \textit{\Progrs} del gruppo \gruppo fornendo direttive e vincoli per la realizzazione del \textit{progetto\ped{G}}.

\subsection{Scopo del prodotto}
Lo scopo del prodotto è di permettere la creazione e gestione di questionari in grado di identificare le lacune dei candidati prima, durante e al termine di un corso di formazione. 
\\Il sistema dovrà offrire le seguenti funzionalità:
\begin{itemize}
	\item
	Archiviare questionari in un server suddivisi per argomento;
	\item
	Somministrare all'utente, tramite un'interfaccia, questionari specifici per argomento scelto;
	\item
	Verificare e valutare i questionari scelti dagli utenti in base alle risposte date.
\end{itemize}
La parte destinata ai creatori di questionari dovrà essere fruibile attraverso un \textit{browser\ped{G}} desktop, abilitato all'utilizzo delle tecnologie \textit{HTML5\ped{G}}, \textit{CSS3\ped{G}} e \textit{JavaScript\ped{G}}. La parte destinata agli esaminandi sarà utilizzabile su qualunque dispositivo: dal personal computer ai tablet e smartphone.

\subsection{Glossario}
Al fine di evitare ogni ambiguità i termini tecnici del dominio del progetto, gli acronimi e le parole che necessitano di ulteriori spiegazioni saranno nei vari documenti marcate con il pedice \ped{G} e quindi presenti nel documento \textit{\G}.


\subsection{Riferimenti}
\subsubsection{Normativi}
\begin{itemize}
	\item \textit{\NdP v2.0.0};
	\item \textit{\AdR v2.0.0};
	\item \textbf{\VE: }.
\end{itemize}
\subsubsection{Informativi}
\begin{itemize}
	\item \textbf{Ingegneria del software - Ian Sommerville - 8a edizione (2007)}: \\
	Parte terza: Progettazione, capitolo 11: Progettazione architetturale, Capitolo 14: Progettazione orientata agli oggetti;
	\item \textbf{Design Patterns} - Erich Gamma, Richard Helm, Ralph Johnson, John Vlissides - 1a edizione italiana (2006);
	\item \textbf{Slide dell'insegnamento - Design patterns:}
	\begin{itemize}
		\item Introduzione:
		\item Strutturali:
		\item Creazionali:
		\item Comportamentali:
		\item Architetturali:
	\end{itemize}
	\item \textbf{Martin Fowler - UML\ped{G} Distilled} - 2nd edition;
	\item \textbf{Slide dell'insegnamento - Diagrammi delle classi}:
	\item \textbf{Slide dell'insegnamento - Diagrammi dei packages:}
	\item \textbf{Slide dell'insegnamento - Diagrammi di sequenza:}
	\item \textbf{Documentazione del \textit{Framework\ped{G}MEAN\ped{G}.js}:}
	\item \textbf{Documentazione della \textit{piattaforma\ped{G}} Node.js:}
	\item \textbf{Giuda all'utilizzo dei middleware Express:}
	\item \textbf{Guida all'utilizzo dei middleware Passport:}
	\item \textbf{Manuale del database \textit{MongoDB\ped{G}}:}
	\item \textbf{Documentazione del \textit{framework\ped{G} AngularJS\ped{G}}:}
	\begin{itemize}
		\item \textit{Documentazione generica:}
	\end{itemize}
	\item \textbf{Documentazione del \textit{framework\ped{G}} Material for Angular:}
	\item \textbf{Documentazione di \textit{ANTLR\ped{G}} per la definizione della grammatica di QML} \\
	\url{http://www.antlr.org/} \\
	\url{https://github.com/antlr/antlr4/blob/master/doc/index.md}
\end{itemize}