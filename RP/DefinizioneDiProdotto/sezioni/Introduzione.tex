\newpage
\section{Introduzione}

\subsection{Scopo del documento}
Il presente documento ha lo scopo di definire in dettaglio la struttura e il funzionamento delle componenti del progetto \progetto. Questo documento servirà come guida per i \textit{\Progrs} del gruppo \gruppo fornendo direttive e vincoli per la realizzazione del \textit{progetto\ped{G}}.

\subsection{Scopo del prodotto}
Lo scopo del prodotto è di permettere la creazione e gestione di questionari in grado di identificare le lacune dei candidati prima, durante e al termine di un corso di formazione. 
\\Il sistema dovrà offrire le seguenti funzionalità:
\begin{itemize}
	\item
	Archiviare questionari in un server suddivisi per argomento;
	\item
	Somministrare all'utente, tramite un'interfaccia, questionari specifici per argomento scelto;
	\item
	Verificare e valutare i questionari scelti dagli utenti in base alle risposte date.
\end{itemize}
La parte destinata ai creatori di questionari dovrà essere fruibile attraverso un \textit{browser\ped{G}} desktop, abilitato all'utilizzo delle tecnologie \textit{HTML5\ped{G}}, \textit{CSS3\ped{G}} e \textit{JavaScript\ped{G}}. La parte destinata agli esaminandi sarà utilizzabile su qualunque dispositivo: dal personal computer ai tablet e smartphone.

\subsection{Glossario}
Al fine di evitare ogni ambiguità i termini tecnici del dominio del progetto, gli acronimi e le parole che necessitano di ulteriori spiegazioni saranno nei vari documenti marcate con il pedice \ped{G} e quindi presenti nel documento \textit{\G}.


\subsection{Riferimenti}
\subsubsection{Normativi}
\begin{itemize}
	\item \textit{\NdPv};
	\item \textit{\AdRvDue};
\end{itemize}
\subsubsection{Informativi}
\begin{itemize}
	\item \textbf{Ingegneria del software - Ian Sommerville - 8a edizione (2007)}: \\
	Parte terza: Progettazione, capitolo 11: Progettazione architetturale, Capitolo 14: Progettazione orientata agli oggetti;
	\item \textbf{Design Patterns} - Erich Gamma, Richard Helm, Ralph Johnson, John Vlissides - 1a edizione italiana (2006);
	\item \textbf{Slide dell'insegnamento - Design patterns:}
	\begin{itemize}
		\item Strutturali: \url{http://www.math.unipd.it/~tullio/IS-1/2015/Dispense/E07.pdf}
		\item Creazionali: \url{http://www.math.unipd.it/~tullio/IS-1/2015/Dispense/E08.pdf}
		\item Comportamentali: \url{http://www.math.unipd.it/~tullio/IS-1/2015/Dispense/E09.pdf}
		\item Architetturali:
			\begin{itemize}
				\item \url{http://www.math.unipd.it/~rcardin/sweb/Design\%20Pattern\%20Architetturali\%20-\%20Model\%20View\%20Controller_4x4.pdf};
				\item \url{http://www.math.unipd.it/~rcardin/sweb/Design\%20Pattern\%20Architetturali\%20-\%20Dependency\%20Injection_4x4.pdf}.
			\end{itemize} 
	\end{itemize}
	\item \textbf{Martin Fowler - UML\ped{G} Distilled} - 2nd edition;
	\item \textbf{Slide dell'insegnamento - Diagrammi delle classi}: \\
		\url{http://www.math.unipd.it/~tullio/IS-1/2015/Dispense/E03.pdf}
	\item \textbf{Slide dell'insegnamento - Diagrammi dei packages:} \\
		\url{http://www.math.unipd.it/~tullio/IS-1/2015/Dispense/E04.pdf}
	\item \textbf{Slide dell'insegnamento - Diagrammi di sequenza:} \\
		\url{http://www.math.unipd.it/~tullio/IS-1/2015/Dispense/E05.pdf}
	\item \textbf{Documentazione del \textit{Framework\ped{G}MEAN\ped{G}.js}:} \\
		\url{http://learn.mean.io/}
	\item \textbf{Documentazione della \textit{piattaforma} Node.js:} \\
		\url{https://nodejs.org/api/}
	\item \textbf{Giuda all'utilizzo dei middleware Express:} \\
		\url{http://expressjs.com/it/guide/using-middleware.html}
	\item \textbf{Guida all'utilizzo dei middleware Passport:} \\
		\url{http://passportjs.org/docs}
	\item \textbf{Manuale del database \textit{MongoDB\ped{G}}:} \\
		\url{https://docs.mongodb.org/manual/}
	\item \textbf{Documentazione dell'interfaccia REST:}
		\begin{itemize}
			\item \textit{Descrizione di REST:} \url{https://it.wikipedia.org/wiki/Representational_State_Transfer}
			\item \textit{Descrizione risorse REST:} \url{http://stashboard.readthedocs.org/en/latest/restapi.html}
		\end{itemize}
	\item \textbf{Documentazione del \textit{framework\ped{G} AngularJS\ped{G}}:} \\
		\begin{itemize}
			\item \textit{Documentazione generica:} \url{https://docs.angularjs.org/guide}
			\item \textit{Documentazione servizio \$http:} \url{https://docs.angularjs.org/api/ng/service/$http}
			\item \textit{Documentazione servizio \$location:} \url{https://docs.angularjs.org/api/ng/service/$location}
			\item \textit{Documentazione servizio \$windows:} \url{https://docs.angularjs.org/api/ng/service/$window}
			\item \textit{Documentazione servizio \$ruoteParams:} \url{https://docs.angularjs.org/api/ngRoute/service/$routeParams}
			\item \textit{Documentazione servizio \$q:} \url{https://docs.angularjs.org/api/ng/service/$q}
		\end{itemize}
	\item \textbf{Documentazione del \textit{framework\ped{G}} Material for Angular:} \\
		\url{https://material.angularjs.org/latest/}
	\item \textbf{Documentazione del \textit{framework\ped{G}} Chart.js} \\
		\url{http://www.chartjs.org/docs/}
	\item \textbf{Documentazione del \textit{wrapper\ped{G}} Angles.js} \\
		\url{https://github.com/gonewandering/angles}
	\item \textbf{Documentazione del \textit{framework\ped{G}} TextAngular.js} \\
		\url{https://github.com/fraywing/textAngular/wiki/textAngular-Docs-v1.1.x}
	\item \textbf{Guida all'utilizzo della direttiva \textit{ng-file-upload}} \\
		\url{https://github.com/danialfarid/ng-file-upload}
	\item \textbf{Documentazione di \textit{jison\ped{G}} per la definizione della grammatica di QML} \\
		\url{http://zaa.ch/jison/docs/}
\end{itemize}