\newpage
\section{QML - Quiz Markup Language}
Uno dei punti fondamentali dell'applicazione è permettere agli utenti di poter creare domande da proporre negli allenamenti e nei questionari. Per rendere più semplice l'aggiunta di una domanda sono stati realizzati degli \textit{wizzard} per alcune tipologie di domande, ma questo limita alla creazione di tipologie di domande specifiche. La soluzione adottata è la realizzazione di un nuovo linguaggio di \textit{markup\ped{G}} che permette la definizione di domande non ordinarie, denominato \textit{QML} acronimo di \textit{Quiz Markup Language}. \\
Per la realizzazione di \textbf{QML} abbiamo adottato il \textit{ANTLR\ped{G}}, uno strumento che ci permette di definire un \textit{lexer\ped{G}} ed un \textit{parser\ped{G}} in grado di costruire e scorrere un albero di analisi al fine di generare un documento \texttt{JSON} a partire da una grammatica definita.

\subsection{Definizione della grammatica}
\subsubsection{Lexer - analizzatore lessicale}
\subsubsection{Parser - analizzatore sintattico}
\subsection{generazione del JSON}
\subsection{gestione degli errori}
\subsubsection{errori di tipo lessicale}
\subsubsection{errori di tipo sintattico}