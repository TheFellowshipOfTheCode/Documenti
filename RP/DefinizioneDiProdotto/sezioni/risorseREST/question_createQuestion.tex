	\item \texttt{/question/createQuestion}
		\begin{itemize}
			\item \textbf{Method}: POST;
			\item \textbf{Livello di Accesso}: utente autenticato/autenticato pro;
			\item \textbf{Descrizione}: fornisce le informazioni che andranno a comporre una domanda da inserire nel sistema;
			\item \textbf{Request}: la richiesta deve contenere i seguenti campi:
\begin{lstlisting}[language=json,firstnumber=1]
{
"userId" : [identificativo dell'utente che inserisce la domanda]
"tipologia" : [tipologia della domanda che si vuole inserire]
"testo" : [testo della domanda]
"option1" : [Array di stringhe contenenti le risposte]
"option2" : [Array di stringhe contenenti associazioni di risposte o la risposta esatta]
"keyword" : [Array di stringhe per indicare le parole chiavi della domanda]
}
\end{lstlisting}
			\item \textbf{Response}: la risposta, se la domanda è stata inserita correttamente, dovrà contenere i seguenti campi:
\begin{lstlisting}[language=json,firstnumber=1]
{
"status" : "ok"
}
\end{lstlisting}
		\end{itemize}