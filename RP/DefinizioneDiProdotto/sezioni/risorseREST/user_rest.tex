\item \texttt{/:lang/user/:userId}
	\begin{itemize}
		\item \textbf{Method}: DELETE;
		\item \textbf{Livello di Accesso}: Utente autenticato/autenticato pro;
		\item \textbf{Descrizione}: elimina l'utente autenticato. Restituisce un messaggio di conferma se viene effettuato correttamente, altrimenti un errore;
	\end{itemize}	
	
	\item \texttt{/:lang/user}
		\begin{itemize}
			\item \textbf{Method}: GET;
			\item \textbf{Livello di Accesso}: Utente autenticato/autenticato pro;
			\item \textbf{Descrizione}: restituisce le informazioni riguardante l'utente autenticato. Restituisce un messaggio di conferma se viene effettuato correttamente, altrimenti un errore;
			\item \textbf{Response}: la risposta deve contenere i seguenti campi:
\begin{lstlisting}[language=json,firstnumber=1]
{
"username" : [username utente da visualizzare]
"name" : [nome utente da visualizzare]
"surname" : [cognome utente da visualizzare]
"email" : [email utente da visualizzare]
"userImg": [rappresenta l'immagine dell'utente da visualizzare]
"levelUser" : [livello utente da visualizzare]
}
\end{lstlisting}
		\end{itemize}
		
	\item \texttt{/:lang/user/:userId}
		\begin{itemize}
			\item \textbf{Method}: PUT;
			\item \textbf{Livello di Accesso}: Utente autenticato/autenticato pro;
			\item \textbf{Descrizione}: modifica le informazioni riguardante l'utente autenticato. Restituisce un messaggio di conferma se viene
effettuato correttamente, altrimenti un errore;
			\item \textbf{Request}: la richiesta deve contenere i seguenti campi:
\begin{lstlisting}[language=json,firstnumber=1]
{
"name" : [nome utente]
"surname" : [cognome utente]
"email" : [email utente]
"userImg": [immagine profilo utente]
}
\end{lstlisting}
		\end{itemize}	
		
	\item \texttt{/:lang/user/:userId/privacy}
		\begin{itemize}
			\item \textbf{Method}: PUT;
			\item \textbf{Livello di Accesso}: Utente autenticato/autenticato pro;
			\item \textbf{Descrizione}: modifica la password di accesso al sistema riguardante l'utente autenticato. Restituisce un messaggio di conferma se viene effettuato correttamente, altrimenti un errore;
			\item \textbf{Request}: la richiesta deve contenere i seguenti campi:	
\begin{lstlisting}[language=json,firstnumber=1]
{
"password" : [identifica la nuova password inserita dallutente]
}	

\end{lstlisting}
		\end{itemize}
		
	\item \texttt{/:lang/user/:userId/statistics}
		\begin{itemize}
			\item \textbf{Method}: GET;
			\item \textbf{Livello di Accesso}: Utente autenticato/autenticato pro;
			\item \textbf{Descrizione}: restituisce le statistiche riguardanti l'utente autenticato; 
			\item \textbf{Response}: la risposta deve contenere i seguenti campi:	
\begin{lstlisting}[language=json,firstnumber=1]
{
"name": [identifica il nome dell'utente]
"surname": [identifica il nome dell'utente]
"userImg": [rappresenta l'immagine dell'utente]
"statistics":[array di JSON, contenente le statische dell'utente di ogni argomento]
"summaries":[array di JSON contenente la cronologia dei questionari svolti dall'utente]   
}	

\end{lstlisting}
		\end{itemize}
		
		
	\item \texttt{/:lang/user/:userId/statistics/:summaryId}
	\begin{itemize}
		\item \textbf{Method}: GET;
		\item \textbf{Livello di Accesso}: Utente autenticato/autenticato pro;
		\item \textbf{Descrizione}: restituisce le statistiche di un questionario svolto; 
		\item \textbf{Response}: la risposta deve contenere i seguenti campi:	
\begin{lstlisting}[language=json,firstnumber=1]
{
"givenAnswer" : [Arrai di JSON contenente le domande del quiz associate alle risposte date dall'utente]
"mark" : [identifica la valutazione finale del questionario svolto]
}	
\end{lstlisting}
	\end{itemize}
	