\item \texttt{/:lang/user/:userId/question}
	\begin{itemize}
		\item \textbf{Method}: GET;
		\item \textbf{Livello di Accesso}: Utente autenticato/autenticato pro;
		\item \textbf{Descrizione}: Restituisce le domande create dall'utente;
		\item \textbf{Response}: la risposta deve avere i seguenti campi:
\begin{lstlisting}[language=json,firstnumber=1]
{
"listQuestion" : [ 
"question_id": [identifica l'identificativo della domanda]
"type" : [identifica tipologia domanda]
"language" : [identifica la lingua della domanda]
"questionText" : [identifica il testo della domanda]
"image" : [identifica l'immagine relativa al testo della domanda]
"option1" : [identifica l'array di opzioni di risposte]
"option2" : [identifica l'array di opzioni di risposte]
"totalAnswer" : [identifica il numero totale di risposte date alla domanda]
"correctAnswer" : [identifica il numero di risposte corrette date alla domanda]
]
}
\end{lstlisting}
	\end{itemize}	

\item \texttt{/:lang/user/:userId/question/:questionId}
	\begin{itemize}
		\item \textbf{Method}: GET;
		\item \textbf{Livello di Accesso}: Utente autenticato/autenticato pro;
		\item \textbf{Descrizione}: Restituisce la domanda creata dall'utente;
		\item \textbf{Response}: la risposta deve avere i seguenti campi:
\begin{lstlisting}[language=json,firstnumber=1]
{
"listQuestion" : [ 
"type" : [identifica tipologia domanda]
"language" : [identifica la lingua della domanda]
"questionText" : [identifica il testo della domanda]
"image" : [identifica l'immagine relativa al testo della domanda]
"option1" : [identifica l'array di opzioni di risposte]
"option2" : [identifica l'array di opzioni di risposte]
"totalAnswer" : [identifica il numero totale di risposte date alla domanda]
"correctAnswer" : [identifica il numero di risposte corrette date alla domanda]
]
}
\end{lstlisting}
	\end{itemize}	
	
	
	\item \texttt{/:lang/user/:userId/question}
		\begin{itemize}
			\item \textbf{Method}: POST;
			\item \textbf{Livello di Accesso}: Utente autenticato/autenticato pro;
			\item \textbf{Descrizione}: Aggiunge una domanda nel sistema, restituisce un messaggio di conferma o di errore;
			\item \textbf{Request}: la richiesta deve contenere i seguenti campi:
\begin{lstlisting}[language=json,firstnumber=1]
{
"listQuestion" : [ 
"type" : [identifica tipologia domanda]
"language" : [identifica la lingua della domanda]
"questionText" : [identifica il testo della domanda]
"image" : [identifica l'immagine relativa al testo della domanda]
"option1" : [identifica l'array di opzioni di risposte]
"option2" : [identifica l'array di opzioni di risposte]
]
}
\end{lstlisting}
		\end{itemize}