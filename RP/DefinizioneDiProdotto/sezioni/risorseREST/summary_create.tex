\item \texttt{/summary/create}
		\begin{itemize}
			\item \textbf{Method}: POST;
			\item \textbf{Livello di Accesso}: utente;
			\item \textbf{Descrizione}: crea un riepilogo in base al questionario svolto e alle risposte date alle domande del questionario;
			\item \textbf{Request}: la richiesta deve avere i seguenti campi:
\begin{lstlisting}[language=json,firstnumber=1]
{
"_summaryId" : [identificativo del riepilogo]
"_quizId" : [identificativo del quiz per cui verrà creato il riepilogo]
"givenAnswers" : [array contenente le risposte date alle domande del questionario svolto]
}
\end{lstlisting}
			\item \textbf{Response}: la risposta deve avere i seguenti dati:
\begin{lstlisting}[language=json,firstnumber=1]
{
"comparedQuestions" : [array contenente booleani (true o false) che definiscono se le risposte date alle domande del questionario sono corrette o meno]
"date" : [momento in cui viene creato il riepilogo]
}
\end{lstlisting}
		\end{itemize}