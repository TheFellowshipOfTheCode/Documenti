	\item \texttt{/:lang/recovery}
		\begin{itemize}
			\item \textbf{Method}: POST;
			\item \textbf{Livello di Accesso}: Utente;
			\item \textbf{Descrizione}: invia una nuova password sulla mail dell'utente generata in automatico, restituisce un messaggio di conferma se viene effettuato correttamente, altrimenti un errore;
			\item \textbf{Request}: la richiesta di recupero della password deve avere i seguenti campi:
\begin{lstlisting}[language=json,firstnumber=1]
{
"e-mail" : [la password associata all'username dell'account dell'utente]
}
\end{lstlisting}
		\end{itemize}
		
	\item \texttt{/:lang/loggedin}
		\begin{itemize}
			\item \textbf{Method}: GET;
			\item \textbf{Livello di Accesso}: Utente;
			\item \textbf{Descrizione}: controlla se la sessione è attiva o meno.
			\item \textbf{Response}: la risposta deve avere i seguenti dati:
\begin{lstlisting}[language=json,firstnumber=1]
{
"user" : [Array di JSON contenente le informazioni riguardanti l'utente autenticato]
}
\end{lstlisting}
		\end{itemize}	