	\item \texttt{/:lang/user/:userId/search/:keyword/users}
	 \begin{itemize}
	 	\item \textbf{Method}: GET;
	 	\item \textbf{Livello di Accesso:} Utente autenticato/autenticato pro;
	 	\item \textbf{Descrizione}: effettua una ricerca di utenti da parte di un utente;
	 	\item \textbf{Request}: la richiesta deve avere i seguenti campi:
\begin{lstlisting}[language=json,firstnumber=1]
{
"keyword" : [stringa da ricercare]
}
\end{lstlisting} 
		\item \textbf{Response:} la risposta deve avere un campo \texttt{array} contenente i risultati degli utenti trovati:
\begin{lstlisting}[language=json,firstnumber=1]
{
"userResult" : [
"_userID" : [identificativi dell'utente trovato]
"name" : [nome dell'utente trovato]
"surname" : [cognome dell'utente trovato]
]
}
\end{lstlisting}
	 \end{itemize}

\item \texttt{/:lang/user/:userId/search/:keyword/quizzes}
	 \begin{itemize}
	 	\item \textbf{Method}: GET;
	 	\item \textbf{Livello di Accesso:} Utente autenticato/autenticato pro;
	 	\item \textbf{Descrizione}: effettua una ricerca di questionari da parte di un utente;
	 	\item \textbf{Request}: la richiesta deve avere i seguenti campi:
\begin{lstlisting}[language=json,firstnumber=1]
{
"keyword" : [stringa da ricercare]
}
\end{lstlisting} 
		\item \textbf{Response:} la risposta deve avere un campo \texttt{array} contenente i risultati dei questionari trovati:
\begin{lstlisting}[language=json,firstnumber=1]
{
"quizResult : [
"_quizID" : [identificativo del questionario]
"title" : [titolo del questionario]
"author" : [username dell'autore del questionario]
]
}
\end{lstlisting}
	 \end{itemize}	 
	 
	\item \texttt{/:lang/user/:userId/search/users/:userId}
	 \begin{itemize}
	 	\item \textbf{Method}: GET;
	 	\item \textbf{Livello di Accesso:} Utente autenticato/autenticato pro;
	 	\item \textbf{Descrizione}: Restituisce le informazioni dell'utente precedentemente trovato tramite una ricerca;
	 	\item \textbf{Response:} 
\begin{lstlisting}[language=json,firstnumber=1]
{
"username" : [username utente da visualizzare]
"name" : [nome utente da visualizzare]
"surname" : [cognome utente da visualizzare]
"email" : [email utente da visualizzare]
"userImg": [rappresenta l'immagine dell'utente da visualizzare]
"levelUser" : [livello utente da visualizzare]
"statistics":[array di JSON, contenente le statische dell'utente di ogni argomento]
}
\end{lstlisting}
	 \end{itemize}
	 
	 
	\item \texttt{/:lang/user/:userId/search/quizzes/:quizId}
	\begin{itemize}
		\item \textbf{Method}: GET;
		\item \textbf{Livello di Accesso:} Utente autenticato/autenticato pro;
		\item \textbf{Descrizione}: Restituisce le informazioni del questionario precedentemente trovato tramite una ricerca;
		\item \textbf{Response:} 
\begin{lstlisting}[language=json,firstnumber=1]
{
"title" : [identifica il titolo del questionario]
"author" : [identifica l'autore del questionario]
"questions" : [identifica l'Array di JSON con le domande relative al questionario]
}
\end{lstlisting}
	\end{itemize}