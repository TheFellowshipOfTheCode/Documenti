\item \texttt{/:lang/user/:userId/quiz}
	\begin{itemize}
		\item \textbf{Method}: GET;
		\item \textbf{Livello di Accesso}: Utente autenticato pro;
		\item \textbf{Descrizione}: Restituisce i questionari creati dall'utente pro;
		\item \textbf{Response}: la risposta deve avere i seguenti campi:
\begin{lstlisting}[language=json,firstnumber=1]
{
"listQuiz" : [ 
"title" : [identifica il titolo del questionario]
"questions" : [Arrai di JSON contenente le domande relative al questionario]
]
}
\end{lstlisting}
	\end{itemize}	
	
	
	\item \texttt{/:lang/user/quiz}
		\begin{itemize}
			\item \textbf{Method}: POST;
			\item \textbf{Livello di Accesso}: Utente autenticato pro;
			\item \textbf{Descrizione}: Aggiunge un questionario nel sistema, restituisce un messaggio di conferma o di errore;
			\item \textbf{Request}: la richiesta deve contenere i seguenti campi:
\begin{lstlisting}[language=json,firstnumber=1]
{
"title" : [titolo del questionario]
"questions" : [Array di JSON contenente gli identificativi delle domande che compongono il questionario]
}
\end{lstlisting}
		\end{itemize}
		
		
	\item \texttt{/:lang/user/quiz/:quizId/addUser}
	\begin{itemize}
		\item \textbf{Method}: POST;
		\item \textbf{Livello di Accesso}: Utente autenticato pro;
		\item \textbf{Descrizione}: Iscrive un utente ad un questionario, restituisce un messaggio di conferma o di errore;
		\item \textbf{Request}: la richiesta deve contenere i seguenti campi:
\begin{lstlisting}[language=json,firstnumber=1]
{
"_userId" : [identificativo dell'utente da iscrivere al questionario]
}
\end{lstlisting}
	\end{itemize}
	
	\item \texttt{/:lang/user/quiz/:quizId/activeUser}
	\begin{itemize}
		\item \textbf{Method}: POST;
		\item \textbf{Livello di Accesso}: Utente autenticato/autenticato pro;
		\item \textbf{Descrizione}: Aggiunge un utente iscritto al questionario nella lista degli utenti che lo hanno eseguito;
		\item \textbf{Request}: la richiesta deve contenere i seguenti campi:
\begin{lstlisting}[language=json,firstnumber=1]
{
"_userId" : [identificativo dell'utente da iscrivere al questionario]
}
\end{lstlisting}
	\end{itemize}
	
	
	\item \texttt{/:lang/user/quiz/:quizId/addUser}
	\begin{itemize}
		\item \textbf{Method}: POST;
		\item \textbf{Livello di Accesso}: Utente autenticato pro;
		\item \textbf{Descrizione}: Iscrive un utente ad un questionario, restituisce un messaggio di conferma o di errore;
		\item \textbf{Request}: la richiesta deve contenere i seguenti campi:
		\begin{lstlisting}[language=json,firstnumber=1]
		{
		"_userId" : [identificativo dell'utente da iscrivere al questionario]
		}
		\end{lstlisting}
	\end{itemize}
	
	
	\item \texttt{/:lang/user/:userId/quiz/:quizId}
	\begin{itemize}
		\item \textbf{Method}: PUT;
		\item \textbf{Livello di Accesso}: Utente autenticato pro;
		\item \textbf{Descrizione}: Modifica in questionario creato in precedenza, restituisce un messaggio di conferma o di errore;
		\item \textbf{Request}: la richiesta deve contenere i seguenti campi:
\begin{lstlisting}[language=json,firstnumber=1]
{
"title" : [titolo del questionario]
"questions" : [Array di JSON contenente gli identificativi delle domande che compongono il questionario]
}
\end{lstlisting}
	\end{itemize}
	
	
	\item \texttt{/:lang/user/quiz/:quizId/test}
	\begin{itemize}
		\item \textbf{Method}: POST;
		\item \textbf{Livello di Accesso}: Utente autenticato/autenticato pro;
		\item \textbf{Descrizione}: Restituisce la valutazione del questionario;
		\item \textbf{Request}: la richiesta deve contenere i seguenti campi:
\begin{lstlisting}[language=json,firstnumber=1]
{
"_userId" : [identificativo dell'utente]
"givenAnswer" : [Arrai di JSON contenente le domande del quiz associate alle risposte date dall'utente]
}
\end{lstlisting}
		\item \textbf{Response}: la risposta deve contenere i seguenti campi:
\begin{lstlisting}[language=json,firstnumber=1]
{
"mark" : [valutazione del questionario]
}
\end{lstlisting}
	\end{itemize}