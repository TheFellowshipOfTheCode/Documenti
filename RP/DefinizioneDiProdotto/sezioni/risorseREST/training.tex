	\item \texttt{/:lang/training}
		\begin{itemize}
			\item \textbf{Method}: POST;
			\item \textbf{Livello di Accesso}: Utente;
			\item \textbf{Descrizione}: restituisce una domanda in base al livello di abilità raggiunto dall'utente;
			\item \textbf{Request}: la richiesta deve contenere i seguenti campi:
\begin{lstlisting}[language=json,firstnumber=1]
{
"topicTraining" : [indica l'argomento scelto per iniziare l'allenamento]
"numberQuestions" : [indica il numero di domande che componono l'allenamento]
"keyword" : [indica un array di parole chiavi per filtrare le domande]
"levelUser" : [indica il livello di abilità sull'argomento scelto dell'utente]
}
\end{lstlisting}
			\item \textbf{Response}: la risposta deve contenere i seguenti campi:
\begin{lstlisting}[language=json,firstnumber=1]
{
"_questionId" : [identificativo della domanda]
"type" : [indica la tipologia della domanda]
"questionText" : [indica il testo della domanda]
"option1" : [indica un array di opzioni di risposta]
"option2" : [indica un array di opzioni di risposta] 
}
\end{lstlisting}
		\end{itemize}
		
		
	\item \texttt{/:lang/user/training}
		\begin{itemize}
			\item \textbf{Method}: POST;
			\item \textbf{Livello di Accesso}: Utente autenticato/autenticato pro;
			\item \textbf{Descrizione}: restituisce una domanda, aggiorna le statistiche della domanda e dell'utente;
			\item \textbf{Request}: la richiesta deve contenere i seguenti campi:
\begin{lstlisting}[language=json,firstnumber=1]
{
"topicTraining" : [indica l'argomento scelto per iniziare l'allenamento]
"numberQuestions" : [indica il numero di domande che componono l'allenamento]
"keyword" : [indica un array di parole chiavi per filtrare le domande]
"levelUser" : [indica il livello di abilità sull'argomento scelto dell'utente]
}
\end{lstlisting}
		\item \textbf{Response}: la risposta deve contenere i seguenti campi:
\begin{lstlisting}[language=json,firstnumber=1]
{
"_questionId" : [identificativo della domanda]
"type" : [indica la tipologia della domanda]
"questionText" : [indica il testo della domanda]
"option1" : [indica un array di opzioni di risposta]
"option2" : [indica un array di opzioni di risposta] 
}
\end{lstlisting}
	\end{itemize}