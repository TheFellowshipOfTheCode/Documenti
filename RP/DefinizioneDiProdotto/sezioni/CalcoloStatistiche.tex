\section{Calcolo delle Statistiche}
Di seguito sarà illustrato come verranno calcolate le statistiche di utente, di una domanda e di un argomento secondo le risposte (corrette o meno) date ad una singola domanda.

\subsection{Statistiche Utente}
\subsubsection{Utente autenticato}
L'utente autenticato possiede due tipologie di livello:
\begin{itemize}
	\item \textbf{Experience level}: livello che cresce sempre man mano che un utente risponde alle domande, secondo un sistema a punti: 
	\begin{itemize}
		\item +1 punto per ogni risposta errata;
		\item +2 punti per ogni risposta esatta;
	\end{itemize}
	Il livello sarà rappresentato da una barra d'esperienza che deve essere riempita totalmente per poter passare al livello successivo; per riempire la barra è necessario accumulare un ammontare di punti tanto maggiore quanto è alto il livello dell'utente. Il livello iniziale è 1.
	\item \textbf{Skill level}: livello che varia a seconda della risposta (corretta o errata) data dall'utente ad una domanda di un certo argomento. La variazione sarà calcolata sulla base della differenza tra lo \textit{skill level} attuale dell'utente e il \textit{difficulty level} della domanda a cui si sta rispondendo secondo i valori indicati in tabella.
	
% TABELLA
\begin{center}
\scalebox{0.88}{
\begin{tabular}{|c|>{\centering\arraybackslash}m{3cm}|>{\centering\arraybackslash}m{3cm}|>{\centering\arraybackslash}m{3cm}|>{\centering\arraybackslash}m{3cm}|}
	\hline
	\multirow{2}{*}{Differenza} & \multicolumn{2}{c|}{$0 \leq \textit{skill level} - \textit{difficulty level} \leq 100$} & \multicolumn{2}{c|}{$0 \leq \textit{difficulty level} - \textit{skill level} \leq 100$} \\ \cline{2-5}
	& corretta & errata & corretta & errata \\ \hline
	91 - 100 & +2 & -11 & +11 & -2 \\ \hline
	81 - 90 & +3 & -10 & +10 & -3 \\ \hline
	71 - 80 & +4 & -9 & +9 & -4 \\ \hline
	61 - 70 & +5 & -8 & +8 & -5 \\ \hline
	51 - 60 & +6 & -7 & +7 & -6 \\ \hline
	41 - 50 & +7 & -6 & +6 & -7 \\ \hline
	31 - 40 & +8 & -5 & +5 & -8 \\ \hline
	21 - 30 & +9 & -4 & +4 & -9 \\ \hline
	11 - 20 & +10 & -3 & +3 & -10 \\ \hline
	1 - 10 & +11 & -2 & +2 & -11 \\ \hline
	& corretta & errata \\ \cline{1-3}
	0 & +1 & -1 \\ \cline{1-3} 
\end{tabular}
}
\end{center}

Ogni utente autenticato avrà uno \textit{skill level} per ogni argomento presente nel sistema.
Inizialmente lo \textit{skill level} sarà uguale a 500 e potrà variare all'interno di un intervallo tra 0 e 1000.
\end{itemize}
I livelli precedentemente descritti verranno salvati e aggiornati automaticamente dal sistema.

\subsubsection{Utente non autenticato}
L'utente non autenticato possiederà solamente la tipologia di livello \textit{skill level}, con la differenza che non verrà salvato dal sistema al termine di un allenamento. 

\subsection{Statistiche Domanda}
La domanda possiederà un \textit{difficulty level} che varia a seconda della risposta (corretta o errata) data dall'utente alla domanda. La variazione sarà calcolata sulla base della differenza tra lo \textit{skill level} dell'utente autenticato e il \textit{difficulty level} attuale della domanda secondo i valori indicati in tabella.

% TABELLA
\begin{center}
\scalebox{0.88}{
\begin{tabular}{|c|>{\centering\arraybackslash}m{3cm}|>{\centering\arraybackslash}m{3cm}|>{\centering\arraybackslash}m{3cm}|>{\centering\arraybackslash}m{3cm}|}
	\hline
	\multirow{2}{*}{Differenza} & \multicolumn{2}{c|}{$0 \leq \textit{skill level} - \textit{difficulty level} \leq 100$} & \multicolumn{2}{c|}{$0 \leq \textit{difficulty level} - \textit{skill level} \leq 100$} \\ \cline{2-5}
	& corretta & errata & corretta & errata \\ \hline
	91 - 100 & -2 & +11 & +11 & -2 \\ \hline
	81 - 90 & -3 & +10 & +10 & -3 \\ \hline
	71 - 80 & -4 & +9 & +9 & -4 \\ \hline
	61 - 70 & -5 & +8 & +8 & -5 \\ \hline
	51 - 60 & -6 & +7 & +7 & -6 \\ \hline
	41 - 50 & -7 & +6 & +6 & -7 \\ \hline
	31 - 40 & -8 & +5 & +5 & -8 \\ \hline
	21 - 30 & -9 & +4 & +4 & -9 \\ \hline
	11 - 20 & -10 & +3 & +3 & -10 \\ \hline
	1 - 10 & -11 & +2 & +2 & -11 \\ \hline
	& corretta & errata \\ \cline{1-3}
	0 & -1 & +1 \\ \cline{1-3} 
\end{tabular}
}
\end{center}

Inizialmente il \textit{difficulty level} sarà uguale a 500 e potrà variare all'interno di un intervallo tra 0 e 1000 e soltanto se a rispondere è un utente autenticato.


\subsection{Statistiche Argomento}
La percentuale di risposte corrette ($\%R\ped{C}$) è data dal rapporto tra risposte corrette totali ($R\ped{CT}$) e risposte totali ($R\ped{T}$) date alle domande relative all'argomento.
\begin{center}
	$\%R\ped{C} = \frac{R\ped{CT}}{R\ped{T}}\times{100}$
\end{center}


\subsection{Statistiche Questionario}
Per un questionario vi saranno due statistiche:
\begin{itemize}
	\item \textbf{Generale}: la percentuale di risposte corrette ($\%R\ped{C}$) è data dal rapporto tra risposte corrette totali ($R\ped{CT}$) e risposte totali ($R\ped{T}$) date alle domande del questionario.
	\begin{center}
	$\%R\ped{C} = \frac{R\ped{CT}}{R\ped{T}}\times{100}$
	\end{center}
	\item \textbf{Per utente}: la percentuale di risposte corrette dell'utente ($\%R\ped{CU}$) è data dal rapporto tra risposte corrette totali dell'utente ($R\ped{CTU}$) e risposte totali dell'utente ($R\ped{TU}$) date alle domande del questionario.
	\begin{center}
	$\%R\ped{CU} = \frac{R\ped{CTU}}{R\ped{TU}}\times{100}$
	\end{center}
\end{itemize}
