\subsection{QuizziPedia::Back-End}
\subsubsection{Informazioni generali}
\label{QuizziPedia::Back-End}
\begin{figure}[ht]
	\centering
	\includegraphics[scale=0.45]{UML/Package/QuizziPedia_Back-End.png}
	\caption{QuizziPedia::Back-End}
\end{figure}
\FloatBarrier
	\begin{itemize}
		\item \textbf{Descrizione} \\ Package contenenti le componenti della parte back-end dell' applicazione;
		\item \textbf{Package contenuti}
		\begin{itemize}
			\item App \\
			Package\ped{G} contenente le componenti del server\ped{G} che implementano il \textit{pattern MVC\ped{G}};
			\item Config \\
			Package\ped{G} contenente le componenti di configurazione del server\ped{G}.
		\end{itemize}
	\end{itemize}
\subsubsection{Classi}
	\paragraph{QuizziPedia::Back-End::Server}
\label{QuizziPedia::Back-End::Server}
\begin{figure}[ht]
	\centering
	\includegraphics[scale=0.45]{UML/Classi/Back-End/QuizziPedia_Back-End_Server.png}
	\caption{QuizziPedia::Back-End::Server}
\end{figure}
\FloatBarrier
	\begin{itemize}
		\item \textbf{Descrizione} \\
		Classe che avvia il server\ped{G}. Nello specifico apre una connessione al database tramite Mongoose\ped{G}, invoca il \textit{middleware\ped{G}} Express\ped{G} passando un riferimento al database MongoDB\ped{G} come parametro in modo  che possa configurarsi con esso, invoca il \textit{middleware\ped{G}} Passport\ped{G} ed infine si mette in ascolto su una determinata porta. E' il componente client\ped{G} del design pattern \textit{Chain of responsibility\ped{G}}. Utilizza i moduli Mongoose\ped{G}, Express\ped{G}, Passport\ped{G};
		\item \textbf{Utilizzo} \\
		Utilizzo per avviare l'applicazione lato server\ped{G}. Inizializza, internamente al back-end, la catena di gestione delle chiamate REST\ped{G} utilizzando le classi contenute nel package\ped{G} \texttt{Routers}
	\end{itemize}
	