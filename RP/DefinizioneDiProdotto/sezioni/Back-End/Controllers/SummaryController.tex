\paragraph{QuizziPedia::Back-End::App::Controllers::SummaryController}
\label{QuizziPedia::Back-End::App::Controllers::SummaryController}
\begin{figure}
	\centering
	\includegraphics[scale=0.45]{UML/Classi/Back-End/QuizziPedia_Back-End_App_Controllers_SummaryController.png}
	\caption{QuizziPedia::Back-End::App::Models::Controllers::TopicController}
\end{figure}

\begin{itemize}
	\item \textbf{Descrizione} \\
	Classe che gestisce la logica applicativa riguardante la visualizzazione e la modifica dei riepiloghi dei questionari.
	\item \textbf{Utilizzo} \\
	Viene utilizzata per implementare le funzionalità necessarie a gestire le richieste REST legate ai riepiloghi dei questionari.
	\item \textbf{Relazione con altre classi}\\
	\begin{itemize}
			\item \textbf{OUT \texttt{SummaryModel}} \\
			Classe che modella i riepiloghi dei questionari.
	\end{itemize}
	\item \textbf{Metodi}\\
	\begin{itemize}
		\item \texttt{+ createSummary(req: Request, res: Response, next: function(QuizziPediaError))}\\
		Crea un riepilogo.\\
		\textbf{Parametri}:
		\begin{itemize}
			\item \texttt{req: Request}\\
			Rappresenta la richiesta inviata al server. Contiene l'identificativo del questionario per cui verrà creato il riepilogo e le risposte date alle domande.
			\item \texttt{res: Response}\\
			Rappresenta la risposta che il server fornirà al termine dell'esecuzione del metodo.
			\item \texttt{next: function(QuizziPediaError)}\\
			Rappresenta la \textit{callback\ped{G}} che il metodo deve chiamare al termine dell'elaborazione per passare il controllo ai successivi \textit{middleware\ped{G}}. La presenza del parametro facoltativo QuizziPediaError attiva la catena di gestione dell'errore in sostituzione della normale catena di gestione delle richieste.
		\end{itemize}
	\end{itemize}
\end{itemize}