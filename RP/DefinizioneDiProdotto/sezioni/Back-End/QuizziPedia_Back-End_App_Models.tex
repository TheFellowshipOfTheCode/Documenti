\subsection{QuizziPedia::Back-End::App::Models}
\subsubsection{Informazioni generali}
\label{QuizziPedia::Back-End::App::Models}
\begin{figure}
	\centering
	\includegraphics[scale=0.45]{UML/Package/QuizziPedia_Back-End_App_Models.png}
	\caption{QuizziPedia::Back-End::App::Models}
\end{figure}
\FloatBarrier
\begin{itemize}
	\item \textbf{Descrizione} \\
	\textit{Package\ped{G}} contenente le classi che definiscono il \textit{model} dell'applicazione. Queste classi sono definite come classi schema di \textit{Mongoose\ped{G}}, il quale permette di utilizzare \textit{MongoDB\ped{G}} tramite oggetti;
	\item \textbf{Padre} \texttt{App}
\end{itemize}

\subsubsection{Classi}
\paragraph{QuizziPedia::Back-End::App::Routers::UserRouter}
	\begin{itemize}
		\item \textbf{Descrizione} \\
		\item \textbf{Utilizzo} \\
		\item \textbf{Relazioni con altre classi} \\
		\item \textbf{Metodi} \\
	\end{itemize}
\paragraph{QuizziPedia::Back-End::App::Models::UserModel}
\begin{itemize}
	\item \textbf{Descrizione} \\
	Classe che modella la creazione e la gestione dei dati utente
	\item \textbf{Utilizzo} \\
	Viene utilizzata per rappresentare i dati degli account dei vari utenti dell’applicazione. Si interfaccia alla libreria Mongoose per la creazione dello schema e dei relativi metodi statici o di istanza.
	\item \textbf{Relazioni con altre classi} \\
		\begin{itemize}
			\item \textbf{IN QuestionModel} \\
			Questa classe rappresenta i dati delle domande create dai vari utenti.
			\item \textbf{IN QuizModel} \\
			Questa classe rappresenta i dati dei questionari creati dai vari utenti.
			\item \textbf{IN QuizSummaryModel} \\
			Questa classe rappresenta i dati dei questionari creati dai vari utenti.
		\end{itemize}
	\item \textbf{Attributi} \\
		\begin{itemize}
			\item \textbf{- userSchema: Schema} \\
			Questo campo dati rappresenta lo schema Mongoose dell'utente QuizziPedia. Lo schema prevede i seguenti attributi:
			\begin{itemize}
				\item 
					name di tipo String, rappresenta il nome  dell'utente registrato;
				\item 
					surname di tipo String, rappresenta il cognome  dell'utente registrato;
				\item 
					email di tipo String, rappresenta l'email  dell'utente registrato;
				\item 
					userImg di tipo String, rappresenta il path della foto profilo dell'utente registrato;
				\item 
					userType di tipo String, rappresenta la tipologia dell'utente registrato;
				\item 
					username di tipo String, rappresenta l'username con cui viene identificato l'utente all'interno dell'applicazione;		
				\item
					password di tipo String, rappresenta la password associata all'utente,    appositamente codificata mediante l'algoritmo bcrypt;  		
				\item
					statistics di tipo Array Mixed, contenente i seguenti attributi:
				\begin{itemize}
					\item
						topic di tipo String, identifica la statistica derivante dalle esercitazioni effettuate dall'utente in un determinato argomento; 
					\item
						 topicLevel di tipo Number, identifica il livello di preparazione dell'utente in un determinato argomento;
					\item
						correctAnswers di tipo Number, identifica il numero di risposte corrette date dall'utente riguardanti domande di un determinato argomento; 
					\item						
						 totalAnswers di tipo Number	, identifica il numero di risposte totali date dall'utente riguardanti domande di un determinato argomento.		
				\end{itemize}		
				\item 
					levelUsers di tipo Number, identifica il livello dell'utente;				
				
				\item
					quizSummaries di tipo Array, contiene oggetti di tipo ObjectId, che rappresentano i riferimenti agli identificativi nel database dei questionari svolti dall'utente;		
			\end{itemize}	
		\end{itemize}	
	\item \textbf{Metodi} \\
		\begin{itemize}
		\item
		- generateHash(password: String): String
		Effettua l'hashing della stringa password se non è già stata criptata tramite campo salt per evitare attacchi di tipo rainbow. \\
		\textbf{Parametri} \\
			\begin{itemize}
			\item
				 password: String
				Rappresenta la password dell'utente.
			\end{itemize}
		\item
		- validPassword(password: String): String 
		Effettua la validità della password inserita comparandola con la password criptata.	\\
		\textbf{Parametri} \\
			\begin{itemize}
			\item
				 password: String
				Rappresenta la password dell'utente.
			\end{itemize}
		\end{itemize}	
\end{itemize}
\paragraph{QuizziPedia::Back-End::App::Models::UserProModel}
\label{QuizziPedia::Back-End::App::Models::UserProModel}
\begin{figure}
	
\end{figure}
\begin{itemize}
	\item \textbf{Descrizione} \\
	Classe che modella i dati dell'utente pro.
	\item \textbf{Utilizzo} \\
	Viene utilizzata per rappresentare i dati dell'utente pro. Si interfaccia alla libreria Mongoose per la creazione dello schema e dei relativi metodi statici o di istanza.
	\item \textbf{Relazioni con altre classi} 
		\begin{itemize}
			\item \textbf{OUT QuizModel} \\
			Questa classe rappresenta i dati dei questionari creati dagli utenti Pro.
			\item \textbf{IN UserPro} \\
			Questa classe rappresenta i dati riguardanti i vari utenti registrati al sistema.
		\end{itemize}
	\item \textbf{Attributi} 
		\begin{itemize}
			\item \textbf{- \texttt{userProSchema}: \texttt{Schema}} \\
			Questo campo dati rappresenta lo schema Mongoose dell'utente pro QuizziPedia. Lo schema prevede i seguenti attributi:
			\begin{itemize}
				\item 
					\texttt{userID} di tipo \texttt{ObjectId}, rappresenta il riferimento all'identificativo nel database contenente i dati dei vari utenti registrati;
			\end{itemize}		
		\end{itemize}	
	\item \textbf{Metodi}
		
\end{itemize}
\paragraph{QuizziPedia::Back-End::App::Models::QuestionModel}
\label{QuizziPedia::Back-End::App::Models::QuestionModel}
\begin{figure}
	\centering
	\includegraphics[scale=0.45]{UML/Package/QuizziPedia_Back-End_App_Models_questionModel.png}
	\caption{QuizziPedia::Back-End::App::Models::QuestionModel}
\end{figure}
\begin{itemize}
\item \textbf{Descrizione}: Questa classe rappresenta le domande;	
\item \textbf{Utilizzo}: Viene utilizzata per rappresentare le domande. Si interfaccia alla libreria Mongoose per la creazione dello schema e dei relativi metodi statici o di istanza;
\item \textbf{Relazione con altra classi}:
	\begin{itemize}
	\item IN UserModel;
	\item IN SummaryModel;
	\item OUT TopicModel;
	\item OUT QuizModel.
	\end{itemize}
\item \textbf{Attributi}:
	\begin{itemize}
	\item \texttt{questionSchema: Schema} \\
	Questo campo dati rappresenta lo schema Mongoose per le domande e prevede i 					seguenti attributi:
		\begin{itemize}
		\item \texttt{author} di tipo \texttt{ObjectId}, rappresenta il riferimento 					all'identificativo nel database dell'utente che ha creato la domanda;
		\item \texttt{type} di tipo \texttt{String}, rappresenta la tipologia di domanda;
		\item \texttt{questionText} di tipo \texttt{String}, rappresenta il testo della 				domanda; 
		\item \texttt{image} di tipo \texttt{String}, rappresenta l'URL dell'immagine 				associata al testo della domanda;
		\item \texttt{options1} di tipo \texttt{Array}, contiene oggetti di tipo String e 			rappresenta informazioni diverse in base alla tipologia della	 domanda;
		\item \texttt{options2} di tipo \texttt{Array}, contiene oggetti di tipo String e 			rappresenta informazioni diverse in base alla tipologia della	 domanda;
		\item \texttt{level} di tipo \texttt{Number}, rappresenta la difficoltà della 				domanda;
		\item \texttt{totalAnswers} di tipo \texttt{Number}, rappresenta le risposte 					totali che tutti gli utenti hanno dato alla domanda;
		\item \texttt{correctAnswers} di tipo \texttt{Number}, rappresenta quante risposte 		corrette hanno dato gli utenti che hanno risposto alla domanda.
		\end{itemize}
	\end{itemize}
\item \textbf{Metodi}:
	\begin{itemize}
	\item \texttt{+ createQuestion(content: JSON, callback: function(JSON))} \\
	;   
	\item \texttt{+ editQuestion(content: JSON, callback: function(JSON))} \\
	.
	\end{itemize}
\end{itemize}
\paragraph{QuizziPedia::Back-End::App::Models::QuizModel}
\label{QuizziPedia::Back-End::App::Models::QuizModel}
\begin{figure}
	\centering
	\includegraphics[scale=0.45]{UML/Package/QuizziPedia_Back-End_App_Models_QuizModel.png}
	\caption{QuizziPedia::Back-End::App::Models::QuizModel}
\end{figure}
\paragraph{QuizziPedia::Back-End::App::Models::TopicModel}
\label{QuizziPedia::Back-End::App::Models::TopicModel}
\begin{figure}
	\centering
	\includegraphics[scale=0.45]{UML/Package/QuizziPedia_Back-End_App_Models_topicModel.png}
	\caption{QuizziPedia::Back-End::App::Models::TopicModel}
\end{figure}
\begin{itemize}
	\item \textbf{Descrizione} \\
	Classe che modella gli argomenti all'interno delle domande.
	\item \textbf{Utilizzo} \\
	Viene utilizzata per rappresentare i dati relativi agli argomenti delle domande. Si interfaccia con la libreria \textit{Mongoose\ped{G}} per la creazione dello schema e dei relativi metodi statici o di istanza.
	\item \textbf{Relazioni con altre classi}
		\begin{itemize}
			\item \textbf{IN \texttt{QuestionModel}} \\
			Questa classe rappresenta i dati delle domande create dai vari utenti;
			\item \textbf{IN \texttt{UserModel}} \\
			Questa classe rappresenta gli utenti.
		\end{itemize}
	\item \textbf{Attributi}
		\begin{itemize}
			\item \textbf{- topicSchema: Schema} \\
			Questo campo dati rappresenta lo schema \textit{Mongoose\ped{G}} per gli argomenti e prevede i seguenti attributi:
				\begin{itemize}
					\item \texttt{name} di tipo \texttt{String}, rappresenta il nome dell'argomento;
					\item \texttt{correctAnswers} di tipo \texttt{Number}, rappresenta il numero totale di domande alle quali gli utenti hanno risposto correttamente durante un allenamento sull'argomento; 
					\item \texttt{totalAnswers} di tipo \texttt{Number}, rappresenta il numero totale di domande alle quali gli utenti hanno risposto durante un allenamento sull'argomento;
					\item \texttt{question} di tipo \texttt{Array}, contiene gli \texttt{ObjectId} delle domande sull'argomento.
				\end{itemize}
		\end{itemize}
	\item \textbf{Metodi}
		\begin{itemize}
			\item \texttt{+ updateCorrect(topicId : ObjectId) : void} \\
			Metodo che consente di tenere aggiornato il numero di risposte esatte date a domande sull'argomento. \\
			\begin{itemize}
			\item \texttt{topicId : ObjectId} \\
			Rappresenta l'identificativo dell'argomento da aggiornare.
			\end{itemize}
			\item \texttt{+ updateTotal(topicId : ObjectId): void} \\
			Metodo che consente di tenere aggiornato il numero totale di risposte date a domande sull'argomento. \\
			\begin{itemize}
			\item \texttt{topicId : ObjectId} \\
			Rappresenta l'identificativo dell'argomento da aggiornare.
			\end{itemize}
			\item \texttt{+ addQuestion(topicId : ObjectId, questionID : ObjectID) : void} \\
			Metodo che consente di aggiunge una domanda tra le domande sull'argomento. \\
			\begin{itemize}
			\item \texttt{topicId : ObjectId} \\
			Rappresenta l'identificativo dell'argomento da aggiornare;
			\item \texttt{questionId : ObjectId} \\
			Rappresenta l'identificativo della domanda da aggiungere.
			\end{itemize}
			\item \texttt{+ getQuestions(topicId : ObjectId, callback : function(JSON), errback : function(QuizziPediaError)) : void} \\
			Metodo che consente di ottenere le domande sull'argomento attraverso la funzione di callback oppure un messaggio di errore. \\
			\begin{itemize}
			\item \texttt{topicId : ObjectId} \\
			Rappresenta l'identificativo dell'argomento da aggiornare;
			\item \texttt{callback : function(JSON)} \\
			Rappresenta la callback che il metodo deve chiamare al termine dell'elaborazione nel caso in cui non si siano verificati errori;
			\item \texttt{errback : function(QuizziPediaError)} \\
			Rappresenta la callback che il metodo deve chiamare qualora si verificassero errori durante l'esecuzione del metodo.
			\end{itemize}
		\end{itemize}
\end{itemize}
\paragraph{QuizziPedia::Back-End::App::Models::SummaryModel}
\label{QuizziPedia::Back-End::App::Models::summaryModel}
\begin{figure}
	\centering
	\includegraphics[scale=0.45]{UML/Package/QuizziPedia_Back-End_App_Models_summaryModel.png}
	\caption{QuizziPedia::Back-End::App::Models::summaryModel}
\end{figure}
\paragraph{QuizziPedia::Back-End::App::Models::LangModel}
\label{QuizziPedia::Back-End::App::Models::LangModel}
\begin{figure}[ht]
	\centering
	\includegraphics[scale=0.45]{UML/Classi/Back-End/QuizziPedia_Back-End_App_Models_langModel.png}
	\caption{QuizziPedia::Back-End::App::Models::LangModel}
\end{figure}
\FloatBarrier
	\begin{itemize}
		\item \textbf{Descrizione}: classe che modella le informazioni riguardanti la lingua dell'applicazione;
		\item \textbf{Utilizzo}: viene utilizzata per scambiare memorizzare le traduzioni delle variabili che andranno visualizzate nella \textit{view\ped{G}}.
		\item \textbf{Relazioni con altre classi}:
			\begin{itemize}
				\item OUT LangController \\
				Classe che gestisce la logica applicativa riguardante la traduzione delle variabili;
			\end{itemize}
		\item \textbf{Attributi}:
			\begin{itemize}
				\item \texttt{LangSchema : Schema} \\
				Questo campo rappresenta lo schema \textit{mongoose\ped{G}} per le variabili della lingua e prevede i seguenti attributi:
					\begin{itemize}
						\item \texttt{lang} di tipo \texttt{String}, rappresenta la lingua scelta per l'applicazione;
						\item \texttt{eng} di tipo \texttt{Array}, contiene oggetti con coppie di tipo \texttt{String} che associano i nomi delle variabili alla loro traduzione in inglese;
						\item \texttt{ita} di tipo \texttt{Array}, contiene oggetti con coppie di tipo \texttt{String} che associano i nomi delle variabili alla loro traduzione in italiano;
					\end{itemize}
			\end{itemize}
		\item \textbf{Metodi}:
			\begin{itemize}
				\item \texttt{+ getVarlist(lang: String, callback: function(JSON), \\ errback: function(QuizziPediaError))} \\
				Metodo che permette di ritornare la traduzione delle variabili; \\
				\textbf{Parametri}:
					\begin{itemize}
						\item \texttt{lang: String} \\
						Rappresenta l'informazione che indica il set di traduzione da ritornare;
						\item \texttt{callback: function(JSON)} \\
						Rappresenta la \textit{callback\ped{G}} che verrà eseguita al termine dell'elaborazione nel caso non si verifichino errori durante l'esecuzione;
						\item \texttt{errback: function(QuizziPediaError)} \\
						Rappresenta la \textit{callback\ped{G}} che verrà eseguito al termine dell'elaborazione nel caso si verifichino errori durante l'esecuzione del metodo;
					\end{itemize}
			\end{itemize}
	\end{itemize}
	