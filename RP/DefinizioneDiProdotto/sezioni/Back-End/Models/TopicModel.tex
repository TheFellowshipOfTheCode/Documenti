\paragraph{QuizziPedia::Back-End::App::Models::TopicModel}
\label{QuizziPedia::Back-End::App::Models::TopicModel}
\begin{figure}[ht]
	\centering
	\includegraphics[scale=0.6]{UML/Classi/Back-End/QuizziPedia_Back-End_App_Models_topicModel.png}
	\caption{QuizziPedia::Back-End::App::Models::TopicModel}
\end{figure}
\FloatBarrier
\begin{itemize}
	\item \textbf{Descrizione}: classe che modella gli argomenti all'interno delle domande;
	\item \textbf{Utilizzo}:	viene utilizzata per rappresentare i dati relativi agli argomenti delle domande. Si interfaccia con la libreria \textit{Mongoose\ped{G}} per la creazione dello schema e dei relativi metodi statici o di istanza;
	\item \textbf{Relazioni con altre classi}:
		\begin{itemize}
			\item \textbf{IN \texttt{TopicController}} \\
			Classe che gestisce la logica applicativa riguardante la visualizzazione e la modifica degli argomenti delle domande;
			\item \textbf{OUT \texttt{QuestionModel}} \\
			Questa classe rappresenta i dati delle domande create dai vari utenti.
		\end{itemize}
	\item \textbf{Attributi}:
		\begin{itemize}
			\item \texttt{- topicSchema: Schema} \\
			Questo campo dati rappresenta lo schema \textit{Mongoose\ped{G}} per gli argomenti e prevede i seguenti attributi:
				\begin{itemize}
					\item \texttt{name: String}\\ Rappresenta il nome dell'argomento;
					\item \texttt{correctAnswers: Number}\\ Rappresenta il numero totale di domande alle quali gli utenti hanno risposto correttamente durante un allenamento sull'argomento; 
					\item \texttt{totalAnswers: Number}\\ Rappresenta il numero totale di domande alle quali gli utenti hanno risposto durante un allenamento sull'argomento;
					\item \texttt{question: Array}\\ Array che ontiene gli \texttt{ObjectId} delle domande sull'argomento.
				\end{itemize}
		\end{itemize}
	\item \textbf{Metodi}:
		\begin{itemize}
			\item \texttt{+ addCorrect(topic : String, callback : function(JSON)) : void} \\
			Metodo che consente di tenere aggiornato il numero di risposte esatte date a domande sull'argomento.\\
			\textbf{Parametri}:
			\begin{itemize}
				\item \texttt{topic : String} \\
				Rappresenta l'argomento a cui si vuole aggiornare il contatore di risposte corrette
				\item \texttt{callback : function(JSON)} \\
				Rappresenta la \textit{callback\ped{G}} che il metodo deve chiamare al termine dell'elaborazione nel caso in cui non si siano verificati errori;	.
			\end{itemize}
			\item \texttt{+ addTotal(topic : String, callback : function(JSON)) : void} \\
			Metodo che consente di tenere aggiornato il numero totale di risposte date a domande sull'argomento.\\
			\begin{itemize}
				\item \texttt{topic : String} \\
				Rappresenta l'argomento a cui si vuole aggiornare il contatore di risposte totali
				\item \texttt{callback : function(JSON)} \\
				Rappresenta la \textit{callback\ped{G}} che il metodo deve chiamare al termine dell'elaborazione nel caso in cui non si siano verificati errori;	.
			\end{itemize}
			
			\item \texttt{+ addQuestion(questionId : ObjectID) : void} \\
			Metodo che consente di aggiunge una domanda tra le domande sull'argomento. \\
			\textbf{Parametri}:
			\begin{itemize}
				\item \texttt{questionId : ObjectId} \\
				Rappresenta l'identificativo della domanda da aggiungere.
			\end{itemize}
			
			\item \texttt{+ getTopic(lang : String,\\callback : function(JSON))} \\
			Metodo che consente di ottenere tutte i nomi degli argomenti presenti. \\
			\textbf{Parametri}:
			\begin{itemize}
				\item \texttt{lang : String} \\
				Rappresenta la lingua in cui sono scritte le domande che si vuole ottenere;
				\item \texttt{callback : function(JSON)} \\
				Rappresenta la \textit{callback\ped{G}} che il metodo deve chiamare al termine dell'elaborazione nel caso in cui non si siano verificati errori;
			\end{itemize}
			
			\item \texttt{+ getQuestions(language : String, keyword : String[], callback : function(JSON))} \\
			Metodo che consente di ottenere le domande sull'argomento attraverso la funzione di \textit{callback\ped{G}} oppure un messaggio di errore. \\
			\textbf{Parametri}:
			\begin{itemize}
				\item \texttt{language : String} \\
				Rappresenta la lingua in cui sono scritte le domande che si vuole ottenere;
				\item \texttt{keyword : Array<String>} \\
				Rappresenta le keyword che possono essere assegnate alle domande che si vuole ottenere;
				\item \texttt{callback : function(JSON)} \\
				Rappresenta la \textit{callback\ped{G}} che il metodo deve chiamare al termine dell'elaborazione nel caso in cui non si siano verificati errori;
			\end{itemize}
			\item \texttt{+ getTopicQuestions(topic : String, keyword : String[],  lang : String, \\callback : function(JSON))} \\
			Metodo che consente di ottenere le domande relative ad un determinato argomento filtrate per keywords. \\
			\textbf{Parametri}:
			\begin{itemize}
				\item \texttt{topic : String} \\
				Rappresenta l'argomento per il quale si vuole ottenere le relative domande;
				\item \texttt{keyword : Array<String>} \\
				Rappresenta le keyword che verranno utilizzate per filtrare le domande;
				\item \texttt{lang : String} \\
				Rappresenta la lingua in cui sono scritte le domande che si vuole ottenere;
				\item \texttt{callback : function(JSON)} \\
				Rappresenta la \textit{callback\ped{G}} che il metodo deve chiamare al termine dell'elaborazione nel caso in cui non si siano verificati errori;
			\end{itemize}
			\item \texttt{+ getNextQuestion(topic:String, alreadyAnswered:Boolean, language:String, keywords:String, skillLevel:Number, \\callback : function(JSON)):: void} \\
			Metodo che consente di ottenere la domanda successiva nella modalità allenamento, in base al livello dell'utente che lo sta svolgendo. \\
			\textbf{Parametri}:
			\begin{itemize}
				\item \texttt{topic:String} \\
				Rappresenta l'argomento  della domanda che si vuole ottenere;	
				\item \texttt{alreadyAnswered:Boolean} \\
				Rappresenta se l'utente ha già risposto o meno alla domanda;
				\item \texttt{keywords:String} \\
				Rappresenta le keywords scelte dall'utente per le quali si vuole filtrare le domande;
				\item \texttt{language : String} \\
				Rappresenta la lingua in cui è scritta la domanda che si vuole ottenere;
				\item \texttt{skillLevel:Number} \\
				Rappresenta il livello dell'utente che sta svolgendo l'allenamento;
				\item \texttt{callback : function(JSON)} \\
				Rappresenta la \textit{callback\ped{G}} che il metodo deve chiamare al termine dell'elaborazione nel caso in cui non si siano verificati errori;
			\end{itemize}
		\end{itemize}
\end{itemize}