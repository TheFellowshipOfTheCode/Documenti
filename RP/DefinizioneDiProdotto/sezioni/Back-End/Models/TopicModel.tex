\paragraph{QuizziPedia::Back-End::App::Models::TopicModel}
\label{QuizziPedia::Back-End::App::Models::TopicModel}
\begin{figure}
	\centering
	\includegraphics[scale=0.45]{UML/Classi/Back-End/QuizziPedia_Back-End_App_Models_topicModel.png}
	\caption{QuizziPedia::Back-End::App::Models::TopicModel}
\end{figure}
\begin{itemize}
	\item \textbf{Descrizione} \\
	Classe che modella gli argomenti all'interno delle domande.
	\item \textbf{Utilizzo} \\
	Viene utilizzata per rappresentare i dati relativi agli argomenti delle domande. Si interfaccia con la libreria \textit{Mongoose\ped{G}} per la creazione dello schema e dei relativi metodi statici o di istanza.
	\item \textbf{Relazioni con altre classi}
		\begin{itemize}
			\item \textbf{IN \texttt{QuestionModel}} \\
			Questa classe rappresenta i dati delle domande create dai vari utenti;
			\item \textbf{IN \texttt{UserModel}} \\
			Questa classe rappresenta gli utenti.
		\end{itemize}
	\item \textbf{Attributi}
		\begin{itemize}
			\item \textbf{- topicSchema: Schema} \\
			Questo campo dati rappresenta lo schema \textit{Mongoose\ped{G}} per gli argomenti e prevede i seguenti attributi:
				\begin{itemize}
					\item \texttt{name} di tipo \texttt{String}, rappresenta il nome dell'argomento;
					\item \texttt{correctAnswers} di tipo \texttt{Number}, rappresenta il numero totale di domande alle quali gli utenti hanno risposto correttamente durante un allenamento sull'argomento; 
					\item \texttt{totalAnswers} di tipo \texttt{Number}, rappresenta il numero totale di domande alle quali gli utenti hanno risposto durante un allenamento sull'argomento;
					\item \texttt{question} di tipo \texttt{Array}, contiene gli \texttt{ObjectId} delle domande sull'argomento.
				\end{itemize}
		\end{itemize}
	\item \textbf{Metodi}
		\begin{itemize}
			\item \texttt{+ updateCorrect() : void} \\
			Metodo che consente di tenere aggiornato il numero di risposte esatte date a domande sull'argomento.
			\item \texttt{+ updateTotal() : void} \\
			Metodo che consente di tenere aggiornato il numero totale di risposte date a domande sull'argomento.
			\item \texttt{+ addQuestion(questionID : ObjectID) : void} \\
			Metodo che consente di aggiunge una domanda tra le domande sull'argomento. \\
			\textbf{Parametri}:
			\begin{itemize}
			\item \texttt{questionId : ObjectId} \\
			Rappresenta l'identificativo della domanda da aggiungere.
			\end{itemize}
			\item \texttt{+ getQuestions(language : String, callback : function(JSON), errback : function(QuizziPediaError)) : void} \\
			Metodo che consente di ottenere le domande sull'argomento attraverso la funzione di callback oppure un messaggio di errore. \\
			\textbf{Parametri}:
			\begin{itemize}
			\item \texttt{language : String} \\
			Rappresenta la lingua in cui sono scritte le domande che si vuole ottenere;
			\item \texttt{callback : function(JSON)} \\
			Rappresenta la callback che il metodo deve chiamare al termine dell'elaborazione nel caso in cui non si siano verificati errori;
			\item \texttt{errback : function(QuizziPediaError)} \\
			Rappresenta la callback che il metodo deve chiamare qualora si verificassero errori durante l'esecuzione del metodo.
			\end{itemize}
			\item \texttt{+ getNextQuestion(language : String, levelUser : Number, callback : function(JSON), errback : function(QuizziPediaError)) : void} \\
			Metodo che consente di ottenere la domanda successiva nella modalità allenamento, in base al livello dell'utente che lo sta svolgendo. \\
			\textbf{Parametri}:
			\begin{itemize}
			\item \texttt{language : String} \\
			Rappresenta la lingua in cui è scritta la domanda che si vuole ottenere;
			\item \texttt{levelUser : Number} \\
			Rappresenta il livello dell'utente che sta svolgendo l'allenamento;
			\item \texttt{callback : function(JSON)} \\
			Rappresenta la callback che il metodo deve chiamare al termine dell'elaborazione nel caso in cui non si siano verificati errori;
			\item \texttt{errback : function(QuizziPediaError)} \\
			Rappresenta la callback che il metodo deve chiamare qualora si verificassero errori durante l'esecuzione del metodo.
			\end{itemize}
		\end{itemize}
\end{itemize}