\paragraph{QuizziPedia::Back-End::App::Routers::UserRouter}
\label{QuizziPedia::Back-End::App::Routers::UserRouter}
\begin{figure}[ht]
	\centering
	\includegraphics[scale=0.8]{UML/Classi/Back-End/QuizziPedia_Back-End_App_Routers_UserRouter.png}
	\caption{QuizziPedia::Back-End::App::Routers::UserRouter}
\end{figure}
\FloatBarrier
	\begin{itemize}
		\item \textbf{Descrizione}: classe che gestisce le richieste relative alla registrazione, alla gestione della sessione e alla cronologia dei questionari svolti di un utente.
Componente ConcreteHandler del design pattern \textit{Chain of responsibility\ped{G}}. Utilizza il modulo \textit{Passport\ped{G}};
		\item \textbf{Utilizzo}: viene utilizzata per chiamare il \textit{controller\ped{G}} che si occupa di gestire le \textit{API\ped{G}} relative alla registrazione, alla gestione della sessione e alla cronologia dei questionari di un utente;
		\item \textbf{Relazioni con altre classi}:
		\begin{itemize}
		\item 
			\textbf{IN	\texttt{Server}}: classe che avvia il server. Nello specifico apre una connessione al database tramite \textit{Mongoose\ped{G}}, invoca il \textit{middleware\ped{G}} \textit{Express\ped{G}} passando un riferimento al database \textit{MongoDB\ped{G}} come parametro in modo che possa configurarsi con esso, invoca il \textit{middleware\ped{G}} \textit{Passport\ped{G}} ed infine si mette in ascolto su una determinata porta. È il componente client del design pattern \textit{Chain of responsibility\ped{G}}. Utilizza i moduli \textit{Mongoose\ped{G}}, \textit{Express\ped{G}}, \textit{Passport\ped{G}}; 
		\item 
		\textbf{	OUT \texttt{ErrorHandler}}: classe \textit{middleware\ped{G}} per la gestione degli errori. Ritorna al client un oggetto di tipo Response con stato \textit{HTTP\ped{G}} 500 e descrizione dell’errore in formato \textit{JSON\ped{G}}. È un
componente ConcreteHandler del design pattern \textit{Chain of responsibility\ped{G}};
		\item 
			\textbf{OUT \texttt{NotFoundHandler}}: classe che si occupa della gestione dell’errore di pagina non trovata. Componente ConcreteHandler del design pattern \textit{Chain of responsibility\ped{G}};
		\item 
			\textbf{OUT \texttt{UserController}}: classe che raggruppa attraverso require i vari \textit{controllers\ped{G}}  responsabili delle operazioni legate alla gestione degli utenti. Si è scelto di predisporre questo raggruppamento per facilitare l'introduzione di nuove funzionalità legate alla gestione degli utenti;
		\item 
		\textbf{	OUT \texttt{SummaryController}}: classe che gestisce la cronologia dei questionari svolti dall'utente.
	\end{itemize}
		\item \textbf{Metodi}:
		\begin{itemize}
		\item 
		\texttt{+ UserRouter(app: Server)} \\
		Contiene diverse \textit{route\ped{G}} che vengono configurate all’avvio del server. Quest’ultime ricevono le richieste del client e passano il controllo al ConcreteHandler successivo.
		\textbf{Parametri}:
			\begin{itemize}
				\item 
				\texttt{app: Server} \\
				Rappresenta l’istanza del server su cui configurare i \textit{route\ped{G}} che mappano i \textit{controllers\ped{G}}  specifici.
			\end{itemize}
		\end{itemize}
\end{itemize}		