\paragraph{QuizziPedia::Back-End::App::Routers::QuestionRouter}
\begin{figure}[ht]
	\centering
	\includegraphics[scale=0.45]{UML/Package/QuizziPedia_Back-End_App_Routers_questionRouter.png}
	\caption{QuizziPedia::Back-End::App::Routers::QuestionRouter}
\end{figure}
\FloatBarrier
	\begin{itemize}
		\item \textbf{Descrizione} \\
		Classe che gestisce le richieste relative alle operazioni riguardanti le domande. Componente \textit{ConcreteHandler\ped{G}} del \textit{design pattern\ped{G}} \textit{Chain of responsibility\ped{G}};
		\item \textbf{Utilizzo} \\
		Viene utilizzata per chiamare il controller che si occupa di gestire le \textit{API\ped{G}} relative alle domande;
		\item \textbf{Relazioni con altre classi} \\
		\begin{itemize}
			\item \textbf{IN} \texttt{Server}\\
			Classe che avvia il \textit{server\ped{G}}. Nello specifico apre una connessione al database tramite \textit{Mongoose\ped{G}}, invoca il \textit{middleware\ped{G}} \textit{Express\ped{G}} passando un riferimento al database \textit{MongoDB\ped{G}} come parametro in modo che possa configurarsi con esso, invoca il \textit{middleware\ped{G}} \textit{Passport\ped{G}} ed infine si mette in ascolto su una determinata porta. È il componente \textit{client\ped{G}} del \textit{design pattern\ped{G}} \textit{Chain of responsibility\ped{G}}. Utilizza i moduli \textit{Mongoose\ped{G}}, \textit{Express\ped{G}} e \textit{Passport\ped{G}};
			\item \textbf{OUT} \texttt{ErrorsHandler}\\
			Classe \textit{middleware\ped{G}} per la gestione degli errori. Ritorna al \textit{client\ped{G}} un oggetto di tipo \texttt{Response} con stato HTTP 500 e descrizione dell'errore in formato JSON. È un componente \textit{ConcreteHandler\ped{G}} del \textit{design pattern\ped{G}} \textit{Chain of responsibility\ped{G}};
			\item \textbf{OUT} \texttt{NotFoundHandler}\\
			Classe che si occupa della gestione dell’errore di pagina non trovata. Componente \textit{ConcreteHandler\ped{G}} del \textit{design pattern\ped{G}} \textit{Chain of responsibility\ped{G}};
			\item \textbf{OUT} \texttt{QuestionController}\\
			Classe che raggruppa i vari \textit{controllers} responsabili delle operazioni riguardanti le domande attraverso \texttt{require};
			\item \textbf{OUT} \texttt{TopicController} \\
			Classe che gestisce la logica applicativa riguardante la visualizzazione e la modifica degli argomenti delle domande. È un componente \textit{ConcreteHandler\ped{G}} del \textit{design pattern\ped{G}} \textit{Chain of responsibility\ped{G}}.
		\end{itemize}
		\item \textbf{Metodi} \\
		\begin{itemize}
			\item \texttt{+ QuestionRouter(app: Server)}\\
			Contiene diverse \textit{route\ped{G}} che vengono configurate all’avvio del \textit{server\ped{G}}. Quest’ultime ricevono le richieste del \textit{client\ped{G}} e passano il controllo al \textit{ConcreteHandler\ped{G}} successivo.\\
			\textbf{Parametri}:
			\begin{itemize}
				\item \texttt{app: Server}\\
				Rappresenta l’istanza del \textit{server\ped{G}} su cui configurare i \textit{route\ped{G}} che mappano i \textit{controllers} specifici.
			\end{itemize}
		\end{itemize}
	\end{itemize}