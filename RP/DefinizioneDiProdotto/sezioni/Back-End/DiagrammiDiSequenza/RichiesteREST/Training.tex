\paragraph{GET /:lang/user/training/question} % richiesta che ritorna la prossima domanda dell'allenamento
\begin{itemize}
\item \textbf{Successo}
% descrizione diagramma e UML

\textbf{Descrizione}: il \texttt{QuestionRouter} gestisce la richiesta \textit{REST\ped{G}} del Front-End passando il controllo al \texttt{TopicController}; viene poi invocato il metodo \texttt{getNextQuestion(req,res,next)} che ritorna la prossima domanda da sottoporre all'utente durante l'Allenamento. 

\begin{figure}[ht]
	\centering
	\includegraphics[scale=0.45]{UML/DiagrammiDiSequenza/Back-end/GET__lang_user_training_question.png}
	\caption{GET /:lang/user/training/question}
\end{figure}
\FloatBarrier

\item \textbf{Fallimento}
% descrizione diagramma e UML
\end{itemize}

\paragraph{POST /:lang/user/training/userstatistics} % richiesta che aggiorna le statistiche dell'utente
\begin{itemize}
\item \textbf{Successo}
% descrizione diagramma e UML
\item \textbf{Fallimento}
% descrizione diagramma e UML
\end{itemize}

\paragraph{POST /:lang/user/training/questionstatistics} % richiesta che aggiorna le statistiche della domanda
\begin{itemize}
\item \textbf{Successo}
% descrizione diagramma e UML
\item \textbf{Fallimento}
% descrizione diagramma e UML
\end{itemize}  