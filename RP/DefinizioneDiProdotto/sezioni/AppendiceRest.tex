\newpage
\section{Interfaccia REST}
Per utilizzare il Web\ped{G} come piattaforma di elaborazione, l'interfaccia per il \textit{back-end\ped{G}} del progetto \textit{\progetto} è realizzata in stile \textit{RESTful\ped{G}}.
L'interfaccia REST\ped{G} propone una visione del Web incentrata sul concetto di \textit{risorsa\ped{G}}, per questo motivo è stato preferito a SOAP\ped{G} che espone un insieme di metodi richiamabili da remoto da parte di un \textit{client\ped{G}}. Inoltre, SOAP\ped{G} sfrutta il protocollo HTTP\ped{G} come semplice protocollo di trasporto; REST\ped{G} lo usa come  protocollo di livello applicativo, per questo ne utilizza appieno le potenzialità.
%qua inserisco il motivo della scelta di usare REST%
I dati scambiati mediante l'interfaccia \textit{REST\ped{G}} sono rappresentati in formato JSON\ped{G}, che si integra facilmente con le tecnologie ed il linguaggio utilizzati per sviluppare \progetto. \\
Se lo scambio di dati avviene correttamente il \textit{server\ped{G}} può fornire in risposta un messaggio di conferma:
\begin{lstlisting}[language=json,firstnumber=1]
{
"status" : "ok"
}
\end{lstlisting}
\subsection{Errori}
In caso di errori il \textit{server\ped{G}} risponde con un messaggio d'errore in formato JSON\ped{G}, definito secondo lo schema:
\begin{lstlisting}[language=json,firstnumber=1]
{
"code" : [codice dell'errore],
"title" : [titolo dell'errore],
"message" : [descrizione testuale dell'errore]
}
\end{lstlisting}
\subsubsection{Errori generici}
Gli errori generici cono gestiti dalla classe \texttt{ErrorHandler} ad eccezione dell'errore \texttt{404} che viene gestito da \texttt{NotFoundHandler}. In questo modo si ottiene una gestione degli errori più flessibile e modulare, in quanto è possibile inserire in coda allo stack\ped{G} di chiamate di ogni ruote la classe \texttt{NotFoundHandler}, riducendo così la complessità della classe \texttt{ErrorHandler}.

\paragraph{Errori lato server}
Nel caso si verifichi un errore lato server, questo risponde con un codice HTTP\ped{G} del tipo \texttt{5xx}, di seguito vengono specificati gli errori che possono essere sollevati:
	\begin{itemize}
		\item \texttt{500 Errore sconosciuto}: il messaggio descrive un errore generico del server che avviene quando si verifica una condizione non gestibile e quindi non identificabile con un errore specifico.
	\end{itemize}
\paragraph{Errori nelle richieste da parte del client}
Nel caso si verifichi un errore riguardo le richieste ricevute dal client, il server risponde con un codice HTTP\ped{G} del tipo \texttt{4xx}, di seguito vengono specificati gli errori che possono essere sollevati:
	\begin{itemize}
		\item \texttt{400}: il server non può gestire la richiesta in seguito ad una generica richiesta errata. Il contenuto del messaggio d'errore varia in base alla tipologia di richiesta ricevuta;
		\item \texttt{401}: Accesso non autorizzato;
		\item \texttt{404}: Pagina non trovata.
	\end{itemize}S
\paragraph{Errori specifici di \progetto}
Questi errori rappresentano situazioni specifiche del sistema \progetto, sono quindi stati definiti dei codici personalizzati per questa tipologia di errori basandosi sull'idea dei codici d'errore HTTP\ped{G} standard. I codici dei messaggi d'errore sono stati assegnati secondo diverse categorie, in modo da poter individuare facilmente quale componente dell'applicazione ha causato l'errore. Ad ogni categoria è stato assegnato successivamente un intervallo di codici possibili, evitando di utilizzare gli intervalli \texttt{4xx} e \texttt{5xx} per non creare ambiguità con i codici d'errore standard descritti nella sezione precedente. \\
Le categorie di errori definite sono le seguenti:
\begin{itemize}
	\item \texttt{1xx}: errori di autenticazione, registrazione, modifica su utente e del middleware \texttt{userById};
		\begin{itemize}
			\item \texttt{101}: credenziali non valide;
			\item \texttt{102}: password assente;
			\item \texttt{103}: password troppo corta;
			\item \texttt{104}: password uguale all'attuale;
			\item \texttt{105}: password e conferma password non coincidono;
			\item \texttt{111}: campo obbligatorio vuoto;
			\item \texttt{112}: username non disponibile;
			\item \texttt{113}: username non valido;
			\item \texttt{114}: indirizzo mail non valido;
			\item \texttt{121}: formato immagine non valido;
			\item \texttt{122}: immagine di dimensione troppo grande;
			\item \texttt{123}: errore nel caricamento dell'immagine;
		\end{itemize}
	\item \texttt{2xx}: errori del middleware \texttt{questionById};
	\item \texttt{3xx}: errori del middleware \texttt{topicById};
	\item \texttt{6xx}: errori del middleware \texttt{quizById};
	\item \texttt{7xx}: errori del middleware \texttt{summaryById};
	\item \texttt{8xx}: errori del controller \texttt{QuestionController};
	\item \texttt{9xx}: errori del controller \texttt{TopicController};
	\item \texttt{10xx}: errori del controller \texttt{QuizController};
	\item \texttt{11xx}: errori del controller \texttt{SummaryController};
	\item \texttt{12xx}: errori degli oggetti \texttt{QuestionModel}:
	\item \texttt{13xx}: errori degli oggetti \texttt{TopicModel};
	\item \texttt{14xx}: errori degli oggetti \texttt{QuizModel};
	\item \texttt{15xx}: errori degli oggetti \texttt{SummaryModel};
\end{itemize}
Gli errori specifici di \textit{\progetto} vengono gestiti dalla classe \texttt{QuizziPediaError} e sono di seguito elencati sotto forma di:
\begin{center}
	\texttt{Codice Titolo}: messaggio d'errore
\end{center}
\begin{itemize}
	\item \texttt{100 Utente non trovato:} l'identificativo utente fornito non è un identificativo valido;
	\item \texttt{101 Credenziali non valide:} è necessario fornire un username ed una password valide;
	\item \texttt{102 Password assente:} è necessario inserire una password;
	\item \texttt{103 Password troppo corta:} è necessario inserire una password di almeno 6 caratteri;
	\item \texttt{104 Password uguale all'attuale:} è necessario inserire una password diversa della precedente;
	\item \texttt{105 Le password non coincidono:} è necessario che 'password' e 'conferma password' siano identiche;
	\item \texttt{111 Campo obbligatorio vuoto:} è necessario riempire tutti i campi obbligatori;
	\item \texttt{112 Username non disponibile:} l 'username è già stato utilizzato, prego inserire un nuovo username;
	\item \texttt{113 Username non valido:} l'username non è valido;
	\item \texttt{114 Indirizzo mail non valido: } è necessario fornire un indirizzo e-mail valido;
	\item \texttt{121 Formato immagine non valido: } Il formato dell'immagine non è valido;
	\item \texttt{122 Immagine di dimensione troppo grande: } La dimensione massima per l'immagine è di 20MB;
	\item \texttt{123 Caricamento immagine fallito}: errore sconosciuto nel caricamento dell'immagine;
	\item \texttt{200 Domanda non valida:} l'identificativo della domanda fornita non è un identificativo valido;
	\item \texttt{300 Argomento non valido:} l'identificativo dell'argomento fornito non è un identificativo valido;
	\item \texttt{600 Questionario non valido:}  l'identificativo del questionario fornito non è un identificativo valido;
	\item \texttt{700 Sommario non valido:} l'identificativo del questionario fornito non è un identificativo valido;
	\item \texttt{800 Dati non validi:} i dati relativi al contenuto della domanda non sono  validi o sono formattati in modo errato;
	\item \texttt{900 Dati non validi:} i dati relativi al contenuto dell'argomento non sono validi;
	\item \texttt{1000 Dati non validi:} i dati relativi al contenuto del questionario non validi;
	\item \texttt{1100 Dati non validi:} i dati relativi al contenuto del  sommario non validi o sono formattati in modo errato;
	\item \texttt{1200 Dati non validi:} i dati per la modifica di una domanda non sono definiti;
	\item \texttt{1300 Dati non validi:} i dati per la gestione dell' argomento non sono definiti;
	\item \texttt{1400 dati non validi:} i dati per la gestione del questionario non sono definiti;
	\item \texttt{1500 Dati non validi:} i dati per la gestione del sommario non sono validi;
\end{itemize}

\subsubsection{Risorse REST}
In seguito vengono riportate le risorse \textit{REST\ped{G}} fornite, associate al tipo di metodo HTTP\ped{G} che è possibile richiedere su di esse e ai permessi necessario per effettuare la richiesta. In particolare, i permessi sono:
\begin{itemize}
	\item \textbf{Utente}: la risorsa può essere acceduta da qualsiasi tipo di utente;
	\item \textbf{Utente autenticato}: la risorsa può essere acceduta solo dagli utenti che hanno effettuato il login;
	\item \textbf{Utente autenticato pro}: la risorsa può essere acceduta solo dagli utenti pro.
\end{itemize}
Inoltre per ogni risorsa sono stati specificati i formati per lo scambio dei dati in JSON\ped{G}:
\begin{itemize}
	\item \textbf{Request}: rappresenta l'oggetto JSON\ped{G} che dovrà essere passato alla risorsa \textit{REST\ped{G}};
	\item \textbf{Response}: rappresenta l'oggetto JSON\ped{G} che fornirà in risposta la risorsa \textit{REST\ped{G}};
		\item \texttt{/:lang}
		\begin{itemize}
			\item \textbf{Method}: GET
			\item \textbf{Livello di Accesso}: Utente;
			\item \textbf{Descrizione}: restituisce le keywords inerenti alla lingua impostata
			\item \textbf{Response}: la risposta deve avere i seguenti campi:
\begin{lstlisting}[language=json,firstnumber=1]
{
"keywords": [Array contenente le keywords inerenti alla lingua impostata ]
}
\end{lstlisting}
		\end{itemize}
	
	
	\item \texttt{/:lang/signup}
		\begin{itemize}
			\item \textbf{Method}: POST
			\item \textbf{Livello di Accesso}: Utente;
			\item \textbf{Descrizione}: crea un nuovo account. Un username univoco inserito dall'utente lo identifica, pertanto non ci saranno più utenti con lo stesso username. Restituisce un messaggio di conferma se viene effettuato correttamente, altrimenti un errore.
			\item \textbf{Request}: lo scambio dei dati dell'utente avviene attraverso una form che deve avere i seguenti campi:
\begin{lstlisting}[language=json,firstnumber=1]
{
"username" : [username univoco scelto dall'utente]
"password" : [la password associata all'account che si vuole registrare]
"e-mail" : [l'indirizzo e-mail con cui si effettua la registrazione]
"name" : [nome dell'utente da registrare]
"surname" : [cognome dell'utente da registrare]
}
\end{lstlisting}
		\end{itemize}
		\item \texttt{/signin}
		\begin{itemize}
			\item \textbf{Method}: POST;
			\item \textbf{Livello di Accesso}: utente/utente pro;
			\item \textbf{descrizione}: effettua l'accesso all'account. Restituisce un messaggio di conferma se viene effettuato correttamente, altrimenti un errore.
			\item \textbf{Request}: lo scambio dei dati dell'utente avviene attraverso una form che deve avere i seguenti campi:
\begin{lstlisting}[language=json,firstnumber=1]
{
"username" : [username che corrisponde all'username inserita durante la registrazione]
"password" : [la password associata all'username dell'account dell'utente]
}
\end{lstlisting}
			\item \textbf{Response}: la risposta deve contenere i seguenti campi:
\begin{lstlisting}[language=json,firstnumber=1]
{
"_userId" : [identificativo dell'utente]
"username" : [username che corrisponde all'username inserita durante la registrazione]
}
\end{lstlisting}
		\end{itemize}
		\item \texttt{/signout}
		\begin{itemize}
			\item \textbf{Method}: GET;
			\item \textbf{Livello di Accesso}: Utente autenticato/autenticato pro;
			\item \textbf{Descrizione}: effettua il logout. restituisce un messaggio di conferma se viene effettuato correttamente, altrimenti un errore;
		\end{itemize}
		\item \texttt{/:lang/recovery}
		\begin{itemize}
			\item \textbf{Method}: POST;
			\item \textbf{Livello di Accesso}: Utente;
			\item \textbf{Descrizione}: invia una nuova password sulla mail dell'utente generata in automatico, restituisce un messaggio di conferma se viene effettuato correttamente, altrimenti un errore;
			\item \textbf{Request}: la richiesta di recupero della password deve avere i seguenti campi:
\begin{lstlisting}[language=json,firstnumber=1]
{
"e-mail" : [la password associata all'username dell'account dell'utente]
}
\end{lstlisting}
		\end{itemize}
		\item \texttt{/:lang/user/:userId/search/:keyword/users}
	 \begin{itemize}
	 	\item \textbf{Method}: GET;
	 	\item \textbf{Livello di Accesso:} Utente autenticato/autenticato pro;
	 	\item \textbf{Descrizione}: effettua una ricerca di utenti da parte di un utente;
	 	\item \textbf{Request}: la richiesta deve avere i seguenti campi:
\begin{lstlisting}[language=json,firstnumber=1]
{
"keyword" : [stringa da ricercare]
}
\end{lstlisting} 
		\item \textbf{Response:} la risposta deve avere un campo \texttt{array} contenente i risultati degli utenti trovati:
\begin{lstlisting}[language=json,firstnumber=1]
{
"userResult" : [
"_userID" : [identificativi dell'utente trovato]
"name" : [nome dell'utente trovato]
"surname" : [cognome dell'utente trovato]
]
}
\end{lstlisting}
	 \end{itemize}

\item \texttt{/:lang/user/:userId/search/:keyword/quizzes}
	 \begin{itemize}
	 	\item \textbf{Method}: GET;
	 	\item \textbf{Livello di Accesso:} Utente autenticato/autenticato pro;
	 	\item \textbf{Descrizione}: effettua una ricerca di questionari da parte di un utente;
	 	\item \textbf{Request}: la richiesta deve avere i seguenti campi:
\begin{lstlisting}[language=json,firstnumber=1]
{
"keyword" : [stringa da ricercare]
}
\end{lstlisting} 
		\item \textbf{Response:} la risposta deve avere un campo \texttt{array} contenente i risultati dei questionari trovati:
\begin{lstlisting}[language=json,firstnumber=1]
{
"quizResult : [
"_quizID" : [identificativo del questionario]
"title" : [titolo del questionario]
"author" : [username dell'autore del questionario]
]
}
\end{lstlisting}
	 \end{itemize}	 
	 
	\item \texttt{/:lang/user/:userId/search/users/:userId}
	 \begin{itemize}
	 	\item \textbf{Method}: GET;
	 	\item \textbf{Livello di Accesso:} Utente autenticato/autenticato pro;
	 	\item \textbf{Descrizione}: restituisce le informazioni dell'utente precedentemente trovato tramite una ricerca;
	 	\item \textbf{Response:} 
\begin{lstlisting}[language=json,firstnumber=1]
{
"username" : [username utente da visualizzare]
"name" : [nome utente da visualizzare]
"surname" : [cognome utente da visualizzare]
"email" : [email utente da visualizzare]
"userImg": [rappresenta l'immagine dell'utente da visualizzare]
"levelUser" : [livello utente da visualizzare]
"statistics":[array di JSON, contenente le statische dell'utente di ogni argomento]
}
\end{lstlisting}
	 \end{itemize}
	 
	 
	\item \texttt{/:lang/user/:userId/search/quizzes/:quizId}
	\begin{itemize}
		\item \textbf{Method}: GET;
		\item \textbf{Livello di Accesso:} Utente autenticato/autenticato pro;
		\item \textbf{Descrizione}: restituisce le informazioni del questionario precedentemente trovato tramite una ricerca;
		\item \textbf{Response:} 
\begin{lstlisting}[language=json,firstnumber=1]
{
"title" : [identifica il titolo del questionario]
"author" : [identifica l'autore del questionario]
"questions" : [identifica l'Array di JSON con le domande relative al questionario]
}
\end{lstlisting}
	\end{itemize}
	\item \texttt{/profile/:userId}
	\begin{itemize}
		\item \textbf{Method}: DELETE;
		\item \textbf{Livello di Accesso}: utente autenticato/autenticato pro;
		\item \textbf{Descrizione}: elimina l'utente autenticato con id pari a :userId. 		Restituisce un messaggio di conferma se viene effettuato correttamente, altrimenti un errore;
		\item \textbf{Response}:la risposta deve contenere i seguenti campi:
\begin{lstlisting}[language=json,firstnumber=1]
{
   	"name" : [identifica il nome dell'utente]
    	"surname" : [identifica il nome dell'utente]
}
\end{lstlisting}
	\end{itemize}	
	
	\item \texttt{/profile/info/:userId}
		\begin{itemize}
			\item \textbf{Method}: GET;
			\item \textbf{Livello di Accesso}: utente autenticato/autenticato pro;
			\item \textbf{Descrizione}: restituisce le informazioni riguardante l'utente autenticato con id pari a :userId. Restituisce un messaggio di conferma se viene
effettuato correttamente, altrimenti un errore;
			\item \textbf{Response}: la risposta deve contenere i seguenti campi:
\begin{lstlisting}[language=json,firstnumber=1]
{
	"username" : [username utente da visualizzare]
	"name" : [nome utente da visualizzare]
	"surname" : [cognome utente da visualizzare]
	"email" : [email utente da visualizzare]
	"userImg": [rappresenta l'immagine dell'utente da visualizzare]
	"levelUser" : [livello utente da visualizzare]
}
\end{lstlisting}
		\end{itemize}
		
	\item \texttt{/profile/info/:userId}
		\begin{itemize}
			\item \textbf{Method}: PUT;
			\item \textbf{Livello di Accesso}: utente autenticato/autenticato pro;
			\item \textbf{Descrizione}: modifica le informazioni riguardante l'utente autenticato con id pari a :userId. Restituisce un messaggio di conferma se viene
effettuato correttamente, altrimenti un errore;
			\item \textbf{Request}: la richiesta deve contenere i seguenti campi:
\begin{lstlisting}[language=json,firstnumber=1]
{
	"name" : [nome utente]
	"surname" : [cognome utente]
	"email" : [email utente]
	"userImg": [immagine profilo utente]
	"levelUser" : [livello utente]
}
\end{lstlisting}
		\end{itemize}	
		
	\item \texttt{/profile/info/privacy/:userId}
		\begin{itemize}
			\item \textbf{Method}: PUT;
			\item \textbf{Livello di Accesso}: utente autenticato/autenticato pro;
			\item \textbf{Descrizione}: modifica la password di accesso al sistema riguardante l'utente autenticato con id pari a :userId. Restituisce un messaggio di conferma se viene effettuato correttamente, altrimenti un errore;
			\item \textbf{Request}: la richiesta deve contenere i seguenti campi:	
\begin{lstlisting}[language=json,firstnumber=1]
{
    "password" : [identifica la nuova password inserita dallutente]
}	

\end{lstlisting}
		\end{itemize}
		
	\item \texttt{/profile/statistics/:userId}
		\begin{itemize}
			\item \textbf{Method}: GET;
			\item \textbf{Livello di Accesso}: utente autenticato/autenticato pro;
			\item \textbf{Descrizione}: restituisce le statistiche riguardanti l'utente autenticato con id pari a :userId. 
			\item \textbf{Response}: la risposta deve contenere i seguenti campi:	
\begin{lstlisting}[language=json,firstnumber=1]
{
    "name": [identifica il nome dell'utente]
    "surname": [identifica il nome dell'utente]
    "userImg": [rappresenta l'immagine dell'utente]
	"statistics":[array di JSON, contenente le statische dell'utente di ogni argomento]
	"summaries":[array di JSON contenente la cronologia dei questionari svolti dall'utente]   
}	

\end{lstlisting}
		\end{itemize}
	
	\item \texttt{/:lang/user/:userId/question}
	\begin{itemize}
		\item \textbf{Method}: GET;
		\item \textbf{Livello di Accesso}: Utente autenticato/autenticato pro;
		\item \textbf{Descrizione}: Restituisce le domande create dall'utente;
		\item \textbf{Response}: la risposta deve avere i seguenti campi:
\begin{lstlisting}[language=json,firstnumber=1]
{
"listQuestion" : [ 
"type" : [identifica tipologia domanda]
"language" : [identifica la lingua della domanda]
"questionText" : [identifica il testo della domanda]
"image" : [identifica l'immagine relativa al testo della domanda]
"option1" : [identifica l'array di opzioni di risposte]
"option2" : [identifica l'array di opzioni di risposte]
"totalAnswer" : [identifica il numero totale di risposte date alla domanda]
"correctAnswer" : [identifica il numero di risposte corrette date alla domanda]
]
}
\end{lstlisting}
	\end{itemize}	
	
	
	\item \texttt{/:lang/user/question}
		\begin{itemize}
			\item \textbf{Method}: POST;
			\item \textbf{Livello di Accesso}: Utente autenticato/autenticato pro;
			\item \textbf{Descrizione}: Aggiunge una domanda nel sistema, restituisce un messaggio di conferma o di errore;
			\item \textbf{Request}: la richiesta deve contenere i seguenti campi:
\begin{lstlisting}[language=json,firstnumber=1]
{
"listQuestion" : [ 
"type" : [identifica tipologia domanda]
"language" : [identifica la lingua della domanda]
"questionText" : [identifica il testo della domanda]
"image" : [identifica l'immagine relativa al testo della domanda]
"option1" : [identifica l'array di opzioni di risposte]
"option2" : [identifica l'array di opzioni di risposte]
]
}
\end{lstlisting}
		\end{itemize}
	\item \texttt{/:lang/user/:userId/quiz}
	\begin{itemize}
		\item \textbf{Method}: GET;
		\item \textbf{Livello di Accesso}: Utente autenticato pro;
		\item \textbf{Descrizione}: Restituisce i questionari creati dall'utente pro;
		\item \textbf{Response}: la risposta deve avere i seguenti campi:
\begin{lstlisting}[language=json,firstnumber=1]
{
"listQuiz" : [ 
"title" : [identifica il titolo del questionario]
"questions" : [Arrai di JSON contenente le domande relative al questionario]
]
}
\end{lstlisting}
	\end{itemize}	
	
	
	\item \texttt{/:lang/user/:userId/quiz}
		\begin{itemize}
			\item \textbf{Method}: POST;
			\item \textbf{Livello di Accesso}: Utente autenticato pro;
			\item \textbf{Descrizione}: Aggiunge un questionario nel sistema, restituisce un messaggio di conferma o di errore;
			\item \textbf{Request}: la richiesta deve contenere i seguenti campi:
\begin{lstlisting}[language=json,firstnumber=1]
{
"title" : [titolo del questionario]
"questions" : [Array di JSON contenente gli identificativi delle domande che compongono il questionario]
}
\end{lstlisting}
		\end{itemize}
		
		
	\item \texttt{/:lang/user/quiz/:quizId/addUser}
	\begin{itemize}
		\item \textbf{Method}: POST;
		\item \textbf{Livello di Accesso}: Utente autenticato pro;
		\item \textbf{Descrizione}: Iscrive un utente ad un questionario, restituisce un messaggio di conferma o di errore;
		\item \textbf{Request}: la richiesta deve contenere i seguenti campi:
\begin{lstlisting}[language=json,firstnumber=1]
{
"_userId" : [identificativo dell'utente da iscrivere al questionario]
}
\end{lstlisting}
	\end{itemize}
	
	\item \texttt{/:lang/user/quiz/:quizId/activeUser}
	\begin{itemize}
		\item \textbf{Method}: POST;
		\item \textbf{Livello di Accesso}: Utente autenticato/autenticato pro;
		\item \textbf{Descrizione}: Aggiunge un utente iscritto al questionario nella lista degli utenti che lo hanno eseguito;
		\item \textbf{Request}: la richiesta deve contenere i seguenti campi:
\begin{lstlisting}[language=json,firstnumber=1]
{
"_userId" : [identificativo dell'utente da iscrivere al questionario]
}
\end{lstlisting}
	\end{itemize}
	
	\item \texttt{/:lang/user/:userId/quiz/:quizId}
	\begin{itemize}
		\item \textbf{Method}: PUT;
		\item \textbf{Livello di Accesso}: Utente autenticato pro;
		\item \textbf{Descrizione}: Modifica in questionario creato in precedenza, restituisce un messaggio di conferma o di errore;
		\item \textbf{Request}: la richiesta deve contenere i seguenti campi:
\begin{lstlisting}[language=json,firstnumber=1]
{
"title" : [titolo del questionario]
"questions" : [Array di JSON contenente gli identificativi delle domande che compongono il questionario]
}
\end{lstlisting}
	\end{itemize}
	
	\item \texttt{/:lang/user/quiz/:quizId/test}
	\begin{itemize}
		\item \textbf{Method}: POST;
		\item \textbf{Livello di Accesso}: Utente autenticato/autenticato pro;
		\item \textbf{Descrizione}: Restituisce il quiz selezionato dall'utente per effettuare l'esercitazione
		\item \textbf{Response}: la richiesta deve contenere i seguenti campi:
\begin{lstlisting}[language=json,firstnumber=1]
{
"title" : [titolo del questionario]
"author" : [identifica l'autore del questionario]
"questions" : [Array di JSON contenente le domande che compongono il questionario]
}
\end{lstlisting}
	\end{itemize}
	
	
	\item \texttt{/:lang/user/quiz/:quizId/summary}
	\begin{itemize}
		\item \textbf{Method}: POST;
		\item \textbf{Livello di Accesso}: Utente autenticato/autenticato pro;
		\item \textbf{Descrizione}: Crea il riepilogo del questionario svolto;
		\item \textbf{Request}: la richiesta deve contenere i seguenti campi:
\begin{lstlisting}[language=json,firstnumber=1]
{
"_quizId" : [identificativo del questionario svolto]
"_userId" : [identificativo dell'utente che ha svolto il questionario]
"givenAnswers" : [Array di JSON contenente le domande del quiz associate alle risposte date dall'utente]
}
\end{lstlisting}
		\item \textbf{Response}: la risposta deve contenere i seguenti campi:
\begin{lstlisting}[language=json,firstnumber=1]
{
"mark" : [valutazione del questionario]
"answers" : [Array di JSON contenente le domande del quiz associate alle risposte date dall'utente e le risposte corrette]
}
\end{lstlisting}
	\end{itemize}
	\item \texttt{/summary/create}
		\begin{itemize}
			\item \textbf{Method}: POST;
			\item \textbf{Livello di Accesso}: utente;
			\item \textbf{Descrizione}: crea un riepilogo in base al questionario svolto e alle risposte date alle domande del questionario;
			\item \textbf{Request}: la richiesta deve avere i seguenti campi:
\begin{lstlisting}[language=json,firstnumber=1]
{
"_quizId" : [identificativo del quiz per cui verra' creato il riepilogo]
"givenAnswers" : [array contenente le risposte date alle domande del questionario svolto]
}
\end{lstlisting}
			\item \textbf{Response}: la risposta deve avere i seguenti dati:
\begin{lstlisting}[language=json,firstnumber=1]
{
"_summaryId" : [identificativo del riepilogo]
"comparedQuestions" : [array contenente booleani (true o false) che definiscono se le risposte date alle domande del questionario sono corrette o meno]
"questions" : [array di identificativi alle domande del questionario svolto]
"date" : [momento in cui viene creato il riepilogo]
}
\end{lstlisting}
		\end{itemize}
		\item \texttt{/:lang/training}
		\begin{itemize}
			\item \textbf{Method}: POST;
			\item \textbf{Livello di Accesso}: Utente;
			\item \textbf{Descrizione}: restituisce una domanda in base al livello di abilità raggiunto dall'utente;
			\item \textbf{Request}: la richiesta deve contenere i seguenti campi:
\begin{lstlisting}[language=json,firstnumber=1]
{
"topicTraining" : [indica l'argomento scelto per iniziare l'allenamento]
"numberQuestions" : [indica il numero di domande che componono l'allenamento]
"keyword" : [indica un array di parole chiavi per filtrare le domande]
"levelUser" : [indica il livello di abilita' sull'argomento scelto dell'utente]
}
\end{lstlisting}
			\item \textbf{Response}: la risposta deve contenere i seguenti campi:
\begin{lstlisting}[language=json,firstnumber=1]
{
"_questionId" : [identificativo della domanda]
"type" : [indica la tipologia della domanda]
"questionText" : [indica il testo della domanda]
"option1" : [indica un array di opzioni di risposta]
"option2" : [indica un array di opzioni di risposta] 
}
\end{lstlisting}
		\end{itemize}
		
		
	\item \texttt{/:lang/user/training}
		\begin{itemize}
			\item \textbf{Method}: POST;
			\item \textbf{Livello di Accesso}: Utente autenticato/autenticato pro;
			\item \textbf{Descrizione}: restituisce una domanda, aggiorna le statistiche della domanda e dell'utente;
			\item \textbf{Request}: la richiesta deve contenere i seguenti campi:
\begin{lstlisting}[language=json,firstnumber=1]
{
"topicTraining" : [indica l'argomento scelto per iniziare l'allenamento]
"numberQuestions" : [indica il numero di domande che componono l'allenamento]
"keyword" : [indica un array di parole chiavi per filtrare le domande]
"levelUser" : [indica il livello di abilita' sull'argomento scelto dell'utente]
}
\end{lstlisting}
		\item \textbf{Response}: la risposta deve contenere i seguenti campi:
\begin{lstlisting}[language=json,firstnumber=1]
{
"_questionId" : [identificativo della domanda]
"type" : [indica la tipologia della domanda]
"questionText" : [indica il testo della domanda]
"option1" : [indica un array di opzioni di risposta]
"option2" : [indica un array di opzioni di risposta] 
}
\end{lstlisting}
	\end{itemize}
\end{itemize}