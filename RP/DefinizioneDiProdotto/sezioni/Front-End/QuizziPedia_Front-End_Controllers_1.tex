\paragraph{QuizziPedia::Front-End::Controllers::LoginController}
\begin{figure} [ht]
	\centering
	\includegraphics[scale=0.45]{UML/Classi/Front-End/QuizziPedia_Front-end_Controller_LoginController.png}
	\caption{QuizziPedia::Front-End::Controllers::LoginController}
\end{figure} \FloatBarrier
\begin{itemize}
	\item \textbf{Descrizione}: questa classe permette di gestire l'autenticazione dell'utente al sistema; 
	\item \textbf{Utilizzo}: fornisce le funzionalità di autenticazione al sistema, compresa la gestione di situazioni di errore di autenticazione;
	\item \textbf{Relazione con altre classi}:
	\begin{itemize}
		\item \textit{IN} \texttt{LoginModelView}: classe di tipo modelview la cui istanzazione è contenuta all'interno della variabile di ambiente \$scope di \textit{Angular.js\ped{G}}. All'interno di essa sono presenti le variabili e i metodi necessari per il \textit{Two-Way Data-Binding\ped{G}} tra la view \texttt{LoginView} e il controller \texttt{LoginController};
		\item \textit{IN} \texttt{AuthService}: questa classe permette di gestire la registrazione e l'autenticazione di un utente;
	\end{itemize}
	\item \textbf{Attributi}:
	\begin{itemize}
		\item \texttt{-} \texttt{\$scope: \$scope} \\
		Campo dati contenente un riferimento all’oggetto \$scope creato da \textit{Angular\ped{G}}. Viene utilizzato come mezzo di comunicazione tra il controller e la view. Contiene gli oggetti che definiscono il viewmodel e il model dell’applicazione;
		\item \texttt{-} \texttt{\$location: \$location} \\
		Campo dati contenente un riferimento al servizio creato da \textit{Angular\ped{G}} che permette di accedere alla barra degli indirizzi del \textit{browser\ped{G}}, i cambiamenti all’URL nella barra degli indirizzi si riflettono in questo oggetto e viceversa;
		\item \texttt{-} \texttt{\$mdDialog: \$mdDialog} \\
		Campo dati contenente un riferimento al servizio della libreria \textit{Material for Angular\ped{G}} che permette di creare delle componenti a popup;
		\item \texttt{-} \texttt{AuthService: AuthService} \\
		Campo dati contenente un riferimento al servizio che si occupa della gestione delle informazioni legate all’autenticazione. Viene utilizzato il metodo \texttt{logIn} di \$texttt{AuthService} a cui vengono passati i parametri \texttt{username} e \texttt{password};
		\item \texttt{+} \texttt{user: LoginModelView} \\
		Oggetto di tipo \texttt{LoginModelView}. All'interno di esso sono presenti le variabili e i metodi necessari per il \textit{Two-Way Data-Binding\ped{G}} tra la view \texttt{LoginView} e il controller \texttt{LoginController};
	\end{itemize}
	\item \textbf{Metodi}:
	\begin{itemize}
		\item \texttt{+} \texttt{LoginController(\$scope: \$scope, \$rootScope: \$rootScope, \$location: \$location, \$mdDialog: \$mdDialog, AuthService: AuthService)} \\
		Metodo costruttore della classe: \\
		\textbf{Parametri}:
			\begin{itemize}
				\item \texttt{\$scope: \$scope} \\
				Parametro contenente un riferimento all’oggetto \$scope creato da \textit{Angular\ped{G}}. Viene utilizzato come mezzo di comunicazione tra il controller e la view. Contiene gli oggetti che definiscono il viewmodel e il model dell’applicazione;
				\item \texttt{\$location: \$location} \\
				Parametro contenente un riferimento al servizio creato da \textit{Angular\ped{G}} che permette di accedere alla barra degli indirizzi del \textit{browser\ped{G}}, i cambiamenti all’URL nella barra degli indirizzi si riflettono in questo oggetto e viceversa;
				\item \texttt{\$mdDialog: \$mdDialog} \\
				Parametro contenente un riferimento al servizio della libreria \textit{Material for Angular\ped{G}} che permette di creare delle componenti a popup;
				\item \texttt{AuthService: AuthService} \\
				Parametro contenente un riferimento al servizio che si occupa della gestione delle informazioni legate all’autenticazione. Viene utilizzato il metodo \texttt{logIn} di \$texttt{AuthService} a cui vengono passati i parametri \texttt{username} e \texttt{password};
			\end{itemize}
		\item \texttt{+} \texttt{logIn(): void} \\
		Metodo che richiama il metodo \texttt{Login} del service \texttt{AuthService} passandogli \texttt{username} e \texttt{password}. Nel caso di buona riuscita dell'operazione viene effettuato il redirect alla homepage dell'applicazione. Nel caso in cui invece avvenga un errore, viene mostrato a video il messaggio di errore;
		\item \texttt{+} \texttt{signUp(): void} \\
		Metodo che gestisce l’evento click sul pulsante di registrazione. Effettua il redirect alla pagina di registrazione;
		\item \texttt{+} \texttt{recoveryPassword(): void} \\
		Metodo che gestisce l’evento click sul pulsante di recupero password. Effettua il redirect alla pagina per il recupero della password; 
	\end{itemize}
\end{itemize}

\paragraph{QuizziPedia::Front-End::Controllers::SignUpController}
\begin{figure} [ht]
	\centering
	\includegraphics[scale=0.45]{UML/Classi/Front-End/QuizziPedia_Front-end_Controller_SignUpController.png}
	\caption{QuizziPedia::Front-End::Controllers::SignUpController}
\end{figure} \FloatBarrier
\begin{itemize}
	\item \textbf{Descrizione}: questa classe permette di gestire la registrazione di un utente al sistema;
	\item \textbf{Utilizzo}: fornisce le funzionalità di registrazione di un utente al sistema;
	\item \textbf{Relazione con altre classi}:
	\begin{itemize}
		\item \textit{IN} \texttt{SignUpModelView}: classe di tipo modelview la cui istanzazione è contenuta all'interno della variabile di ambiente \$scope di \textit{Angular.js\ped{G}}. All'interno di essa sono presenti le variabili e i metodi necessari per il \textit{Two-Way Data-Binding\ped{G}} tra la view \texttt{SignUpView} e il controller \texttt{SignUpController};
		\item \textit{IN} \texttt{AuthService}: questa classe permette di gestire la registrazione e l'autenticazione di un utente;
	\end{itemize}
	\item \textbf{Attributi}:
	\begin{itemize}
		\item \texttt{-} \texttt{\$scope: \$scope} \\
		Campo dati contenente un riferimento all’oggetto \$scope creato da \textit{Angular\ped{G}}. Viene utilizzato come mezzo di comunicazione tra il controller e la view. Contiene gli oggetti che definiscono il viewmodel e il model dell’applicazione;
		\item \texttt{-} \texttt{\$location: \$location} \\
		Campo dati contenente un riferimento al servizio creato da \textit{Angular\ped{G}} che permette di accedere alla barra degli indirizzi del \textit{browser\ped{G}}, i cambiamenti all’URL nella barra degli indirizzi si riflettono in questo oggetto e viceversa;
		\item \texttt{-} \texttt{\$mdDialog: \$mdDialog} \\
		Campo dati contenente un riferimento al servizio della libreria \textit{Material for Angular\ped{G}} che permette di creare delle componenti a popup;
		\item \texttt{-} \texttt{AuthService: AuthService} \\
		Campo dati contenente un riferimento al servizio che si occupa della gestione delle informazioni legate all’autenticazione. Viene utilizzato il metodo \texttt{signUp} di \$texttt{AuthService} a cui viene passato come parametro un oggetto di tipo \texttt{SignUpModelView};
		\item \texttt{+} \texttt{newUser: SignUpModelView} \\
		Oggetto di tipo \texttt{SignUpModelView}. All'interno di esso sono presenti le variabili e i metodi necessari per il \textit{Two-Way Data-Binding\ped{G}} tra la view \texttt{SignUpView} e il controller \texttt{SignUpController};
	\end{itemize}
	\item \textbf{Metodi}:
	\begin{itemize}
		\item \texttt{+} \texttt{SignUpController(\$scope: \$scope, \$location: \$location, \$mdDialog: \$mdDialog, AuthService: AuthService)} \\
		Metodo costruttore della classe: \\
		\textbf{Parametri}:
		\begin{itemize}
			\item \texttt{\$scope: \$scope} \\
			Parametro contenente un riferimento all’oggetto \$scope creato da \textit{Angular\ped{G}}. Viene utilizzato come mezzo di comunicazione tra il controller e la view. Contiene gli oggetti che definiscono il viewmodel e il model dell’applicazione;
			\item \texttt{\$location: \$location} \\
			Parametro contenente un riferimento al servizio creato da \textit{Angular\ped{G}} che permette di accedere alla barra degli indirizzi del \textit{browser\ped{G}}, i cambiamenti all’URL nella barra degli indirizzi si riflettono in questo oggetto e viceversa;
			\item \texttt{\$mdDialog: \$mdDialog} \\
			Parametro contenente un riferimento al servizio della libreria \textit{Material for Angular\ped{G}} che permette di creare delle componenti a popup;
			\item \texttt{AuthService: AuthService} \\
			Campo dati contenente un riferimento al servizio che si occupa della gestione delle informazioni legate all’autenticazione. Viene utilizzato il metodo \texttt{logIn} di \$texttt{AuthService} a cui vengono passati i parametri \texttt{username} e \texttt{password};
		\end{itemize}
		\item \texttt{+} \texttt{signUp(): void} \\
		Metodo che richiama il metodo \texttt{signUp} del service \texttt{AuthService} passandogli un oggetto di tipo \texttt{SignUpModelView}. Nel caso di buona riuscita dell'operazione viene mostrato un messaggio di successo. Con l'azione di click sul bottone presentato dal messaggio di successo è possibile effettuare il redirect alla pagina di login dell'applicazione. Nel caso in cui invece avvenga un errore, viene mostrato a video il messaggio di errore;
		\item \texttt{+} \texttt{logIn(): void} \\
		Metodo che gestisce l’evento click sul pulsante di login. Effettua il redirect alla pagina di login;
	
	\end{itemize}
\end{itemize}

\paragraph{QuizziPedia::Front-End::Controllers::HomeController}
\begin{figure} [ht]
	\centering
	\includegraphics[scale=0.45]{UML/Classi/Front-End/QuizziPedia_Front-end_Controller_HomeController.png}
	\caption{QuizziPedia::Front-End::Controllers::HomeController}
\end{figure} \FloatBarrier
\begin{itemize}
	\item \textbf{Descrizione}: questa classe permette di gestire la home page;
	\item \textbf{Utilizzo}: fornisce tutte le informazioni da mostrare nella homepage;
	\item \textbf{Relazione con altre classi}:
	\begin{itemize}
		\item \textit{IN} \texttt{HomeModelView}: classe di tipo modelview la cui istanzazione è contenuta all'interno della variabile di ambiente \$scope di \textit{Angular.js\ped{G}}. All'interno di essa sono presenti le variabili e i metodi necessari per il \textit{Two-Way Data-Binding\ped{G}} tra la view \texttt{HomeView} e il controller \texttt{HomeController};
	\end{itemize}
	\item \textbf{Attributi}:
	\begin{itemize}
		\item \texttt{-} \texttt{\$scope: \$scope} \\
		Campo dati contenente un riferimento all’oggetto \$scope creato da \textit{Angular\ped{G}}, viene utilizzato come mezzo di comunicazione tra il controller e la view. Contiene gli oggetti che definiscono il model dell’applicazione;
		\item \texttt{-} \texttt{\$location: \$location} \\
		Campo dati contenente un riferimento al servizio creato da \textit{Angular\ped{G}} che permette di accedere alla barra degli indirizzi del \textit{browser\ped{G}}, i cambiamenti all’URL nella barra degli indirizzi si riflettono in questo oggetto e viceversa;
	\end{itemize}
	\item \textbf{Metodi}:
	\begin{itemize}
		\item \texttt{+} \texttt{HomeController(\$scope: \$scope, \$location: \$location)} \\
		Metodo costruttore della classe: \\
		\textbf{Parametri}:
		\begin{itemize}
			\item \texttt{\$scope: \$scope} \\
			Parametro contenente un riferimento all’oggetto \$scope creato da \textit{Angular\ped{G}}. Viene utilizzato come mezzo di comunicazione tra il controller e la view. Contiene gli oggetti che definiscono il viewmodel e il model dell’applicazione;
			\item \texttt{\$location: \$location} \\
			Parametro contenente un riferimento al servizio creato da \textit{Angular\ped{G}} che permette di accedere alla barra degli indirizzi del \textit{browser\ped{G}}, i cambiamenti all’URL nella barra degli indirizzi si riflettono in questo oggetto e viceversa;
		\end{itemize}
		\item \texttt{+} \texttt{trainingMode(): void} \\
		Metodo che gestisce l’evento click sul pulsante di allenamento. Effettua il redirect alla pagina di allenamento; 
	\end{itemize}
\end{itemize}

\paragraph{QuizziPedia::Front-End::Controllers::SearchController}
\begin{figure} [ht]
	\centering
	\includegraphics[scale=0.45]{UML/Classi/Front-End/QuizziPedia_Front-end_Controller_SearchController.png}
	\caption{QuizziPedia::Front-End::Controllers::SearchController}
\end{figure} \FloatBarrier
\begin{itemize}
	\item \textbf{Descrizione}: questa classe permette di gestire la ricerca di questionari e utenti all'interno dell'applicazione;
	\item \textbf{Utilizzo}: fornisce all'utente le funzionalità di ricerca per utenti e questionari;
	\item \textbf{Relazione con altre classi}:
	\begin{itemize}
		\item \textit{IN} \texttt{ResultsModelView}: classe di tipo modelview la cui istanzazione è contenuta all'interno della variabile di ambiente \$scope di \textit{Angular.js\ped{G}}. All'interno di essa sono presenti le variabili e i metodi necessari per il \textit{Two-Way Data-Binding\ped{G}} tra la view \texttt{ResultsView}, la directive \texttt{SearchDirective} e il controller \texttt{ResultsController};
		\item \textit{IN} \texttt{SearchService}: questa classe permette di eseguire una ricerca tra i questionari e gli utenti presenti ritornando un \textit{Object} contenente i risultati di tale operazione;
		\item \textit{IN} \texttt{QuizService}: questa classe permette di ottenere i dati di un quiz tramite delle parole chiave inserite dall'utente nella barra di ricerca. Permette inoltre di iscriversi ad un questionario;
	\end{itemize}
	\item \textbf{Attributi}:
	\begin{itemize}
		\item \texttt{-} \texttt{\$scope: \$scope} \\
		Campo dati contenente un riferimento all’oggetto \$scope creato da \textit{Angular\ped{G}}, viene utilizzato come mezzo di comunicazione tra il controller e la view. Contiene gli oggetti che definiscono il model dell’applicazione;
		\item \texttt{-} \texttt{\$location: \$location} \\
		Campo dati contenente un riferimento al servizio creato da \textit{Angular\ped{G}} che permette di accedere alla barra degli indirizzi del \textit{browser\ped{G}}, i cambiamenti all’URL nella barra degli indirizzi si riflettono in questo oggetto e viceversa;
		\item \texttt{-} \texttt{\$mdDialog: \$mdDialog} \\
		Campo dati contenente un riferimento al servizio della libreria \textit{Material for Angular\ped{G}} che permette di creare delle componenti a popup;
		\item \texttt{-} \texttt{SearchService: SearchService} \\
		Campo dati contenente un riferimento al servizio che si occupa della gestione delle informazioni legate alla ricerca. Viene utilizzato il metodo \texttt{search} di \$texttt{SearchService} a cui viene passato come parametro la stringa di ricerca;
		\item \texttt{-} \texttt{QuizService: QuizService} \\
		Campo dati contenente un riferimento al servizio che si occupa della gestione delle informazioni legate ai questionari. Viene utilizzato il metodo \texttt{subscribeQuestionnaire} di \$texttt{QuizService} per iscrivere un utente ad un questionario;
		\item \texttt{+} \texttt{result: SearchModelView} \\
		Oggetto di tipo \texttt{SearchModelView}. All'interno di esso sono presenti le variabili e i metodi necessari per il \textit{Two-Way Data-Binding\ped{G}} tra la view \texttt{ResultView} e il controller \texttt{SearchController};
	\end{itemize}
	\item \textbf{Metodi}:
	\begin{itemize}
		\item \texttt{+} \texttt{SearchController(\$scope: \$scope, \$location: \$location, \$mdDialog: \$mdDialog, SearchService: SearchService)} \\
		Metodo costruttore della classe. Viene eseguita la ricerca per poter poi popolare il campo dati \texttt{result}. \\
		\textbf{Parametri}:
		\begin{itemize}
			\item \texttt{\$scope: \$scope} \\
			Parametro contenente un riferimento all’oggetto \$scope creato da \textit{Angular\ped{G}}. Viene utilizzato come mezzo di comunicazione tra il controller e la view. Contiene gli oggetti che definiscono il viewmodel e il model dell’applicazione;
			\item \texttt{\$location: \$location} \\
			Parametro contenente un riferimento al servizio creato da \textit{Angular\ped{G}} che permette di accedere alla barra degli indirizzi del \textit{browser\ped{G}}, i cambiamenti all’URL nella barra degli indirizzi si riflettono in questo oggetto e viceversa;
			\item \texttt{\$mdDialog: \$mdDialog} \\
			Parametro contenente un riferimento al servizio della libreria \textit{Material for Angular\ped{G}} che permette di creare delle componenti a popup;
			\item \texttt{SearchService: SearchService} \\
			Parametro contenente un riferimento al servizio che si occupa della gestione delle informazioni legate alla ricerca. Viene utilizzato il metodo \texttt{search} di \$texttt{SearchService} a cui viene passato come parametro la stringa di ricerca;
		\end{itemize} 
		\item \texttt{-} \texttt{getSearch(stringSearch: String): SearchModelView} \\
		Metodo che esegue la ricerca utilizzando un metodo fornito dalla classe SearchService. \\
		\textbf{Parametri}:
		\begin{itemize}
			\item \texttt{stringSearch: String} \\
			Parametro contenente la stringa della quale effettuare la ricerca;
		\end{itemize} 
		\item \texttt{+} \texttt{goToUser(idUser: String): void} \\
		Metodo che gestisce l’evento click sul bottone per visualizzare il profilo dell'utente selezionato. Effettua il redirect alla pagina dell'utente.\\
		\textbf{Parametri}:
		\begin{itemize}
			\item \texttt{idUser: String} \\
			Parametro contenente l'id dell'utente di cui si vuole visualizzare il profilo;
		\end{itemize} 
		\item \texttt{+} \texttt{registrationToQuiz(idQuiz: String): void} \\
		Metodo che gestisce l’evento click sul pulsante di registrazione al questionario.\\
		\textbf{Parametri}:
		\begin{itemize}
			\item \texttt{idQuiz: String} \\
			Parametro contenente l'id del questionario di cui si vuole effettuare l'iscrizione;
		\end{itemize} 
		\item \texttt{+} \texttt{goToResultsPage(stringSearch: String): void} \\
		Metodo che gestisce l’evento click sul pulsante per effettuare una ricerca.\\
		\textbf{Parametri}:
		\begin{itemize}
			\item \texttt{stringSearch: String} \\
			Parametro contenente la stringa della quale effettuare la ricerca;
		\end{itemize} 
	\end{itemize}
\end{itemize}

\paragraph{QuizziPedia::Front-End::Controllers::ProfileManagementController}
\begin{figure} [ht]
	\centering
	\includegraphics[scale=0.45]{UML/Classi/Front-End/QuizziPedia_Front-end_Controller_ProfileManagementController.png}
	\caption{QuizziPedia::Front-End::Controllers::ProfileManagementController}
\end{figure} \FloatBarrier
\begin{itemize}
	\item \textbf{Descrizione}: questa classe permette di gestire il profilo personale di un utente; 
	\item \textbf{Utilizzo}: fornisce le funzionalità all'utente per poter gestire i propri dati;
	\item \textbf{Relazione con altre classi}:
	\begin{itemize}
		\item \textit{IN} \texttt{ProfileManagementModelView}: classe di tipo modelview la cui istanzazione è contenuta all'interno della variabile di ambiente \$scope di \textit{Angular.js\ped{G}}. All'interno di essa sono presenti le variabili e i metodi necessari per il \textit{Two-Way Data-Binding\ped{G}} tra la view \texttt{ProfileManagementView} e il controller \texttt{ProfileManagementController};
		\item \textit{IN} \texttt{UserDetailsService}: questa classe permette di ottenere i dati personali degli utenti;
		\item \textit{IN} \texttt{UserDetailsModel}: questa classe rappresenta il tipo dell'utente autenticato della pagina; 
	\end{itemize}
	\item \textbf{Attributi}:
	\begin{itemize}
		\item \texttt{-} \texttt{\$scope: \$scope} \\
		Campo dati contenente un riferimento all’oggetto \$scope creato da \textit{Angular\ped{G}}, viene utilizzato come mezzo di comunicazione tra il controller e la view. Contiene gli oggetti che definiscono il model dell’applicazione;
		\item \texttt{-} \texttt{\$rootScope: \$rootScope} \\
		Campo dati contenente il riferimento all'oggetto globale \$rootScope creato da \textit{Angular\ped{G}}. Viene utilizzato per rendere accessibile a tutti i controller e a tutte le view l'oggetto \texttt{UserDetailsModel};
		\item \texttt{-} \texttt{\$mdDialog: \$mdDialog} \\
		Campo dati contenente un riferimento al servizio della libreria \textit{Material for Angular\ped{G}} che permette di creare delle componenti a popup;		
		\item \texttt{-} \texttt{UserDetailsService: UserDetailsService}: \\
		Campo dati contenente un riferimento al servizio che si occupa della gestione delle informazioni legate agli utenti;
		\item \texttt{+} \texttt{user: UserDetailsModel}: \\
		Oggetto di tipo \texttt{UserDetailsModel}. Viene mantenuto all'interno del \$rootScope; 
		\item \texttt{-} \texttt{Upload: Upload} \\
		Campo dati contenente un riferimento alla libreria \textit{ng-file-upload\ped{G}} necessaria per il caricamento della foto profilo dell'utente;
		\item \texttt{-} \texttt{\$timeout: \$timeout} \\
		Campo dati contenente il riferimento all'oggetto globale \$timeout creato da \textit{Angular.js\ped{G}}. 
		Il valore di ritorno di una chiamata alla funzione di \texttt{\$timeout} è una promise, la quale sarà risolta quando avverrà il ritardo e la funzione di timeout eseguita; 
	\end{itemize}
	\item \textbf{Metodi}:
	\begin{itemize}
		\item \texttt{+} \texttt{ProfileManagementController(\$scope: \$scope, \$rootScope: \$rootScope, \$mdDialog: \$mdDialog, UserDetailsService: UserDetailsService)} \\
		Metodo costruttore della classe. \\
		\textbf{Parametri}:
		\begin{itemize}
			\item \texttt{\$scope: \$scope} \\
			Parametro contenente un riferimento all’oggetto \$scope creato da \textit{Angular\ped{G}}. Viene utilizzato come mezzo di comunicazione tra il controller e la view. Contiene gli oggetti che definiscono il viewmodel e il model dell’applicazione;
			\item \texttt{\$rootScope: \$rootScope} \\
			Parametro contenente il riferimento all'oggetto globale \$rootScope creato da \textit{Angular\ped{G}}. Viene utilizzato per rendere accessibile a tutti i controller e a tutte le view l'oggetto \texttt{UserDetailsModel}. In questo caso viene utilizzato per aggiornare in \$rootScope l'oggetto che rappresenta l'utente autenticato all'interno dell'applicazione;
			\item \texttt{\$location: \$location} \\
			Parametro contenente un riferimento al servizio creato da \textit{Angular\ped{G}} che permette di accedere alla barra degli indirizzi del \textit{browser\ped{G}}, i cambiamenti all’URL nella barra degli indirizzi si riflettono in questo oggetto e viceversa;
			\item \texttt{\$mdDialog: \$mdDialog} \\
			Parametro contenente un riferimento al servizio della libreria \textit{Material for Angular\ped{G}} che permette di creare delle componenti a popup;
			\item \texttt{UserDetailsService: UserDetailsService} \\
			Parametro contenente un riferimento al servizio che si occupa della gestione delle informazioni legate all’utente;
			\item \texttt{Upload: Upload} \\
			Parametro contenente un riferimento alla libreria \textit{ng-file-upload} necessaria per il caricamento della foto profilo dell'utente;
			\item \texttt{\$timeout: \$timeout} \\
			Parametro contenente il riferimento all'oggetto globale \$timeout creato da \textit{Angular.js\ped{G}}. 
			Il valore di ritorno di una chiamata alla funzione di \texttt{\$timeout} è una promise, la quale sarà risolta quando avverrà il ritardo e la funzione di timeout eseguita; 
		\end{itemize}
		\item \texttt{+} \texttt{confirm(): void} \\
		Metodo che gestisce l’evento click sul pulsante di conferma modifica. Aggiorna, in caso di modifiche, l'oggetto locale \texttt{UserDetailsModel}. Inoltre, utilizzando il metodo dell'\texttt{UserDetailsService}, aggiorna anche nel server i dati dell'utente.
	\end{itemize}
\end{itemize}

\paragraph{QuizziPedia::Front-End::Controllers::PasswordForgotController}
\begin{figure} [ht]
	\centering
	\includegraphics[scale=0.45]{UML/Classi/Front-End/QuizziPedia_Front-end_Controller_PasswordForgotController.png}
	\caption{QuizziPedia::Front-End::Controllers::PasswordForgotController}
\end{figure} \FloatBarrier
\begin{itemize}
	\item \textbf{Descrizione}: questa classe permette di gestire il ripristino della password dimenticata;
	\item \textbf{Utilizzo}: fornisce tutte le funzionalità per ripristinare la password dopo aver verificato l'identità dell'utente;
	\item \textbf{Relazione con altre classi}:
	\begin{itemize}
		\item \textit{IN} \texttt{PasswordForgotModelView}: classe di tipo modelview la cui istanzazione è contenuta all'interno della variabile di ambiente \$scope di \texttt{Angular.js}. All'interno di essa sono presenti le variabili e i metodi necessari per il \textit{Two-Way Data-Binding\ped{G}} tra la view \texttt{PasswordForgotView} e il controller \texttt{PasswordForgotController};
		\item \textit{IN} \texttt{AuthService}: questa classe permette di gestire la registrazione e l'autenticazione di un utente;
	\end{itemize}
	\item \textbf{Attributi}:
	\begin{itemize}
		\item \texttt{-} \texttt{\$scope: \$scope} \\
		Campo dati contenente un riferimento all’oggetto \$scope creato da \textit{Angular\ped{G}}, viene utilizzato come mezzo di comunicazione tra il controller e la view. Contiene gli oggetti che definiscono il model dell’applicazione;
		\item \texttt{-} \texttt{\$location: \$location} \\
		Campo dati contenente un riferimento al servizio creato da \textit{Angular\ped{G}} che permette di accedere alla barra degli indirizzi del \textit{browser\ped{G}}, i cambiamenti all’URL nella barra degli indirizzi si riflettono in questo oggetto e viceversa;
		\item \texttt{-} \texttt{\$mdDialog: \$mdDialog} \\
		Campo dati contenente un riferimento al servizio della libreria \textit{Material for Angular\ped{G}} che permette di creare delle componenti a popup;
		\item \texttt{-} \texttt{AuthService: AuthService} \\
		Campo dati contenente un riferimento al servizio che si occupa della gestione delle informazioni legate all’autenticazione. Viene utilizzato il metodo \texttt{passwordForgot} di \$texttt{AuthService} a cui viene passato il parametro \texttt{email};
	\end{itemize}
	\item \textbf{Metodi}:
	\begin{itemize}
		\item \texttt{+} \texttt{PasswordForgotController(\$scope: \$scope, \$location: \$location, \$mdDialog: \$mdDialog, AuthService: AuthService)} \\
		Metodo costruttore della classe; \\
		\textbf{Parametri}:
		\begin{itemize}
			\item \texttt{\$scope: \$scope} \\
			Parametro contenente un riferimento all’oggetto \$scope creato da \textit{Angular\ped{G}}. Viene utilizzato come mezzo di comunicazione tra il controller e la view. Contiene gli oggetti che definiscono il viewmodel e il model dell’applicazione;
			\item \texttt{\$location: \$location} \\
			Parametro contenente un riferimento al servizio creato da \textit{Angular\ped{G}} che permette di accedere alla barra degli indirizzi del \textit{browser\ped{G}}, i cambiamenti all’URL nella barra degli indirizzi si riflettono in questo oggetto e viceversa;
			\item \texttt{\$mdDialog: \$mdDialog} \\
			Parametro contenente un riferimento al servizio della libreria \textit{Material for Angular\ped{G}} che permette di creare delle componenti a popup;
			\item \texttt{AuthService: AuthService} \\
			Campo dati contenente un riferimento al servizio che si occupa della gestione delle informazioni legate all’autenticazione. Viene utilizzato il metodo \texttt{passwordForgot} di \$texttt{AuthService} a cui viene passato il parametro \texttt{email};
		\end{itemize}
		\item \texttt{+} \texttt{passwordForgot(): void} \\
		Metodo che richiama il metodo \texttt{passwordForgot} del service \texttt{AuthService} passandogli il parametro \texttt{email}. Nel caso di buona riuscita dell'operazione, viene mostrato un messaggio di successo il cui corpo contiene anche un bottone per effettuare il redirect alla pagina di login. Nel caso in cui invece avvenga un errore, viene mostrato a video il messaggio di errore;
		\item \texttt{+} \texttt{logIn(): void} \\
		Metodo che gestisce l’evento click sul pulsante di login. Effettua il redirect alla pagina di login;
	\end{itemize}
\end{itemize}

\paragraph{QuizziPedia::Front-End::Controllers::TrueFalseQuestionsController}
\begin{figure} [ht]
	\centering
	\includegraphics[scale=0.45]{UML/Classi/Front-End/QuizziPedia_Front-end_Controller_TrueFalseQuestionsController.png}
	\caption{QuizziPedia::Front-End::Controllers::TrueFalseQuestionsController}
\end{figure} \FloatBarrier
\begin{itemize}
	\item \textbf{Descrizione}: questa classe permette di gestire la creazione e la modifica di una domanda vero/falso;
	\item \textbf{Utilizzo}: fornisce le funzionalità per inserire una nuova domanda vero/falso nel database e per modificarne una esistente;
	\item \textbf{Relazione con altre classi}:
	\begin{itemize}
		\item \textit{IN} \texttt{TrueFalseQuestionsModelView}: classe di tipo modelview la cui istanzazione è contenuta all'interno della variabile di ambiente \$scope di \textit{Angular.js\ped{G}}. All'interno di essa sono presenti le variabili e i metodi necessari per il \textit{Two-Way Data-Binding\ped{G}} tra la view \texttt{TrueFalseQuestionsView} e il controller \texttt{TrueFalseQuestionsController};
		\item \textit{IN} \texttt{QuestionService}: questa classe permette di:
			\begin{itemize}
				\item Ottenere una domanda attraverso il metodo dedicato;
				\item Caricare una domanda modificata;
				\item Caricare una nuova domanda.
			\end{itemize}
		\item \textit{IN} \texttt{QuestionItemModel}: questa classe rappresenta il modello di una domanda;
	\end{itemize}
	\item \textbf{Attributi}:
	\begin{itemize}
		\item \texttt{-} \texttt{\$scope: \$scope} \\
		Campo dati contenente un riferimento all’oggetto \$scope creato da \textit{Angular\ped{G}}, viene utilizzato come mezzo di comunicazione tra il controller e la view. Contiene gli oggetti che definiscono il model dell’applicazione;
		\item \texttt{-} \texttt{QuestionItemModel: QuestionItemModel} \\
		Campo dati che si riferisce alla classe che rappresenta il modello della classe;
		\item \texttt{-} \texttt{\$mdDialog: \$mdDialog} \\
		Campo dati contenente un riferimento al servizio della libreria \textit{Material for Angular\ped{G}} che permette di creare delle componenti a popup;
		\item \texttt{-} \texttt{QuestionService: QuestionService}: \\
		Campo dati contenente un riferimento al servizio che si occupa della gestione delle informazioni legate alle domande;
		\item \texttt{-} \texttt{Upload: Upload} \\
		Campo dati contenente un riferimento alla libreria \textit{ng-file-upload\ped{G}} necessaria per il caricamento della foto profilo dell'utente;
		\item \texttt{-} \texttt{\$timeout: \$timeout} \\
		Campo dati contenente il riferimento all'oggetto globale \$timeout creato da \textit{Angular.js\ped{G}}. 
		Il valore di ritorno di una chiamata alla funzione di \texttt{\$timeout} è una promise, la quale sarà risolta quando avverrà il ritardo e la funzione di timeout eseguita; 
		\item \texttt{-} \texttt{\$routeParams: \$routeParams} \\
		Campo dati contenente il riferimento all'oggetto globale \$routeParams creato da \textit{Angular.js\ped{G}}. Tale servizio permette di recuperare il set di variabili presenti nell'url; 
	\end{itemize}
	\item \textbf{Metodi}:
	\begin{itemize}
		\item \texttt{+} \texttt{TrueFalseQuestionsController(\$scope: \$scope, QuestionItemModel: QuestionItemModel, \$mdDialog: \$mdDialog, QuestionService: QuestionService, Upload: Upload, \$timeout: \$timeout, \$routeParams: \$routeParams)} \\ 
		Metodo costruttore della classe. Se in \texttt{\$routeParams} sarà presente il codice univoco che rappresenta una domanda e di questa il creatore è l'utente autenticato, allora verrà scaricato attraverso il \texttt{QuestionService} il contenuto della domanda così da poterlo modificare. In caso contrario verrà mostrato un errore attraverso \texttt{\$mdDialog} indicando che i privilegi per tale operazione sono negati. Nel caso in cui non ci sarà tale parametro in \texttt{\$routeParams} verrà caricata la view vuota così da poter creare una nuova domanda; \\
		\textbf{Parametri}:
		\begin{itemize}
			\item \texttt{\$scope: \$scope} \\
			Parametro contenente un riferimento all’oggetto \$scope creato da \textit{Angular\ped{G}}, viene utilizzato come mezzo di comunicazione tra il controller e la view. Contiene gli oggetti che definiscono il model dell’applicazione;
			\item \texttt{QuestionItemModel: QuestionItemModel} \\ 
			Parametro che si riferisce alla classe che rappresenta il modello della classe;
			\item \texttt{\$mdDialog: \$mdDialog} \\
			Parametro contenente un riferimento al servizio della libreria \textit{Material for Angular\ped{G}} che permette di creare delle componenti a popup;
			\item \texttt{QuestionService: QuestionService}: \\
			Parametro contenente un riferimento al servizio che si occupa della gestione delle informazioni legate alle domande;
			\item \texttt{Upload: Upload} \\
			Parametro contenente un riferimento alla libreria \textit{ng-file-upload\ped{G}} necessaria per il caricamento della foto profilo dell'utente;
			\item \texttt{\$timeout: \$timeout} \\
			Parametro contenente il riferimento all'oggetto globale \$timeout creato da \textit{Angular.js\ped{G}}. 
			Il valore di ritorno di una chiamata alla funzione di \texttt{\$timeout} è una promise, la quale sarà risolta quando avverrà il ritardo e la funzione di timeout eseguita; 
			\item \texttt{\$routeParams: \$routeParams} \\
			Parametro contenente il riferimento all'oggetto globale \$routeParams creato da \textit{Angular.js\ped{G}}. Tale servizio permette di recuperare il set di variabili presenti nell'url; 
		\end{itemize}
		\item \texttt{+} \texttt{submitQuestion(): void}\\ 
		Metodo che gestisce l’evento click sul pulsante di conferma sulla domanda. Raccoglie i dati dal modelview e li manda al server attraverso \texttt{QuestionService}. Poi verrà effettuato il redirect alla pagina di gestione delle domande oppure al questionario che si stava creando; 
	\end{itemize}
\end{itemize}

\paragraph{QuizziPedia::Front-End::Controllers::MultipleQuestionsController}
\begin{figure} [ht]
	\centering
	\includegraphics[scale=0.45]{UML/Classi/Front-End/QuizziPedia_Front-end_Controller_MultipleQuestionsController.png}
	\caption{QuizziPedia::Front-End::Controllers::MultipleChoiceQuestion}
\end{figure} \FloatBarrier
\begin{itemize}
	\item \textbf{Descrizione}: questa classe permette di gestire la creazione e la modifica di una domanda a risposta multipla;
	\item \textbf{Utilizzo}: fornisce le funzionalità per inserire una nuova domanda a risposta multipla nel database e per modificarne una esistente;
	\item \textbf{Relazione con altre classi}:
	\begin{itemize}
		\item \textit{IN} \texttt{MultipleQuestionsModelView}: classe di tipo modelview la cui istanzazione è contenuta all'interno della variabile di ambiente \$scope di \textit{Angular.js\ped{G}}. All'interno di essa sono presenti le variabili e i metodi necessari per il \textit{Two-Way Data-Binding\ped{G}} tra la view \texttt{MultipleQuestionsView} e il controller \texttt{MultipleQuestionsController};
		\item \textit{IN} \texttt{QuestionService}: questa classe permette di:
		\begin{itemize}
			\item Ottenere una domanda attraverso il metodo dedicato;
			\item Caricare una domanda modificata;
			\item Caricare una nuova domanda.
		\end{itemize}
		\item \textit{IN} \texttt{QuestionItemModel}: questa classe rappresenta il modello di una domanda;
	\end{itemize}
	\item \textbf{Attributi}:
	\begin{itemize}
		\item \texttt{-} \texttt{\$scope: \$scope} \\
		Campo dati contenente un riferimento all’oggetto \$scope creato da \textit{Angular\ped{G}}, viene utilizzato come mezzo di comunicazione tra il controller e la view. Contiene gli oggetti che definiscono il model dell’applicazione;
		\item \texttt{-} \texttt{QuestionItemModel: QuestionItemModel} \\
		Campo dati che si riferisce alla classe che rappresenta il modello della classe;
		\item \texttt{-} \texttt{\$mdDialog: \$mdDialog} \\
		Campo dati contenente un riferimento al servizio della libreria \textit{Material for Angular\ped{G}} che permette di creare delle componenti a popup;
		\item \texttt{-} \texttt{QuestionService: QuestionService}: \\
		Campo dati contenente un riferimento al servizio che si occupa della gestione delle informazioni legate alle domande;
		\item \texttt{-} \texttt{Upload: Upload} \\
		Campo dati contenente un riferimento alla libreria \textit{ng-file-upload\ped{G}} necessaria per il caricamento della foto profilo dell'utente;
		\item \texttt{-} \texttt{\$timeout: \$timeout} \\
		Campo dati contenente il riferimento all'oggetto globale \$timeout creato da \textit{Angular.js\ped{G}}. 
		Il valore di ritorno di una chiamata alla funzione di \texttt{\$timeout} è una promise, la quale sarà risolta quando avverrà il ritardo e la funzione di timeout eseguita; 
		\item \texttt{-} \texttt{\$routeParams: \$routeParams} \\
		Campo dati contenente il riferimento all'oggetto globale \$routeParams creato da \textit{Angular.js\ped{G}}. Tale servizio permette di recuperare il set di variabili presenti nell'url; 
	\end{itemize}
	\item \textbf{Metodi}:
	\begin{itemize}
		\item \texttt{+} \texttt{MultipleQuestionsController(\$scope: \$scope, QuestionItemModel: QuestionItemModel, \$mdDialog: \$mdDialog, QuestionService: QuestionService, Upload: Upload, \$timeout: \$timeout, \$routeParams: \$routeParams)} \\ 
		Metodo costruttore della classe. Se in \texttt{\$routeParams} sarà presente il codice univoco che rappresenta una domanda e di questa il creatore è l'utente autenticato, allora verrà scaricato attraverso il \texttt{QuestionService} il contenuto della domanda così da poterlo modificare. In caso contrario verrà mostrato un errore attraverso \texttt{\$mdDialog} indicando che i privilegi per tale operazione sono negati. Nel caso in cui non ci sarà tale parametro in \texttt{\$routeParams} verrà caricata la view vuota così da poter creare una nuova domanda; \\
		\textbf{Parametri}:
		\begin{itemize}
			\item \texttt{\$scope: \$scope} \\
			Parametro contenente un riferimento all’oggetto \$scope creato da \textit{Angular\ped{G}}, viene utilizzato come mezzo di comunicazione tra il controller e la view. Contiene gli oggetti che definiscono il model dell’applicazione;
			\item \texttt{QuestionItemModel: QuestionItemModel} \\ 
			Parametro che si riferisce alla classe che rappresenta il modello della classe;
			\item \texttt{\$mdDialog: \$mdDialog} \\
			Parametro contenente un riferimento al servizio della libreria \textit{Material for Angular\ped{G}} che permette di creare delle componenti a popup;
			\item \texttt{QuestionService: QuestionService}: \\
			Parametro contenente un riferimento al servizio che si occupa della gestione delle informazioni legate alle domande;
			\item \texttt{Upload: Upload} \\
			Parametro contenente un riferimento alla libreria \textit{ng-file-upload\ped{G}} necessaria per il caricamento della foto profilo dell'utente;
			\item \texttt{\$timeout: \$timeout} \\
			Parametro contenente il riferimento all'oggetto globale \$timeout creato da \textit{Angular.js\ped{G}}. 
			Il valore di ritorno di una chiamata alla funzione di \texttt{\$timeout} è una promise, la quale sarà risolta quando avverrà il ritardo e la funzione di timeout eseguita; 
			\item \texttt{\$routeParams: \$routeParams} \\
			Parametro contenente il riferimento all'oggetto globale \$routeParams creato da \textit{Angular.js\ped{G}}. Tale servizio permette di recuperare il set di variabili presenti nell'url; 
		\end{itemize}
		\item \texttt{+} \texttt{submitQuestion(): void}\\ 
		Metodo che gestisce l’evento click sul pulsante di conferma sulla domanda. Raccoglie i dati dal modelview e li manda al server attraverso \texttt{QuestionService}. Poi verrà effettuato il redirect alla pagina di gestione delle domande oppure al questionario che si stava creando; 
	\end{itemize}

\end{itemize}

\paragraph{QuizziPedia::Front-End::Controllers::ConnectionQuestionsController}
\begin{figure} [ht]
	\centering
	\includegraphics[scale=0.45]{UML/Classi/Front-End/QuizziPedia_Front-end_Controller_ConnectionQuestionsController.png}
	\caption{QuizziPedia::Front-End::Controllers::ConnectionQuestionsController}
\end{figure} \FloatBarrier
\begin{itemize}
	\item \textbf{Descrizione}: questa classe permette di gestire la creazione e la modifica di una domanda a collegamento;
	\item \textbf{Utilizzo}: fornisce le funzionalità per inserire una nuova domanda a collegamento nel database e per modificarne una esistente;
	\item \textbf{Relazione con altre classi}:
	\begin{itemize}
		\item \textit{IN} \texttt{ConnectionQuestionsModelView}: classe di tipo modelview la cui istanzazione è contenuta all'interno della variabile di ambiente \$scope di \textit{Angular.js\ped{G}}. All'interno di essa sono presenti le variabili e i metodi necessari per il \textit{Two-Way Data-Binding\ped{G}} tra la view \texttt{ConnectionQuestionsView} e il controller \texttt{ConnectionQuestionsController};
		\item \textit{IN} \texttt{QuestionService}: questa classe permette di:
		\begin{itemize}
			\item Ottenere una domanda attraverso il metodo dedicato;
			\item Caricare una domanda modificata;
			\item Caricare una nuova domanda.
		\end{itemize}
		\item \textit{IN} \texttt{QuestionItemModel}: questa classe rappresenta il modello di una domanda;
	\end{itemize}
	\item \textbf{Attributi}:
	\begin{itemize}
		\item \texttt{-} \texttt{\$scope: \$scope} \\
		Campo dati contenente un riferimento all’oggetto \$scope creato da \textit{Angular\ped{G}}, viene utilizzato come mezzo di comunicazione tra il controller e la view. Contiene gli oggetti che definiscono il model dell’applicazione;
		\item \texttt{-} \texttt{QuestionItemModel: QuestionItemModel} \\
		Campo dati che si riferisce alla classe che rappresenta il modello della classe;
		\item \texttt{-} \texttt{\$mdDialog: \$mdDialog} \\
		Campo dati contenente un riferimento al servizio della libreria \textit{Material for Angular\ped{G}} che permette di creare delle componenti a popup;
		\item \texttt{-} \texttt{QuestionService: QuestionService}: \\
		Campo dati contenente un riferimento al servizio che si occupa della gestione delle informazioni legate alle domande;
		\item \texttt{-} \texttt{Upload: Upload} \\
		Campo dati contenente un riferimento alla libreria \textit{ng-file-upload\ped{G}} necessaria per il caricamento della foto profilo dell'utente;
		\item \texttt{-} \texttt{\$timeout: \$timeout} \\
		Campo dati contenente il riferimento all'oggetto globale \$timeout creato da \textit{Angular.js\ped{G}}. 
		Il valore di ritorno di una chiamata alla funzione di \texttt{\$timeout} è una promise, la quale sarà risolta quando avverrà il ritardo e la funzione di timeout eseguita; 
		\item \texttt{\$routeParams: \$routeParams} \\
		Campo dati contenente il riferimento all'oggetto globale \$routeParams creato da \textit{Angular.js\ped{G}}. Tale servizio permette di recuperare il set di variabili presenti nell'url; 
	\end{itemize}
	\item \textbf{Metodi}:
	\begin{itemize}
		\item \texttt{+} \texttt{ConnectionQuestionsController(\$scope: \$scope, QuestionItemModel: QuestionItemModel, \$mdDialog: \$mdDialog, QuestionService: QuestionService, Upload: Upload, \$timeout: \$timeout, \$routeParams: \$routeParams)} \\ 
		Metodo costruttore della classe. Se in \texttt{\$routeParams} sarà presente il codice univoco che rappresenta una domanda e di questa il creatore è l'utente autenticato, allora verrà scaricato attraverso il \texttt{QuestionService} il contenuto della domanda così da poterlo modificare. In caso contrario verrà mostrato un errore attraverso \texttt{\$mdDialog} indicando che i privilegi per tale operazione sono negati. Nel caso in cui non ci sarà tale parametro in \texttt{\$routeParams} verrà caricata la view vuota così da poter creare una nuova domanda; \\
		\textbf{Parametri}:
		\begin{itemize}
			\item \texttt{\$scope: \$scope} \\
			Parametro contenente un riferimento all’oggetto \$scope creato da \textit{Angular\ped{G}}, viene utilizzato come mezzo di comunicazione tra il controller e la view. Contiene gli oggetti che definiscono il model dell’applicazione;
			\item \texttt{QuestionItemModel: QuestionItemModel} \\ 
			Parametro che si riferisce alla classe che rappresenta il modello della classe;
			\item \texttt{\$mdDialog: \$mdDialog} \\
			Parametro contenente un riferimento al servizio della libreria \textit{Material for Angular\ped{G}} che permette di creare delle componenti a popup;
			\item \texttt{QuestionService: QuestionService}: \\
			Parametro contenente un riferimento al servizio che si occupa della gestione delle informazioni legate alle domande;
			\item \texttt{Upload: Upload} \\
			Parametro contenente un riferimento alla libreria \textit{ng-file-upload\ped{G}} necessaria per il caricamento della foto profilo dell'utente;
			\item \texttt{\$timeout: \$timeout} \\
			Parametro contenente il riferimento all'oggetto globale \$timeout creato da \textit{Angular.js\ped{G}}. 
			Il valore di ritorno di una chiamata alla funzione di \texttt{\$timeout} è una promise, la quale sarà risolta quando avverrà il ritardo e la funzione di timeout eseguita; 
			\item \texttt{\$routeParams: \$routeParams} \\
			Parametro contenente il riferimento all'oggetto globale \$routeParams creato da \textit{Angular.js\ped{G}}. Tale servizio permette di recuperare il set di variabili presenti nell'url; 
		\end{itemize}
		\item \texttt{+} \texttt{submitQuestion(): void}\\ 
		Metodo che gestisce l’evento click sul pulsante di conferma sulla domanda. Raccoglie i dati dal modelview e li manda al server attraverso \texttt{QuestionService}. Poi verrà effettuato il redirect alla pagina di gestione delle domande oppure al questionario che si stava creando; 
	\end{itemize}
\end{itemize}

\paragraph{QuizziPedia::Front-End::Controllers::ImagesSortingQuestionsController}
\begin{figure} [ht]
	\centering
	\includegraphics[scale=0.45]{UML/Classi/Front-End/QuizziPedia_Front-end_Controller_ImagesSortingQuestionsController.png}
	\caption{QuizziPedia::Front-End::Controllers::ImagesSortingQuestionsController}
\end{figure} \FloatBarrier
\begin{itemize}
	\item \textbf{Descrizione}: questa classe permette di gestire la creazione e la modifica di una domanda a ordinamento immagini;
	\item \textbf{Utilizzo}: fornisce le funzionalità per inserire una nuova domanda a ordinamento immagini nel database e per modificarne una esistente;
	\item \textbf{Relazione con altre classi}:
	\begin{itemize}
		\item \textit{IN} \texttt{ImageSortingQuestionsModelView}: classe di tipo modelview la cui istanzazione è contenuta all'interno della variabile di ambiente \$scope di \textit{Angular.js\ped{G}}. All'interno di essa sono presenti le variabili e i metodi necessari per il \textit{Two-Way Data-Binding\ped{G}} tra la view \texttt{ImagesSortingQuestionsView} e il controller \texttt{ImagesSortingQuestionsController};
		\item \textit{IN} \texttt{QuestionService}: questa classe permette di:
		\begin{itemize}
			\item Ottenere una domanda attraverso il metodo dedicato;
			\item Caricare una domanda modificata;
			\item Caricare una nuova domanda.
		\end{itemize}
		\item \textit{IN} \texttt{QuestionItemModel}: questa classe rappresenta il modello di una domanda;
	\end{itemize}
	\item \textbf{Attributi}:
	\begin{itemize}
		\item \texttt{-} \texttt{\$scope: \$scope} \\
		Campo dati contenente un riferimento all’oggetto \$scope creato da \textit{Angular\ped{G}}, viene utilizzato come mezzo di comunicazione tra il controller e la view. Contiene gli oggetti che definiscono il model dell’applicazione;
		\item \texttt{-} \texttt{QuestionItemModel: QuestionItemModel} \\
		Campo dati che si riferisce alla classe che rappresenta il modello della classe;
		\item \texttt{-} \texttt{\$mdDialog: \$mdDialog} \\
		Campo dati contenente un riferimento al servizio della libreria \textit{Material for Angular\ped{G}} che permette di creare delle componenti a popup;
		\item \texttt{-} \texttt{QuestionService: QuestionService}: \\
		Campo dati contenente un riferimento al servizio che si occupa della gestione delle informazioni legate alle domande;
		\item \texttt{-} \texttt{Upload: Upload} \\
		Campo dati contenente un riferimento alla libreria \textit{ng-file-upload\ped{G}} necessaria per il caricamento della foto profilo dell'utente;
		\item \texttt{-} \texttt{\$timeout: \$timeout} \\
		Campo dati contenente il riferimento all'oggetto globale \$timeout creato da \textit{Angular.js\ped{G}}. 
		Il valore di ritorno di una chiamata alla funzione di \texttt{\$timeout} è una promise, la quale sarà risolta quando avverrà il ritardo e la funzione di timeout eseguita; 
		\item \texttt{\$routeParams: \$routeParams} \\
		Campo dati contenente il riferimento all'oggetto globale \$routeParams creato da \textit{Angular.js\ped{G}}. Tale servizio permette di recuperare il set di variabili presenti nell'url; 
	\end{itemize}
	\item \textbf{Metodi}:
	\begin{itemize}
		\item \texttt{+} \texttt{ImageSortingQuestionsController(\$scope: \$scope, QuestionItemModel: QuestionItemModel, \$mdDialog: \$mdDialog, QuestionService: QuestionService, Upload: Upload, \$timeout: \$timeout, \$routeParams: \$routeParams)} \\ 
		Metodo costruttore della classe. Se in \texttt{\$routeParams} sarà presente il codice univoco che rappresenta una domanda e di questa il creatore è l'utente autenticato, allora verrà scaricato attraverso il \texttt{QuestionService} il contenuto della domanda così da poterlo modificare. In caso contrario verrà mostrato un errore attraverso \texttt{\$mdDialog} indicando che i privilegi per tale operazione sono negati. Nel caso in cui non ci sarà tale parametro in \texttt{\$routeParams} verrà caricata la view vuota così da poter creare una nuova domanda; \\
		\textbf{Parametri}:
		\begin{itemize}
			\item \texttt{\$scope: \$scope} \\
			Parametro contenente un riferimento all’oggetto \$scope creato da \textit{Angular\ped{G}}, viene utilizzato come mezzo di comunicazione tra il controller e la view. Contiene gli oggetti che definiscono il model dell’applicazione;
			\item \texttt{QuestionItemModel: QuestionItemModel} \\ 
			Parametro che si riferisce alla classe che rappresenta il modello della classe;
			\item \texttt{\$mdDialog: \$mdDialog} \\
			Parametro contenente un riferimento al servizio della libreria \textit{Material for Angular\ped{G}} che permette di creare delle componenti a popup;
			\item \texttt{QuestionService: QuestionService}: \\
			Parametro contenente un riferimento al servizio che si occupa della gestione delle informazioni legate alle domande;
			\item \texttt{Upload: Upload} \\
			Parametro contenente un riferimento alla libreria \textit{ng-file-upload\ped{G}} necessaria per il caricamento della foto profilo dell'utente;
			\item \texttt{\$timeout: \$timeout} \\
			Parametro contenente il riferimento all'oggetto globale \$timeout creato da \textit{Angular.js\ped{G}}. 
			Il valore di ritorno di una chiamata alla funzione di \texttt{\$timeout} è una promise, la quale sarà risolta quando avverrà il ritardo e la funzione di timeout eseguita; 
			\item \texttt{\$routeParams: \$routeParams} \\
			Parametro contenente il riferimento all'oggetto globale \$routeParams creato da \textit{Angular.js\ped{G}}. Tale servizio permette di recuperare il set di variabili presenti nell'url; 
		\end{itemize}
		\item \texttt{+} \texttt{submitQuestion(): void}\\ 
		Metodo che gestisce l’evento click sul pulsante di conferma sulla domanda. Raccoglie i dati dal modelview e li manda al server attraverso \texttt{QuestionService}. Poi verrà effettuato il redirect alla pagina di gestione delle domande oppure al questionario che si stava creando; 
	\end{itemize}
\end{itemize}

\paragraph{QuizziPedia::Front-End::Controllers::StringsSortingQuestionsController}
\begin{figure} [ht]
	\centering
	\includegraphics[scale=0.45]{UML/Classi/Front-End/QuizziPedia_Front-end_Controller_StringSortingQuestionsController.png}
	\caption{QuizziPedia::Front-End::Controllers::StringsSortingQuestionsController}
\end{figure} \FloatBarrier
\begin{itemize}
	\item \textbf{Descrizione}: questa classe permette di gestire la creazione e la modifica di una domanda a ordinamento di stringhe;
	\item \textbf{Utilizzo}: fornisce le funzionalità per inserire una nuova domanda a ordinamento di stringhe nel database e per modificarne una esistente;
	\item \textbf{Relazione con altre classi}:
	\begin{itemize}
		\item \textit{IN} \texttt{StringsSortingQuestionsModelView}: classe di tipo modelview la cui istanzazione è contenuta all'interno della variabile di ambiente \$scope di \textit{Angular.js\ped{G}}. All'interno di essa sono presenti le variabili e i metodi necessari per il \textit{Two-Way Data-Binding\ped{G}} tra la view \texttt{StringsSortingQuestionsView} e il controller \texttt{StringsSortingQuestionsController};
		\item \textit{IN} \texttt{QuestionService}: questa classe permette di:
		\begin{itemize}
			\item Ottenere una domanda attraverso il metodo dedicato;
			\item Caricare una domanda modificata;
			\item Caricare una nuova domanda.
		\end{itemize}
		\item \textit{IN} \texttt{QuestionItemModel}: questa classe rappresenta il modello di una domanda;
	\end{itemize}
	\item \textbf{Attributi}:
	\begin{itemize}
		\item \texttt{-} \texttt{\$scope: \$scope} \\
		Campo dati contenente un riferimento all’oggetto \$scope creato da \textit{Angular\ped{G}}, viene utilizzato come mezzo di comunicazione tra il controller e la view. Contiene gli oggetti che definiscono il model dell’applicazione;
		\item \texttt{-} \texttt{QuestionItemModel: QuestionItemModel} \\
		Campo dati che si riferisce alla classe che rappresenta il modello della classe;
		\item \texttt{-} \texttt{\$mdDialog: \$mdDialog} \\
		Campo dati contenente un riferimento al servizio della libreria \textit{Material for Angular\ped{G}} che permette di creare delle componenti a popup;
		\item \texttt{-} \texttt{QuestionService: QuestionService}: \\
		Campo dati contenente un riferimento al servizio che si occupa della gestione delle informazioni legate alle domande;
		\item \texttt{\$routeParams: \$routeParams} \\
		Campo dati contenente il riferimento all'oggetto globale \$routeParams creato da \textit{Angular.js\ped{G}}. Tale servizio permette di recuperare il set di variabili presenti nell'url; 
	\end{itemize}
	\item \textbf{Metodi}:
	\begin{itemize}
		\item \texttt{+} \texttt{StringsSortingQuestionsController(\$scope: \$scope, QuestionItemModel: QuestionItemModel, \$mdDialog: \$mdDialog, QuestionService: QuestionService, \$routeParams: \$routeParams)} \\ 
		Metodo costruttore della classe. Se in \texttt{\$routeParams} sarà presente il codice univoco che rappresenta una domanda e di questa il creatore è l'utente autenticato, allora verrà scaricato attraverso il \texttt{QuestionService} il contenuto della domanda così da poterlo modificare. In caso contrario verrà mostrato un errore attraverso \texttt{\$mdDialog} indicando che i privilegi per tale operazione sono negati. Nel caso in cui non ci sarà tale parametro in \texttt{\$routeParams} verrà caricata la view vuota così da poter creare una nuova domanda; \\
		\textbf{Parametri}:
		\begin{itemize}
			\item \texttt{\$scope: \$scope} \\
			Parametro contenente un riferimento all’oggetto \$scope creato da \textit{Angular\ped{G}}, viene utilizzato come mezzo di comunicazione tra il controller e la view. Contiene gli oggetti che definiscono il model dell’applicazione;
			\item \texttt{QuestionItemModel: QuestionItemModel} \\ 
			Parametro che si riferisce alla classe che rappresenta il modello della classe;
			\item \texttt{\$mdDialog: \$mdDialog} \\
			Parametro contenente un riferimento al servizio della libreria \textit{Material for Angular\ped{G}} che permette di creare delle componenti a popup;
			\item \texttt{QuestionService: QuestionService}: \\
			Parametro contenente un riferimento al servizio che si occupa della gestione delle informazioni legate alle domande;
			\item \texttt{\$routeParams: \$routeParams} \\
			Parametro contenente il riferimento all'oggetto globale \$routeParams creato da \textit{Angular.js\ped{G}}. Tale servizio permette di recuperare il set di variabili presenti nell'url; 
		\end{itemize}
		\item \texttt{+} \texttt{submitQuestion(): void}\\ 
		Metodo che gestisce l’evento click sul pulsante di conferma sulla domanda. Raccoglie i dati dal modelview e li manda al server attraverso \texttt{QuestionService}. Poi verrà effettuato il redirect alla pagina di gestione delle domande oppure al questionario che si stava creando; 
	\end{itemize}
\end{itemize}

\paragraph{QuizziPedia::Front-End::Controllers::FillingQuestionsController}
\begin{figure} [ht]
	\centering
	\includegraphics[scale=0.45]{UML/Classi/Front-End/QuizziPedia_Front-end_Controller_FillingQuestionsController.png}
	\caption{QuizziPedia::Front-End::Controllers::FillingQuestionsController}
\end{figure} \FloatBarrier
\begin{itemize}
	\item \textbf{Descrizione}: questa classe permette di gestire la creazione e la modifica di una domanda a riempimento di spazi;
	\item \textbf{Utilizzo}: fornisce le funzionalità per inserire una nuova domanda ariempimento di spazi nel database e per modificarne una esistente;
	\item \textbf{Relazione con altre classi}:
	\begin{itemize}
		\item \textit{IN} \texttt{FillingQuestionsModelView}: classe di tipo modelview la cui istanzazione è contenuta all'interno della variabile di ambiente \$scope di \textit{Angular.js\ped{G}}. All'interno di essa sono presenti le variabili e i metodi necessari per il \textit{Two-Way Data-Binding\ped{G}} tra la view \texttt{FillingQuestionsView} e il controller \texttt{FillingQuestionsController};
		\item \textit{IN} \texttt{QuestionService}: questa classe permette di:
		\begin{itemize}
			\item Ottenere una domanda attraverso il metodo dedicato;
			\item Caricare una domanda modificata;
			\item Caricare una nuova domanda.
		\end{itemize}
		\item \textit{IN} \texttt{QuestionItemModel}: questa classe rappresenta il modello di una domanda;
	\end{itemize}
	\item \textbf{Attributi}:
	\begin{itemize}
		\item \texttt{-} \texttt{\$scope: \$scope} \\
		Campo dati contenente un riferimento all’oggetto \$scope creato da \textit{Angular\ped{G}}, viene utilizzato come mezzo di comunicazione tra il controller e la view. Contiene gli oggetti che definiscono il model dell’applicazione;
		\item \texttt{-} \texttt{QuestionItemModel: QuestionItemModel} \\
		Campo dati che si riferisce alla classe che rappresenta il modello della classe;
		\item \texttt{-} \texttt{\$mdDialog: \$mdDialog} \\
		Campo dati contenente un riferimento al servizio della libreria \textit{Material for Angular\ped{G}} che permette di creare delle componenti a popup;
		\item \texttt{-} \texttt{QuestionService: QuestionService}: ;
		\item \texttt{\$routeParams: \$routeParams} \\
		Campo dati contenente il riferimento all'oggetto globale \$routeParams creato da \textit{Angular.js\ped{G}}. Tale servizio permette di recuperare il set di variabili presenti nell'url; 
	\end{itemize}
	\item \textbf{Metodi}:
	\begin{itemize}
		\item \texttt{+} \texttt{FillingQuestionsController(\$scope: \$scope, QuestionItemModel: QuestionItemModel, \$mdDialog: \$mdDialog, QuestionService: QuestionService)} \\ 
		Metodo costruttore della classe. Se in \texttt{\$routeParams} sarà presente il codice univoco che rappresenta una domanda e di questa il creatore è l'utente autenticato, allora verrà scaricato attraverso il \texttt{QuestionService} il contenuto della domanda così da poterlo modificare. In caso contrario verrà mostrato un errore attraverso \texttt{\$mdDialog} indicando che i privilegi per tale operazione sono negati. Nel caso in cui non ci sarà tale parametro in \texttt{\$routeParams} verrà caricata la view vuota così da poter creare una nuova domanda; \\
		\textbf{Parametri}:
		\begin{itemize}
			\item \texttt{\$scope: \$scope} \\
			Parametro contenente un riferimento all’oggetto \$scope creato da \textit{Angular\ped{G}}, viene utilizzato come mezzo di comunicazione tra il controller e la view. Contiene gli oggetti che definiscono il model dell’applicazione;
			\item \texttt{QuestionItemModel: QuestionItemModel} \\ 
			Parametro che si riferisce alla classe che rappresenta il modello della classe;
			\item \texttt{\$mdDialog: \$mdDialog} \\
			Parametro contenente un riferimento al servizio della libreria \textit{Material for Angular\ped{G}} che permette di creare delle componenti a popup;
			\item \texttt{QuestionService: QuestionService}: \\
			Parametro contenente un riferimento al servizio che si occupa della gestione delle informazioni legate alle domande;
			\item \texttt{Upload: Upload} \\
			Parametro contenente un riferimento alla libreria \textit{ng-file-upload\ped{G}} necessaria per il caricamento della foto profilo dell'utente;
			\item \texttt{\$timeout: \$timeout} \\
			Parametro contenente il riferimento all'oggetto globale \$timeout creato da \textit{Angular.js\ped{G}}. 
			Il valore di ritorno di una chiamata alla funzione di \texttt{\$timeout} è una promise, la quale sarà risolta quando avverrà il ritardo e la funzione di timeout eseguita; 
			\item \texttt{\$routeParams: \$routeParams} \\
			Parametro contenente il riferimento all'oggetto globale \$routeParams creato da \textit{Angular.js\ped{G}}. Tale servizio permette di recuperare il set di variabili presenti nell'url; 
		\end{itemize}
		\item \texttt{+} \texttt{submitQuestion(): void}\\ 
		Metodo che gestisce l’evento click sul pulsante di conferma sulla domanda. Raccoglie i dati dal modelview e li manda al server attraverso \texttt{QuestionService}. Poi verrà effettuato il redirect alla pagina di gestione delle domande oppure al questionario che si stava creando; 
		\item \texttt{+} \texttt{choseThatWord(word:String, number: Integer): void}\\
		Metodo che gestisce l’evento click su una parola del testo. Una volta selezionata essa verrà inserita nell'array che conterrà le parole che dovranno essere nascoste quando l'esercizio sarà proposto; \\
		\textbf{Parametri}:
		\begin{itemize}
			\item \texttt{word:String} \\
			Parametro contenente la parola scelta da nascondere;
			\item \texttt{number: Integer} \\ 
			Parametro che si riferisce al numero della parola scelta da nascondere;
		\end{itemize}
	\end{itemize}
\end{itemize}

\paragraph{QuizziPedia::Front-End::Controllers::ClickableAreaQuestionsController}
\begin{figure} [ht]
	\centering
	\includegraphics[scale=0.45]{UML/Classi/Front-End/QuizziPedia_Front-end_Controller_ClickableAreaQuestionsController.png}
	\caption{QuizziPedia::Front-End::Controllers::ClickableAreaQuestionsController}
\end{figure} \FloatBarrier
\begin{itemize}
	\item \textbf{Descrizione}: questa classe permette di gestire la creazione e la modifica di una domanda ad area cliccabile;
	\item \textbf{Utilizzo}: fornisce le funzionalità per inserire una nuova domanda ad area cliccabile nel database e per modificarne una esistente;
	\begin{itemize}
		\item \textit{IN} \texttt{ClickableAreaQuestionsModelView}: classe di tipo modelview la cui istanzazione è contenuta all'interno della variabile di ambiente \$scope di \textit{Angular.js\ped{G}}. All'interno di essa sono presenti le variabili e i metodi necessari per il \textit{Two-Way Data-Binding\ped{G}} tra la view \texttt{ClickableAreaQuestionsView} e il controller \texttt{ClickableAreaQuestionsController};
		\item \textit{IN} \texttt{QuestionService}: questa classe permette di:
		\begin{itemize}
			\item Ottenere una domanda attraverso il metodo dedicato;
			\item Caricare una domanda modificata;
			\item Caricare una nuova domanda.
		\end{itemize}
		\item \textit{IN} \texttt{QuestionItemModel}: questa classe rappresenta il modello di una domanda;
	\end{itemize}
	\item \textbf{Attributi}:
	\begin{itemize}
		\item \texttt{-} \texttt{\$scope: \$scope} \\
		Campo dati contenente un riferimento all’oggetto \$scope creato da \textit{Angular\ped{G}}, viene utilizzato come mezzo di comunicazione tra il controller e la view. Contiene gli oggetti che definiscono il model dell’applicazione;
		\item \texttt{-} \texttt{QuestionItemModel: QuestionItemModel} \\
		Campo dati che si riferisce alla classe che rappresenta il modello della classe;
		\item \texttt{-} \texttt{\$mdDialog: \$mdDialog} \\
		Campo dati contenente un riferimento al servizio della libreria \textit{Material for Angular\ped{G}} che permette di creare delle componenti a popup;
		\item \texttt{-} \texttt{QuestionService: QuestionService}: \\
		Campo dati contenente un riferimento al servizio che si occupa della gestione delle informazioni legate alle domande;
		\item \texttt{-} \texttt{Upload: Upload} \\
		Campo dati contenente un riferimento alla libreria \textit{ng-file-upload\ped{G}} necessaria per il caricamento della foto profilo dell'utente;
		\item \texttt{-} \texttt{\$timeout: \$timeout} \\
		Campo dati contenente il riferimento all'oggetto globale \$timeout creato da \textit{Angular.js\ped{G}}. 
		Il valore di ritorno di una chiamata alla funzione di \texttt{\$timeout} è una promise, la quale sarà risolta quando avverrà il ritardo e la funzione di timeout eseguita; 
		\item \texttt{\$routeParams: \$routeParams} \\
		Campo dati contenente il riferimento all'oggetto globale \$routeParams creato da \textit{Angular.js\ped{G}}. Tale servizio permette di recuperare il set di variabili presenti nell'url; 
	\end{itemize}
	\item \textbf{Metodi}:
	\begin{itemize}
		\item \texttt{+} \texttt{ClickableAreaQuestionsController(\$scope: \$scope, QuestionItemModel: QuestionItemModel, \$mdDialog: \$mdDialog, QuestionService: QuestionService, Upload: Upload, \$timeout: \$timeout, \$routeParams: \$routeParams)} \\ 
		Metodo costruttore della classe. Se in \texttt{\$routeParams} sarà presente il codice univoco che rappresenta una domanda e di questa il creatore è l'utente autenticato, allora verrà scaricato attraverso il \texttt{QuestionService} il contenuto della domanda così da poterlo modificare. In caso contrario verrà mostrato un errore attraverso \texttt{\$mdDialog} indicando che i privilegi per tale operazione sono negati. Nel caso in cui non ci sarà tale parametro in \texttt{\$routeParams} verrà caricata la view vuota così da poter creare una nuova domanda; \\
		\textbf{Parametri}:
		\begin{itemize}
			\item \texttt{\$scope: \$scope} \\
			Parametro contenente un riferimento all’oggetto \$scope creato da \textit{Angular\ped{G}}, viene utilizzato come mezzo di comunicazione tra il controller e la view. Contiene gli oggetti che definiscono il model dell’applicazione;
			\item \texttt{QuestionItemModel: QuestionItemModel} \\ 
			Parametro che si riferisce alla classe che rappresenta il modello della classe;
			\item \texttt{\$mdDialog: \$mdDialog} \\
			Parametro contenente un riferimento al servizio della libreria \textit{Material for Angular\ped{G}} che permette di creare delle componenti a popup;
			\item \texttt{QuestionService: QuestionService}: \\
			Parametro contenente un riferimento al servizio che si occupa della gestione delle informazioni legate alle domande;
			\item \texttt{Upload: Upload} \\
			Parametro contenente un riferimento alla libreria \textit{ng-file-upload\ped{G}} necessaria per il caricamento della foto profilo dell'utente;
			\item \texttt{\$timeout: \$timeout} \\
			Parametro contenente il riferimento all'oggetto globale \$timeout creato da \textit{Angular.js\ped{G}}. 
			Il valore di ritorno di una chiamata alla funzione di \texttt{\$timeout} è una promise, la quale sarà risolta quando avverrà il ritardo e la funzione di timeout eseguita; 
			\item \texttt{\$routeParams: \$routeParams} \\
			Parametro contenente il riferimento all'oggetto globale \$routeParams creato da \textit{Angular.js\ped{G}}. Tale servizio permette di recuperare il set di variabili presenti nell'url; 
		\end{itemize}
		\item \texttt{+} \texttt{submitQuestion(): void}\\ 
		Metodo che gestisce l’evento click sul pulsante di conferma sulla domanda. Raccoglie i dati dal modelview e li manda al server attraverso \texttt{QuestionService}. Poi verrà effettuato il redirect alla pagina di gestione delle domande oppure al questionario che si stava creando; 
		\item \texttt{+} \texttt{choseThatPoint(x:Integer, y: Integer): void}\\
		Metodo che gestisce l’evento click su un punto dell'immagine. Una volta selezionato esso verrà inserito nell'array di punti; \\
		\textbf{Parametri}:
		\begin{itemize}
			\item \texttt{x: Integer} \\
			Parametro contenente la coordinata x del punto;
			\item \texttt{y: Integer} \\ 
			Parametro contenente la coordinata y del punto;
		\end{itemize}
	\end{itemize}
\end{itemize}

\paragraph{QuizziPedia::Front-End::Controllers::EditorQMLController}
\begin{figure} [ht]
	\centering
	\includegraphics[scale=0.45]{UML/Classi/Front-End/QuizziPedia_Front-end_Controller_EditorQMLController.png}
	\caption{QuizziPedia::Front-End::Controllers::EditorQMLController}
\end{figure} \FloatBarrier
\begin{itemize}
	\item \textbf{Descrizione}: questa classe permette di gestire la creazione e la modifica di domande create tramite editor QML;
	\item \textbf{Utilizzo}: fornisce le funzionalità per creare e modificare una domanda tramite editor QML;
	\item \textbf{Relazione con altre classi}:
	\begin{itemize}
		\item \textit{IN} \texttt{EditorQMLModelView}: classe di tipo modelview la cui istanzazione è contenuta all'interno della variabile di ambiente \$scope di \textit{Angular.js\ped{G}}. All'interno di essa sono presenti le variabili e i metodi necessari per il \textit{Two-Way Data-Binding\ped{G}} tra la view \texttt{EditorQMLView} e il controller \texttt{EditorQMLController};
		\item \textit{IN} \texttt{QuestionService}: questa classe permette di:
		\begin{itemize}
			\item Ottenere una domanda attraverso il metodo dedicato;
			\item Caricare una domanda modificata;
			\item Caricare una nuova domanda.
		\end{itemize}
		\item \textit{IN} \texttt{ParserQML}: questa classe rappresenta il parser QML. Essa fa diventale un oggetto di tipo \texttt{QuestionItemModel} in linguaggio QML e viceversa;
		\item \textit{IN} \texttt{QuestionItemModel}: questa classe rappresenta il modello di una domanda;
	\end{itemize}
	\item \textbf{Attributi}:
	\begin{itemize}
		\item \texttt{-} \texttt{\$scope: \$scope} \\
		Campo dati contenente un riferimento all’oggetto \$scope creato da \textit{Angular\ped{G}}, viene utilizzato come mezzo di comunicazione tra il controller e la view. Contiene gli oggetti che definiscono il model dell’applicazione;
		\item \texttt{-} \texttt{QuestionItemModel: QuestionItemModel} \\
		Campo dati che si riferisce alla classe che rappresenta il modello della classe;
		\item \texttt{-} \texttt{\$mdDialog: \$mdDialog} \\
		Campo dati contenente un riferimento al servizio della libreria \textit{Material for Angular\ped{G}} che permette di creare delle componenti a popup;
		\item \texttt{-} \texttt{QuestionService: QuestionService}: \\
		Campo dati contenente un riferimento al servizio che si occupa della gestione delle informazioni legate alle domande; 
		\item \texttt{-} \texttt{\$routeParams: \$routeParams} \\
		Campo dati contenente il riferimento all'oggetto globale \$routeParams creato da \textit{Angular.js\ped{G}}. Tale servizio permette di recuperare il set di variabili presenti nell'url; 
		\item \texttt{-} \texttt{ParserQML: ParserQML} \\
		Campo dati che si riferisce alla classe che rappresenta il parser QML. Essa fa diventale un oggetto di tipo \texttt{QuestionItemModel} in linguaggio QML e viceversa;
	\end{itemize}
	\item \textbf{Metodi}:
	\begin{itemize}
		\item \texttt{+} \texttt{EditorQMLController(\$scope: \$scope, QuestionItemModel: QuestionItemModel, \$mdDialog: \$mdDialog, QuestionService: QuestionService, \$routeParams: \$routeParams, ParserQML: ParserQML)} \\ 
		Metodo costruttore della classe. Se in \texttt{\$routeParams} sarà presente il codice univoco che rappresenta una domanda e di questa il creatore è l'utente autenticato, allora verrà scaricato attraverso il \texttt{QuestionService} il contenuto della domanda così da poterlo modificare in formato QML. In caso contrario verrà mostrato un errore attraverso \texttt{\$mdDialog} indicando che i privilegi per tale operazione sono negati. Nel caso in cui non ci sarà tale parametro in \texttt{\$routeParams} verrà caricata la view vuota così da poter creare una nuova domanda; \\
		\textbf{Parametri}:
		\begin{itemize}
			\item \texttt{\$scope: \$scope} \\
			Parametro contenente un riferimento all’oggetto \$scope creato da \textit{Angular\ped{G}}, viene utilizzato come mezzo di comunicazione tra il controller e la view. Contiene gli oggetti che definiscono il model dell’applicazione;
			\item \texttt{QuestionItemModel: QuestionItemModel} \\ 
			Parametro che si riferisce alla classe che rappresenta il modello della classe;
			\item \texttt{\$mdDialog: \$mdDialog} \\
			Parametro contenente un riferimento al servizio della libreria \textit{Material for Angular\ped{G}} che permette di creare delle componenti a popup;
			\item \texttt{QuestionService: QuestionService}: \\
			Parametro contenente un riferimento al servizio che si occupa della gestione delle informazioni legate alle domande;
			\item \texttt{\$routeParams: \$routeParams} \\
			Parametro contenente il riferimento all'oggetto globale \$routeParams creato da \textit{Angular.js\ped{G}}. Tale servizio permette di recuperare il set di variabili presenti nell'url; 
			\item \texttt{ParserQML: ParserQML} \\
			Parametro contenente un riferimento alla classe che rappresenta il parser QML. Essa fa diventale un oggetto di tipo \texttt{QuestionItemModel} in linguaggio QML e viceversa;
		\end{itemize}
		\item \texttt{+} \texttt{submitQuestion(): void}\\ 
		Metodo che gestisce l’evento click sul pulsante di conferma sulla domanda. Raccoglie i dati dal modelview, li converte attraverso il parser QML, e li manda al server attraverso \texttt{QuestionService}. Poi verrà effettuato il redirect alla pagina di gestione delle domande oppure al questionario che si stava creando; 
	\end{itemize}
\end{itemize}


\paragraph{QuizziPedia::Front-End::Controllers::QuestionsManagementController}
\begin{figure} [ht]
	\centering
	\includegraphics[scale=0.45]{UML/Classi/Front-End/QuizziPedia_Front-end_Controller_QuestionsManagementController.png}
	\caption{QuizziPedia::Front-End::Controllers::QuestionsManagementController}
\end{figure} \FloatBarrier
\begin{itemize}
	\item \textbf{Descrizione}: questa classe permette di gestire le domande create dall'utente e di crearne di nuove;
	\item \textbf{Utilizzo}: fornisce le funzionalità per richiedere al server le domande create dall'utente e mostrarle nella pagina dedicata. Inoltre permette di catturare gli eventi per modificare le domande esistenti e per crearne una di nuova; 
	\item \textbf{Relazione con altre classi}:
	\begin{itemize}
		\item \textit{IN} \texttt{QuestionsManagementModelView}: classe di tipo modelview la cui istanzazione è contenuta all'interno della variabile di ambiente \$scope di \textit{Angular.js\ped{G}}. All'interno di essa sono presenti le variabili e i metodi necessari per il \textit{Two-Way Data-Binding\ped{G}} tra la view \texttt{QuestionsManagementView} e il controller \texttt{QuestionsManagementController}; 
		\item \textit{IN} \texttt{QuestionService}: questa classe permette di ottenere domande esistenti e salvare nuove domande;
	\end{itemize}
	\item \textbf{Attributi}:
	\begin{itemize}
		\item \texttt{-} \texttt{\$scope: \$scope} \\
		Campo dati contenente un riferimento all’oggetto \$scope creato da \textit{Angular\ped{G}}, viene utilizzato come mezzo di comunicazione tra il controller e la view. Contiene gli oggetti che definiscono il model dell’applicazione;
		\item \texttt{-} \texttt{\$location: \$location} \\
		Campo dati contenente un riferimento al servizio creato da \textit{Angular\ped{G}} che permette di accedere alla barra degli indirizzi del \textit{browser\ped{G}}, i cambiamenti all’URL nella barra degli indirizzi si riflettono in questo oggetto e viceversa;
		\item \texttt{-} \texttt{QuestionService}\\
		Campo dati contenente un riferimento al servizio che si occupa della gestione delle informazioni legate alle domande;
	\end{itemize}
	\item \textbf{Metodi}:
	\begin{itemize}
		\item \texttt{+} \texttt{QuestionsManagementsController(\$scope: \$scope, \$location: \$location, QuestionService: QuestionService)} \\ 
		Metodo costruttore della classe; \\
		\textbf{Parametri}:
		\begin{itemize}
			\item \texttt{\$scope: \$scope} \\
			Parametro contenente un riferimento all’oggetto \$scope creato da \textit{Angular\ped{G}}. Viene utilizzato come mezzo di comunicazione tra il controller e la view. Contiene gli oggetti che definiscono il viewmodel e il model dell’applicazione;
			\item \texttt{\$location: \$location} \\
			Parametro contenente un riferimento al servizio creato da \textit{Angular\ped{G}} che permette di accedere alla barra degli indirizzi del \textit{browser\ped{G}}, i cambiamenti all’URL nella barra degli indirizzi si riflettono in questo oggetto e viceversa;
			\item \texttt{QuestionService: QuestionService} \\
			Parametro contenente un riferimento al servizio che si occupa della gestione delle informazioni legate alle domande;
		\end{itemize}
		\item \texttt{-} \texttt{getQuestionsByUser(username: String)} \\ 
		Metodo che acquisisce le domande create dall'utente attraverso il \texttt{QuestionService};\\
		\textbf{Parametri}:
		\begin{itemize}
			\item \texttt{username: String} \\
			Parametro di tipo \texttt{String} contenente l'username dell'utente;
		\end{itemize}
		\item \texttt{+} \texttt{editQuestion(idQuestion: String)} \\ 
		Metodo che gestisce l’evento click sul pulsante per modificare la domanda. Effettua il redirect alla pagina di modifica della domanda. \\
		\textbf{Parametri}:
		\begin{itemize}
			\item \texttt{idQuestion: username} \\
			Parametro di tipo \texttt{String} contenente l'id della domanda da modificare;
		\end{itemize}
	\end{itemize}
\end{itemize}

\paragraph{QuizziPedia::Front-End::Controllers::NewQuestionsButtonController}
\begin{figure} [ht]
	\centering
	\includegraphics[scale=0.45]{UML/Classi/Front-End/QuizziPedia_Front-end_Controller_NewQuestionsButtonController.png}
	\caption{QuizziPedia::Front-End::Controllers::NewQuestionButtonController}
\end{figure} \FloatBarrier
\begin{itemize}
	\item \textbf{Descrizione}: questa classe permette di effettuare il redirect alla pagina di creazione nuova domanda;
	\item \textbf{Utilizzo}: effettua il redirect alla pagina di creazione di una nuova domanda quando l'utente seleziona interagisce con il bottone a cui è collegato il corrispettivo evento;
	\item \textbf{Relazione con altre classi}:
	\begin{itemize}
		\item \textit{IN} \texttt{NewQuestionButtonsModelView}: classe di tipo modelview la cui istanzazione è contenuta all'interno della variabile di ambiente \$scope di \textit{Angular.js\ped{G}}. All'interno di essa sono presenti le variabili e i metodi necessari per il \textit{Two-Way Data-Binding\ped{G}} tra la directive \texttt{NewQuestionButtonsDirective} e il controller \texttt{NewQuestionsButtonController}; 
	\end{itemize}
	\item \textbf{Attributi}:
	\begin{itemize}
		\item \texttt{-} \texttt{\$scope: \$scope} \\
		Campo dati contenente un riferimento all’oggetto \$scope creato da \textit{Angular\ped{G}}, viene utilizzato come mezzo di comunicazione tra il controller e la view. Contiene gli oggetti che definiscono il model dell’applicazione;
		\item \texttt{-} \texttt{\$location: \$location} \\
		Campo dati contenente un riferimento al servizio creato da \textit{Angular\ped{G}} che permette di accedere alla barra degli indirizzi del \textit{browser\ped{G}}, i cambiamenti all’URL nella barra degli indirizzi si riflettono in questo oggetto e viceversa;
	\end{itemize}
	\item \textbf{Metodi}:
	\begin{itemize}
		\item \texttt{+} \texttt{NewQuestionButtonsController(\$scope: \$scope, \$location: \$location)} \\ 
		Metodo costruttore della classe. \\
		\textbf{Parametri}:
		\begin{itemize}
			\item \texttt{\$scope: \$scope} \\
			Parametro contenente un riferimento all’oggetto \$scope creato da \textit{Angular\ped{G}}. Viene utilizzato come mezzo di comunicazione tra il controller e la view. Contiene gli oggetti che definiscono il viewmodel e il model dell’applicazione;
			\item \texttt{\$location: \$location} \\
			Parametro contenente un riferimento al servizio creato da \textit{Angular\ped{G}} che permette di accedere alla barra degli indirizzi del \textit{browser\ped{G}}, i cambiamenti all’URL nella barra degli indirizzi si riflettono in questo oggetto e viceversa;
		\end{itemize}
		\item \texttt{+} \texttt{newQuestion()} \\ 
		Metodo che gestisce l’evento click sul pulsante per creare una nuova domanda. Effettua il redirect alla pagina di creazione di una domanda;
	\end{itemize}
	
\end{itemize}

\paragraph{QuizziPedia::Front-End::Controllers::StatisticsController}
\begin{figure} [ht]
	\centering
	\includegraphics[scale=0.45]{UML/Classi/Front-End/QuizziPedia_Front-end_Controller_StatisticsController.png}
	\caption{QuizziPedia::Front-End::Controllers::StatisticsController}
\end{figure} \FloatBarrier
\begin{itemize}
	\item \textbf{Descrizione}: questa classe permette di le statistiche di un utente;
	\item \textbf{Utilizzo}: fornisce le funzionalità per ottenere le statistiche di un utente per poterle mostrare nella view;
	\item \textbf{Relazione con altre classi}:
	\begin{itemize}
		\item \textit{IN} \texttt{StatisticsModelView}: classe di tipo modelview la cui istanzazione è contenuta all'interno della variabile di ambiente \$scope di \textit{Angular.js\ped{G}}. All'interno di essa sono presenti le variabili e i metodi necessari per il \textit{Two-Way Data-Binding\ped{G}} tra la directive \texttt{StatisticsDirective} e il controller \texttt{StatisticsController}; 
		\item \textit{IN} \texttt{StatisticsService}: questa classe permette di ottenere le statistiche dell'utente;
	\end{itemize}
	\item \textbf{Attributi}:
	\begin{itemize}
		\item \texttt{-} \texttt{\$scope: \$scope} \\
		Campo dati contenente un riferimento all’oggetto \$scope creato da \textit{Angular\ped{G}}, viene utilizzato come mezzo di comunicazione tra il controller e la view. Contiene gli oggetti che definiscono il model dell’applicazione;
		\item \texttt{-} \texttt{StatisticsService: StatisticsService} \\
		Campo dati contenente un riferimento al servizio che si occupa della gestione delle informazioni legate alle statistiche da visualizzare;
	\end{itemize}	
	\begin{itemize}
		\item \textbf{Metodi}:
		\item \texttt{+} \texttt{StatisticsController(\$scope: \$scope, StatisticsService: StatisticsService)} \\ 
		Metodo costruttore della classe. \\
		\begin{itemize}
			\item \texttt{\$scope: \$scope} \\
			Parametro contenente un riferimento all’oggetto \$scope creato da \textit{Angular\ped{G}}. Viene utilizzato come mezzo di comunicazione tra il controller e la view. Contiene gli oggetti che definiscono il viewmodel e il model dell’applicazione;
			\item \texttt{StatisticsService: StatisticsService} \\
			Parametro contenente un riferimento al servizio che si occupa della gestione delle informazioni legate alle statistiche da visualizzare;
		\end{itemize}
		\item \texttt{-} \texttt{getStatistics(username: String)} \\ 
		Metodo che permette di ottenere le statistiche si un utente grazie all'utilizzo di \texttt{StatisticsService}; \\
		\textbf{Parametri}: 
		Metodo costruttore della classe. \\
		\begin{itemize}
			\item \texttt{username: String} \\
			Parametro contenente la stringa username utilizzata per poter recuperare le giuste statistiche attraverso lo \texttt{StatisticsService}; 
		\end{itemize}
	\end{itemize}
\end{itemize}

\paragraph{QuizziPedia::Front-End::Controllers::MenuBarController}
\begin{figure} [ht]
	\centering
	\includegraphics[scale=0.45]{UML/Classi/Front-End/QuizziPedia_Front-end_Controller_MenuBarController.png}
	\caption{QuizziPedia::Front-End::Controllers::MenuBarController}
\end{figure} \FloatBarrier
\begin{itemize}
	\item \textbf{Descrizione}: questa classe permette di gestire il menù fisso per ogni pagina;
	\item \textbf{Utilizzo}: fornisce le funzionalità per aggiornare, a seconda della pagina, il contenuto del menù;
	\item \textbf{Relazione con altre classi}:
	\begin{itemize}
		\item \textit{IN} \texttt{MenuBarModelView}: classe di tipo modelview la cui istanzazione è contenuta all'interno della variabile di ambiente \$scope di \textit{Angular.js\ped{G}}. All'interno di essa sono presenti le variabili e i metodi necessari per il \textit{Two-Way Data-Binding\ped{G}} tra la directive \texttt{MenuBarDirective} e il controller \texttt{MenuBarController}. Rappresenta il menù, presente in ogni pagina dell'applicazione, generato in base agli oggetti passati nello \$scope. Fornisce un pulsante per ogni oggetto ricevuto come parametro, ogni pulsante viene rappresentato con un’icona e con un testo. Al click di un pulsante viene invocata la funzione ad esso associata; 
		\item \textit{IN} \texttt{AuthService}: questa classe permette di gestire la registrazione e l'autenticazione di un utente;
		\item \textit{IN} \texttt{MenuBarModel}: questa classe rappresenta la classe che contiene le informazioni per la giusta visualizzazione della barra;
	\end{itemize}
	\item \textbf{Attributi}:
	\begin{itemize}
		\item \texttt{-} \texttt{\$scope: \$scope} \\
		Campo dati contenente un riferimento all’oggetto \$scope creato da \textit{Angular\ped{G}}, viene utilizzato come mezzo di comunicazione tra il controller e la view. Contiene gli oggetti che definiscono il model dell’applicazione;
		\item \texttt{-} \texttt{\$location: \$location} \\
		Campo dati contenente un riferimento al servizio creato da \textit{Angular\ped{G}} che permette di accedere alla barra degli indirizzi del \textit{browser\ped{G}}, i cambiamenti all’URL nella barra degli indirizzi si riflettono in questo oggetto e viceversa; 
		\item \texttt{-} \texttt{\$mdDialog: \$mdDialog} \\
		Campo dati contenente un riferimento al servizio della libreria \textit{Material for Angular\ped{G}} che permette di creare delle componenti a popup;
		\item \texttt{-} \texttt{AuthService: AuthService} \\
		Campo dati contenente un riferimento al servizio che si occupa della gestione delle informazioni legate all’autenticazione;
		\item \texttt{-} \texttt{MenuBarModel: MenuBarModel}: \\
		Campo dati contenente un riferimento all'oggetto che contiene le informazioni per la giusta visualizzazione della barra;
		
	\end{itemize}
	\item \textbf{Metodi}:
	\begin{itemize}
		\item \texttt{+} \texttt{MenuBarController(\$scope: \$scope, \$location: \$location, \$mdDialog: \$mdDialog, AuthService: AuthService, MenuBarModel: MenuBarModel)} \\
		Metodo costruttore della classe; \\
		\textbf{Parametri}:
		\begin{itemize}
			\item \texttt{\$scope: \$scope} \\
			Parametro contenente un riferimento all’oggetto \$scope creato da \textit{Angular\ped{G}}. Viene utilizzato come mezzo di comunicazione tra il controller e la view. Contiene gli oggetti che definiscono il viewmodel e il model dell’applicazione;
			\item \texttt{\$location: \$location} \\
			Parametro contenente un riferimento al servizio creato da \textit{Angular\ped{G}} che permette di accedere alla barra degli indirizzi del \textit{browser\ped{G}}, i cambiamenti all’URL nella barra degli indirizzi si riflettono in questo oggetto e viceversa;
			\item \texttt{\$mdDialog: \$mdDialog} \\
			Parametro contenente un riferimento al servizio della libreria \textit{Material for Angular\ped{G}} che permette di creare delle componenti a popup;
			\item \texttt{AuthService: AuthService} \\
			Parametro contenente un riferimento al servizio che si occupa della gestione delle informazioni legate all’autenticazione.  Viene utilizzato il metodo \texttt{logOut} di \$texttt{AuthService} a cui viene passato il parametro \texttt{username};
			\item \texttt{MenuBarModel: MenuBarModel}: \\
			Parametro contenente un riferimento all'oggetto che contiene le informazioni per la giusta visualizzazione della barra;
		\end{itemize}
		\item \texttt{+} \texttt{logOut(): void} \\
		Metodo che richiama il metodo \texttt{logOut} del service \texttt{AuthService} passandogli lo \texttt{username}. Prima di effettuare questa operazione viene mostrato a video un messaggio di conferma per il proseguo dell'operazione; 
		\item \texttt{+} \texttt{logIn(): void} \\
		Metodo che gestisce l’evento click sul pulsante per effettuare il login. Effettua il redirect alla pagina per effettuare il login; 
		\item \texttt{+} \texttt{signUp(): void} \\
		Metodo che gestisce l’evento click sul pulsante per effettuare la registrazione. Effettua il redirect alla pagina per effettuare la registrazione; 
		\item \texttt{+} \texttt{goToUserPage(): void} \\
		Metodo che gestisce l’evento click sul pulsante di visualizzazione della pagina utente. Effettua il redirect alla pagina di visualizzazione della pagina utente; 
		\item \texttt{+} \texttt{goToUserManagemetPage(): void} \\
		Metodo che gestisce l’evento click sul pulsante di gestione del profilo utente. Effettua il redirect alla pagina di gestione del profilo utente; 
		\item \texttt{+} \texttt{goToQuestionsManagementPage(): void} \\
		Metodo che gestisce l’evento click sul pulsante di gestione delle domande. Effettua il redirect alla pagina di gestione delle domande; 
		\item \texttt{+} \texttt{goToQuizManagementPage(): void} \\
		Metodo che gestisce l’evento click sul pulsante di gestione dei questionari. Effettua il redirect alla pagina di gestione dei questionari; 
	\end{itemize}
	
\end{itemize}