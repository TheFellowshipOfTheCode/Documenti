\paragraph{QuizziPedia::Front-End::Directives::AnswerChoiceDirective}
\begin{itemize}
	\item \textbf{Descrizione}: directive contenente il componente grafico per le risposte a scelta;
	\item \textbf{Utilizzo}: permette all'utente di utilizzare quest'unica direttiva per creare domande vero/falso e risposte a scelta multipla;
	\item \textbf{Relazioni con altre classi}:
	\begin{itemize}
		\item \textit{IN} \texttt{TrueFalseQuestionsView} \\
		\item \textit{IN} \texttt{MultipleQuestionsView} \\
	\end{itemize}
	\item \textbf{Attributi}
\end{itemize}

\paragraph{QuizziPedia::Front-End::Directives::CreationAndModifyDirective}
\begin{itemize}
	\item \textbf{Descrizione}: componente grafico contenente il bottone per creare un questionario uno per modificare un questionario;
	\item \textbf{Utilizzo}: permette di creare un nuovo questionario o di modificarne uno esistente;
	\item \textbf{Relazioni con altre classi}:
	\begin{itemize}
		\item \textit{IN} \texttt{QuestionnaireManagementView} \\
		\item \textit{IN} \texttt{QuizEventController} \\
	\end{itemize}
	\item \textbf{Attributi}
\end{itemize}

\paragraph{QuizziPedia::Front-End::Directives::ErrorsDirective}
\begin{itemize}
	\item \textbf{Descrizione}: directive che mostra l'eventuale errore dopo un'azione;
	\item \textbf{Utilizzo}: permette all'\texttt{ErrorController} di utilizzare un template unico per la visualizzazione dei diversi errori;
	\item \textbf{Relazioni con altre classi}:
	\begin{itemize}
		\item \textit{IN} \texttt{ErrorsController}
		\item \textit{IN} \texttt{Index}
	\end{itemize}
	\item \textbf{Attributi}
\end{itemize}

\paragraph{QuizziPedia::Front-End::Directives::ExamModalityDirective}
\begin{itemize}
	\item \textbf{Descrizione}: directive contenete i componenti grafici per attivare la modalità esame su un questionario e gestire le iscrizioni;
	\item \textbf{Utilizzo}: permette di attivare la modalità esame su un questionario e di gestirne le iscrizioni;
	\item \textbf{Relazioni con altre classi}:
	\begin{itemize}
		\item \textit{IN} \texttt{QuestionnaireManagementView}: view principale per la gestione dei questionari;
		\item \textit{IN} \texttt{QuizEventController}: questa classe permette di reagire ai comandi dell'utente durante la gestione dei suoi questionari;
	\end{itemize}
	\item \textbf{Attributi}
\end{itemize}

\paragraph{QuizziPedia::Front-End::Directives::FooterDirective}
\begin{itemize}
	\item \textbf{Descrizione}: directive che mostra il footer dell'applicazione che sarà presente in ogni pagina;
	\item \textbf{Utilizzo}: permette all'\texttt{ErrorController} di utilizzare un template unico per la visualizzazione dei diversi errori;
	\item \textbf{Relazioni con altre classi}:
	\begin{itemize}
		\item \textit{IN} \texttt{FooterController}
		\item \textit{IN} \texttt{ErrorsController}
		\item \textit{IN} \texttt{Index}
	\end{itemize}
	\item \textbf{Attributi}
\end{itemize}

\paragraph{QuizziPedia::Front-End::Directives::ImageInTheQuestionDirective}
\begin{itemize}
	\item \textbf{Descrizione}: directive per l'inserimento dell'immagine nella creazione delle domande;
	\item \textbf{Utilizzo}: permette di utilizzare un unico template per inserire un immagine in una domanda;
	\item \textbf{Relazioni con altre classi}:
	\begin{itemize}
		\item \textit{IN} \texttt{TrueFalseQuestionsView} \\
		\item \textit{IN} \texttt{MultipleQuestionsView} \\
		\item \textit{IN} \texttt{ImagesSortingQuestionsView} \\
		\item \textit{IN} \texttt{ClickableAreaQuestionsView} \\
	\end{itemize}
	\item \textbf{Attributi}
\end{itemize}