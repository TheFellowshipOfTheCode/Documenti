	\paragraph{QuizziPedia::Front-End::Directives::ClickableAnswerDirective}
		
		\label{QuizziPedia::Front-End::Directives::ClickableAnswerDirective}
		
		\begin{figure}[ht]
			\centering
			\includegraphics[scale=0.80,keepaspectratio]{UML/Classi/Front-End/QuizziPedia_Front-end_Templates_ClickableAnswerTemplate.png}
			\caption{QuizziPedia::Front-End::Directives::ClickableAnswerTemplate}
		\end{figure} \FloatBarrier
		
		\begin{itemize}
			\item \textbf{Descrizione}: rappresenta il componente grafico che permette all'utente di visualizzare la domanda ad area cliccabile nell'immagine. Viene visualizzato dinamicamente all'interno delle views TrainingView e FillingQuestionnaireView mediante il controller QuestionsController;
			\item \textbf{Utilizzo}: viene utilizzato per consentire all'utente la compilazione della domanda ad area cliccabile nell'immagine;
			\item \textbf{Relazioni con altre classi}: 
			\begin{itemize}
				\item \textit{IN} \texttt{TrainingView}: view principale della modalità allenamento; 
				\item \textit{IN} \texttt{FillingQuestionnaireView}: view principale per la compilazione del questionario;
				\item \textit{IN} \texttt{QuestionsController}: questa classe permette di gestire il recupero delle domande per poterle stampare nella modalità allenamento;
				
			\end{itemize}
			\item \textbf{Attributi}: 
				\begin{itemize}
					\item \texttt{+ controller: String} \\ Stringa contenente il nome del controller della direttiva;
					\item \texttt{+ restrict: String} \\ Stringa che permette di definire le modalità di inserimento della direttiva all'interno della pagina;
					\item \texttt{+ scope: Scope} \\ Oggetto scope interno della direttiva, contiene:
					\begin{itemize}	
						\item \texttt{+ questionText: String} \\ Identifica il testo della domanda;
						\item \texttt{+ image: String} \\ Identifica l'url di una possibile immagine nella domanda;
						\item \texttt{+ answers: Array} \\ Array che contiene coppie di valori. Queste coppie sono formate da:
						\begin{itemize}
							\item \texttt{+ type: String} \\ Indica la tipologia della risposta;
							\item \texttt{+ text: String} \\ Contiene il testo dell'affermazione;
							\item \texttt{+ url: String} \\ Rappresenta l'immagine della risposta;
							\item \texttt{+ attributesForClickableArea: Mixed} \\ Contiene i seguenti attributi:
							\begin{enumerate}
								\item \texttt{+ x: Number} \\ Rappresenta la coordinata x di una area cliccabile;
								\item \texttt{+ y: Number} \\ Rappresenta la coordinata y di una area cliccabile.
							\end{enumerate}
						\end{itemize}
					\end{itemize}
					\item \texttt{+ templateUrl: String} \\ Stringa contenente il percorso del file \textit{HTML\ped{G}} che contiene la direttive.
				
				\end{itemize}
		\end{itemize}
	
		\paragraph{QuizziPedia::Front-End::Directives::EmptySpaceAnswerDirective}
		
		\label{QuizziPedia::Front-End::Directives::EmptySpaceAnswerDirective}
		
		\begin{figure}[ht]
			\centering
			\includegraphics[scale=0.80,keepaspectratio]{UML/Classi/Front-End/QuizziPedia_Front-end_Templates_EmptySpaceAnswerTemplate.png}
			\caption{QuizziPedia::Front-End::Directives::EmptyChoiceAnswerDirective}
		\end{figure} \FloatBarrier
		
		\begin{itemize}
			\item \textbf{Descrizione}: rappresenta il componente grafico che permette all'utente di visualizzare l'esercizio a riempimento di spazi vuoti. Viene visualizzato dinamicamente all'interno delle views TrainingView e FillingQuestionnaireView mediante il controller QuestionsController;
			\item \textbf{Utilizzo}: viene utilizzato per consentire all'utente la compilazione dell'esercizio a riempimento di spazi vuoti;
			\item \textbf{Relazioni con altre classi}: 
			\begin{itemize}
				\item \textit{IN} \texttt{TrainingView}: view principale della modalità allenamento; 
				\item \textit{IN} \texttt{FillingQuestionnaireView}: view principale per la compilazione del questionario;
				\item \textit{IN} \texttt{QuestionsController}: questa classe permette di gestire il recupero delle domande per poterle stampare nella modalità allenamento;
			\end{itemize}
			\item \textbf{Attributi}: 
			\begin{itemize}
				\item \texttt{+ controller: String} \\ Stringa contenente il nome del controller della direttiva;
				\item \texttt{+ restrict: String} \\ Stringa che permette di definire le modalità di inserimento della direttiva all'interno della pagina;
				\item \texttt{+ scope: Scope} \\ Oggetto scope interno della direttiva, contiene:
				\begin{itemize}
					\item \texttt{+ questionText: String} \\ Identifica il testo della domanda;
					\item \texttt{+ image: String} \\ Identifica l'url di una possibile immagine nella domanda;
					\item \texttt{+ answers: Array} \\ Array che contiene coppie di valori. Queste coppie sono formate da:
					\begin{itemize}
						\item \texttt{+ type: String} \\ Indica la tipologia della risposta;
						\item \texttt{+ text: String} \\ Contiene il testo dell'affermazione;
						\item \texttt{+ url: String} \\ Rappresenta l'immagine della risposta;
						\item \texttt{+ attributesForEmptySpaces: Mixed} \\ Contiene i seguenti attributi:
						\begin{enumerate}
							\item \texttt{+ wordNumber: Number} \\ Rappresenta la posizione dello spazio vuoto in cui deve andare inserita la parola.
						\end{enumerate}
					\end{itemize}
				\end{itemize}
				\item \texttt{+ templateUrl: String}: stringa contenente il percorso del file \textit{HTML\ped{G}} che contiene la direttive.
			\end{itemize}
		\end{itemize}
		
		\paragraph{QuizziPedia::Front-End::Directives::HeaderTextQuestionDirective}
		
		\label{QuizziPedia::Front-End::Directives::HeaderTextQuestionDirective}
		
		\begin{figure}[ht]
			\centering
			\includegraphics[scale=0.80,keepaspectratio]{UML/Classi/Front-End/QuizziPedia_Front-end_Templates_HeaderTextQuestionTemplate.png}
			\caption{QuizziPedia::Front-End::Directives::HeaderTextQuestionDirective}
		\end{figure} \FloatBarrier
		
		\begin{itemize}
			\item \textbf{Descrizione}: rappresenta il componente grafico che presenta all'utente l'argomento e le parole chiave. Viene visualizzato dinamicamente all'interno delle views TrainingView e FillingQuestionnaireView mediante il controller QuestionsController;
			\item \textbf{Utilizzo}: viene utilizzato per consentire all'utente la visualizzazione dell'argomento della domanda e le parole chiave associate ad essa;
			\item \textbf{Relazioni con altre classi}: 
			\begin{itemize}
				\item \textit{IN} \texttt{TrainingModelView}: classe di tipo modelview la cui istanziazione è contenuta all'interno della variabile di ambiente \$scope di \textit{Angular.js\ped{G}}. All'interno di essa sono presenti le variabili e i metodi necessari per il \textit{Two-Way Data-Binding\ped{G}} tra la view \texttt{TrainingView} e il controller \texttt{TrainingController}; 
				\item \textit{IN} \texttt{FillingQuestionnaireModelView}: classe di tipo modelview la cui istanziazione è contenuta all'interno della variabile di ambiente \$scope di \textit{Angular.js\ped{G}}. All'interno di essa sono presenti le variabili e i metodi necessari per il \textit{Two-Way Data-Binding\ped{G}} tra la view \texttt{FillingQuestionnaireView} e il controller \texttt{FillingQuestionnaireController};
				\item \textit{IN} \texttt{QuestionsController}: questa classe permette di gestire il recupero delle domande per poterle stampare nella modalità allenamento;
				\item \textit{IN} \texttt{LangModel}: rappresenta il modello delle informazioni per la giusta traduzione dell'applicazione.
			\end{itemize}
			\item \textbf{Attributi}: 
			\begin{itemize}
				\item \texttt{+ controller: String} \\ Stringa contenente il nome del controller della direttiva;
				\item \texttt{+ restrict: String} \\ Stringa che permette di definire le modalità di inserimento della direttiva all'interno della pagina;
				\item \texttt{+ scope: Scope} \\ Oggetto scope interno della direttiva, contiene:
				\begin{itemize}
					\item \textit{+ header: Object} \\ Oggetto contenente i campi dati da visualizzare nella direttiva, ovvero:
					\begin{itemize}
						\item \texttt{topic};
						\item \texttt{keywords}.
					\end{itemize}
				\end{itemize}
				\item \texttt{+ templateUrl: String} \\ Stringa contenente il percorso del file \textit{HTML\ped{G}} che contiene la direttive.
			\end{itemize}
		\end{itemize}
		
		\paragraph{QuizziPedia::Front-End::Directives::InfoQuestionnaireDirective}
		
		\label{QuizziPedia::Front-End::Directives::InfoQuestionnaireDirective}
		
		\begin{figure}[ht]
			\centering
			\includegraphics[scale=0.80,keepaspectratio]{UML/Classi/Front-End/QuizziPedia_Front-end_Templates_InfoQuestionnaireTemplate.png}
			\caption{QuizziPedia::Front-End::Directives::InfoQuestionnaireDirective}
		\end{figure} \FloatBarrier
		
		\begin{itemize}
			\item \textbf{Descrizione}: rappresenta il componente grafico che permette all'utente di visualizzare le informazioni principali del questionario che si sta per svolgere. Viene visualizzato dinamicamente all'interno della view FillingQuestionnaireView mediante il controller FillingQuestionsController;
			\item \textbf{Utilizzo}: viene utilizzato per consentire all'utente di visualizzare le informazioni principali del questionario che si sta per svolgere. Informazioni come:
			\begin{itemize}
				\item Nome del questionario;
				\item Nome dell'autore del questionario;
				\item Data di creazione del questionario;
				\item Argomento del questionario;
				\item Bottone per iniziare il questionario;
			\end{itemize}
			\item \textbf{Relazioni con altre classi}: 
			\begin{itemize}
				\item \textit{IN} \texttt{FillingQuestionnaireModelView}: classe di tipo modelview la cui istanziazione è contenuta all'interno della variabile di ambiente \$scope di \textit{Angular.js\ped{G}}. All'interno di essa sono presenti le variabili e i metodi necessari per il \textit{Two-Way Data-Binding\ped{G}} tra la view \texttt{FillingQuestionnaireView} e il controller \texttt{FillingQuestionnaireController};
				\item \textit{IN} \texttt{FillingQuestionsController}: questa classe permette di gestire la creazione e la modifica di una domanda a riempimento di spazi;
				\item \textit{IN} \texttt{LangModel}: rappresenta il modello delle informazioni per la giusta traduzione dell'applicazione.
			\end{itemize}
			\item \textbf{Attributi}: 
			\begin{itemize}
				\item \texttt{+ controller: String} \\ Stringa contenente il nome del controller della direttiva;
				\item \texttt{+ restrict: String} \\ Stringa che permette di definire le modalità di inserimento della direttiva all'interno della pagina;
				\item \texttt{+ scope: Scope} \\ Oggetto scope interno della direttiva, contiene:
				\begin{itemize}
					\item \texttt{+ info: Object} \\ Oggetto contenente tutte le informazioni sul questionario, ovvero:
					\begin{itemize}
						\item \texttt{name};
						\item \texttt{author};
						\item \texttt{date};
						\item \texttt{topic}.
					\end{itemize}
					\item \texttt{+ startButton: String} \\ Attributo che viene utilizzato per visualizzare la giusta traduzione della \textit{label\ped{G}} per il bottone di inizio del questionario selezionato, in italiano o in inglese; 
				\end{itemize}
				\item \texttt{+ templateUrl: String} \\ Stringa contenente il percorso del file \textit{HTML\ped{G}} che contiene la direttive.
			\end{itemize}
		\end{itemize}
		
		\paragraph{QuizziPedia::Front-End::Directives::LinkingAnswerDirective}
		
		\label{QuizziPedia::Front-End::Directives::LinkingAnswerDirective}
		
		\begin{figure}[ht]
			\centering
			\includegraphics[scale=0.80,keepaspectratio]{UML/Classi/Front-End/QuizziPedia_Front-end_Templates_LinkingAnswerTemplate.png}
			\caption{QuizziPedia::Front-End::Directives::LinkingAnswerDirective}
		\end{figure} \FloatBarrier		
		
		\begin{itemize}
			\item \textbf{Descrizione}: rappresenta il componente grafico che permette all'utente di visualizzare la domanda di collegamento. Viene visualizzato dinamicamente all'interno delle views TrainingView e FillingQuestionnaireView mediante il controller QuestionsController;
			\item \textbf{Utilizzo}: viene utilizzato per consentire all'utente la compilazione della domanda di collegamento;
			\item \textbf{Relazioni con altre classi}: 
			\begin{itemize}
				\item \textit{IN} \texttt{TrainingModelView}: classe di tipo modelview la cui istanziazione è contenuta all'interno della variabile di ambiente \$scope di \textit{Angular.js\ped{G}}. All'interno di essa sono presenti le variabili e i metodi necessari per il \textit{Two-Way Data-Binding\ped{G}} tra la view \texttt{TrainingView} e il controller \texttt{TrainingController}; 
				\item \textit{IN} \texttt{FillingQuestionnaireModelView}: classe di tipo modelview la cui istanziazione è contenuta all'interno della variabile di ambiente \$scope di \textit{Angular.js\ped{G}}. All'interno di essa sono presenti le variabili e i metodi necessari per il \textit{Two-Way Data-Binding\ped{G}} tra la view \texttt{FillingQuestionnaireView} e il controller \texttt{FillingQuestionnaireController};
				\item \textit{IN} \texttt{QuestionsController}: questa classe permette di gestire il recupero delle domande per poterle stampare nella modalità allenamento;
			\end{itemize}
			\item \textbf{Attributi}: 
			\begin{itemize}
				\item \texttt{+ controller: String} \\ Stringa contenente il nome del controller della direttiva;
				\item \texttt{+ restrict: String} \\ Stringa che permette di definire le modalità di inserimento della direttiva all'interno della pagina;
				\item \texttt{+ scope: Scope}: oggetto scope interno della direttiva, contiene:
				\begin{itemize}
					\item \texttt{+ questionText: String} \\ Identifica il testo della domanda;
					\item \texttt{+ image: String} \\ Identifica l'url di una possibile immagine nella domanda;
					\item \texttt{+ answers: Array} \\ Array che contiene coppie di valori. Queste coppie sono formate da:
					\begin{itemize}
						\item \texttt{+ type: String} \\ Indica la tipologia della risposta;
						\item \texttt{+ text: String} \\ Contiene il testo dell'affermazione;
						\item \texttt{+ url: String} \\ Rappresenta l'immagine della risposta;
						\item \texttt{+ attributesForLinking: Mixed}: contiene i seguenti attributi:
						\begin{enumerate}
							\item \texttt{+ text1: String}: rappresenta il primo elemento testuale che deve essere collegato con il secondo elemento testuale o rappresentato da un’immagine;
							\item \texttt{+ text2: String}: rappresenta il secondo elemento testuale che deve essere collegato con il primo elemento testuale o rappresentato da un’immagine;
							\item \texttt{+ url1: String}: rappresenta il primo elemento rappresentato da un'immagine che deve essere collegato con il secondo elemento testuale o rappresentato da un’immagine;
							\item \texttt{+ url2: String}: rappresenta il secondo elemento rappresentato da un'immagine che deve essere collegato con il primo elemento testuale o rappresentato da un’immagine.
						\end{enumerate}
					\end{itemize}
				\end{itemize}
				\item \texttt{+ templateUrl: String}: stringa contenente il percorso del file \textit{HTML\ped{G}} che contiene la direttive.
					
			\end{itemize}
		\end{itemize}
		
		\paragraph{QuizziPedia::Front-End::Directives::MultipleChoiceAnswerDirective}
		
		\label{QuizziPedia::Front-End::Directives::MultipleChoiceAnswerDirective}
		
		\begin{figure}[ht]
			\centering
			\includegraphics[scale=0.80,keepaspectratio]{UML/Classi/Front-End/QuizziPedia_Front-end_Templates_MultipleChoiceAnswerTemplate.png}
			\caption{QuizziPedia::Front-End::Directives::MultipleChoiceAnswerDirective}
		\end{figure} \FloatBarrier
		
		\begin{itemize}
			\item \textbf{Descrizione}: rappresenta il componente grafico che permette all'utente di visualizzare la domanda a risposta multipla. Viene visualizzato dinamicamente all'interno delle views TrainingView e FillingQuestionnaireView mediante il controller QuestionsController;
			\item \textbf{Utilizzo}: viene utilizzato per consentire all'utente la compilazione della domanda a risposta multipla;
			\item \textbf{Relazioni con altre classi}: 
			\begin{itemize}
				\item \textit{IN} \texttt{TrainingModelView}: classe di tipo modelview la cui istanziazione è contenuta all'interno della variabile di ambiente \$scope di \textit{Angular.js\ped{G}}. All'interno di essa sono presenti le variabili e i metodi necessari per il \textit{Two-Way Data-Binding\ped{G}} tra la view \texttt{TrainingView} e il controller \texttt{TrainingController}; 
				\item \textit{IN} \texttt{FillingQuestionnaireModelView}: classe di tipo modelview la cui istanziazione è contenuta all'interno della variabile di ambiente \$scope di \textit{Angular.js\ped{G}}. All'interno di essa sono presenti le variabili e i metodi necessari per il \textit{Two-Way Data-Binding\ped{G}} tra la view \texttt{FillingQuestionnaireView} e il controller \texttt{FillingQuestionnaireController};
				\item \textit{IN} \texttt{QuestionsController}: questa classe permette di gestire il recupero delle domande per poterle stampare nella modalità allenamento;
			\end{itemize}
			\item \textbf{Attributi}: 
			\begin{itemize}
				\item \texttt{+ controller: String} \\ Stringa contenente il nome del controller della direttiva;
				\item \texttt{+ restrict: String} \\ Stringa che permette di definire le modalità di inserimento della direttiva all'interno della pagina;
				\item \texttt{+ scope: Scope} \\ Oggetto scope interno della direttiva, contiene:
				\begin{itemize}
					\item \texttt{questionText: String} \\ Identifica il testo della domanda;
					\item \texttt{image: String} \\ Identifica l'url di una possibile immagine nella domanda;
					\item \texttt{answers: Array}\\ Array che contiene coppie di valori. Queste coppie sono formate da:
					\begin{itemize}
						\item \texttt{type: String} \\ Indica la tipologia della risposta;
						\item \texttt{text: String} \\ Contiene il testo dell'affermazione;
						\item \texttt{url: String} \\ Rappresenta l'immagine della risposta;
						\item \texttt{attributesForTForMultiple: Mixed} \\ Contiene i seguenti attributi:
						\begin{enumerate}
							\item \texttt{isItRight: Boolean} \\ Contiene se la risposta è vera o falsa.
						\end{enumerate}
					\end{itemize}
				\end{itemize}
				\item \texttt{+ templateUrl: String} \\ Stringa contenente il percorso del file \textit{HTML\ped{G}} che contiene la direttive.
				
			\end{itemize}
		\end{itemize}
		
		\paragraph{QuizziPedia::Front-End::Directives::SortImagesAnswerDirective}
		
		\label{QuizziPedia::Front-End::Directives::SortImagesAnswerDirective}
		
		\begin{figure}[ht]
			\centering
			\includegraphics[scale=0.80,keepaspectratio]{UML/Classi/Front-End/QuizziPedia_Front-end_Templates_SortImagesAnswerTemplate.png}
			\caption{QuizziPedia::Front-End::Directives::SortImagesAnswerDirective}
		\end{figure} \FloatBarrier
		
		\begin{itemize}
			\item \textbf{Descrizione}: rappresenta il componente grafico che permette all'utente di visualizzare la domanda ad ordinamento di immagini. Viene visualizzato dinamicamente all'interno delle views TrainingView e FillingQuestionnaireView mediante il controller QuestionsController;
			\item \textbf{Utilizzo}: viene utilizzato per consentire all'utente la compilazione della domanda ad ordinamento di immagini;
			\item \textbf{Relazioni con altre classi}: 
			\begin{itemize}
				\item \textit{IN} \texttt{TrainingModelView}: classe di tipo modelview la cui istanziazione è contenuta all'interno della variabile di ambiente \$scope di \textit{Angular.js\ped{G}}. All'interno di essa sono presenti le variabili e i metodi necessari per il \textit{Two-Way Data-Binding\ped{G}} tra la view \texttt{TrainingView} e il controller \texttt{TrainingController}; 
				\item \textit{IN} \texttt{FillingQuestionnaireModelView}: classe di tipo modelview la cui istanziazione è contenuta all'interno della variabile di ambiente \$scope di \textit{Angular.js\ped{G}}. All'interno di essa sono presenti le variabili e i metodi necessari per il \textit{Two-Way Data-Binding\ped{G}} tra la view \texttt{FillingQuestionnaireView} e il controller \texttt{FillingQuestionnaireController};
				\item \textit{IN} \texttt{QuestionsController}: questa classe permette di gestire il recupero delle domande per poterle stampare nella modalità allenamento;
			\end{itemize}
			\item \textbf{Attributi}: 
			\begin{itemize}
				\item \texttt{+ controller: String} \\ Stringa contenente il nome del controller della direttiva;
				\item \texttt{+ restrict: String} \\ Stringa che permette di definire le modalità di inserimento della direttiva all'interno della pagina;
				\item \texttt{+ scope: Scope} \\ Oggetto scope interno della direttiva, contiene:
				\begin{itemize}
					\item \texttt{+ questionText: String} \\ Identifica il testo della domanda;
					\item \texttt{+ image: String} \\ Identifica l'url di una possibile immagine nella domanda;
					\item \texttt{+ answers: Array} \\ Array che contiene coppie di valori. Queste coppie sono formate da:
					\begin{itemize}
						\item \texttt{+ type: String} \\ Indica la tipologia della risposta;
						\item \texttt{+ text: String} \\ Contiene il testo dell'affermazione;
						\item \texttt{+ url: String} \\ Rappresenta l'immagine della risposta;
						\item \texttt{+ attributesForSorting: Mixed} \\ Contiene i seguenti attributi:
						\begin{enumerate}
							\item \texttt{+ position: Boolean} \\ Contiene la giusta posizione del testo o dell'immagine nell'esercizio di ordinamento.
						\end{enumerate}
					\end{itemize}
				\end{itemize}
				\item \texttt{+ templateUrl: String} \\ Stringa contenente il percorso del file \textit{HTML\ped{G}} che contiene la direttive.
			\end{itemize}
		\end{itemize}
		
		\paragraph{QuizziPedia::Front-End::Directives::SortTextAnswerDirective}
		
		\label{QuizziPedia::Front-End::Directives::SortTextAnswerDirective}
		
		\begin{figure}[ht]
			\centering
			\includegraphics[scale=0.80,keepaspectratio]{UML/Classi/Front-End/QuizziPedia_Front-end_Templates_SortTextAnswerTemplate.png}
			\caption{QuizziPedia::Front-End::Directives::SortTextAnswerDirective}
		\end{figure} \FloatBarrier
		
		\begin{itemize}
			\item \textbf{Descrizione}: rappresenta il componente grafico che permette all'utente di visualizzare la domanda ad ordinamento di stringhe. Viene visualizzato dinamicamente all'interno delle views TrainingView e FillingQuestionnaireView mediante il controller QuestionsController;
			\item \textbf{Utilizzo}: viene utilizzato per consentire all'utente la compilazione della domanda ad ordinamento di stringhe;
			\item \textbf{Relazioni con altre classi}: 
			\begin{itemize}
				\item \textit{IN} \texttt{TrainingModelView}: classe di tipo modelview la cui istanziazione è contenuta all'interno della variabile di ambiente \$scope di \textit{Angular.js\ped{G}}. All'interno di essa sono presenti le variabili e i metodi necessari per il \textit{Two-Way Data-Binding\ped{G}} tra la view \texttt{TrainingView} e il controller \texttt{TrainingController}; 
				\item \textit{IN} \texttt{FillingQuestionnaireModelView}: classe di tipo modelview la cui istanziazione è contenuta all'interno della variabile di ambiente \$scope di \textit{Angular.js\ped{G}}. All'interno di essa sono presenti le variabili e i metodi necessari per il \textit{Two-Way Data-Binding\ped{G}} tra la view \texttt{FillingQuestionnaireView} e il controller \texttt{FillingQuestionnaireController};
				\item \textit{IN} \texttt{QuestionsController}: questa classe permette di gestire il recupero delle domande per poterle stampare nella modalità allenamento;
			\end{itemize}
			\item \textbf{Attributi}: 
			\begin{itemize}
				\item \texttt{+ controller: String} \\ Stringa contenente il nome del controller della direttiva;
				\item \texttt{+ restrict: String} \\ Stringa che permette di definire le modalità di inserimento della direttiva all'interno della pagina;
				\item \texttt{+ scope: Scope} \\ Oggetto scope interno della direttiva, contiene:
				\begin{itemize}
					\item \texttt{+ questionText: String} \\ Identifica il testo della domanda;
					\item \texttt{+ image: String} \\ Identifica l'url di una possibile immagine nella domanda;
					\item \texttt{+ answers: Array} \\ Array che contiene coppie di valori. Queste coppie sono formate da:
					\begin{itemize}
						\item \texttt{+ type: String} \\ Indica la tipologia della risposta;
						\item \texttt{+ text: String} \\ Contiene il testo dell'affermazione;
						\item \texttt{+ url: String} \\ Rappresenta l'immagine della risposta;
						\item \texttt{+ attributesForSorting: Mixed} \\ Contiene i seguenti attributi:
						\begin{enumerate}
							\item \texttt{+ position: Boolean} \\ Contiene la giusta posizione del testo o dell'immagine nell'esercizio di ordinamento.
						\end{enumerate}
					\end{itemize}
				\end{itemize}
				\item \texttt{+ templateUrl: String} \\ Stringa contenente il percorso del file \textit{HTML\ped{G}} che contiene la direttive.
			\end{itemize}
		\end{itemize}
		
		\paragraph{QuizziPedia::Front-End::Directives::TrainingSetUpDirective}
		
		\label{QuizziPedia::Front-End::Directives::TrainingSetUpDirective}
		
		\begin{figure}[ht]
			\centering
			\includegraphics[scale=0.80,keepaspectratio]{UML/Classi/Front-End/QuizziPedia_Front-end_Templates_TrainingSetUpTemplate.png}
			\caption{QuizziPedia::Front-End::Directives::TrainingSetUpDirective}
		\end{figure} \FloatBarrier
		
		\begin{itemize}
			\item \textbf{Descrizione}: rappresenta il componente grafico che permette all'utente di selezionare l'argomento e le parole chiave per iniziare un allenamento con queste caratteristiche. Viene visualizzato dinamicamente all'interno della view TrainingView mediante il controller TrainingController;
			\item \textbf{Utilizzo}: viene utilizzato per consentire all'utente di selezionare l'argomento e le parole chiave di un allenamento;
			\item \textbf{Relazioni con altre classi}: 
			\begin{itemize}
				\item \textit{IN} \texttt{TrainingModelView}: classe di tipo modelview la cui istanziazione è contenuta all'interno della variabile di ambiente \$scope di \textit{Angular.js\ped{G}}. All'interno di essa sono presenti le variabili e i metodi necessari per il \textit{Two-Way Data-Binding\ped{G}} tra la view \texttt{TrainingView} e il controller \texttt{TrainingController};  
				\item \textit{IN} \texttt{TrainingController}: questa classe permette di gestire la modalità allenamento sottoponendo all'utente le giuste domande adatte al suo livello;
			\end{itemize}
			\item \textbf{Attributi}: 
			\begin{itemize}
				\item \texttt{+ controller: String} \\ Stringa contenente il nome del controller della direttiva;
				\item \texttt{+ restrict: String} \\ Stringa che permette di definire le modalità di inserimento della direttiva all'interno della pagina;
				\item \texttt{+ scope: Scope} \\ Oggetto scope interno della direttiva, contiene:
				\begin{itemize}
					\item \textit{+ topic: String} \\ Valore ottenuto come risultato dal ciclo effettuato sull'array degli argomenti;
					\item \textit{+ keywords: String} \\ Valore ottenuto come risultato dal ciclo effettuato sull'array delle keywords.
				\end{itemize}
				\item \texttt{+ templateUrl: String} \\ Stringa contenente il percorso del file \textit{HTML\ped{G}} che contiene la direttive.
			\end{itemize}
		\end{itemize}
		
		\paragraph{QuizziPedia::Front-End::Directives::TrueFalseAnswerDirective}
		
		\label{QuizziPedia::Front-End::Directives::TrueFalseAnswerDirective}
		
		\begin{figure}[ht]
			\centering
			\includegraphics[scale=0.80,keepaspectratio]{UML/Classi/Front-End/QuizziPedia_Front-end_Templates_TrueFalseAnswerTemplate.png}
			\caption{QuizziPedia::Front-End::Directives::TrueFalseAnswerDirective}
		\end{figure} \FloatBarrier
		
		\begin{itemize}
			\item \textbf{Descrizione}: rappresenta il componente grafico che permette all'utente di visualizzare la domanda vero e falso. Viene visualizzato dinamicamente all'interno delle views TrainingView e FillingQuestionnaireView mediante il controller QuestionsController;
			\item \textbf{Utilizzo}: viene utilizzato per consentire all'utente la compilazione della domanda vero e falso;
			\item \textbf{Relazioni con altre classi}: 
			\begin{itemize}
				\item \textit{IN} \texttt{TrainingModelView}: classe di tipo modelview la cui istanziazione è contenuta all'interno della variabile di ambiente \$scope di \textit{Angular.js\ped{G}}. All'interno di essa sono presenti le variabili e i metodi necessari per il \textit{Two-Way Data-Binding\ped{G}} tra la view \texttt{TrainingView} e il controller \texttt{TrainingController}; 
				\item \textit{IN} \texttt{QuestionsController}: questa classe permette di gestire il recupero delle domande per poterle stampare nella modalità allenamento;
			\end{itemize}
			\item \textbf{Attributi}: 
			\begin{itemize}
				\item \texttt{+ controller: String}: stringa contenente il nome del controller della direttiva;
				\item \texttt{+ restrict: String}: stringa che permette di definire le modalità di inserimento della direttiva all'interno della pagina;
				\item \texttt{+ scope: Scope}: oggetto scope interno della direttiva, contiene:
				\begin{itemize}
					\item \texttt{questionText: String}: identifica il testo della domanda;
					\item \texttt{image: String}: identifica l'url di una possibile immagine nella domanda;
					\item \texttt{answers: Array}: array che contiene coppie di valori. Queste coppie sono formate da:
					\begin{itemize}
						\item \texttt{type: String}: indica la tipologia della risposta;
						\item \texttt{text: String}: contiene il testo dell'affermazione;
						\item \texttt{url: String}: rappresenta l'immagine della risposta;
						\item \texttt{attributesForTForMultiple: Mixed}: contiene i seguenti attributi:
						\begin{enumerate}
							\item \texttt{isItRight: Boolean}: contiene se la risposta è vera o falsa.
						\end{enumerate}
					\end{itemize}
				\end{itemize}
				\item \texttt{+ templateUrl: String}: stringa contenente il percorso del file \textit{HTML\ped{G}} che contiene la direttive.
			\end{itemize}
		\end{itemize}																	