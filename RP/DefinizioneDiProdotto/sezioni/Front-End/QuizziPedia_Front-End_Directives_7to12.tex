\paragraph{QuizziPedia::Front-End::Directives::MenuBarDirective}

\label{QuizziPedia::Front-End::Directives::MenuBarDirective}

\begin{figure}[h]
	\centering
	\includegraphics[scale=0.5,keepaspectratio]{UML/Classi/Front-End/QuizziPedia_Front-end_Directives_MenuBarDirective.png}
	\caption{QuizziPedia::Front-End::Directives::MenuBarDirective}
\end{figure}

\begin{itemize}
	\item \textbf{Descrizione}: rappresenta un menù, presente in ogni pagina dell'applicazione, generato in base agli oggetti passati nello scope isolato. Fornisce un pulsante per ogni oggetto ricevuto come parametro, ogni pulsante viene rappresentato con un’icona e con del testo. Al click di un pulsante viene invocata la funzione ad esso associata;
	\item \textbf{Utilizzo}: viene utilizzato per realizzare il menù, presente in ogni pagina dell'applicazione, che permette all'utente di selezionare un'opzione in base al contesto in cui si trova:
		\begin{itemize}
			\item Login;
			\item Registrazione;
			\item Ricerca;
			\item Visualizzare il proprio profilo utente;
			\item Gestire le domande create;
			\item Gestire i questionari creati.
		\end{itemize}
	\item \textbf{Relazioni con altre classi}: 
	\begin{itemize}
		\item \textit{IN} \texttt{Index}: 
		\item \textit{IN} \texttt{SearchDirective}: 
		\item \textit{IN} \texttt{LogoutController}: 
		\item \textit{IN} \texttt{MenuBarController}: 
	\end{itemize}
	\item \textbf{Attributi}: 
	\begin{itemize}
		\item ;
	\end{itemize}
	\item \textbf{Metodi}: 
	\begin{itemize}
		\item ;
	\end{itemize}
\end{itemize}

\paragraph{QuizziPedia::Front-End::Directives::NewQuestionButtonDirective}

\label{QuizziPedia::Front-End::Directives::NewQuestionButtonDirective}

\begin{figure}[h]
	\centering
	\includegraphics[scale=0.5,keepaspectratio]{UML/Classi/Front-End/QuizziPedia_Front-end_Directives_NewQuestionButtonDirective.png}
	\caption{QuizziPedia::Front-End::Directives::NewQuestionButtonDirective}
\end{figure}

\begin{itemize}
	\item \textbf{Descrizione}: rappresenta il componente grafico che permette all'utente di posizionarsi nella view di creazione di una nuova domanda;
	\item \textbf{Utilizzo}: viene utilizzato per permette all'utente di posizionarsi nella view di creazione di una nuova domanda;
	\item \textbf{Relazioni con altre classi}: 
	\begin{itemize}
		\item \textit{IN} \texttt{QuestionsManagementView}: 
		\item \textit{IN} \texttt{NewQuestionButtonsController}: 
	\end{itemize}
	\item \textbf{Attributi}: 
	\begin{itemize}
		\item ;
	\end{itemize}
	\item \textbf{Metodi}: 
	\begin{itemize}
		\item ;
	\end{itemize}
\end{itemize}

\paragraph{QuizziPedia::Front-End::Directives::OneQuestionDirective}

\label{QuizziPedia::Front-End::Directives::OneQuestionDirective}

\begin{figure}[h]
	\centering
	\includegraphics[scale=0.5,keepaspectratio]{UML/Classi/Front-End/QuizziPedia_Front-end_Directives_OneQuestionDirective.png}
	\caption{QuizziPedia::Front-End::Directives::OneQuestionDirective}
\end{figure}

\begin{itemize}
	\item \textbf{Descrizione}: rappresenta il componente grafico che visualizza all'utente l'anteprima della domanda che ha creato. Eseguendo l'azione di click su di essa sarà possibile modificare tale domanda. All'interno di QuestionsManagementsView verranno stampati a video tanti componenti quanti presenti nello scope isolato ad esso associato;
	\item \textbf{Utilizzo}: viene utilizzato per permettere all'utente di visualizzare le domande che ha creato;
	\item \textbf{Relazioni con altre classi}: 
	\begin{itemize}
		\item \textit{IN} \texttt{QuestionsManagementView}: 
	\end{itemize}
	\item \textbf{Attributi}: 
	\begin{itemize}
		\item ;
	\end{itemize}
	\item \textbf{Metodi}: 
	\begin{itemize}
		\item ;
	\end{itemize}
\end{itemize}

\paragraph{QuizziPedia::Front-End::Directives::QuestionTextDirective}

\label{QuizziPedia::Front-End::Directives::QuestionTextDirective}

\begin{figure}[h]
	\centering
	\includegraphics[scale=0.5,keepaspectratio]{UML/Classi/Front-End/QuizziPedia_Front-end_Directives_QuestionTextDirective.png}
	\caption{QuizziPedia::Front-End::Directives::QuestionTextDirective}
\end{figure}

\begin{itemize}
	\item \textbf{Descrizione}: rappresenta il componente grafico che permette all'utente di scrivere o modificare il testo di una domanda;
	\item \textbf{Utilizzo}: viene usato per permettere all'utente di scrivere o modificare il testo di una domanda;
	\item \textbf{Relazioni con altre classi}: 
	\begin{itemize}
		\item \textit{IN} \texttt{TrueFalseQuestionsView}: 
		\item \textit{IN} \texttt{MultipleQuestionsView}: 
		\item \textit{IN} \texttt{ConnectionQuestionsView}: 
		\item \textit{IN} \texttt{ImagesSortingQuestionsView}: 
		\item \textit{IN} \texttt{StringsSortingQuestionsView}: 
		\item \textit{IN} \texttt{FillingQuestionsView}: 
		\item \textit{IN} \texttt{ClickableAreaQuestionsView}: 
	\end{itemize}
	\item \textbf{Attributi}: 
	\begin{itemize}
		\item ;
	\end{itemize}
	\item \textbf{Metodi}: 
	\begin{itemize}
		\item ;
	\end{itemize}
\end{itemize}

\paragraph{QuizziPedia::Front-End::Directives::QuestionnaireDetailsDirective}

\label{QuizziPedia::Front-End::Directives::QuestionnaireDetailsDirective}

\begin{figure}[h]
	\centering
	\includegraphics[scale=0.5,keepaspectratio]{UML/Classi/Front-End/QuizziPedia_Front-end_Directives_QuestionnaireDetailsDirective.png}
	\caption{QuizziPedia::Front-End::Directives::QuestionnaireDetailsDirective}
\end{figure}

\begin{itemize}
	\item \textbf{Descrizione}: rappresenta il componente grafico che permette all'utente di visualizzare la lista di questionari che può compilare. Ogni item di questa lista contiene:
		\begin{itemize}
			\item Nome del questionario;
			\item Autore del questionario;
			\item Argomento del questionario;
			\item Parole chiave del questionario;
		\end{itemize}
	Al verificarsi dell'evento click su un item della lista l'utente verrà indirizzato alla view per la compilazione del questionario selezionato;
	\item \textbf{Utilizzo}: viene utilizzato per permettere all'utente di visualizzare la lista di questionari che può compilare;
	\item \textbf{Relazioni con altre classi}: 
	\begin{itemize}
		\item \textit{IN} \texttt{UserView}: 
		\item \textit{IN} \texttt{QuestionnaireDetailsController}: 
	\end{itemize}
	\item \textbf{Attributi}: 
	\begin{itemize}
		\item ;
	\end{itemize}
	\item \textbf{Metodi}: 
	\begin{itemize}
		\item ;
	\end{itemize}
\end{itemize}

\paragraph{QuizziPedia::Front-End::Directives::QuestionnaireDoneDetailsDirective}

\label{QuizziPedia::Front-End::Directives::QuestionnaireDoneDetailsDirective}

\begin{figure}[h]
	\centering
	\includegraphics[scale=0.5,keepaspectratio]{UML/Classi/Front-End/QuizziPedia_Front-end_Directives_QuestionnaireDetailsDoneDirective.png}
	\caption{QuizziPedia::Front-End::Directives::QuestionnaireDetailsDoneDirective}
\end{figure}

\begin{itemize}
	\item \textbf{Descrizione}: rappresenta il componente grafico che permette all'utente di visualizzare la lista di questionari che ha già compilato e di conseguenza vederne le valutazioni. Ogni item di questa lista contiene:
	\begin{itemize}
		\item Nome del questionario;
		\item Autore del questionario;
		\item Argomento del questionario;
		\item Parole chiave del questionario;
		\item Valutazione del questionario.
	\end{itemize};
	\item \textbf{Utilizzo}: viene utilizzato per permettere all'utente di visualizzare la lista di questionari che ha compilato;
	\item \textbf{Relazioni con altre classi}: 
	\begin{itemize}
		\item \textit{IN} \texttt{UserView}: 
		\item \textit{IN} \texttt{QuestionnaireDetailsController}: 
	\end{itemize}
	\item \textbf{Attributi}: 
	\begin{itemize}
		\item ;
	\end{itemize}
	\item \textbf{Metodi}: 
	\begin{itemize}
		\item ;
	\end{itemize}
\end{itemize}

\paragraph{QuizziPedia::Front-End::Directives::QuestionsManagementQuestionnaireDirective}

\label{QuizziPedia::Front-End::Directives::QuestionsManagementQuestionnaireDirective}

\begin{figure}[h]
	\centering
	\includegraphics[scale=0.5,keepaspectratio]{UML/Classi/Front-End/QuizziPedia_Front-end_Directives_QuestionsManagementQuestionnaireDirective.png}
	\caption{QuizziPedia::Front-End::Directives::QuestionsManagementQuestionnaireDirective}
\end{figure}

\begin{itemize}
	\item \textbf{Descrizione}: rappresenta il componente grafico che permette all'utente di:
		\begin{itemize}
			\item Effettuare delle ricerche sul database di domande;
			\item Selezionare le domande da inserire nel questionario;
			\item Mostrare le domande già inserite e permettere all'utente di eliminarle da tale lista.
		\end{itemize}
		Questo componente si presta sia per la creazione che per la modifica di un questionario;
	\item \textbf{Utilizzo}: viene utilizzato per gestire le domande di un questionario. Esso permette di ricercare, inserire e togliere domande dalla lista di domande che andranno a comporre il questionario;
	\item \textbf{Relazioni con altre classi}: 
	\begin{itemize}
		\item \textit{IN} \texttt{CreateQuestionnaireView}: 
		\item \textit{IN} \texttt{QuestionnaireQuestionsManagementController}: 
	\end{itemize}
	\item \textbf{Attributi}: 
	\begin{itemize}
		\item ;
	\end{itemize}
	\item \textbf{Metodi}: 
	\begin{itemize}
		\item ;
	\end{itemize}
\end{itemize}