\paragraph{QuizziPedia::Front-End::ModelViews::MultipleQuestionsModelView}
\begin{figure} [ht]
	\centering
	%\includegraphics[scale=0.80]{UML/Classi/Front-End/QuizziPedia_Front-end_Views_MultipleQuestionsModelView.png}
	\caption{QuizziPedia::Front-End::ModelViews::MultipleQuestionsModelView}
\end{figure} \FloatBarrier
\begin{itemize}
	\item \textbf{Descrizione}: classe di tipo modelview la cui istanziazione è contenuta all'interno della variabile di ambiente \$scope di \textit{Angular.js\ped{G}}. All'interno di essa sono presenti le variabili e i metodi necessari per il \textit{Two-Way Data-Binding\ped{G}} tra la view \texttt{MultipleQuestionsView} e il controller \texttt{MultipleQuestionsController}; 
	\item \textbf{Utilizzo}: viene utilizzata per effettuare il \textit{Two-Way Data-Binding\ped{G}} tra la view \texttt{MultipleQuestionsView} e il controller \texttt{MultipleQuestionsController} rendendo disponibili variabili e metodi;
	\item \textbf{Relazioni con altre classi}:
	\begin{itemize}
		\item \textit{IN} \texttt{MultipleQuestionsView}: view contenente le direttive per creare una domanda a risposta multipla; 
		\item \textit{IN} \texttt{MultipleQuestionsController}: questa classe permette di gestire la creazione e la modifica di una domanda a risposta multipla;
	\end{itemize}
	\item \textbf{Attributi}:
	\begin{itemize}
		\item {+ image: String} \\ Attributo contenete l'\textit{URL\ped{G}} dell'immagine caricata dall'utente;
		\item {+ questionText: String} \\ Attributo contenente il testo della domanda;
		\item \texttt{questionText: String} \\ Identifica il testo della domanda;
		\item \texttt{image: String} \\ Identifica l'url di una possibile immagine nella domanda;
		\item \texttt{answers: Array}\\ Array che contiene coppie di valori. Queste coppie sono formate da:
		\begin{itemize}
			\item \texttt{type: String} \\ Indica la tipologia della risposta;
			\item \texttt{text: String} \\ Contiene il testo dell'affermazione;
			\item \texttt{url: String} \\ Rappresenta l'immagine della risposta;
			\item \texttt{attributesForTForMultiple: Mixed} \\ Contiene i seguenti attributi:
			\begin{enumerate}
				\item \texttt{isItRight: Boolean} \\ Contiene se la risposta è vera o falsa.
			\end{enumerate}
		\end{itemize}
	\end{itemize}
	\item \textbf{Metodi}:
	\begin{itemize}
		\item \texttt{+} \texttt{submitQuestion(): void}\\ 
		Metodo che gestisce l’evento click sul pulsante di conferma sulla domanda. Raccoglie i dati dal modelview e li manda al server attraverso \texttt{QuestionService}. Poi verrà effettuato il redirect alla pagina di gestione delle domande oppure al questionario che si stava creando; 
	\end{itemize}
\end{itemize}

