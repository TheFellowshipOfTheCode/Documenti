\paragraph{QuizziPedia::Front-End::ModelViews::QuestionsModelView}

\label{QuizziPedia::Front-End::ModelViews::QuestionsModelView}

\begin{figure}[ht]
	\centering
	\includegraphics[scale=0.5,keepaspectratio]{UML/Classi/Front-End/QuizziPedia_Front-end_Templates_ClickableAnswerTemplate.png}
	\caption{QuizziPedia::Front-End::ModelViews::QuestionsModelView}
\end{figure} \FloatBarrier

\begin{itemize}
	\item \textbf{Descrizione}: classe di tipo modelview la cui istanziazione è contenuta all'interno della variabile di ambiente \$scope di \textit{Angular.js\ped{G}}. All'interno di essa sono presenti le variabili e i metodi necessari per il \textit{Two-Way Data-Binding\ped{G}} tra le directive che compongono dinamicamente la vista della domanda e il controller \texttt{QuestionsController};
	\item \textbf{Utilizzo}: viene utilizzata per effettuare il \textit{Two-Way Data-Binding\ped{G}} tra le directive che compongono dinamicamente la vista della domanda e il controller \texttt{QuestionsController} rendendo disponibili variabili e metodi;
	\item \textbf{Relazioni con altre classi}: 
	\begin{itemize} 
		\item \textit{OUT} \texttt{QuestionsController}: questa classe permette di gestire il recupero delle domande per poterle stampare nella modalità allenamento;
		\item \textit{OUT} \texttt{ClickableAnswerDirective}: rappresenta il componente grafico che permette all'utente di visualizzare la domanda ad area cliccabile nell'immagine. Viene visualizzato dinamicamente all'interno delle views TrainingView e FillingQuestionnaireView mediante il controller QuestionsController;
		\item \textit{OUT} \texttt{EmptySpaceAnswerDirective}: rappresenta il componente grafico che permette all'utente di visualizzare l'esercizio a riempimento di spazi vuoti. Viene visualizzato dinamicamente all'interno delle views TrainingView e FillingQuestionnaireView mediante il controller QuestionsController;
		\item \textit{OUT} \texttt{HeaderTextQuestionDirective}: rappresenta il componente grafico che presenta all'utente l'argomento, le parole chiave e il numero di domande complessive. Viene visualizzato dinamicamente all'interno delle views TrainingView e FillingQuestionnaireView mediante il controller QuestionsController;
		\item \textit{OUT} \texttt{LinkingAnswerDirective}: rappresenta il componente grafico che permette all'utente di visualizzare la domanda di collegamento. Viene visualizzato dinamicamente all'interno delle views TrainingView e FillingQuestionnaireView mediante il controller QuestionsController;
		\item \textit{OUT} \texttt{MultipleChoiceAnswerDirective}: rappresenta il componente grafico che permette all'utente di visualizzare la domanda a risposta multipla. Viene visualizzato dinamicamente all'interno delle views TrainingView e FillingQuestionnaireView mediante il controller QuestionsController;
		\item \textit{OUT} \texttt{SortImagesAnswerDirective}: rappresenta il componente grafico che permette all'utente di visualizzare la domanda ad ordinamento di immagini. Viene visualizzato dinamicamente all'interno delle views TrainingView e FillingQuestionnaireView mediante il controller QuestionsController;
		\item \textit{OUT} \texttt{SortTextAnswerDirective}: rappresenta il componente grafico che permette all'utente di visualizzare la domanda ad ordinamento di stringhe. Viene visualizzato dinamicamente all'interno delle views TrainingView e FillingQuestionnaireView mediante il controller QuestionsController;
		\item \textit{OUT} \texttt{TrueFalseAnswareDirective}: rappresenta il componente grafico che permette all'utente di visualizzare la domanda vero e falso. Viene visualizzato dinamicamente all'interno delle views TrainingView e FillingQuestionnaireView mediante il controller QuestionsController;	
		\item \textit{OUT} \texttt{TrainingView}: view principale della modalità allenamento, conterrà i vari templates di ogni domanda dell'allenamento.				
	\end{itemize}
	\item \textbf{Attributi}: 
	\begin{itemize}
		\item \texttt{+ piecesOfQuestion: Array[Object]} \\
		Questo attributo è un \texttt{array} di \texttt{Object} contenente la domanda da visualizzare dinamicamente attraverso le direttive all'interno le direttive di allenamento e di compilazione dei questionari;
		\item \texttt{+ objAnswer: Array[Object]} \\
		Questo attributo è un \texttt{array} di \texttt{Object} contenente le risposte date fino a quel momento dall'utente in una domanda. L'\texttt{Object} è così formato: \\
		\begin{itemize}
			\item \texttt{+ typeQuestion: String} \\
			Questo attributo rappresenta il tipo della domanda;
			\item \texttt{+ answerGiven: Array[String]} \\
			Questo attributo rappresenta le riposte scelte dall'utente fino a quel momento. Può essere creato con una funzione di \texttt{callback}.
		\end{itemize}
	\end{itemize}
	\item \textbf{Metodi}: 
	\begin{itemize}
		\item \texttt{+} \texttt{addAnswer(index: Number, typeQuestion: String, answerGiven: Array[String]): void} \\
		Metodo che gestisce l'evento di selezione delle risposte. \\
		\textbf{Parametri}:
		\begin{itemize}
			\item \texttt{index: Number} \\
			Parametro contenente l'indice della risposta di cui si vuole tenere traccia. Rappresenta anche l'indice dell'\texttt{array objAswer} in cui verrà inserito l'oggetto delle risposte date;
			\item \texttt{typeQuestion: String} \\
			Parametro contenente una stringa la quale indica la tipologia della domanda;
			\item \texttt{answerGiven: Array[String]} \\
			Parametro contenente l'array di risposte date dall'utente aggiornato all'ultima iterazione.
		\end{itemize};
		\item \texttt{+} \texttt{answerGiven(index: Number): Array[String]} \\
		Metodo di supporto che ritorna un \texttt{array} di stringhe contenente le risposte date. Si occupa di recuperare le risposte date nelle domande vero/falso, risposta multipla e ad area cliccabile.\\
		\textbf{Parametri}:
		\begin{itemize}
			\item \texttt{index: Number} \\
			Parametro contenente l'indice della risposta di cui si vuole raccogliere le risposte date. 
		\end{itemize}
		\item \texttt{+} \texttt{orderChosen(index: Number): Array[String]} \\
		Metodo di supporto che ritorna un \texttt{array} di stringhe contenente le risposte date. Si occupa di recuperare le risposte date nelle domande ad ordinamento e di riempimento di spazi.\\
		\textbf{Parametri}:
		\begin{itemize}
			\item \texttt{index: Number} \\
			Parametro contenente l'indice della risposta di cui si vuole raccogliere le risposte date. 
		\end{itemize}
		\item \texttt{+} \texttt{linkingMade(index: Number): Array[String]} \\
		Metodo di supporto che ritorna un \texttt{array} di stringhe contenente le risposte date. Si occupa di recuperare le risposte date nelle domande a collegamento.\\
		\textbf{Parametri}:
		\begin{itemize}
			\item \texttt{index: Number} \\
			Parametro contenente l'indice della risposta di cui si vuole raccogliere le risposte date. 
		\end{itemize}
		\item \texttt{+} \texttt{loadNewQuestionBy(topic: String, keywords: Array[String], level: Number): void} \\
		Metodo che gestisce l'evento per scaricare una nuova domanda in base ai parametri passati. Evoca l'evento per inserire la domanda in \texttt{TrainingModelView}. \\
		\textbf{Parametri}:
		\begin{itemize}
			\item \texttt{topic: String} \\
			Parametro contenente l'argomento della domanda;
			\item \texttt{keywords: Array[String]} \\
			Parametro contenente un\texttt{array} di stringhe che rappresenta le keywords scelte per l'allenamento;
			\item \texttt{level: Number} \\
			Parametro contenente il livello dell'utente.
		\end{itemize}
		\item \texttt{+} \texttt{loadNewQuestion(question: QuestionItemModel): void} \\
		Metodo che gestisce l'evento per visualizzare una nuova domanda. \\
		\textbf{Parametri}:
		\begin{itemize}
			\item \texttt{question: QuestionItemModel} \\
			Parametro contenente un riferimento all'oggetto di tipo \texttt{QuestionItemModel}.
		\end{itemize}
		\item \texttt{+} \texttt{checkAnswer(): boolean} \\ 
		Metodo che controlla che le risposte date siano corrette.
	\end{itemize}
\end{itemize}

	