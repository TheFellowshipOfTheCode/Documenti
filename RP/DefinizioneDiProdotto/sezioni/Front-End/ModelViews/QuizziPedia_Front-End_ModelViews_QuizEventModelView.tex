\paragraph{QuizziPedia::Front-End::ModelViews::QuizEventModelView}
							
							\label{QuizziPedia::Front-End::ModelViews::QuizEventModelView}
							
							\begin{figure}[ht]
								\centering
								\includegraphics[scale=0.5,keepaspectratio]{UML/Classi/Front-End/QuizziPedia_Front-end_Templates_ClickableAnswerTemplate.png}
								\caption{QuizziPedia::Front-End::ModelViews::QuizEventModelView}
							\end{figure} \FloatBarrier
							
							\begin{itemize}
								\item \textbf{Descrizione}: classe di tipo modelview la cui istanziazione è contenuta all'interno della variabile di ambiente \$scope di \textit{Angular.js\ped{G}}. All'interno di essa sono presenti le variabili e i metodi necessari per il \textit{Two-Way Data-Binding\ped{G}} tra la view \texttt{QuizEventView} e il controller \texttt{QuizEventController};
								\item \textbf{Utilizzo}: viene utilizzata per effettuare il \textit{Two-Way Data-Binding\ped{G}} tra la view \texttt{QuizEventView} e il controller \texttt{QuizEventController} rendendo disponibili variabili e metodi;
								\item \textbf{Relazioni con altre classi}: 
								\begin{itemize}
									\item \textit{IN} \texttt{QuestionnaireManagementView}: view principale per la gestione dei questionari; 
									\item \textit{IN} \texttt{QuizEventController}: questa classe permette di reagire ai comandi dell'utente durante la gestione dei suoi questionari;
								\end{itemize}
								\item \textbf{Attributi}: 
								\begin{itemize}
									\item ;
								\end{itemize}
								\item \textbf{Metodi}: 
								\begin{itemize}
									\item \texttt{+} \texttt{modifyQuestionnaire(quizId: String): void} \\
									Metodo che gestisce l’evento click sul pulsante di modifica questionario. Effettua il redirect alla pagina di gestione questionari;
									\begin{itemize}
										\item \texttt{quizId: String}: parametro che indica l'identificativo univoco di un questionario.
									\end{itemize}
									\item \texttt{+} \texttt{deleteQuestionnaire(quizId: String): void} \\
									Metodo che gestisce l’evento click sul pulsante di eliminazione questionario. Effettua il redirect alla pagina di gestione questionari;  
									\begin{itemize}
										\item \texttt{quizId: String}: parametro che indica l'identificativo univoco di un questionario.
									\end{itemize}
									\item \texttt{+} \texttt{subscribeManagement(quizId: String): void} \\
									Metodo che gestisce l’evento click sul pulsante di gestione iscrizioni. Effettua il redirect alla pagina di gestione iscrizioni;
									\item \texttt{+} \texttt{examModalityquizId: String(): void} \\
									\begin{itemize}
										\item \texttt{quizId: String}: parametro che indica l'identificativo univoco di un questionario.
									\end{itemize}
									Metodo che gestisce l’evento click sul pulsante di attivazione modalità esame. Effettua il redirect alla pagina di gestione questionari;
									\begin{itemize}
										\item \texttt{quizId: String}: parametro che indica l'identificativo univoco di un questionario.
									\end{itemize}
									\item \texttt{+} \texttt{resultsQuestionnaire(quizId: String): void} \\
									Metodo che gestisce l’evento click sul pulsante di allenamento. Effettua il redirect alla pagina di gestine questionari;
									\begin{itemize}
										\item \texttt{quizId: String}: parametro che indica l'identificativo univoco di un questionario.
									\end{itemize}   
								\end{itemize}
							\end{itemize}