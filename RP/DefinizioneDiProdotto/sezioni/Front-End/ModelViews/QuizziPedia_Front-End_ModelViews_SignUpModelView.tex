\paragraph{QuizziPedia::Front-End::ModelViews::SignUpModelView}
	
	\label{QuizziPedia::Front-End::ModelViews::SignUpModelView}
	
	\begin{figure}[ht]
		\centering
		\includegraphics[scale=0.5,keepaspectratio]{UML/Classi/Front-End/QuizziPedia_Front-end_Templates_ClickableAnswerTemplate.png}
		\caption{QuizziPedia::Front-End::ModelViews::SignUpModelView}
	\end{figure} \FloatBarrier
	
	\begin{itemize}
		\item \textbf{Descrizione}: classe di tipo modelview la cui istanziazione è contenuta all'interno della variabile di ambiente \$scope di \textit{Angular.js\ped{G}}. All'interno di essa sono presenti le variabili e i metodi necessari per il \textit{Two-Way Data-Binding\ped{G}} tra la view \texttt{SignUpView} e il controller \texttt{SignUpController};
		\item \textbf{Utilizzo}: viene utilizzata per effettuare il \textit{Two-Way Data-Binding\ped{G}} tra la view \texttt{SignUpView} e il controller \texttt{SignUpController} rendendo disponibili variabili e metodi;
		\item \textbf{Relazioni con altre classi}: 
		\begin{itemize}
			\item \textit{IN} \texttt{SignUpView}: view contenente le form dedicate alla registrazione utente. Contiene inoltre un link alla pagina di login; 
			\item \textit{IN} \texttt{SignUpController}: questa classe permette di gestire la registrazione di un utente al sistema;
		\end{itemize}
		\item \textbf{Attributi}: 
		\begin{itemize}
			\item \texttt{+ name} \\ Campo dati contenente il nome;
			\item \texttt{+ surname} \\ Campo dati contenente il cognome;
			\item \texttt{+ username} \\ Campo dati contenente lo username;
			\item \texttt{+ email} \\ Campo dati contenente la email;
			\item \texttt{+ password} \\ Campo dati contenente la password;
			\item \texttt{+ passwordCheck} \\ Campo dati contenente la variabile di conferma password.
		\end{itemize}
		\item \textbf{Metodi}: 
		\begin{itemize}
			\item \texttt{+} \texttt{signUp(): void} \\
			Metodo che richiama il metodo \texttt{signUp} del service \texttt{AuthService} passandogli un oggetto di tipo \texttt{SignUpModelView}. Nel caso di buona riuscita dell'operazione viene mostrato un messaggio di successo. Con l'azione di click sul bottone presentato dal messaggio di successo è possibile effettuare il redirect alla pagina di login dell'applicazione. Nel caso in cui invece avvenga un errore, viene mostrato a video il messaggio di errore;
			\item \texttt{+} \texttt{logIn(): void} \\
			Metodo che gestisce l’evento click sul pulsante di login. Effettua il redirect alla pagina di login;
		\end{itemize}
	\end{itemize}
	
	