\paragraph{QuizziPedia::Front-End::Views::ResultsView}
\begin{figure} [ht]
	\centering
	\includegraphics[scale=0.80]{UML/Classi/Front-End/QuizziPedia_Front-end_Views_ResultsView.png}
	\caption{QuizziPedia::Front-End::Views::ResultsView}
\end{figure} \FloatBarrier
\begin{itemize}
	\item \textbf{Descrizione}: view contenente i risultati della ricerca effettuata, sia gli utenti che i questionari trovati;
	\item \textbf{Utilizzo}: viene visualizzata dopo aver effettuato la ricerca di un utente o di un questionario nella barra di ricerca presente nella SearchDirective e permette di selezionare un risultato presente al suo interno; 
	\item \textbf{Relazioni con altre classi}:
	\begin{itemize}
		\item \textit{IN} \texttt{SearchController}: questa classe permette di gestire la ricerca di questionari e utenti all'interno dell'applicazione;
		\item \textit{IN} \texttt{ResultsModelView}: classe di tipo modelview la cui istanzazione è contenuta all'interno della variabile di ambiente \$scope di \texttt{Angular.js}. All'interno di essa sono presenti le variabili e i metodi necessari per il \textit{Two-Way Data-Binding\ped{G}} tra la view \texttt{ResultsView} e il controller \texttt{SearchController};
		\item \textit{IN} \texttt{SubscribeResultDirective}: directive che permette di visualizzare e iscriversi ai questionari ricercati;
		\item \textit{IN} \texttt{UserResultsDirective}: directive che permette di visualizzare la lista degli utenti ricercati dopo aver utilizzato l'apposita funzione di ricerca;
		\item \textit{IN} \texttt{LangModel}: rappresenta il modello delle informazioni per la giusta traduzione dell'applicazione.
	\end{itemize}
	\item \textbf{Attributi}:
		\begin{itemize}
			\item \texttt{+ titleLang: String} \\ Attributo che viene utilizzato per visualizzare la giusta traduzione del titolo della pagina, in italiano o in inglese.
		\end{itemize}
\end{itemize}