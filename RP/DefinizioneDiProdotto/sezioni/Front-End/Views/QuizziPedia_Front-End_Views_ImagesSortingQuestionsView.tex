\paragraph{QuizziPedia::Front-End::Views::ImagesSortingQuestionsView}
\begin{figure} [ht]
	\centering
	\includegraphics[scale=0.80]{UML/Classi/Front-End/QuizziPedia_Front-end_Views_ImagesSortingQuestionsView.png}
	\caption{QuizziPedia::Front-End::Views::ImagesSortingQuestionsView}
\end{figure} \FloatBarrier
\begin{itemize}
	\item \textbf{Descrizione}: view contenente i campi e le direttive per creare una domanda a ordinamento immagini;
	\item \textbf{Utilizzo}: permette all'utente di creare una domanda a ordinamento immagini compilando i campi proposti;
	\item \textbf{Relazioni con altre classi}:
	\begin{itemize}
		\item \textit{IN} \texttt{ImageSortingQuestionsModelView}: classe di tipo modelview la cui istanziazione è contenuta all'interno della variabile di ambiente \$scope di \texttt{Angular.js}. All'interno di essa sono presenti le variabili e i metodi necessari per il \textit{Two-Way Data-Binding\ped{G}} tra la view \texttt{ImagesSortingQuestionsModelView} e il controller \texttt{ImagesSortingQuestionsController};
		\item \textit{IN} \texttt{InputToListModelView}: classe di tipo modelview la cui istanziazione è contenuta all'interno della variabile di ambiente \$scope di \texttt{Angular.js}. All'interno di essa sono presenti le variabili e i metodi necessari per il \textit{Two-Way Data-Binding\ped{G}} tra la view \texttt{ImageSortingQuestionsView} e il controller \texttt{InputToListController};
		\item \textit{IN} \texttt{ImageInTheQuestionDirective}: directive per l'inserimento dell'immagine nella creazione delle domande;
		\item \textit{IN} \texttt{TopicKeywordsDirective}: directive che permette di gestire l'inserimento di keywords al momento della creazione della domanda;
		\item \textit{IN} \texttt{QuestionTextDirective}: rappresenta il componente grafico che permette all'utente di scrivere o modificare il testo di una domanda;
		\item \textit{IN} \texttt{LangModel}: rappresenta il modello delle informazioni per la giusta traduzione dell'applicazione.
	\end{itemize}
	\item \textbf{Attributi}:
	\begin{itemize}
			\item \texttt{+ question: Object} \\ Oggetto contenente gli attributi per la creazione della domanda:
			\begin{itemize}
				\item \texttt{answer}: array contenente oggetti che rappresentano le risposte. Ogni oggetto risposta contiene:
				\begin{itemize}
					\item \texttt{urlSorting}: attributo di tipo \texttt{String} che contiene l'\textit{URL\ped{G}} dell'immagine associata alla risposta;
					\item \texttt{position}: attributo di tipo \texttt{Number} che indica la giusta posizione dell'immagine.
				\end{itemize}
			\end{itemize}	  
		\item \texttt{+ titleLangImages: String} \\ Attributo che viene utilizzato per visualizzare la giusta traduzione del titolo della pagina, in italiano o in inglese;
		\item \texttt{+ buttonConfirmLangImages: String} \\ Attributo che viene utilizzato per visualizzare la giusta traduzione della \textit{label\ped{G}} per il bottone di conferma, in italiano o in inglese;
		\item \texttt{+ buttonLoadImageLangImages: String} \\ Attributo che viene utilizzato per visualizzare la giusta traduzione della \textit{label\ped{G}} per il bottone di caricamento dell'immagine nel testo della domanda, in italiano o in inglese;
		\item \texttt{+ successCreation: String} \\ Attributo che visualizza un messaggio di conferma avvenuta creazione della domanda;
		\item \texttt{+ errorCreation: String} \\ Attributo che visualizza un messaggio d'errore per la creazione della domanda.
	\end{itemize}
\end{itemize}
