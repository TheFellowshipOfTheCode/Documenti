\paragraph{QuizziPedia::Front-End::Views::FillingQuestionnaireView}
\begin{figure} [ht]
	\centering
	\includegraphics[scale=0.80]{UML/Classi/Front-End/QuizziPedia_Front-end_Views_FillingQuestionnaireView.png}
	\caption{QuizziPedia::Front-End::Views:FillingQuestionnaireView}
\end{figure} \FloatBarrier
\begin{itemize}
	\item \textbf{Descrizione}: \textit{view\ped{G}} principale per la compilazione del questionario; conterrà i vari templates di ogni domanda appartenente al questionario;
	\item \textbf{Utilizzo}: all'interno di essa verrà caricato inizialmente il template contenente le informazioni generali relative al questionario; verranno poi caricati i templates di ogni domanda presente nel questionario;
	\item \textbf{Relazioni con altre classi}: 
	\begin{itemize}
		\item \textbf{IN \texttt{FillingQuestionnaireController}}: questa classe permette di gestire la compilazione del questionario;
		\item \textbf{IN \texttt{FillingQuestionnaireModelView}}: classe di tipo modelview la cui istanziazione è contenuta all'interno della variabile di ambiente \texttt{\$scope} di \textit{Angular\ped{G}}. All'interno di essa sono presenti le variabili e i metodi necessari per il \textit{Two-Way Data-Binding\ped{G}} tra la \textit{view\ped{G}} \texttt{FillingQuestionnaireView} e il \textit{controller\ped{G}} \texttt{FillingQuestionnaireController};
		\item \textbf{IN \texttt{QuestionsModelView}}: classe di tipo modelview la cui istanziazione è contenuta all'interno della variabile di ambiente \texttt{\$scope} di \textit{Angular\ped{G}}. All'interno di essa sono presenti le variabili e i metodi necessari per il \textit{Two-Way Data-Binding\ped{G}} tra le \textit{directives\ped{G}} che compongono dinamicamente la vista della domanda e il \textit{controller\ped{G}} \texttt{QuestionsController};
		\item \textbf{IN \texttt{InfoQuestionnaireDirective}}: rappresenta il componente grafico che permette all'utente di visualizzare le informazioni principali del questionario che si sta per svolgere. Viene visualizzato dinamicamente all'interno delle \textit{views\ped{G}} \texttt{TrainingView} e \texttt{FillingQuestionnaireView} mediante il \textit{controller\ped{G}} \texttt{QuestionsController};
		\item \textbf{IN \texttt{HeaderTextQuestionDirective}}:rappresenta il componente grafico che presenta all'utente il testo della domanda, l'argomento e le parole chiave. Viene visualizzato dinamicamente all'interno delle \textit{views\ped{G}} \texttt{TrainingView} e \texttt{FillingQuestionnaireView} mediante il \textit{controller\ped{G}} \texttt{QuestionsController};
		\item \textbf{IN \texttt{TrueFalseAnswerDirective}}: rappresenta il componente grafico che permette all'utente di visualizzare la domanda vero e falso. Viene visualizzato dinamicamente all'interno delle \textit{views\ped{G}} \texttt{TrainingView} e \texttt{FillingQuestionnaireView} mediante il \textit{controller\ped{G}} \texttt{QuestionsController};
		\item \textbf{IN \texttt{MultipleChoiceAnswerDirective}}: rappresenta il componente grafico che permette all'utente di visualizzare la domanda a risposta multipla. Viene visualizzato dinamicamente all'interno delle \textit{views\ped{G}} \texttt{TrainingView} e \texttt{FillingQuestionnaireView} mediante il \textit{controller\ped{G}} \texttt{QuestionsController};
		\item \textbf{IN \texttt{LinkingAnswerDirective}}: rappresenta il componente grafico che permette all'utente di visualizzare la domanda di collegamento. Viene visualizzato dinamicamente all'interno delle \textit{views\ped{G}} \texttt{TrainingView} e \texttt{FillingQuestionnaireView} mediante il \textit{controller\ped{G}} \texttt{QuestionsController};
		\item \textbf{IN \texttt{SortImagesAnswerDirective}}: rappresenta il componente grafico che permette all'utente di visualizzare la domanda ad ordinamento di immagini. Viene visualizzato dinamicamente all'interno delle \textit{views\ped{G}} \texttt{TrainingView} e \texttt{FillingQuestionnaireView} mediante il \textit{controller\ped{G}} \texttt{QuestionsController};
		\item \textbf{IN \texttt{SortTextAnswerDirective}}: rappresenta il componente grafico che permette all'utente di visualizzare la domanda ad ordinamento di stringhe. Viene visualizzato dinamicamente all'interno delle \textit{views\ped{G}} \texttt{TrainingView} e \texttt{FillingQuestionnaireView} mediante il \textit{controller\ped{G}} \texttt{QuestionsController};
		\item \textbf{IN \texttt{EmptySpaceAnswerDirective}}: rappresenta il componente grafico che permette all'utente di visualizzare l'esercizio a riempimento di spazi vuoti. Viene visualizzato dinamicamente all'interno delle \textit{views\ped{G}} \texttt{TrainingView} e \texttt{FillingQuestionnaireView} mediante il \textit{controller\ped{G}} \texttt{QuestionsController};
		\item \textbf{IN \texttt{ClickableAnswerDirective}}: rappresenta il componente grafico che permette all'utente di visualizzare la domanda ad area cliccabile nell'immagine. Viene visualizzato dinamicamente all'interno delle \textit{views\ped{G}} \texttt{TrainingView} e \texttt{FillingQuestionnaireView} mediante il \textit{controller\ped{G}} \texttt{QuestionsController};
		\item \textbf{IN \texttt{LangModel}}: rappresenta il modello delle informazioni per la giusta traduzione dell'applicazione.
	\end{itemize}
		\item \textbf{Attributi}:
		\begin{itemize}
			\item \texttt{+ titleLangQuiz: String} \\ Attributo che viene utilizzato per visualizzare la giusta traduzione del titolo della pagina, in italiano o in inglese;
			\item \texttt{+ questionNumberLang: String} \\ Attributo che viene utilizzato per visualizzare la giusta traduzione della frase "domanda numero";
			\item \texttt{+ ofLang: String} \\ Attributo che viene utilizzato per visualizzare la giusta traduzione della parola "di";
			\item \texttt{+ buttonNextQuestionLang: String} \\ Attributo che viene utilizzato per visualizzare la giusta traduzione della \textit{label\ped{G}} per il bottone di avanzamento a domanda successiva, in italiano o in inglese;
			\item \texttt{+ buttonBackQuestionLang: String} \\ Attributo che viene utilizzato per visualizzare la giusta traduzione della \textit{label\ped{G}} per il bottone di ritorno alla domanda precedente, in italiano o in inglese;
			\item \texttt{+ buttonEndQuizLang: String} \\ Attributo che viene utilizzato per visualizzare la giusta traduzione della \textit{label\ped{G}} per il bottone di conclusione del questionario, in italiano o in inglese;
			\item \texttt{+ quiz: QuestionnaireModelView} \\ Oggetto di tipo \texttt{QuestionnaireModelView} contenente al suo interno i seguenti campi:
			\begin{itemize}
				\item \texttt{+ title: String} \\ Attributo che rappresenta il titolo del questionario;
				\item \texttt{+ argument: String} \\ Attributo che rappresenta l'argomento del questionario;
				\item \texttt{+ keywords: Array[String]} \\ \texttt{array} di stringhe che contiene le parole chiave del questionario;
				\item \texttt{+ questionNumber: String} \\ Attributo che rappresenta il numero progressivo della domanda attuale;
				\item \texttt{+ numberOfQuestions: String} \\ Attributo che rappresenta il numero di domande.
			\end{itemize}
		\end{itemize}
\end{itemize}