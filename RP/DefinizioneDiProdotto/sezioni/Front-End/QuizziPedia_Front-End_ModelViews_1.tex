	\paragraph{QuizziPedia::Front-End::ModelViews::LoginModelView}
	
	\label{QuizziPedia::Front-End::ModelViews::LoginModelView}
	
	\begin{figure}[ht]
		\centering
		\includegraphics[scale=0.5,keepaspectratio]{UML/Classi/Front-End/QuizziPedia_Front-end_Templates_ClickableAnswerTemplate.png}
		\caption{QuizziPedia::Front-End::ModelViews::LoginModelView}
	\end{figure} \FloatBarrier
	
	\begin{itemize}
		\item \textbf{Descrizione}: classe di tipo modelview la cui istanziazione è contenuta all'interno della variabile di ambiente \$scope di \textit{Angular.js\ped{G}}. All'interno di essa sono presenti le variabili e i metodi necessari per il \textit{Two-Way Data-Binding\ped{G}} tra la view \texttt{LoginView} e il controller \texttt{LoginController};
		\item \textbf{Utilizzo}: viene utilizzata per effettuare il \textit{Two-Way Data-Binding\ped{G}} tra la view \texttt{LoginView} e il controller \texttt{LoginController} rendendo disponibili variabili e metodi;
		\item \textbf{Relazioni con altre classi}: 
		\begin{itemize}
			\item \textit{OUT} \texttt{LoginView}: view contenente le form necessarie per effettuare il login. Contiene inoltre un link alla pagina di registrazione e uno alla pagina per il recupero della password; 
			\item \textit{OUT} \texttt{LoginController}: questa classe permette di gestire l'autenticazione dell'utente al sistema;
		\end{itemize}
		\item \textbf{Attributi}: 
		\begin{itemize}
			\item \texttt{+ user: Object} \\ Campo dati contenente due attributi: \texttt{email} e \texttt{password};
		\end{itemize}
		\item \textbf{Metodi}: 
		\begin{itemize}
			\item \texttt{+} \texttt{logIn(): void} \\
			Metodo che richiama il metodo \texttt{Login} del service \texttt{AuthService} passandogli \texttt{username} e \texttt{password}. Nel caso di buona riuscita dell'operazione viene effettuato il redirect alla homepage dell'applicazione. Nel caso in cui invece avvenga un errore, viene mostrato a video il messaggio di errore;
			\item \texttt{+} \texttt{signUp(): void} \\
			Metodo che gestisce l’evento click sul pulsante di registrazione. Effettua il redirect alla pagina di registrazione;
			\item \texttt{+} \texttt{recoveryPassword(): void} \\
			Metodo che gestisce l’evento click sul pulsante di recupero password. Effettua il redirect alla pagina per il recupero della password; 
		\end{itemize}
	\end{itemize}
	
	\paragraph{QuizziPedia::Front-End::ModelViews::SignUpModelView}
	
	\label{QuizziPedia::Front-End::ModelViews::SignUpModelView}
	
	\begin{figure}[ht]
		\centering
		\includegraphics[scale=0.5,keepaspectratio]{UML/Classi/Front-End/QuizziPedia_Front-end_Templates_ClickableAnswerTemplate.png}
		\caption{QuizziPedia::Front-End::ModelViews::SignUpModelView}
	\end{figure} \FloatBarrier
	
	\begin{itemize}
		\item \textbf{Descrizione}: classe di tipo modelview la cui istanziazione è contenuta all'interno della variabile di ambiente \$scope di \textit{Angular.js\ped{G}}. All'interno di essa sono presenti le variabili e i metodi necessari per il \textit{Two-Way Data-Binding\ped{G}} tra la view \texttt{SignUpView} e il controller \texttt{SignUpController};
		\item \textbf{Utilizzo}: viene utilizzata per effettuare il \textit{Two-Way Data-Binding\ped{G}} tra la view \texttt{SignUpView} e il controller \texttt{SignUpController} rendendo disponibili variabili e metodi;
		\item \textbf{Relazioni con altre classi}: 
		\begin{itemize}
			\item \textit{IN} \texttt{SignUpView}: view contenente le form dedicate alla registrazione utente. Contiene inoltre un link alla pagina di login; 
			\item \textit{IN} \texttt{SignUpController}: questa classe permette di gestire la registrazione di un utente al sistema;
		\end{itemize}
		\item \textbf{Attributi}: 
		\begin{itemize}
			\item \texttt{+ user: Object} \\ Campo dati contenente i seguenti attributi: \texttt{name}, \texttt{surname}, \texttt{username}, \texttt{email}, \texttt{password} e \texttt{passwordCheck}.
		\end{itemize}
		\item \textbf{Metodi}: 
		\begin{itemize}
			\item \texttt{+} \texttt{signUp(): void} \\
			Metodo che richiama il metodo \texttt{signUp} del service \texttt{AuthService} passandogli un oggetto di tipo \texttt{SignUpModelView}. Nel caso di buona riuscita dell'operazione viene mostrato un messaggio di successo. Con l'azione di click sul bottone presentato dal messaggio di successo è possibile effettuare il redirect alla pagina di login dell'applicazione. Nel caso in cui invece avvenga un errore, viene mostrato a video il messaggio di errore;
			\item \texttt{+} \texttt{logIn(): void} \\
			Metodo che gestisce l’evento click sul pulsante di login. Effettua il redirect alla pagina di login;
		\end{itemize}
	\end{itemize}
	
	\paragraph{QuizziPedia::Front-End::ModelViews::PasswordForgotModelView}
	
	\label{QuizziPedia::Front-End::ModelViews::PasswordForgotModelView}
	
	\begin{figure}[ht]
		\centering
		\includegraphics[scale=0.5,keepaspectratio]{UML/Classi/Front-End/QuizziPedia_Front-end_Templates_ClickableAnswerTemplate.png}
		\caption{QuizziPedia::Front-End::ModelViews::PasswordForgotModelView}
	\end{figure} \FloatBarrier
	
	\begin{itemize}
		\item \textbf{Descrizione}: classe di tipo modelview la cui istanziazione è contenuta all'interno della variabile di ambiente \$scope di \textit{Angular.js\ped{G}}. All'interno di essa sono presenti le variabili e i metodi necessari per il \textit{Two-Way Data-Binding\ped{G}} tra la view \texttt{PasswordForgotView} e il controller \texttt{PasswordForgotController};
		\item \textbf{Utilizzo}: viene utilizzata per effettuare il \textit{Two-Way Data-Binding\ped{G}} tra la view \texttt{PasswordForgotView} e il controller \texttt{PasswordForgotController} rendendo disponibili variabili e metodi;
		\item \textbf{Relazioni con altre classi}: 
		\begin{itemize}
			\item \textit{IN} \texttt{PasswordForgotView}: view contenente le form necessarie per il recupero della password dimenti- cata; 
			\item \textit{IN} \texttt{PasswordForgotController}: questa classe permette di gestire il ripristino della password dimenticata;
		\end{itemize}
		\item \textbf{Attributi}: 
		\begin{itemize}
			\item ;
		\end{itemize}
		\item \textbf{Metodi}: 
		\begin{itemize}
			\item ;
		\end{itemize}
	\end{itemize}
	
	\paragraph{QuizziPedia::Front-End::ModelViews::HomeModelView}
	
	\label{QuizziPedia::Front-End::ModelViews::HomeModelView}
	
	\begin{figure}[ht]
		\centering
		\includegraphics[scale=0.5,keepaspectratio]{UML/Classi/Front-End/QuizziPedia_Front-end_Templates_ClickableAnswerTemplate.png}
		\caption{QuizziPedia::Front-End::ModelViews::HomeModelView}
	\end{figure} \FloatBarrier
	
	\begin{itemize}
		\item \textbf{Descrizione}: classe di tipo modelview la cui istanziazione è contenuta all'interno della variabile di ambiente \$scope di \textit{Angular.js\ped{G}}. All'interno di essa sono presenti le variabili e i metodi necessari per il \textit{Two-Way Data-Binding\ped{G}} tra la view \texttt{HomeView} e il controller \texttt{HomeController};
		\item \textbf{Utilizzo}: viene utilizzata per effettuare il \textit{Two-Way Data-Binding\ped{G}} tra la view \texttt{HomeView} e il controller \texttt{HomeController} rendendo disponibili variabili e metodi;
		\item \textbf{Relazioni con altre classi}: 
		\begin{itemize}
			\item \textit{IN} \texttt{HomeView}: view contenente la direttiva per barra di ricerca degli utenti e questionari e il bottone che porterà l’utente nella modalità allenamento; 
			\item \textit{IN} \texttt{HomeController}: questa classe permette di gestire la home page;
		\end{itemize}
		\item \textbf{Attributi}: 
		\begin{itemize}
			\item ;
		\end{itemize}
		\item \textbf{Metodi}: 
		\begin{itemize}
			\item ;
		\end{itemize}
	\end{itemize}
	
	\paragraph{QuizziPedia::Front-End::ModelViews::ResultsModelView}
	
	\label{QuizziPedia::Front-End::ModelViews::ResultsModelView}
	
	\begin{figure}[ht]
		\centering
		\includegraphics[scale=0.5,keepaspectratio]{UML/Classi/Front-End/QuizziPedia_Front-end_Templates_ClickableAnswerTemplate.png}
		\caption{QuizziPedia::Front-End::ModelViews::ResultsModelView}
	\end{figure} \FloatBarrier
	
	\begin{itemize}
		\item \textbf{Descrizione}: classe di tipo modelview la cui istanziazione è contenuta all'interno della variabile di ambiente \$scope di \textit{Angular.js\ped{G}}. All'interno di essa sono presenti le variabili e i metodi necessari per il \textit{Two-Way Data-Binding\ped{G}} tra la view \texttt{ResultsView} e il controller \texttt{ResultsController};
		\item \textbf{Utilizzo}: viene utilizzata per effettuare il \textit{Two-Way Data-Binding\ped{G}} tra la view \texttt{ResultsView} e il controller \texttt{ResultsController} rendendo disponibili variabili e metodi;
		\item \textbf{Relazioni con altre classi}: 
		\begin{itemize}
			\item \textit{IN} \texttt{ResultsView}: view contenente i risultati della ricerca effettuata, sia gli utenti che i questionari trovati; 
			\item \textit{IN} \texttt{SearchController}: questa classe permette di gestire la ricerca di questionari e utenti all’interno dell’applicazione;
		\end{itemize}
		\item \textbf{Attributi}: 
		\begin{itemize}
			\item ;
		\end{itemize}
		\item \textbf{Metodi}: 
		\begin{itemize}
			\item ;
		\end{itemize}
	\end{itemize}
	
	\paragraph{QuizziPedia::Front-End::ModelViews::SearchQuestionnaireModelView}
	
	\label{QuizziPedia::Front-End::ModelViews::SearchQuestionnaireModelView}
	
	\begin{figure}[ht]
		\centering
		\includegraphics[scale=0.5,keepaspectratio]{UML/Classi/Front-End/QuizziPedia_Front-end_Templates_ClickableAnswerTemplate.png}
		\caption{QuizziPedia::Front-End::ModelViews::SearchQuestionnaire}
	\end{figure} \FloatBarrier
	
	\begin{itemize}
		\item \textbf{Descrizione}: classe di tipo modelview la cui istanziazione è contenuta all'interno della variabile di ambiente \$scope di \textit{Angular.js\ped{G}}. All'interno di essa sono presenti le variabili e i metodi necessari per il \textit{Two-Way Data-Binding\ped{G}} tra la view \texttt{ResultsView} e il controller \texttt{SearchController};
		\item \textbf{Utilizzo}: viene utilizzata per effettuare il \textit{Two-Way Data-Binding\ped{G}} tra la view \texttt{ResultsView} e il controller \texttt{SearchController} rendendo disponibili variabili e metodi;
		\item \textbf{Relazioni con altre classi}: 
		\begin{itemize}
			\item \textit{IN} \texttt{ResultsView}: view contenente i risultati della ricerca effettuata, sia gli utenti che i que- stionari trovati; 
			\item \textit{IN} \texttt{SearchController}: questa classe permette di gestire la ricerca di questionari e utenti all’interno dell’applicazione;
		\end{itemize}
		\item \textbf{Attributi}: 
		\begin{itemize}
			\item ;
		\end{itemize}
		\item \textbf{Metodi}: 
		\begin{itemize}
			\item ;
		\end{itemize}
	\end{itemize}
	
	\paragraph{QuizziPedia::Front-End::ModelViews::SearchModelView}
	
	\label{QuizziPedia::Front-End::ModelViews::SearchModelView}
	
	\begin{figure}[ht]
		\centering
		\includegraphics[scale=0.5,keepaspectratio]{UML/Classi/Front-End/QuizziPedia_Front-end_Templates_ClickableAnswerTemplate.png}
		\caption{QuizziPedia::Front-End::ModelViews::SearchModelView}
	\end{figure} \FloatBarrier
	
	\begin{itemize}
		\item \textbf{Descrizione}: classe di tipo modelview la cui istanziazione è contenuta all'interno della variabile di ambiente \$scope di \textit{Angular.js\ped{G}}. All'interno di essa sono presenti le variabili e i metodi necessari per il \textit{Two-Way Data-Binding\ped{G}} tra la view \texttt{ResultsView} e il controller \texttt{SearchController};
		\item \textbf{Utilizzo}: viene utilizzata per effettuare il \textit{Two-Way Data-Binding\ped{G}} tra la view \texttt{ResultsView} e il controller \texttt{SearchController} rendendo disponibili variabili e metodi;
		\item \textbf{Relazioni con altre classi}: 
		\begin{itemize}
			\item \textit{IN} \texttt{ResultsView}: view contenente i risultati della ricerca effettuata, sia gli utenti che i que- stionari trovati; 
			\item \textit{IN} \texttt{SearchController}: questa classe permette di gestire la ricerca di questionari e utenti all’interno dell’applicazione;
		\end{itemize}
		\item \textbf{Attributi}: 
		\begin{itemize}
			\item ;
		\end{itemize}
		\item \textbf{Metodi}: 
		\begin{itemize}
			\item ;
		\end{itemize}
	\end{itemize}	


	\paragraph{QuizziPedia::Front-End::ModelViews::ProfileManagementModelView}
	
	\label{QuizziPedia::Front-End::ModelViews::ProfileManagementModelView}
	
	\begin{figure}[ht]
		\centering
		\includegraphics[scale=0.5,keepaspectratio]{UML/Classi/Front-End/QuizziPedia_Front-end_Templates_ClickableAnswerTemplate.png}
		\caption{QuizziPedia::Front-End::ModelViews::ProfileManagementModelView}
	\end{figure} \FloatBarrier
	
	\begin{itemize}
		\item \textbf{Descrizione}: classe di tipo modelview la cui istanziazione è contenuta all'interno della variabile di ambiente \$scope di \textit{Angular.js\ped{G}}. All'interno di essa sono presenti le variabili e i metodi necessari per il \textit{Two-Way Data-Binding\ped{G}} tra la view \texttt{ProfileManagementView} e il controller \texttt{ProfileManagementController};
		\item \textbf{Utilizzo}: viene utilizzata per effettuare il \textit{Two-Way Data-Binding\ped{G}} tra la view \texttt{ProfileManagementView} e il controller \texttt{ProfileManagementController} rendendo disponibili variabili e metodi;
		\item \textbf{Relazioni con altre classi}: 
		\begin{itemize}
			\item \textit{IN} \texttt{ProfileManagementView}: view contenente i dati personali che un utente può modificare dopo essersi registrato al sistema; 
			\item \textit{IN} \texttt{ProfileManagementController}: questa classe permette di gestire il profilo personale di un utente;
		\end{itemize}
		\item \textbf{Attributi}: 
		\begin{itemize}
			\item ;
		\end{itemize}
		\item \textbf{Metodi}: 
		\begin{itemize}
			\item ;
		\end{itemize}
	\end{itemize}	

\paragraph{QuizziPedia::Front-End::ModelViews::QuestionsManagementModelView}

\label{QuizziPedia::Front-End::ModelViews::QuestionsManagementModelView}

\begin{figure}[ht]
	\centering
	\includegraphics[scale=0.5,keepaspectratio]{UML/Classi/Front-End/QuizziPedia_Front-end_Templates_ClickableAnswerTemplate.png}
	\caption{QuizziPedia::Front-End::ModelViews::QuestionsManagementModelView}
\end{figure} \FloatBarrier

\begin{itemize}
	\item \textbf{Descrizione}: classe di tipo modelview la cui istanziazione è contenuta all'interno della variabile di ambiente \$scope di \textit{Angular.js\ped{G}}. All'interno di essa sono presenti le variabili e i metodi necessari per il \textit{Two-Way Data-Binding\ped{G}} tra la view \texttt{QuestionsManagementView} e il controller \texttt{QuestionsManagementController};
	\item \textbf{Utilizzo}: viene utilizzata per effettuare il \textit{Two-Way Data-Binding\ped{G}} tra la view \texttt{QuestionsManagementView} e il controller \texttt{QuestionsManagementController} rendendo disponibili variabili e metodi;
	\item \textbf{Relazioni con altre classi}: 
	\begin{itemize}
		\item \textit{IN} \texttt{QuestionsManagementView}: view contenente l’elenco delle domande create; 
		\item \textit{IN} \texttt{QuestionsManagementController}: questa classe permette di gestire le domande create dall’utente e di crearne di nuove;
	\end{itemize}
	\item \textbf{Attributi}: 
	\begin{itemize}
		\item ;
	\end{itemize}
	\item \textbf{Metodi}: 
	\begin{itemize}
		\item ;
	\end{itemize}
\end{itemize}	

\paragraph{QuizziPedia::Front-End::ModelViews::StatisticsModelView}

\label{QuizziPedia::Front-End::ModelViews::StatisticsModelView}

\begin{figure}[ht]
	\centering
	\includegraphics[scale=0.5,keepaspectratio]{UML/Classi/Front-End/QuizziPedia_Front-end_Templates_ClickableAnswerTemplate.png}
	\caption{QuizziPedia::Front-End::ModelViews::StatisticsModelView}
\end{figure} \FloatBarrier

\begin{itemize}
	\item \textbf{Descrizione}: classe di tipo modelview la cui istanziazione è contenuta all'interno della variabile di ambiente \$scope di \textit{Angular.js\ped{G}}. All'interno di essa sono presenti le variabili e i metodi necessari per il \textit{Two-Way Data-Binding\ped{G}} tra la view \texttt{StatisticsView} e il controller \texttt{StatisticsController};
	\item \textbf{Utilizzo}: viene utilizzata per effettuare il \textit{Two-Way Data-Binding\ped{G}} tra la view \texttt{StatisticsView} e il controller \texttt{StatisticsController} rendendo disponibili variabili e metodi;
	\item \textbf{Relazioni con altre classi}: 
	\begin{itemize}
		\item \textit{IN} \texttt{StatisticsView}: view che visualizza le statistiche di un utente; 
		\item \textit{IN} \texttt{StatisticsController}: questa classe permette di le statistiche di un utente;
	\end{itemize}
	\item \textbf{Attributi}: 
	\begin{itemize}
		\item ;
	\end{itemize}
	\item \textbf{Metodi}: 
	\begin{itemize}
		\item ;
	\end{itemize}
\end{itemize}	

\paragraph{QuizziPedia::Front-End::ModelViews::MenuBarModelView}

\label{QuizziPedia::Front-End::ModelViews::MenuBarModelView}

\begin{figure}[ht]
	\centering
	\includegraphics[scale=0.5,keepaspectratio]{UML/Classi/Front-End/QuizziPedia_Front-end_Templates_ClickableAnswerTemplate.png}
	\caption{QuizziPedia::Front-End::ModelViews::MenuBarModelView}
\end{figure} \FloatBarrier

\begin{itemize}
	\item \textbf{Descrizione}: classe di tipo modelview la cui istanziazione è contenuta all'interno della variabile di ambiente \$scope di \textit{Angular.js\ped{G}}. All'interno di essa sono presenti le variabili e i metodi necessari per il \textit{Two-Way Data-Binding\ped{G}} tra la view \texttt{Index} e il controller \texttt{MenuBarController};
	\item \textbf{Utilizzo}: viene utilizzata per effettuare il \textit{Two-Way Data-Binding\ped{G}} tra la view \texttt{Index} e il controller \texttt{MenuBarController} rendendo disponibili variabili e metodi;
	\item \textbf{Relazioni con altre classi}: 
	\begin{itemize}
		\item \textit{IN} \texttt{Index}: pagina alla base di tutte le view dell'applicazione; 
		\item \textit{IN} \texttt{MenuBarController}: questa classe permette di gestire il menù fisso per ogni pagina;
	\end{itemize}
	\item \textbf{Attributi}: 
	\begin{itemize}
		\item ;
	\end{itemize}
	\item \textbf{Metodi}: 
	\begin{itemize}
		\item ;
	\end{itemize}
\end{itemize}



%ModelView di tutte le tipologie di DOMANDE


