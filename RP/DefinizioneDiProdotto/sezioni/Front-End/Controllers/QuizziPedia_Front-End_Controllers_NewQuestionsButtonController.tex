\paragraph[QuizziPedia::Front-End::Controllers\\::NewQuestionsButtonController]{QuizziPedia::Front-End::Controllers::NewQuestionsButtonController}
\begin{figure} [ht]
	\centering
	\includegraphics[scale=0.8]{UML/Classi/Front-End/QuizziPedia_Front-end_Controller_NewQuestionsButtonController.png}
	\caption{QuizziPedia::Front-End::Controllers::NewQuestionsButtonController}
\end{figure} \FloatBarrier
\begin{itemize}
	\item \textbf{Descrizione}: questa classe permette di effettuare il redirect alla pagina di creazione nuova domanda;
	\item \textbf{Utilizzo}: effettua il redirect alla pagina di creazione di una nuova domanda quando l'utente seleziona interagisce con il bottone a cui è collegato il corrispettivo evento;
	\item \textbf{Relazione con altre classi}:
	\begin{itemize}
		\item \textbf{OUT \texttt{NewQuestionsButtonDirective}}: rappresenta il componente grafico che permette all'utente di posizionarsi nella \textit{view\ped{G}} di creazione di una nuova domanda.
	\end{itemize}
	\item \textbf{Attributi}:
	\begin{itemize}
		\item \texttt{-} \texttt{\$scope: \$scope} \\
		Campo dati contenente un riferimento all'oggetto \$scope creato da \textit{Angular\ped{G}}, viene utilizzato come mezzo di comunicazione tra il \textit{controller\ped{G}} e la \textit{view\ped{G}}. Contiene gli oggetti che definiscono il \textit{model\ped{G}} dell'applicazione;
		\item \texttt{-} \texttt{\$location: \$location} \\
		Campo dati contenente un riferimento al servizio creato da \textit{Angular\ped{G}} che permette di accedere alla barra degli indirizzi del \textit{browser\ped{G}}, i cambiamenti all'URL nella barra degli indirizzi si riflettono in questo oggetto e viceversa.
	\end{itemize}
	\item \textbf{Metodi}:
	\begin{itemize}
		\item \texttt{+} \texttt{NewQuestionButtonsController(\$scope: \$scope, \$location: \$location)} \\ 
		Metodo costruttore della classe. \\
		\textbf{Parametri}:
		\begin{itemize}
			\item \texttt{\$scope: \$scope} \\
			Parametro contenente un riferimento all'oggetto \$scope creato da \textit{Angular\ped{G}}. Viene utilizzato come mezzo di comunicazione tra il \textit{controller\ped{G}} e la \textit{view\ped{G}}. Contiene gli oggetti che definiscono il \textit{viewmodel\ped{G}} e il \textit{model\ped{G}} dell'applicazione;
			\item \texttt{\$location: \$location} \\
			Parametro contenente un riferimento al servizio creato da \textit{Angular\ped{G}} che permette di accedere alla barra degli indirizzi del \textit{browser\ped{G}}, i cambiamenti all'URL nella barra degli indirizzi si riflettono in questo oggetto e viceversa.
		\end{itemize}
		\item \texttt{+} \texttt{newQuestion(): void} \\ 
		Metodo che gestisce l'evento click sul pulsante per creare una nuova domanda. Effettua il redirect alla pagina di creazione di una domanda.
	\end{itemize}
	
\end{itemize}

