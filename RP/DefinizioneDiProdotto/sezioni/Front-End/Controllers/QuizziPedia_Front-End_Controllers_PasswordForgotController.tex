\paragraph{QuizziPedia::Front-End::Controllers::PasswordForgotController}
\begin{figure} [ht]
	\centering
	\includegraphics[scale=0.7]{UML/Classi/Front-End/QuizziPedia_Front-end_Controller_PasswordForgotController.png}
	\caption{QuizziPedia::Front-End::Controllers::PasswordForgotController}
\end{figure} \FloatBarrier
\begin{itemize}
	\item \textbf{Descrizione}: questa classe permette di gestire il ripristino della password dimenticata;
	\item \textbf{Utilizzo}: fornisce tutte le funzionalità per ripristinare la password dopo aver verificato l'identità dell'utente;
	\item \textbf{Relazione con altre classi}:
	\begin{itemize}
		\item \textbf{IN \texttt{PasswordForgotModelView}}: classe di tipo \textit{modelview\ped{G}} la cui istanziazione è contenuta all'interno della variabile di ambiente \$scope di \textit{Angular\ped{G}}. All'interno di essa sono presenti le variabili e i metodi necessari per il \textit{Two-Way Data-Binding\ped{G}} tra la \textit{view\ped{G}} \texttt{PasswordForgotView} e il \textit{controller\ped{G}} \texttt{PasswordForgotController};
		\item \textbf{OUT \texttt{AuthService}}: questa classe permette di gestire la registrazione e l'autenticazione di un utente.
	\end{itemize}
	\item \textbf{Attributi}:
	\begin{itemize}
		\item \texttt{-} \texttt{\$scope: \$scope} \\
		Campo dati contenente un riferimento all'oggetto \$scope creato da \textit{Angular\ped{G}}, viene utilizzato come mezzo di comunicazione tra il \textit{controller\ped{G}} e la \textit{view\ped{G}}. Contiene gli oggetti che definiscono il \textit{model\ped{G}} dell'applicazione;
		\item \texttt{-} \texttt{\$location: \$location} \\
		Campo dati contenente un riferimento al servizio creato da \textit{Angular\ped{G}} che permette di accedere alla barra degli indirizzi del \textit{browser\ped{G}}, i cambiamenti all'URL nella barra degli indirizzi si riflettono in questo oggetto e viceversa;
		\item \texttt{-} \texttt{\$mdDialog: \$mdDialog} \\
		Campo dati contenente un riferimento al servizio della libreria \textit{Material for Angular\ped{G}} che permette di creare delle componenti a pop-up;
		\item \texttt{-} \texttt{AuthService: AuthService} \\
		Campo dati contenente un riferimento al servizio che si occupa della gestione delle informazioni legate all'autenticazione. Viene utilizzato il metodo \texttt{passwordForgot} di \texttt{AuthService} a cui viene passato il parametro \texttt{email}.
	\end{itemize}
	\item \textbf{Metodi}:
	\begin{itemize}
		\item \texttt{+} \texttt{PasswordForgotController(\$scope: \$scope, \$location: \$location, \$mdDialog: \$mdDialog, AuthService: AuthService)} \\
		Metodo costruttore della classe. \\
		\textbf{Parametri}:
		\begin{itemize}
			\item \texttt{\$scope: \$scope} \\
			Parametro contenente un riferimento all'oggetto \$scope creato da \textit{Angular\ped{G}}. Viene utilizzato come mezzo di comunicazione tra il \textit{controller\ped{G}} e la \textit{view\ped{G}}. Contiene gli oggetti che definiscono il \textit{viewmodel\ped{G}} e il \textit{model\ped{G}} dell'applicazione;
			\item \texttt{\$location: \$location} \\
			Parametro contenente un riferimento al servizio creato da \textit{Angular\ped{G}} che permette di accedere alla barra degli indirizzi del \textit{browser\ped{G}}, i cambiamenti all'URL nella barra degli indirizzi si riflettono in questo oggetto e viceversa;
			\item \texttt{\$mdDialog: \$mdDialog} \\
			Parametro contenente un riferimento al servizio della libreria \textit{Material for Angular\ped{G}} che permette di creare delle componenti a pop-up;
			\item \texttt{AuthService: AuthService} \\
			Campo dati contenente un riferimento al servizio che si occupa della gestione delle informazioni legate all'autenticazione. Viene utilizzato il metodo \texttt{passwordForgot} di \texttt{AuthService} a cui viene passato il parametro \texttt{email}.
		\end{itemize}
		\item \texttt{+} \texttt{passwordForgot(): void} \\
		Metodo che richiama il metodo \texttt{getNewPassword} del \textit{service\ped{G}} \texttt{AuthService} passandogli il parametro \texttt{email}. Nel caso di buona riuscita dell'operazione, viene mostrato un messaggio di successo il cui corpo contiene anche un bottone per effettuare il redirect alla pagina di login. Nel caso in cui invece avvenga un errore, viene mostrato a video il messaggio di errore;
		\item \texttt{+} \texttt{logIn(): void} \\
		Metodo che gestisce l'evento click sul pulsante di login. Effettua il redirect alla pagina di login.
	\end{itemize}
\end{itemize}

