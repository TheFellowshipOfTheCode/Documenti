\section{Consuntivo}
In questa sezione viene presentato il bilancio tra il preventivo e il consuntivo. Tale bilancio può essere:
\begin{itemize}
	\item \textbf{Positivo}: il preventivo ha superato il consuntivo;
	\item \textbf{Negativo}: il consuntivo ha superato il preventivo;
	\item \textbf{In pari}: il consuntivo coincide con il preventivo.
\end{itemize}

\subsection{Analisi dei Requisiti}
Di seguito verrà presentato il consuntivo per l'attività di \textit{\AdR}.
\\\\
La tabella sottostante riporta le ore preventivate e, tra parentesi, la differenza di tali ore con quelle effettivamente impiegate per ciascun componente del gruppo \gruppo.

\begin{table}[H]
	\begin{center}
		\begin{tabular}{|c|c|c|c|c|c|c|c|}
			\hline
			\textbf{Nominativo} & \multicolumn{6}{c|}{\textbf{Ore per ruolo}} & \textbf{Ore totali} \\
			& \textbf{Re} & \textbf{Am} & \textbf{An} & \textbf{Pj} & \textbf{Pr} & \textbf{Ve} & \\
			\hline	
			\FB		&			&	4 (+1)	&	23		&		&		&		&	27 (+1)	\\
			\hline
			\AF		&			&	6 		&	21		&	 	&		&		& 	27		\\
			\hline
			\GN		&	20		&			&	9 (+2)	&		&		&		&	29		\\
			\hline
			\GR		&	20 (+1)	&	 		&	8 		&		&	 	& 		&	28		\\
			\hline
			\SM 	&			&	3		&	5		&		&		& 	20	&	28		\\
			\hline
			\MP		& 			&			&	7		&		&		&	20	&	27 (+1)	\\
			\hline
			\MV 	&			&	4		&	5		&		&		&	20	& 	29		\\
			\hline
		\end{tabular}
	\end{center}
	\caption{Differenza preventivo-consuntivo per componente, \AdR}
\end{table}

La tabella sottostante, invece, riporta le ore preventivate e  tra parentesi la differenza di ore tra preventivo e consuntivo, divise per ruolo.

\begin{table}[H]
	\begin{center}
		\begin{tabular}{|l|c|c|}
			\hline
			\textbf{Ruolo}	& \textbf{Ore} & \textbf{Costo} \\
			\hline
			\textit{Responsabile}		&	40 (+1)	&	1200 (+30)	\\
			\hline
			\textit{Amministratore}		&	17 (+1)	&	340 (+20)	\\
			\hline
			\textit{Analista}			&	78 (+2)	&	1950 (+50)	\\
			\hline
			\textit{Verificatore}		&	60 (+1)	&	900 (+15)	\\
			\hline
			\textbf{Totale preventivo}	&	195		&	4390 		\\
			\hline
			\textbf{Totale consuntivo}	&	200		&   4505		\\
			\hline
			\textbf{Differenza} 		&	-5		&	-115		\\
			\hline
		\end{tabular}
	\end{center}
	\caption{Differenza preventivo-consuntivo per ruolo, \AdR}
\end{table}
 
\subsubsection{Conclusioni}

Come mostrato dalla tabella, risulta che il gruppo ha utilizzato complessivamente cinque ore un più rispetto alla pianificazione nel periodo di \textit{Analisi dei Requisiti}, con un bilancio in passivo pari a 115€. Tale passivo non andrà influenzare il costo totale del progetto in quanto le ore impiegate in questo periodo non vengono poste a carico del proponente.

\subsection{Analisi dei Requisiti in Dettaglio}
Di seguito verrà presentato il consuntivo per l'attività di \textit{\AD}.
\\\\
La tabella sottostante riporta le ore preventivate e, tra parentesi, la differenza di tali ore con quelle effettivamente impiegate per ciascun componente del gruppo \gruppo.

\begin{table}[H]
	\begin{center}
		\begin{tabular}{|c|c|c|c|c|c|c|c|}
			\hline
			\textbf{Nominativo} & \multicolumn{6}{c|}{\textbf{Ore per ruolo}} & \textbf{Ore totali} \\
			& \textbf{Re} & \textbf{Am} & \textbf{An} & \textbf{Pj} & \textbf{Pr} & \textbf{Ve} & \\
			\hline
			\FB			&			&		&	3 (-2)	&		&		&			&	3 (-2)	\\
			\hline
			\AF			&			&		&	3 (-1)	&	 	&		&			& 	3 (-1)		\\
			\hline
			\GN			&	1 		&		&			&		&		&			&	1		\\
			\hline
			\GR			&	2 (-2)	&	 	&	 		&		&	 	& 			&	2 (-2)	\\
			\hline
			\SM 		&			&	1	&			&		&		& 	1		&	2		\\
			\hline
			\MP			& 			&		&			&		&		&	3 (-2)	&	3 (-2)	\\
			\hline
			\MV 		&			&		&			&		&		&	1		& 	1		\\
			\hline
		\end{tabular}
	\end{center}
	\caption{Differenza preventivo-consuntivo per componente, \AD}
\end{table}

La tabella sottostante, invece, riporta le ore preventivate e  tra parentesi la differenza di ore tra preventivo e consuntivo, divise per ruolo.

\begin{table}[H]
	\begin{center}
		\begin{tabular}{|l|c|c|}
			\hline
			\textbf{Ruolo}	& \textbf{Ore} & \textbf{Costo} \\
			\hline
			\textit{Responsabile}		&	3 (-2)	&	90 (-60) 	\\
			\hline
			\textit{Amministratore}		&	1		&	20 			\\
			\hline
			\textit{Analista}			&	6 (-3)	&	150 (-75) 	\\
			\hline
			\textit{Verificatore}		&	5 (-2)	&	75 (-30)	\\
			\hline
			\textbf{Totale preventivo}	&	15		& 	335			\\
			\hline
			\textbf{Totale consuntivo}	&	8		&  	170			\\
			\hline
			\textbf{Differenza} 		&	7		&	165			\\
			\hline
		\end{tabular}
	\end{center}
	\caption{Differenza preventivo-consuntivo per ruolo, \AD}
\end{table}

\subsubsection{Conclusioni}
Come mostrato dalla tabella, risulta che il gruppo ha utilizzato complessivamente 7 ore in meno rispetto la pianificazione dell'attività di \textit{\AD}, con un bilancio in attivo pari a 165€, che sommato al bilancio dell'attività di \textit{\AR} risulta un attivo di 50€.

\subsection{Progettazione Architetturale}

Di seguito verrà presentato il consuntivo per l'attività di \textit{\PA}.
\\\\
La tabella sottostante riporta le ore preventivate e, tra parentesi, la differenza di tali ore con quelle effettivamente impiegate per ciascun componente del gruppo \gruppo.

\begin{table}[H]
	\begin{center}
		\begin{tabular}{|c|c|c|c|c|c|c|c|}
			\hline
			\textbf{Nominativo} & \multicolumn{6}{c|}{\textbf{Ore per ruolo}} & \textbf{Ore totali} \\
			& \textbf{Re} & \textbf{Am} & \textbf{An} & \textbf{Pj} & \textbf{Pr} & \textbf{Ve} & \\
			\hline
			\FB			&		&		&		&	10(-3)	&		&	23	&	33(-3)	\\
			\hline
			\AF			&		&		&		&	9(+2) 	&		&	23	& 	32(+2)	\\
			\hline
			\GN			&		&		&		&	8	&		&	24	&	32	\\
			\hline
			\GR			&		&	4	&  		&	28	&	 	& 		&	32	\\
			\hline
			\SM 		&	3(-1)	&		&		&	29	&		& 		&	32(-1)	\\
			\hline
			\MP 		& 		&	4	&		&	28	&		&		&	32	\\
			\hline
			\MV 		&	3	&		&		&	29	&		&		& 	32	\\
			\hline
		\end{tabular}
	\end{center}
	\caption{Differenza preventivo-consuntivo per componente, Progettazione Architetturale}
\end{table}

La tabella sottostante, invece, riporta le ore preventivate e  tra parentesi la differenza di ore tra preventivo e consuntivo, divise per ruolo.

\begin{table}[H]
	\begin{center}
		\begin{tabular}{|l|c|c|}
			\hline
			\textbf{Ruolo}	& \textbf{Ore} & \textbf{Costo} \\
			\hline
			\textit{Responsabile}		&	6 (-1)	&	180 (-30) 	\\
			\hline
			\textit{Amministratore}		&	8		&	160			\\
			\hline
			\textit{Progettista}		&	141 (-1)&	3102 (-22) 	\\
			\hline
			\textit{Verificatore}		&	70 		&	1050 		\\
			\hline
			\textbf{Totale preventivo}	&	225		& 	4492		\\
			\hline
			\textbf{Totale consuntivo}	&	222		&  	4440		\\
			\hline
			\textbf{Differenza} 		&	2		&	52			\\
			\hline
		\end{tabular}
	\end{center}
	\caption{Differenza preventivo-consuntivo per ruolo, \PA}
\end{table}

\subsubsection{Conclusioni}

Come mostrato dalla tabella, risulta che il gruppo ha utilizzato complessivamente 2 ore in meno rispetto la pianificazione dell'attività di \textit{\PA}, con un bilancio in attivo pari a 52€, che sommato al bilancio all'attività di \textit{\AR} risulta un attivo di 102€.

\subsection{Progettazione di Dettaglio}
Di seguito verrà presentato il consuntivo per l'attività di \textit{\PD}.
\\\\
La tabella sottostante riporta le ore preventivate e, tra parentesi, la differenza di tali ore con quelle effettivamente impiegate per ciascun componente del gruppo \gruppo.

\begin{table}[H]
	\begin{center}
		\begin{tabular}{|c|c|c|c|c|c|c|c|}
			\hline
			\textbf{Nominativo} & \multicolumn{6}{c|}{\textbf{Ore per ruolo}} & \textbf{Ore totali} \\
			& \textbf{Re} & \textbf{Am} & \textbf{An} & \textbf{Pj} & \textbf{Pr} & \textbf{Ve} & \\
			\hline
			\FB			&	3		&			&		&	17 (+3)	&		&		&	20 (+3)		\\
			\hline
			\AF			&	4 (-2)	&			&		&	16 		&		&		& 	20 (-2)		\\
			\hline
			\GN			&			&	2		&		&	18		&		&		&	20			\\
			\hline
			\GR			&			&	 		&		&	6		&	 	& 	14	&	20			\\
			\hline
			\SM 		&			&	3 (+1)	&		&	17		&		& 		&	20 (+1)		\\
			\hline
			\MP 		& 			&			&		&	6		&		&	14	&	20			\\
			\hline
			\MV 		&			&			&		&	6		&		&	14	& 	20			\\
			\hline
		\end{tabular}
	\end{center}
	\caption{Differenza preventivo-consuntivo per componente, Progettazione di Dettaglio}
\end{table}


La tabella sottostante, invece, riporta le ore preventivate e  tra parentesi la differenza di ore tra preventivo e consuntivo, divise per ruolo.

\begin{table}[H]
	\begin{center}
		\begin{tabular}{|l|c|c|}
			\hline
			\textbf{Ruolo}	& \textbf{Ore} & \textbf{Costo} \\
			\hline
			\textit{Responsabile}		&	7 (-2)	&	210 (-60) 		\\
			\hline
			\textit{Amministratore}		&	5 (+1)	&	100	(+20)		\\
			\hline
			\textit{Progettista}		&	86 (+3)	&	1892 (+66) 		\\
			\hline
			\textit{Verificatore}		&	42 		&	630 			\\
			\hline
			\textbf{Totale preventivo}	&	140		& 	2832			\\
			\hline
			\textbf{Totale consuntivo}	&	142		&  	2858			\\
			\hline
			\textbf{Differenza} 		&	-2		&	-26				\\
			\hline
		\end{tabular}
	\end{center}
	\caption{Differenza preventivo-consuntivo per ruolo, \PD}
\end{table}

\subsubsection{Conclusioni}
Come mostrato dalla tabella, risulta che il gruppo ha utilizzato complessivamente 2 ore in più rispetto la pianificazione dell'attività di \textit{\PD}, con un bilancio in passivo pari a 26€, che sommato al bilancio all'attività di \textit{\PA} risulta un attivo di 76€.
