\newpage
\subsection{Caso d'uso UC6: Ricerca questionario}
\label{UC6}
\begin{itemize}
\item\textbf{Attori}: utente autenticato, utente autenticato pro;
\item\textbf{Descrizione}: nella schermata principale l'attore può ricercare un questionario, a cui si vuole iscrivere, attraverso la barra di ricerca. Le chiavi di ricerca sono l'autore e il titolo del questionario;	
\item\textbf{Precondizione}: l'attore si trova nella pagina principale dell'applicazione;
\item\textbf{Postcondizione}: l'attore visualizza i questionari che contengono, nel titolo o nell'username dell'autore, il testo scritto nella barra di ricerca;
\item\textbf{Scenario principale}:
l'attore utilizza la barra di ricerca per cercare un questionario;
\end{itemize}