\subsection{M}
\begin{itemize} 
	\item
	\textbf{Mac OS}: è il sistema operativo di Apple dedicato ai computer Macintosh; il nome è l'acronimo di Macintosh Operating System.
	\item
	\textbf{Material for Angular}: per gli sviluppatori Angular, è sia un framework per le componenti UI che un'implementazione di riferimento di Google's Material Design Specification. Questo progetto fornisce un insieme di componenti dell'interfaccia utente riutilizzabili, ben collaudato, e un insieme di componenti UI basati sul Material Design.
	\item
	\textbf{Microsoft Edge}: è un browser web sviluppato da Microsoft e incluso in Windows 10. Ufficialmente presentato il 21 gennaio 2015, ha sostituito Internet Explorer come browser predefinito di Windows. Sarà anche il browser predefinito per Windows 10 Mobile, la versione per smartphone e tablet.
	\item
	\textbf{Microsoft Windows}: è una famiglia di ambienti operativi e sistemi operativi dedicati ai personal computer, alle workstation, ai server e agli smartphone. Il sistema operativo si chiama così per via della sua interfaccia a finestre.
	\'E software proprietario della Microsoft Corporation che lo rende disponibile esclusivamente a pagamento.
	\item
	\textbf{Middleware}: si intende un insieme di programmi informatici che fungono da intermediari tra diverse applicazioni e componenti software. Sono spesso utilizzati come supporto per sistemi distribuiti complessi.
	\item
	\textbf{Model}: il model fornisce i metodi per accedere ai dati utili all'applicazione;
	\item
	\textbf{MongoDB}: è un DBMS non relazionale, orientato ai documenti. Classificato come un database di tipo NoSQL, MongoDB si allontana dalla struttura tradizionale basata su tabelle dei database relazionali in favore di documenti in stile JSON con schema dinamico (BSON), rendendo l'integrazione di dati di alcuni tipi di applicazioni più facile e veloce. 
	\item
	\textbf{Mongoose}: è un web server multipiattaforma. Mongoose viene distribuito sotto licenza commerciale e GPLv2. 
	\item 
	\textbf{Mozilla Firefox}: è un web browser opensource multipiattaforma prodotto da Mozilla Foundation. \'E disponibile per sistemi operativi Windows, Linux, Mac OS X, Android e Firefox OS.
	\item
	\textbf{MVC}: è un pattern architetturale molto diffuso nello sviluppo di sistemi software, in particolare nell'ambito della programmazione orientata agli oggetti, in grado di separare la logica di presentazione dei dati dalla logica di business. 
\end{itemize}
