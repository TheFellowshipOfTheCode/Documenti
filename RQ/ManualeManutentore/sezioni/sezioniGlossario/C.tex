\subsection{C}
\begin{itemize}
	\item
	\textbf{Chain of responsibility}: design pattern che permette di separare gli oggetti che invocano richieste dagli oggetti che le gestiscono, dando ad ognuno la possibilità di gestire queste richieste. Viene utilizzato il termine catena perché di fatto la richiesta viene inviata e "segue la catena" di oggetti, passando da uno all'altro, finché non trova quello che la gestisce.
	\item
	\textbf{Client}: indica una componente che accede ai servizi o alle risorse di un'altra componente detta server. In questo contesto si può quindi parlare di client riferendosi all'hardware oppure al software. Esso fa parte dunque dell'architettura logica di rete detta client-server.
	Il termine client indica anche il software usato sul computer-client per accedere alle funzionalità offerte da un server.
	\item
	\textbf{ConcreteHandler}:  componente del design pattern Chain of Responsibility che rappresenta l'effettiva implementazione della gestione degli eventi per un oggetto.
	\item
	\textbf{Controller}: riceve i comandi dell'utente (in genere attraverso una view) e li attua modificando lo stato degli altri due componenti (model e view) secondo il design pattern MVC. 
	\item
	\textbf{CSS}: è un linguaggio usato per definire la formattazione di documenti HTML, XHTML e XML ad esempio i siti web e relative pagine web. Le regole per comporre il CSS sono contenute in un insieme di direttive (Recommendations) emanate a partire dal 1996 dal W3C.
	L'introduzione del CSS si è resa necessaria per separare i contenuti delle pagine HTML dalla loro formattazione e permettere una programmazione più chiara e facile da utilizzare, sia per gli autori delle pagine stesse sia per gli utenti, garantendo contemporaneamente anche il riutilizzo di codice ed una sua più facile manutenzione. 
	\item
	\textbf{CSS3}: è un linguaggio utilizzato per definire la formattazione di documenti HTML, XHTML e XML.
	Questo linguaggio istruisce un browser su come il documento debba essere presentato all'utente, per esempio definendone la formattazione del testo, il posizionamento degli elementi rispetto a diversi media e device eccetera.
\end{itemize}