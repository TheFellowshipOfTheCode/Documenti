\newpage
\section{Introduzione}
\subsection{Scopo del documento}
Il documento rappresenta il manuale del manutentore per il software \progetto{} sviluppato dal gruppo \gruppo.
\subsection{Scopo del Prodotto}
Lo scopo del prodotto è di permettere la creazione e gestione di questionari in grado di identificare le lacune dei candidati prima, durante e al termine di un corso di formazione. 
\\Il sistema dovrà offrire le seguenti funzionalità:
\begin{itemize}
	\item
	Archiviare questionari in un server suddivisi per argomento;
	\item
	Somministrare all'utente, tramite un'interfaccia, questionari specifici per argomento scelto;
	\item
	Verificare e valutare i questionari scelti dagli utenti in base alle risposte date.
\end{itemize}
La parte destinata ai creatori di questionari dovrà essere fruibile attraverso un \textit{browser\ped{G}} desktop, abilitato all'utilizzo delle tecnologie \textit{HTML5\ped{G}}, \textit{CSS3\ped{G}} e \textit{JavaScript\ped{G}}. La parte destinata agli esaminandi sarà utilizzabile su qualunque dispositivo: dal personal computer ai tablet e smartphone.
\subsection{Glossario}
Al fine di evitare ogni ambiguità i termini tecnici del dominio del progetto, gli acronimi e le parole che necessitano di ulteriori spiegazioni saranno marcate con il pedice \ped{G} e quindi presenti nella sezione Glossario.
\subsection{Riferimenti}
\subsubsection{Normativi}
\subsubsection{Informativi}