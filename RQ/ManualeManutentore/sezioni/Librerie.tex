\newpage
\section{Librerie e Frameworks utilizzati}
\subsection{Front-End}
Nella parte Front-End dell'applicazione sono state utilizzate le seguenti librerie e \textit{frameworks\ped{G}}:
\begin{itemize}
	\item \texttt{MaterialAngular.js}: \textit{framework\ped{G}} per lo sviluppo dei componenti grafici dell'applicazione;
	\item \texttt{Charts.js}: libreria per lo sviluppo dei grafici per la visualizzazione delle statistiche
	utente;
	\item \texttt{Angles.js}: libreria necessaria per integrare la libreria Charts.js all'interno
	dell'ambiente \textit{Angular\ped{G}};
	\item \texttt{TextAngular.js}: libreria per la creazione di un editor di testo all'interno delle pagine web per permettere all'utente di creare domande custom in linguaggio \textit{QML\ped{G}}.
\end{itemize}
\subsection{Back-End}
Nella parte Back-End dell'applicazione sono state utilizzate le seguenti librerie e \textit{frameworks\ped{G}}:
\begin{itemize}
	\item \texttt{Express}: \textit{framework\ped{G}} Web di routing e \textit{middleware\ped{G}}, con funzionalità sua propria minima: un'applicazione Express è essenzialmente una serie di chiamate a funzioni \textit{middleware\ped{G}};
	\item \texttt{Passport}: \textit{middleware\ped{G}} per l'autenticazione in ambiente \textit{Node.js\ped{G}}.
\end{itemize}
