\newpage
\section{Installazione e utilizzo}
\subsection{Installazione}
Per l'utilizzo del software è necessaria l'installazione della piattaforma \textit{Node.js\ped{G}}, del Node \textit{package\ped{G}} manager \textit{npm\ped{G}}, di \textit{MongoDB\ped{G}} (in alternativa si può usare un servizio come MongoLab) e del il sistema software di controllo di versione distribuito \textit{Git\ped{G}}. 
Inoltre nell'ambiente di lavoro in cui verrà eseguito QuizziPedia dovrà essere installato Python 2.7 per il corretto funzionamento dei test. Per impostare l'ambiente di lavoro è necessario eseguire il comando da terminale:\\
\\
\centerline{\texttt{npm install -g node-gyp}}\\
\\
e installare, dipendentemente dal sistema operativo in uso, i seguenti pacchetti:
\begin{itemize}
	\item \textbf{UNIX}
	\begin{itemize}
		\item Python (v2.7 raccomandata, v3.x.x non è supportata);
		\item \texttt{make};
		\item Un compilatore C/C++, come GCC.
	\end{itemize}
	\item \textbf{Mac OS X}
	\begin{itemize}
		\item Python (v2.7 raccomandata, v3.x.x non è supportata);
		\item Xcode: è necessario installare il Command Line Tools tramite Xcode. Lo si può trovare nel menu Xcode -> Preferences -> Downloads;
		\item Questo passo installerà gcc e il relativo pacchetto di programmi contiene il \texttt{make}.
	\end{itemize}
	\item \textbf{Windows 10}
	\begin{itemize}
		\item Python (v2.7.10 raccomandato, v3.x.x non è supportata): assicurarsi che la variabile d'ambiente PYTHON abbia il valore: \verb|\path\to\python.exe|;
		\item Installare l'ultima versione di \textit{npm\ped{G}};
		\item Installare Visual Studio Community 2015 Edition;
		\item Impostare la variabile d'ambiente \verb|GYP_MSVS_VERSION=2015|;
		\item Avviare il prompt dei comandi come Amministratore ed eseguire il comando: \texttt{npm install};
		\item Per i sistemi a 64-bit è necessario anche Windows 7 64-bit SDK;
		\item Potrebbe essere necessario eseguire uno dei seguenti comandi se WindowsSDKDir lamenta di non essere impostato:
	\end{itemize}
\end{itemize}
	\begin{center}
		\verb|call "C:\Program Files\Microsoft SDKs\Windows\v7.1\bin\Setenv.cmd"/Release/x86|\\
		\verb|call "C:\Program Files\Microsoft SDKs\Windows\v7.1\bin\Setenv.cmd"/Release/x64|\\
	\end{center}
La guida originale è visibile sul link presentato all'interno della sezione Riferimenti -> Informativi.
Successivamente da terminale eseguire il comando:\\
\\
\centerline{\texttt{npm install -g bower}}\\
\\
all'interno della cartella QuizziPedia. Infine eseguire sempre da terminale, dentro alla cartella QuizziPedia, i comandi:\\
\\
\centerline{\texttt{npm install}}\\
\\
e
\\
\centerline{\texttt{npm start}}\\
\\
A questo punto il software è avviabile sulla porta \textit{localhost\ped{G}} indicata dal terminale.

\subsection{Modalità di utilizzo}
Vi sono tre modalità di utilizzo del software, ognuna delle quali ha differenti privilegi:
\begin{itemize}
	\item \textbf{Utente non autenticato}: non è necessaria l'iscrizione all'applicazione ed è possibile solo effettuare la modalità allenamento e ricercare utenti e questionari;
	\item \textbf{Utente autenticato}: è necessaria l'iscrizione attraverso l'apposita funzionalità visibile nell'home page, compilandone tutti i campi dati. Una volta effettuata l'autenticazione, è possibile accedere a tutte le funzionalità che l'applicazione offre tranne la creazione di questionari;
	\item \textbf{Utente autenticato pro}: il software permette di effettuare un upgrade del proprio account all'interno della funzionalità Gestione profilo. Un utente autenticato pro può accedere ad ogni funzionalità dell'applicazione. 
\end{itemize}

