\newpage
\section{Installazione e utilizzo}
Per l'utilizzo del software è necessaria l'installazione della piattaforma \textit{Node.js\ped{G}}, del Node package manager \textit{npm\ped{G}} e del il sistema software di controllo di versione distribuito \textit{Git\ped{G}}. Successivamente da terminale eseguire il comando:\\
\\
\centerline{\texttt{npm install -g bower}}\\
\\
all'interno della cartella QuizziPedia. Infine eseguire sempre da terminale, dentro alla cartella QuizziPedia, i comandi:\\
\\
\centerline{\texttt{npm install}}\\
\\
e
\\
\centerline{\texttt{npm start}}\\
\\
A questo punto il software è avviabile sulla porta \textit{localhost\ped{G}} indicata dal terminale.
\\
Vi sono tre modalità di utilizzo del software, ognuna delle quali ha differenti privilegi:
\begin{itemize}
	\item \textbf{Utente non autenticato}: non è necessaria l'iscrizione all'applicazione ed è possibile solo effettuare la modalità allenamento e ricercare utenti e questionari;
	\item \textbf{Utente autenticato}: è necessaria l'iscrizione attraverso l'apposita funzionalità visibile nell'home page, compilandone tutti i campi dati. Una volta effettuata l'autenticazione, è possibile accedere a tutte le funzionalità che l'applicazione offre tranne la creazione di questionari;
	\item \textbf{Utente autenticato pro}: il software permette di effettuare un upgrade del proprio account all'interno della funzionalità Gestione profilo. Un utente autenticato pro può accedere ad ogni funzionalità dell'applicazione. 
\end{itemize}

