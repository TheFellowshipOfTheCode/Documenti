\section{Aggiunta di nuove funzionalità}
Il software QuizziPedia è stato progettato e sviluppato seguendo il principio di estendibilità del codice. Pertanto è possibile aggiungere nuove classi, sia nella parte Front-End che Back-End, per lo sviluppo di nuove funzionalità.

\subsection{Front-End}

\subsubsection{Inserimento nuova view}
Per inserire una nuova view è necessario creare un file .html all'interno della cartella View del path: QuizziPedia/Front-End/View. Se si decide di utilizzare la stessa metodologia dell'intero progetto per la visualizzazione della giusta traduzione delle parole chiave è indispensabile che il controller associato alla view sia \texttt{AppController} e che le keywords che saranno visualizzate a video siano del tipo: \texttt{$listOfKey.nomevariabile$}. Infine è necessario aggiornare il file \texttt{AppRouter} inserendo il codice:

\begin{lstlisting}[language=Java,firstnumber=1]
.when('/:lang/nuovaview', {
templateUrl: '/Views/NuovaView.html',
controller:"NuovaViewController",
css: [
{
href: 'css/auth-main.css'
},
{
href: 'css/auth-medium.css',
media: 'handheld, screen and (max-width:960px), only screen and (max-device-width:960px)'
},
{
href: 'css/auth-small.css',
media: 'handheld, screen and (max-width:480px), only screen and (max-device-width:480px)'
}
]
})
\end{lstlisting}

\subsubsection{Inserimento nuova directive}

\subsubsection{Inserimento nuovo controller}  
Per inserire un nuovo controller è necessario creare un file .js all'interno della cartella Controller del path: QuizziPedia/Front-End/Controller. Infine si deve aggiungere all'interno del file \texttt{Index}, presente nella cartella QuizziPedia/Front-End, la seguente linea di codice nella sezione Controllers:
\begin{lstlisting}[language=Java,firstnumber=1]
	<script src="Controllers/NuovoController.js"></script>
\end{lstlisting}

\subsubsection{Inserimento nuovo service}

\subsubsection{Inserimento nuovo model}

\subsection{Back-End}

nuovo controller dentro Controller -> creare un router e associarlo al controller: var QuizController = require('../Controller/QuizController.js');
nuovo model dentro a Model -> associare il model appena creato al relativo controller