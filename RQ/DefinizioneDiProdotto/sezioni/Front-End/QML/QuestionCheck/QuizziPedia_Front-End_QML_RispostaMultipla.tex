\paragraph[QuizziPedia::Front-End::QML:: \\ QuestionCheck::RispostaMultipla]{QuizziPedia::Front-End::QML::QuestionCheck::RispostaMultipla}
\begin{figure} [ht]
	\centering
	\includegraphics[scale=0.32]{UML/Classi/Front-End/QuizziPedia_Front-end_QML_QuestionCheck_RispostaMultipla.png}
	\caption{QuizziPedia::Front-End::QML::QuestionCheck::RispostaMultipla}
\end{figure} \FloatBarrier
\begin{itemize}
	\item \textbf{Descrizione}: questa classe implementa il metodo che permette di controllare la validità semantica del codice \textit{QML\ped{G}} relativo alla tipologia di domanda "RispostaMultipla";
	\item \textbf{Utilizzo}: usata per controllare la validità semantica del codice \textit{QML\ped{G}} nello specifico caso di tipologia "RispostaMultipla";
	\item \textbf{Relazioni con altre classi}:
	\begin{itemize}
		\item \textbf{IN} \texttt{CheckQML}: classe che fornisce le funzionalità per controllare la validità semantica del codice \textit{QML\ped{G}}.
	\end{itemize}
	\item \textbf{Metodi}:
	\begin{itemize}
		\item \texttt{+} \texttt{rispostaMultipla(req : JSON, res : JSON)} \\
		Questo metodo permette di controllare la semantica del codice \textit{QML\ped{G}}, restituisce un JSON con la domanda valida oppure un JSON contenente un messaggio d'errore. \\
		\textbf{Parametri}:
		\begin{itemize}
			\item \texttt{req : JSON} \\
			Parametro contenente il codice QML scritto dall'utente nell'editor di testo e sintatticamente valido;
			\item \texttt{res : JSON} \\
			Parametro che contiene il JSON specifico per la domanda da creare o un JSON con un messaggio d'errore;
		\end{itemize}
	\end{itemize}
\end{itemize}

