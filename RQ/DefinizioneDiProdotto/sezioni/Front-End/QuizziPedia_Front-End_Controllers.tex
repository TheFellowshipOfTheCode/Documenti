\newcommand{\scopeA}{\texttt{-} \texttt{\$scope: \$scope} \\
	Campo dati contenente un riferimento all'oggetto \$scope creato da \textit{Angular\ped{G}}, viene utilizzato come mezzo di comunicazione tra il \textit{controller\ped{G}} e la \textit{view\ped{G}}. Contiene gli oggetti che definiscono il \textit{model\ped{G}} dell’applicazione}
\newcommand{\scopeP}{\texttt{\$scope: \$scope} \\
	Parametro contenente un riferimento all’oggetto \$scope creato da \textit{Angular\ped{G}}. Viene utilizzato come mezzo di comunicazione tra il \textit{controller\ped{G}} e la \textit{view\ped{G}}. Contiene gli oggetti che definiscono il viewmodel e il \textit{model\ped{G}} dell'applicazione}
\newcommand{\rootscopeA}{\texttt{-} \texttt{\$rootScope: \$rootScope} \\
	Campo dati contenente il riferimento all'oggetto globale \$rootScope creato da \textit{Angular\ped{G}}}
\newcommand{\rootscopeP}{\texttt{\$rootScope: \$rootScope} \\
	Parametro contenente il riferimento all'oggetto globale \$rootScope creato da \textit{Angular\ped{G}}}
\newcommand{\errorinfomodelA}{\texttt{-} \texttt{ErrorInfoModel: ErrorInfoModel} \\
Campo dati contenente il riferimento alla classe che rappresenta le informazioni di un errore che si è verificato eseguendo una determinata operazione}
\newcommand{\errorinfomodelP}{\texttt{ErrorInfoModel: ErrorInfoModel} \\
	Parametro contenente il riferimento alla classe che rappresenta le informazioni di un errore che si è verificato eseguendo una determinata operazione}
\newcommand{\userdetailsmodelA}{\texttt{-} \texttt{UserDetailsModel: UserDetailsModel} \\ Campo dati che rappresenta un utente. Contiene tutte le informazioni necessarie alla presentazione del contenuto di un utente sia nella visualizzazione che nella gestione di un profilo}
\newcommand{\userdetailsmodelP}{\texttt{UserDetailsModel: UserDetailsModel} \\
	Parametro contenente un riferimento alla classe per poter istanziare un oggetto di tipo \texttt{UserDetailsModel}}
\newcommand{\mdBottomSheetA}{\texttt{-} \texttt{\$mdBottomSheet: \$mdBottomSheet}: \\
	Servizio offerto dalla libreria \texttt{Angular Material} che permette di aprire una tendina a scorrimento sopra la vista principale per mostrare un set di bottoni. Implementa le \texttt{promise}. In \textit{QuizziPedia} serve per poter scegliere la lingua con sui visualizzare l'applicazione}
\newcommand{\mdBottomSheetP}{\texttt{\$mdBottomSheet: \$mdBottomSheet}: \\
	Parametro contenente un riferimento al servizio offerto dalla libreria \texttt{Angular Material} che permette di aprire una tendina a scorrimento sopra la vista principale per mostrare un set di bottoni. Implementa le \texttt{promise}. In \textit{QuizziPedia} serve per poter scegliere la lingua con cui visualizzare l'applicazione}
\newcommand{\locationA}{\texttt{-} \texttt{\$location: \$location} \\
	Campo dati contenente un riferimento al servizio creato da \textit{Angular\ped{G}} che permette di accedere alla barra degli indirizzi del \textit{browser\ped{G}}, i cambiamenti all'URL nella barra degli indirizzi si riflettono in questo oggetto e viceversa}
\newcommand{\locationP}{\texttt{\$location: \$location} \\
	Parametro contenente un riferimento al servizio creato da \textit{Angular\ped{G}} che permette di accedere alla barra degli indirizzi del \textit{browser\ped{G}}, i cambiamenti all'URL nella barra degli indirizzi si riflettono in questo oggetto e viceversa}
\newcommand{\routeparamsA}{\texttt{-} \texttt{\$routeParams: \$routeParams} \\
	Campo dati contenente il riferimento all'oggetto globale \$routeParams creato da \textit{Angular\ped{G}}. Tale servizio permette di recuperare il set di variabili presenti nell'url}
\newcommand{\routeparamsP}{\texttt{\$routeParams: \$routeParams} \\
	Parametro contenente il riferimento all'oggetto globale \$routeParams creato da \textit{Angular\ped{G}}. Tale servizio permette di recuperare il set di variabili presenti nell'url}
\newcommand{\questionnairemodelA}{\texttt{-} \texttt{QuestionnaireModel: QuestionnaireModel} \\ Campo dati che rappresenta un riferimento ad un questionario. Contiene tutte le informazioni necessarie alla presentazione del contenuto del questionario}
\newcommand{\questionnairemodelP}{\texttt{QuestionnaireModel: QuestionnaireModel} \\Parametro che rappresenta un riferimento ad un questionario. Contiene tutte le informazioni necessarie alla presentazione del contenuto del questionario}
\newcommand{\timeoutA}{\texttt{-} \texttt{\$timeout: \$timeout} \\
	Campo dati contenente il riferimento all'oggetto globale \$timeout creato da \textit{Angular\ped{G}}. 
	Il valore di ritorno di una chiamata alla funzione di \texttt{\$timeout} è una \textit{promise\ped{G}}, la quale sarà risolta quando avverrà il ritardo e la funzione di timeout eseguita}
\newcommand{\timeoutP}{\texttt{\$timeout: \$timeout} \\
	Parametro contenente il riferimento all'oggetto globale \$timeout creato da \textit{Angular\ped{G}}. 
	Il valore di ritorno di una chiamata alla funzione di \texttt{\$timeout} è una promise, la quale sarà risolta quando avverrà il ritardo e la funzione di timeout eseguita}
\newcommand{\questionitemmodelA}{\texttt{-} \texttt{QuestionItemModel: QuestionItemModel} \\ Campo dati che rappresenta una domanda. Contiene tutte le informazioni necessarie alla presentazione del contenuto della domanda}
\newcommand{\questionitemmodelP}{\texttt{QuestionItemModel: QuestionItemModel} \\ Parametro che rappresenta una domanda. Contiene tutte le informazioni necessarie alla presentazione del contenuto della domanda}
\newcommand{\utilsA}{\texttt{-} \texttt{Utils: Utils} \\ Campo dati che rappresenta un oggetto contenente metodi che non appartengono a nessuna classe ma sono utili per svariati scopi}
\newcommand{\utilsP}{\texttt{Utils: Utils} \\Parametro che rappresenta un oggetto contenente metodi che non appartengono a nessuna classe ma sono utili per svariati scopi}
\newcommand{\questionsserviceA}{\texttt{-} \texttt{QuestionService: QuestionsService}\\ Permette di ottenere domande esistenti tramite chiamata di metodo specifici}
\newcommand{\questionsserviceP}{\texttt{QuestionService: QuestionService} \\ Parametro che permette di ottenere domande esistenti tramite chiamata di metodo specifici}
\newcommand{\mddialogA}{\texttt{-} \texttt{\$mdDialog: \$mdDialog} \\
	Campo dati contenente un riferimento al servizio della libreria \textit{Material for Angular\ped{G}} che permette di creare delle componenti a pop-up}
\newcommand{\mddialogP}{\texttt{\$mdDialog: \$mdDialog} \\
	Parametro contenente un riferimento al servizio della libreria \textit{Material for Angular\ped{G}} che permette di creare delle componenti a pop-up}

\newpage
\subsection{QuizziPedia::Front-End::Controllers}


\begin{figure} [ht]
	\centering
	\includegraphics[scale=0.42]{UML/Package/QuizziPedia_Front-End_Controllers.png}
	\caption{QuizziPedia::Front-End::Controllers}
\end{figure} \FloatBarrier

\subsubsection{Informazioni generali}
\begin{itemize}
	\item \textbf{Descrizione}: \textit{package\ped{G}} che contiene i controller individuati per la parte front-end dell'applicazione;
	\item \textbf{Padre}: \texttt{Front-End};
	\item \textbf{Interazione con altri componenti}:
	\begin{itemize}
		\item \texttt{Models}: \textit{package\ped{G}} che contiene le classi model individuate;
		\item \texttt{Services}: \textit{package\ped{G}} che contiene le classi services individuate.
	\end{itemize} 
\end{itemize}
\subsubsection{Classi}

\input{sezioni/Front-End/Controllers/QuizziPedia_Front-End_Controllers_ClickableAreaQuestionsController.tex}
\input{sezioni/Front-End/Controllers/QuizziPedia_Front-End_Controllers_ConnectionQuestionsController.tex}
\paragraph{QuizziPedia::Front-End::Controllers::CreateQuestionnaireController}
\begin{figure} [ht]
	\centering
	\includegraphics[scale=0.45]{UML/Classi/Front-End/QuizziPedia_Front-end_Controller_CreateQuestionnaireController.png}
	\caption{QuizziPedia::Front-End::Controllers::CreateQuestionnaireController}
\end{figure} \FloatBarrier
\begin{itemize}
	\item \textbf{Descrizione}: questa classe permette di gestire la creazione di un questionario;
	\item \textbf{Utilizzo}: fornisce tutte le funzionalità per la creazione di un nuovo questionario e per la modifica di uno esistente;
	\item \textbf{Relazione con altre classi}:
	\begin{itemize}
		\item \textit{IN} \texttt{CreateQuestionnaireModelView}: Oggetto di tipo \texttt{CreateQuestionnaireModelView}. All'interno di esso sono presenti le variabili e i metodi necessari per il \textit{Two-Way Data-Binding\ped{G}} tra la view \texttt{CreateQuestionnaireView} e il controller \texttt{CreateQuestionnaireController}; 
		\item \textit{IN} \texttt{QuizService}: questa classe permette di ottenere i dati di un quiz tramite delle parole chiave inserite dall'utente nella barra di ricerca;
		\item \textit{IN} \texttt{QuestionnaireModel}: model relativo ai questionari.
	\end{itemize}
	\item \textbf{Attributi}:
	\begin{itemize}
		\item \texttt{-} \texttt{scope: Scope} \\
		Campo dati contenente un riferimento all’oggetto \$scope creato da \textit{Angular\ped{G}}, viene utilizzato come mezzo di comunicazione tra il controller e la view. Contiene gli oggetti che definiscono il model dell’applicazione;
		\item \texttt{-} \texttt{\$rootScope: \$rootScope} \\
		Campo dati contenente il riferimento all'oggetto globale \$rootScope creato da \textit{Angular\ped{G}}. Viene utilizzato per rendere accessibile a tutti i controller e a tutte le view l'oggetto \texttt{QuestionnaireModel}. In questo caso viene utilizzato per inserire in \$rootScope l'oggetto di ritorno della chiamata a \texttt{getQuestiontionnaire} e la lista dei questionari ottenuta dalla chiamata \texttt{getQuestionnairePreview};
		\item \texttt{-} \texttt{\$mdDialog: \$mdDialog} \\
		Campo dati contenente un riferimento al servizio della libreria \textit{Material for Angular\ped{G}} che permette di creare delle componenti a popup;
		\item \texttt{-} \texttt{QuizService} \\ questa classe permette di ottenere i dati di un quiz tramite delle parole chiave inserite dall'utente nella barra di ricerca;

		
	\end{itemize}
	\item \textbf{Metodi}:
	\begin{itemize}
		\item \texttt{+} \texttt{CreateQuestionnaireController(\$scope: \$scope, \$rootscope: \$rootscope, \$mdDialog: \$mdDialog, QuizService: QuizService)}: \\ Metodo costruttore della classe. \\
		\textbf{Parametri}:
		\begin{itemize}
			\item \texttt{-} \texttt{\$scope: \$scope} \\
			Campo dati contenente un riferimento all’oggetto \$scope creato da \textit{Angular\ped{G}}. Viene utilizzato come mezzo di comunicazione tra il controller e la view. Contiene gli oggetti che definiscono il viewmodel e il model dell’applicazione;
				\item \texttt{-} \texttt{\$rootScope: \$rootScope} \\
				Campo dati contenente il riferimento all'oggetto globale \$rootScope creato da \textit{Angular\ped{G}}. Viene utilizzato per rendere accessibile a tutti i controller e a tutte le view l'oggetto \texttt{QuestionnaireModel}. In questo caso viene utilizzato per inserire in \$rootScope l'oggetto di ritorno della chiamata a \texttt{getQuestiontionnaire} e la lista dei questionari ottenuta dalla chiamata \texttt{getQuestionnairePreview};
			\item \texttt{-} \texttt{\$mdDialog: \$mdDialog} \\
			Campo dati contenente un riferimento al servizio della libreria \textit{Material for Angular\ped{G}} che permette di creare delle componenti a popup;
			\item \texttt{QuizService: QuizService}: parametro che permette di ottenere, tramite il service, la lista di tutte le domande presenti nel quiz;
		\end{itemize}
		\item \texttt{+} \texttt{createQuestionnaire(title: String, quiz: QuestionnaireModel)}: \\Metodo che permette di inserire un questionario nel database tramite richiesta al service; \\
			\textbf{Parametri}:
			\begin{itemize}
				\item 
			\end{itemize}
		\item \texttt{+} \texttt{modifyQuestionnaire(quizId: QuestionnaireModel): Void} \\ Metodo che serve per modificare un questionario; \\
			\textbf{Parametri}:
			\begin{itemize}
				\item \texttt{quiz: QuestionnaireModel}: parametro che rappresenta l'oggetto questionario;
			\end{itemize}
		\item \texttt{+} \texttt{getQuestionnaire(quizId: String): QuestionnaireModel} \\Metodo che serve per ottenere un questionario tramite l'id in modo da poterlo modificare; \\
			\textbf{Parametri}:
			\begin{itemize}
				\item \texttt{quizId: String}: parametro che rappresenta l'id del questionario da richiedere.
			\end{itemize}
		\item \texttt{+} \texttt{getQuestionnairePreview(username: String): QuestionnaireModel[]} \\ Metodo che serve per ottenere la lista di tutti i questionari di un utente; \\
			\textbf{Parametri}:
			\begin{itemize}
				\item \texttt{username: String}: parametro che indica l'utente del quale vogliamo caricare tutti i questionari.
			\end{itemize}
		\item \texttt{+} \texttt{deleteQuestionnaire(quizId: String): Void} \\Metodo che elimina un questionario.
		\textbf{Parametri}:
		\texttt{quizId: String}: identificativo del questionario da eliminare.
	\end{itemize}
\end{itemize}


\paragraph{QuizziPedia::Front-End::Controllers::EditorQMLController}
\begin{figure} [ht]
	\centering
	\includegraphics[scale=0.35]{UML/Classi/Front-End/QuizziPedia_Front-end_Controller_EditorQMLController.png}
	\caption{QuizziPedia::Front-End::Controllers::EditorQMLController}
\end{figure} \FloatBarrier
\begin{itemize}
	\item \textbf{Descrizione}: questa classe permette di gestire la creazione e la modifica di domande create tramite editor QML;
	\item \textbf{Utilizzo}: fornisce le funzionalità per creare e modificare una domanda tramite editor QML;
	\item \textbf{Relazione con altre classi}:
	\begin{itemize}
		\item \textbf{IN} \texttt{EditorQMLModelView}: classe di tipo modelview la cui istanziazione è contenuta all'interno della variabile di ambiente \$scope di \textit{Angular\ped{G}}. All'interno di essa sono presenti le variabili e i metodi necessari per il \textit{Two-Way Data-Binding\ped{G}} tra la \textit{view\ped{G}} \texttt{EditorQMLView} e il \textit{controller\ped{G}} \texttt{EditorQMLController};
		\item \textbf{IN} \texttt{QuestionService}: questa classe permette di:
		\begin{itemize}
			\item Ottenere una domanda attraverso il metodo dedicato;
			\item Caricare una domanda modificata;
			\item Caricare una nuova domanda.
		\end{itemize}
		\item \textbf{IN} \texttt{ParserQML}: questa classe rappresenta il parser QML. Essa fa diventale un oggetto di tipo \texttt{QuestionItemModel} in linguaggio QML e viceversa;
		\item \textbf{IN} \texttt{QuestionItemModel}: questa classe rappresenta il modello di una domanda.
	\end{itemize}
	\item \textbf{Attributi}:
	\begin{itemize}
		\item \texttt{-} \texttt{\$scope: \$scope} \\
		Campo dati contenente un riferimento all'oggetto \$scope creato da \textit{Angular\ped{G}}, viene utilizzato come mezzo di comunicazione tra il \textit{controller\ped{G}} e la \textit{view\ped{G}}. Contiene gli oggetti che definiscono il model dell'applicazione;
		\item \texttt{-} \texttt{QuestionItemModel: QuestionItemModel} \\
		Campo dati che si riferisce alla classe che rappresenta il modello della classe;
		\item \texttt{-} \texttt{\$mdDialog: \$mdDialog} \\
		Campo dati contenente un riferimento al servizio della libreria \textit{Material for Angular\ped{G}} che permette di creare delle componenti a pop-up;
		\item \texttt{-} \texttt{QuestionService: QuestionService}: \\
		Campo dati contenente un riferimento al servizio che si occupa della gestione delle informazioni legate alle domande; 
		\item \texttt{-} \texttt{\$routeParams: \$routeParams} \\
		Campo dati contenente il riferimento all'oggetto globale \$routeParams creato da \textit{Angular\ped{G}}. Tale servizio permette di recuperare il set di variabili presenti nell'url; 
		\item \texttt{-} \texttt{ParserQML: ParserQML} \\
		Campo dati che si riferisce alla classe che rappresenta il parser QML. Essa fa diventale un oggetto di tipo \texttt{QuestionItemModel} in linguaggio QML e viceversa.
	\end{itemize}
	\item \textbf{Metodi}:
	\begin{itemize}
		\item \texttt{+} \texttt{EditorQMLController(\$scope: \$scope, QuestionItemModel: QuestionItemModel, \$mdDialog: \$mdDialog, QuestionService: QuestionService, \$routeParams:\\ \$routeParams, ParserQML: ParserQML)} \\ 
		Metodo costruttore della classe. Se in \texttt{\$routeParams} sarà presente il codice univoco che rappresenta una domanda e di questa il creatore è l'utente autenticato, allora verrà scaricato attraverso il \texttt{QuestionService} il contenuto della domanda così da poterlo modificare in formato QML. In caso contrario verrà mostrato un errore attraverso \texttt{\$mdDialog} indicando che i privilegi per tale operazione sono negati. Nel caso in cui non ci sarà tale parametro in \texttt{\$routeParams} verrà caricata la \textit{view\ped{G}} vuota così da poter creare una nuova domanda. \\
		\textbf{Parametri}:
		\begin{itemize}
			\item \texttt{\$scope: \$scope} \\
			Parametro contenente un riferimento all'oggetto \$scope creato da \textit{Angular\ped{G}}, viene utilizzato come mezzo di comunicazione tra il \textit{controller\ped{G}} e la \textit{view\ped{G}}. Contiene gli oggetti che definiscono il \textit{model\ped{G}} dell'applicazione;
			\item \texttt{QuestionItemModel: QuestionItemModel} \\ 
			Parametro che si riferisce alla classe che rappresenta il modello della classe;
			\item \texttt{\$mdDialog: \$mdDialog} \\
			Parametro contenente un riferimento al servizio della libreria \textit{Material for Angular\ped{G}} che permette di creare delle componenti a pop-up;
			\item \texttt{QuestionService: QuestionService}: \\
			Parametro contenente un riferimento al servizio che si occupa della gestione delle informazioni legate alle domande;
			\item \texttt{\$routeParams: \$routeParams} \\
			Parametro contenente il riferimento all'oggetto globale \$routeParams creato da \textit{Angular\ped{G}}. Tale servizio permette di recuperare il set di variabili presenti nell'url; 
			\item \texttt{ParserQML: ParserQML} \\
			Parametro contenente un riferimento alla classe che rappresenta il parser QML. Essa fa diventale un oggetto di tipo \texttt{QuestionItemModel} in linguaggio QML e viceversa.
		\end{itemize}
		\item \texttt{+} \texttt{submitQuestion(): void}\\ 
		Metodo che gestisce l'evento click sul pulsante di conferma sulla domanda. Raccoglie i dati dal \textit{modelview\ped{G}}, li converte attraverso il parser QML, e li manda al server attraverso \texttt{QuestionService}. Poi verrà effettuato il redirect alla pagina di gestione delle domande oppure al questionario che si stava creando.
	\end{itemize}
\end{itemize}



\paragraph[QuizziPedia::Front-End::Controllers\\::FillingQuestionnaireController]{QuizziPedia::Front-End::Controllers::FillingQuestionnaireController}
\begin{figure} [ht]
	\centering
	\includegraphics[scale=0.25]{UML/Classi/Front-End/QuizziPedia_Front-end_Controller_FillingQuestionnaireController.png}
	\caption{QuizziPedia::Front-End::Controllers::FillingQuestionnaireController}
\end{figure} \FloatBarrier
\begin{itemize}
	\item \textbf{Descrizione}: questa classe permette di gestire la compilazione del questionario;
	\item \textbf{Utilizzo}: fornisce le funzionalità per compilare un questionario e per gestire il cambio di domanda;
	\item \textbf{Relazione con altre classi}:
	\begin{itemize}
		\item \textbf{IN} \texttt{FillingQuestionnaireModelView}: classe di tipo modelview la cui istanziazione è contenuta all'interno della variabile di ambiente \$scope di \textit{Angular\ped{G}}. All'interno di essa sono presenti le variabili e i metodi necessari per il \textit{Two-Way Data-Binding\ped{G}} tra la \textit{view\ped{G}} \texttt{FillingQuestionnaireView} e il \textit{controller\ped{G}} \texttt{FillingQuestionnaireController};  
		\item \textbf{IN} \texttt{InfoQuestionnaireTemplate}: rappresenta il componente grafico che permette all'utente di visualizzare le informazioni principali del questionario che si sta per svolgere. Viene gestito dinamicamente all'interno della \textit{view\ped{G}} \texttt{TrainingView} attraverso il \textit{controller\ped{G}} \texttt{TrainingController};
		\item \textbf{IN} \texttt{QuizService}: permette di ottenere i dati di un quiz tramite delle parole chiave inserite dall'utente nella barra di ricerca. Permette inoltre di iscriversi ad un questionario e di scaricare l'intera lista di domande di un questionario a partire dal suo id univoco;
		\item \textbf{IN} \texttt{QuestionnaireModel}: rappresenta un questionario. Contiene tutte le informazioni necessarie alla presentazione del contenuto del questionario;
		\item \textbf{IN} \texttt{QuestionsController}: questa classe permette di gestire il recupero delle domande per far si che possano essere visualizzate nella modalità allenamento e nella compilazione dei questionari; 
		\item \textbf{IN} \texttt{QuestionItemModel}: rappresenta una domanda. Contiene tutte le informazioni necessarie alla presentazione del contenuto della domanda.
	\end{itemize}
	\item \textbf{Attributi}:
	\begin{itemize}
		\item \texttt{-} \texttt{\$scope: \$scope} \\
		Campo dati contenente un riferimento all'oggetto \$scope creato da \textit{Angular\ped{G}}, viene utilizzato come mezzo di comunicazione tra il \textit{controller\ped{G}} e la \textit{view\ped{G}}. Contiene gli oggetti che definiscono il \textit{model\ped{G}} dell'applicazione;
		\item \texttt{-} \texttt{\$rootScope: \$rootScope} \\
		Campo dati contenente il riferimento all'oggetto globale \$rootScope creato da \textit{Angular\ped{G}}. Viene utilizzato per rendere accessibile a tutti i \textit{controller\ped{G}} e a tutte le \textit{view\ped{G}} l'oggetto \texttt{QuestionnaireModel}. In questo caso viene utilizzato per inserire in \$rootScope l'oggetto di ritorno della chiamata a \texttt{getNextQuestion} e l'intero questionario ritornato dalla chiamata a \texttt{getQuestionnaire};
		\item \texttt{-} \texttt{\$mdDialog: \$mdDialog} \\
		Campo dati contenente un riferimento al servizio della libreria \textit{Material for Angular\ped{G}} che permette di creare delle componenti a pop-up;
		\item \texttt{-} \texttt{QuizService: QuizService}: questa classe permette di ottenere i dati di un quiz tramite delle parole chiave inserite dall'utente nella barra di ricerca. Permette inoltre di iscriversi ad un questionario e di scaricare l'intera lista di domande di un questionario a partire dal suo id univoco;
		\item \texttt{+} \texttt{fillQuiz: FillingQuestionnaireModelView} \\
		Oggetto di tipo \texttt{FillingQuestionnaireModelView}. All'interno di esso sono presenti le variabili e i metodi necessari per il \textit{Two-Way Data-Binding\ped{G}} tra la \textit{view\ped{G}} \texttt{FillingQuestionnaireView} e il \textit{controller\ped{G}} \texttt{FillingQuestionnaireController};
		\item \locationA;
		\item \routeparamsA;
		\item \errorinfomodelA;
		\item \questionnairemodelA;
		\item \timeoutA;
		\item \texttt{-} \texttt{quizIsLoaded: Boolean} \\ Campo dati che indica se il questionario è stato scaricato;
		\item \texttt{-} \texttt{questionNumberOnQuiz: Number} \\ Campo dati che indica il numero progressivo della domanda mostrata;
		\item \texttt{-} \texttt{startQuiz: Boolean} \\ Campo dati che indica se il questionario è pronto per essere iniziato;
		\item \texttt{-} \texttt{noStart: Boolean} \\ Campo dati che indica se il questionario è abilitato alla compilazione;
		\item \texttt{-} \texttt{started: Boolean} \\ Campo dati che indica se il questionario è stato iniziato;
		\item \texttt{-} \texttt{quizIsFinished: Boolean} \\ Campo dati che indica se il questionario è finito;
		\item \texttt{-} \texttt{noAuth: Boolean} \\ Campo dati che indica se l'utente che si posiziona nel questionario ha o meno l'autorizzazione;
		\item \texttt{-} \texttt{questions: Array<QuestionItemModel>} \\ Campo dati che rappresenta le domande che verranno poste all'utente;
		\item \texttt{-} \texttt{quiz: QuestionnaireModel} \\ Campo dati che rappresenta il questionario;
		\item \texttt{-} \texttt{myChartDataDoughnut: Object} \\ Campo dati che rappresenta i dati che verranno mostrati nel grafico finale;
		\item \texttt{-} \texttt{myChartOptionsDoughnut: Object} \\ Campo dati che rappresenta le opzioni per la rappresentazione del grafico finale.
	\end{itemize}
	\item \textbf{Metodi}:
	\begin{itemize}
		\item \texttt{+} \texttt{FillingQuestionnaireController(\$scope: \$scope, \$rootScope: \$rootScope, \$mdDialog: \$mdDialog, QuizService: QuizService, \$timeout:\$timeout, \$location:\$location, \$routeParams: \$routeParams, ErrorInfoModel: ErrorInfoModel, QuestionnaireModel: QuestionnaireModel)} \\Metodo costruttore della classe.\\
		\textbf{Parametri}:
		\begin{itemize}
			\item \texttt{\$scope: \$scope} \\
			Parametro contenente un riferimento all'oggetto \$scope creato da \textit{Angular\ped{G}}. Viene utilizzato come mezzo di comunicazione tra il \textit{controller\ped{G}} e la \textit{view\ped{G}}. Contiene gli oggetti che definiscono il viewmodel e il \textit{model\ped{G}} dell'applicazione;
			\item \texttt{-} \texttt{\$location: \$location} \\
			Parametro contenente un riferimento al servizio creato da \textit{Angular\ped{G}} che permette di accedere alla barra degli indirizzi del \textit{browser\ped{G}}, i cambiamenti all’URL nella barra degli indirizzi si riflettono in questo oggetto e viceversa;
			\item \texttt{\$mdDialog: \$mdDialog} \\
			Parametro contenente un riferimento al servizio della libreria \textit{Material for Angular\ped{G}} che permette di creare delle componenti a pop-up;
			\item \texttt{QuizService: QuizService}:\\ Parametro che permette di ottenere, tramite il service, la lista di tutte le domande presenti nel quiz;
			\item \locationP;
			\item \routeparamsP;
			\item \errorinfomodelP;
			\item \questionnairemodelP;
			\item \timeoutP.
		\end{itemize}
		\item \texttt{-} \texttt{downloadQuiz(): void}: \\ Metodo che ottiene l'intero questionario; \\
		\item \texttt{+} \texttt{nextQuestion(): void}: \\ Metodo che carica la domanda successiva del quiz tramite chiamata a \texttt{QuestionController}; \\
		\item \texttt{+} \texttt{startQuizClick(): void} \\
		Metodo che gestisce l'evento per iniziare il questionario;
		\item \texttt{+} \texttt{endQuiz(): void} \\
		Metodo che gestisce l'evento per terminare il questionario in modo anticipato;
		\item \texttt{+} \texttt{goToHome(): void} \\
		Metodo che gestisce l'evento per tornare alla pagina principale dell'applicazione;  
		\item \texttt{+} \texttt{\$on('\$locationChangeStart': String, callback: function): void} \\
		Metodo che gestisce l'evento per tornare alla pagina principale dell'applicazione.\\
		\textbf{Parametri}:
		\begin{itemize}
			  	\item \texttt{-} \texttt{'\$locationChangeStart': String} \\
			  	Parametro che indica su quale evento rimanere in ascolto;
			  	\item \texttt{-} \texttt{callback: function} \\
			  	Parametro che indica una funzione da eseguire.
		\end{itemize}
		\item \texttt{+} \texttt{\$on('addResult', callback: function): void} \\
		Metodo che gestisce l'evento per aggiornare i risultati dati alle domande. \\
		\textbf{Parametri}:
		\begin{itemize}
			\item \texttt{-} \texttt{'addResult': String} \\
			Parametro che indica su quale evento rimanere in ascolto;
			\item \texttt{-} \texttt{callback: function} \\
			Parametro che indica una funzione da eseguire.
		\end{itemize}
		\item \texttt{+} \texttt{graphResultAfterFinishedAQuiz(): void} \\
		Metodo imposta le variabili per la visualizzazione del grafico finale. 
	\end{itemize}
\end{itemize}


\paragraph{QuizziPedia::Front-End::Controllers::FillingQuestionsController}
\begin{figure} [ht]
	\centering
	\includegraphics[scale=0.40]{UML/Classi/Front-End/QuizziPedia_Front-end_Controller_FillingQuestionsController.png}
	\caption{QuizziPedia::Front-End::Controllers::FillingQuestionsController}
\end{figure} \FloatBarrier
\begin{itemize}
	\item \textbf{Descrizione}: questa classe permette di gestire la creazione e la modifica di una domanda a riempimento di spazi;
	\item \textbf{Utilizzo}: fornisce le funzionalità per inserire una nuova domanda a riempimento di spazi nel database e per modificarne una esistente;
	\item \textbf{Relazione con altre classi}:
	\begin{itemize}
		\item \textit{IN} \texttt{FillingQuestionsModelView}: classe di tipo modelview la cui istanziazione è contenuta all'interno della variabile di ambiente \$scope di \textit{Angular\ped{G}}. All'interno di essa sono presenti le variabili e i metodi necessari per il \textit{Two-Way Data-Binding\ped{G}} tra la \textit{view\ped{G}} \texttt{FillingQuestionsView} e il \textit{controller\ped{G}} \texttt{FillingQuestionsController};
		\item \textit{IN} \texttt{QuestionService}: questa classe permette di:
		\begin{itemize}
			\item Ottenere una domanda attraverso il metodo dedicato;
			\item Caricare una domanda modificata;
			\item Caricare una nuova domanda.
		\end{itemize}
		\item \textit{IN} \texttt{QuestionItemModel}: questa classe rappresenta il modello di una domanda.
	\end{itemize}
	\item \textbf{Attributi}:
	\begin{itemize}
		\item \texttt{-} \texttt{\$scope: \$scope} \\
		Campo dati contenente un riferimento all'oggetto \$scope creato da \textit{Angular\ped{G}}, viene utilizzato come mezzo di comunicazione tra il \textit{controller\ped{G}} e la \textit{view\ped{G}}. Contiene gli oggetti che definiscono il \textit{model\ped{G}} dell'applicazione;
		\item \texttt{-} \texttt{QuestionItemModel: QuestionItemModel} \\
		Campo dati che si riferisce alla classe che rappresenta il modello della classe;
		\item \texttt{- \$timeout: \$timeout} \\
		Attributo contenente il riferimento all'oggetto globale \$timeout creato da \textit{Angular\ped{G}}. 
		Il valore di ritorno di una chiamata alla funzione di \texttt{\$timeout} è una \textit{promise\ped{G}}, la quale sarà risolta quando avverrà il ritardo e la funzione di timeout eseguita; 
		Campo dati contenente il riferimento all'oggetto globale \$routeParams creato da \textit{Angular\ped{G}}. Tale servizio permette di recuperare il set di variabili presenti nell'url;
		\item \texttt{-} \texttt{\$mdDialog: \$mdDialog} \\
		Campo dati contenente un riferimento al servizio della libreria \textit{Material for Angular\ped{G}} che permette di creare delle componenti a pop-up;
		\item \texttt{-} \texttt{QuestionService: QuestionService}: \\Parametro contenente un riferimento al servizio che si occupa della gestione delle informazioni legate alle domande;
		\item \texttt{- \$routeParams: \$routeParams} \\ Parametro contenente il riferimento all'oggetto globale \$routeParams creato da \textit{Angular\ped{G}}. Tale servizio permette di recuperare il set di variabili presenti nell'url.
	\end{itemize}
	\item \textbf{Metodi}:
	\begin{itemize}
		\item \texttt{+} \texttt{FillingQuestionsController(\$scope: \$scope, QuestionItemModel: QuestionItemModel, \$mdDialog: \$mdDialog, \$timeout: \$timeout, QuestionService: QuestionService, \$routeParams: \$routeParams)} \\ 
		Metodo costruttore della classe. Se in \texttt{\$routeParams} sarà presente il codice univoco che rappresenta una domanda e di questa il creatore è l'utente autenticato, allora verrà scaricato attraverso il \texttt{QuestionService} il contenuto della domanda così da poterlo modificare. In caso contrario verrà mostrato un errore attraverso \texttt{\$mdDialog} indicando che i privilegi per tale operazione sono negati. Nel caso in cui non ci sarà tale parametro in \texttt{\$routeParams} verrà caricata la \textit{view\ped{G}} vuota così da poter creare una nuova domanda; \\
		Parametro contenente il riferimento all'oggetto globale \$routeParams creato da \textit{Angular\ped{G}}. Tale servizio permette di recuperare il set di variabili presenti nell'url.\\
		\textbf{Parametri}:
		\begin{itemize}
			\item \texttt{\$scope: \$scope} \\
			Parametro contenente un riferimento all'oggetto \$scope creato da \textit{Angular\ped{G}}, viene utilizzato come mezzo di comunicazione tra il \textit{controller\ped{G}} e la \textit{view\ped{G}}. Contiene gli oggetti che definiscono il \textit{model\ped{G}} dell'applicazione;
			\item \texttt{QuestionItemModel: QuestionItemModel} \\ 
			Parametro che si riferisce alla classe che rappresenta il modello della classe;
			\item \texttt{\$mdDialog: \$mdDialog} \\
			Parametro contenente un riferimento al servizio della libreria \textit{Material for Angular\ped{G}} che permette di creare delle componenti a pop-up;
			\item \texttt{QuestionService: QuestionService}: \\
			Parametro contenente un riferimento al servizio che si occupa della gestione delle informazioni legate alle domande;
			\item \texttt{\$timeout: \$timeout} \\
			Parametro contenente il riferimento all'oggetto globale \$timeout creato da \textit{Angular\ped{G}}. 
			Il valore di ritorno di una chiamata alla funzione di \texttt{\$timeout} è una promise, la quale sarà risolta quando avverrà il ritardo e la funzione di timeout eseguita; 
			\item \texttt{\$routeParams: \$routeParams} \\
			Parametro contenente il riferimento all'oggetto globale \$routeParams creato da \textit{Angular\ped{G}}. Tale servizio permette di recuperare il set di variabili presenti nell'url. 
		\end{itemize}
		\item \texttt{+} \texttt{submitQuestion(): void}\\ 
		Metodo che gestisce l’evento click sul pulsante di conferma sulla domanda. Raccoglie i dati dal modelview e li manda al server attraverso \texttt{QuestionService}. Poi verrà effettuato il redirect alla pagina di gestione delle domande oppure al questionario che si stava creando; 
		\item \texttt{+} \texttt{choseThatWord(word:String, number: Integer): void}\\
		Metodo che gestisce l’evento click su una parola del testo. Una volta selezionata essa verrà inserita nell'array che conterrà le parole che dovranno essere nascoste quando l'esercizio sarà proposto. \\
		\textbf{Parametri}:
		\begin{itemize}
			\item \texttt{word:String} \\
			Parametro contenente la parola scelta da nascondere;
			\item \texttt{number: Integer} \\ 
			Parametro che si riferisce al numero della parola scelta da nascondere.
		\end{itemize}
	\end{itemize}
\end{itemize}


\paragraph{QuizziPedia::Front-End::Controllers::HomeController}
\begin{figure} [ht]
	\centering
	\includegraphics[scale=0.6]{UML/Classi/Front-End/QuizziPedia_Front-end_Controller_HomeController.png}
	\caption{QuizziPedia::Front-End::Controllers::HomeController}
\end{figure} \FloatBarrier
\begin{itemize}
	\item \textbf{Descrizione}: questa classe permette di gestire la home page;
	\item \textbf{Utilizzo}: fornisce tutte le informazioni da mostrare nella homepage;
	\item \textbf{Relazione con altre classi}:
	\begin{itemize}
		\item \textbf{IN} \texttt{HomeModelView}: classe di tipo modelview la cui istanziazione è contenuta all'interno della variabile di ambiente \$scope di \textit{Angular.js\ped{G}}. All'interno di essa sono presenti le variabili e i metodi necessari per il \textit{Two-Way Data-Binding\ped{G}} tra la \textit{view\ped{G}} \texttt{HomeView} e il \textit{controller\ped{G}} \texttt{HomeController};
	\end{itemize}
	\item \textbf{Attributi}:
	\begin{itemize}
		\item \texttt{-} \texttt{\$scope: \$scope} \\
		Campo dati contenente un riferimento all'oggetto \$scope creato da \textit{Angular\ped{G}}, viene utilizzato come mezzo di comunicazione tra il \textit{controller\ped{G}} e la \textit{view\ped{G}}. Contiene gli oggetti che definiscono il \textit{model\ped{G}} dell'applicazione;
		\item \texttt{-} \texttt{\$location: \$location} \\
		Campo dati contenente un riferimento al servizio creato da \textit{Angular\ped{G}} che permette di accedere alla barra degli indirizzi del \textit{browser\ped{G}}, i cambiamenti all'URL nella barra degli indirizzi si riflettono in questo oggetto e viceversa.
	\end{itemize}
	\item \textbf{Metodi}:
	\begin{itemize}
		\item \texttt{+} \texttt{HomeController(\$scope: \$scope, \$location: \$location)} \\
		Metodo costruttore della classe: \\
		\textbf{Parametri}:
		\begin{itemize}
			\item \texttt{\$scope: \$scope} \\
			Parametro contenente un riferimento all'oggetto \$scope creato da \textit{Angular\ped{G}}. Viene utilizzato come mezzo di comunicazione tra il \textit{controller\ped{G}} e la \textit{view\ped{G}}. Contiene gli oggetti che definiscono il viewmodel e il \textit{model\ped{G}} dell'applicazione;
			\item \texttt{\$location: \$location} \\
			Parametro contenente un riferimento al servizio creato da \textit{Angular\ped{G}} che permette di accedere alla barra degli indirizzi del \textit{browser\ped{G}}, i cambiamenti all'URL nella barra degli indirizzi si riflettono in questo oggetto e viceversa.
		\end{itemize}
		\item \texttt{+} \texttt{trainingMode(): void} \\
		Metodo che gestisce l’evento click sul pulsante di allenamento. Effettua il redirect alla pagina di allenamento.
	\end{itemize}
\end{itemize}


\paragraph{QuizziPedia::Front-End::Controllers::ImagesSortingQuestionsController}
\begin{figure} [ht]
	\centering
	\includegraphics[scale=0.33]{UML/Classi/Front-End/QuizziPedia_Front-end_Controller_ImagesSortingQuestionsController.png}
	\caption{QuizziPedia::Front-End::Controllers::ImagesSortingQuestionsController}
\end{figure} \FloatBarrier
\begin{itemize}
	\item \textbf{Descrizione}: questa classe permette di gestire la creazione e la modifica di una domanda a ordinamento immagini;
	\item \textbf{Utilizzo}: fornisce le funzionalità per inserire una nuova domanda a ordinamento immagini nel database e per modificarne una esistente;
	\item \textbf{Relazione con altre classi}:
	\begin{itemize}
		\item \textbf{IN} \texttt{ImageSortingQuestionsModelView}: classe di tipo modelview la cui istanziazione è contenuta all'interno della variabile di ambiente \$scope di \textit{Angular.js\ped{G}}. All'interno di essa sono presenti le variabili e i metodi necessari per il \textit{Two-Way Data-Binding\ped{G}} tra la view \texttt{ImagesSortingQuestionsView} e il controller \texttt{ImagesSortingQuestionsController};
		\item \textit{IN} \texttt{QuestionService}: questa classe permette di:
		\begin{itemize}
			\item Ottenere una domanda attraverso il metodo dedicato;
			\item Caricare una domanda modificata;
			\item Caricare una nuova domanda.
		\end{itemize}
		\item \textbf{IN} \texttt{QuestionItemModel}: questa classe rappresenta il modello di una domanda.
	\end{itemize}
	\item \textbf{Attributi}:
	\begin{itemize}
		\item \texttt{-} \texttt{\$scope: \$scope} \\
		Campo dati contenente un riferimento all’oggetto \$scope creato da \textit{Angular\ped{G}}, viene utilizzato come mezzo di comunicazione tra il \textit{controller\ped{G}} e la \textit{view\ped{G}}. Contiene gli oggetti che definiscono il \textit{model\ped{G}} dell’applicazione;
		\item \texttt{-} \texttt{QuestionItemModel: QuestionItemModel} \\
		Campo dati che si riferisce alla classe che rappresenta il modello della classe;
		\item \texttt{-} \texttt{\$mdDialog: \$mdDialog} \\
		Campo dati contenente un riferimento al servizio della libreria \textit{Material for Angular\ped{G}} che permette di creare delle componenti a pop-up;
		\item \texttt{-} \texttt{QuestionService: QuestionService}: \\
		Campo dati contenente un riferimento al servizio che si occupa della gestione delle informazioni legate alle domande;
		\item \texttt{-} \texttt{Upload: Upload} \\
		Campo dati contenente un riferimento alla libreria \textit{ng-file-upload\ped{G}} necessaria per il caricamento della foto profilo dell'utente;
		\item \texttt{-} \texttt{\$timeout: \$timeout} \\
		Campo dati contenente il riferimento all'oggetto globale \$timeout creato da \textit{Angular\ped{G}}. 
		Il valore di ritorno di una chiamata alla funzione di \texttt{\$timeout} è una \textit{promise\ped{G}}, la quale sarà risolta quando avverrà il ritardo e la funzione di timeout eseguita; 
		\item \texttt{\$routeParams: \$routeParams} \\
		Campo dati contenente il riferimento all'oggetto globale \$routeParams creato da \textit{Angular\ped{G}}. Tale servizio permette di recuperare il set di variabili presenti nell'url; 
	\end{itemize}
	\item \textbf{Metodi}:
	\begin{itemize}
		\item \texttt{+} \texttt{ImageSortingQuestionsController(\$scope: \$scope, QuestionItemModel: \\QuestionItemModel, \$mdDialog: \$mdDialog, QuestionService: QuestionService, Upload: Upload, \$timeout: \$timeout, \$routeParams: \$routeParams)} \\ 
		Metodo costruttore della classe. Se in \texttt{\$routeParams} sarà presente il codice univoco che rappresenta una domanda e di questa il creatore è l'utente autenticato, allora verrà scaricato attraverso il \texttt{QuestionService} il contenuto della domanda così da poterlo modificare. In caso contrario verrà mostrato un errore attraverso \texttt{\$mdDialog} indicando che i privilegi per tale operazione sono negati. Nel caso in cui non ci sarà tale parametro in \texttt{\$routeParams} verrà caricata la view vuota così da poter creare una nuova domanda; \\
		\textbf{Parametri}:
		\begin{itemize}
			\item \texttt{\$scope: \$scope} \\
			Parametro contenente un riferimento all’oggetto \$scope creato da \textit{Angular\ped{G}}, viene utilizzato come mezzo di comunicazione tra il \textit{controller\ped{G}} e la \textit{view\ped{G}}. Contiene gli oggetti che definiscono il model dell'applicazione;
			\item \texttt{QuestionItemModel: QuestionItemModel} \\ 
			Parametro che si riferisce alla classe che rappresenta il modello della classe;
			\item \texttt{\$mdDialog: \$mdDialog} \\
			Parametro contenente un riferimento al servizio della libreria \textit{Material for Angular\ped{G}} che permette di creare delle componenti a pop-up;
			\item \texttt{QuestionService: QuestionService}: \\
			Parametro contenente un riferimento al servizio che si occupa della gestione delle informazioni legate alle domande;
			\item \texttt{Upload: Upload} \\
			Parametro contenente un riferimento alla libreria \textit{ng-file-upload\ped{G}} necessaria per il caricamento della foto profilo dell'utente;
			\item \texttt{\$timeout: \$timeout} \\
			Parametro contenente il riferimento all'oggetto globale \$timeout creato da \textit{Angular\ped{G}}. 
			Il valore di ritorno di una chiamata alla funzione di \texttt{\$timeout} è una \textit{promise\ped{G}}, la quale sarà risolta quando avverrà il ritardo e la funzione di timeout eseguita; 
			\item \texttt{\$routeParams: \$routeParams} \\
			Parametro contenente il riferimento all'oggetto globale \$routeParams creato da \textit{Angular\ped{G}}. Tale servizio permette di recuperare il set di variabili presenti nell'url. 
		\end{itemize}
		\item \texttt{+} \texttt{submitQuestion(): void}\\ 
		Metodo che gestisce l’evento click sul pulsante di conferma sulla domanda. Raccoglie i dati dal modelview e li manda al server attraverso \texttt{QuestionService}. Poi verrà effettuato il redirect alla pagina di gestione delle domande oppure al questionario che si stava creando. 
	\end{itemize}
\end{itemize}


\paragraph{QuizziPedia::Front-End::Controllers::InputToListController}
\begin{figure} [ht]
	\centering
	\includegraphics[scale=0.45]{UML/Classi/Front-End/QuizziPedia_Front-end_Controller_InputToListController.png}
	\caption{QuizziPedia::Front-End::Controllers::InputToListController}
\end{figure} \FloatBarrier
\begin{itemize}
	\item \textbf{Descrizione}: questa classe permette di gestire l'inserimento di una lista di risposte durante la creazione di una domanda;
	\item \textbf{Utilizzo}: fornisce le funzionalità per confermare porzioni di domanda durante la creazione;
	\item \textbf{Relazione con altre classi}:
	\begin{itemize}
		\item \textit{OUT} \texttt{MultipleQuestionsModelView}: classe di tipo modelview la cui istanziazione è contenuta all'interno della variabile di ambiente \$scope di \textit{Angular.js\ped{G}}. All'interno di essa sono presenti le variabili e i metodi necessari per il \textit{Two-Way Data-Binding\ped{G}} tra la view \texttt{MultipleQuestionsView} e il controller \texttt{MultipleQuestionsController};
		\item \textit{OUT} \texttt{ConnectionQuestionsModelView}: classe di tipo modelview la cui istanziazione è contenuta all'interno della variabile di ambiente \$scope di \textit{Angular.js\ped{G}}. All'interno di essa sono presenti le variabili e i metodi necessari per il \textit{Two-Way Data-Binding\ped{G}} tra la view \texttt{ConnectionQuestionView} e il controller \texttt{ConnectionQuestionController};
		\item \textit{OUT} \texttt{StringsSortingQuestionsModelView}: classe di tipo modelview la cui istanziazione è contenuta all'interno della variabile di ambiente \$scope di \textit{Angular.js\ped{G}}. All'interno di essa sono presenti le variabili e i metodi necessari per il \textit{Two-Way Data-Binding\ped{G}} tra la view \texttt{StringsSortingQuestionView} e il controller \texttt{StringsSortingQuestionController}; 
		\item \textit{OUT} \texttt{ImagesSortingQuestionsModelView}: permette all'utente di creare una domanda a ordinamento immagini compilando i campi proposticlasse di tipo modelview la cui istanziazione è contenuta all'interno della variabile di ambiente \$scope di \textit{Angular.js\ped{G}}. All'interno di essa sono presenti le variabili e i metodi necessari per il \textit{Two-Way Data-Binding\ped{G}} tra la view \texttt{ImagesSortingQuestionView} e il controller \texttt{ImagesSortingQuestionController};
	\end{itemize}
	\item \textbf{Attributi}:
	\begin{itemize}
		\item \texttt{-} \texttt{\$scope: \$scope} \\
		Campo dati contenente un riferimento all’oggetto \$scope creato da \textit{Angular\ped{G}}, viene utilizzato come mezzo di comunicazione tra il controller e la view. Contiene gli oggetti che definiscono il model dell’applicazione.
	\end{itemize}
	\item \textbf{Metodi}:
	\begin{itemize}
		\item \texttt{+} \texttt{InputToListController(\$scope: \$scope)} \\Metodo costruttore della classe. \\
		\textbf{Parametri}:
		\begin{itemize}
			\item \texttt{-} \texttt{\$scope: \$scope} \\
			Parametro contenente un riferimento all’oggetto \$scope creato da \textit{Angular\ped{G}}, viene utilizzato come mezzo di comunicazione tra il controller e la view. Contiene gli oggetti che definiscono il model dell’applicazione. 
		\end{itemize}
		\item \texttt{-} \texttt{putDownAnswer(): void} \\Metodo che reagisce all'evento di aggiunta nuova risposta e la salva in modo che venga caricata e visualizzata nella pagina.
	\end{itemize}
\end{itemize}


\paragraph{QuizziPedia::Front-End::Controllers::LoginController}
\begin{figure} [ht]
	\centering
	\includegraphics[scale=0.60]{UML/Classi/Front-End/QuizziPedia_Front-end_Controller_LoginController.png}
	\caption{QuizziPedia::Front-End::Controllers::LoginController}
\end{figure} \FloatBarrier
\begin{itemize}
	\item \textbf{Descrizione}: questa classe permette di gestire l'autenticazione dell'utente al sistema; 
	\item \textbf{Utilizzo}: fornisce le funzionalità di autenticazione al sistema, compresa la gestione di situazioni di errore di autenticazione;
	\item \textbf{Relazione con altre classi}:
	\begin{itemize}
		\item \textbf{IN} \texttt{LoginModelView}: classe di tipo modelview la cui istanziazione è contenuta all'interno della variabile di ambiente \$scope di \textit{Angular\ped{G}}. All'interno di essa sono presenti le variabili e i metodi necessari per il \textit{Two-Way Data-Binding\ped{G}} tra la \textit{view\ped{G}} \texttt{LoginView} e il \textit{controller\ped{G}} \texttt{LoginController};
		\item \textbf{IN} \texttt{AuthService}: questa classe permette di gestire la registrazione e l'autenticazione di un utente;
	\end{itemize}
	\item \textbf{Attributi}:
	\begin{itemize}
		\item \texttt{-} \texttt{\$scope: \$scope} \\
		Campo dati contenente un riferimento all’oggetto \$scope creato da \textit{Angular\ped{G}}. Viene utilizzato come mezzo di comunicazione tra il \textit{controller\ped{G}} e la \textit{view\ped{G}}. Contiene gli oggetti che definiscono il viewmodel e il \textit{model\ped{G}} dell’applicazione;
		\item \texttt{-} \texttt{\$location: \$location} \\
		Campo dati contenente un riferimento al servizio creato da \textit{Angular\ped{G}} che permette di accedere alla barra degli indirizzi del \textit{browser\ped{G}}, i cambiamenti all’URL nella barra degli indirizzi si riflettono in questo oggetto e viceversa;
		\item \texttt{-} \texttt{\$mdDialog: \$mdDialog} \\
		Campo dati contenente un riferimento al servizio della libreria \textit{Material for Angular\ped{G}} che permette di creare delle componenti a pop-up;
		\item \texttt{-} \texttt{AuthService: AuthService} \\
		Campo dati contenente un riferimento al servizio che si occupa della gestione delle informazioni legate all'autenticazione. Viene utilizzato il metodo \texttt{logIn()} di \$texttt{AuthService} a cui vengono passati i parametri \texttt{username} e \texttt{password};
		\item \texttt{+} \texttt{user: LoginModelView} \\
		Oggetto di tipo \texttt{LoginModelView}. All'interno di esso sono presenti le variabili e i metodi necessari per il \textit{Two-Way Data-Binding\ped{G}} tra la \textit{view\ped{G}} \texttt{LoginView} e il \textit{controller\ped{G}} \texttt{LoginController};
	\end{itemize}
	\item \textbf{Metodi}:
	\begin{itemize}
		\item \texttt{+} \texttt{LoginController(\$scope: \$scope, \$rootScope: \$rootScope, \$location:\\ \$location, \$mdDialog: \$mdDialog, AuthService: AuthService)} \\
		Metodo costruttore della classe. \\
		\textbf{Parametri}:
			\begin{itemize}
				\item \texttt{\$scope: \$scope} \\
				Parametro contenente un riferimento all’oggetto \$scope creato da \textit{Angular\ped{G}}. Viene utilizzato come mezzo di comunicazione tra il \textit{controller\ped{G}} e la \textit{view\ped{G}}. Contiene gli oggetti che definiscono il viewmodel e il \textit{model\ped{G}} dell'applicazione;
				\item \texttt{\$location: \$location} \\
				Parametro contenente un riferimento al servizio creato da \textit{Angular\ped{G}} che permette di accedere alla barra degli indirizzi del \textit{browser\ped{G}}, i cambiamenti all’URL nella barra degli indirizzi si riflettono in questo oggetto e viceversa;
				\item \texttt{\$mdDialog: \$mdDialog} \\
				Parametro contenente un riferimento al servizio della libreria \textit{Material for Angular\ped{G}} che permette di creare delle componenti a pop-up;
				\item \texttt{AuthService: AuthService} \\
				Parametro contenente un riferimento al servizio che si occupa della gestione delle informazioni legate all'autenticazione. Viene utilizzato il metodo \texttt{logIn()} di \$texttt{AuthService} a cui vengono passati i parametri \texttt{email} e \texttt{password};
			\end{itemize}
		\item \texttt{+} \texttt{logIn(email: String, password: String): void} \\
		Metodo che richiama il metodo \texttt{signin} del service \texttt{AuthService} passandogli \texttt{email} e \texttt{password}. Nel caso di buona riuscita dell'operazione viene effettuato il redirect alla homepage dell'applicazione. Nel caso in cui invece avvenga un errore, viene mostrato a video il messaggio di errore;
		\item \texttt{+} \texttt{signUp(): void} \\
		Metodo che gestisce l'evento click sul pulsante di registrazione. Effettua il redirect alla pagina di registrazione;
		\item \texttt{+} \texttt{recoveryPassword(): void} \\
		Metodo che gestisce l'evento click sul pulsante di recupero password. Effettua il redirect alla pagina per il recupero della password; 
	\end{itemize}
\end{itemize}


\paragraph{QuizziPedia::Front-End::Controllers::MenuBarController}
\begin{figure} [ht]
	\centering
	\includegraphics[scale=0.45]{UML/Classi/Front-End/QuizziPedia_Front-end_Controller_MenuBarController.png}
	\caption{QuizziPedia::Front-End::Controllers::MenuBarController}
\end{figure} \FloatBarrier
\begin{itemize}
	\item \textbf{Descrizione}: questa classe permette di gestire il menù fisso per ogni pagina;
	\item \textbf{Utilizzo}: fornisce le funzionalità per aggiornare, a seconda della pagina, il contenuto del menù;
	\item \textbf{Relazione con altre classi}:
	\begin{itemize}
		\item \textbf{IN} \texttt{MenuBarModelView}: classe di tipo \textit{modelview\ped{G}} la cui istanziazione è contenuta all'interno della variabile di ambiente \$scope di \textit{Angular\ped{G}}. All'interno di essa sono presenti le variabili e i metodi necessari per il \textit{Two-Way Data-Binding\ped{G}} tra la \textit{directive\ped{G}} \texttt{MenuBarDirective} e il \textit{controller\ped{G}} \texttt{MenuBarController}. Rappresenta il menù, presente in ogni pagina dell'applicazione, generato in base agli oggetti passati nello \$scope. Fornisce un pulsante per ogni oggetto ricevuto come parametro, ogni pulsante viene rappresentato con un'icona e con un testo. Al click di un pulsante viene invocata la funzione ad esso associata; 
		\item \textbf{IN} \texttt{AuthService}: questa classe permette di gestire la registrazione e l'autenticazione di un utente;
		\item \textbf{IN} \texttt{MenuBarModel}: questa classe rappresenta la classe che contiene le informazioni per la giusta visualizzazione della barra.
	\end{itemize}
	\item \textbf{Attributi}:
	\begin{itemize}
		\item \texttt{-} \texttt{\$scope: \$scope} \\
		Campo dati contenente un riferimento all'oggetto \$scope creato da \textit{Angular\ped{G}}, viene utilizzato come mezzo di comunicazione tra il \textit{controller\ped{G}} e la \textit{view\ped{G}}. Contiene gli oggetti che definiscono il \textit{model\ped{G}} dell’applicazione;
		\item \texttt{-} \texttt{\$rootScope: \$rootScope} \\
		Campo dati contenente il riferimento all'oggetto globale \$rootScope creato da \textit{Angular\ped{G}};
		\item \texttt{-} \texttt{\$location: \$location} \\
		Campo dati contenente un riferimento al servizio creato da \textit{Angular\ped{G}} che permette di accedere alla barra degli indirizzi del \textit{browser\ped{G}}, i cambiamenti all'URL nella barra degli indirizzi si riflettono in questo oggetto e viceversa; 
		\item \texttt{-} \texttt{\$mdDialog: \$mdDialog} \\
		Campo dati contenente un riferimento al servizio della libreria \textit{Material for Angular\ped{G}} che permette di creare delle componenti a pop-up;
		\item \texttt{-} \texttt{AuthService: AuthService} \\
		Campo dati contenente un riferimento al servizio che si occupa della gestione delle informazioni legate all'autenticazione;
		\item \texttt{-} \texttt{menuBarModel: MenuBarModel}: \\
		Campo dati contenente un riferimento all'oggetto che contiene le informazioni per la giusta visualizzazione della barra;
		\item \rootscopeA;
		\item \errorinfomodelA;
		\item \userdetailsmodelA;
		\item \mdBottomSheetA;
	\end{itemize}
	\item \textbf{Metodi}:
	\begin{itemize}
		\item \texttt{+} \texttt{MenuBarController(\$scope: \$scope, \$location: \$location, \\ \$mdDialog:  \$mdDialog, AuthService: AuthService, MenuBarModel: MenuBarModel, ErrorInfoModel, UserDetailsModel, \$mdBottomSheet, \$rootScope: \$rootScope)} \\
		Metodo costruttore della classe. \\
		\textbf{Parametri}:
		\begin{itemize}
			\item \texttt{\$scope: \$scope} \\
			Parametro contenente un riferimento all’oggetto \$scope creato da \textit{Angular\ped{G}}. Viene utilizzato come mezzo di comunicazione tra il \textit{controller\ped{G}} e la \textit{view\ped{G}}. Contiene gli oggetti che definiscono il viewmodel e il \textit{model\ped{G}} dell'applicazione;
			\item \texttt{\$rootScope: \$rootScope} \\
			Campo dati contenente il riferimento all'oggetto globale \$rootScope creato da \textit{Angular\ped{G}};
			\item \texttt{\$location: \$location} \\
			Parametro contenente un riferimento al servizio creato da \textit{Angular\ped{G}} che permette di accedere alla barra degli indirizzi del \textit{browser\ped{G}}, i cambiamenti all’URL nella barra degli indirizzi si riflettono in questo oggetto e viceversa;
			\item \texttt{\$mdDialog: \$mdDialog} \\
			Parametro contenente un riferimento al servizio della libreria \textit{Material for Angular\ped{G}} che permette di creare delle componenti a pop-up;
			\item \texttt{AuthService: AuthService} \\
			Parametro contenente un riferimento al servizio che si occupa della gestione delle informazioni legate all'autenticazione.  Viene utilizzato il metodo \texttt{logOut} di \$texttt{AuthService} a cui viene passato il parametro \texttt{username};
			\item \texttt{MenuBarModel: MenuBarModel}: \\
			Parametro contenente un riferimento all'oggetto che contiene le informazioni per la giusta visualizzazione della barra;
			\item \rootscopeP;
			\item \errorinfomodelP;
			\item \userdetailsmodelP;
			\item \mdBottomSheetP;
		\end{itemize}
		\item \texttt{+} \texttt{logOut(): void} \\
		Metodo che richiama il metodo \texttt{logOut} del service \texttt{AuthService} passandogli lo \texttt{username}. Prima di effettuare questa operazione viene mostrato a video un messaggio di conferma per il proseguo dell'operazione; 
		\item \texttt{+} \texttt{logIn(): void} \\
		Metodo che gestisce l'evento click sul pulsante per effettuare il login. Effettua il redirect alla pagina per effettuare il login; 
		\item \texttt{+} \texttt{signUp(): void} \\
		Metodo che gestisce l'evento click sul pulsante per effettuare la registrazione. Effettua il redirect alla pagina per effettuare la registrazione; 
		\item \texttt{+} \texttt{goToUserPage(): void} \\
		Metodo che gestisce l'evento click sul pulsante di visualizzazione della pagina utente. Effettua il redirect alla pagina di visualizzazione della pagina utente; 
		\item \texttt{+} \texttt{goToUserManagemetPage(): void} \\
		Metodo che gestisce l'evento click sul pulsante di gestione del profilo utente. Effettua il redirect alla pagina di gestione del profilo utente; 
		\item \texttt{+} \texttt{goToQuestionsManagementPage(): void} \\
		Metodo che gestisce l'evento click sul pulsante di gestione delle domande. Effettua il redirect alla pagina di gestione delle domande; 
		\item \texttt{+} \texttt{goToQuizManagementPage(): void} \\
		Metodo che gestisce l'evento click sul pulsante di gestione dei questionari. Effettua il redirect alla pagina di gestione dei questionari;
		\item \texttt{+ \$on('\$routeChangeStart', function(next, current)): void} \\
		Metodo che cattura i cambiamenti dell'url e che richiede al \texttt{MenuBarModel} le giuste direttive da inserire in \texttt{MenuBarDirective}.\\
		\textbf{Parametri};
		\begin{itemize}
			\item \texttt{'\$routeChangeStart': String}	\\ Servizio offerto da \textit{Angular.js\ped{G}} per catturare gli eventi sulla barra degli URL;
			\item \texttt{callback: function}	\\ Funzione di callback per gestire i cambiamenti della barra degli indirizzi.
		\end{itemize}
		\item \texttt{+} \texttt{backToHome(): void} \\
		Metodo che gestisce l'evento per effettuare il redirect alla pagina principale dell'applicazione; 
		\item \texttt{+} \texttt{showListBottomSheet(): void} \\
		Metodo che gestisce l'evento per mostrare il menù a tendina per scegliere con quale lingua cambiare l'intera applicazione; 
		\item \texttt{+} \texttt{isOpenRight(): void} \\
		Metodo che gestisce l'evento per aprire e chiudere la barra di menù dell'applicazione principale; 
		\item \texttt{-} \texttt{debounce(func: Object, wait: Object, context: Object): void} \\
		Metodo ausiliario che gestisce l'evento per aprire e chiudere la barra di menù dell'applicazione principale; 
		\item \texttt{-} \texttt{buildDelayedToggler(navID: Object): void} \\
		Metodo ausiliario che gestisce l'evento per aprire e chiudere la barra di menù dell'applicazione principale; 
		\item \texttt{-} \texttt{buildToggler(navID: Object): void} \\
		Metodo ausiliario che gestisce l'evento per aprire e chiudere la barra di menù dell'applicazione principale;
	\end{itemize}
	
\end{itemize}
\input{sezioni/Front-End/Controllers/QuizziPedia_Front-End_Controllers_MultipleQuestionsController.tex}
\paragraph[QuizziPedia::Front-End::Controllers\\::NewQuestionsButtonController]{QuizziPedia::Front-End::Controllers::NewQuestionsButtonController}
\begin{figure} [ht]
	\centering
	\includegraphics[scale=0.8]{UML/Classi/Front-End/QuizziPedia_Front-end_Controller_NewQuestionsButtonController.png}
	\caption{QuizziPedia::Front-End::Controllers::NewQuestionsButtonController}
\end{figure} \FloatBarrier
\begin{itemize}
	\item \textbf{Descrizione}: questa classe permette di effettuare il redirect alla pagina di creazione nuova domanda;
	\item \textbf{Utilizzo}: effettua il redirect alla pagina di creazione di una nuova domanda quando l'utente seleziona interagisce con il bottone a cui è collegato il corrispettivo evento;
	\item \textbf{Relazione con altre classi}:
	\begin{itemize}
		\item \textbf{OUT \texttt{NewQuestionsButtonDirective}}: rappresenta il componente grafico che permette all'utente di posizionarsi nella \textit{view\ped{G}} di creazione di una nuova domanda.
	\end{itemize}
	\item \textbf{Attributi}:
	\begin{itemize}
		\item \texttt{-} \texttt{\$scope: \$scope} \\
		Campo dati contenente un riferimento all'oggetto \$scope creato da \textit{Angular\ped{G}}, viene utilizzato come mezzo di comunicazione tra il \textit{controller\ped{G}} e la \textit{view\ped{G}}. Contiene gli oggetti che definiscono il \textit{model\ped{G}} dell'applicazione;
		\item \texttt{-} \texttt{\$location: \$location} \\
		Campo dati contenente un riferimento al servizio creato da \textit{Angular\ped{G}} che permette di accedere alla barra degli indirizzi del \textit{browser\ped{G}}, i cambiamenti all'URL nella barra degli indirizzi si riflettono in questo oggetto e viceversa.
	\end{itemize}
	\item \textbf{Metodi}:
	\begin{itemize}
		\item \texttt{+} \texttt{NewQuestionButtonsController(\$scope: \$scope, \$location: \$location)} \\ 
		Metodo costruttore della classe. \\
		\textbf{Parametri}:
		\begin{itemize}
			\item \texttt{\$scope: \$scope} \\
			Parametro contenente un riferimento all'oggetto \$scope creato da \textit{Angular\ped{G}}. Viene utilizzato come mezzo di comunicazione tra il \textit{controller\ped{G}} e la \textit{view\ped{G}}. Contiene gli oggetti che definiscono il \textit{viewmodel\ped{G}} e il \textit{model\ped{G}} dell'applicazione;
			\item \texttt{\$location: \$location} \\
			Parametro contenente un riferimento al servizio creato da \textit{Angular\ped{G}} che permette di accedere alla barra degli indirizzi del \textit{browser\ped{G}}, i cambiamenti all'URL nella barra degli indirizzi si riflettono in questo oggetto e viceversa.
		\end{itemize}
		\item \texttt{+} \texttt{newQuestion(): void} \\ 
		Metodo che gestisce l'evento click sul pulsante per creare una nuova domanda. Effettua il redirect alla pagina di creazione di una domanda.
	\end{itemize}
	
\end{itemize}


\paragraph{QuizziPedia::Front-End::Controllers::PasswordForgotController}
\begin{figure} [ht]
	\centering
	\includegraphics[scale=0.7]{UML/Classi/Front-End/QuizziPedia_Front-end_Controller_PasswordForgotController.png}
	\caption{QuizziPedia::Front-End::Controllers::PasswordForgotController}
\end{figure} \FloatBarrier
\begin{itemize}
	\item \textbf{Descrizione}: questa classe permette di gestire il ripristino della password dimenticata;
	\item \textbf{Utilizzo}: fornisce tutte le funzionalità per ripristinare la password dopo aver verificato l'identità dell'utente;
	\item \textbf{Relazione con altre classi}:
	\begin{itemize}
		\item \textbf{IN \texttt{PasswordForgotModelView}}: classe di tipo \textit{modelview\ped{G}} la cui istanziazione è contenuta all'interno della variabile di ambiente \$scope di \textit{Angular\ped{G}}. All'interno di essa sono presenti le variabili e i metodi necessari per il \textit{Two-Way Data-Binding\ped{G}} tra la \textit{view\ped{G}} \texttt{PasswordForgotView} e il \textit{controller\ped{G}} \texttt{PasswordForgotController};
		\item \textbf{OUT \texttt{AuthService}}: questa classe permette di gestire la registrazione e l'autenticazione di un utente.
	\end{itemize}
	\item \textbf{Attributi}:
	\begin{itemize}
		\item \texttt{-} \texttt{\$scope: \$scope} \\
		Campo dati contenente un riferimento all'oggetto \$scope creato da \textit{Angular\ped{G}}, viene utilizzato come mezzo di comunicazione tra il \textit{controller\ped{G}} e la \textit{view\ped{G}}. Contiene gli oggetti che definiscono il \textit{model\ped{G}} dell'applicazione;
		\item \texttt{-} \texttt{\$location: \$location} \\
		Campo dati contenente un riferimento al servizio creato da \textit{Angular\ped{G}} che permette di accedere alla barra degli indirizzi del \textit{browser\ped{G}}, i cambiamenti all'URL nella barra degli indirizzi si riflettono in questo oggetto e viceversa;
		\item \texttt{-} \texttt{\$mdDialog: \$mdDialog} \\
		Campo dati contenente un riferimento al servizio della libreria \textit{Material for Angular\ped{G}} che permette di creare delle componenti a pop-up;
		\item \texttt{-} \texttt{AuthService: AuthService} \\
		Campo dati contenente un riferimento al servizio che si occupa della gestione delle informazioni legate all'autenticazione. Viene utilizzato il metodo \texttt{passwordForgot} di \texttt{AuthService} a cui viene passato il parametro \texttt{email}.
	\end{itemize}
	\item \textbf{Metodi}:
	\begin{itemize}
		\item \texttt{+} \texttt{PasswordForgotController(\$scope: \$scope, \$location: \$location, \$mdDialog: \$mdDialog, AuthService: AuthService)} \\
		Metodo costruttore della classe. \\
		\textbf{Parametri}:
		\begin{itemize}
			\item \texttt{\$scope: \$scope} \\
			Parametro contenente un riferimento all'oggetto \$scope creato da \textit{Angular\ped{G}}. Viene utilizzato come mezzo di comunicazione tra il \textit{controller\ped{G}} e la \textit{view\ped{G}}. Contiene gli oggetti che definiscono il \textit{viewmodel\ped{G}} e il \textit{model\ped{G}} dell'applicazione;
			\item \texttt{\$location: \$location} \\
			Parametro contenente un riferimento al servizio creato da \textit{Angular\ped{G}} che permette di accedere alla barra degli indirizzi del \textit{browser\ped{G}}, i cambiamenti all'URL nella barra degli indirizzi si riflettono in questo oggetto e viceversa;
			\item \texttt{\$mdDialog: \$mdDialog} \\
			Parametro contenente un riferimento al servizio della libreria \textit{Material for Angular\ped{G}} che permette di creare delle componenti a pop-up;
			\item \texttt{AuthService: AuthService} \\
			Campo dati contenente un riferimento al servizio che si occupa della gestione delle informazioni legate all'autenticazione. Viene utilizzato il metodo \texttt{passwordForgot} di \texttt{AuthService} a cui viene passato il parametro \texttt{email}.
		\end{itemize}
		\item \texttt{+} \texttt{passwordForgot(): void} \\
		Metodo che richiama il metodo \texttt{getNewPassword} del \textit{service\ped{G}} \texttt{AuthService} passandogli il parametro \texttt{email}. Nel caso di buona riuscita dell'operazione, viene mostrato un messaggio di successo il cui corpo contiene anche un bottone per effettuare il redirect alla pagina di login. Nel caso in cui invece avvenga un errore, viene mostrato a video il messaggio di errore;
		\item \texttt{+} \texttt{logIn(): void} \\
		Metodo che gestisce l'evento click sul pulsante di login. Effettua il redirect alla pagina di login.
	\end{itemize}
\end{itemize}


\paragraph{QuizziPedia::Front-End::Controllers::ProfileManagementController}
\begin{figure} [ht]
	\centering
	\includegraphics[scale=0.80]{UML/Classi/Front-End/QuizziPedia_Front-end_Controller_ProfileManagementController.png}
	\caption{QuizziPedia::Front-End::Controllers::ProfileManagementController}
\end{figure} \FloatBarrier
\begin{itemize}
	\item \textbf{Descrizione}: questa classe permette di gestire il profilo personale di un utente; 
	\item \textbf{Utilizzo}: fornisce le funzionalità all'utente per poter gestire i propri dati;
	\item \textbf{Relazione con altre classi}:
	\begin{itemize}
		\item \textit{IN} \texttt{ProfileManagementModelView}: classe di tipo modelview la cui istanziazione è contenuta all'interno della variabile di ambiente \$scope di \textit{Angular.js\ped{G}}. All'interno di essa sono presenti le variabili e i metodi necessari per il \textit{Two-Way Data-Binding\ped{G}} tra la view \texttt{ProfileManagementView} e il controller \texttt{ProfileManagementController};
		\item \textit{IN} \texttt{UserDetailsService}: questa classe permette di ottenere i dati personali degli utenti;
		\item \textit{IN} \texttt{UserDetailsModel}: questa classe rappresenta il tipo dell'utente autenticato della pagina; 
	\end{itemize}
	\item \textbf{Attributi}:
	\begin{itemize}
		\item \texttt{-} \texttt{\$scope: \$scope} \\
		Campo dati contenente un riferimento all’oggetto \$scope creato da \textit{Angular\ped{G}}, viene utilizzato come mezzo di comunicazione tra il controller e la view. Contiene gli oggetti che definiscono il model dell’applicazione;
		\item \texttt{-} \texttt{\$rootScope: \$rootScope} \\
		Campo dati contenente il riferimento all'oggetto globale \$rootScope creato da \textit{Angular\ped{G}}. Viene utilizzato per rendere accessibile a tutti i controller e a tutte le view l'oggetto \texttt{UserDetailsModel};
		\item \texttt{-} \texttt{\$mdDialog: \$mdDialog} \\
		Campo dati contenente un riferimento al servizio della libreria \textit{Material for Angular\ped{G}} che permette di creare delle componenti a popup;		
		\item \texttt{-} \texttt{UserDetailsService: UserDetailsService}: \\
		Campo dati contenente un riferimento al servizio che si occupa della gestione delle informazioni legate agli utenti;
		\item \texttt{+} \texttt{user: UserDetailsModel}: \\
		Oggetto di tipo \texttt{UserDetailsModel}. Viene mantenuto all'interno del \$rootScope; 
		\item \texttt{-} \texttt{Upload: Upload} \\
		Campo dati contenente un riferimento alla libreria \textit{ng-file-upload\ped{G}} necessaria per il caricamento della foto profilo dell'utente;
		\item \texttt{-} \texttt{\$timeout: \$timeout} \\
		Campo dati contenente il riferimento all'oggetto globale \$timeout creato da \textit{Angular.js\ped{G}}. 
		Il valore di ritorno di una chiamata alla funzione di \texttt{\$timeout} è una promise, la quale sarà risolta quando avverrà il ritardo e la funzione di timeout eseguita; 
	\end{itemize}
	\item \textbf{Metodi}:
	\begin{itemize}
		\item \texttt{+} \texttt{ProfileManagementController(\$scope: \$scope, \$rootScope: \$rootScope, \$mdDialog: \$mdDialog, UserDetailsService: UserDetailsService)} \\
		Metodo costruttore della classe. \\
		\textbf{Parametri}:
		\begin{itemize}
			\item \texttt{\$scope: \$scope} \\
			Parametro contenente un riferimento all’oggetto \$scope creato da \textit{Angular\ped{G}}. Viene utilizzato come mezzo di comunicazione tra il controller e la view. Contiene gli oggetti che definiscono il viewmodel e il model dell’applicazione;
			\item \texttt{\$rootScope: \$rootScope} \\
			Parametro contenente il riferimento all'oggetto globale \$rootScope creato da \textit{Angular\ped{G}}. Viene utilizzato per rendere accessibile a tutti i controller e a tutte le view l'oggetto \texttt{UserDetailsModel}. In questo caso viene utilizzato per aggiornare in \$rootScope l'oggetto che rappresenta l'utente autenticato all'interno dell'applicazione;
			\item \texttt{\$location: \$location} \\
			Parametro contenente un riferimento al servizio creato da \textit{Angular\ped{G}} che permette di accedere alla barra degli indirizzi del \textit{browser\ped{G}}, i cambiamenti all’URL nella barra degli indirizzi si riflettono in questo oggetto e viceversa;
			\item \texttt{\$mdDialog: \$mdDialog} \\
			Parametro contenente un riferimento al servizio della libreria \textit{Material for Angular\ped{G}} che permette di creare delle componenti a popup;
			\item \texttt{UserDetailsService: UserDetailsService} \\
			Parametro contenente un riferimento al servizio che si occupa della gestione delle informazioni legate all’utente;
			\item \texttt{Upload: Upload} \\
			Parametro contenente un riferimento alla libreria \textit{ng-file-upload} necessaria per il caricamento della foto profilo dell'utente;
			\item \texttt{\$timeout: \$timeout} \\
			Parametro contenente il riferimento all'oggetto globale \$timeout creato da \textit{Angular.js\ped{G}}. 
			Il valore di ritorno di una chiamata alla funzione di \texttt{\$timeout} è una promise, la quale sarà risolta quando avverrà il ritardo e la funzione di timeout eseguita; 
		\end{itemize}
		\item \texttt{+} \texttt{confirm(user: Object, imageObject: Object): void} \\
		Metodo che gestisce l’evento click sul pulsante di conferma modifica. Aggiorna, in caso di modifiche, l'oggetto locale \texttt{UserDetailsModel}. Inoltre, utilizzando il metodo dell'\texttt{UserDetailsService}, aggiorna anche nel server i dati dell'utente.
	\end{itemize}
\end{itemize}


\paragraph[QuizziPedia::Front-End::Controllers\\::QuestionnaireDetailsController]{QuizziPedia::Front-End::Controllers::QuestionnaireDetailsController}
\begin{figure} [ht]
	\centering
	\includegraphics[scale=0.6]{UML/Classi/Front-End/QuizziPedia_Front-end_Controller_QuestionnaireDetailsController.png}
	\caption{QuizziPedia::Front-End::Controllers::QuestionnaireDetailsController}
\end{figure} \FloatBarrier
\begin{itemize}
	\item \textbf{Descrizione}: questa classe permette di gestire i dettagli di un questionario; 
	\item \textbf{Utilizzo}: fornisce le funzionalità per recuperare dal back-end i dettagli di un questionario creato da un utente al fine di poterli visualizzare nel suo profilo;
	\item \textbf{Relazione con altre classi}:
	\begin{itemize}
		\item \textbf{IN \texttt{QuestionnaireDetailsModelView}}: classe di tipo \textit{modelview\ped{G}} la cui istanziazione è contenuta all'interno della variabile di ambiente \$scope di \textit{Angular\ped{G}}. All'interno di essa sono presenti le variabili e i metodi necessari per il \textit{Two-Way Data-Binding\ped{G}} tra la \textit{view\ped{G}} \texttt{UserView} e il \textit{controller\ped{G}} \texttt{QuestionnaireDetailsController};
		\item \textbf{IN \texttt{QuizService}}: questa classe permette di ottenere i dati di un quiz tramite delle parole chiave inserite dall'utente nella barra di ricerca;
		\item \textbf{IN \texttt{QuestionnaireModel}}: rappresenta un questionario. Contiene tutte le informazioni necessarie alla presentazione del contenuto del questionario.
	\end{itemize}
	\item \textbf{Attributi}:
	\begin{itemize}
		\item \texttt{-} \texttt{\$scope: \$scope} \\
		Campo dati contenente un riferimento all'oggetto \$scope creato da \textit{Angular\ped{G}}, viene utilizzato come mezzo di comunicazione tra il \textit{controller\ped{G}} e la \textit{view\ped{G}}. Contiene gli oggetti che definiscono il \textit{model\ped{G}} dell'applicazione;
		\item \texttt{-} \texttt{\$rootScope: \$rootScope} \\
		Campo dati contenente il riferimento all'oggetto globale \$rootScope creato da \textit{Angular\ped{G}}. Viene utilizzato per rendere accessibile a tutti i \textit{controller\ped{G}} e a tutte le \textit{view\ped{G}} l'oggetto \texttt{QuestionnaireModel}. In questo caso viene utilizzato per inserire in \$rootScope l'oggetto di ritorno della chiamata a \texttt{getQuestionnaireDetails} del \textit{service\ped{G}} \texttt{QuizService};
		\item \texttt{-} \texttt{QuizService: QuizService} \\ Questa classe permette di ottenere i dati di un quiz tramite delle parole chiave inserite dall'utente nella barra di ricerca;
		\item \texttt{-} \texttt{\$mdDialog: \$mdDialog} \\
		Parametro contenente un riferimento al servizio della libreria \textit{Material for Angular\ped{G}} che permette di creare delle componenti a pop-up;
		\item \texttt{+} \texttt{details: QuestionnaireDetailsModelView} \\
		Oggetto di tipo \texttt{QuestionnaireDetailsModelView}. All'interno di esso sono presenti le variabili e i metodi necessari per il \textit{Two-Way Data-Binding\ped{G}} tra la \textit{view\ped{G}} \texttt{UserView} e il \textit{controller\ped{G}} \texttt{QuestionnaireDetailsController}.
	\end{itemize}
	\item \textbf{Metodi}:
	\begin{itemize}
		\item \texttt{+} \texttt{QuestionnaireDetailsController(\$scope: \$scope, \$rootScope: \$rootScope, \$mdDialog: \$mdDialog, QuizService: QuizService)} \\ Metodo costruttore della classe. \\
		\textbf{Parametri}: 
		\begin{itemize}
			\item \texttt{\$scope: \$scope} \\
			Parametro contenente un riferimento all'oggetto \$scope creato da \textit{Angular\ped{G}}. Viene utilizzato come mezzo di comunicazione tra il \textit{controller\ped{G}} e la \textit{view\ped{G}}. Contiene gli oggetti che definiscono il \textit{viewmodel\ped{G}} e il \textit{model\ped{G}} dell'applicazione;
			\item \texttt{\$rootScope: \$rootScope} \\
			Parametro contenente il riferimento all'oggetto globale \$rootScope creato da \textit{Angular\ped{G}}. Viene utilizzato per rendere accessibile a tutti i \textit{controller\ped{G}} e a tutte le \textit{view\ped{G}} l'oggetto \texttt{QuestionnaireModel}. In questo caso viene utilizzato per inserire in \$rootScope l'oggetto di ritorno della chiamata a \texttt{getQuestionnaireDetails} del \textit{service\ped{G}} \texttt{QuizService};	
			\item \texttt{\$mdDialog: \$mdDialog} \\
			Parametro contenente un riferimento al servizio della libreria \textit{Material for Angular\ped{G}} che permette di creare delle componenti a pop-up;
			\item \texttt{QuizService: QuizService}\\ Parametro che permette di ottenere, tramite il \textit{service\ped{G}}, la lista di tutte le domande presenti nel quiz.
		\end{itemize}
		
		\item \texttt{-} \texttt{getQuestionnaireDetails(username: String): Object} \\ Metodo che richiede al \textit{service\ped{G}} i dettagli dei questionari eseguiti dall'utente. \\
		\textbf{Parametri}:
		\begin{itemize}
			\item \texttt{username: String}: username dell'utente del quale caricare i questionari.
		\end{itemize}
		\item \texttt{-} \texttt{getSubscribedQuestionnaire(username: String): Object} \\Metodo che ritorna i questionari a cui l'utente è iscritto. \\
		\textbf{Parametri}:
		\begin{itemize}
			\item \texttt{username: String}: username dell'utente del quale scaricare i questionari a cui è iscritto.
		\end{itemize}
	\end{itemize}
\end{itemize}


\paragraph[QuizziPedia::Front-End::Controllers\\::QuestionnaireManagementController]{QuizziPedia::Front-End::Controllers::QuestionnaireManagementController}
\begin{figure} [ht]
	\centering
	\includegraphics[scale=0.6]{UML/Classi/Front-End/QuizziPedia_Front-end_Controller_QuestionnaireManagementController.png}
	\caption{QuizziPedia::Front-End::Controllers::QuestionnaireManagementController}
\end{figure} \FloatBarrier
\begin{itemize}
	\item \textbf{Descrizione}: questa classe permette di gestire tutti i questionari creati da un utente; 
	\item \textbf{Utilizzo}: fornisce le funzionalità per recuperare dal back-end tutti i questionari creati da un utente;
	\item \textbf{Relazione con altre classi}:
	\begin{itemize}
		\item \textbf{IN \texttt{QuestionnaireManagementModelView}}: classe di tipo \textit{modelview\ped{G}} la cui istanziazione è contenuta all'interno della variabile di ambiente \$scope di \textit{Angular\ped{G}}. All'interno di essa sono presenti le variabili e i metodi necessari per il \textit{Two-Way Data-Binding\ped{G}} tra la \textit{view\ped{G}} \texttt{QuestionnaireManagementView} e il \textit{controller\ped{G}} \texttt{QuestionnaireManage-\\mentController};
		\item \textbf{IN \texttt{QuizService}}: questa classe permette di ottenere i dati di un quiz tramite delle parole chiave inserite dall'utente nella barra di ricerca;
		\item \textbf{IN \texttt{QuestionnaireModel}}: rappresenta un questionario. Contiene tutte le informazioni necessarie alla presentazione del contenuto del questionario;
	\end{itemize}
	\item \textbf{Attributi}:
	\begin{itemize}
		\item \texttt{-} \texttt{\$scope: \$scope} \\
		Campo dati contenente un riferimento all'oggetto \$scope creato da \textit{Angular\ped{G}}, viene utilizzato come mezzo di comunicazione tra il \textit{controller\ped{G}} e la \textit{view\ped{G}}. Contiene gli oggetti che definiscono il \textit{model\ped{G}} dell'applicazione;
		\item \texttt{-} \texttt{\$mdDialog: \$mdDialog} \\
		Parametro contenente un riferimento al servizio della libreria \textit{Material for Angular\ped{G}} che permette di creare delle componenti a pop-up;
		\item \texttt{-}	 \texttt{QuizService: QuizService}: permette di ottenere i dati di un quiz tramite delle parole chiave inserite dall'utente nella barra di ricerca;
		\item \rootscopeA;
		\item \routeparamsA;
		\item \locationA;
		\item \errorinfomodelA.
	\end{itemize}
	\item \textbf{Metodi}:
	\begin{itemize}
		\item \texttt{+} \texttt{QuestionnaireManagementController(\$scope: \$scope, \$mdDialog: \$mdDia-\\log, QuizService: QuizService, \$rootScope: \$rootScope, \$routeParams: \$routeParams, \$location: \$location, ErrorInfoModel: ErrorInfoModel)} \\Metodo costruttore della classe.\\
		\textbf{Parametri}: 
		\begin{itemize}
			\item \texttt{\$scope: \$scope} \\
			Campo dati contenente un riferimento all'oggetto \$scope creato da \textit{Angular\ped{G}}. Viene utilizzato come mezzo di comunicazione tra il \textit{controller\ped{G}} e la \textit{view\ped{G}}. Contiene gli oggetti che definiscono il \textit{viewmodel\ped{G}} e il \textit{model\ped{G}} dell'applicazione;
			\item \texttt{\$mdDialog: \$mdDialog} \\
			Campo dati contenente un riferimento al servizio della libreria \textit{Material for Angu-\\lar\ped{G}} che permette di creare delle componenti a pop-up;
			\item \texttt{QuizService: QuizService}: parametro che permette di ottenere, tramite il \textit{service\ped{G}}, la lista di tutte le domande presenti nel quiz;
			\item \rootscopeP;
			\item \routeparamsP;
			\item \locationP;
			\item \errorinfomodelP.
		\end{itemize}
		\item \texttt{+} \texttt{showAllQuizzes(username: String, lang: String): void} \\Metodo che ritorna tutti i questionari creati da un utente in un array di \texttt{Questionnair-\\eModel}. \\
		\textbf{Parametri}:
		\begin{itemize}
			\item \texttt{username: String}: parametro che indica l'username dell'utente del quale vogliamo scaricare tutti i questionari;
			\item \texttt{lang: String}: parametro che indica la lingua del sistema. 
		\end{itemize}
		\item \texttt{+} \texttt{goToQuiz(): void} \\Metodo che permette di andare alla pagina che gestisce le iscrizioni di un questionario;
		\item \texttt{+} \texttt{goToCreateQuestionnaire(): void} \\Metodo che permette di andare alla pagina che permette di creare un questionario.		
	\end{itemize}
\end{itemize}


\paragraph{QuizziPedia::Front-End::Controllers::QuestionsController}
\begin{figure} [ht]
	\centering
	\includegraphics[scale=0.45]{UML/Classi/Front-End/QuizziPedia_Front-end_Controller_QuestionsController.png}
	\caption{QuizziPedia::Front-End::Controllers::QuestionsController}
\end{figure} \FloatBarrier
\begin{itemize}
	\item \textbf{Descrizione}: questa classe permette di gestire il recupero delle domande per far si che possano essere visualizzate nella modalità allenamento e nella compilazione dei questionari;
	\item \textbf{Utilizzo}: fornisce le funzionalità per il recupero delle domande esistenti nel database al fine di poterle visualizzare nella modalità allenamento e nella compilazione dei questionari;
	\item \textbf{Relazione con altre classi}:
	\begin{itemize}
		\item \textit{IN} \texttt{QuestionsModelView}: classe di tipo modelview la cui istanziazione è contenuta all'interno della variabile di ambiente \$scope di \textit{Angular.js\ped{G}}. All'interno di essa sono presenti le variabili e i metodi necessari per il \textit{Two-Way Data-Binding\ped{G}} tra le directive che compongono dinamicamente la vista della domanda e il controller \texttt{QuestionsController};
		\item \textit{IN} \texttt{QuestionServices}: questa classe permette di ottenere domande esistenti e salvare nuove domande;
		\item \textit{IN} \texttt{QuestionItemModel}: rappresenta una domanda. Contiene tutte le informazioni necessarie alla presentazione del contenuto della domanda;
		\item \textit{OUT} \texttt{FillingQuestionnaireController}: questa classe permette di gestire la compilazione del questionario;
		\item \textit{OUT} \texttt{TrainingController}: questa classe permette di gestire la modalità allenamento sottoponendo all'utente le giuste domande adatte al suo livello;
	\end{itemize}
	\item \textbf{Attributi}:
	\begin{itemize}
		\item \texttt{-} \texttt{\$scope: \$scope} \\
		Campo dati contenente un riferimento all’oggetto \$scope creato da \textit{Angular\ped{G}}, viene utilizzato come mezzo di comunicazione tra il controller e la view. Contiene gli oggetti che definiscono il model dell’applicazione;
		\item \texttt{-} \texttt{\$rootScope: \$rootScope} \\
		Campo dati contenente il riferimento all'oggetto globale \$rootScope creato da \textit{Angular\ped{G}}. Viene utilizzato per rendere accessibile a tutti i controller e a tutte le view l'oggetto \texttt{QuestionItemModel}. In questo caso viene utilizzato per inserire in \$rootScope l'oggetto di ritorno della chiamata a \texttt{getQuestion} del service \texttt{QuestionsService};
		\item \texttt{-} \texttt{\$mdDialog: \$mdDialog} \\
		Campo dati contenente un riferimento al servizio della libreria \textit{Material for Angular\ped{G}} che permette di creare delle componenti a popup;
		\item \texttt{-} \texttt{QuestionService: QuestionsService}\\ Permette di ottenere domande esistenti tramite chiamata di metodo specifici;
		\item \texttt{-} \texttt{question: QuestionsModelView}:classe di tipo modelview la cui istanziazione è contenuta all'interno della variabile di ambiente \$scope di \textit{Angular.js\ped{G}}. All'interno di essa sono presenti le variabili e i metodi necessari per il \textit{Two-Way Data-Binding\ped{G}} tra le directive che compongono dinamicamente la vista della domanda e il controller \texttt{QuestionsController}.
	\end{itemize}
	\item \textbf{Metodi}:
	\begin{itemize}
		\item \texttt{+} \texttt{QuestionsController(\$scope: \$scope, \$rootScope: \$rootScope,\$mdDialog: \$mdDialog, QuestionService: QuestionService)}: \\ Metodo costruttore della classe. Interagendo con l'oggetto \texttt{QuestionItemModel} ricevuto si interfaccia con la view andando a visualizzare la giusta directive della tipologia di domanda. \\
		\textbf{Parametri}:
		\begin{itemize}
			\item \texttt{-} \texttt{\$scope: \$scope} \\
			Campo dati contenente un riferimento all’oggetto \$scope creato da \textit{Angular\ped{G}}. Viene utilizzato come mezzo di comunicazione tra il controller e la view. Contiene gli oggetti che definiscono il viewmodel e il model dell’applicazione;
			\item \texttt{-} \texttt{\$rootScope: \$rootScope} \\
			Parametro contenente il riferimento all'oggetto globale \$rootScope creato da \textit{Angular\ped{G}}. Viene utilizzato per rendere accessibile a tutti i controller e a tutte le view l'oggetto \texttt{QuestionItemModel}. In questo caso viene utilizzato per inserire in \$rootScope l'oggetto di ritorno della chiamata a \texttt{getQuestion} del service \texttt{QuestionsService}; 
			\item \texttt{-} \texttt{\$mdDialog: \$mdDialog} \\
			Parametro contenente un riferimento al servizio della libreria \textit{Material for Angular\ped{G}} che permette di creare delle componenti a popup;
			\item \texttt{QuestionService: QuestionService} \\ Parametro che permette di ottenere domande esistenti tramite chiamata di metodo specifici;
		\end{itemize}
		\item \texttt{-} \texttt{getQuestionBy(topic: String, keywords: Array[String], level: Number): QuestionItemModel} \\ Metodo che richiede al back-end una domanda. \\
		\textbf{Parametri}:
		\begin{itemize}
			\item \texttt{topic: String} \\
			Parametro contenente l'argomento della domanda;
			\item \texttt{keywords: Array[String]} \\
			Parametro contenente un\texttt{array} di stringhe che rappresenta le keywords scelte per l'allenamento;
			\item \texttt{level: Number} \\
			Parametro contenente il livello dell'utente.
		\end{itemize}
		\item \texttt{+} \texttt{addAnswer(index: Number, typeQuestion: String, answerGiven: Array[String]): void} \\
		Metodo che gestisce l'evento di selezione delle risposte. \\
		\textbf{Parametri}:
		\begin{itemize}
			\item \texttt{index: Number} \\
			Parametro contenente l'indice della risposta di cui si vuole tenere traccia. Rappresenta anche l'indice dell'\texttt{array objAswer} in cui verrà inserito l'oggetto delle risposte date;
			\item \texttt{typeQuestion: String} \\
			Parametro contenente una stringa la quale indica la tipologia della domanda;
			\item \texttt{answerGiven: Array[String]} \\
			Parametro contenente l'array di risposte date dall'utente aggiornato all'ultima iterazione.
		\end{itemize};
		\item \texttt{+} \texttt{answerGiven(index: Number): Array[String]} \\
		Metodo di supporto che ritorna un \texttt{array} di stringhe contenente le risposte date. Si occupa di recuperare le risposte date nelle domande vero/falso, risposta multipla e ad area cliccabile.\\
		\textbf{Parametri}:
		\begin{itemize}
			\item \texttt{index: Number} \\
			Parametro contenente l'indice della risposta di cui si vuole raccogliere le risposte date. 
		\end{itemize}
		\item \texttt{+} \texttt{orderChosen(index: Number): Array[String]} \\
		Metodo di supporto che ritorna un \texttt{array} di stringhe contenente le risposte date. Si occupa di recuperare le risposte date nelle domande ad ordinamento e di riempimento di spazi.\\
		\textbf{Parametri}:
		\begin{itemize}
			\item \texttt{index: Number} \\
			Parametro contenente l'indice della risposta di cui si vuole raccogliere le risposte date. 
		\end{itemize}
		\item \texttt{+} \texttt{linkingMade(index: Number): Array[String]} \\
		Metodo di supporto che ritorna un \texttt{array} di stringhe contenente le risposte date. Si occupa di recuperare le risposte date nelle domande a collegamento.\\
		\textbf{Parametri}:
		\begin{itemize}
			\item \texttt{index: Number} \\
			Parametro contenente l'indice della risposta di cui si vuole raccogliere le risposte date. 
		\end{itemize}
		\item \texttt{+} \texttt{loadNewQuestionBy(topic: String, keywords: Array[String], level: Number): void} \\
		Metodo che gestisce l'evento per scaricare una nuova domanda in base ai parametri passati. Evoca l'evento per inserire la domanda in \texttt{TrainingModelView}. \\
		\textbf{Parametri}:
		\begin{itemize}
			\item \texttt{topic: String} \\
			Parametro contenente l'argomento della domanda;
			\item \texttt{keywords: Array[String]} \\
			Parametro contenente un\texttt{array} di stringhe che rappresenta le keywords scelte per l'allenamento;
			\item \texttt{level: Number} \\
			Parametro contenente il livello dell'utente.
		\end{itemize}
		\item \texttt{+} \texttt{loadNewQuestion(question: QuestionItemModel): void} \\
		Metodo che gestisce l'evento per visualizzare una nuova domanda. \\
		\textbf{Parametri}:
		\begin{itemize}
			\item \texttt{question: QuestionItemModel} \\
			Parametro contenente un riferimento all'oggetto di tipo \texttt{QuestionItemModel}.
		\end{itemize}
		\item \texttt{+} \texttt{checkAnswer(): boolean} \\ 
		Metodo che controlla che le risposte date siano corrette.
	\end{itemize}
\end{itemize}


\paragraph[QuizziPedia::Front-End::Controllers\\::QuestionsManagementController]{QuizziPedia::Front-End::Controllers::QuestionsManagementController}
\begin{figure} [ht]
	\centering
	\includegraphics[scale=0.6]{UML/Classi/Front-End/QuizziPedia_Front-end_Controller_QuestionsManagementController.png}
	\caption{QuizziPedia::Front-End::Controllers::QuestionsManagementController}
\end{figure} \FloatBarrier
\begin{itemize}
	\item \textbf{Descrizione}: questa classe permette di gestire le domande create dall'utente e di crearne di nuove;
	\item \textbf{Utilizzo}: fornisce le funzionalità per richiedere al \textit{server\ped{G}} le domande create dall'utente e mostrarle nella pagina dedicata. Inoltre permette di catturare gli eventi per modificare le domande esistenti e per crearne una di nuova; 
	\item \textbf{Relazione con altre classi}:
	\begin{itemize}
		\item \textbf{IN \texttt{QuestionsManagementModelView}}: classe di tipo \textit{modelview\ped{G}} la cui istanziazione è contenuta all'interno della variabile di ambiente \$scope di \textit{Angular\ped{G}}. All'interno di essa sono presenti le variabili e i metodi necessari per il \textit{Two-Way Data-Binding\ped{G}} tra la \textit{view\ped{G}} \texttt{QuestionsManagementView} e il \textit{controller\ped{G}} \texttt{QuestionsManagementController}; 
		\item \textbf{IN \texttt{QuestionService}}: questa classe permette di ottenere domande esistenti e salvare nuove domande;
		\item \textbf{IN \texttt{QuestionsService}}: questa classe permette di ottenere domande esistenti e salvare nuove domande.
	\end{itemize}
	\item \textbf{Attributi}:
	\begin{itemize}
		\item \texttt{-} \texttt{\$scope: \$scope} \\
		Campo dati contenente un riferimento all'oggetto \$scope creato da \textit{Angular\ped{G}}, viene utilizzato come mezzo di comunicazione tra il \textit{controller\ped{G}} e la \textit{view\ped{G}}. Contiene gli oggetti che definiscono il \textit{model\ped{G}} dell'applicazione;
		\item \texttt{-} \texttt{\$location: \$location} \\
		Campo dati contenente un riferimento al servizio creato da \textit{Angular\ped{G}} che permette di accedere alla barra degli indirizzi del \textit{browser\ped{G}}, i cambiamenti all'URL nella barra degli indirizzi si riflettono in questo oggetto e viceversa;
		\item \texttt{-} \texttt{QuestionService: QuestionsService}\\
		Campo dati contenente un riferimento al servizio che si occupa della gestione delle informazioni legate alle domande.
	\end{itemize}
	\item \textbf{Metodi}:
	\begin{itemize}
		\item \texttt{+} \texttt{QuestionsManagementsController(\$scope: \$scope, \$location: \$location, QuestionService: QuestionService)} \\ 
		Metodo costruttore della classe. \\
		\textbf{Parametri}:
		\begin{itemize}
			\item \texttt{\$scope: \$scope} \\
			Parametro contenente un riferimento all'oggetto \$scope creato da \textit{Angular\ped{G}}. Viene utilizzato come mezzo di comunicazione tra il \textit{controller\ped{G}} e la \textit{view\ped{G}}. Contiene gli oggetti che definiscono il \textit{viewmodel\ped{G}} e il \textit{model\ped{G}} dell'applicazione;
			\item \texttt{\$location: \$location} \\
			Parametro contenente un riferimento al servizio creato da \textit{Angular\ped{G}} che permette di accedere alla barra degli indirizzi del \textit{browser\ped{G}}, i cambiamenti all'URL nella barra degli indirizzi si riflettono in questo oggetto e viceversa;
			\item \texttt{QuestionService: QuestionService} \\
			Parametro contenente un riferimento al servizio che si occupa della gestione delle informazioni legate alle domande.
		\end{itemize}
		\item \texttt{+} \texttt{getQuestionsByUser(username: String) : Array[QuestionItemMode]} \\ 
		Metodo che acquisisce le domande create dall'utente attraverso il \texttt{QuestionService}.\\
		\textbf{Parametri}:
		\begin{itemize}
			\item \texttt{username: String} \\
			Parametro contenente l'username dell'utente.
		\end{itemize}
		\item \texttt{+} \texttt{editQuestion(idQuestion: String) : void} \\ 
		Metodo che gestisce l'evento click sul pulsante per modificare la domanda. Effettua il redirect alla pagina di modifica della domanda. \\
		\textbf{Parametri}:
		\begin{itemize}
			\item \texttt{idQuestion: String} \\
			Parametro contenente l'id della domanda da modificare.
		\end{itemize}
	\end{itemize}
\end{itemize}


\paragraph{QuizziPedia::Front-End::Controllers::QuizEventController}
\begin{figure} [ht]
	\centering
	\includegraphics[scale=0.45]{UML/Classi/Front-End/QuizziPedia_Front-end_Controller_QuizEventController.png}
	\caption{QuizziPedia::Front-End::Controllers::QuizEventController}
\end{figure} \FloatBarrier
\begin{itemize}
	\item \textbf{Descrizione}: questa classe permette di reagire ai comandi dell'utente durante la gestione dei suoi questionari;
	\item \textbf{Utilizzo}: fornisce le funzionalità per reagire ai comandi dell'utente, effettua redirect alle pagine richieste, come la visualizzazione delle statistiche di un questionario e iniziare un questionario in modalità esame.
	\item \textbf{Relazione con altre classi}:
	\begin{itemize}
		\item \textit{IN} \texttt{CreationAndModifyDirective}:  
		\item \textit{IN} \texttt{ExamModalityDirective}:
		\item \textit{IN} \texttt{QuestionnaireResultsDirective}:
	\end{itemize}
	\item \textbf{Attributi}:
	\begin{itemize}
		\item \texttt{-} \texttt{\$scope: \$scope} \\
		Campo dati contenente un riferimento all’oggetto \$scope creato da \textit{Angular\ped{G}}, viene utilizzato come mezzo di comunicazione tra il controller e la view. Contiene gli oggetti che definiscono il model dell’applicazione;
		\item \texttt{-} \texttt{\$location: \$location} \\
		Campo dati contenente un riferimento al servizio creato da \textit{Angular\ped{G}} che permette di accedere alla barra degli indirizzi del \textit{browser\ped{G}}, i cambiamenti all’URL nella barra degli indirizzi si riflettono in questo oggetto e viceversa;
		\item \texttt{-} \texttt{\$mdDialog: \$mdDialog} \\
		Campo dati contenente un riferimento al servizio della libreria \textit{Material for Angular\ped{G}} che permette di creare delle componenti a popup;
	\end{itemize}
	\item \textbf{Metodi}:
	\begin{itemize}
		\item \texttt{+} \texttt{QuizEventController(\$scope: \$scope, \$location: \$location, \$mdDialog: \$mdDialog)} \\ Metodo costruttore della classe.
		\textbf{Parametri}:
		\begin{itemize}
			\item \texttt{-} \texttt{\$scope: \$scope} \\
			Parametro contenente un riferimento all’oggetto \$scope creato da \textit{Angular\ped{G}}, viene utilizzato come mezzo di comunicazione tra il controller e la view. Contiene gli oggetti che definiscono il model dell’applicazione;
			\item \texttt{-} \texttt{\$location: \$location} \\
		    Parametro contenente un riferimento al servizio creato da \textit{Angular\ped{G}} che permette di accedere alla barra degli indirizzi del \textit{browser\ped{G}}, i cambiamenti all’URL nella barra degli indirizzi si riflettono in questo oggetto e viceversa;
			\item \texttt{-} \texttt{\$mdDialog: \$mdDialog} \\
			Parametro contenente un riferimento al servizio della libreria \textit{Material for Angular\ped{G}} che permette di creare delle componenti a popup;
		\end{itemize}
		\item \texttt{+} \texttt{modifyQuestionnaire(quizId: String): void} \\
		Metodo che gestisce l’evento click sul pulsante di modifica questionario. Effettua il redirect alla pagina di gestione questionari;
		\begin{itemize}
			\item \texttt{quizId: String}: parametro che indica l'identificativo univoco di un questionario.
		\end{itemize}
		\item \texttt{+} \texttt{deleteQuestionnaire(quizId: String): void} \\
		Metodo che gestisce l’evento click sul pulsante di eliminazione questionario. Effettua il redirect alla pagina di gestione questionari;  
		\begin{itemize}
			\item \texttt{quizId: String}: parametro che indica l'identificativo univoco di un questionario.
		\end{itemize}
		\item \texttt{+} \texttt{subscribeManagement(quizId: String): void} \\
		Metodo che gestisce l’evento click sul pulsante di gestione iscrizioni. Effettua il redirect alla pagina di gestione iscrizioni;
		\item \texttt{+} \texttt{examModalityquizId: String(): void} \\
		\begin{itemize}
			\item \texttt{quizId: String}: parametro che indica l'identificativo univoco di un questionario.
		\end{itemize}
		Metodo che gestisce l’evento click sul pulsante di attivazione modalità esame. Effettua il redirect alla pagina di gestione questionari;
		\begin{itemize}
			\item \texttt{quizId: String}: parametro che indica l'identificativo univoco di un questionario.
		\end{itemize}
		\item \texttt{+} \texttt{resultsQuestionnaire(quizId: String): void} \\
		Metodo che gestisce l’evento click sul pulsante di allenamento. Effettua il redirect alla pagina di gestine questionari;
		\begin{itemize}
			\item \texttt{quizId: String}: parametro che indica l'identificativo univoco di un questionario.
		\end{itemize}   
	\end{itemize}
\end{itemize}
\paragraph[QuizziPedia::Front-End::Controllers\\::RegistrationManagementController]{QuizziPedia::Front-End::Controllers::RegistrationManagementController}
\begin{figure} [ht]
	\centering
	\includegraphics[scale=0.80]{UML/Classi/Front-End/QuizziPedia_Front-end_Controller_RegistrationManagementController.png}
	\caption{QuizziPedia::Front-End::Controllers::RegistrationManagementController}
\end{figure} \FloatBarrier
\begin{itemize}
	\item \textbf{Descrizione}: questa classe permette di gestire le iscrizione degli utenti ai questionari;
	\item \textbf{Utilizzo}: fornisce le funzionalità di iscrizione ad un questionario;
	\item \textbf{Relazione con altre classi}:
	\begin{itemize}
		\item \textbf{IN \texttt{RegistratioManagementModelView}}: classe di tipo \textit{modelview\ped{G}} la cui istanziazione è contenuta all'interno della variabile di ambiente \$scope di \textit{Angular\ped{G}}. All'interno di essa sono presenti le variabili e i metodi necessari per il \textit{Two-Way Data-Binding\ped{G}} tra la \textit{view\ped{G}} \texttt{RegistrationManagementView} e il \textit{controller\ped{G}} \texttt{RegistrationManagementController}; 
		\item \textbf{OUT \texttt{QuizService}}: questa classe permette di ottenere i dati di un quiz tramite delle parole chiave inserite dall'utente nella barra di ricerca.
	\end{itemize}
	\item \textbf{Attributi}:
	\begin{itemize}
		\item \texttt{-} \texttt{\$scope: \$scope} \\
		Campo dati contenente un riferimento all'oggetto \$scope creato da \textit{Angular\ped{G}}, viene utilizzato come mezzo di comunicazione tra il \textit{controller\ped{G}} e la \textit{view\ped{G}}. Contiene gli oggetti che definiscono il \textit{model\ped{G}} dell'applicazione;
		\item \texttt{-} \texttt{\$mdDialog: \$mdDialog} \\
		Campo dati contenente un riferimento al servizio della libreria \textit{Material for Angular\ped{G}} che permette di creare delle componenti a pop-up;
		\item \texttt{-} \texttt{QuizService: QuizService}\\ parametro che permette di ottenere, tramite il \textit{service\ped{G}}, la lista di tutte le domande presenti nel quiz;
		\item \texttt{+} \texttt{fillQuiz: RegistrationManagementModelView} \\
		Oggetto di tipo \texttt{RegistrationManagementModelView}. All'interno di esso sono presenti le variabili e i metodi necessari per il \textit{Two-Way Data-Binding\ped{G}} tra la \textit{view\ped{G}} \texttt{RegistrationManagemenView} e il \textit{controller\ped{G}} \texttt{RegistrationManagemenController}.
	\end{itemize}
	\item \textbf{Metodi}:
	\begin{itemize}
		\item \texttt{+} \texttt{RegistrationmanagementController(\$scope: \$scope, \$mdDialog: \$mdDialog, QuizService: QuizService)}: \\Metodo costruttore della classe. \\
			\textbf{Parametri}:
			\begin{itemize}
					\item \texttt{\$scope: \$scope} \\
					Campo dati contenente un riferimento all'oggetto \$scope creato da \textit{Angular\ped{G}}. Viene utilizzato come mezzo di comunicazione tra il \textit{controller\ped{G}} e la \textit{view\ped{G}}. Contiene gli oggetti che definiscono il \textit{viewmodel\ped{G}} e il \textit{model\ped{G}} dell'applicazione;
					\item \texttt{\$mdDialog: \$mdDialog} \\
					Campo dati contenente un riferimento al servizio della libreria \textit{Material for Angular\ped{G}} che permette di creare delle componenti a pop-up;
					\item \texttt{QuizService: QuizService}: parametro che permette di ottenere, tramite il \textit{service\ped{G}}, la lista di tutte le domande presenti nel quiz. 
			\end{itemize}
		\item \texttt{+} \texttt{subscribeQuestionnaire(username: String): void} \\ Metodo che permette l'iscrizione ad un questionario. Richiama la funzionalità del \texttt{QuizService}. \\
		\textbf{Parametri}:
		\begin{itemize}
			\item \texttt{username: String}: parametro che indica l'utente da iscrivere al questionario.
		\end{itemize}
	\end{itemize}
\end{itemize}


\paragraph{QuizziPedia::Front-End::Controllers::ResultsQuestionnaireController}
\begin{figure} [ht]
	\centering
	\includegraphics[scale=0.45]{UML/Classi/Front-End/QuizziPedia_Front-end_Controller_ResultsQuestionnaireController.png}
	\caption{QuizziPedia::Front-End::Controllers::ResultsQuestionnaireController}
\end{figure} \FloatBarrier
\begin{itemize}
	\item \textbf{Descrizione}: questa classe permette di gestire la visualizzazione dei risultati di un singolo questionario;
	\item \textbf{Utilizzo}: fornisce le funzionalità per recuperare i dati dal back-end e mostrarli all'utente nella view;
	\item \textbf{Relazione con altre classi}:
	\begin{itemize}
		\item \textit{IN} \texttt{ResultsQuestionnaireModelView}: classe di tipo modelview la cui istanziazione è contenuta all'interno della variabile di ambiente \$scope di \textit{Angular.js\ped{G}}. All'interno di essa sono presenti le variabili e i metodi necessari per il \textit{Two-Way Data-Binding\ped{G}} tra la view \texttt{ResultsQuestionnaireView} e il controller \texttt{ResultsQuestionnaireController}; 
		\item \textit{IN} \texttt{QuizService}: questa classe permette di ottenere i dati di un quiz tramite delle parole chiave inserite dall'utente nella barra di ricerca;
	\end{itemize}
	\item \textbf{Attributi}:
	\begin{itemize}
		\item \texttt{-} \texttt{scope: Scope} \\
		Campo dati contenente un riferimento all’oggetto \$scope creato da \textit{Angular\ped{G}}, viene utilizzato come mezzo di comunicazione tra il controller e la view. Contiene gli oggetti che definiscono il model dell’applicazione;
		\item \texttt{-} \texttt{\$mdDialog: \$mdDialog} \\
		Campo dati contenente un riferimento al servizio della libreria \textit{Material for Angular\ped{G}} che permette di creare delle componenti a popup;
		\item \textit{-} \texttt{QuizService: QuizService}: questa classe permette di ottenere i dati di un quiz tramite delle parole chiave inserite dall'utente nella barra di ricerca;
	\end{itemize}
	\item \textbf{Metodi}:
	\begin{itemize}
		\item \texttt{+} \texttt{ResultsQuestionnaireController(\$scope: \$scope, \$mdDialog: \$mdDialog, QuizService: QuizService)}: \\Metodo costruttore della classe. \\
		\textbf{Parametri}: 
		\begin{itemize}
			\item \texttt{-} \texttt{\$scope: \$scope} \\
			Campo dati contenente un riferimento all’oggetto \$scope creato da \textit{Angular\ped{G}}. Viene utilizzato come mezzo di comunicazione tra il controller e la view. Contiene gli oggetti che definiscono il viewmodel e il model dell’applicazione;
			\item \texttt{-} \texttt{\$mdDialog: \$mdDialog} \\
			Campo dati contenente un riferimento al servizio della libreria \textit{Material for Angular\ped{G}} che permette di creare delle componenti a popup;
			\item \texttt{-} \texttt{QuizService: QuizService}: parametro che permette di ottenere, tramite il service, la lista di tutte le domande presenti nel quiz;
		\end{itemize}
		\item \texttt{+ getQuizResults(quizId: String): Object} \\ Metodo che ritorna i risultati di un questionario.
		\textbf{Parametri}:
		\begin{itemize}
			\item \texttt{quiId: String} \\ Id del questionario del quale recuperare i risultati.
		\end{itemize}
		\item \texttt{+ getUserForThisQuestionnaire(quizId: String): Array} \\ Metodo che ritorna tutti gli utenti che hanno eseguito il questionario, con il loro risultato.
		\textbf{Parametri}:
		\begin{itemize}
			\item \texttt{quizId: String} \\ Id del questionario del quale recuperare gli utenti.
		\end{itemize}
	\end{itemize}
\end{itemize}


\paragraph{QuizziPedia::Front-End::Controllers::SearchController}
\begin{figure} [ht]
	\centering
	\includegraphics[scale=0.4]{UML/Classi/Front-End/QuizziPedia_Front-end_Controller_SearchController.png}
	\caption{QuizziPedia::Front-End::Controllers::SearchController}
\end{figure} \FloatBarrier
\begin{itemize}
	\item \textbf{Descrizione}: questa classe permette di gestire la ricerca di questionari e utenti all'interno dell'applicazione;
	\item \textbf{Utilizzo}: fornisce all'utente le funzionalità di ricerca per utenti e questionari;
	\item \textbf{Relazione con altre classi}:
	\begin{itemize}
		\item \textbf{IN \texttt{ResultsModelView}}: classe di tipo \textit{modelview\ped{G}} la cui istanziazione è contenuta all'interno della variabile di ambiente \$scope di \textit{Angular\ped{G}}. All'interno di essa sono presenti le variabili e i metodi necessari per il \textit{Two-Way Data-Binding\ped{G}} tra la \textit{view\ped{G}} \texttt{ResultsView}, la \textit{directive\ped{G}} \texttt{SearchDirective} e il \textit{controller\ped{G}} \texttt{ResultsController};
		\item \textbf{IN \texttt{SearchService}}: questa classe permette di eseguire una ricerca tra i questionari e gli utenti presenti ritornando un Object contenente i risultati di tale operazione;
		\item \textbf{IN \texttt{QuizService}}: questa classe permette di ottenere i dati di un quiz tramite delle parole chiave inserite dall'utente nella barra di ricerca. Permette inoltre di iscriversi ad un questionario.
	\end{itemize}
	\item \textbf{Attributi}:
	\begin{itemize}
		\item \texttt{-} \texttt{\$scope: \$scope} \\
		Campo dati contenente un riferimento all'oggetto \$scope creato da \textit{Angular\ped{G}}, viene utilizzato come mezzo di comunicazione tra il \textit{controller\ped{G}} e la \textit{view\ped{G}}. Contiene gli oggetti che definiscono il \textit{model\ped{G}} dell'applicazione;
		\item \texttt{-} \texttt{\$location: \$location} \\
		Campo dati contenente un riferimento al servizio creato da \textit{Angular\ped{G}} che permette di accedere alla barra degli indirizzi del \textit{browser\ped{G}}, i cambiamenti all'URL nella barra degli indirizzi si riflettono in questo oggetto e viceversa;
		\\item \texttt{-} \texttt{\$mdDialog: \$mdDialog} \\
		Campo dati contenente un riferimento al servizio della libreria \textit{Material for Angular\ped{G}} che permette di creare delle componenti a pop-up;
		\item \texttt{-} \texttt{\$routeParams: \$routeParams} \\
		Campo dati contente un riferimento al servizio creato da \textit{Angular\ped{G}} che permette di accedere alla barra degli indirizzi e recuperare i parametri passati; 
		\item \texttt{-} \texttt{SearchService: SearchService} \\
		Campo dati contenente un riferimento al servizio che si occupa della gestione delle informazioni legate alla ricerca. Viene utilizzato il metodo \texttt{search} di \texttt{SearchService} a cui viene passato come parametro la stringa di ricerca;
		\item \texttt{-} \texttt{QuizService: QuizService} \\
		Campo dati contenente un riferimento al servizio che si occupa della gestione delle informazioni legate ai questionari. Viene utilizzato il metodo \texttt{subscribeQuestionnaire} di \texttt{QuizService} per iscrivere un utente ad un questionario;
		\item \texttt{+} \texttt{result: ResultsModelView} \\
		Oggetto di tipo \texttt{ResultsModelView}. All'interno di esso sono presenti le variabili e i metodi necessari per il \textit{Two-Way Data-Binding\ped{G}} tra la \textit{view\ped{G}} \texttt{ResultView} e il \textit{controller\ped{G}} \texttt{SearchController}.
	\end{itemize}
	\item \textbf{Metodi}:
	\begin{itemize}
		\item \texttt{+} \texttt{SearchController(\$scope: \$scope, \$location: \$location, \$mdDialog:\\ \$mdDialog, \$routeParams: \$routeParams, SearchService: SearchService,\\ QuizService: QuizService)} \\
		Metodo costruttore della classe. Viene eseguita la ricerca per poter poi popolare il campo dati \texttt{result}. \\
		\textbf{Parametri}:
		\begin{itemize}
			\item \texttt{\$scope: \$scope} \\
			Parametro contenente un riferimento all'oggetto \$scope creato da \textit{Angular\ped{G}}. Viene utilizzato come mezzo di comunicazione tra il \textit{controller\ped{G}} e la \textit{view\ped{G}}. Contiene gli oggetti che definiscono il \textit{viewmodel\ped{G}} e il \textit{model\ped{G}} dell'applicazione;
			\item \texttt{\$location: \$location} \\
			Parametro contenente un riferimento al servizio creato da \textit{Angular\ped{G}} che permette di accedere alla barra degli indirizzi del \textit{browser\ped{G}}, i cambiamenti all'URL nella barra degli indirizzi si riflettono in questo oggetto e viceversa;
			\item \texttt{\$mdDialog: \$mdDialog} \\
			Parametro contenente un riferimento al servizio della libreria \textit{Material for Angular\ped{G}} che permette di creare delle componenti a pop-up;
			\item \texttt{\$routeParams: \$routeParams} \\
			Parametro contente un riferimento al servizio creato da \textit{Angular\ped{G}} che permette di accedere alla barra degli indirizzi e recuperare i parametri passati;
			\item \texttt{SearchService: SearchService} \\
			Parametro contenente un riferimento al servizio che si occupa della gestione delle informazioni legate alla ricerca. Viene utilizzato il metodo \texttt{search} di \texttt{SearchServ-\\ice} a cui viene passato come parametro la stringa di ricerca;
			\item \texttt{QuizService: QuizService} \\
			Parametro contenente un riferimento al servizio che si occupa della gestione delle informazioni legate ai questionari. Viene utilizzato il metodo \texttt{subscribeQuestio-\\nnaire} di \texttt{QuizService} per iscrivere un utente ad un questionario.
		\end{itemize} 
		\item \texttt{-} \texttt{searchUser(stringSearch: String, lang: String): Array<UserDetailsModel>} \\
		Metodo che esegue la ricerca tra gli utenti tramite un metodo fornito dalla classe \texttt{SearchService}. \\
		\textbf{Parametri}:
		\begin{itemize}
			\item \texttt{stringSearch: String} \\
			Parametro contenente la stringa sulla quale si deve basare la ricerca;
			\item \texttt{lang: String} \\
			Parametro contenente la lingua per la richiesta al backend.
		\end{itemize} 
		\item \texttt{-} \texttt{searchQuestionnaire(stringSearch: String): QuestionnaireModel} \\
		Metodo che esegue la ricerca tra i questionari tramite un metodo fornito dalla classe \texttt{SearchService}. \\
		\textbf{Parametri}:
		\begin{itemize}
			\item \texttt{stringSearch: String} \\
			Parametro contenente la stringa sulla quale si deve basare la ricerca;
			\item \texttt{lang: String} \\
			Parametro contenente la lingua per la richiesta al backend.
		\end{itemize} 
		\item \texttt{+} \texttt{goToUser(username: String): void} \\
		Metodo che gestisce l'evento click sul bottone per visualizzare il profilo dell'utente selezionato. Effettua il redirect alla pagina dell'utente.\\
		\textbf{Parametri}:
		\begin{itemize}
			\item \texttt{username: String} \\
			Parametro contenente l'username dell'utente di cui si vuole visualizzare il profilo.
		\end{itemize} 
		\item \texttt{+} \texttt{registrationToQuiz(idQuiz: String): void} \\
		Metodo che gestisce l'evento click sul pulsante di registrazione al questionario.\\
		\textbf{Parametri}:
		\begin{itemize}
			\item \texttt{idQuiz: String} \\
			Parametro contenente l'id del questionario di cui si vuole effettuare l'iscrizione.
		\end{itemize} 
		\item \texttt{+} \texttt{goToResultsPage(stringSearch: String): void} \\
		Metodo che gestisce l'evento click sul pulsante per effettuare una ricerca.\\
		\textbf{Parametri}:
		\begin{itemize}
			\item \texttt{stringSearch: String} \\
			Parametro contenente la stringa sulla quale si deve basare la ricerca.
		\end{itemize} 
	\end{itemize}
\end{itemize}


\paragraph{QuizziPedia::Front-End::Controllers::SignUpController}
\begin{figure} [ht]
	\centering
	\includegraphics[scale=0.6]{UML/Classi/Front-End/QuizziPedia_Front-end_Controller_SignUpController.png}
	\caption{QuizziPedia::Front-End::Controllers::SignUpController}
\end{figure} \FloatBarrier
\begin{itemize}
	\item \textbf{Descrizione}: questa classe permette di gestire la registrazione di un utente al sistema;
	\item \textbf{Utilizzo}: fornisce le funzionalità di registrazione di un utente al sistema;
	\item \textbf{Relazione con altre classi}:
	\begin{itemize}
		\item \textbf{IN \texttt{SignUpModelView}}: classe di tipo \textit{modelview\ped{G}} la cui istanziazione è contenuta all'interno della variabile di ambiente \$scope di \textit{Angular\ped{G}}. All'interno di essa sono presenti le variabili e i metodi necessari per il \textit{Two-Way Data-Binding\ped{G}} tra la \textit{view\ped{G}} \texttt{SignUpView} e il \textit{controller\ped{G}} \texttt{SignUpController};
		\item \textbf{IN \texttt{AuthService}}: questa classe permette di gestire la registrazione e l'autenticazione di un utente.
	\end{itemize}
	\item \textbf{Attributi}:
	\begin{itemize}
		\item \texttt{-} \texttt{\$scope: \$scope} \\
		Campo dati contenente un riferimento all'oggetto \$scope creato da \textit{Angular\ped{G}}. Viene utilizzato come mezzo di comunicazione tra il \textit{controller\ped{G}} e la \textit{view\ped{G}}. Contiene gli oggetti che definiscono il \textit{viewmodel\ped{G}} e il \textit{model\ped{G}} dell'applicazione;
		\item \texttt{-} \texttt{\$location: \$location} \\
		Campo dati contenente un riferimento al servizio creato da \textit{Angular\ped{G}} che permette di accedere alla barra degli indirizzi del \textit{browser\ped{G}}, i cambiamenti all'URL nella barra degli indirizzi si riflettono in questo oggetto e viceversa;
		\item \texttt{-} \texttt{\$mdDialog: \$mdDialog} \\
		Campo dati contenente un riferimento al servizio della libreria \textit{Material for Angular\ped{G}} che permette di creare delle componenti a pop-up;
		\item \texttt{-} \texttt{AuthService: AuthService} \\
		Campo dati contenente un riferimento al servizio che si occupa della gestione delle informazioni legate all'autenticazione. Viene utilizzato il metodo \texttt{signUp} di \texttt{AuthService} a cui viene passato come parametro un oggetto di tipo \texttt{SignUpModelView};
		\item \texttt{+} \texttt{newUser: SignUpModelView} \\
		Oggetto di tipo \texttt{SignUpModelView}. All'interno di esso sono presenti le variabili e i metodi necessari per il \textit{Two-Way Data-Binding\ped{G}} tra la \textit{view\ped{G}} \texttt{SignUpView} e il \textit{controller\ped{G}} \texttt{SignUpController}.
	\end{itemize}
	\item \textbf{Metodi}:
	\begin{itemize}
		\item \texttt{+} \texttt{SignUpController(\$scope: \$scope, \$location: \$location, \$mdDialog:\\ \$mdDialog, AuthService: AuthService)} \\
		Metodo costruttore della classe. \\
		\textbf{Parametri}:
		\begin{itemize}
			\item \texttt{\$scope: \$scope} \\
			Parametro contenente un riferimento all'oggetto \$scope creato da \textit{Angular\ped{G}}. Viene utilizzato come mezzo di comunicazione tra il \textit{controller\ped{G}} e la \textit{view\ped{G}}. Contiene gli oggetti che definiscono il \textit{viewmodel\ped{G}} e il \textit{model\ped{G}} dell'applicazione;
			Parametro contenente un riferimento al servizio creato da \textit{Angular\ped{G}} che permette di accedere alla barra degli indirizzi del \textit{browser\ped{G}}, i cambiamenti all'URL nella barra degli indirizzi si riflettono in questo oggetto e viceversa;
			\item \texttt{\$mdDialog: \$mdDialog} \\
			Parametro contenente un riferimento al servizio della libreria \textit{Material for Angular\ped{G}} che permette di creare delle componenti a pop-up;
			\item \texttt{AuthService: AuthService} \\
			Campo dati contenente un riferimento al servizio che si occupa della gestione delle informazioni legate all'autenticazione. Viene utilizzato il metodo \texttt{logIn} di \$texttt{AuthService} a cui vengono passati i parametri \texttt{username} e \texttt{password};
		\end{itemize}
		\item \texttt{+} \texttt{signUp(user: Object): void} \\
		Metodo che richiama il metodo \texttt{signup} del service \texttt{AuthService} passandogli un oggetto di tipo \texttt{SignUpModelView}. Nel caso di buona riuscita dell'operazione viene mostrato un messaggio di successo. Con l'azione di click sul bottone presentato dal messaggio di successo è possibile effettuare il redirect alla pagina di login dell'applicazione. Nel caso in cui invece avvenga un errore, viene mostrato a video il messaggio di errore.
		\textbf{Parametri}:
		\begin{itemize}
		\item \texttt{user: Object} \\
		Parametro che rappresenta un oggetto contenente tutti i parametri per la registrazione.
		\end{itemize}
		\item \texttt{+} \texttt{logIn(): void} \\
		Metodo che gestisce l'evento click sul pulsante di login. Effettua il redirect alla pagina di login.
	\end{itemize}
\end{itemize}


\paragraph{QuizziPedia::Front-End::Controllers::StatisticsController}
\begin{figure} [ht]
	\centering
	\includegraphics[scale=0.45]{UML/Classi/Front-End/QuizziPedia_Front-end_Controller_StatisticsController.png}
	\caption{QuizziPedia::Front-End::Controllers::StatisticsController}
\end{figure} \FloatBarrier
\begin{itemize}
	\item \textbf{Descrizione}: questa classe permette di gestire le statistiche di un utente;
	\item \textbf{Utilizzo}: fornisce le funzionalità per ottenere le statistiche di un utente per poterle mostrare nella view;
	\item \textbf{Relazione con altre classi}:
	\begin{itemize}
		\item \textit{IN} \texttt{StatisticsModelView}: classe di tipo modelview la cui istanziazione è contenuta all'interno della variabile di ambiente \$scope di \textit{Angular.js\ped{G}}. All'interno di essa sono presenti le variabili e i metodi necessari per il \textit{Two-Way Data-Binding\ped{G}} tra la directive \texttt{StatisticsDirective} e il controller \texttt{StatisticsController}; 
		\item \textit{IN} \texttt{StatisticsService}: questa classe permette di ottenere le statistiche dell'utente;
	\end{itemize}
	\item \textbf{Attributi}:
	\begin{itemize}
		\item \texttt{-} \texttt{\$scope: \$scope} \\
		Campo dati contenente un riferimento all’oggetto \$scope creato da \textit{Angular\ped{G}}, viene utilizzato come mezzo di comunicazione tra il controller e la view. Contiene gli oggetti che definiscono il model dell’applicazione;
		\item \texttt{-} \texttt{StatisticsService: StatisticsService} \\
		Campo dati contenente un riferimento al servizio che si occupa della gestione delle informazioni legate alle statistiche da visualizzare;
		\item \texttt{+} \texttt{userStatistic: StatisticsModelView} \\ Oggetto di tipo \texttt{StatisticsModelView} contenente le informazioni delle statistiche. All'interno di esso sono presenti le variabili e i metodi necessari per il \textit{Two-Way Data-Binding\ped{G}} tra la directive e il controller \texttt{StatisticsController}.
	\end{itemize}	
	\begin{itemize}
		\item \textbf{Metodi}:
		\item \texttt{+} \texttt{StatisticsController(\$scope: \$scope, StatisticsService: StatisticsService)} \\ 
		Metodo costruttore della classe. \\
		\begin{itemize}
			\item \texttt{\$scope: \$scope} \\
			Parametro contenente un riferimento all’oggetto \$scope creato da \textit{Angular\ped{G}}. Viene utilizzato come mezzo di comunicazione tra il controller e la view. Contiene gli oggetti che definiscono il viewmodel e il model dell’applicazione;
			\item \texttt{StatisticsService: StatisticsService} \\
			Parametro contenente un riferimento al servizio che si occupa della gestione delle informazioni legate alle statistiche da visualizzare.
		\end{itemize}
		\item \texttt{-} \texttt{getStatistics(username: String): Object} \\ 
		Metodo che permette di ottenere le statistiche si un utente grazie all'utilizzo di \texttt{StatisticsService}; \\
		\textbf{Parametri}: 
		Metodo costruttore della classe. \\
		\begin{itemize}
			\item \texttt{username: String} \\
			Parametro contenente la stringa username utilizzata per poter recuperare le giuste statistiche attraverso lo \texttt{StatisticsService}.
		\end{itemize}
	\end{itemize}
\end{itemize}


\paragraph[QuizziPedia::Front-End::Controllers\\::StringsSortingQuestionsController]{QuizziPedia::Front-End::Controllers::StringsSortingQuestionsController}
\begin{figure} [ht]
	\centering
	\includegraphics[scale=0.3]{UML/Classi/Front-End/QuizziPedia_Front-end_Controller_StringSortingQuestionsController.png}
	\caption{QuizziPedia::Front-End::Controllers::StringsSortingQuestionsController}
\end{figure} \FloatBarrier
\begin{itemize}
	\item \textbf{Descrizione}: questa classe permette di gestire la creazione e la modifica di una domanda a ordinamento di stringhe;
	\item \textbf{Utilizzo}: fornisce le funzionalità per inserire una nuova domanda a ordinamento di stringhe nel database e per modificarne una esistente;
	\item \textbf{Relazione con altre classi}:
	\begin{itemize}
		\item \textbf{IN \texttt{StringsSortingQuestionsModelView}}: classe di tipo \textit{modelview \ped{G}} la cui istanziazione è contenuta all'interno della variabile di ambiente \$scope di \textit{Angular\ped{G}}. All'interno di essa sono presenti le variabili e i metodi necessari per il \textit{Two-Way Data-Binding\ped{G}} tra la \textit{view\ped{G}} \texttt{StringsSortingQuestionsView} e il \textit{controller\ped{G}} \texttt{StringsSortingQuestio-\\nsController};
		\item \textbf{IN \texttt{QuestionService}}: questa classe permette di:
		\begin{itemize}
			\item Ottenere una domanda attraverso il metodo dedicato;
			\item Caricare una domanda modificata;
			\item Caricare una nuova domanda.
		\end{itemize}
		\item \textbf{IN \texttt{QuestionItemModel}}: questa classe rappresenta il modello di una domanda.
	\end{itemize}
	\item \textbf{Attributi}:
	\begin{itemize}
		\item \texttt{-} \texttt{\$scope: \$scope} \\
		Campo dati contenente un riferimento all'oggetto \$scope creato da \textit{Angular\ped{G}}, viene utilizzato come mezzo di comunicazione tra il \textit{controller\ped{G}} e la \textit{view\ped{G}}. Contiene gli oggetti che definiscono il \textit{model\ped{G}} dell'applicazione;
		\item \texttt{-} \texttt{QuestionItemModel: QuestionItemModel} \\
		Campo dati che si riferisce alla classe che rappresenta il modello della domanda;
		\item \texttt{-} \texttt{\$mdDialog: \$mdDialog} \\
		Campo dati contenente un riferimento al servizio della libreria \textit{Material for Angular\ped{G}} che permette di creare delle componenti a pop-up;
		\item \texttt{-} \texttt{QuestionService: QuestionService}: \\
		Campo dati contenente un riferimento al servizio che si occupa della gestione delle informazioni legate alle domande;
		\item \texttt{\$routeParams: \$routeParams} \\
		Campo dati contenente il riferimento all'oggetto globale \$routeParams creato da \textit{Angular\ped{G}}. Tale servizio permette di recuperare il set di variabili presenti nell'URL. 
	\end{itemize}
	\item \textbf{Metodi}:
	\begin{itemize}
		\item \texttt{+} \texttt{StringsSortingQuestionsController(\$scope: \$scope, QuestionItemModel:\\ QuestionItemModel, \$mdDialog: \$mdDialog, QuestionService: QuestionSe-\\rvice, \$routeParams: \$routeParams)} \\ 
		Metodo costruttore della classe. Se in \$routeParams sarà presente il codice univoco che rappresenta una domanda e di questa il creatore è l'utente autenticato, allora verrà scaricato attraverso il \texttt{QuestionService} il contenuto della domanda così da poterlo modificare. In caso contrario verrà mostrato un errore attraverso \$mdDialog indicando che i privilegi per tale operazione sono negati. Nel caso in cui non ci sarà tale parametro in \$routeParams verrà caricata la \textit{view\ped{G}} vuota così da poter creare una nuova domanda. \\
		\textbf{Parametri}:
		\begin{itemize}
			\item \texttt{\$scope: \$scope} \\
			Parametro contenente un riferimento all'oggetto \$scope creato da \textit{Angular\ped{G}}. Viene utilizzato come mezzo di comunicazione tra il \textit{controller\ped{G}} e la \textit{view\ped{G}}. Contiene gli oggetti che definiscono il \textit{viewmodel\ped{G}} e il \textit{model\ped{G}} dell'applicazione;
			\item \texttt{QuestionItemModel: QuestionItemModel} \\ 
			Parametro che si riferisce alla classe che rappresenta il modello della domanda;
			\item \texttt{\$mdDialog: \$mdDialog} \\
			Parametro contenente un riferimento al servizio della libreria \textit{Material for Angular\ped{G}} che permette di creare delle componenti a pop-up;
			\item \texttt{QuestionService: QuestionService}: \\
			Parametro contenente un riferimento al servizio che si occupa della gestione delle informazioni legate alle domande;
			\item \texttt{\$routeParams: \$routeParams} \\
			Parametro contenente il riferimento all'oggetto globale \$routeParams creato da \textit{Angular\ped{G}}. Tale servizio permette di recuperare il set di variabili presenti nell'URL. 
		\end{itemize}
		\item \texttt{+} \texttt{submitQuestion(): void}\\ 
		Metodo che gestisce l'evento click sul pulsante di conferma sulla domanda. Raccoglie i dati dal \textit{modelview\ped{G}} e li manda al \textit{server\ped{G}} attraverso \texttt{QuestionService}. Poi verrà effettuato il redirect alla pagina di gestione delle domande oppure al questionario che si stava creando. 
	\end{itemize}
\end{itemize}


\paragraph{QuizziPedia::Front-End::Controllers::TopicKeywordsController}
\begin{figure} [ht]
	\centering
	\includegraphics[scale=0.45]{UML/Classi/Front-End/QuizziPedia_Front-end_Controller_TopicKeywordsController.png}
	\caption{QuizziPedia::Front-End::Controllers::TopicKeywordsController}
\end{figure} \FloatBarrier
\begin{itemize}
	\item \textbf{Descrizione}: questa classe permette di gestire il recupero delle parole chiave di un questionario;
	\item \textbf{Utilizzo}: fornisce le funzionalità per il recupero delle parole chiave durante la creazione di un questionario;
	\item \textbf{Relazione con altre classi}:
	\begin{itemize}
		\item \textit{IN} \texttt{TopicKeywordsModelView}: ; 
		\item \textit{IN} \texttt{QuestionsService}: questa classe permette di ottenere domande esistenti e salvare nuove domande;
	\end{itemize}
	\item \textbf{Attributi}:
	\begin{itemize}
		\item \texttt{-} \texttt{\$scope: \$scope} \\
		Campo dati contenente un riferimento all’oggetto \$scope creato da \textit{Angular\ped{G}}, viene utilizzato come mezzo di comunicazione tra il controller e la view. Contiene gli oggetti che definiscono il model dell’applicazione;
		\item \texttt{-} \texttt{QuestionService: QuestionService}\\ 
		Permette, tra le altre cose, di ottenere le parole chiave a partire da una stringa passata.
		\item \texttt{+} \texttt{key: TopicKeywordsModelView} \\
		Oggetto di tipo \texttt{CreateQuestionnaireView}. All'interno di esso sono presenti le variabili e i metodi necessari per il \textit{Two-Way Data-Binding\ped{G}} tra la direttiva \texttt{TopicKeywordsDirective} e il controller \texttt{TopicKeywordsController};
	\end{itemize}
	\item \textbf{Metodi}:
	\begin{itemize}
		\item \texttt{+} \texttt{TopicKeywordsController(\$scope: \$scope)} \\Metodo costruttore della classe. \\
		\textbf{Parametri}:
		\begin{itemize}
			\item \texttt{-} \texttt{\$scope: \$scope} \\
			Campo dati contenente un riferimento all’oggetto \$scope creato da \textit{Angular\ped{G}}. Viene utilizzato come mezzo di comunicazione tra il controller e la view. Contiene gli oggetti che definiscono il viewmodel e il model dell’applicazione; 
		\end{itemize}
		\item \texttt{-} \texttt{getKeywords(key: String): String[]}: \\ Metodo che ritorna le parole che hanno a che fare con key
	\end{itemize}
	\textbf{Parametri}:
	\begin{itemize}
		\item \texttt{key: String}: parametro che identifica la stringa con la quale cercare le keywords. 
	\end{itemize}
\end{itemize}
\paragraph{QuizziPedia::Front-End::Controllers::TrainingController}
\begin{figure} [ht]
	\centering
	\includegraphics[scale=0.45]{UML/Classi/Front-End/QuizziPedia_Front-end_Controller_TrainingController.png}
	\caption{QuizziPedia::Front-End::Controllers::TrainingController}
\end{figure} \FloatBarrier
\begin{itemize}
	\item \textbf{Descrizione}: questa classe permette di gestire la modalità allenamento sottoponendo all'utente le giuste domande adatte al suo livello;
	\item \textbf{Utilizzo}: fornisce le funzionalità per recuperare le domande che siano in accordo con il livello dell'utente;
	\item \textbf{Relazione con altre classi}:
	\begin{itemize}
		\item \textit{IN} \texttt{TrainingModelView}: classe di tipo modelview la cui istanziazione è contenuta all'interno della variabile di ambiente \$scope di \textit{Angular.js\ped{G}}. All'interno di essa sono presenti le variabili e i metodi necessari per il \textit{Two-Way Data-Binding\ped{G}} tra la view \texttt{TrainingView} e il controller \texttt{TrainingController};
		\item \textit{IN} \texttt{TrainingModeModel}: rappresenta un allenamento. Contiene tutte le informazioni necessarie alla	presentazione del contenuto di un allenamento;
		\item \textit{IN} \texttt{QuestionController}: questa classe permette di gestire il recupero delle domande per far si che possano essere visualizzate nella modalità allenamento e nella compilazione dei questionari;
	\end{itemize}
	\item \textbf{Attributi}:
	\begin{itemize}
		\item \texttt{-} \texttt{\$scope: \$scope} \\
		Campo dati contenente un riferimento all’oggetto \$scope creato da \textit{Angular\ped{G}}, viene utilizzato come mezzo di comunicazione tra il controller e la view. Contiene gli oggetti che definiscono il model dell’applicazione;
		\item \texttt{-} \texttt{\$rootScope: \$rootScope} \\
		Campo dati contenente il riferimento all'oggetto globale \$rootScope creato da \textit{Angular\ped{G}}. Viene utilizzato per rendere accessibile a tutti i controller e a tutte le view l'oggetto \texttt{TrainingModeModel}. In questo caso viene utilizzato per inserire in \$rootScope l'oggetto di ritorno della chiamata a \texttt{getNextQuestion};
		\item \texttt{-} \texttt{\$mdDialog: \$mdDialog} \\
		Campo dati contenente un riferimento al servizio della libreria \textit{Material for Angular\ped{G}} che permette di creare delle componenti a popup;
		\item \texttt{+} \texttt{training: TrainingModelView} \\
		Oggetto di tipo \texttt{TrainingModelView}. All'interno di esso sono presenti le variabili e i metodi necessari per il \textit{Two-Way Data-Binding\ped{G}} tra la view \texttt{TrainingView} e il controller \texttt{TrainingController};
	\end{itemize}
	\item \textbf{Metodi}:
	\begin{itemize}
		\item \texttt{+} \texttt{TrainingController(\$scope: \$scope, \$rootscope: \$rootscope, \$mdDialog: \$mdDialog, TrainingModeModel: TrainingModeModel)} \\ Metodo costruttore della classe; \\
		\textbf{Parametri:}
		\begin{itemize}
			\item \texttt{\$scope: \$scope} \\
			Parametro contenente un riferimento all’oggetto \$scope creato da \textit{Angular\ped{G}}. Viene utilizzato come mezzo di comunicazione tra il controller e la view. Contiene gli oggetti che definiscono il viewmodel e il model dell’applicazione;
			\item \texttt{\$rootscope: \$rootscope}\\
			Parametro contenente il riferimento all'oggetto globale \$rootScope creato da \textit{Angular\ped{G}}. Viene utilizzato per rendere accessibile a tutti i controller e a tutte le view l'oggetto \texttt{UserDetailsModel}. In questo caso viene utilizzato per aggiornare in \$rootScope l'oggetto che rappresenta l'utente autenticato all'interno dell'applicazione;
			\item \texttt{\$mdDialog: \$mdDialog} \\
			Parametro contenente un riferimento al servizio della libreria \textit{Material for Angular\ped{G}} che permette di creare delle componenti a popup;
			\item \texttt{TrainingModeModel: TrainingModeModel} \\ Rappresenta un allenamento. Contiene tutte le informazioni necessarie alla presentazione del contenuto di un allenamento;
		\end{itemize}
		\item \texttt{+} \texttt{addQuestion(question: QuestionItemModel): void} \\
		Metodo che gestisce l'evento per inserire una domanda nella cronologia delle domande. \\
		\textbf{Parametri}:
		\begin{itemize}
			\item \texttt{question: QuestionItemModel} \\
			Parametro contenente un riferimento all'oggetto di tipo \texttt{QuestionItemModel};
		\end{itemize}
		\item \texttt{+} \texttt{loadNewQuestionBy(topic: String, keywords: Array[String], level: Number): void} \\
		Metodo che emette l'evento per scaricare una nuova domanda in base ai parametri passati. Fa una richiesta al \texttt{QuestionsController} che andrà a scaricare la domanda e ad inserirla in \texttt{TrainingModeModel} nello \texttt{\$scope} mediante il metodo \texttt{addQuestion}; \\
		\textbf{Parametri}:
		\begin{itemize}
			\item \texttt{topic: String} \\
			Parametro contenente l'argomento della domanda;
			\item \texttt{keywords: Array[String]} \\
			Parametro contenente un\texttt{array} di stringhe che rappresenta le keywords scelte per l'allenamento;
			\item \texttt{level: Number} \\
			Parametro contenente il livello dell'utente.
		\end{itemize}
		\item \texttt{+} \texttt{addResult(questionNumber: Number, result: Boolean): void} \\
		Metodo che gestisce l'evento per inserire il risultato di una domanda. \\
		\textbf{Parametri}:
		\begin{itemize}
			\item \texttt{questionNumber: Number} \\
			Parametro contenente il numero della domanda risposta;
			\item \texttt{result: Boolean} \\
			Parametro contenente il risultato della domanda risposta.
		\end{itemize}
	\end{itemize}
\end{itemize}


\paragraph{QuizziPedia::Front-End::Controllers::TrueFalseQuestionsController}
\begin{figure} [ht]
	\centering
	\includegraphics[scale=0.3]{UML/Classi/Front-End/QuizziPedia_Front-end_Controller_TrueFalseQuestionsController.png}
	\caption{QuizziPedia::Front-End::Controllers::TrueFalseQuestionsController}
\end{figure} \FloatBarrier
\begin{itemize}
	\item \textbf{Descrizione}: questa classe permette di gestire la creazione e la modifica di una domanda vero/falso;
	\item \textbf{Utilizzo}: fornisce le funzionalità per inserire una nuova domanda vero/falso nel database e per modificarne una esistente;
	\item \textbf{Relazione con altre classi}:
	\begin{itemize}
		\item \textbf{IN \texttt{QuestionService}}: questa classe permette di:
			\begin{itemize}
				\item Ottenere una domanda attraverso il metodo dedicato;
				\item Caricare una domanda modificata;
				\item Caricare una nuova domanda.
			\end{itemize}
		\item \textbf{IN \texttt{QuestionItemModel}}: questa classe rappresenta il modello di una domanda;
		\item \textbf{OUT \texttt{TrueFalseQuestionsModelView}}: classe di tipo \textit{modelview\ped{G}} la cui istanziazione è contenuta all'interno della variabile di ambiente \$scope di \textit{Angular\ped{G}}. All'interno di essa sono presenti le variabili e i metodi necessari per il \textit{Two-Way Data-Binding\ped{G}} tra la \textit{view\ped{G}} \texttt{TrueFalseQuestionsView} e il \textit{controller\ped{G}} \texttt{TrueFalseQuestionsController}.
	\end{itemize}
	\item \textbf{Attributi}:
	\begin{itemize}
		\item \texttt{-} \texttt{\$scope: \$scope} \\
		Campo dati contenente un riferimento all'oggetto \$scope creato da \textit{Angular\ped{G}}, viene utilizzato come mezzo di comunicazione tra il \textit{controller\ped{G}} e la \textit{view\ped{G}}. Contiene gli oggetti che definiscono il \textit{model\ped{G}} dell'applicazione;
		\item \texttt{-} \texttt{QuestionItemModel: QuestionItemModel} \\
		Campo dati che si riferisce alla classe che rappresenta il modello della domanda;
		\item \texttt{-} \texttt{\$mdDialog: \$mdDialog} \\
		Campo dati contenente un riferimento al servizio della libreria \textit{Material for Angular\ped{G}} che permette di creare delle componenti a pop-up;
		\item \texttt{-} \texttt{QuestionService: QuestionService}: \\
		Campo dati contenente un riferimento al servizio che si occupa della gestione delle informazioni legate alle domande; 
		\item \texttt{-} \texttt{Upload: Upload} \\
		Campo dati contenente un riferimento alla libreria \textit{ng-file-upload\ped{G}} necessaria per il caricamento della foto profilo dell'utente;
		\item \texttt{-} \texttt{\$timeout: \$timeout} \\
		Campo dati contenente il riferimento all'oggetto globale \$timeout creato da \textit{Angular\ped{G}}. 
		Il valore di ritorno di una chiamata alla funzione di \$timeout è una \textit{promise\ped{G}}, la quale sarà risolta quando avverrà il ritardo e la funzione di \$timeout sarà eseguita; 
		\item \texttt{\$routeParams: \$routeParams} \\
		Campo dati contenente il riferimento all'oggetto globale \$routeParams creato da \textit{Angular\ped{G}}. Tale servizio permette di recuperare il set di variabili presenti nell'URL.
	\end{itemize}
	\item \textbf{Metodi}:
	\begin{itemize}
		\item \texttt{+} \texttt{TrueFalseQuestionsController(\$scope: \$scope, QuestionItemModel: Que-\\stionItemModel, \$mdDialog: \$mdDialog, QuestionService: QuestionServi-\\ce, Upload: Upload, \$timeout: \$timeout, \$routeParams: \$routeParams)} \\ 
		Metodo costruttore della classe. Se in \$routeParams sarà presente il codice univoco che rappresenta una domanda e di questa il creatore è l'utente autenticato, allora verrà scaricato attraverso il \texttt{QuestionService} il contenuto della domanda così da poterlo modificare. In caso contrario verrà mostrato un errore attraverso \$mdDialog indicando che i privilegi per tale operazione sono negati. Nel caso in cui non ci sarà tale parametro in \$routeParams verrà caricata la \textit{view\ped{G}} vuota così da poter creare una nuova domanda. \\
		\textbf{Parametri}:
		\begin{itemize}
			\item \texttt{\$scope: \$scope} \\
			Parametro contenente un riferimento all'oggetto \$scope creato da \textit{Angular\ped{G}}. Viene utilizzato come mezzo di comunicazione tra il \textit{controller\ped{G}} e la \textit{view\ped{G}}. Contiene gli oggetti che definiscono il \textit{viewmodel\ped{G}} e il \textit{model\ped{G}} dell'applicazione;
			\item \texttt{QuestionItemModel: QuestionItemModel} \\ 
			Parametro che si riferisce alla classe che rappresenta il modello della domanda;
			\item \texttt{\$mdDialog: \$mdDialog} \\
			Parametro contenente un riferimento al servizio della libreria \textit{Material for Angul-\\ar\ped{G}} che permette di creare delle componenti a pop-up;
			\item \texttt{QuestionService: QuestionService}: \\
			Parametro contenente un riferimento al servizio che si occupa della gestione delle informazioni legate alle domande;
			\item \texttt{Upload: Upload} \\
			Parametro contenente un riferimento alla libreria \textit{ng-file-upload\ped{G}} necessaria per il caricamento della foto profilo dell'utente;
			\item \texttt{\$timeout: \$timeout} \\
			Parametro contenente il riferimento all'oggetto globale \$timeout creato da \textit{Angul-\\ar\ped{G}}. 
			Il valore di ritorno di una chiamata alla funzione di \$timeout è una \textit{promise\ped{G}}, la quale sarà risolta quando avverrà il ritardo e la funzione di \$timeout sarà eseguita;
			\item \texttt{\$routeParams: \$routeParams} \\
			Parametro contenente il riferimento all'oggetto globale \$routeParams creato da \textit{Angular\ped{G}}. Tale servizio permette di recuperare il set di variabili presenti nell'URL.
		\end{itemize}
		\item \texttt{+} \texttt{submitQuestion(): void}\\ 
		Metodo che gestisce l'evento click sul pulsante di conferma sulla domanda. Raccoglie i dati dal \textit{modelview\ped{G}} e li manda al \textit{server\ped{G}} attraverso \texttt{QuestionService}. Poi verrà effettuato il redirect alla pagina di gestione delle domande oppure al questionario che si stava creando. 
	\end{itemize}
\end{itemize}


\paragraph{QuizziPedia::Front-End::Controllers::UserDetailsController}
\begin{figure} [ht]
	\centering
	\includegraphics[scale=0.5]{UML/Classi/Front-End/QuizziPedia_Front-end_Controller_UserDetailController.png}
	\caption{QuizziPedia::Front-End::Controllers::UserDetailController}
\end{figure} \FloatBarrier
\begin{itemize}
	\item \textbf{Descrizione}: questa classe permette di gestire i dati di un utente da mostrare nella pagina di un profilo;
	\item \textbf{Utilizzo}: fornisce le funzionalità per ottenere i dati di un utente per poterle mostrare nella \textit{view\ped{G}};
	\item \textbf{Relazione con altre classi}:
	\begin{itemize}
		\item \textbf{IN \texttt{UserDetailsModelView}}: \textit{directive\ped{G}} che permette di visualizzare i dati di un utente; 
		\item \textbf{IN \texttt{UserDetailsService}}: questa classe permette di ottenere i dati dell'utente;
		\item \textbf{IN \texttt{UserDetailsModel}}: rappresenta un utente. Contiene tutte le informazioni necessarie alla presentazione del contenuto di un utente sia nella visualizzazione che nella gestione di un profilo.
	\end{itemize}
	\item \textbf{Attributi}:
	\begin{itemize}
		\item \texttt{-} \texttt{\$scope: \$scope} \\
		Campo dati contenente un riferimento all'oggetto \$scope creato da \textit{Angular\ped{G}}, viene utilizzato come mezzo di comunicazione tra il \textit{controller\ped{G}} e la \textit{view\ped{G}}. Contiene gli oggetti che definiscono il \textit{model\ped{G}} dell'applicazione;
		\item \texttt{-} \texttt{\$rootScope: \$rootScope} \\
		Campo dati contenente il riferimento all'oggetto globale \$rootScope creato da \textit{Angular\ped{G}}. Viene utilizzato per rendere accessibile a tutti i \textit{controller\ped{G}} e a tutte le \textit{view\ped{G}} l'oggetto \texttt{UserDetailsModel}. In questo caso viene utilizzato per inserire in \$rootScope l'oggetto di ritorno della chiamata a \texttt{getUserDetails} del \textit{service\ped{G}} \texttt{UserDetailsService};
		\item \texttt{-} \texttt{userDetailsService: userDetailsService} \\ Questa classe permette di ottenere i dati personali degli utenti;
		\item \texttt{+} \texttt{details: UserDetailsModelView} \\
		Oggetto di tipo \texttt{UserDetailsModelView}. All'interno di esso sono presenti le variabili e i metodi necessari per il \textit{Two-Way Data-Binding\ped{G}} tra la \textit{directive\ped{G}} \texttt{UserDetailsDire-\\ctive} e il \textit{controller\ped{G}} \texttt{UserDetailsController}.
	\end{itemize}	
		\item \textbf{Metodi}:
		\begin{itemize}
		\item \texttt{+} \texttt{UserDetailsController(\$scope: \$scope, \$rootScope:\$rootScope,\\ \$mdDialog: \$mdDialog, UserDetailsService: UserDetailsService)} \\ Metodo costruttore della classe.\\
		\textbf{Parametri}: 
		\begin{itemize}
			\item \texttt{\$scope: \$scope} \\
			Parametro contenente un riferimento all'oggetto \$scope creato da \textit{Angular\ped{G}}. Viene utilizzato come mezzo di comunicazione tra il \textit{controller\ped{G}} e la \textit{view\ped{G}}. Contiene gli oggetti che definiscono il \textit{viewmodel\ped{G}} e il \textit{model\ped{G}} dell'applicazione;
			\item \texttt{\$rootScope: \$rootScope} \\
			Parametro contenente il riferimento all'oggetto globale \$rootScope creato da \textit{An-\\gular\ped{G}}. Viene utilizzato per rendere accessibile a tutti i \textit{controller\ped{G}} e a tutte le \textit{view\ped{G}} l'oggetto \texttt{UserDetailsModel}. In questo caso viene utilizzato per inserire in \$rootScope l'oggetto di ritorno della chiamata a \texttt{getUserDetails} del \textit{service\ped{G}} \texttt{UserDetai-\\lsService};
			\item \texttt{\$mdDialog: \$mdDialog} \\
			Campo dati contenente un riferimento al servizio della libreria \textit{Material for Angul-\\ar\ped{G}} che permette di creare delle componenti a pop-up;
			\item \texttt{userDetailsService: userDetailsService}: parametro che permette di ottenere, tramite il \textit{service\ped{G}}, la lista di tutti i dati dell'utente.
		\end{itemize}
		\item \texttt{-} \texttt{getUserDetails(username: String): UserDetailsModel} \\ Metodo che permette di ottenere i dati con una chiamata a \texttt{UserDetailsService}.
		\textbf{Parametri}:
		\begin{itemize}
			\item \texttt{username: String}: parametro che identifica l'utente del quale saranno scaricati i dati.
		\end{itemize}
	\end{itemize}
\end{itemize}


\paragraph{QuizziPedia::Front-End::Controllers::AppController}
\begin{figure} [ht]
	\centering
	\includegraphics[scale=0.35]{UML/Classi/Front-End/QuizziPedia_Front-end_Controller_AppController.png}
	\caption{QuizziPedia::Front-End::Controllers::AppController}
\end{figure} \FloatBarrier
\begin{itemize}
	\item \textbf{Descrizione}: questa classe permette di gestire per ogni pagina dell'applicazione l'autenticazione e l'autorizzazione dell'utente che si posiziona in essa e mostrare la lingua corretta;
	\item \textbf{Utilizzo}: fornisce le funzionalità per gestire per ogni pagina dell'applicazione l'autenticazione e l'autorizzazione dell'utente che si posiziona in essa e mostrare la lingua corretta;
	\item \textbf{Relazione con altre classi}:
	\begin{itemize}
		\item \textbf{IN \texttt{UserDetailsModelView}}: classe di tipo \textit{modelview\ped{G}} la cui istanziazione è contenuta all'interno della variabile di ambiente \$rootScope di \textit{Angular\ped{G}}. All'interno di essa sono presenti le variabili e i metodi necessari per il \textit{Two-Way Data-Binding\ped{G}} tra le \textit{views\ped{G}} e i \textit{controllers\ped{G}} che necessitano di utilizzare l'utente autenticato;
		\item \textbf{IN \texttt{UserDetailsModel}}: rappresenta un utente. Contiene tutte le informazioni necessarie alla presentazione del contenuto di un utente sia nella visualizzazione che nella gestione di un profilo;
		\item \textbf{IN} \texttt{LangService}: questa classe permette di gestire la lingua nella quale si è scelto di utilizzare l'applicazione;
		\item \textbf{IN} \texttt{LangModel}: rappresenta le informazioni per la giusta traduzione dell'applicazione;
		\item \textbf{IN} \texttt{AuthService}: questa classe permette di gestire la registrazione e l'autenticazione di un utente;
		\item \textbf{IN} \texttt{MenuBarModel}: questa classe rappresenta la classe che contiene le informazioni per la giusta visualizzazione della barra;
		\item \textbf{IN} \texttt{MenuBarModelView}: classe di tipo modelview la cui istanziazione è contenuta all'interno della variabile di ambiente \texttt{\$rootScope} di \textit{Angular\ped{G}}. All'interno di essa sono presenti le variabili e i metodi necessari per il \textit{Two-Way Data-Binding\ped{G}} tra la \textit{view\ped{G}} \texttt{Index} e il \textit{controller\ped{G}} \texttt{MenuBarController};
		\item \textbf{OUT} \texttt{AppRouter}: classe che gestisce i routes dell’applicazione, utilizza il servizio \texttt{\$routeProvider} per associare ad ogni route un \textit{controller\ped{G}} e una \textit{view\ped{G}}. \texttt{Appcontroller} viene chiamato ogni volta che \texttt{AppRouter} instrada una vista.
	\end{itemize}
	\item \textbf{Attributi}:
	\begin{itemize}
		\item \texttt{-} \texttt{\$rootScope: \$rootScope} \\
		Campo dati contenente il riferimento all'oggetto globale \$rootScope creato da \textit{Angular\ped{G}}. Viene utilizzato per rendere accessibile a tutti i \textit{controller\ped{G}} e a tutte le \textit{view\ped{G}} l'oggetto \texttt{UserDetailsModel};
		\item \texttt{-} \texttt{\$location: \$location} \\
		Campo dati contenente un riferimento al servizio creato da \textit{Angular\ped{G}} che permette di accedere alla barra degli indirizzi del \textit{browser\ped{G}}, i cambiamenti all'URL nella barra degli indirizzi si riflettono in questo oggetto e viceversa; 
		\item \texttt{-} \texttt{\$routeParams: \$routeParams} \\
		Campo dati contenente il riferimento all'oggetto globale \$routeParams creato da \textit{Angular\ped{G}}. Tale servizio permette di recuperare il set di variabili presenti nell'url;
		\item \texttt{-} \texttt{UserDetailsModel: UserDetailsModel} \\
		Campo dati contenente un riferimento alla classe per poter istanziare un oggetto di tipo \texttt{UserDetailsModel};
		\item \texttt{-} \texttt{AuthService: AuthService} \\
		Campo dati contenente un riferimento al servizio che si occupa della gestione delle informazioni legate all’autenticazione;
		\item \texttt{-} \texttt{LangModel: LangModel} \\
		Campo dati contenente un riferimento alla classe rappresenta le informazioni per la giusta traduzione dell'applicazione;
		\item \texttt{-} \texttt{LangService: LangService} \\
		Campo dati contenente un riferimento alla classe che permette di gestire la lingua nella quale si è scelto di utilizzare l'applicazione;
		\item \texttt{-} \texttt{directivesChoose: MenuBarModelView}: \\
		Campo dati contenente un oggetto di tipo modelview la cui istanziazione è contenuta all'interno della variabile di ambiente \texttt{\$rootScope} di \textit{Angular\ped{G}}. All'interno di essa sono presenti le variabili e i metodi necessari per il \textit{Two-Way Data-Binding\ped{G}} tra la \textit{view\ped{G}} \texttt{Index} e il \textit{controller\ped{G}} \texttt{MenuBarController}. Rappresenta un oggetto di tipo \texttt{MenuBarModel};
		\item \texttt{+} \texttt{systemLang: String} \\
		Campo dati contenente una stringa in cui viene memorizzata la lingua del sistama;
		\item \texttt{+} \texttt{listOfKeys: LangModelView}\\ 
		Classe di tipo \textit{modelview\ped{G}} la cui istanziazione è contenuta all'interno della variabile di ambiente \$rootScope di \textit{Angular\ped{G}}. All'interno di essa sono presenti le variabili e i metodi necessari per il \textit{Two-Way Data-Binding\ped{G}} tra le \textit{views\ped{G}} e i \textit{controllers\ped{G}} per la visualizzazione nella lingua corretta dell'applicazione;
		\item \texttt{+} \texttt{userLogged: UserDetailsModelView} \\
		Oggetto di tipo \texttt{UserDetailsModelView}. All'interno di essa sono presenti le variabili e i metodi necessari per il \textit{Two-Way Data-Binding\ped{G}} tra le \textit{views\ped{G}} e i \textit{controllers\ped{G}} che necessitano di utilizzare l'utente autenticato. Rappresenta nello \$rootScope un oggetto di tipo \texttt{UserDetailsModel}.
	\end{itemize}	
		\item \textbf{Metodi}:
		\begin{itemize}
		\item \texttt{+} \texttt{AppController (\$rootScope: \$rootScope, \$location: \$location, \$routeParams: \$routeParams, UserDetailsModel: UserDetailsModel, AuthService: AuthService, LangModel: LangModel, LangService: LangService, MenuBarModel: MenuBarModel )} \\ Metodo costruttore della classe. Una volta che viene creato l'oggetto esso esegue tutte le operazioni possibili per capire il livello di autorizzazione e di scaricare la giusta lingua per l'applicazione.\\
		\textbf{Parametri}: 
		\begin{itemize}
			\item \texttt{\$rootScope: \$rootScope} \\
			Parametro contenente il riferimento all'oggetto globale \$rootScope creato da \textit{An-\\gular\ped{G}}. Viene utilizzato per rendere accessibile a tutti i \textit{controller\ped{G}} e a tutte le \textit{view\ped{G}} l'oggetto \texttt{UserDetailsModel}. In questo caso viene utilizzato per inserire in \$rootScope l'oggetto di ritorno della chiamata a \texttt{getUserDetails} del \textit{service\ped{G}} \texttt{UserDetai-\\lsService};
			\item \texttt{-} \texttt{\$location: \$location} \\
			Campo dati contenente un riferimento al servizio creato da \textit{Angular\ped{G}} che permette di accedere alla barra degli indirizzi del \textit{browser\ped{G}}, i cambiamenti all'URL nella barra degli indirizzi si riflettono in questo oggetto e viceversa; 
			\item \texttt{\$routeParams: \$routeParams} \\
			Campo dati contenente il riferimento all'oggetto globale \$routeParams creato da \textit{Angular\ped{G}}. Tale servizio permette di recuperare il set di variabili presenti nell'url.
			\item \texttt{UserDetailsModel: UserDetailsModel} \\
			Parametro contenente un riferimento alla classe per poter istanziare un oggetto di tipo \texttt{UserDetailsModel};
			\item \texttt{AuthService: AuthService} \\
			Parametro contenente un riferimento al servizio che si occupa della gestione delle informazioni legate all’autenticazione;
			\item \texttt{LangModel: LangModel} \\
			Parametro contenente un riferimento alla classe rappresenta le informazioni per la giusta traduzione dell'applicazione;;
			\item \texttt{LangService: LangService} \\
			Parametro contenente un riferimento alla classe che permette di gestire la lingua nella quale si è scelto di utilizzare l'applicazione
			\item \texttt{MenuBarModel: MenuBarModel}: \\
			Parametro contenente un riferimento all'oggetto che contiene le informazioni per la giusta visualizzazione della barra.
		\end{itemize}
		\item \texttt{-} \texttt{getLang(lang: String): LangModel} \\ Metodo che permette di ottenere i dati con una chiamata a \texttt{LangService}.\\
		\textbf{Parametri}:
		\begin{itemize}
			\item \texttt{lang: String}: parametro che identifica la lingua del sistema.
		\end{itemize}
		\item \texttt{-} \texttt{checkUrl(lang: \$location.path): void} \\ Metodo che gestisce l'autorizzazione dell'utente nella pagina in cui si trova;
		\item \texttt{-} \texttt{checkLang(): void} \\ Metodo che controlla se la lingua presente nell'url sia tra una di quelle supportate dal sistema.
	\end{itemize}
\end{itemize}


\paragraph{QuizziPedia::Front-End::Controllers::LangController}
\begin{figure} [ht]
	\centering
	\includegraphics[scale=0.5]{UML/Classi/Front-End/QuizziPedia_Front-end_Controller_LangController.png}
	\caption{QuizziPedia::Front-End::Controllers::LangController}
\end{figure} \FloatBarrier
\begin{itemize}
	\item \textbf{Descrizione}: questa classe permette di ottenere un nuovo set di variabili per la nuova traduzione dell'applicazione;
	\item \textbf{Utilizzo}: fornisce le funzionalità per cambiare la lingua del sistema;
	\item \textbf{Relazione con altre classi}:
	\begin{itemize}
		\item \textbf{IN \texttt{AppController}}: questa classe permette di gestire per ogni pagina dell'applicazione l'autenticazione e l'autorizzazione dell'utente che si posiziona in essa e mostrare la lingua corretta;
		\item \textbf{IN \texttt{MenuBarController}}: questa classe permette di gestire il menù fisso per ogni pagina.
	\end{itemize}
	\item \textbf{Attributi}:
	\begin{itemize}
		\item \texttt{-} \texttt{\$rootScope: \$rootScope} \\
		Campo dati contenente il riferimento all'oggetto globale \$rootScope creato da \textit{Angular\ped{G}}. Viene utilizzato per rendere accessibile a tutti i \textit{controller\ped{G}} e a tutte le \textit{view\ped{G}} l'oggetto \texttt{UserDetailsModel}. In questo caso viene utilizzato per inserire in \$rootScope l'oggetto di ritorno della chiamata a \texttt{getUserDetails} del \textit{service\ped{G}} \texttt{UserDetailsService};
		\item \texttt{-} \texttt{\$location: \$location} \\
		Campo dati contenente un riferimento al servizio creato da \textit{Angular\ped{G}} che permette di accedere alla barra degli indirizzi del \textit{browser\ped{G}}, i cambiamenti all'URL nella barra degli indirizzi si riflettono in questo oggetto e viceversa; 
		\item \texttt{-} \texttt{LangService: LangService} \\
		Campo dati contenente un riferimento alla classe che permette di gestire la lingua nella quale si è scelto di utilizzare l'applicazione;
		\item \texttt{-} \texttt{\$mdBottomSheet: \$mdBottomSheet}: \\
		Servizio offerto dalla libreria \texttt{Angular Material} che permette di aprire una tendina a scorrimento sopra la vista principale per mostrare un set di bottoni. Implementa le \texttt{promise}. In \textit{QuizziPedia} serve per poter scegliere la lingua con sui visualizzare l'applicazione;
		\item \scopeA.
	\end{itemize}	
		\item \textbf{Metodi}:
		\begin{itemize}
		\item \texttt{+} \texttt{LangController(\$rootScope: \$rootScope, \$location: \$location, LangService: LangService, \$mdBottomSheet: \$mdBottomSheet, \$scope: \$scope)} \\ Metodo costruttore della classe.\\
		\textbf{Parametri}: 
		\begin{itemize}
			\item \texttt{\$rootScope: \$rootScope} \\
			Parametro contenente il riferimento all'oggetto globale \$rootScope creato da \textit{An-\\gular\ped{G}}. Viene utilizzato per rendere accessibile a tutti i \textit{controller\ped{G}} e a tutte le \textit{view\ped{G}} l'oggetto \texttt{UserDetailsModel}. In questo caso viene utilizzato per inserire in \$rootScope l'oggetto di ritorno della chiamata a \texttt{getUserDetails} del \textit{service\ped{G}} \texttt{UserDetai-\\lsService};
			\item \texttt{-} \texttt{\$location: \$location} \\
			Parametro contenente un riferimento al servizio creato da \textit{Angular\ped{G}} che permette di accedere alla barra degli indirizzi del \textit{browser\ped{G}}, i cambiamenti all'URL nella barra degli indirizzi si riflettono in questo oggetto e viceversa; 
			\item \texttt{LangService: LangService} \\
			Parametro contenente un riferimento alla classe che permette di gestire la lingua nella quale si è scelto di utilizzare l'applicazione;
			\item \texttt{\$mdBottomSheet: \$mdBottomSheet}: \\
			Parametro contenente un riferimento al servizio offerto dalla libreria \texttt{Angular Material} che permette di aprire una tendina a scorrimento sopra la vista principale per mostrare un set di bottoni. Implementa le \texttt{promise}. In \textit{QuizziPedia} serve per poter scegliere la lingua con sui visualizzare l'applicazione;
			\item \scopeP.
		\end{itemize}
		\item \texttt{-} \texttt{goToNewLang(lang: String): void} \\ Metodo che permette di cambiare lingua al sistema con una chiamata a \texttt{LangService}.\\
		\textbf{Parametri}:
		\begin{itemize}
			\item \texttt{lang: String}: parametro che identifica la lingua del sistema.
		\end{itemize}
	
	\end{itemize}
\end{itemize}



