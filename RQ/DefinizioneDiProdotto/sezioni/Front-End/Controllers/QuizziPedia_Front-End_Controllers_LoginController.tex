\paragraph{QuizziPedia::Front-End::Controllers::LoginController}
\begin{figure} [ht]
	\centering
	\includegraphics[scale=0.60]{UML/Classi/Front-End/QuizziPedia_Front-end_Controller_LoginController.png}
	\caption{QuizziPedia::Front-End::Controllers::LoginController}
\end{figure} \FloatBarrier
\begin{itemize}
	\item \textbf{Descrizione}: questa classe permette di gestire l'autenticazione dell'utente al sistema; 
	\item \textbf{Utilizzo}: fornisce le funzionalità di autenticazione al sistema, compresa la gestione di situazioni di errore di autenticazione;
	\item \textbf{Relazione con altre classi}:
	\begin{itemize}
		\item \textbf{IN} \texttt{LoginModelView}: classe di tipo modelview la cui istanziazione è contenuta all'interno della variabile di ambiente \$scope di \textit{Angular\ped{G}}. All'interno di essa sono presenti le variabili e i metodi necessari per il \textit{Two-Way Data-Binding\ped{G}} tra la \textit{view\ped{G}} \texttt{LoginView} e il \textit{controller\ped{G}} \texttt{LoginController};
		\item \textbf{IN} \texttt{AuthService}: questa classe permette di gestire la registrazione e l'autenticazione di un utente;
	\end{itemize}
	\item \textbf{Attributi}:
	\begin{itemize}
		\item \texttt{-} \texttt{\$scope: \$scope} \\
		Campo dati contenente un riferimento all’oggetto \$scope creato da \textit{Angular\ped{G}}. Viene utilizzato come mezzo di comunicazione tra il \textit{controller\ped{G}} e la \textit{view\ped{G}}. Contiene gli oggetti che definiscono il viewmodel e il \textit{model\ped{G}} dell’applicazione;
		\item \texttt{-} \texttt{\$location: \$location} \\
		Campo dati contenente un riferimento al servizio creato da \textit{Angular\ped{G}} che permette di accedere alla barra degli indirizzi del \textit{browser\ped{G}}, i cambiamenti all’URL nella barra degli indirizzi si riflettono in questo oggetto e viceversa;
		\item \texttt{-} \texttt{\$mdDialog: \$mdDialog} \\
		Campo dati contenente un riferimento al servizio della libreria \textit{Material for Angular\ped{G}} che permette di creare delle componenti a pop-up;
		\item \texttt{-} \texttt{AuthService: AuthService} \\
		Campo dati contenente un riferimento al servizio che si occupa della gestione delle informazioni legate all'autenticazione. Viene utilizzato il metodo \texttt{logIn()} di \$texttt{AuthService} a cui vengono passati i parametri \texttt{username} e \texttt{password};
		\item \texttt{+} \texttt{user: LoginModelView} \\
		Oggetto di tipo \texttt{LoginModelView}. All'interno di esso sono presenti le variabili e i metodi necessari per il \textit{Two-Way Data-Binding\ped{G}} tra la \textit{view\ped{G}} \texttt{LoginView} e il \textit{controller\ped{G}} \texttt{LoginController};
	\end{itemize}
	\item \textbf{Metodi}:
	\begin{itemize}
		\item \texttt{+} \texttt{LoginController(\$scope: \$scope, \$rootScope: \$rootScope, \$location:\\ \$location, \$mdDialog: \$mdDialog, AuthService: AuthService)} \\
		Metodo costruttore della classe. \\
		\textbf{Parametri}:
			\begin{itemize}
				\item \texttt{\$scope: \$scope} \\
				Parametro contenente un riferimento all’oggetto \$scope creato da \textit{Angular\ped{G}}. Viene utilizzato come mezzo di comunicazione tra il \textit{controller\ped{G}} e la \textit{view\ped{G}}. Contiene gli oggetti che definiscono il viewmodel e il \textit{model\ped{G}} dell'applicazione;
				\item \texttt{\$location: \$location} \\
				Parametro contenente un riferimento al servizio creato da \textit{Angular\ped{G}} che permette di accedere alla barra degli indirizzi del \textit{browser\ped{G}}, i cambiamenti all’URL nella barra degli indirizzi si riflettono in questo oggetto e viceversa;
				\item \texttt{\$mdDialog: \$mdDialog} \\
				Parametro contenente un riferimento al servizio della libreria \textit{Material for Angular\ped{G}} che permette di creare delle componenti a pop-up;
				\item \texttt{AuthService: AuthService} \\
				Parametro contenente un riferimento al servizio che si occupa della gestione delle informazioni legate all'autenticazione. Viene utilizzato il metodo \texttt{logIn()} di \$texttt{AuthService} a cui vengono passati i parametri \texttt{email} e \texttt{password};
			\end{itemize}
		\item \texttt{+} \texttt{logIn(email: String, password: String): void} \\
		Metodo che richiama il metodo \texttt{signin} del service \texttt{AuthService} passandogli \texttt{email} e \texttt{password}. Nel caso di buona riuscita dell'operazione viene effettuato il redirect alla homepage dell'applicazione. Nel caso in cui invece avvenga un errore, viene mostrato a video il messaggio di errore;
		\item \texttt{+} \texttt{signUp(): void} \\
		Metodo che gestisce l'evento click sul pulsante di registrazione. Effettua il redirect alla pagina di registrazione;
		\item \texttt{+} \texttt{recoveryPassword(): void} \\
		Metodo che gestisce l'evento click sul pulsante di recupero password. Effettua il redirect alla pagina per il recupero della password; 
	\end{itemize}
\end{itemize}

