\newpage
\section{Resoconto attività di verifica}
In questa sezione del documento vengono descritti e analizzati gli esiti delle attività di verifica svolte su tutti i documenti che vengono consegnati nelle varie revisioni di avanzamento del progetto.

\subsection{Revisione dei Requisiti}

\subsubsection{Tracciamento}
Il \textit{team\ped{G}} ha deciso di utilizzare il software interno DocumentsDB in modo da facilitare il tracciamento sia delle relazioni fra casi d'uso e requisiti, sia delle relazioni fra requisiti e fonti.

\subsubsection{Analisi statica dei documenti}
L'analisi dei documenti mediante \textit{Walkthrough\ped{G}} ha portato all'individuazione di alcuni errori frequenti a partire dai quali è stata stilata una lista di controllo che è stata inserita all'interno dei Processi Organizzativi nelle \textit{\NdP}. Grazie a questa sarà possibile applicare l'\textit{Inspection\ped{G}} per le future attività di verifica.

\subsubsection{Esiti verifiche automatizzate}
Vengono qui riportati gli esiti delle verifiche automatizzate per il calcolo dell'\textit{indice Gulpease\ped{G}}, alle quali sono stati sottoposti tutti i documenti.
\begin{table}[h]
\begin{center}
\begin{tabular}{|c|c|c|c|}
\hline \textbf{Documento} & \textbf{Indice Gulpease} & \textbf{Esito}\\
\hline
\emph{Norme di Progetto} & 68 & Superato \\
\emph{Studio di Fattibilità} & 56 & Superato \\
\emph{Piano di Progetto} & 60 & Superato \\
\emph{Piano di Qualifica} & 57  & Superato \\
\emph{Analisi dei Requisiti} & 72 & Superato \\
\emph{Glossario} & 48 & Superato \\
\emph{Verbale Interno 2015-12-03} & 75 & Superato \\
\emph{Verbale Interno 2015-12-11} & 74 & Superato \\
\emph{Verbale Interno 2015-12-29} & 75 & Superato \\
\emph{Verbale Interno 2016-01-12} & 75 & Superato \\
\emph{Verbale Esterno 2016-01-11} & 76 & Superato \\
\hline
\end{tabular}
\caption{Resoconto verifiche automatizzate - Revisione dei Requisiti}
\end{center}
\end{table}

\subsection{Revisione di Progettazione}

\subsubsection{Tracciamento}
Il \textit{team\ped{G}} attraverso il software interno DocumentsDB è riuscito a effettuare il tracciamento sia delle relazioni fra requisiti e componenti che fra requisiti e classi. Questo software è stato utilizzato inoltre per generare le tabelle dei vari tipi di test e dei relativi tracciamenti con requisiti, componenti, classi e metodi.

\subsubsection{Analisi statica dei documenti}
L'analisi dei documenti mediante \textit{Walkthrough\ped{G}} ha portato all'individuazione di alcuni errori frequenti a partire dai quali è stata ampliata la lista di controllo presente all'interno dei Processi Organizzativi nelle \textit{\NdP}. Grazie a questa sarà possibile applicare l'\textit{Inspection\ped{G}} per le future attività di verifica.

\subsubsection{Esiti verifiche automatizzate}
Vengono qui riportati gli esiti delle verifiche automatizzate per il calcolo dell'\textit{indice Gulpease\ped{G}}, alle quali sono stati sottoposti tutti i documenti.
\begin{table}[h]
\begin{center}
\begin{tabular}{|c|c|c|c|}
\hline \textbf{Documento} & \textbf{Indice Gulpease} & \textbf{Esito}\\
\hline
\emph{Definizione di Prodotto} & 61 & Superato \\
\emph{Norme di Progetto} & 70 & Superato \\
\emph{Piano di Progetto} & 61 & Superato \\
\emph{Piano di Qualifica} & 68 & Superato \\
\emph{Analisi dei Requisiti} & 70 & Superato \\
\emph{Glossario} & 49 & Superato \\
\emph{Verbale Interno 2015-12-03} & 75 & Superato \\
\emph{Verbale Interno 2015-12-11} & 74 & Superato \\
\emph{Verbale Interno 2015-12-29} & 75 & Superato \\
\emph{Verbale Interno 2016-01-12} & 75 & Superato \\
\emph{Verbale Interno 2016-02-25} & 76 & Superato \\
\emph{Verbale Interno 2016-03-01} & 74 & Superato \\
\emph{Verbale Interno 2016-03-07} & 76 & Superato \\
\emph{Verbale Esterno 2016-01-11} & 75 & Superato \\
\emph{Verbale Esterno 2016-03-11} & 76 & Superato \\
\hline
\end{tabular}
\caption{Resoconto verifiche automatizzate - Revisione di Progettazione}
\end{center}
\end{table}

\subsection{Revisione di Qualifica}

\subsubsection{Tracciamento}
Il \textit{team\ped{G}} attraverso il software interno DocumentsDB è riuscito a effettuare il tracciamento sia delle relazioni fra requisiti e componenti che fra requisiti e classi. Questo software è stato utilizzato inoltre per generare le tabelle dei vari tipi di test e dei relativi tracciamenti con requisiti, componenti, classi e metodi.

\subsubsection{Analisi statica dei documenti}
La lista degli errori comuni individuati nelle fasi precedenti ha permesso di effettuare sui documenti un'analisi mediante l'\textit{Inspection\ped{G}} che è stata comunque affiancata a una parte di analisi mediante \textit{Walkthrough\ped{G}}, la quale ha portato all'individuazione di alcuni nuovi errori frequenti che sono stati aggiunti alla lista di controllo presente nell'apposita sezione dei Processi Organizzativi delle \textit{\NdP}.

\subsubsection{Esiti verifiche automatizzate}
Vengono qui riportati gli esiti delle verifiche automatizzate per il calcolo dell'\textit{indice Gulpease\ped{G}}, alle quali sono stati sottoposti tutti i documenti.
\begin{table}[h]
\begin{center}
\begin{tabular}{|c|c|c|c|}
\hline \textbf{Documento} & \textbf{Indice Gulpease} & \textbf{Esito}\\
\hline
\emph{Definizione di Prodotto} &  & Superato \\
\emph{Norme di Progetto} &  & Superato \\
\emph{Piano di Progetto} &  & Superato \\
\emph{Piano di Qualifica} &  & Superato \\
\emph{Analisi dei Requisiti} & 70 & Superato \\
\emph{Glossario} & 49 & Superato \\
\emph{Verbale Interno 2015-12-03} & 75 & Superato \\
\emph{Verbale Interno 2015-12-11} & 74 & Superato \\
\emph{Verbale Interno 2015-12-29} & 75 & Superato \\
\emph{Verbale Interno 2016-01-12} & 75 & Superato \\
\emph{Verbale Interno 2016-02-25} & 76 & Superato \\
\emph{Verbale Interno 2016-03-01} & 74 & Superato \\
\emph{Verbale Interno 2016-03-07} & 76 & Superato \\
\emph{Verbale Esterno 2016-01-11} & 75 & Superato \\
\emph{Verbale Esterno 2016-03-11} & 76 & Superato \\
\hline
\end{tabular}
\caption{Resoconto verifiche automatizzate - Revisione di Progettazione}
\end{center}
\end{table}

\newpage
\subsubsection{Soddisfacimento metriche}

\paragraph{Qualità di processo}
\begin{longtable}{|>{\centering}m{5cm}|c|c|c|c|}
\hline
\textbf{Metrica} & \textbf{Unità di misura} & \textbf{Valore} & \textbf{Accettazione} & \textbf{Ottimalità}\\
\hline
\endhead
\emph{Disponibilità DocumentsDB} & {Percentuale} & \textcolor{Green}{100} & $80 - 100$ & $90 - 100$\\ \hline
\emph{Tempo di correzione incoerenze in DocumentsDB} & {Giorni} & \textcolor{Orange}{1.3} & $0 - 3$ & $0 - 1$\\ \hline
\emph{Schedule Variance} & {Attività} & \textcolor{Green}{0} & $\geq 0$  & $\geq 0$\\ \hline
\emph{Budget Variance} & {Euro} & \textcolor{Green}{30.00} & $\geq 0$ & $\geq 0$\\ \hline
\emph{Rischi non preventivati} & {Rischi} & \textcolor{Green}{0} & $0 - 5$ & $0$\\ \hline
%\emph{Efficienza di gestione dei rischi} & {Giorni} & \textcolor{Orange}{21.3} & $\geq 20$ & $\geq 60$\\ \hline
\emph{Requisiti obbligatori soddisfatti} & {Percentuale} & \textcolor{Green}{100} & $100$ & $100$\\ \hline
%\emph{Structural Fan-In} & {Moduli} & \textcolor{Orange}{5} & $\geq 0$ & $\geq 2$\\ \hline
%\emph{Structural Fan-Out} & {Moduli} & \textcolor{Orange}{4.71} & $0 - 5$ & $0 - 1$\\ \hline
\emph{Numero di metodi per classe} & {Metodi} & \textcolor{Green}{3.2} & $1 - 10$ & $1 - 7$\\ \hline
\emph{Numero di parametri per metodo} & {Parametri} & \textcolor{Green}{2.8} & $0 - 8$ & $0 - 4$\\ \hline
\emph{Produttività di codifica} & {Linee} & \textcolor{Orange}{9.1} & $\geq 3$ & $\geq 10$\\ \hline
%\emph{Complessità Ciclomatica media} & {Cammini} & \textcolor{Green}{0} & $1 - 15$ & $1 - 10$\\ \hline
%\emph{Livelli di annidamento medi} & {Chiamate} & \textcolor{Green}{1} & $1 - 6$ & $1 - 3$\\ \hline
\emph{Linee di codice per linee di commento} & {Percentuale} & \textcolor{Green}{31} & $\geq 25$ & $\geq 30$\\ \hline
\emph{Variabili inutilizzate} & {Variabili} & \textcolor{Green}{0} & $0$ & $0$\\ \hline
\emph{Dipendenze} & {Chiamate require} & \textcolor{Green}{2.2} & $0 - 10$ & $0 - 5$\\ \hline
%\emph{Halstead Difficulty media} & {Percentuale} & \textcolor{Green}{0} & $0 - 25$ & $0 - 15$\\ \hline
%\emph{Halstead Volume media} & {Percentuale} & \textcolor{Green}{20} & $20 - 1500$ & $20 - 1000$\\ \hline
%\emph{Halstead Effort media} & {Percentuale} & \textcolor{Green}{0} & $0 - 400$ & $0 - 300$\\ \hline
%\emph{Indice di manutenibilità} & {Percentuale} & \textcolor{Orange}{5.14} & $100 - 171$ & $120 - 171$\\ \hline
\emph{Componenti integrate} & {Percentuale} & \textcolor{Green}{100} & $100$ & $100$\\ \hline
\emph{Test di Unità eseguiti} & {Percentuale} & \textcolor{Red}{36.6} & $90 - 100$ & $100$\\ \hline
\emph{Test di Integrazione eseguiti} & {Percentuale} & \textcolor{Red}{0} & $60 - 100$ & $70 - 100$\\ \hline
\emph{Test di Sistema eseguiti} & {Percentuale} & \textcolor{Red}{0} & $70 - 100$ & $80 - 100$\\ \hline
\emph{Test di Validazione eseguiti} & {Percentuale} & \textcolor{Red}{0} & $100$ & $100$\\ \hline
\emph{Test superati} & {Percentuale} & \textcolor{Green}{100} & $90 - 100$ & $100$\\ \hline
%\emph{Branch Coverage} & {Percentuale} & \textcolor{Orange}{73.3} & $70 - 100$ & $80 - 100$\\ \hline
%\emph{Code Coverage} & {Percentuale} & \textcolor{Green}{76.57} & $60 - 100$ & $70 - 100$\\ \hline
\caption{Metriche principali di qualità di processo}
\end{longtable}

\newpage
\paragraph{Qualità di prodotto}
\begin{longtable}{|>{\centering}m{5cm}|c|c|c|c|}
\hline
\textbf{Metrica} & \textbf{Unità di misura} & \textbf{Valore} & \textbf{Accettazione} & \textbf{Ottimalità}\\
\hline
\endhead
\emph{Completezza dell'implementazione funzionale} & {Percentuale} & \textcolor{Green}{100} & $100$ & $100$\\ \hline
\emph{Accuratezza rispetto alle attese} & {Percentuale} & \textcolor{Green}{100} & $90 - 100$ & $100$\\ \hline
\emph{Controllo degli accessi} & {Percentuale} & \textcolor{Green}{100} & $90 - 100$ & $100$\\ \hline
\emph{Densità di failure} & {Percentuale} & \textcolor{Green}{0} & $0 - 10$  & $0$\\ \hline
\emph{Blocco di operazioni non corrette} & {Percentuale} & \textcolor{Green}{100} & $80 - 100$  & $100$\\ \hline
%\emph{Comprensibilità delle funzioni offerte} & {Percentuale} & \textcolor{Green}{92.28} & $80 - 100$  & $90 - 100$\\ \hline
%\emph{Facilità di apprendimento delle funzionalità} & {Minuti} & \textcolor{Green}{7.42} & $0 - 30$ & $0 - 15$\\ \hline
%\emph{Consistenza operazionale in uso} & {Percentuale} & \textcolor{Orange}{82.75} & $80 - 100$ & $90 - 100$\\ \hline
\emph{Tempo di risposta} & {Secondi} & \textcolor{Green}{2.7} & $0 - 8$ & $0 - 3$\\ \hline
\emph{Capacità di analisi di failure} & {Percentuale} & \textcolor{Orange}{75} & $60 - 100$ & $80 - 100$\\ \hline
\emph{Impatto delle modifiche} & {Percentuale} & \textcolor{Orange}{15} & $0 - 20$ & $0 - 10$\\ \hline
\emph{Versioni dei browser supportate} & {Percentuale} & \textcolor{Green}{100} & $100$ & $100$\\ \hline
%\emph{Inclusione di funzionalità da altri prodotti} & {Percentuale} & \textcolor{Red}{68.87} & $80 - 100$ & $90 - 100$\\ \hline
\caption{Metriche principali di qualità di prodotto}
\end{longtable}