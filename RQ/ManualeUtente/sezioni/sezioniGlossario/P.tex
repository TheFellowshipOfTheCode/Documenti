\subsection{P}
\begin{itemize} 
	\item
	\textbf{Package}: è un meccanismo per organizzare classi Java, logicamente correlate o che forniscono servizi simili, all’interno di sottogruppi ordinati. Questi package possono essere compressi permettendo la trasmissione di più classi in una sola volta. In UML, analogamente, è un raggruppamento arbitrario di elementi in una unità di livello più alto.
	\item
	\textbf{Parser}: il parsing, analisi sintattica o parsificazione è un processo che analizza un flusso continuo di dati in ingresso (input) (letti per esempio da un file o una tastiera) in modo da determinare la sua struttura grazie ad una data grammatica formale. Un parser è un programma che esegue questo compito.
	\textbf{Passport}: è un middleware per l'autenticazione degli utenti al sistema.
	\item
	\textbf{Promise}: oggetti che rappresentano il risultato di una chiamata di funzione asincrona, più semplicemente rappresentano una promessa che un risultato verrà fornito non appena disponibile.
\end{itemize}

