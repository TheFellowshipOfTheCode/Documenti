\subsection{D}
\begin{itemize} 
	\item
	\textbf{Design Pattern}: è un concetto che può essere definito "una soluzione progettuale generale ad un problema ricorrente". Si tratta di una descrizione o modello logico da applicare per la risoluzione di un problema che può presentarsi in diverse situazioni durante le fasi di progettazione e sviluppo del software, ancor prima della definizione dell'algoritmo risolutivo della parte computazionale. È un approccio spesso efficace nel contenere o ridurre il debito tecnico.
	I design pattern orientati agli oggetti tipicamente mostrano relazioni ed interazioni tra classi o oggetti, senza specificare le classi applicative finali coinvolte, risiedendo quindi nel dominio dei moduli e delle interconnessioni. Ad un livello più alto sono invece i pattern architetturali che hanno un ambito ben più ampio, descrivendo un pattern complessivo adottato dall'intero sistema, la cui implementazione logica dà vita ad un framework.
	\item
	\textbf{Dependency Injection}: è un design pattern della programmazione orientata agli oggetti il cui scopo è quello di semplificare lo sviluppo e migliorare la testabilità di software di grandi dimensioni. Il pattern Dependency Injection coinvolge almeno tre elementi: una componente dipendente, la dichiarazione delle dipendenze del componente, definite come interface contracts e un injector (chiamato anche provider o container) che crea, a richiesta, le istanze delle classi che implementano delle dependency interfaces.
	\item
	\textbf{Directive}: esse rappresentano un modo di estendere il linguaggio HTML con elementi e attributi personalizzati. Questo si riallaccia inoltre all’ottica per cui ogni componente Angular ha un suo ruolo specifico. In particolare il ruolo delle direttive consiste nell’estendere le potenzialità dell’approccio dichiarativo dell’HTML nella costruzione di interfacce grafiche.
\end{itemize}