\newpage
\section{QML - Quiz Markup Language}
Uno dei punti fondamentali dell'applicazione è permettere agli utenti di poter creare domande da proporre negli allenamenti e nei questionari. Per rendere più semplice l'aggiunta di una domanda sono stati realizzati dei wizard per alcune tipologie di domande, ma questo limita l'utente alla sola creazione di tali tipologie. La soluzione adottata è la realizzazione di un nuovo linguaggio di markup che permette la definizione di domande non ordinarie, denominato \textit{QML\ped{G}} acronimo di Quiz Markup Language.

\subsection{Definizione della grammatica}
Per mantenere la semplicità di realizzazione di una domanda tramite \textit{QML\ped{G}} è stato deciso di mantenere la sintassi \textit{JSON\ped{G}}, con l'aggiunta delle sole parole chiave necessarie per mantenere coerente la grammatica con lo scopo del \textit{QML\ped{G}}. Alla grammatica è stata data possibilità di inserire commenti che saranno rimossi prima della creazione del \textit{JSON\ped{G}}.

\subsubsection{Sintassi domanda Vero/Falso}
\begin{lstlisting}[language=json,firstnumber=1]
{	
"type": "veroFalso",
"topic" : "Patente",
"image": "/img/veroFalso/example.png", //immagine possibile nel testo della domanda vero e falso
"answers":
	[{
	"text": "In Italia la guida e' a destra",
	"isItRight": true
	}],
"keywords":
	[{
	"guida",
	"legge",
	"italia"
	}] 
}
\end{lstlisting}

\subsubsection{Sintassi domanda a Risposta Multipla}
\begin{lstlisting}[language=json,firstnumber=1]
{	
"type" : "rispostaMultipla",
"topic" : "Matematica",
"questionText": "Quali di questi numeri e' pari?",
"url": "/img/rispostaMultipla/D0_3.png", //immagine possibile nel testo della domanda risposta multipla
"answers":
	[{
	"text": "1",
	"url": "/img/rispostaMultipla/D0_1.png", //url dell'immagine, possibile campo facoltativo
	"isItRight": false
	 },{
	"text": "2",
	"url": "/img/rispostaMultipla/D0_2.png", //url dell'immagine, possibile campo facoltativo
	"isItRight": true
	 },{
	"text": "7",
	"url": "/img/rispostaMultipla/D0_3.png", //url dell'immagine, possibile campo facoltativo
	"isItRight": false
	 },{
	"text": "9",
	"url": "/img/rispostaMultipla/D0_4.png", //url dell'immagine, possibile campo facoltativo
	"isItRight": false
	 }],
"keywords":
	[{
	"numeri",
	"matematica",
	"scuola"
	}]
}
\end{lstlisting}

\subsubsection{Sintassi domanda a Ordinamento di Stringhe}
\begin{lstlisting}[language=json,firstnumber=1]
{
"type": "ordinamentoStringhe",
"topic" : "Matematica",
"questionText": "Ordina questi numeri in modo decrescente.",
"answer":
	[{
	"text": "1",
	"position": 4
	},{
	"text": "2",
	"position": 3
	},{
	"text": "7",
	"position": 2
	},{
	"text": "9",
	"position": 1
	}],
"keywords":
	[{
	"ordinamento",
	"numeri",
	}]
}
\end{lstlisting}

\subsubsection{Sintassi domanda a Ordinamento Immagini}
\begin{lstlisting}[language=json,firstnumber=1]
{
"type": "ordinamentoImmagini",
"topic" : "Grammatica Italiana",
"questionText": "Ordina queste immagini in modo da ottenere la parola "CIAO".",
"url": "/img/ordinamentoImmagini/D0_1.png", //immagine possibile nel testo della domanda ad ordianamento di immagini
"answer":
	[{
	"url": "/img/domandeOrdinamentoImmagini/I.png", //url dell'immagine
	"position": 2
	},{
	"url": "/img/domandeOrdinamentoImmagini/A.png",
	"position": 3
	},{
	"url": "/img/domandeOrdinamentoImmagini/O.png",
	"position": 4
	},{
	"url": "/img/domandeOrdinamentoImmagini/C.png",
	"position": 1
	}],
"keywords":
	[{
	"parole",
	"grammatica"
	}]
}
\end{lstlisting}

\subsubsection{Sintassi domanda a Collegamento di Elementi}
\begin{lstlisting}[language=json,firstnumber=1]
{
"type": "collegamentoElementi",
"topic" : "Cultura Generale",
"questionText": "Collega queste coppie di nemici classici.",
"answer":
	[{
	"text_1_A": "cane",
	"text_1_B": "gatto"
	},{
	"url_2_A": "/img/collegamento/uncino.png",
	"text_2_B": "peter pan"
	},{
	"url_3_A": "/img/collegamento/D0_1.png",
	"url_3_B": "/img/collegamento/D0_5.png"
	}],
"keywords":
	[{
	"nemici"
	}]
}
\end{lstlisting}

\subsubsection{Sintassi domanda ad Area Cliccabile}
\begin{lstlisting}[language=json,firstnumber=1]
{
"type": "areaCliccabile",
"topic" : "Medicina",
"questionText": "Clicca quale tra le seguenti scelte e' il bicipide.",
"image": "/img/areaCliccabile/D0_1.png", //sfondo dell'area cliccabile
"resolution": { "x":400, "y":500 },
"answer":
	[{
	"x": "200",
	"y": "50",
	"text": "testo facoltativo di arricchimento"
	},{
	"x": "300",
	"y": "120",
	"text": "testo facoltativo di arricchimento"
	},{
	"x": "200",
	"y": "200",
	"text": "testo facoltativo di arricchimento"
	}],
"keywords":
	[{
	"corpo umano",
	"medicina",
	"scienze" ,
	"muscoli"
	}]
}
\end{lstlisting}

\subsubsection{Sintassi domanda a Riempimento spazi vuoti}
\begin{lstlisting}[language=json,firstnumber=1]
{
"type": "spaziVuoti",
"topic" : "Storia"
"questionText": "Giulio Cesare era un console romano.",
"answer":
	[{
	"parolaNumero": 2 //sta ad indicare quale parola e' da oscurare. In questo caso la numero 2
	},{
	"parolaNumero": 5
	}],
"keywords":
	[{
	"storia",
	}] 
}
\end{lstlisting}