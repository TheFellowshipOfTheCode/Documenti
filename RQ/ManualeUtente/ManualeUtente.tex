%Generazione delle variabili che andranno a sostituire quelle del template `HomePage.tex`
\newcommand{\documento}{Manuale Utente}
\newcommand{\nomedocumentofisico}{ManualeUtente\_v1\_0\_0.pdf}
\newcommand{\redazione}{\GR\\ & \GN\\ & \SM\\ & \MV\\ & \FB\\ & \AF}
\newcommand{\verifica}{\GR\\ & \MP\\ & \MV}
\newcommand{\approvazione}{\FB}
\newcommand{\uso}{Esterno}
\newcommand{\destinateTo}{\TV, \\ & \RC, \\ & \ZU}
\newcommand{\datacreazione}{07 Maggio 2016}
\newcommand{\datamodifica}{07 Maggio 2016}
\newcommand{\stato}{In redazione}
\newcommand{\versione}{1}

\def\TABELLE{false}	%abilita - disabilita l'indice delle tabelle
\def\FIGURE{false} 	%abilita - disabilita l'indice delle figure

%Layout del documento 
\documentclass[a4paper,11pt]{article}

%***IMPORTAZIONE PACKAGE***
\usepackage{ifthen}
\usepackage[italian]{babel}
\usepackage[utf8]{inputenc}
\usepackage[T1]{fontenc}
\usepackage{float}
\usepackage{chapterbib}
\usepackage{graphicx}
\usepackage[a4paper,top=2.5cm,bottom=2.5cm,left=2.5cm,right=2.5cm]{geometry}
\usepackage[colorlinks=true, urlcolor=black, citecolor=black, linkcolor=black]{hyperref}
\usepackage{booktabs}
\usepackage{fancyhdr}
\usepackage{totpages}
\usepackage{tabularx, array}
\usepackage{dcolumn}
\usepackage{epstopdf}
\usepackage{booktabs}
\usepackage{fancyhdr}
\usepackage{longtable}
\usepackage{calc}
\usepackage{datatool}
\usepackage[bottom]{footmisc}
\usepackage{listings} 

%***STILE PAGINA***
\pagestyle{fancy}
%no indentazione paragrafo
\setlength{\parindent}{0pt}

%***INTESTAZIONE***
\lhead{\Large{\progetto} \\ \footnotesize{\documento}}
\rhead{\includegraphics[keepaspectratio = true, width = 25px] {../../Template/icone/logo.jpg}}
\renewcommand{\headrulewidth}{0.4pt}  %Linea sotto l'intestazione

%***PIÈ DI PAGINA***
\lfoot{\textit{\gruppoLink}\\ \footnotesize{\email}}
\rfoot{\thepage} %per le prime pagine: mostra solo il numero romano
\cfoot{}
\renewcommand{\footrulewidth}{0.4pt}   %Linea sopra il piè di pagina



%Comandi generali
%Generali
\newcommand{\progetto}{QuizziPedia}
\newcommand{\gruppo}{TheFellowshipOfTheCode}
\newcommand{\gruppoLink}{\href{http://thefellowshipofthecode.github.io/}{TheFellowshipOfTheCode}}
\newcommand{\email}{\href{mailto:thefellowshipofthecode@gmail.com}{thefellowshipofthecode@gmail.com}}

%Documenti
\newcommand{\AdR}{Analisi dei Requisiti}
\newcommand{\NdP}{Norme di Progetto}
\newcommand{\PdP}{Piano di Progetto}
\newcommand{\SdF}{Studio di Fattibilità}
\newcommand{\PdQ}{Piano di Qualifica}
\newcommand{\VI}{Verbale Interno}
\newcommand{\VE}{Verbale Esterno}
\newcommand{\ST}{Specifica Tecnica}
\newcommand{\DDP}{Definizione di Prodotto}
\newcommand{\MU}{Manuale Utente}
\newcommand{\G}{Glossario}
\newcommand{\LdP}{Lettera di Presentazione}

%Componenti del gruppo
\newcommand{\AF}{Alberto Ferrara}
\newcommand{\SM}{Simone Magagna}
\newcommand{\FB}{Franco Berton}
\newcommand{\MP}{Marco Prelaz}
\newcommand{\MV}{Mattia Varotto}
\newcommand{\GN}{Matteo Gnoato}
\newcommand{\GR}{Matteo Granzotto}

%Ruoli
\newcommand{\RdP}{Responsabile di Progetto}
\newcommand{\Res}{Responsabile}
\newcommand{\Amm}{Amministratore}
\newcommand{\Ver}{Verificatore}
\newcommand{\Prog}{Progettista}
\newcommand{\Progr}{Programmatore}
\newcommand{\Ana}{Analista}
\newcommand{\RdPs}{Responsabili di Progetto}
\newcommand{\Ress}{Responsabile}
\newcommand{\Amms}{Amministratori}
\newcommand{\Vers}{Verificatori}
\newcommand{\Progs}{Progettisti}
\newcommand{\Progrs}{Programmatori}
\newcommand{\Anas}{Analisti}

%Professori e proponente
\newcommand{\TV}{Tullio Vardanega}
\newcommand{\RC}{Riccardo Cardin}
\newcommand{\ZU}{Zucchetti S.P.A.}
\newcommand{\proponente}{Zucchetti S.P.A}

\newcommand{\diaryEntry}[5]{#2 & \emph{#4} & #3 & #5 & #1\\ \hline}

%comando per una nuova riga nella tabella del diario delle modifiche
\newcommand{\specialcell}[2][c]{%
	\begin{tabular}[#1]{@{}c@{}}#2\end{tabular}}

\renewcommand*\sectionmark[1]{\markboth{#1}{}}
\renewcommand*\subsectionmark[1]{\markright{#1}}

%Pediodi di lavoro 
\newcommand{\AR}{Analisi dei Requisiti}
\newcommand{\AD}{Analisi dei Requisiti in Dettaglio}
\newcommand{\PA}{Progettazione Architetturale}
\newcommand{\PD}{Progettazione di Dettaglio}
\newcommand{\CO}{Codifica}
\newcommand{\VV}{Verifica e Validazione}

% Revisioni
\newcommand{\RR}{Revisione dei Requisiti}
\newcommand{\RP}{Revisione di Progettazione}
\newcommand{\RQ}{Revisione di Qualifica}
\newcommand{\RA}{Revisione di Accettazione}

% Comandi analisi dei requisiti
\newcommand{\uau}{utente autenticato}
\newcommand{\uaupro}{utente autenticato pro}

\begin{document}
	
%inclusione template HomePage
\begin{center}

\includegraphics[width=1em]{../../Template/icone/logo_fotc.jpg}
\begin{large} \textbf{\gruppo} \end{large}
\includegraphics[width=1em]{../../Template/icone/logo_fotc.jpg}
\vspace{0.2em}

\hrule
\vspace{3em}

\includegraphics[width=15em]{../../Template/icone/logo.jpg}


\begin{center} 
  \begin{Huge}
  {\fontsize{15mm}{20mm}\selectfont \progetto} 
  \end{Huge}
\end{center}

\begin{Huge} \documento \end{Huge}

\begin{center}
\textbf{Informazioni sul documento} \\ \vspace{2em}
\small
\begin{tabular}{r|l}
	\textbf{Nome Documento} & \nomedocumentofisico \\
	\textbf{Versione}	& 1\\
	\textbf{Data di Creazione} & \datacreazione\\
	\textbf{Data ultima modifica} & \datamodifica\\
	\textbf{Stato} & \stato \\
	\textbf{Redazione}	& \redazione\\
	\textbf{Verifica}	& \verifica\\
	\textbf{Approvazione}	& \approvazione\\
	\textbf{Uso}  & \uso\\
	\textbf{Distribuzione} & \gruppo \\
	\textbf{Destinato a}  &  \destinateTo
\end{tabular}
\end{center}
\normalsize

\vspace{4em}

\textbf{Sommario\\} 
Documento contenente il sesto verbale interno dell'07 Marzo 2016 per il progetto \progetto{} del gruppo \gruppo{}.


\vfill

\end{center}



%Registro delle modifiche e indice 
%si usa la numerazione romana per gli indici e la tabella delle modifiche
\pagenumbering{Roman}
\newcommand{\modifiche}
{
	1.0.0 & Approvazione documento & \specialcell[t]{\GR\\Responsabile} & 2015-12-10 \\\midrule
	0.1.0 & Verifica documento & \specialcell[t]{\MV\\Verificatore} & 2015-12-09 \\\midrule
	0.0.1 & Creazione e completamento stesura del documento & \specialcell[t]{\AF\\Amministratore} & 2015-12-03 \\
}
\input{../../Template/index.tex}
\newpage
\listoffigures
\newpage
\listoftables
\newpage
%sezioni documento
\newpage

\section{Introduzione}

\subsection{Scopo del documento}
Il presente documento ha lo scopo di definire in dettaglio la struttura e il funzionamento delle componenti del progetto \progetto. Questo documento servirà come guida per i \textit{\Progrs} del gruppo \gruppo fornendo direttive e vincoli per la realizzazione del \textit{progetto\ped{G}}.

\subsection{Scopo del prodotto}
Lo scopo del prodotto è di permettere la creazione e gestione di questionari in grado di identificare le lacune dei candidati prima, durante e al termine di un corso di formazione. 
\\Il sistema dovrà offrire le seguenti funzionalità:
\begin{itemize}
	\item
	Archiviare questionari in un server suddivisi per argomento;
	\item
	Somministrare all'utente, tramite un'interfaccia, questionari specifici per argomento scelto;
	\item
	Verificare e valutare i questionari scelti dagli utenti in base alle risposte date.
\end{itemize}
La parte destinata ai creatori di questionari dovrà essere fruibile attraverso un \textit{browser\ped{G}} desktop, abilitato all'utilizzo delle tecnologie \textit{HTML5\ped{G}}, \textit{CSS3\ped{G}} e \textit{JavaScript\ped{G}}. La parte destinata agli esaminandi sarà utilizzabile su qualunque dispositivo: dal personal computer ai tablet e smartphone.

\subsection{Glossario}
Al fine di evitare ogni ambiguità i termini tecnici del dominio del progetto, gli acronimi e le parole che necessitano di ulteriori spiegazioni saranno nei vari documenti marcate con il pedice \ped{G} e quindi presenti nel documento \textit{\G}.


\subsection{Riferimenti}
\subsubsection{Normativi}
\begin{itemize}
	\item \textit{\NdPv};
	\item \textit{\AdRvDue};
\end{itemize}
\subsubsection{Informativi}
\begin{itemize}
	\item \textbf{Ingegneria del software - Ian Sommerville - 8a edizione (2007)}: \\
	Parte terza: Progettazione, capitolo 11: Progettazione architetturale, Capitolo 14: Progettazione orientata agli oggetti;
	\item \textbf{Design Patterns} - Erich Gamma, Richard Helm, Ralph Johnson, John Vlissides - 1a edizione italiana (2006);
	\item \textbf{Slide dell'insegnamento - Design patterns:}
	\begin{itemize}
		\item Strutturali: \url{http://www.math.unipd.it/~tullio/IS-1/2015/Dispense/E07.pdf}
		\item Creazionali: \url{http://www.math.unipd.it/~tullio/IS-1/2015/Dispense/E08.pdf}
		\item Comportamentali: \url{http://www.math.unipd.it/~tullio/IS-1/2015/Dispense/E09.pdf}
		\item Architetturali:
			\begin{itemize}
				\item \burl{http://www.math.unipd.it/~rcardin/sweb/Design\%20Pattern\%20Architetturali\%20-\%20Model\%20View\%20Controller_4x4.pdf};
				\item \burl{http://www.math.unipd.it/~rcardin/sweb/Design\%20Pattern\%20Architetturali\%20-\%20Dependency\%20Injection_4x4.pdf}.
			\end{itemize} 
	\end{itemize}
	\item \textbf{Martin Fowler - UML\ped{G} Distilled} - 2nd edition;
	\item \textbf{Slide dell'insegnamento - Diagrammi delle classi}: \\
		\url{http://www.math.unipd.it/~tullio/IS-1/2015/Dispense/E03.pdf}
	\item \textbf{Slide dell'insegnamento - Diagrammi dei packages:} \\
		\url{http://www.math.unipd.it/~tullio/IS-1/2015/Dispense/E04.pdf}
	\item \textbf{Slide dell'insegnamento - Diagrammi di sequenza:} \\
		\url{http://www.math.unipd.it/~tullio/IS-1/2015/Dispense/E05.pdf}
	\item \textbf{Documentazione del \textit{Framework\ped{G}MEAN\ped{G}.js}:} \\
		\url{http://learn.mean.io/}
	\item \textbf{Documentazione della \textit{piattaforma} Node.js:} \\
		\url{https://nodejs.org/api/}
	\item \textbf{Giuda all'utilizzo dei middleware Express:} \\
		\url{http://expressjs.com/it/guide/using-middleware.html}
	\item \textbf{Guida all'utilizzo dei middleware Passport:} \\
		\url{http://passportjs.org/docs}
	\item \textbf{Manuale del database \textit{MongoDB\ped{G}}:} \\
		\url{https://docs.mongodb.org/manual/}
	\item \textbf{Documentazione dell'interfaccia REST:}
		\begin{itemize}
			\item \textit{Descrizione di REST:} \url{https://it.wikipedia.org/wiki/Representational_State_Transfer}
			\item \textit{Descrizione risorse REST:} \url{http://stashboard.readthedocs.org/en/latest/restapi.html}
		\end{itemize}
	\item \textbf{Documentazione del \textit{framework\ped{G} AngularJS\ped{G}}:} \\
		\begin{itemize}
			\item \textit{Documentazione generica:} \url{https://docs.angularjs.org/guide}
			\item \textit{Documentazione servizio \$http:} \burl{https://docs.angularjs.org/api/ng/service/$http}
			\item \textit{Documentazione servizio \$location:} \burl{https://docs.angularjs.org/api/ng/service/$location}
			\item \textit{Documentazione servizio \$windows:} \url{https://docs.angularjs.org/api/ng/service/$window}
			\item \textit{Documentazione servizio \$ruoteParams:} \burl{https://docs.angularjs.org/api/ngRoute/service/$routeParams}
			\item \textit{Documentazione servizio \$q:} \burl{https://docs.angularjs.org/api/ng/service/$q}
		\end{itemize}
	\item \textbf{Documentazione del \textit{framework\ped{G}} Material for Angular:} \\
		\url{https://material.angularjs.org/latest/}
	\item \textbf{Documentazione del \textit{framework\ped{G}} Chart.js} \\
		\url{http://www.chartjs.org/docs/}
	\item \textbf{Documentazione del \textit{wrapper\ped{G}} Angles.js} \\
		\url{https://github.com/gonewandering/angles}
	\item \textbf{Documentazione del \textit{framework\ped{G}} TextAngular.js} \\
		\url{https://github.com/fraywing/textAngular/wiki/textAngular-Docs-v1.1.x}
	\item \textbf{Guida all'utilizzo della direttiva \textit{ng-file-upload}} \\
		\url{https://github.com/danialfarid/ng-file-upload}
	\item \textbf{Documentazione di \textit{jison\ped{G}} per la definizione della grammatica di QML} \\
		\url{http://zaa.ch/jison/docs/}
\end{itemize} 
\input{sezioni/QuizziPedia.tex}
\newpage
\section{Requisiti di Sistema}
\subsection{Dispositivi supportati}
Il software \progetto{} è compatibile con i seguenti sistemi operativi desktop: Ubuntu 16.04 LTS, OS X El Capitan 10.11.4 e Windows 10 e con i seguenti sistemi operativi mobile: Android 6.0 Marshmallow, IOS 9 e Windows 10 Mobile.
\subsection{Browser supportati}
Il software \progetto{} supporta i browser: Google Chrome 50, Mozilla Firefox 46, Microsoft Edge 25 e Opera 37.
\newpage
\section{Registrazione}
\section{Login/Logout}
\newpage
\section{Gestione Profilo}
Il servizio permette ad un utente autenticato di aggiornare e/o modificare i propri dati personali. Cliccando la voce \textit{Gestione Profilo}, presente nella barra a sinistra, il sistema porta l'utente all'interno della pagina dedicata alla gestione del proprio profilo e account. All'interno di questa sezione è possibile:
\begin{itemize}
	\item Inserire un'immagine di profilo;
	\item Modificare il proprio nome;
	\item Modificare il proprio cognome;
	\item Modificare l'email;
	\item Inserire una nuova password.
\end{itemize}

\label{GestioneProfilo}
\begin{figure}[ht]
	\centering
	\includegraphics[scale=0.45]{img/gestione_profilo.png}
	\caption{Gestione profilo}
\end{figure}
\FloatBarrier


\newpage
\section{Svolgere un allenamento}
\subsection{Segnalare una domanda}
\input{sezioni/questionario.tex}
\newpage
\section{Creazione domande}
\subsection{Creazione di domande tramite wizard guidati}
\subsection{creazione di domande tramite il linguaggio QML}
\input{sezioni/FAQ.tex}
\appendix
%\newpage
\section{Organigramma}
\subsection{Redazione}
\begin{table}[htbp]
	\begin{center}
		\setlength{\extrarowheight}{\jot}
		\begin{tabular}{|c|c|p{6cm}|}
			\hline
			\textbf{Nominativo} & \textbf{Data di redazione} & \textbf{Firma} \\[1ex]
			\hline
			Matteo Granzotto & 2016-01-07 & \\[1ex]
			\hline
		\end{tabular}
	\end{center}
	\caption{Redazione}
\end{table}

\subsection{Approvazione}
\begin{table}[htbp]
	\begin{center}
		\setlength{\extrarowheight}{\jot}
		\begin{tabular}{|c|c|p{5cm}|}
			\hline
			\textbf{Nominativo}     & \textbf{Data di approvazione} & \textbf{Firma}  \\[1ex]
			\hline
			Matteo Granzotto		& 2016-01-07					&			\\[1ex]
			\hline
			Prof. Tullio Vardanega	&								&			\\[1ex]
			\hline
		\end{tabular}
	\end{center}
	\caption{Approvazione}
\end{table}

\subsection{Accettazione dei componenti}
\begin{table}[htbp]
	\begin{center}
		\setlength{\extrarowheight}{\jot}
		\begin{tabular}{|c|c|p{6cm}|}
			\hline
			\textbf{Nominativo} & \textbf{Data di accettazione} & \textbf{Firma} \\[1ex]
			\hline
			Matteo Granzotto	&	2016-01-07	&		\\[1ex]
			\hline
			Matteo Gnoato		&	2016-01-07	&		\\[1ex]
			\hline
			Alberto Ferrara		&	2016-01-07	&		\\[1ex]
			\hline
			Franco Berton		&	2016-01-07	&		\\[1ex]
			\hline
			Marco Prelaz		&	2016-01-07	&		\\[1ex]
			\hline
			Simone Magagna		&	2016-01-07	&		\\[1ex]
			\hline
			Mattia Varotto		&	2016-01-07	&		\\[1ex]
			\hline
		\end{tabular}
	\end{center}
	\caption{Accettazione}
\end{table}

\subsection{Componenti}
\begin{table}[H]
	\begin{center}
		\setlength{\extrarowheight}{\jot}
		\begin{tabular}{|c|c|p{5cm}|p{4.3cm}|}
			\hline
			\textbf{Nominativo} & \textbf{Matricola} & \raggedright \textbf{Indirizzo di posta elettronica} & \textbf{Ruoli} \\[1ex]
			\hline
	 		Matteo Granzotto	& 1051540	& \href{mailto:granzotto.matteo@gmail.com}{granzotto.matteo@gmail.com} 	& Responsabile, Analista 	\\[1ex]
			\hline
			Matteo Gnoato		& 1051873	& \href{mailto:gnoatomatteo@gmail.com}{gnoatomatteo@gmail.com} 			& Responsabile, Analista 	\\[1ex]
			\hline
			Alberto Ferrara		& 1049378	& \href{mailto:albertoferrara92@gmail.com}{albertoferrara92@gmail.com} 	& Amministratore, Analista 	\\[1ex]
			\hline
			Franco Berton 		& 1052574	& \href{mailto:franco.berton93@gmail.com}{franco.berton93@gmail.com} 	& Amministratore, Analista	\\[1ex]
			\hline
			Marco Prelaz		& 1047343	& \href{mailto:marcomak91@hotmail.it}{marcomak91@hotmail.it} 			& Verificatore, Analista	\\[1ex]
			\hline
			Simone Magagna		& 1009467	& \href{mailto:simone.magagna91@gmail.com}{simone.magagna91@gmail.com} 	& Verificatore, Analista 	\\[1ex]
			\hline
			Mattia Varotto		& 1051619	& \href{mailto:varots93@hotmail.it}{varots93@hotmail.it} & Verificatore, Analista \\[1ex]
			\hline	
		\end{tabular}
	\end{center}
	\caption{Componenti}
\end{table}



\end{document}