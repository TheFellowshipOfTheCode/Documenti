\section{P}
\begin{itemize} 
	\item
	\textbf{Package}: è un meccanismo per organizzare classi Java, logicamente correlate o che forniscono servizi simili, all’interno di sottogruppi ordinati. Questi package possono essere compressi permettendo la trasmissione di più classi in una sola volta. In UML, analogamente, è un raggruppamento arbitrario di elementi in una unità di livello più alto.
	\item
	\textbf{Parser}: il parsing, analisi sintattica o parsificazione è un processo che analizza un flusso continuo di dati in ingresso (input) (letti per esempio da un file o una tastiera) in modo da determinare la sua struttura grazie ad una data grammatica formale. Un parser è un programma che esegue questo compito.
	\textbf{Passport}: è un middleware per l'autenticazione degli utenti al sistema.
	\item
	\textbf{Path}: la sequenza delle cartelle a cui bisogna accedere per raggiungere la posizione di un determinato file.
	\item
	\textbf{PERT}: è un metodo statistico di determinazione dei tempi e dei costi delle attività di progetto. Rispetto alla semplice stima a valore singolo, il metodo presuppone la determinazione di valori di stima ottimale, probabile e pessimistico che risultano più adeguati a valutare tempi e costi di attività di progetto che presentano incertezza o complessità.
	\item
	\textbf{PDCA}: plan-do-check-act è un metodo di gestione in quattro fasi iterativo, utilizzato in attività per il controllo e il miglioramento continuo dei processi e dei prodotti. È noto anche come il Ciclo di Deming, o di Shewhart. Il ciclo PDCA è fondamentale nel processo di creazione di una Customer Relationship Management avanzata. 
	\item
	\textbf{PDF}: il Portable Document Format, comunemente abbreviato PDF, è un formato di file basato su un linguaggio di descrizione di pagina sviluppato da Adobe Systems nel 1993 per rappresentare documenti in modo indipendente dall’hardware e dal software utilizzati per generarli o per visualizzarli.
	\item
	\textbf{Play Framework}: play è un framework open source, scritto in Java e Scala, che implementa il pattern model-view-controller. Il suo scopo è quello di migliorare la produttività degli sviluppatori usando il paradigma Convention Over Configuration, il caricamento del codice a caldo e la visualizzazione degli errori nel browser.
	\item
	\textbf{PNG}: il Portable Network Graphics (abbreviato PNG) è un formato di file per memorizzare immagini.
	\item
	\textbf{PostgreSQL}: è un completo DBMS ad oggetti rilasciato con licenza libera (BSD).
	PostgreSQL è una reale alternativa sia rispetto ad altri prodotti liberi come MySQL, Firebird SQL e MaxDB che a quelli a codice chiuso come Oracle, Informix o DB2 ed offre caratteristiche uniche nel suo genere che lo pongono per alcuni aspetti all'avanguardia nel settore dei database.
	\item
	\textbf{Promise}: oggetti che rappresentano il risultato di una chiamata di funzione asincrona, più semplicemente rappresentano una promessa che un risultato verrà fornito non appena disponibile.
\end{itemize}

