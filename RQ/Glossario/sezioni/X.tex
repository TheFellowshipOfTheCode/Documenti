\section{X}
\begin{itemize} 
	\item
	\textbf{XHTML}: (eXtensible HyperText Markup Language, Linguaggio di marcatura di ipertesti estensibile) è un linguaggio di marcatura che associa alcune proprietà dell'XML con le caratteristiche dell'HTML: un file XHTML è un pagina HTML scritta in conformità con lo standard XML.
	\item
	\textbf{XML}: (eXtensible Markup Language) è un linguaggio di markup, ovvero un linguaggio marcatore basato su un meccanismo sintattico che consente di definire e controllare il significato degli elementi contenuti in un documento. Il nome indica che si tratta di un linguaggio marcatore estensibile in quanto permette di creare tag personalizzati. Rispetto all’HTML, l’XML ha uno scopo ben diverso: mentre il primo definisce una grammatica per la descrizione e la formattazione di pagine web e, in generale,di ipertesti, il secondo è un metalinguaggio utilizzato per creare nuovi linguaggi,atti a descrivere documenti strutturati. Mentre l’HTML ha un insieme ben definito e ristretto di tag, con l’XML è invece possibile definirne di propri a seconda delle esigenze. Viene spesso utilizzato anche nello scambio di dati tra software diversi.
\end{itemize}