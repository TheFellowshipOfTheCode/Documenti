\section{U}
\begin{itemize} 
	\item
	\textbf{UML}: unified modelling language, linguaggio di modellazione unificato, è un linguaggio di modellazione e specifica basato sul paradigma orientato agli oggetti, ed una famiglia di notazioni grafiche che si basano su un singolo meta-modello e servono a supportare la descrizione e il progetto dei sistemi software. Il linguaggio nacque con l'intento di unificare approcci precedenti, raccogliendo le migliori prassi nel settore e definendo così uno standard industriale unificato.
	\item
	\textbf{Unicode}:  è un sistema di codifica che assegna un numero univoco ad ogni carattere usato per la scrittura di testi, in maniera indipendente dalla lingua, dalla piattaforma informatica e dal programma utilizzato.
	È stato compilato e viene aggiornato e pubblicizzato dall'Unicode Consortium, un consorzio internazionale di aziende interessate alla interoperabilità nel trattamento informatico dei testi in lingue diverse.
\end{itemize}