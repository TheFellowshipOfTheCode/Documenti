\section{F}
\begin{itemize}
	\item
	\textbf{Facade}: letteralmente facade significa "facciata", ed infatti nella programmazione ad oggetti indica un oggetto che permette, attraverso un'interfaccia più semplice, l'accesso a sottosistemi che espongono interfacce complesse e molto diverse tra loro, nonché a blocchi di codice complessi. 
	\item
	\textbf{Fault-tollerance}: la tolleranza ai guasti è la capacità di un sistema di non subire avarie, cioè interruzioni di servizio, anche in presenza di guasti. La tolleranza ai guasti è uno degli aspetti che costituiscono l'affidabilità. È importante notare che la tolleranza ai guasti non garantisce l'immunità da tutti i guasti, ma solo che i guasti per cui è stata progettata una protezione non causino fallimenti.
	I controlli di protezione (che vengono effettuati a tempo di esecuzione), assieme a controlli analoghi effettuati staticamente (come a tempo di progettazione o di compilazione), sono una metodologia molto efficace per ottenere un'elevata robustezza (rapida rilevazione degli errori e loro confinamento) in un sistema.
	La tolleranza ai guasti può portare al peggioramento delle prestazioni, per cui nella progettazione di un sistema è necessario trovare adeguate ottimizzazioni e compromessi.
	\item
	\textbf{Flexible and Adaptive Text To Speech}: permette la creazione semplice e veloce di sintesi vocale basato su un input di testo. La tecnologia utilizzata permette la manipolazione di diversi parametri acustici e linguistici per ottenere la voce sintetica che è più adatta per una situazione specifica. 
	\item
	\textbf{Framework}: architettura logica di supporto per lo sviluppo. Alla base di un framework c’è sempre una serie di librerie di codice utilizzabili con uno o più linguaggi di programmazione.
\end{itemize}