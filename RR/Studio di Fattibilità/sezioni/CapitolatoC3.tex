\section{Capitolato C3}
\subsection{Descrizione}
Il capitolato proposto dall'azienda Zero12 riguarda la realizzazione di un algoritmo predittivo in grado di analizzare i dati provenienti da \textsf{oggetti} inseriti in diversi contesti. L'algoritmo, una volta analizzati i dati, provvederà a fornire delle previsioni su possibili guasti, interazioni con nuovi utenti ed identificare dei pattern di comportamento degli utenti per prevedere le azioni degli stessi su altri oggetti.

\subsection{Dominio Applicativo}
Il dominio applicativo del software in questione è l'interpretazione dei dati analizzati dall'algoritmo, ricevuti da macchine  (industriali o quali esse siano). Tutto questo allo scopo di poter prevenire eventuali guasti oppure azioni compiute dagli utenti.

\subsection{Dominio Tecnologico}
Per realizzare l’oggetto del capitolato vengono richieste al gruppo conoscenze legate all’ambito web:
\begin{itemize}
\item \textbf{Amazon Web Services} e \textbf{Database NoSQL} (MongoDB oppure OrientDB);
\item \textbf{Java o Scala} come linguaggio di programmazione;
\item \textbf{Play Framework} come piattaforma di sviluppo;
\item \textbf{HTML5, CSS3 e Javascript} per la parte di visualizzazione e comportamentale.
\end{itemize}

\subsection{Criticità}
Sono state individuate le seguenti criticità sulla scelta del capitolato:
\begin{itemize}
\item L'individuazione e la realizzazione di un'algoritmo predittivo che sia efficiente richiede un utilizzo profondo della matematica, il che può essere un punto critico che potrebbe creare delle difficoltà in fase di progettazione e realizzazione.
\item Non è molto chiaro come questo algoritmo debba essere realizzato, la descrizione del capitolato lascia molto spazio all'interpretazione di chi legge e non pone molti vincoli. Questo potrebbe rivelarsi un problema nella fase di studio e realizzazione del suddetto software.
\end{itemize}

\subsection{Valutazione Finale}
Il capitolato in se sarebbe interessante, visto che tratta argomenti che potenzialmente saranno alla base dell'internet del futuro, ma dopo un'attenta analisi si è preferito scartarlo. Questo perchè, da parte di tutti i componenti del team, c'era il timore di trovarsi ad affrontare un problema troppo grande o comunque non facilmente gestibile per garantire la consegna entro i termini prefissati.