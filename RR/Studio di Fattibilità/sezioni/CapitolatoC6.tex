\section{Capitolato C6}
\subsection{Descrizione}
L’obbiettivo del capitolato è rendere disponibile, sui dispositivi mobile, nuove funzionalità legate alla sintesi vocale da testo, normalmente non presenti su questo tipo di dispositivi, come la possibilità di  applicare effetti alle voci e/o poter utilizzare la voce degli utenti.

\subsection{Dominio Applicativo}
La sintesi vocale è una tecnologia che permette la conversione di un qualsiasi file di testo in un file sonoro. Negli ultimi anni si è assistito ad un rapido diffondersi di questa tecnologia in numerosi ambiti: dai navigatori satellitari agli annunci nei mezzi pubblici agli assistenti vocali nei più moderni sistemi operativi, centralini telefonici e lettori di messaggi.

\subsection{Dominio Tecnologico}
Il progetto prevede la realizzazione di una applicazione per dispositivi mobili che sfrutti appieno le potenzialità offerte dal motore di sintesi vocale opensource “Flexible and Adaptive Text To Speech”, che offre un’applicazione web che espone le proprie funzionalità mediante interfaccia HTTP. 

\subsection{Criticità}
Nell'analisi del capitolato, il team ha rilevato le seguenti criticità:
\begin{itemize}
\item Uso del mototre di sintesi FA-TTS completamente sconosciuto.
\item Nessuna competenza del team sulla sintesi vocale.
\item Utilizzo di framework sconosciuti.
\end{itemize}

\subsection{Valutazione Finale}
Per quanto interessante e innovativo possa essere la sintesi vocale, il team ritiene che le scarse competenze sull’argomento possano essere un vincolo sulla realizzazione di questo capitolato. Inoltre, il gruppo ritiene che lo studio del motore di sintesi FA-TTS non dia delle competenze tecniche spendibili in un futuro ambito lavorativo d’interesse.
La parte dello sviluppo su piattaforma mobile aveva attirato in parte l’attenzione del team, ma non riteniamo sia sufficiente per la scelta del capitolato.
