\section{H}
\begin{itemize} 
	\item
	\textbf{HTML}: l'HyperText Markup Language (HTML) (traduzione letterale: linguaggio a marcatori per ipertesti), è il linguaggio di markup solitamente usato per la formattazione e impaginazione di documenti ipertestuali disponibili nel World Wide Web sotto forma di pagine web.
	È un linguaggio di pubblico dominio, la cui sintassi è stabilita dal World Wide Web Consortium (W3C), e che è derivato da un altro linguaggio avente scopi più generici, l'SGML. 
	\item
	\textbf{HTML5}: è un linguaggio di markup per la strutturazione delle pagine web, derivato dallo standard che definisce HTML.
	Il termine rappresenta due concetti differenti:
	\begin{itemize}
		\item
		Una nuova versione del linguaggio HTML, con nuovi elementi, attributi e comportamenti;
		\item
		Un più ampio insieme di tecnologie che permettono siti web e applicazioni più diversificate e potenti.
	\end{itemize}
	HTML5 si presenta come un linguaggio pronto ad essere plasmato secondo le più recenti necessità, sia dal lato della strutturazione del contenuto che da quello dello sviluppo di vere e proprie applicazioni.
	\item
	\textbf{HTTP}: l'HyperText Transfer Protocol (protocollo di trasferimento di un ipertesto) è usato come principale sistema per la trasmissione di informazioni sul web ovvero in un'architettura tipica client-server. Le specifiche del protocollo sono gestite dal W3C. Un server HTTP generalmente resta in ascolto delle richieste dei client sulla porta 80 usando il protocollo TCP a livello di trasporto.
\end{itemize}
