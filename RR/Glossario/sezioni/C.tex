\section{C}
\begin{itemize}
	\item
	\textbf{Complessità ciclomatica}: La Complessità Ciclomatica (o complessità condizionale) è una metrica software. Viene utilizzata per misurare la complessità di un programma. Misura direttamente il numero di cammini linearmente indipendenti attraverso il grafo di controllo di flusso. La complessità ciclomatica è calcolata utilizzando il grafo di controllo di flusso del programma: i nodi del grafo corrispondono a gruppi indivisibili di istruzioni, mentre gli archi connettono due nodi se il secondo gruppo di istruzioni può essere eseguito immediatamente dopo il primo gruppo. La complessità ciclomatica può inoltre essere applicata a singole funzioni, moduli, metodi o classi di un programma.
	\item
	\textbf{Complexity-report}: software di analisi della complessità dei progetti JavaScript. 
	\item
	\textbf{CSS}: è un linguaggio usato per definire la formattazione di documenti HTML, XHTML e XML ad esempio i siti web e relative pagine web. Le regole per comporre il CSS sono contenute in un insieme di direttive (Recommendations) emanate a partire dal 1996 dal W3C.
	L'introduzione del CSS si è resa necessaria per separare i contenuti delle pagine HTML dalla loro formattazione e permettere una programmazione più chiara e facile da utilizzare, sia per gli autori delle pagine stesse sia per gli utenti, garantendo contemporaneamente anche il riutilizzo di codice ed una sua più facile manutenzione. 
	\item
	\textbf{CSS3}: è un linguaggio utilizzato per definire la formattazione di documenti HTML\ped{G}, XHTML\ped{G} e XML\ped{G}.
	Questo linguaggio istruisce un browser\ped{G} su come il documento debba essere presentato all'utente, per esempio definendone la formattazione del testo, il posizionamento degli elementi rispetto a diversi media e device eccetera.
	\item
	\textbf{CSSHint}: è uno strumento che aiuta a rilevare possibili errori nel codice CSS\ped{G}.
\end{itemize}