\section{C}
\begin{itemize} 
	\item
	\textbf{CSS}: è un linguaggio usato per definire la formattazione di documenti HTML, XHTML e XML ad esempio i siti web e relative pagine web. Le regole per comporre il CSS sono contenute in un insieme di direttive (Recommendations) emanate a partire dal 1996 dal W3C.
	L'introduzione del CSS si è resa necessaria per separare i contenuti delle pagine HTML dalla loro formattazione e permettere una programmazione più chiara e facile da utilizzare, sia per gli autori delle pagine stesse sia per gli utenti, garantendo contemporaneamente anche il riutilizzo di codice ed una sua più facile manutenzione. 
	\item
	\textbf{CSS3}: è un linguaggio utilizzato per definire la formattazione di documenti HTML\ped{G}, XHTML\ped{G} e XML\ped{G}.
	Questo linguaggio istruisce un browser\ped{G} su come il documento debba essere presentato all'utente, per esempio definendone la formattazione del testo, il posizionamento degli elementi rispetto a diversi media e device eccetera.
\end{itemize}