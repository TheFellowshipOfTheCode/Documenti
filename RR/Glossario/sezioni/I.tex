\section{I}
\begin{itemize} 
	\item
	\textbf{IDE}: l’ambiente di sviluppo integrato (in inglese Integrated Development Environment, ovvero IDE) è un software che, in fase di programmazione, aiuta i programmatori nello sviluppo del codice sorgente di un programma. Spesso l’IDE aiuta lo sviluppatore segnalando errori di sintassi del codice direttamente in fase di scrittura, oltre a tutta una serie di strumenti e funzionalità di supporto alla fase di sviluppo e debugging.
	\item
	\textbf{Incapsulamento}:si definisce incapsulamento (o encapsulation) la tecnica di nascondere il funzionamento interno – deciso in fase di progetto – di una parte di un programma, in modo da proteggere le altre parti del programma dai cambiamenti che si produrrebbero in esse nel caso che questo funzionamento fosse difettoso, oppure si decidesse di implementarlo in modo diverso. Per avere una protezione completa è necessario disporre di una robusta interfaccia che protegga il resto del programma dalla modifica delle funzionalità soggette a più frequenti cambiamenti.
	\item
	\textbf{Inspection}: tecnica di analisi statica che consiste in una lettura dettagliata e mirata dei documenti o del codice, utilizzando come supporto fondamentale la lista di controllo contenente gli errori più frequenti.
	\item
	\textbf{Instant Messaging}:è una categoria di sistemi di comunicazione in tempo reale in rete, tipicamente Internet o una rete locale, che permette ai suoi utilizzatori lo scambio di brevi messaggi.
	\item
	\textbf{ISO}: è un sistema per localizzare oggetti o persone all'interno di un edificio tramite onde radio, campi magnetici, segnali acustici o altre informazioni sensoriali raccolte da dispositivi mobili.
	\item
	\textbf{ISOG 8601:2004}: l'ISO 8601 (Data elements and interchange formats - Information interchange - Representation of dates and times) è uno standard internazionale per la rappresentazione di date ed orari.
	\item
	\textbf{ISO/IEC 15504}: anche chiamato Software Process Improvement and Capability determinazione (SPICE), è un insieme di documenti tecnici standard per il processo di sviluppo software e le funzioni di gestione di business correlati. E' uno degli standard congiunti dell'Organizzazione Internazionale per la Standardizzazione (ISO) e della Commissione Elettrotecnica Internazionale (IEC).
	\item
	\textbf{ISO/IEC 9126}: Con la sigla ISO/IEC 9126 si individua una serie di normative e linee guida, sviluppate dall’ ISO (Organizzazione Internazionale per la Normazione) in collaborazione con l'IEC (Commissione Elettrotecnica Internazionale), preposte a descrivere un modello di qualità del software. Lo standard è stato sostituito dalla ISO/IEC 25010:2011. Il modello propone un approccio alla qualità in modo tale che le società di software possano migliorare l'organizzazione e i processi e, quindi come conseguenza concreta, la qualità del prodotto sviluppato.
\end{itemize}