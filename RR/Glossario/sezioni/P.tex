\section{P}
\begin{itemize} 
	\item
	\textbf{Package}: É un meccanismo per per organizzare classi Java\ped{G}, logicamente correlate o che forniscono servizi simili, all’interno di sottogruppi ordinati. Questi package possono essere compressi permettendo la trasmissione di più classi in una sola volta.
	In UML\ped{G} analogamente è un raggruppamento arbitrario di elementi in una unità di livello più alto.
	\item
	\textbf{PERT}: É un metodo statistico di determinazione dei tempi delle attività di progetto (ma può essere applicato anche ai costi). Rispetto alla semplice stima a valore singolo, il metodo presuppone la determinazione di valori di stima ottimale, probabile e pessimistico che risultano più adeguati a valutare tempi e costi di attività di progetto che presentano incertezza o complessità.
	\item
	\textbf{PDCA}: Plan-Do-Check-Act è un metodo di gestione in quattro fasi iterativo, utilizzato in attività per il controllo e il miglioramento continuo dei processi e dei prodotti. È noto anche come il Ciclo di Deming, o di Shewhart, (Plan-Do-Study-Act). Un'altra versione di questo ciclo viene chiamata, OPDCA. La "O", ha il significato sia di "osservazione", che di "Afferrare la condizione attuale." Il ciclo PDCA è fondamentale nel processo di creazione di una Customer relationship management avanzata. Può anche essere sostenuto utilizzando strumenti cognitivi, come le mappe mentali, e coadiuvando gli analisti con flussi semplificati per l'implementazione di micro cicli in fase di brainstorming.
	\item
	\textbf{PDF}: Il Portable Document Format, comunemente abbreviato PDF, è un formato di file basato su un linguaggio di descrizione di pagina sviluppato da Adobe Systems nel 1993 per rappresentare documenti in modo indipendente dall’hardware e dal software utilizzati per generarli o per visualizzarli.
	\item
	\textbf{Play Framework}: Play è un framework open source, scritto in Java e Scala, che implementa il pattern model-view-controller. Il suo scopo è quello di migliorare la produttività degli sviluppatori usando il paradigma Convention Over Configuration, il caricamento del codice a caldo e la visualizzazione degli errori nel browser.
	\item
	\textbf{PNG}: Il Portable Network Graphics (abbreviato PNG) è un formato di file per memorizzare immagini.
	\item
	\textbf{PostgreSQL}: É un completo DBMS ad oggetti rilasciato con licenza libera (stile Licenza BSD). Spesso viene abbreviato come "Postgres", sebbene questo sia un nome vecchio dello stesso progetto.
	PostgreSQL è una reale alternativa sia rispetto ad altri prodotti liberi come MySQL, Firebird SQL e MaxDB che a quelli a codice chiuso come Oracle, Informix o DB2 ed offre caratteristiche uniche nel suo genere che lo pongono per alcuni aspetti all'avanguardia nel settore dei database.
\end{itemize}

