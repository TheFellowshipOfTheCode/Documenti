\section{P}
\begin{itemize} 
	\item
	\textbf{Package}: è un meccanismo per per organizzare classi Java\ped{G}, logicamente correlate o che forniscono servizi simili, all’interno di sottogruppi ordinati. Questi package possono essere compressi permettendo la trasmissione di più classi in una sola volta.
	In UML\ped{G} analogamente è un raggruppamento arbitrario di elementi in una unità di livello più alto.
	\item
	\textbf{PDF}: il Portable Document Format, comunemente abbreviato PDF, è un formato di file basato su un linguaggio di descrizione di pagina sviluppato da Adobe Systems nel 1993 per rappresentare documenti in modo indipendente dall’hardware e dal software utilizzati per generarli o per visualizzarli.
	\item
	\textbf{PNG}: il Portable Network Graphics (abbreviato PNG) è un formato di file per memorizzare immagini.
\end{itemize}

