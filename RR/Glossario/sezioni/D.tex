\section{D}
\begin{itemize} 
	\item
	\textbf{Debugger}: è un programma/software specificatamente progettato per l'analisi e l'eliminazione dei bug (debugging), ovvero errori di programmazione interni al codice di altri programmi. Assieme al compilatore è fra i più importanti strumenti di sviluppo a disposizione di un programmatore, spesso compreso all'interno di un ambiente integrato di sviluppo (IDE), in quanto in grado di aiutare il programmatore ad individuare errori di semantica all'interno del codice sorgente del programma, altrimenti di difficile individuazione in fase di runtime.
	\item
	\textbf{Device}:può indicare degli elementi concernenti un sistema operativo:
	 \begin{itemize} 
	\item
    device, astrazione di un dispositivo hardware al quale il sistema operativo fornisce accesso mediante un driver; nei sistemi Unix storicamente si sono distinti i block device (i dispositivi che accedono ai dati in gruppi di byte e non necessariamente in sequenza) dai character device (ai quali l'astrazione in oggetto fornisce primitive per accedere ai byte strettamente in sequenza);
 	\item  
    device file o file speciale, un'interfaccia ad un driver che appare nel file system come se fosse un file ordinario; anch'essi sono divisi storicamente in block e character.
	\end{itemize}
	\item
	\textbf{Driver}: l'insieme di procedure, spesso scritte in assembly, che permette ad un sistema operativo di dialogare con un dispositivo hardware attraverso un'interfaccia che astrae dall'implementazione fisica e che ne considera soltanto il funzionamento logico. 
\end{itemize}