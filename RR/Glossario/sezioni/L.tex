\section{L}
\begin{itemize} 
	\item
	\textbf{\LaTeX}: É un linguaggio di markup utilizzato per la produzione di documentazione tecnica e scientifica. \LaTeX è lo standard de facto per la comunicazione e la pubblicazione di documenti scientifici. \LaTeX è disponibile come software libero.
	\item
	\textbf{Lato client}: Nell'ambito delle reti di calcolatori, il termine lato client (client-side in inglese) indica le operazioni di elaborazione effettuate da un client in un'architettura client-server.
	\item
	\textbf{Lato server}: Nell'ambito delle reti di calcolatori, il termine lato server (server-side in inglese) indica le operazioni di elaborazione effettuate dal server in un'architettura client-server.
	\item
	\textbf{Linguaggio di markup}: É un insieme di regole che descrivono i meccanismi di rappresentazione (strutturali, semantici o presentazionali) di un testo che, utilizzando convenzioni standardizzate, sono utilizzabili su più supporti. La tecnica di formattazione per mezzo di marcatori (o espressioni codificate) richiede quindi una serie di convenzioni, ovvero appunto di un linguaggio a marcatori di documenti.
	\item
	\textbf{Linguaggio di scripting}: Un linguaggio di scripting, in informatica, è un linguaggio di programmazione interpretato destinato in genere a compiti di automazione del sistema operativo (batch) o delle applicazioni (macro), o a essere usato all'interno delle pagine web.
	I programmi sviluppati con questi linguaggi sono detti script, termine della lingua inglese utilizzato in ambito teatrale per indicare il testo (anche detto canovaccio) in cui sono tracciate le parti che devono essere interpretate dagli attori.
	\item
	\textbf{Linux}: É una famiglia di sistemi operativi di tipo Unix-like, rilasciati sotto varie possibili distribuzioni, aventi la caratteristica comune di utilizzare come nucleo il kernel Linux.
	\item
	\textbf{Logger}: Framework utilizzato durante l'esecuzione di un programma per registrare e riportare informazioni sul sistema, messaggi di errore e tracciamento dell'output.
\end{itemize}