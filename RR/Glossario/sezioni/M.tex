\section{M}
\begin{itemize} 
	\item
	\textbf{Mac OS}: è il sistema operativo di Apple dedicato ai computer Macintosh; il nome è l'acronimo di Macintosh Operating System.
	\item
	\textbf{Media}: è un dispositivo di memorizzazione su cui si registrano informazioni (dati). Il termine è utilizzato soprattutto in riferimento a file, audio e video, ma teoricamente la registrazione può avvenire con qualunque grandezza fisica, anche un foglio di carta per scrivere.
	\item
	\textbf{Microsoft Windows}: Microsoft Windows (abbreviazioni comunemente utilizzate: "Windows" o "Win") è una famiglia di ambienti operativi e sistemi operativi dedicati ai personal computer, alle workstation, ai server e agli smartphone. Il sistema operativo si chiama così per via della sua interfaccia di programmazione di un'applicazione a finestre (che si chiamano "windows" in lingua inglese).
	In particolare Microsoft Windows nasce come ambiente operativo per i sistemi operativi MS-DOS e PC DOS (dedicati ai primi home computer), e diventa sistema operativo con Windows NT (dedicato alle workstation e ai server) e Windows 95 (dedicato ai personal computer). È software proprietario della Microsoft Corporation che lo rende disponibile esclusivamente a pagamento.
	\item
	\textbf{Modello ad attori}: Il modello di attore in informatica è un modello matematico di calcolo concorrente che tratta "attori" come i primitivi universali di computazione concorrente: in risposta ad un messaggio che riceve, un attore può prendere decisioni locali, creare più attori, inviare più messaggi, e capire come rispondere al messaggio successivo ricevuto. 
	\item
	\textbf{MongoDB}: è un DBMS non relazionale, orientato ai documenti. Classificato come un database di tipo NoSQL, MongoDB si allontana dalla struttura tradizionale basata su tabelle dei database relazionali in favore di documenti in stile JSON con schema dinamico (MongoDB chiama il formato BSON), rendendo l'integrazione di dati di alcuni tipi di applicazioni più facile e veloce.  
\end{itemize}
