\section{B}
\begin{itemize} 
	\item
	\textbf{Backup}:si indica la replicazione, su un qualunque supporto di memorizzazione, di materiale informativo archiviato nella memoria di massa dei computer, siano essi stazione di lavoro o server, al fine di prevenire la perdita definitiva dei dati in caso di eventi malevoli accidentali o intenzionali. Si tratta dunque di una misura di ridondanza fisica dei dati, tipica delle procedure di disaster recovery.
	\item
	\textbf{Baseline}: è una descrizione delle caratteristiche di un prodotto, in un punto nel tempo, che serve come base per definire il cambiamento. Un "cambiamento" è un movimento da questo stato base ad uno stato successivo. L'identificazione dei cambiamenti rilevanti rispetto alla condizione di base è lo scopo centrale della baseline.
	\item
	\textbf{Beacon}: I Beacon sono piccoli dispositivi che, attraverso la tecnologia Bluetooth Low Energy, sono in grado di trasmettere informazioni a smartphone e tablet con un raggio di azione regolabile dai 10cm ai 70m. Attraverso i Beacons, è possibile veicolare informazioni ed una vasta gamma di contenuti (foto, video, documenti, questionari, sondaggi etc.) agli utenti che abbiano scaricato un'App.
	\item
	\textbf{BLE}: Bluetooth a basso consumo energetico, è una tecnologia di rete personale wireless progettato e commercializzato dal Bluetooth Special Interest Group finalizzato a nuove applicazioni nel settore sanitario, fitness, beacons, di sicurezza e le industrie di intrattenimento domestico. Rispetto al classico Bluetooth, Bluetooth smart è destinato a fornire il consumo di energia notevolmente ridotto e costi pur mantenendo un range di comunicazione simile.
	\item
	\textbf{Browser}: il web browser, o più semplicemente browser è un'applicazione per il recupero, la presentazione e la navigazione di risorse web. Tali risorse, come pagine web, immagini o video, sono messe a disposizione sulla World Wide Web, la rete globale che si appoggia su Internet, o su una rete locale, o ancora sullo stesso computer dove il browser è in esecuzione. Il programma implementa da un lato le funzionalità di client per il protocollo HTTP, che regola lo scaricamento delle risorse dai server web a partire dal loro indirizzo URL; dall'altro quelle di visualizzazione dei contenuti ipertestuali, solitamente all'interno di documenti HTML, e di riproduzione multimediali.
\end{itemize}