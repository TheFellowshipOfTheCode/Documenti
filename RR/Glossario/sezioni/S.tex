\section{S}
\begin{itemize} 
	\item
	\textbf{Sass}: Sass è un'estensione del linguaggio CSS che permette di utilizzare variabili, di creare funzioni e di organizzare i fogli di stile in più file. Il linguaggio Sass si basa sul concetto di preprocessore CSS, il quale serve a definire fogli di stile con una forma più semplice, completa e potente rispetto ai CSS e a generare file CSS ottimizzati, aggregando le strutture definite anche in modo complesso.
	\item
	\textbf{SPICE}: è un insieme di documenti tecnici standard per il processo di sviluppo software e le funzioni di gestione di business correlati. E' uno degli standard congiunti dell'Organizzazione Internazionale per la Standardizzazione (ISO) e della Commissione Elettrotecnica Internazionale (IEC).
	\item
	\textbf{SyncTex}: SyncTeX è un nuovo metodo di TeXLive 2008 per permettere la sincronizzazione tra i file di TeX sorgente e il file DVI o PDF risultante. SyncTeX ha il vantaggio di essere molto più compatibile con i pacchetti (altri file di stile) essendo incorporato nel motore TeX e inoltre non è soggetto a modifiche del layout del documento.
	
\end{itemize}