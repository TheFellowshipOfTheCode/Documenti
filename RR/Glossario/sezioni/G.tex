\section{G}
\begin{itemize}
	\item
	\textbf{Git}: è un sistema software di controllo di versione distribuito, creato da Linus Torvalds nel 2005.
	La progettazione di Git è stata ispirata da BitKeeper e da Monotone, pensato 	inizialmente solamente come motore a basso livello che altri potevano usare per scrivere un front-end. In seguito diventato un sistema di controllo versione, direttamente utilizzabile da riga di comando; vari progetti software adesso usano Git per tale attività, principalmente il kernel Linux.
	\item
	\textbf{GitHub}: è un servizio web di hosting per lo sviluppo di progetti software, che usa il sistema di controllo di versione Git. Può essere utilizzato anche per la condivisione e la modifica di file di testo e documenti revisionabili. 
	\item
	\textbf{Google Chrome}: google Chrome è un browser sviluppato da Google, basato sul motore di rendering Blink.
	\item
	\textbf{Google Chrome DevTools}: gli strumenti di Chrome Developer (DevTools in breve), sono un insieme di web authoring e strumenti di debug integrato in Google Chrome. I DevTools offrono agli sviluppatori web di accesso in profondità nelle parti interne del browser e la loro applicazione web. Utilizzare i DevTools di rintracciare in modo efficace verso il basso problemi di layout, impostare punti di interruzione JavaScript, e ottenere spunti per l'ottimizzazione del codice.
	\item
	\textbf{Google Drive}: è un servizio, in ambiente cloud computing, di memorizzazione e sincronizzazione online introdotto da Google il 24 aprile 2012. Il servizio comprende il file hosting, il file sharing e la modifica collaborativa di documenti inizialmente fino a 5 GB, da ottobre 2013 fino a 15 GB gratuiti (inclusivi dello spazio di memorizzazione di Gmail e delle foto di Google+) estendibili fino a 30 TB in totale. Il servizio può essere usato via Web, caricando e visualizzando i file tramite il web browser, oppure tramite l'applicazione installata su computer che sincronizza automaticamente una cartella locale del file system con quella condivisa. Su Google Drive sono presenti anche i documenti creati con Google Documenti.
\end{itemize}
