\section{R}
\begin{itemize} 
\item
\textbf{Runtime}: indica il momento in cui un programma per computer viene eseguito, in contrapposizione ad altre fasi del ciclo di vita del software.
Tradizionalmente, questa contrapposizione era soprattutto intesa rispetto al tempo di compilazione o compile-time, relativa alla stesura e traduzione del programma sorgente; in questa accezione, spesso si usano anche gli aggettivi dinamico e statico per riferirsi rispettivamente al run-time e al compile-time. Nello scenario della programmazione moderna, che comprende una catena di produzione del software più varia e articolata, run-time può essere contrapposto anche ad altri stadi della vita di un programma, come il deployment time, il linking time, il loading time.
\end{itemize}