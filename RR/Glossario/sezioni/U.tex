\section{U}
\begin{itemize} 
	\item
	\textbf{UML}: unified Modelling Language, linguaggio di modellazione unificato, è un linguaggio di modellazione e specifica basato sul paradigma orientato agli oggetti e una famiglia di notazioni grafiche che si basano su un singolo meta-modello e servono a supportare la descrizione e il progetto dei sistemi software.  Il linguaggio nacque con l'intento di unificare approcci precedenti (dovuti ai tre padri di UML e altri), raccogliendo le migliori prassi nel settore e definendo così uno standard industriale unificato.
	L'UML svolge un'importantissima funzione di "lingua franca" nella comunità della progettazione e programmazione a oggetti. Gran parte della letteratura di settore usa UML per descrivere soluzioni analitiche e progettuali in modo sintetico e comprensibile a un vasto pubblico.
	\item
	\textbf{Unicode}:  è un sistema di codifica che assegna un numero univoco ad ogni carattere usato per la scrittura di testi, in maniera indipendente dalla lingua, dalla piattaforma informatica e dal programma utilizzato.
	È stato compilato e viene aggiornato e pubblicizzato dall'Unicode Consortium[1], un consorzio internazionale di aziende interessate alla interoperabilità nel trattamento informatico dei testi in lingue diverse.
\end{itemize}