\section{N}
\begin{itemize}
	\item
	\textbf{Networking}:  l’insieme delle problematiche, e/o regole, cui due o più entità sono soggette quando richiedano di scambiare e condividere informazioni e risorse.  
	\item
	\textbf{Node.js}:  è un framework\ped{G} relativo all'utilizzo lato server\ped{G} di Javascript\ped{G}.
	 La caratteristica principale di Node.js\ped{G} risiede nella possibilità di accedere alle risorse del sistema operativo in modalità event-driven\ped{G} e non sfruttando il classico modello basato su processi o threads\ped{G} concorrenti, utilizzato dai classici web server. Il modello event-driven\ped{G}, o "programmazione ad eventi", si basa su un concetto piuttosto semplice: si lancia una azione quando accade qualcosa. Ogni azione quindi risulta asincrona a differenza dei pattern di programmazione più comune in cui una azione succede ad un'altra solo dopo che essa è stata completata.
	 \item
	 \textbf{NoSQL}: è un movimento che promuove sistemi software dove la persistenza dei dati è caratterizzata dal fatto di non utilizzare il modello relazionale, di solito usato dai database tradizionali (RDBMS). L'espressione NoSQL fa riferimento al linguaggio SQL, che è il più comune linguaggio di interrogazione dei dati nei database relazionali, qui preso a simbolo dell'intero paradigma relazionale.
	 Questi archivi di dati il più delle volte non richiedono uno schema fisso (schemaless), evitano spesso le operazioni di unione (join) e puntano a scalare in modo orizzontale. Gli accademici e gli articoli si riferiscono a queste basi di dati come memorizzazione strutturata (structured storage).
\end{itemize}