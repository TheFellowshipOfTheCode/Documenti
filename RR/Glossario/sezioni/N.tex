\section{N}
\begin{itemize}
	\item
	\textbf{Networking}:  l’insieme delle problematiche, e/o regole, cui due o più entità sono soggette quando richiedano di scambiare e condividere informazioni e risorse.  
	\item
	\textbf{Node.js}:  è un framework\ped{G} relativo all'utilizzo lato server\ped{G} di Javascript\ped{G}.
	 La caratteristica principale di Node.js\ped{G} risiede nella possibilità di accedere alle risorse del sistema operativo in modalità event-driven\ped{G} e non sfruttando il classico modello basato su processi o threads\ped{G} concorrenti, utilizzato dai classici web server. Il modello event-driven\ped{G}, o "programmazione ad eventi", si basa su un concetto piuttosto semplice: si lancia una azione quando accade qualcosa. Ogni azione quindi risulta asincrona a differenza dei pattern di programmazione più comune in cui una azione succede ad un'altra solo dopo che essa è stata completata.
\end{itemize}