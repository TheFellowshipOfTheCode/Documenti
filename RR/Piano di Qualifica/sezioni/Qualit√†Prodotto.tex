\section{Qualità di Prodotto}
Nei punti seguenti vengono elencate e descritte le diverse caratteristiche che il team ha individuato dallo standard ISO/IEC 9126 e che si impegna a perseguire, per garantire al prodotto finale una buona qualità. Per ognuna di queste sono stati identificati sia gli obiettivi primari sia le metriche di riferimento che ci permettano di definire se questi obiettivi potranno considerarsi soddisfatti o meno. 	

\subsection{Funzionalità}
La funzionalità rappresenta la capacità del software prodotto di fornire tutte le funzioni richieste dal Proponente, che sono state individuate attraverso l'\textit{Analisi dei Requisiti}.

\subsubsection{Obiettivi di Qualità}
Il team si impegna a rispettare le seguenti caratteristiche:
\begin{itemize}
\item \textbf{Adeguatezza}: le funzionalità fornite devono essere conformi rispetto le aspettative;
\item \textbf{Accuratezza}: il prodotto deve fornire i risultati attesi, con il livello di precisione richiesto;
\item \textbf{Sicurezza}: il prodotto deve proteggere le informazioni e i dati da accessi e modifiche non autorizzati.
\end{itemize}

\subsubsection{Metriche}

\paragraph{•}
\begin{itemize}
\item \textbf{Misurazione}:;
\item \textbf{Valori ottimali}:;
\item \textbf{Valori accettati}:.
\end{itemize}

\paragraph{•}
\begin{itemize}
\item \textbf{Misurazione}:;
\item \textbf{Valori ottimali}:;
\item \textbf{Valori accettati}:.
\end{itemize}

\paragraph{•}
\begin{itemize}
\item \textbf{Misurazione}:;
\item \textbf{Valori ottimali}:;
\item \textbf{Valori accettati}:.
\end{itemize}

\subsection{Affidabilità}
L’affidabilità rappresenta la capacità del prodotto software di svolgere correttamente le sue funzioni durante il suo utilizzo, anche nel caso in cui si presentino situazioni anomale.

\subsubsection{Obiettivi di Qualità}
Il software deve presentare le seguenti caratteristiche in fase di esecuzione:	
\begin{itemize}
\item \textbf{Maturità}: il prodotto deve evitare che si verifichino malfunzionamenti o che vengano prodotti risultati non corretti;
\item \textbf{Tolleranza agli errori}: nel caso in cui si presentino degli errori, dovuti a guasti o ad un uso scorretto dell'applicativo, questi devono essere gestiti in modo da mantenere alto il livello di prestazione.
\end{itemize}

\subsubsection{Metriche}

\paragraph{•}
\begin{itemize}
\item \textbf{Misurazione}:;
\item \textbf{Valori ottimali}:;
\item \textbf{Valori accettati}:.
\end{itemize}

\paragraph{•}
\begin{itemize}
\item \textbf{Misurazione}:;
\item \textbf{Valori ottimali}:;
\item \textbf{Valori accettati}:.
\end{itemize}

\subsection{Usabilità}
L’usabilità rappresenta la capacità del prodotto di essere facilmente comprensibile e attraente in ogni sua parte per qualsiasi utente che lo andrà ad utilizzare.

\subsubsection{Obiettivi di Qualità}
Il prodotto punta ad avere le seguenti caratteristiche:
\begin{itemize}
\item \textbf{Comprensibilità}: l'utente deve essere in grado di riconoscerne le funzionalità offerte dal software e deve comprenderne le modalità di utilizzo per riuscire a raggiungere i risultati attesi;
\item \textbf{Apprendibilità}: deve essere data la possibilità all'utente di imparare ad utilizzare l'applicazione senza troppo impegno;
\item \textbf{Operabilità}: le funzionalità presenti devono essere coerenti con le aspettative dell'utente;
\item \textbf{Attrattiva}: il software deve essere piacevole per chi ne fa uso.
\end{itemize}

\subsubsection{Metriche}

\paragraph{•}
\begin{itemize}
\item \textbf{Misurazione}:;
\item \textbf{Valori ottimali}:;
\item \textbf{Valori accettati}:.
\end{itemize}

\paragraph{•}
\begin{itemize}
\item \textbf{Misurazione}:;
\item \textbf{Valori ottimali}:;
\item \textbf{Valori accettati}:.
\end{itemize}

\paragraph{•}
\begin{itemize}
\item \textbf{Misurazione}:;
\item \textbf{Valori ottimali}:;
\item \textbf{Valori accettati}:.
\end{itemize}

\paragraph{•}
\begin{itemize}
\item \textbf{Misurazione}:;
\item \textbf{Valori ottimali}:;
\item \textbf{Valori accettati}:.
\end{itemize}

\subsection{Efficienza}
L’efficienza rappresenta la capacità di eseguire le funzionalità offerte dal software nel minor tempo possibile utilizzando al tempo stesso il minor numero di risorse possibili.

\subsubsection{Obiettivi di Qualità}
Il prodotto deve essere efficiente nel:
\begin{itemize}
\item \textbf{Comportamento rispetto al tempo}: per svolgere le sue funzioni il software deve fornire adeguati tempi di risposta ed elaborazione;
\item \textbf{Utilizzo delle risorse}: il software quando esegue le sue funzionalità deve utilizzare un appropriato numero e tipo di risorse.
\end{itemize}

\subsubsection{Metriche}

\paragraph{•}
\begin{itemize}
\item \textbf{Misurazione}:;
\item \textbf{Valori ottimali}:;
\item \textbf{Valori accettati}:.
\end{itemize}

\paragraph{•}
\begin{itemize}
\item \textbf{Misurazione}:;
\item \textbf{Valori ottimali}:;
\item \textbf{Valori accettati}:.
\end{itemize}

\subsection{Manutenibilità}
La manutenibilità rappresenta la capacità del prodotto di essere modificato, tramite correzioni, miglioramenti o adattamenti del software a cambiamenti negli ambienti, nei requisiti e nelle specifiche funzionali.

\subsubsection{Obiettivi di Qualità}
Per consentire una manutenzione il più agevole possibile bisogna rispettare i seguenti obiettivi:
\begin{itemize}
\item \textbf{Analizzabilità}: il software deve consentire una rapida identificazione delle possibili cause di errori e malfunzionamenti;
\item \textbf{Modificabilità}: il prodotto originale deve permettere eventuali cambiamenti in alcune sue parti;
\item \textbf{Stabilità}: non devono insorgere effetti indesiderati in seguito a modifiche effettuate sul software;
\item \textbf{Testabilità}: il software deve poter essere facilmente testato per validare le modifiche effettuate.
\end{itemize}

\subsubsection{Metriche}

\paragraph{•}
\begin{itemize}
\item \textbf{Misurazione}:;
\item \textbf{Valori ottimali}:;
\item \textbf{Valori accettati}:.
\end{itemize}

\paragraph{•}
\begin{itemize}
\item \textbf{Misurazione}:;
\item \textbf{Valori ottimali}:;
\item \textbf{Valori accettati}:.
\end{itemize}

\paragraph{•}
\begin{itemize}
\item \textbf{Misurazione}:;
\item \textbf{Valori ottimali}:;
\item \textbf{Valori accettati}:.
\end{itemize}

\paragraph{•}
\begin{itemize}
\item \textbf{Misurazione}:;
\item \textbf{Valori ottimali}:;
\item \textbf{Valori accettati}:.
\end{itemize}

\subsection{Portabilità}
La portabilità rappresenta la capacità del software di poter essere utilizzato su diversi ambienti.

\subsubsection{Obiettivi di Qualità}
Il software deve avere le seguenti proprietà:
\begin{itemize}
\item \textbf{Adattabilità}:il prodotto deve adattarsi a tutti quegli ambienti di lavoro nei quali è stato previsto un suo utilizzo, senza dover apportare modifiche allo stesso;
\item \textbf{Sostituibilità}: l'applicativo deve poter sostituire un altro software che ha lo stesso scopo e lavora nel medesimo ambiente.
\end{itemize}	

\subsubsection{Metriche}

\paragraph{•}
\begin{itemize}
\item \textbf{Misurazione}:;
\item \textbf{Valori ottimali}:;
\item \textbf{Valori accettati}:.
\end{itemize}

\paragraph{•}
\begin{itemize}
\item \textbf{Misurazione}:;
\item \textbf{Valori ottimali}:;
\item \textbf{Valori accettati}:.
\end{itemize}