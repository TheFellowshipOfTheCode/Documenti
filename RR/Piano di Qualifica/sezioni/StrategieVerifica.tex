\section{Strategie di verifica}
\subsection{Procedure di controllo qualità di processo}


\subsection{Procedure di controllo qualità di prodotto}


\subsection{Organizzazione}
Per ottenere la qualità prefissata saranno necessarie attività di verifica e di validazione che seguiranno tutto lo sviluppo del prodotto. Queste attività saranno legate al tipo di revisione (interna o esterna) che il prodotto deve affrontare lungo il periodo di sviluppo. Il teamG ha deciso di utilizzare un ciclo di vita incrementale per lo sviluppo del progetto. L’organizzazione della strategia di verifica sarà divisa in base alle revisioni e definita come segue:
\begin{itemize}
\item \textbf{Revisione dei Requisiti}: le attività di verifica dovranno essere eseguite su ogni documento destinato alla \textbf{Revisione dei Requisiti} e ogniqualvolta il documento abbia subito un cambio di versione. Le attività che verranno svolte sui documenti sono:
\begin{itemize}
\item[\-] Verifica della correttezza ortografica;
\item[\-] Verifica della correttezza lessicale;
\item[\-] Verifica della correttezza dei contenuti e la loro conformità alla natura del documento;
\item[\-] Verificare che ogni requisito abbia corrispondenza con una fonte (\textit{caso d’uso, capitolato, ecc.});
\item[\-] Verificare che ogni documento rispetti le \textit{Norme di Progetto}.
\end{itemize}
Inoltre sarà garantito il completo tracciamento dei requisiti e dei corrispondenti casi d’uso.
\item \textbf{Revisione di Progettazione}: arrivati a questa fase, sarà verificata e garantita la corrispondenza tra i requisiti definiti nelle fasi di \textbf{Analisi} e \textbf{Analisi di Dettaglio}, la cui consistenza e validità è data dallo svolgimento delle attività del punto precedente, e i moduli software derivanti dalla progettazione.
\item \textbf{Revisione di Qualifica}: verranno svolte le attività di codifica e di esecuzione dei test di unicità per la verifica del codice nella maniera più automatizzata possibile, rispettando i vincoli statici. In parallelo i verificatori dovranno controllare la presenza di eventuali anomalie. Arrivati in questa fase verrà garantito che:
\begin{itemize}
\item[\-] Ogni modulo software sia privo di errori e si comporti in maniera corretta con gli altri moduli;
\item[\-] Ogni modulo sia direttamente riconducibile ad un requisito;
\item[\-] Ogni requisito obbligatorio sia stato soddisfatto.
\end{itemize}
\item \textbf{Revisione di Accettazione}: tutte le attività svolte prima di questa fase garantiranno che il prodotto che è stato realizzato dal team Pragma funzioni in maniera corretta e conforme alle richieste e che tutti i requisiti siano stati adempiti.
\end{itemize}


\subsection{Pianificazione strategica e temporale}
L’obiettivo di rispettare le scadenze fissate nel Piano di Progetto v1.2.0 crea la necessità di avere un processo di verifica sistematico, organizzato e semplice. Per rispettare tali termini è richiesto che ogni membro del team studi a fondo le attività da eseguire, in modo da evitare errori comuni, interpretazioni sbagliate dei contenuti e improvvisazioni riguardo il compito da svolgere. Sarà necessario uno studio, a posteriori, atto a migliorare la qualità dei processi e ad aumentare la velocità di verifica e di svolgimento dei compiti. Questa attività andrà a classificare e quantificare gli errori più comuni e a verificare se sia necessaria l’introduzione di metodi aggiuntivi di verifica. Inoltre, permetterà di vedere quali attività richiedono più risorse così da poter migliorare la parallelizzazione delle attività ed aumentare l’efficienza. Di seguito vengono presentate le scadenze fissate:
\begin{itemize}
\item \textbf{Revisioni interne}
\begin{itemize}
\item[\-] \textbf{Revisione di Progettazione}: % inserire la data
\item[\-] \textbf{Revisione di Qualifica}: % inserire la data
\end{itemize}
\item \textbf{Revisioni esterne}
\begin{itemize}
\item[\-] \textbf{Revisione dei Requisiti}: % inserire la data
\item[\-] \textbf{Revisione di Accettazione}: % inserire la data
\end{itemize}
\end{itemize}


\subsection{Responsabilità}
Il \textit{Responsabile di Progetto} ha la responsabilità relativa all’assegnazione degli incarichi e quella derivata dai rapporti con gli stakeholders. A carico dell’\textit{Amministratore} è invece la responsabilità di fornire un ambiente di lavoro adeguato per lo svolgimento in modo corretto di tutti i compiti necessari alla realizzazione del progetto. Concluso il processo di verifica, il \textit{Verificatore} si prende la responsabilità sul documento o modulo software preso in esame. Inoltre ogni membro del team è responsabile per tutto il materiale da esso prodotto.


\subsection{Risorse}
Verranno di seguito elencate le risorse che serviranno per il corretto sviluppo del prodotto software.Tali risorse si possono racchiudere in due categorie.

\subsubsection{Necessarie}
Alcune risorse sono considerate necessarie per lo svolgimento del prodotto, queste sono:
\begin{itemize}
\item \textbf{Risorse umane}: sono le figure professionali necessarie nel team.
\begin{itemize}
\item[\-] \textit{Responsabile di Progetto};
\item[\-] \textit{Amministratore};
\item[\-] \textit{Analista};
\item[\-] \textit{Progettista};
\item[\-] \textit{Programmatore};
\item[\-] \textit{Verificatore}.
\end{itemize}
\item \textbf{Risorse software}: sono le applicazioni necessarie per:
\begin{itemize}
\item[\-] La stesura dei documenti in formato LATEX; 
\item[\-] La creazione di diagrammi UML;
\item[\-] Lo sviluppo di codice nei vari linguaggi scelti;
\item[\-] Effettuare l'analisi statica del codice;
\item[\-] Effettuare i vari test sul codice;
\item[\-] Automatizzare il più possibile i processi di verifica.
\end{itemize}
Tutti i software che verranno utilizzati sono descritti nelle Norme di Progetto.
\item \textbf{Risorse Hardware}: Sono necessari computer per poter eseguire le applicazioni citate in precedenza, server che conterranno gli applicativi per la gestione del ticketing, per il continuos integration, le mail, il repository e il database.
\end{itemize}

\subsubsection{Disponibili}
\begin{itemize}
\item \textbf{Risorse umane}: tutte le figure professionali descritte in precedenza verranno rappresentate dai membri del team.
\item \textbf{Risorse software}: sono disponibili per il team tutti i software descritti nelle \textit{Norme di Progetto}, gran parte di loro sono free e per altri è stato possibile ottenere una licenza per studenti.
\item \textbf{Risorse Hardware}: ogni membro del teamG dispone di un computer proprio, in alternativa sono disponibili quelli presenti nei laboratori dell’Università degli Studi di Padova. Per quanto riguarda i server,
% da inserire i server impiegati
\end{itemize}


\subsection{Strumenti}
Gli strumenti software sono descritti in dettaglio nel documento \textit{Norme di Progetto}. Essi vengono impiegati per:
\begin{itemize}
\item \textbf{Versionamento};
\item \textbf{Redazione dei documenti};
\item \textbf{Verifica dei documenti};
\item \textbf{Codifica del software};
\item \textbf{Verifica del codice};
\item \textbf{Organizzazione del gruppo};
\item \textbf{Condivisione dei file};
\end{itemize}


\subsection{Tecniche di analisi}

\subsubsection{Analisi statica}
L’analisi statica è una tecnica di verifica applicabile ai documenti e al codice. Verrà impiegata durante tutto lo sviluppo del sistema e sarà automatizzata il più possibile mediante gli strumenti descritti nelle \textit{Norme di Progetto}. L’analisi statica applicata al software non necessita che i programmi vengano eseguiti, bensì mira a trovare anomalie ed errori di sintassi e a fare predizioni sulla qualità e la manutenibilità del codice prodotto. In seguito vengono riportate le metodologie di applicazione dell’analisi statica.

\subsubsection{Walkthrough}
Questo metodo di applicazione dell’analisi statica consiste in una lettura del documento o del codice ricercando anomalie ed errori a largo spettro, ovvero senza una conoscenza precisa dei tipi di errori che è possibile riscontrare. Il walkthrough verrà applicato nelle prime fasi dello sviluppo poiché non si possiede ancora la concezione dei possibili e più frequenti errori. Mediante questa tecnica, verrà stilata una lista di controllo contenente gli errori rilevati più spesso. Quando la lista di controllo dei verificatori sarà sufficientemente completa, verrà allegata in appendice al documento \textit{Norme di Progetto} e dal walkthrough si passerà all’utilizzo della tecnica di inspection.

\subsubsection{Inspection}
Questa tecnica consiste in una lettura mirata dei documenti o del codice. Viene impiegata una lista di controllo contenente gli errori più probabili o commessi con maggior frequenza, al fine di rilevarli durante l’inspection. Poiché questa lista verrà ampliata con l’acquisizione di esperienza nella verifica, l’inspection diverrà sempre più efficace.

\subsubsection{Analisi dinamica}
Questa tipologia di analisi è applicata solamente al software prodotto e alle sue componenti. L’analisi dinamica viene svolta mediante test che verificano il funzionamento di componenti software e che ne identificano eventuali errori. Per ogni test è necessario definire:
\begin{itemize}
\item \textbf{Ambiente}: è il sistema hardware e software sul quale è pianificata l’esecuzione del test. Inoltre, è necessario specificare uno stato iniziale di partenza per il test;
\item \textbf{Specifica}: rappresenta l’insieme degli input e dei corrispondenti output attesi;
\item \textbf{Procedure}: si tratta di una specifica di istruzioni su come eseguire il test e di come i risultati devono essere interpretati e analizzati.
\end{itemize}
Affinché si possano ottenere risultati attendibili, è necessario che i test siano ripetibili, ovvero dato un certo input la loro esecuzione nello stesso ambiente deve produrre in output sempre gli stessi risultati. In seguito sono elencati i tipi di test effettuati sui prodotti software.

\subsubsubsection{Test di unità}
Per unità di prodotto software si intende la più piccola quantità software che risulta conveniente verificare singolarmente, tipicamente quella prodotta da un singolo programmatore. Ad esempio solitamente un modulo è parte dell’unità ed il componente invece integra più unità. I test di unità verificano che ogni unità funzioni correttamente, evidenziando errori di implementazione. Questi test verranno effettuati sui moduli base che compongono il software.

\subsubsubsection{Test di integrazione}
Questo tipo di test verifica che due o più unità, tipicamente moduli, precedentemente verificati, una volta assemblati funzionino correttamente. I test di integrazione possono aiutare a rilevare errori residui sui moduli e verificano anche che l’eventuale cooperazione di essi con componenti esterni, quali framework e librerie, non produca anomalie.

\subsubsubsection{Test di regressione}
Questo test consiste nel rieseguire tutti i test su un componente che ha subito una modifica. In questo modo si vuole verificare che il resto dei moduli continui a funzionare correttamente. Inoltre eseguire dei test di regressione consente di capire quali test sono a rischio di inesattezza in caso di modifiche al codice dei prodotti.

\subsubsubsection{Test di sistema}
Un test di sistema viene eseguito su un prodotto che si ritiene essere giunto ad una versione definitiva. Viene perciò verificato che il prodotto soddisfi tutti i requisiti imposti. Questo test è quindi la validazione dei prodotti software.

\subsubsubsection{accettazione}
Il test di accettazione coincide con il collaudo del software in presenza del Proponente. In caso di esito positivo, questo test determina un grado di maturità del prodotto tale da consentirne il rilascio.


\subsection{Misure e metriche}
% da inserire
\subsubsection{Metriche per i processi}

\subsubsection{Metriche per i documenti}

\subsubsection{Metriche per il codice}






