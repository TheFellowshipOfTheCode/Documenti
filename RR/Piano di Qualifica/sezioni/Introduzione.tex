\newpage
\section{Introduzione}
\subsection{Scopo del Documento}
Lo scopo del presente documento è quello di illustrare le strategie di verifica e validazione che il gruppo \gruppo \hspace{1mm} ha deciso di adottare per raggiungere gli obiettivi qualitativi prefissati sul prodotto da sviluppare. Per raggiungere tali obiettivi è necessaria una verifica continua sulle attività svolte. In questo modo sarà possibile individuare rapidamente eventuali anomalie che si potrebbero riscontrare e quindi risolverle in modo tempistico e senza spreco di risorse, garantendo la correttezza del prodotto e il soddisfacimento del cliente.

\subsection{Scopo del prodotto}
Lo scopo del prodotto è di permettere la creazione e gestione di questionari in grado di identificare le lacune dei candidati prima, durante e al termine di un corso di formazione. 
\\Il sistema dovrà offrire le seguenti funzionalità:
\begin{itemize}
	\item
	Archiviare questionari in un server suddivisi per argomento;
	\item
	Somministrare all'utente, tramite un'interfaccia, questionari specifici per argomento scelto;
	\item
	Verificare e valutare i questionari scelti dagli utenti in base alle risposte date.
\end{itemize}
La parte destinata ai creatori di questionari dovrà essere fruibile attraverso un \textit{browser\ped{G}} desktop, abilitato all'utilizzo delle tecnologie \textit{HTML5\ped{G}}, \textit{CSS3\ped{G}} e \textit{JavaScript\ped{G}}. La parte destinata agli esaminandi sarà utilizzabile su qualunque dispositivo: dal personal computer ai tablet e smartphone.

\subsection{Glossario}
Al fine di evitare ogni ambiguità i termini tecnici del dominio del progetto, gli acronimi e le parole che necessitano di ulteriori spiegazioni saranno nei vari documenti marcate con il pedice \ped{G} e quindi presenti nel documento \textit{\G}.
\subsection{Riferimenti}
\subsubsection{Normativi}
\begin{itemize}
\item \textit{\NdPv};
\item \textbf{Capitolato}: \color{blue}{\url{http://www.math.unipd.it/~tullio/IS-1/2014/Progetto/C4.pdf}}.
\end{itemize}

\subsubsection{Informativi}
\begin{itemize}
\item \textbf{\PdP}: \textit{Piano di Progetto};
\item \textbf{Capacity Maturity Model}: \url{http://en.wikipedia.org/wiki/Capability_Maturity_Model};
\item \textbf{Capacity Maturity Model Integration}: \url{http://en.wikipedia.org/wiki/Capability_Maturity_Model_Integration};
\item \textbf{PDCA (Plan-Do-Check-Act)}: \url{http://it.wikipedia.org/wiki/PDCA};
\item \textbf{Standard ISO/IEC 12207:2008 - IEEE Std 12207-2008}: \url{http://ieeexplore.ieee.org/xpl/mostRecentIssue.jsp?punumber=4475822};
\item \textbf{Standard ISO/IEC 9126}: \url{http://it.wikipedia.org/wiki/ISO/IEC_9126};
\item \textbf{Indice di Gulpease}: \url{http://it.wikipedia.org/wiki/Indice_Gulpease};
\item \textbf{MI and MINC - Maintainability Index}: \url{http://www.virtualmachinery.com/sidebar4.htm}.
\end{itemize}
