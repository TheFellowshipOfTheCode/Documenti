\section{Definizione Obiettivi di Qualità}
Nei punti seguenti vengono elencate e descritte le diverse caratteristiche che il software prodotto mira ad avere. In particolare il team si impegna a garantire le seguenti qualità in riferimento allo standard ISOG/IEC 9126:

\subsection{Funzionalità}
La funzionalità rappresenta la capacità del software prodotto di fornire le funzioni richieste da parte del Proponente.

\subsection{Affidabilità}
L’affidabilità rappresenta la capacità del prodotto software di svolgere correttamente le sue funzioni quando viene utilizzato sotto determinate condizioni.

\subsection{Usabilità}
L’usabilità rappresenta la capacità del prodotto di essere comprensibile, apprendibile, e attraente per l’utente.

\subsection{Efficienza}
L’efficienza rappresenta la capacità del prodotto software di fornire le funzionalità offerte utilizzando il minor numero di risorse possibili.

\subsection{Manutenibilità}
La manutenibilità rappresenta la capacità del prodotto di essere modificato, tramite correzioni, miglioramenti o adattamenti del software a modifiche negli ambienti, nei requisiti e nelle specifiche funzionali.

\subsection{Portabilità}
La portabilità rappresenta la capacità del software di poter essere utilizzato su più ambienti di lavoro differenti.
