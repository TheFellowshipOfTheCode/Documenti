\section{Definizione Obiettivi di Qualità}
Nei punti seguenti vengono elencate e descritte le diverse caratteristiche che il team ha individuato dallo standard ISO/IEC 9126 e che si impegna a perseguire, per garantire al prodotto finale una buona qualità.	

\subsection{Funzionalità}
La funzionalità rappresenta la capacità del software prodotto di fornire tutte le funzioni richieste dal Proponente, che saranno individuate attraverso l'\textit{Analisi dei Requisiti}. \\ Affinché questa qualità possa ritenersi soddisfatta, il sistema dovrà superare positivamente tutti i test che verranno predisposti.

\subsection{Affidabilità}
L’affidabilità rappresenta la capacità del prodotto software di svolgere correttamente le sue funzioni durante il suo utilizzo, anche nel caso in cui si presentino situazioni anomale. \\ Ogni utilizzo del software in questione, nelle sue molteplici funzionalità, deve portare ad un risultato che corrisponda a ciò che si è atteso o, in caso di uso scorretto, al sollevamento di un' eccezione che generi un opportuno messaggio di errore; in questo modo non potranno esserci comportamenti che non siano gestiti dal sistema.

\subsection{Usabilità}
L’usabilità rappresenta la capacità del prodotto di essere facilmente comprensibile e attraente per qualsiasi utente che lo andrà ad utilizzare.

\subsection{Efficienza}
L’efficienza rappresenta la capacità di eseguire le funzionalità offerte dal software nel minor tempo possibile utilizzando al tempo stesso il minor numero di risorse possibili.

\subsection{Manutenibilità}
La manutenibilità rappresenta la capacità del prodotto di essere modificato, tramite correzioni e/o miglioramenti. \\ Il software dovrà inoltre consentire a coloro che si occuperanno di manutenerlo di identificare velocemente le possibili cause di errori e malfunzionamenti.

\subsection{Portabilità}
La portabilità rappresenta la capacità del software di poter essere utilizzato su tutti quegli ambienti di lavoro nei quali è stato previsto un suo utilizzo.
