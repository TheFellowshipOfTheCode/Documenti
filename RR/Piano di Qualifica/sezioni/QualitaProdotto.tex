\newpage
\section{Qualità di prodotto}
Per garantire una buona qualità di prodotto, il \textit{team} ha individuato dallo standard \textit{ISO/IEC 9126} le qualità che ritiene più importanti nell'arco del ciclo di vita del prodotto e le ha istanziate individuando obiettivi e metriche coerenti con i livelli di qualità perseguiti.


\subsection{Funzionalità (6.1)}
La \textit{funzionalità} rappresenta la capacità del software prodotto di fornire le funzioni richieste da parte del \textit{Proponente}.

\subsubsection{Obiettivi di qualità}
Il \textit{team} si impegnerà affinché:
\begin{itemize}
\item \textbf{Appropriatezza (6.1.1)}: tutte le funzionalità fornite dal prodotto saranno coerenti con le aspettative del cliente;
\item \textbf{Accuratezza (6.1.2)}: il prodotto fornirà i risultati attesi, con il livello di dettaglio richiesto;
\item \textbf{Sicurezza (6.1.4)}: il prodotto si occuperà di proteggere le informazioni e i dati in modo tale che persone o sistemi non autorizzati non possano accedervi per leggerli o modificarli.
\end{itemize}
\subsubsection{Metriche}
\paragraph{Completezza dell'implementazione funzionale}
Indica la percentuale di requisiti funzionali coperti dall'implementazione.
\begin{itemize}
\item \textbf{Misurazione}: $C=(1-\frac{N_{FM}}{N_{FI}}) \cdot 100$, dove $N_{FM}$ è il numero di funzionalità mancanti nell'implementazione e $N_{FI}$ è il numero di funzionalità individuate nell'attività di analisi;
\item \textbf{Range-ottimale}: $100$;
\item \textbf{Range-accettazione}: $100$.
\end{itemize}
\paragraph{Accuratezza rispetto alle attese}
Indica la percentuale di risultati concordi alle attese.
\begin{itemize}
\item \textbf{Misurazione}: $A=(1-\frac{N_{RD}}{N_{TE}}) \cdot 100$, dove $N_{RD}$ è il numero di test che producono risultati discordanti rispetto alle attese e $N_{TE}$ è il numero di test-case eseguiti;
\item \textbf{Range-ottimale}: $100$;
\item \textbf{Range-accettazione}: $90 - 100$.
\end{itemize}
\paragraph{Controllo degli accessi}
Indica la percentuale di operazioni illegali non bloccate.
\begin{itemize}
\item \textbf{Misurazione}: $I=\frac{N_{IE}}{N_{II}} \cdot 100$, dove $N_{IE}$ è il numero di operazioni illegali effettuabili dai test e $N_{II}$ è il numero di operazioni illegali individuate;
\item \textbf{Range-ottimale}: $0$;
\item \textbf{Range-accettazione}: $0 - 10$.
\end{itemize}


\subsection{Affidabilità (6.2)}
L'\textit{affidabilità} rappresenta la capacità del prodotto software di svolgere correttamente le sue funzioni quando viene utilizzato sotto determinate condizioni.
\subsubsection{Obiettivi di qualità}
L'esecuzione del prodotto dovrà presentare le seguenti caratteristiche:
\begin{itemize}
\item \textbf{Maturità (6.2.1)}: il software impedirà l'esecuzione di malfunzionamenti, operazioni illegali e restituzione di risultati errati (\textit{failure}) in seguito a \textit{fault};
\item \textbf{Fault tolerance (6.2.2)}: il prodotto sarà in grado di gestire situazioni di errore anche in seguito all'insorgere di guasti.
\end{itemize}
\subsubsection{Metriche}
\paragraph{Densità di failure}
Indica la percentuale di operazioni di testing conclusesi in failure.
\begin{itemize}
\item \textbf{Misurazione}: $F=\frac{N_{FR}}{N_{TE}} \cdot 100$, dove $N_{FR}$ è il numero di failure rilevati durante l'attività di testing e $N_{TE}$ è il numero di test-case eseguiti;
\item \textbf{Range-ottimale}: $0$;
\item \textbf{Range-accettazione}: $0 - 10$.
\end{itemize}
\paragraph{Blocco di operazioni non corrette}
Indica la percentuale di funzionalità in grado di gestire correttamente i fault che potrebbero verificarsi.
\begin{itemize}
\item \textbf{Misurazione}: $B=\frac{N_{FE}}{N_{ON}} \cdot 100$, dove $N_{FE}$ è il numero di failure evitati durante i test effettuati e $N_{ON}$ è il numero di test-case eseguiti che prevedono l'esecuzione di operazioni non corrette, causa di possibili failure;
\item \textbf{Range-ottimalite}: $100$;
\item \textbf{Range-accettazione}: $80 - 100$.
\end{itemize}

\subsection{Usabilità (6.3)}
L'\textit{usabilità} rappresenta la capacità del prodotto di essere comprensibile, apprendibile, e attraente per l'utente.
\subsubsection{Obiettivi di qualità}
Il prodotto dovrà puntare ai seguenti obiettivi di usabilità:
\begin{itemize}
\item \textbf{Comprensibilità (6.3.1)}: il software permetterà all'utente di riconoscerne le funzionalità e comprenderne le modalità di utilizzo per raggiungere i risultati attesi;
\item \textbf{Apprendibilità (6.3.2)}: il prodotto semplificherà all'utente l'apprendimento del suo utilizzo;
\item \textbf{Operabilità (6.3.3)}: il prodotto metterà a proprio agio l'utente presentandogli messaggi e funzionalità coerenti con le sue aspettative;
\item \textbf{Attrattiva (6.3.4)}: il prodotto presenterà delle funzionalità e delle interfacce utente che lo renderanno attrattivo per l'utilizzatore.
\end{itemize}
\subsubsection{Metriche}
\paragraph{Comprensibilità delle funzioni offerte}
Indica la percentuale di operazioni comprese in modo immediato dall'utente, senza la consultazione del manuale.
\begin{itemize}
\item \textbf{Misurazione}: $C=\frac{N_{FC}}{N_{FO}} \cdot 100$, dove $N_{FC}$ è il numero di funzionalità comprese in modo immediato dall'utente durante l'attività di testing del prodotto e $N_{FO}$ è il numero di funzionalità offerte dal sistema;
\item \textbf{Range-ottimale}: $90 - 100$;
\item \textbf{Range-accettazione}: $80 - 100$.
\end{itemize}
\paragraph{Facilità di apprendimento delle funzionalità}
Indica il tempo medio impiegato dall'utente nell'imparare ad usare correttamente una data funzionalità.
\begin{itemize}
\item \textbf{Misurazione}: indicatore numerico (espresso in \textit{minuti}) che tiene traccia del tempo medio impiegato dall'utente nell'apprendere il corretto utilizzo di una funzionalità offerta dal sistema;
\item \textbf{Range-ottimale}: $0 - 15$;
\item \textbf{Range-accettazione}: $0 - 30$.
\end{itemize}
\paragraph{Consistenza operazionale in uso}
Indica la percentuale di messaggi e funzionalità offerte all'utente che rispettano le sue aspettative riguardo al comportamento del software.
\begin{itemize}
\item \textbf{Misurazione}: $C=(1-\frac{N_{MFI}}{N_{MFO}}) \cdot 100$, dove $N_{MFI}$ è il numero di messaggi e funzionalità che non rispettano le aspettative dell'utente e $N_{MFO}$ è il numero di messaggi e funzionalità offerti dal sistema;
\item \textbf{Range-ottimale}: $90 - 100$;
\item \textbf{Range-accettazione}: $80 - 100$.
\end{itemize}
\paragraph{Elementi personalizzabili}
Indica la percentuale di elementi personalizzabili rispetto a quanto l'utente vorrebbe personalizzare.
\begin{itemize}
\item \textbf{Misurazione}: $C=\frac{N_{EC}}{N_{PD}} \cdot 100$, dove $N_{EC}$ è il numero di elementi per i quali è stata prevista personalizzazione da parte dell'utente e $N_{PD}$ è il numero di elementi che l'utente desidererebbe personalizzare;
\item \textbf{Range-ottimale}: $90 - 100$;
\item \textbf{Range-accettazione}: $60 - 100$.
\end{itemize}

\subsection{Efficienza (6.4)}
\label{efficienza}
L'\textit{efficienza} rappresenta la capacità del prodotto software di fornire le funzionalità offerte con performance appropriate, relativamente a risorse usate, sotto determinate condizioni.
\subsubsection{Obiettivi di qualità}
Il prodotto dovrà essere efficiente, in particolare:
\begin{itemize}
\item \textbf{Efficienza nel tempo (6.4.1)}: il software dovrà fornire appropriati tempi di risposta e  di elaborazione nell'eseguire le funzionalità previste sotto determinate condizioni di utilizzo;
\item \textbf{Utilizzo delle risorse (6.4.2)}: il prodotto dovrà rispettare dei limiti in termini di utilizzo di risorse, per evitare che \textit{client} e \textit{server} siano sottoposti a stress di calcolo, o debbano attendere a lungo se vi è uno scambio eccessivo di dati attraverso la connessione internet.
\end{itemize}
\subsubsection{Metriche}
\paragraph{Tempo di risposta}
Indica il periodo temporale medio che intercorre fra la richiesta al software di una determinata funzionalità e la restituzione del risultato all'utente.
\begin{itemize}
\item \textbf{Misurazione}: $T_{RISP} = \frac{\sum_{i=1}^{n} T_{i}}{n}$ (con $T_{RISP}$ espresso in \textit{secondi}) dove $T_{i}$ è il tempo intercorso fra la richiesta $i$ di una funzionalità ed il completamento delle operazioni necessarie a restituire un risultato a tale richiesta;
\item \textbf{Range-ottimale}: $0 - 3$;
\item \textbf{Range-accettazione}: $0 - 8$.
\end{itemize}

\subsection{Manutenibilità (6.5)}
La \textit{manutenibilità} rappresenta la capacità del prodotto di essere modificato, tramite correzioni, miglioramenti o adattamenti del software a modifiche negli ambienti, nei requisiti e nelle specifiche funzionali.
\subsubsection{Obiettivi di qualità}
Le operazioni di manutenzione andranno agevolate il più possibile adottando le seguenti caratteristiche:
\begin{itemize}
\item \textbf{Analizzabilità (6.5.1)}: il software dovrà consentire a coloro che si occuperanno di manutenerlo di identificare velocemente le possibili cause di errori e malfunzionamenti;
\item \textbf{Modificabilità (6.5.2)}: gli obiettivi di qualità fissati per le attività di progettazione e di codifica permetteranno ai manutentori di applicare cambiamenti a componenti del prodotto;
\item \textbf{Stabilità (6.5.3)}: gli obiettivi di qualità fissati per le attività di progettazione e di codifica permetteranno ai manutentori di evitare l'insorgenza di effetti indesiderati a seguito di modifiche effettuate ad alcune componenti del prodotto;
\item \textbf{Testabilità (6.5.4)}: gli obiettivi di qualità fissati per le attività di progettazione e di codifica permetteranno ai manutentori di verificare e validare attraverso test i cambiamenti effettuati a componenti del prodotto.
\end{itemize}
\subsubsection{Metriche}
\paragraph{Capacità di analisi di failure}
\label{capacitaAnalisiFailure}
Indica la percentuale di failure registrate, delle quali sono state individuate le cause.
\begin{itemize}
\item \textbf{Misurazione}: $I=\frac{N_{FI}}{N_{FR}} \cdot 100$, dove $N_{FI}$ è il numero di failure delle quali sono state individuate le cause e $N_{FR}$ è il numero di failure rilevate;
\item \textbf{Range-ottimale}: $80 - 100$;
\item \textbf{Range-accettazione}: $60 - 100$.
\end{itemize}
\paragraph{Impatto delle modifiche}
Indica la percentuale di modifiche effettuate in risposta a failure che hanno portato all'introduzione di nuove failure in altre componenti del sistema.
\begin{itemize}
\item \textbf{Misurazione}: $I=\frac{N_{FRF}}{N_{FR}} \cdot 100$, dove $N_{FRF}$ è il numero di failure risolte con l'introduzione di nuove failure e $N_{FR}$ è il numero di failure risolte;
\item \textbf{Range-ottimale}: $0 - 10$;
\item \textbf{Range-accettazione}: $0 - 20$.
\end{itemize}

\subsection{Portabilità (6.6)}
\label{portabilita}
La \textit{portabilità} rappresenta la capacità del software di poter essere trasportato da un ambiente ad un altro.
\subsubsection{Obiettivi di qualità}
Sarà agevolata la portabilità del prodotto adottando i seguenti obiettivi:
\begin{itemize}
\item \textbf{Adattabilità (6.6.1)}: il software sarà in grado di adattare le sue funzionalità a tutti gli ambienti individuati nell'analisi dei requisiti;
\item \textbf{Sostituibilità (6.6.4)}: il software permetterà all'utente di mantenere lo stesso metodo di lavoro e la stessa visione ad alto livello adottata nell'utilizzo di un prodotto precedentemente utilizzato per i medesimi scopi e nello stesso ambiente.
\end{itemize}
\subsubsection{Metriche}
\paragraph{Versioni dei browser supportate}
Indica la percentuale di versioni di \textit{browser} attualmente supportate, fra quelle individuate dai requisiti.
\begin{itemize}
\item \textbf{Misurazione}: $S=\frac{N_{VS}}{N_{VI}} \cdot 100$, dove $N_{VS}$ è il numero di versioni di \textit{browser} supportate dal prodotto e $N_{VI}$ è il numero di versioni di \textit{browser} che devono essere supportate dal prodotto;
\item \textbf{Range-ottimale}: $100$;
\item \textbf{Range-accettazione}: $100$.
\end{itemize}
\paragraph{Inclusione di funzionalità da altri prodotti}
Indica la percentuale di funzionalità del software utilizzato in precedenza dall'utente che produce risultati simili a quelli ottenuti dal prodotto in oggetto.
\begin{itemize}
\item \textbf{Misurazione}: $I=\frac{N_{FPA}}{N_{FPP}} \cdot 100$, dove $N_{FPA}$ è il numero di funzionalità del software utilizzato in precedenza dall'utente che produce risultati simili a quelli ottenuti dal prodotto in oggetto e $N_{FPP}$ è il numero di funzionalità offerte dal software utilizzato in precedenza dall'utente;
\item \textbf{Range-ottimale}: $90 - 100$;
\item \textbf{Range-accettazione}: $80 - 100$.
\end{itemize}
