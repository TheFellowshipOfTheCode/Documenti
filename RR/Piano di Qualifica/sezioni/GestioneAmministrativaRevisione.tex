\section{Gestione Amministrativa della Revisione}

\subsection{Comunicazione e Risoluzione delle Anomalie}
Il \textit{Verificatore} ha il compito di individuare le eventuali anomalie, presenti nella documentazione o nel codice che sta verificando. Dovrà quindi segnalarle al \textit{Responsabile di Progetto}, attraverso il sistema di ticketing descritto nelle \textit{Norme di Progetto}, che incaricherà un membro del team alla risoluzione del problema individuato.\\ I tipi di anomalie che si possono presentare sono:
\begin{itemize} 
\item Errori di battitura;
\item Errori di tipo ortografico, sintattico o un qualsiasi imprecisione del linguaggio; 
\item Errori di formattazione del testo;
\item Errori di compilazione o a runtime di una determinata sezione di codice, dovuta ad un' inesattezza logica o ad un uso scorretto del linguaggio utilizzato;
\item Violazioni delle regole fissate nelle \textit{Norme di Progetto}; 
\item Mancato rispetto dei range nella valutazione di una determinata caratteristica di qualità;
\item Incongruenza del prodotto rispetto le funzionalità individuate nell’\textit{Analisi dei Requisiti}.
\end{itemize}
