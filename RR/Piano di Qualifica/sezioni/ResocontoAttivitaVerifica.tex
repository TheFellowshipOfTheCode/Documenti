\newpage
\section{Resoconto attività di verifica}
In questa sezione del documento vengono descritti e analizzati gli esiti delle attività di verifica svolte su tutti i documenti che vengono consegnati nelle varie revisioni di avanzamento del progetto.

\subsection{Revisione dei Requisiti}

\subsubsection{Tracciamento casi d'uso e requisiti}
Il team ha deciso di utilizzare il software interno \textit{DocumentsDB} in modo da facilitare il tracciamento sia delle relazioni fra casi d'uso e requisiti, sia delle relazioni fra requisiti e fonti.

\subsubsection{Analisi statica dei documenti}
L'analisi dei documenti mediante \textit{walkthrough\ped{G}} ha portato all'individuazione di alcuni errori frequenti a partire dai quali è stata stilata una lista di controllo che è stata inserita all'interno dei Processi Organizzativi nelle \textit{\NdP}. Grazie a questa, sarà possibile applicare l'\textit{inspection\ped{G}} per le future attività di verifica.

\subsubsection{Esiti verifiche automatizzate}
Vengono qui riportati gli esiti delle verifiche automatizzate per il calcolo dell'\textit{indice Gulpease\ped{G}}, alle quali sono stati sottoposti tutti i documenti.
\begin{table}[h]
\begin{center}
\begin{tabular}{|c|c|c|c|}
\hline Documento & Indice Gulpease & Esito\\
\hline
\emph{Norme di Progetto} &  & Superato \\
\emph{Studio di Fattibilità} &  & Superato \\
\emph{Piano di Progetto} &  & Superato \\
\emph{Piano di Qualifica} &   & Superato \\
\emph{Analisi dei Requisiti} &  & Superato \\
\emph{Glossario} &  & Superato \\
\emph{Verbale} &  & Superato \\
\hline
\end{tabular}
\caption{Resoconto verifiche automatizzate - Revisione dei Requisiti}
\end{center}
\end{table}