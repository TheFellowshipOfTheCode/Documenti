\section{Resoconto Attività di Verifica}
In questa sezione del documento vengono descritti e analizzati gli esiti delle attività di verifica svolte su tutti i documenti che vengono consegnati nelle varie revisioni di avanzamento del progetto.

\subsection{Revisione dei Requisiti}


\subsubsection{Tracciamento Casi d’Uso e Requisiti}
Per il tracciamento delle relazioni fra casi d’uso e requisiti, così come per il tracciamento delle relazioni fra requisiti e fonti, si è deciso di sfruttare il software interno \textit{DocumetsDB}.

\subsubsection{Analisi Statica dei Documenti}
L'analisi dei documenti mediante \textit{walkthrough} ha portato all'individuazione di alcuni errori frequenti a partire dai quali è stata stilata una lista di controllo da usare per le future attività di verifica. \\
In ordine di frequenza sono stati riscontrati i seguenti errori:
\begin{itemize}
\item Mancato utilizzo dei comandi \LaTeX~ personalizzati;
\item Mancato rispetto delle norme riguardanti gli elenchi puntati:
\begin{itemize}
\item La prima parola di una voce dell'elenco non iniziava con una lettera maiuscola;
\item La voce dell'elenco terminava con un punto anziché con un punto e virgola (o viceversa).
\end{itemize}
\item Errori riguardanti la struttura delle frasi: alcune frasi erano scritte in modo più colloquiale che formale o con tempi verbali errati;
\item Errori di battitura.
\end{itemize}

\subsubsection{Esiti Verifiche Automatizzate}
Ogni documento è stato sottoposto a delle verifiche automatizzate per il calcolo dell'indice di Gulpease e per il controllo ortografico.
\begin{table}[h]
\begin{center}
\begin{tabular}{|c|c|c|c|}
\hline Documento & Indice Gulpease & Esito\\
\hline
\emph{Analisi dei Requisiti} & 71,70 & Superato \\
\emph{Glossario} & 43,87 & Superato \\
\emph{Norme di Progetto} & 56,89 & Superato \\
\emph{Piano di Progetto} & 51,03 & Superato \\
\emph{Piano di Qualifica} & 52,57  & Superato \\
\emph{Studio di Fattibilità} & 63,87 & Superato \\
\emph{Verbale} & 58,40 & Superato \\
\hline
\end{tabular}
\caption{Resoconto verifiche automatizzate - Revisione dei Requisiti}
\end{center}
\end{table}
\\Dal calcolo dell'indice di Gulpease sono state escluse le tabelle e le didascalie.

\subsubsection{Considerazioni Finali}
\begin{itemize}
\item Lo svolgimento delle attività di verifica è ancora poco automatizzato, questo perché senza una prima lista di errori comuni non è stato possibile predisporre automatismi per rilevarli e correggerli. Per le verifiche successive si cercherà di automatizzare ulteriormente questo processo;
\item La stesura dei documenti ha raggiunto un buon grado di automatizzazione; per quanto riguarda l'\textit{Analisi dei Requisiti} e il \textit{Glossario}, infatti, la maggior parte del codice \LaTeX~ viene generato in modo automatico a partire dai dati presenti all'interno di \textit{DocumentsDB}. Non è stato possibile raggiungere lo stesso livello di automazione per gli altri documenti a causa della natura del loro contenuto.
\end{itemize}