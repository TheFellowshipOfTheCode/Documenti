\section{Specifica dei test}
Il \textit{team\ped{G}} \gruppo, al fine di implementare del software che sia di qualità, ha strutturato dei test atti a verificare che il software prodotto rispecchi le funzionalità a fronte di risultati attesi.
Questi test sono il frutto dell'applicazione delle tecniche di verifica dinamica che sono state introdotte nel documento \NdP.
Tutte le attività di testing prodotte devono poter essere ripetibili e devono essere deterministiche, al fine di poter fornire delle informazioni utili a intraprendere azioni di correzione, nel caso in cui i risultati ottenuti siano diversi da quelli attesi.
Per avere un tracciamento dei test prodotti e dei risultati ottenuti si è scelto di classificare il tutto producendo dei log che siano di facile consultazione e che possano fornire una precisa indicazione di quelli che sono stati gli output di queste attività di verifica, eventuali errori o eventuali risultati che siano non coerenti con quanto in precedenza fissato.
\subsection{Tipi di test}
Sono stati individuati quattro livelli di testing e sono rispettivamente:
\begin{itemize}
\item \textbf{Test di unità [TU]:} con questa tipologia di test si cerca di verificare la più piccola parte di lavoro prodotta da un programmatore. Questo si traduce tendenzialmente a verificare i metodi e le funzioni scritte;
\item \textbf{Test di integrazione [TI]:} con questa tipologia di test si cerca di verificare le componenti di sistema. Più precisamente, l'obiettivo è quello di testare il funzionamento dei vari package prodotti, sia singolarmente che nel loro insieme;
\item \textbf{Test di sistema [TS]:} con questa tipologia di test si cerca di verificare che il comportamento e il funzionamento dell'architettura siano corretti;
\item \textbf{Test di validazione [TV]:} con questa tipologia di test si vuole verificare che il lavoro prodotto soddisfi quanto richiesto dal proponente.
\end{itemize}

%
% INSERIRE TABELLE TEST \input{tabelletest.tex}
%