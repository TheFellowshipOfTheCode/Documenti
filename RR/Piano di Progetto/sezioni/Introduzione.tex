\section{Introduzione}
\subsection{Scopo del documento}
Questo documento contiene la pianificazione delle attività che saranno svolte dai membri del gruppo \textit{\gruppo} per realizzare il
progetto \textit{\progetto}. In particolare, questo documento contiene:

	\begin{itemize}
		\item Analisi e trattamento dei rischi;
		\item Il preventivo delle risorse necessarie allo svolgimento del \textit{progetto\ped{G}};
		\item il \textit{consuntivo\ped{G}} delle attività finora risolte.
	\end{itemize}
	
\subsection{Scopo del prodotto}
Lo scopo del prodotto è di permettere la creazione e la gestione di questionari per sperimentare un metodo innovativo di valutazione delle competenze, in ambito lavorativo e in ambito scolastico. In particolare, il prodotto vuole verificare la possibilità di svolgere questionari prima e durante un corso di formazione, in modo da identificare le lacune dei candidati in modo anticipato rispetto la sede d'esame.
\subsection{Glossario}
Al fine di evitare ogni ambiguità di linguaggio e massimizzare la comprensione dei documenti, i termini tecnici, di dominio, gli acronimi e le parole che necessitano di essere chiarite, sono riportate nel documento \textit{Glossario\ped{G}}. Ogni occorrenza dei vocaboli presenti nel \textit{Glossario\ped{G}} è maracata da una \textit{G} maiuscola in pedice ed è scritta in corsivo (es: \textit{Esempio\ped{G}}).
\subsection{Riferimenti}
\subsubsection{Normativi}
	\begin{itemize}
		\item Norme di progetto: documento \textit{norme di progetto\ped{G}};
		\item Capitolato d'appalto C5: \textit{\progetto}, reperibile all'indirizzo:\\
		\url{http://www.math.unipd.it/~tullio/IS-1/2015/Progetto/C5.pdf};
		\item Regolamento di \textit{progetto didattico\ped{G}}:\\
		\url{http://www.math.unipd.it/~tullio/IS-1/2015/Dispense/PD01.pdf};
		\item Vincoli di organigramma e dettagli tecnico-economici:
		\url{http://www.math.unipd.it/~tullio/IS-1/2015/Progetto/PD01b.html}.
	\end{itemize}
\subsubsection{Informativi}
	\begin{itemize}
		\item Ingegneria del Software - Ian Sommerville - Ottava edizione:
		\begin{itemize}
			\item Capitolo 4: \textit{Software management};
			\item Capitolo 5: \textit{Gestione di progetti}.
		\end{itemize}
		\item Slide del corso \textit{Ingegneria del Software modulo A}:
		\url{http://www.math.unipd.it/~tullio/IS-1/2015/};
	\end{itemize}
\subsection{Ciclo di Vita}
Il modello di ciclo di vita scelto è il \textbf{modello incrementale}. Esso prevede che:
\begin{itemize}
	\item L'\textit{Analisi dei Requisiti} e la \textit{Progettazione Architetturale\ped{G}} costituiscano una base solida, esse infatti non vengono ripetute: I requisiti e l'architettura\ped{G} del sistema\ped{G} sono identificati e fisati definitivamente e sono essenziali per la pianificazione dei \textit{cicli incrementali\ped{G}};
	\item La \textit{Progettazione di Dettaglio\ped{G}}, la \textit{codifica{G}} e i \textit{test\ped{G}} vengono ripetuti, permettendo il miglioramento di parti del sistema\ped{G} già esistenti e l'aggiunta di funzionalità del prodotto portando ad un miglioramento della base del sistema\ped{G}.
\end{itemize}
I vantaggi attesi dalla scelta di tale modello sono i seguenti:
\begin{itemize}
	\item I requisiti utenti sono realizzati in base all'importanza strategica, ovvero vengono soddisfatti per primi quelli di maggiore rilevanza;
	\item Ogni incremento può produrre valore e riduce il rischio di fallimento, in quando esso consolida ulteriormente la base ed eventualmente ne aumenta la qualità;
	\item Esecuzione più dettagliata dei \textit{test\ped{G}} che quindi risulteranno essere maggiormente esaustivi;
	\item Rilasci multipli e successivi che inizialmente punteranno a soddisfare i requisiti di primaria importanza mentre successivamente andranno ad adempiere a possibili funzionalità aggiuntive ed opzionali. Si ha, quindi, la possibilità di presentare il prima possibile al proponente\ped{G} un prototipo\ped{G} che realizza le funzionalità primarie per permettergli di fornire una prima valutazione del lavoro nel periodo di produzione.
\end{itemize}
\subsection{Scadenze}
Il gruppo \textit{\gruppo} ha deciso di rispettare le seguenti scadenze:
\begin{itemize}
	\item \textbf{Revisione dei Requisiti}: 2016-02-16;
	\item \textbf{Revisione di Progettazione}: 2016-01-18 presentandosi con \textit{Revisione di Progettazione massima}\ped{G};
	\item \textbf{Revisione di Qualifica}: 2016-05-23;
	\item \textbf{Revisione di Accettazione}: 2016-06-17.
\end{itemize}
