\newpage
\section{Introduzione}
\subsection{Scopo del documento}
Questo documento contiene la pianificazione delle attività che saranno svolte dai membri del gruppo \gruppo\ per realizzare il progetto \progetto. In particolare, questo documento contiene:

	\begin{itemize}
		\item Analisi e trattamento dei rischi;
		\item Il preventivo delle risorse necessarie allo svolgimento del progetto;
		\item Il consuntivo delle attività finora svolte.
	\end{itemize}
	
\subsection{Scopo del prodotto}
Lo scopo del prodotto è quello di permettere la creazione e gestione di questionari in grado di identificare le lacune dei candidati prima, durante e al termine di un corso di formazione. 
\\Il sistema dovrà offrire le seguenti funzionalità:
\begin{itemize}
	\item
	Archiviare questionari in un server suddivisi per argomento;
	\item
	Somministrare all'utente, tramite un'interfaccia, questionari specifici per argomento scelto;
	\item
	Verificare e valutare i questionari scelti dagli utenti in base alle risposte date.
\end{itemize}
La parte destinata ai creatori di questionari dovrà essere fruibile attraverso un \textit{browser\ped{G}} desktop, abilitato all'utilizzo delle tecnologie \textit{HTML5\ped{G}}, \textit{CSS3\ped{G}} e \textit{JavaScript\ped{G}}. La parte destinata agli esaminandi sarà utilizzabile su qualunque dispositivo: dal personal computer ai tablet e smartphone.

\subsection{Glossario}
Al fine di evitare ogni ambiguità i termini tecnici del dominio del progetto, gli acronimi e le parole che necessitano di ulteriori spiegazioni saranno nei vari documenti marcate con il pedice \ped{G} e quindi presenti nel documento \textit{\G}.
\subsection{Riferimenti}
\subsubsection{Normativi}
	\begin{itemize}
		\item \textit{\NdPv};
		\item \textbf{Capitolato d'appalto C5: \progetto}:\\
		\url{http://www.math.unipd.it/~tullio/IS-1/2015/Progetto/C5.pdf};
		\item \textbf{Regolamento del progetto didattico}:\\
		\url{http://www.math.unipd.it/~tullio/IS-1/2015/Dispense/PD01.pdf};
		\item \textbf{Vincoli di organigramma e dettagli tecnico-economici}:\\
		\url{http://www.math.unipd.it/~tullio/IS-1/2015/Progetto/PD01b.html}.
	\end{itemize}
\subsubsection{Informativi}
	\begin{itemize}
		\item \textbf{Ingegneria del Software - Ian Sommerville - Ottava edizione}:
		\begin{itemize}
			\item Capitolo 4: Software management;
			\item Capitolo 5: Gestione di progetti.
		\end{itemize}
		\item \textbf{Slides del corso di Ingegneria del Software modulo A}:\\
		\url{http://www.math.unipd.it/~tullio/IS-1/2015/}.
	\end{itemize}
\subsection{Ciclo di Vita}
Il modello di ciclo di vita scelto è il Modello Incrementale. Esso prevede che:
\begin{itemize}
	\item L'Analisi e la Progettazione Architetturale costituiscano una base solida: i requisiti e l'architettura del sistema sono identificati e fissati definitivamente e sono essenziali per la pianificazione dei cicli incrementali;
	\item La Progettazione di Dettaglio, la codifica e i test vengono ripetuti più volte, permettendo sia il miglioramento di parti del sistema già esistenti che l'aggiunta di nuove funzionalità.
\end{itemize}
I vantaggi attesi dalla scelta di tale modello sono i seguenti:
\begin{itemize}
	\item I requisiti utenti sono realizzati in base all'importanza strategica, ovvero vengono soddisfatti per primi quelli di maggiore rilevanza;
	\item Ogni incremento può produrre valore e riduce il rischio di fallimento, in quando esso consolida ulteriormente la base ed eventualmente ne aumenta la qualità;
	\item Esecuzione più dettagliata dei test che quindi risulteranno essere maggiormente esaustivi;
	\item Rilasci multipli e in successione, che inizialmente punteranno a soddisfare i requisiti di primaria importanza mentre successivamente andranno ad adempiere a possibili funzionalità aggiuntive ed opzionali. Grazie a questi prototipi il proponente avrà la possibilità di fornire una prima valutazione del lavoro nel periodo di produzione.
\end{itemize}
\subsection{Scadenze}
Il gruppo \gruppo\ ha deciso di rispettare le seguenti scadenze:
\begin{itemize}
	\item \textbf{\RR}: 2016-02-16;
	\item \textbf{\RP}: 2016-04-18 presentandosi con Revisione di Progettazione massima;
	\item \textbf{\RQ}: 2016-05-23;
	\item \textbf{\RA}: 2016-06-17.
\end{itemize}
