\section{Introduzione}
\subsection{Scopo del documento}
Questo documento contiene la pianificazione delle attività che saranno svolte dai membri del gruppo \textit{\gruppo} per realizzare il
progetto \textit{quizzipedia}. In particolare, questo documento contiene:

	\begin{itemize}
		\item Analisi e trattamento dei rischi.
		\item Il preventivo delle risorse necessarie allo svolgimento del \textit{progetto\ped{G}}.
		\item il consuntivo delle attività finora risolte.
	\end{itemize}
	
\subsection{Scopo del prodotto}
Lo scopo del prodotto è di permettere la creazione e la gestione di questionari per sperimentare un metodo innovativo di valutazione delle competenze, in ambito lavorativo e in ambito scolastico. In particolare, il prodotto vuole verificare la possibilità di svolgere questionari prima e durante un corso di formazione, in modo da identificare le lacune dei candidati in modo anticipato rispetto la sede d'esame.
\subsection{Glossario}
Al fine di evitare ogni ambiguità di linguaggio e massimizzare la comprensione dei documenti, i termini tecnici, di dominio, gli acronimi e le parole che necessitano di essere chiarite, sono riportate nel documento \textit{Glossario\ped{G}}. Ogni occorrenza dei vocaboli presenti nel \textit{Glossario\ped{G}} è maracata da una \textit{G} maiuscola in pedice ed è scritta in corsivo (es: \textit{Esempio\ped{G}}).
\subsection{Riferimenti}
\subsubsection{Normativi}
	\begin{itemize}
		\item Norme di progetto: documento \textit{norme di progetto\ped{G}}.
		\item Capitolato d'appalto C5: \textit{Quizzipedia}, reperibile all'indirizzo:
		\item Regolamento di \textit{progetto didattico\ped{G}}:
		\item Vincoli di organigramma e dettagli tecnico-economici:
	\end{itemize}
\subsubsection{Informativi}
	\begin{itemize}
		\item Ingegneria del Software - Ian Sommerville - Ottava edizione:
		\begin{itemize}
			\item Capitolo 5: \textit{Gestione di progetti}.
		\end{itemize}
	\end{itemize}
\subsection{Ciclo di Vita}
\subsection{Scadenze}
