\newpage
\section{Pianificazione}
Per migliorare lo sviluppo del progetto, il team ha deciso di suddividere il carico di lavoro in sei periodi:
\begin{itemize}
	\item \textbf{\AdR (AR)};
	\item \textbf{Analisi dei Requisiti in dettaglio (AD)};
	\item \textbf{Progettazione architetturale (PA)};
	\item \textbf{Progettazione di dettaglio (PD)};
	\item \textbf{Codifica (CO)};
	\item \textbf{Verifica e Validazione (VV)}.
\end{itemize}

Per evidenziare le attività principali di un periodo, ad ognuno di essi viene associato un diagramma di \textit{Gantt\ped{G}}. 
Ogni attività, a propria volta, può essere suddivisa in sotto-attività e fare riferimento ad una o più risorse.
Una \textit{milestone\ped{G}} può essere esterna, coincidendo con le date di consegna dei documenti o con l'approvazione del lavoro svolto fino a quel moneto, o interna, ovvero un punto di revisione stabilito dal team. 
La rappresentazione temporale di queste nel diagramma di \textit{Gantt\ped{G}} avviene mediante linee nere le cui estremità sono delle frecce verso il basso. 
Oppure con dei rombi neri nel caso in cui la finestra di tempo che rappresentano sia di piccola durata. 
Le attività corrispondono a delle \textit{milestone\ped{G}} interne. 
Il momento in cui un periodo termina coincide con una \textit{milestone\ped{G}}.\\
I diagrammi di \textit{Gantt\ped{G}} permettono, grazie all'uso di frecce, di rappresentare le dipendenze tra le attività.
La conseguenza di un ritardo su un'attività è lo slittamento temporale di tutte le attività ad essa correlate . \\
Si è deciso di non riportare i diagrammi di \textit{PERT\ped{G}} in quanto poco leggibili data la moltitudine
di nodi presenti; quindi si è scelto di presentare i soli diagrammi di \textit{Gantt\ped{G}} riportando anche le
risorse impegnate per ciascuna attività.

\subsection{Suddivisione delle attività}

\subsubsection{Analisi dei Requisiti}
\textbf{Periodo:} dall'8 Dicembre 2015 al 22 Gennaio 2016.\\
Questo perdiodo inizia in concomitanza alla formazione del gruppo e termina con la consegna dei documenti per la \textbf{Revisione dei Requisiti}.\\ 
Questo periodo prevede di stilare i seguenti documenti:
\begin{itemize}
		\item \textbf{Norme di Progetto}: questo è il primo documento redatto in ordine cronologico poiché norma tutto l'operato del team rigaurdo la stesura dei documenti, delle comunicazioni, etc. ed è indipendente dal capitolato scelto. E' l’Amministratore di Progetto a redigere questo documento inserendovi le norme che il team dovrà seguire durante lo svolgimento di tutte le attività. I Verificatori certificheranno che tutte le norme siano state effettivamente osservate durante le diverse attività;
		\item \textbf{Studio di Fattibilità}: in questo documento vengono analizzati tutti i capitolati proposti. Per ognuno viene analizzato il dominio tecnologico e applicativo valutandone i fattori positivi e negativi. Risulta essere un'attività	critica perché definisce il progetto sul quale il gruppo andrà a lavorare e blocca la stesura del documento di Analisi dei Requisiti;
		\item \textbf{Analisi dei Requisiti}: redatto dagli Analisti, è l'analisi approfondita del capitolato scelto con lo Studio di Fattibilità;
		\item \textbf{Piano di Progetto}: redatto dal Responsabile di Progetto, individua tutte le attività necessarie allo svolgimento del progetto e le assegna alle risorse disponibili distribuendo il carico di lavoro in maniera uniforme.
		La priorità di questo documento è alta poiché vincola tutte attività svolte dal team;
		\item \textbf{Piano di Qualifica}: steso dal Verificatore, definisce come devono essere effettuate le verifiche al fine di consegnare un prodotto di qualità;
		\item \textbf{Glossario}: scritto in maniera incrementale dai redattori dei diversi documenti e contiene la spiegazione di alcuni termini utilizzati nei vari documenti, al fine di eliminare ogni possibile ambiguità di significato;
		\item \textbf{Lettera di presentazione}: documento che dichiara l’interesse del gruppo a partecipare alla gara d’appalto.
\end{itemize}
Ogni documento sopraccitato oltre che stilato verrà anche verificato; la verifica dei presenti documenti è considerata un'attività critica.
In questa fase i ruoli maggiormente interessati sono quelli di Amministratore, Responsabile, Analista e Verificatore. 
\begin{sidewaysfigure}
	\centering
	\includegraphics[keepaspectratio = true, width=23cm]{immagini/PdP_AnalisiDeiRequisitiGantt.png}
	\caption{Diagramma di Gantt relativo al periodo di Analisi dei Requisiti.}\label{etichetta}
\end{sidewaysfigure}
\newpage

\subsubsection{Analisi dei Requisiti in Dettaglio}
\textbf{Periodo:} dal 23 Gennaio 2016 al 22 Febbraio 2016. \\
Il periodo di Analisi dei Requisiti in Dettaglio inizia la consegna dei documenti per la Revisione dei Requisiti e termina con l'inizio del periodo successivo, quello della Progettazione Architturale. Il termine fissato corrisponde ad una \textit{milestone\ped{G}} interna. \\
In questo periodo il gruppo mira a consolidare ed ampliare i requisiti richiesti dal sistema e migliorare il documento di \AdR attuando le correzioni in base all'esito della Revisione dei Requisiti.
Vengono inoltre corretti e verificati anche gli altri documenti. 
 
\begin{center}
	\includegraphics[keepaspectratio = true, width=16cm]{immagini/PdP_AnalisiDeiRequisitiInDettaglioGantt.png}
\end{center}
\begin{figure}[h]
	\caption{Diagramma di Gantt relativo al periodo di Analisi dei Requisiti in Dettaglio.}\label{etichetta}
\end{figure}

\subsubsection{Progettazione Architetturale}
\textbf{Periodo:} dal 23 Febbrario 2016 al 20 Marzo 2016. \\
Questo periodo, di Progettazione Architetturale, inizia dopo il periodo di Analisi dei Requisiti in Dettaglio e si conclude con una \textit{milestone\ped{G}} interna di Revisione di Progettazione minima. L'obbiettivo di questo periodo è la stesura della progettazione ad alto livello del sistema. \\
Questo periodo prevede di svolgere le seguenti attività:
\begin{itemize}
	\item Redigire il documento di \textbf{Specifica Tecnica}: il \Prog deve descrivere le scelte progettuali di alto livello effettuale, i design patter scelti per la realizzazione del prodotto e l'archittura generale del software. Inoltre deve viene effettuato il tracciamento dei requisiti.  
	\item Incrementare i documenti di \textbf{\NdP},\textbf{\PdP} e \textbf{\PdQ}.
	\item Verifica di tutti i documenti sopraccitati.
\end{itemize}
\begin{center}
	\includegraphics[keepaspectratio = true, width=15cm]{immagini/PdP_ProgettazioneArchitetturaleGantt.png}
\end{center}
\begin{figure}[h]
	\caption{Diagramma di Gantt relativo al periodo di Progettazione Architetturale.}\label{etichetta}
\end{figure}

\subsubsection{Progettazione di Dettaglio}
\textbf{Periodo:} dal 21 Marzo all'11 Aprile 2016. \\
Questo periodo, di Progettazione di Dettaglio, inizia dopo il periodo di Progettazione Architetturale e si conclude con la consegna dei documenti per la Revisione di Progettazione. L'obbiettivo di questo periodo è la stesura, in modo dettagliato, dell'intero sistema, specificando in modo approfondito il comportamento e l'iterazionetra tra i vari componenti. \\
Prevede di svolgere le seguenti attività:
\begin{itemize}
	\item Redigire il documento di \textbf{\DDP}: il \Prog deve descrivere il comportamento e le iterazioni tra i vari componenti del sistema basandosi sul documento di \ST.  
	\item Incrementare i documenti di \textbf{\NdP},\textbf{\PdP}, \textbf{\PdQ}, \textbf{\ST} e \textbf{Glossario}.
	\item Verifica di tutti i documenti sopraccitati.
\end{itemize}
\begin{center}
	\includegraphics[keepaspectratio = true, width=15cm]{immagini/PdP_ProgettazioneDiDettaglioGantt.png}
\end{center}
\begin{figure}[h]
	\caption{Diagramma di Gantt relativo al periodo di Progettazione di Dettaglio.}\label{etichetta}
\end{figure}

\subsubsection{Codifica}
\textbf{Periodo:} dal 19 Aprile 2015 al 16 Maggio 2015. \\
Questo periodo, di Codifica, inizia dopo il periodo di Progettazione di Dettaglio e si conclude con la consegna del prodotto alla Revisione di Qualità. L'obbiettivo in questo periodo è di consegnare un prodotto qualificato e prevede di svolgere le seguenti attività:
\begin{itemize}
	\item   
	\item Incrementare i documenti di \textbf{\NdP},\textbf{\PdP}, \textbf{\PdQ} e \textbf{Glossario}.
	\item Verifica di tutti i documenti sopraccitati.
\end{itemize}
\begin{center}
	\includegraphics[keepaspectratio = true, width=15cm]{immagini/PdP_CodificaGantt.png}
\end{center}
\begin{figure}[h]
	\caption{Diagramma di Gantt relativo al periodo di Codifica.}\label{etichetta}
\end{figure}

\subsubsection{Verifica e Validazione}
\textbf{Periodo:} dal 24 Maggio 2016 al 10 Giugno 2016. \\
Questo periodo, di Verifica e Validazione, inizia dopo il periodo di Codifica e si conclude con la consegna del prodotto alla Revisione di Accettazione. In questo periodo vengono effettuati tutti i test necessari per garantire che il prodotto soddisfi tutti i requisiti dell'\AdR.  
Le attività sono di:
\begin{itemize}
	\item   
	\item Incrementare i documenti di \textbf{\MU},\textbf{\NdP},\textbf{\PdP}, \textbf{\PdQ} e \textbf{Glossario}.
	\item Verifica di tutti i documenti sopraccitati.
\end{itemize}
\begin{center}
	\includegraphics[keepaspectratio = true, width=15cm]{immagini/PdP_VerificaEValidazioneGantt.png}
\end{center}
\begin{figure}[h]
	\caption{Diagramma di Gantt relativo al periodo di Verifica e Validazione.}\label{etichetta}
\end{figure}