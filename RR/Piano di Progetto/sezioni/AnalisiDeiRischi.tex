\newpage
\section{Analisi dei rischi}

	In questa sezione del documento vengono elencati tutti i possibili rischi che potrebbero colpire il gruppo \gruppo{} nella realizzazione del prodotto \progetto. Per gestire i rischi è stata attuata la seguente procedura che prevede: 
	
	\begin{itemize}
		
		\item \textbf{Identificazione dei Rischi}: trovare i potenziali rischi che possono presentarsi durante lo sviluppo del progetto e studiarne la natura. Possono essere di tre tipi:
			\begin{itemize}
				\item \textbf{Progetto}: relativi a pianificazione, strumenti ed risorse;
				\item \textbf{Prodotto}: relativi a conformità e aspettative del committente;
				\item \textbf{Business}: relativi a costi e concorrenza;
			\end{itemize} 
				
		\item \textbf{Analisi dei Rischi}: studiare per ogni rischio le: 	
			\begin{itemize}
				\item \textbf{Probabilità di occorrenza};
				\item \textbf{Conseguenze}: comprendere che peso hanno sul progetto e quindi capirne le criticità; 			
			\end{itemize}
		
		\item \textbf{Pianificazione di controllo e mitigazione}: istituire metodi di controllo per i rischi, così da poterli evitare. Facendo:
			\begin{itemize}
				\item \textbf{Verifica costante del livello di rischio};
				\item \textbf{Riconoscimento e trattamento};
			\end{itemize}
			
		\item \textbf{Attuazione nel periodo}: viene progressivamente descritto se il rischio si è verificato, in che modo il gruppo ha reagito e cosa ha comportato. 
		
	\end{itemize}
	
	Un rischio viene identificato: 
	
	\begin{itemize}
		\item A livello tecnologico; 
		\item A livello del personale;
		\item A livello organizzativo;
		\item A livello dei requisiti;
		\item A livello di valutazione dei costi.
	\end{itemize}
	
	Per ogni rischio viene stilato un elenco di informazioni necessario per comprenderne la natura. Esso comprende: 
	
	\begin{itemize}
		\item Nome; 
		\item Descrizione;
		\item Probabilità di occorrenza;
		\item Grado di pericolosità;
		\item Riconoscimento;
		\item Trattazione;
		\item Attuazione nel periodo.
	\end{itemize}

	\subsection{A livello tecnologico}
	
		\subsubsection{Tecnologie adottate}
		\begin{itemize}
			\item \textbf{Descrizione}: il gruppo è interessato allo studio delle tecnologie web ed ha una sufficiente preparazione in merito. Non è da escludere però un'eventuale difficoltà nell'utilizzarle e nel padroneggiare gli strumenti che ne fanno uso;
			\item \textbf{Probabilità di occorrenza}: media;
			\item \textbf{Grado di pericolosità}: alto;
			\item \textbf{Riconoscimento}: il \textit{\Res} deve verificare il grado di preparazione di ogni membro del gruppo relativo alle tecnologie utilizzate;	
			\item \textbf{Trattamento}: ogni componente del gruppo deve, in maniera autonoma, studiare tutte le tecnologie necessarie per la realizzazione del prodotto facendo uso dei documenti forniti dall'\textit{\Amm};
			\item \textbf{Attuazione nel periodo}:
				\begin{itemize}
					\item \textbf{Analisi dei requisiti}: il rischio non si è ancora verificato dato che non sono state usate tali tecnologie in questo periodo di sviluppo.
				\end{itemize}
		\end{itemize}
		
		\subsubsection{Rotture Hardware}
		\begin{itemize}
			\item \textbf{Descrizione}: il gruppo è dotato di computer portatili di tipo non professionale, il rischio di rottura di uno di questi è possibile ed è da tenere in considerazione. Un altro rischio di fallibilità hardware è quello del server usato per ospitare \textit{DocumentsDB\ped{G}}, un malfunzionamento su tale macchina mette a rischio il lavoro dell'intero gruppo;
			\item \textbf{Probabilità di occorrenza}: bassa;
			\item \textbf{Grado di pericolosità}: medio;
			\item \textbf{Riconoscimento}: ogni membro del gruppo dove porre un'elevata attenzione verso i propri strumenti hardware;	
			\item \textbf{Trattamento}: ogni membro del gruppo possiede un altro dispositivo per poter continuare il lavoro in caso di malfunzionamenti o rotture hardware. Per il server che ospita \textit{DocumentsDB\ped{G}} è previsto un sistema di backup automatico e in caso di malfunzionamenti sarà compito dell'\textit{\Amm} riportare tale macchina in uno stato funzionante nel minor tempo possibile;
			\item \textbf{Attuazione nel periodo}:
			\begin{itemize}
				\item \textbf{Analisi dei requisiti}: questo rischio non si è mai verificato.
			\end{itemize}
		\end{itemize}	
	
	\subsection{A livello del personale}
	
		\subsubsection{Problemi tra componenti del team}
		\begin{itemize}
			\item \textbf{Descrizione}: per ogni componente del gruppo è la prima esperienza di lavoro in un gruppo di grandi dimensioni. Tale fattore potrebbe causare problemi di collaborazione causando squilibri negli equilibri interni provocando dei ritardi nei lavori e creare un clima non proficuo; 
			\item \textbf{Probabilità di occorrenza}: bassa;
			\item \textbf{Grado di pericolosità}: alto;
			\item \textbf{Riconoscimento}: la comunicazione costante con il \textit{\Res} può far sì che quest'ultimo monitori ogni tipo di problematica sul nascere;  
			\item \textbf{Trattamento}: il \textit{\Res} provvederà, in caso di contrasti tra membri del gruppo, ad affidare alle persone coinvolte attività che non li faccia collaborare assieme. Ripristinando così il clima corretto per svolgere in modo idoneo le attività nei tempi stabiliti;
			\item \textbf{Attuazione nel periodo}:
			\begin{itemize}
				\item \textbf{Analisi dei requisiti}: questo rischio non si è mai verificato.
			\end{itemize}
		\end{itemize}
		
		\subsubsection{Problemi dei componenti del team}
		\begin{itemize}
			\item \textbf{Descrizione}: ogni membro del gruppo ha impegni e necessità proprie. Risulta inevitabile il verificarsi di problemi organizzativi in seguito a sovrapposizioni di tali impegni;
			\item \textbf{Probabilità di occorrenza}: bassa;
			\item \textbf{Grado di pericolosità}: alto;
			\item \textbf{Riconoscimento}: per creare un calendario sincronizzato e condiviso tra i membri del gruppo è necessario che vengano notificati al \textit{\Res} in maniera preventiva e tempestiva gli impegni di ognuno. Grazie a questa pratica è possibile scongiurare tale rischio;
			\item \textbf{Trattamento}: ad ogni impegno notificato il \textit{\Res} si prenderà la responsabilità di rifare, in tempi brevi, una parte di pianificazione per coprire l'assenza creatasi;
			\item \textbf{Attuazione nel periodo}:
			\begin{itemize}
				\item \textbf{Analisi dei requisiti}: questo rischio non si è mai verificato, se non nelle prime battute quando un membro del team ha avuto un'influenza per un paio di settimane. Il lavoro è stato, in ogni caso, portato a termine grazie ad una pianificazione alternativa.
			\end{itemize}
		\end{itemize}
		
		\subsubsection{Problemi di inesperienza}
		\begin{itemize}
			\item \textbf{Descrizione}: per ogni componente del gruppo è la prima esperienza con questo metodo di lavoro. Sono richieste capacità di pianificazione e di analisi che il gruppo non possiede a causa dell'inesperienza. \'E inoltre richiesto l'utilizzo di strumenti mai utilizzati prima. E alcune tecnologie richiedono del tempo per essere apprese. 
			\item \textbf{Probabilità di occorrenza}: alta;
			\item \textbf{Grado di pericolosità}: alto;
			\item \textbf{Riconoscimento}: quando nasce la necessità di utilizzare una nuova tecnologia, questa deve essere riferita al \textit{\Res} per far si che la integri nelle armonie di sviluppo del progetto e del gruppo; 
			\item \textbf{Trattamento}: ogni  membro del gruppo si impegna a studiare il materiale necessario per l'utilizzo di tecnologie e strumenti richiesti durante lo svolgimento del progetto. Nel caso in cui questo non fosse sufficiente il \textit{\Res} dovrà preparare un piano di studi per sopperire ogni tipo di lacuna;	
			\item \textbf{Attuazione nel periodo}:
			\begin{itemize}
				\item \textbf{Analisi dei requisiti}: il rischio si è verificato soprattutto nella prima fase, facendo impiegare più tempo del previsto nell'apprendimento dei software da utilizzare. 
			\end{itemize}
		\end{itemize}
	
	\subsection{A livello organizzativo}
		
		\subsubsection{Pianificazione errata}
		\begin{itemize}
			\item \textbf{Descrizione}: durante la pianificazione è possibile che i tempi, per l’esecuzione di alcune attività, vengano calcolati in modo errato;
			\item \textbf{Probabilità di occorrenza}: media;
			\item \textbf{Grado di pericolosità}: alta;
			\item \textbf{Riconoscimento}: la caratteristica dinamica del rischio impone che si debba controllare lo stato delle attività nel programma di \textit{project management\ped{G}} periodicamente, in modo da verificare eventuali ritardi nello sviluppo delle attività;	
			\item \textbf{Trattamento}: si è deciso di prevedere per ogni attività con maggior criticità un periodo maggiore di quanto normalmente richiesto, in modo tale che un eventuale ritardo non influenzi la durata totale del progetto;
			\item \textbf{Attuazione nel periodo}:
			\begin{itemize}
				\item \textbf{Analisi dei requisiti}:questo rischio non si è mai verificato.
			\end{itemize}
		\end{itemize}
	
	\subsection{A livello dei requisiti}
	
		\subsubsection{Incomprensioni e scelte non congrue}
		\begin{itemize}
			\item \textbf{Descrizione}: è possibile che alcuni requisiti individuati dal capitolato vengano interpretati in modo erroneo o in maniera incompleta da parte degli \textit{\Anas} rispetto alle aspettative del proponente. \'E inoltre possibile che alcuni requisiti vengano tolti, aggiunti o modificati durante il corso del progetto;
			\item \textbf{Probabilità di occorrenza}: media;
			\item \textbf{Grado di pericolosità}: alto;
			\item \textbf{Riconoscimento}: per ridurre al minimo la probabilità che si verifichi un errore nella fase di \textit{\AdR} si effettueranno, durante tale fase, degli incontri con il proponente in modo da assicurare la totale concordanza sulle necessità del prodotto. Inoltre, i documenti verranno consegnati	e valutati dal committente ad ogni revisione;
			\item \textbf{Trattamento}: sarà necessario effettuare degli incontri con il proponente in modo da poter definire con chiarezza ogni requisito necessario al corretto sviluppo del progetto. Sarà inoltre indispensabile correggere eventuali errori o imprecisioni indicati dal committente all'esito di ogni revisione;
			\item \textbf{Attuazione nel periodo}:
			\begin{itemize}
				\item \textbf{Analisi dei requisiti}: questo rischio non si è mai verificato.
			\end{itemize}
		\end{itemize}
	
	\subsection{A livello di valutazione dei costi}
	
		\subsubsection{Errore nelle previsioni}
		\begin{itemize} 
			\item \textbf{Descrizione}: è possibile che i tempi delle attività pianificate per lo svolgimento del progetto siano sovrastimate o sottostimate. Un valutazione errata di queste può comportare una variazione sul costo preventivo presentato;
			\item \textbf{Probabilità di occorrenza}: media;
			\item \textbf{Grado di pericolosità}: medio;
			\item \textbf{Riconoscimento}: quando un'attività occupa più tempo di quello previsto significa che è stata sottostimata. Quando invece ne occupa considerevolmente meno significa che è stata sovrastimata. \'E necessario che il \textit{\Res} monitori con grande attenzione il programma di \textit{project management\ped{G}} e che faccia tempestive modifiche alla pianificazione e al rendiconto dei costi;
			\item \textbf{Trattamento}: è necessario che ogni membro del gruppo rispetti i tempi delle attività assegnatogli;
			\item \textbf{Attuazione nel periodo}:
			\begin{itemize}
				\item \textbf{Analisi dei requisiti}: questo rischio non si è mai verificato.
			\end{itemize}
		\end{itemize}