\newpage
\section{Analisi dei rischi}

	In questa sezione del documento vengono elencati tutti i possibili rischi che potrebbero colpire il gruppo \gruppo nella realizzazione del prodotto \progetto. Per gestire i rischi è stata attuata la seguente procedura che prevede: 
	
	\begin{itemize}
		
		\item \textbf{L'Identificazione dei Rischi}: trovare i potenziali rischi che possono presentarsi durante lo sviluppo del progetto e studiarne la natura. Possono di essere di tre tipi:
			\begin{itemize}
				\item \textbf{Progetto}: relativi a pianificazione, strumenti ed alle risorse;
				\item \textbf{Prodotto}: relativi a conformità alle aspettative del committente;
				\item \textbf{Business}: relativi a costi e concorrenza;
			\end{itemize} 
				
		\item \textbf{L'Analisi dei Rischi}: studiare per ogni rischio le: 	
			\begin{itemize}
				\item \textbf{Probabilità di occorrenza};
				\item \textbf{Conseguenze}: comprendere che peso hanno sul progetto e quindi capire le criticità; 			
			\end{itemize}
		
		\item \textbf{La Pianificazione di controllo e mitigazione}: istituire metodi di controllo per i rischi, cosi da poterli evitare. Facendo:
			\begin{itemize}
				\item \textbf{Verifica costante del livello di rischio};
				\item \textbf{Riconoscimento e trattamento};
			\end{itemize}
			
		\item \textbf{L'Attuazione nel periodo}: viene progressivamente descritto se il rischio si è verificato, in che modo il gruppo ha reagito e cosa ha comportato. 
		
	\end{itemize}
	
	Un rischio viene identificato: 
	
	\begin{itemize}
		\item A livello tecnologico; 
		\item A livello del personale;
		\item A livello organizzativo;
		\item A livello dei requisiti;
		\item A livello di valutazione dei costi.
	\end{itemize}
	
	Per ogni rischio viene stilato un elenco di informazioni necessario per comprenderne la natura. Esso comprende: 
	
	\begin{itemize}
		\item Nome; 
		\item Descrizione;
		\item Probabilità di occorrenza;
		\item Grado di pericolosità;
		\item Riconoscimento;
		\item Trattazione;
		\item Attuazione nel periodo.
	\end{itemize}

	\subsection{A livello tecnologico}
	
		\subsubsection{Tecnologie adottate}
		\begin{itemize}
			\item \textbf{Descrizione}: il gruppo è interessato allo studio delle tecnologie web ed ha una sufficiente preparazione in merito. Non è da escludere però un'eventuale difficoltà nell'utilizzarle e nel padroneggiare gli strumenti che ne fanno uso;
			\item \textbf{Probabilità di occorrenza}: Media;
			\item \textbf{Grado di pericolosità}:Alto;
			\item \textbf{Riconoscimento}: il \textit{\Res} deve verificare il grado di preparazione di ogni membro del gruppo relativo alle tecnologie utilizzate;	
			\item \textbf{Trattamento}: ogni componente del gruppo deve, in maniera autonoma, studiare tutte le tecnologie necessarie per la realizzazione del prodotto facendo uso dei documenti forniti dall'\textit{\Amm};
			\item \textbf{Attuazione nel periodo}:
				\begin{itemize}
					\item \textbf{Analisi dei requisiti}: il rischio non si è ancora verificato dato che non sono state usate tali tecnologie in questo periodo di sviluppo.
				\end{itemize}
		\end{itemize}
		
		\subsubsection{Rotture Hardware}
		\begin{itemize}
			\item \textbf{Descrizione}: il gruppo è dotato di computer portatili di tipo non professionale, il rischio di rottura di uno di questi è possibile ed è da tenere in considerazione. Un altro rischio di fallibilità hardware è quello del server usato per ospitare \textit{DocumentsDB\ped{G}}, un malfunzionamento su tale macchina mette a rischio il lavoro dell'intero gruppo;
			\item \textbf{Probabilità di occorrenza}:Bassa;
			\item \textbf{Grado di pericolosità}:Medio;
			\item \textbf{Riconoscimento}: ogni membro del gruppo dovrà porre un'elevata attenzione verso i propri strumenti hardware;	
			\item \textbf{Trattamento}: ogni membro del gruppo possiede un altro dispositivo per poter continuare il lavoro in caso di malfunzionamenti o rotture hardware. Per il server che ospita \textit{DocumentsDB\ped{G}} è previsto un sistema di backup automatico e in caso di malfunzionamenti sarà compito dell'\Amm riportare tale macchina in uno stato funzionante nel minor tempo possibile;
			\item \textbf{Attuazione nel periodo}:
			\begin{itemize}
				\item \textbf{Analisi dei requisiti}: questo rischio non si è mai verificato.
			\end{itemize}
		\end{itemize}	
	
	\subsection{A livello del personale}
	
		\subsubsection{Problemi tra componenti del team}
		\begin{itemize}
			\item \textbf{Descrizione}:;
			\item \textbf{Probabilità di occorrenza}:Bassa;
			\item \textbf{Grado di pericolosità}:Alto;
			\item \textbf{Riconoscimento}:	
			\item \textbf{Trattamento}:
			\item \textbf{Attuazione nel periodo}:
			\begin{itemize}
				\item \textbf{Analisi dei requisiti}:.
			\end{itemize}
		\end{itemize}
		
		\subsubsection{Problemi dei componenti del team}
		\begin{itemize}
			\item \textbf{Descrizione}:;
			\item \textbf{Probabilità di occorrenza}:Bassa;
			\item \textbf{Grado di pericolosità}:Alto;
			\item \textbf{Riconoscimento}:;
			\item \textbf{Trattamento}:;
			\item \textbf{Attuazione nel periodo}:
			\begin{itemize}
				\item \textbf{Analisi dei requisiti}:.
			\end{itemize}
		\end{itemize}
		
		\subsubsection{Problemi di inesperienza}
		\begin{itemize}
			\item \textbf{Descrizione}:;
			\item \textbf{Probabilità di occorrenza}:Alta;
			\item \textbf{Grado di pericolosità}:Alto;
			\item \textbf{Riconoscimento}:;	
			\item \textbf{Trattamento}:;
			\item \textbf{Attuazione nel periodo}:
			\begin{itemize}
				\item \textbf{Analisi dei requisiti}:.
			\end{itemize}
		\end{itemize}
	
	\subsection{A livello organizzativo}
		
		\subsubsection{Pinificazione errata}
		\begin{itemize}
			\item \textbf{Descrizione}:;
			\item \textbf{Probabilità di occorrenza}:Media;
			\item \textbf{Grado di pericolosità}:Alta;
			\item \textbf{Riconoscimento}:;	
			\item \textbf{Trattamento}:;
			\item \textbf{Attuazione nel periodo}:
			\begin{itemize}
				\item \textbf{Analisi dei requisiti}:.
			\end{itemize}
		\end{itemize}
	
	\subsection{A livello dei requisiti}
	
		\subsubsection{Incomprensioni e scelte non congrue}
		\begin{itemize}
			\item \textbf{Descrizione}:;
			\item \textbf{Probabilità di occorrenza}:Media;
			\item \textbf{Grado di pericolosità}:Alto;
			\item \textbf{Riconoscimento}:;	
			\item \textbf{Trattamento}:;
			\item \textbf{Attuazione nel periodo}:
			\begin{itemize}
				\item \textbf{Analisi dei requisiti}:.
			\end{itemize}
		\end{itemize}
	
	\subsection{A livello di valutazione dei costi}
	
		\subsubsection{Errore nelle previsioni}
		\begin{itemize} 
			\item \textbf{Descrizione}:;
			\item \textbf{Probabilità di occorrenza}:Media;
			\item \textbf{Grado di pericolosità}:Medio;
			\item \textbf{Riconoscimento}:;
			\item \textbf{Trattamento}:;
			\item \textbf{Attuazione nel periodo}:
			\begin{itemize}
				\item \textbf{Analisi dei requisiti}:.
			\end{itemize}
		\end{itemize}