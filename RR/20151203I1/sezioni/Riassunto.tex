\section{Riassunto della riunione}
\subsection{Descrizione}

Durante la riunione sono stati discussi i seguenti argomenti: il nome e il logo del gruppo, il capitolato da scegliere, le modalità con le quali verranno distribuiti i ruoli ai vari componenti del \textit{team\ped{G}} e gli strumenti che verranno utilizzati per l'organizzazione e il versionamento.

\subsection{Decisioni prese}
\begin{itemize}
\item Scelta del nome del gruppo: dopo varie proposte è stato scelto all'unaminità il seguente nome: TheFellowshipOfTheCode;
\item Si è scelto il capitolato C5, QuizziPedia proposto dall'azienda Zucchetti S.P.A.;
\item Si è discusso su come distribuire i ruoli ai vari componenti del \textit{team\ped{G}}. Questa decisione è stata riportata ed è consultabile nel file \textsl{Ruoli.txt} in \textit{Google Drive\ped{G}};
\item Sono stati definiti gli strumenti e i servizi che si andranno ad utilizzare nel corso del progetto. Come strumento di ticketing abbiamo scelto di appoggiarci al servizio offerto da \textit{Zoho\ped{G}} in quanto gratuito e ricco di strumenti utili per l'organizzazione del gruppo (tra cui la creazione automatica dei diagrammi Gantt). Per quanto riguarda il versionamento, abbiamo scelto di utilizzare \textit{GitHub\ped{G}} perché offre un ottimo servizio gratuitamente. Come software per generare i diagrammi \textit{UML\ped{G}} abbiamo deciso di utilizzare \textit{Astah\ped{G}}, in quanto è risultato molto intuitivo e disponibile gratuitamente su tutte le piattaforme.
\end{itemize}
