\section{Riassunto della riunione}
\subsection{Descrizione}

Durante la riunione sono stati discussi, in questo ordine, i seguenti argomenti: il nome del gruppo, il capitolato da scegliere, le modalità con le quali verranno distribuiti i ruoli ai vari componenti del team e gli strumenti che verranno utilizzati.

\subsection{Decisioni effettuate}
\begin{itemize}
\item Scelta del nome del gruppo: dopo varie proposte è stato scelto all'unaminità il seguente nome: \textbf{The Fellowship Of The Code};
\item Si è deciso il capitolato C5, \textbf{QuizziPedia} proposto dall'azienda Zucchetti S.p.A.;
\item Si è discusso su come distribuire i ruoli ai vari componenti del team. Questa decisione è stata riportata ed è consultabile nel file \textsl{Ruoli.txt};
\item Sono stati definiti gli strumenti e i servizi che si andranno ad utilizzare nel corso del progetto. Come strumento di ticketing abbiamo scelto di appoggiarci al servizio offerto da \textsl{Zoho} in quanto gratuito e con molti strumenti messi a disposizone (tra cui la creazione automatica dei diagrammi Gantt). Per quanto riguarda il versionamento, abbiamo scelto di utilizzare \textsl{GitHub} perchè offre un ottimo servizio gratuitamente. Come software per generare i diagrammi UML abbiamo deciso di utilizzare \textsl{Micosoft Visio}, in quanto è risultato molto intuitivo e in più viene offerto gratuitamente dall'Università degli Studi di Padova.
\end{itemize}