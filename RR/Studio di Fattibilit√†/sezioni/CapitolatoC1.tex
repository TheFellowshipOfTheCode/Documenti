\section{Capitolato C1}
\subsection{Descrizione}
L'obiettivo del progetto è di implementare un modello di database NoSQL di tipo Key‐value utilizzando il modello ad attori(G).
Tale progetto richiede la definizione di mappe, attori ed operazioni basilari. 
\\Ogni mappa deve essere caratterizzata da:
\begin{itemize}
\item
un numero arbitrario di coppie chiave / valore. 
\item 
un nome che la contraddistingua. 
\item
un parametro di configurazione, che determinerà il numero di attori da utilizzare per l'implementazione della mappa e ogni quante coppie, la mappa debba essere suddivisa fra più attori.
\end{itemize}
Ogni attore deve appartenere al seguente insieme di attori:
\begin{itemize}
\item
\textbf{STOREKEEPER}: incaricati di mantenere fisicamente al proprio interno le coppie chiave / valore.
\item
\textbf{STOREFINDER}: incaricati di ricevere le richieste dall’esterno e instradarle ai 
rispettivi \textbf{STOREKEEPER}, in modo da soddisfarle. 
\item
\textbf{WAREHOUSEMEN}: incaricati ad interfacciarsi con gli \textbf{STOREKEEPER} e persistere su disco le rispettive mappe.
\end{itemize}

Le operazioni da implementare sul database devono essere le seguenti: 
\begin{itemize}
\item
Inserimento
\item
Cancellazione 
\item
Aggiornamento 
\end{itemize}

L'implementazione del suddetto sistema ad attori deve prendere il nome di ACTORDB. In particolare, per l'implementazione degli attori viene richiesto l'utilizzo della libreria Akka(G).
Il linguaggio da utilizzare può essere scelto fra \textbf{Scala(G)} o \textbf{Java(G)}. 
Viene richiesto la definizione di un \textbf{domain specific language (DSL)} da utilizzare da riga di comando per poter interagire con il database.

\subsection{Dominio Applicativo}
Questa tipologia di database da implementare presenta caratteristiche tecniche ottimali, come: la semplicità strutturale, l'affidabilità, le performance elevate e costanti.
Ciò determina la preferenza nell'utilizzo di questi DB per la gestione di applicazioni di Big Data.
Quindi,a questa tipologia di DB possiamo attribuire svariati ambiti di utilizzo, come ad esempio:
\begin{itemize}
\item
\textbf{Big Data} 
\item
\textbf{Social & Mobile Infrastructure}
\item
\textbf{Content Management & Delivery}
\item
\textbf{Data Hub}
\item
\textbf{User Data Management} 
\end{itemize}

\subsection{Dominio Tecnologico}
Il capitolato prevedeva l'utilizzo delle seguenti tecnologie: 
\begin{itemize}
\item
\textbf{Scala(G)} o \textbf{Java(G)}:linguaggi di programmazione consigliati
\item
\textbf{Akka(G)}: libreria consigliata per l'implementazione degli attori, in quanto fornisce un'implementazione del modello ad attori su JVM.
\end{itemize}

\subsection{Criticità}
Le criticità trovate sono:
\begin{itemize}
\item
Definizione di un \textbf{domain specific language(DSL)} da utilizzare da riga di comando per interagire con il database
\end{itemize}

\subsection{Valutazione Finale}
Nonostante l'interesse per questo capitolato perché ritenuto semplice e ben strutturato, alla fine è stato scartato perché:
\begin{itemize}
\item
Non presentava alcun stimolo positivo da parte di tutti i componenti del gruppo verso le tecnologie richieste.
\item
Il gruppo preferiva sviluppare un progetto presentato da un azienda anziché uno presentato da un membro interno all'università.
\end{itemize}