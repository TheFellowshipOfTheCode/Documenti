\newpage
\section{Capitolato C6}
\subsection{Descrizione}
L’obbiettivo del capitolato proposto da MIVOQ S.R.L. è quello di realizzare, su tablet e smartphone, un’applicazione innovativa che utilizzi la sintesi vocale da testo, normalmente non presenti su questo tipo di dispositivi. In particolare sarà possibile applicare effetti e stili alle voci e verrà data all’utente la possibilità di creare la propria voce sintetica personale.

\subsection{Dominio Applicativo}
La sintesi vocale è una tecnologia che permette la conversione di un qualsiasi file di testo in un file sonoro. Negli ultimi anni si è assistito ad un rapido diffondersi di questa tecnologia in numerosi ambiti: dai navigatori satellitari agli annunci nei mezzi pubblici agli assistenti vocali nei più moderni sistemi operativi, centralini telefonici e lettori di messaggi.

\subsection{Dominio Tecnologico}
Il progetto prevede che l’applicazione sfrutti appieno le potenzialità offerte dal motore di sintesi vocale opensource “Flexible and Adaptive Text To Speech”, che espone molteplici funzionalità reperibili mediante interfaccia HTTP. Non vengono invece imposti vincoli tecnologici su progettazione e implementazione dell’applicazione.

\subsection{Criticità}
Nell'analisi del capitolato, il team ha rilevato le seguenti criticità:
\begin{itemize}
\item Uso del motore di sintesi FA-TTS completamente sconosciuto.
\item Nessuna competenza del team sulla sintesi vocale.
\item Utilizzo di framework sconosciuti.
\end{itemize}

\subsection{Valutazione Finale}
Per quanto interessante e innovativo possa essere la sintesi vocale, il team ritiene che le scarse competenze sull’argomento possano essere un vincolo sulla realizzazione di questo capitolato. Inoltre, il gruppo ritiene che lo studio del motore di sintesi FA-TTS non dia delle competenze tecniche spendibili in un futuro ambito lavorativo d’interesse.
La parte dello sviluppo su piattaforma mobile aveva attirato in parte l’attenzione di alcuni membri del gruppo, ma non riteniamo sia sufficiente per la scelta del capitolato.
