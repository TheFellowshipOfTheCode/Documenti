\section{Capitolato C3}
\subsection{Descrizione}
Il capitolato proposto dall'azienda Zero12 riguarda la realizzazione di un algoritmo predittivo in grado di analizzare i dati provenienti da \textsf{oggetti} inseriti in diversi contesti. L'algoritmo, una volta analizzati i dati, provvederà a fornire delle previsioni su possibili guasti, interazioni con nuovi utenti ed identificare dei pattern di comportamento degli utenti per prevedere le azioni degli stessi su altri oggetti.

\subsection{Dominio Applicativo}
Come suggerito dal capitolato, il dominio applicativo di tale software è davvero ampio: domotica, robotica, avionica, industria automobilistica, biomedicale, monitoraggio in ambito industriale, telemetria, reti wireless di sensori, sorveglianza e security, smart grid, smart city, sistemi embedded, telematica e telecontrollo.

\subsection{Dominio Tecnologico}
Per realizzare l'oggetto del capitolato vengono richieste al gruppo conoscenze legate all'ambito web:
\begin{itemize}
\item \textbf{Amazon Web Services} e \textbf{Database NoSQL} (MongoDB oppure OrientDB);
\item \textbf{Java o Scala} come linguaggio di programmazione;
\item \textbf{Play Framework} come piattaforma di sviluppo;
\item \textbf{HTML5, CSS3 e Javascript} per la parte di visualizzazione e comportamentale.
\end{itemize}

\subsection{Criticità}
Sono state individuate le seguenti criticità:
\begin{itemize}
\item L'individuazione e la realizzazione di un algoritmo predittivo efficiente richiede una profonda conoscenza matematica. Inoltre il capitolato non suggerisce minimamente come implementarlo, lasciando completa libertà al fornitore.
\item Il dominio applicativo del software è considerevolmente vasto e uno studio approfondito di tutti i suoi campi applicativi sarebbe troppo oneroso date le tempistiche ristrette.
\end{itemize}

\subsection{Valutazione Finale}
Il tema proposto dal capitolato risulta essere interessante per il gruppo poiché tratta argomenti che potenzialmente saranno alla base dell'internet del futuro, ma dopo un'attenta analisi sugli obbiettivi richiesti si è preferito scartarlo. La vastità del dominio applicativo e l'incertezza sull'implementazione dell'algoritmo predittivo avrebbero potuto portare a non rispettare i limiti di consegna del progetto