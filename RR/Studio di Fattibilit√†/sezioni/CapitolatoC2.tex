\section{Capitolato C2 - CLIPS}
\subsection{Descrizione}
L'obiettivo del progetto è la realizzazione di un software di navigazione/comunicazione/gaming/ecc, operante attraverso dispositivi beacon opportunamente posizionati all'interno di un'area indoor, basato sui concetti di proximity, IPS (Indoor Positioning System) e smart places per conto della società Miriade. Tale prototipo dovrà venire sviluppato appoggiandosi alla piattaforma di prossimità fornita da Miriade: Ubiika, la quale permette di gestire la microlocalizzazione del luogo fisico e di veicolare contenuti contestualizzati nelle aree coperte da tecnologia BLE beacon. Miriade inoltre si occuperà di fornire al team il numero di beacons necessario all'allestimento di un laboratorio per effettuare i vari test pratici, atti ad individuare problematiche legate all'hardware, alla costruzione del software e al lato user experience.     

\subsection{Dominio Applicativo}
Il servizio fornito sarà disponibile in ambienti indoor, generalmente non molto grandi. Viene lasciata ampia libertà su che tipologia di scenario realizzare, sia esso applicato alla navigazione, alla comunicazione/interazione o al social gaming. Il dominio applicativo risulta quindi essere un qualunque ambito social. 

\subsection{Dominio Tecnologico}
\begin{itemize}
\item Ubiika API: specifiche che permettono di interfacciarsi con la piattaforma Ubiika e di accederne il database;
\item BLE (Bluetooth Low Energy): tecnologia Bluetooth a consumo notevolmente ridotto di energia e costi e che riesce a mantenere un range di comunicazione molto simile a quello fornito dal Bluetooth classico;
\item Beacon (iBeacon/Eddystone): dispositivi beacon Apple/Google per gestire la microlocalizzazione indoor;
\item HTML5,CSS3,jQuery,JavaScript: consigliati per l'implementazione dell'interfaccia mobile;
\item Android/iOS API: librerie che permettono di interfacciarsi ai vari dispositivi beacon con i dispositivi mobile su cui verranno scaricate le applicazioni per interagire con il servizio.
\end{itemize}

  
\subsection{Criticità}
Nell'analisi del capitolato il team ha rilevato le seguenti criticità:
\begin{itemize} 
\item API Proprietarie: per la realizzazione è necessario l'uso di API prorietarie realizzate da Miriade e difficilmente la conoscenza di queste API tornerà utile in futuro;
\item Allestimento di un laboratorio: è richiesto di definire un'area indoor che verrà coperta dal nuovo servizio in cui poi eseguire i test di fattibilità tecnica;
\item Beacon e BLE: è richiesta una conoscenza approfondita dei dispositivi beacon e del protocollo Bluetooth BLE.
\end{itemize}


\subsection{Valutazione Finale}
Il gruppo ha deciso di scartare questo capitolato a causa delle tecnologie impiegate che sono sconosciute alla totalità dei membri del gruppo, considerando che l'utilizzo di una tecnologia proprietaria sia vincolante ed inoltre ritenuta di scarsa utilità in futuro. Infine si ritiene che l'interfacciamento hardware e i test di fattibilità tecnica potrebbero risultare onerosi per un gruppo con poca esperienza in questo ambito.  
