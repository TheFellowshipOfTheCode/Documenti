\newpage
\section{Capitolato C2}
\subsection{Descrizione}
L'obiettivo del capitolato è la realizzazione di un software di navigazione/comunicazione/gaming/eccetera, operante attraverso dispositivi beacons opportunamente posizionati all'interno di un'area indoor, basato sui concetti di proximity \textit{IPS\ped{G}} (Indoor Positioning System) e smart places per conto della società Miriade. Tale prototipo dovrà essere sviluppato appoggiandosi alla piattaforma di prossimità fornita da Miriade: Ubiika, la quale permette di gestire la microlocalizzazione del luogo fisico e di veicolare contenuti contestualizzati nelle aree coperte da tecnologia \textit{BLE\ped{G}} \textit{beacon\ped{G}}. Miriade inoltre si occuperà di fornire al team il numero di \textit{beacons\ped{G}} necessario all'allestimento di un laboratorio per effettuare i vari test pratici, atti ad individuare problematiche legate all'hardware, alla costruzione del software e al lato user experience.     

\subsection{Dominio Applicativo}
Il servizio fornito sarà disponibile in ambienti indoor, generalmente non molto grandi. Viene lasciata ampia libertà su che tipologia di scenario realizzare, sia esso applicato alla navigazione, alla comunicazione/interazione o al social gaming. Il dominio applicativo risulta quindi essere un qualunque ambiente social. 

\subsection{Dominio Tecnologico}
Viene richiesto l'utilizzo delle seguenti tecnologie:
\begin{itemize}
\item \textbf{Ubiika API}: specifiche che permettono di interfacciarsi con la piattaforma Ubiika e di accederne il database;
\item \textbf{\textit{BLE\ped{G}} (Bluetooth Low Energy)}: tecnologia Bluetooth a consumo ridotto di energia e costi che riesce a mantenere un range di comunicazione molto simile a quello fornito dal Bluetooth classico;
\item \textbf{\textit{Beacon\ped{G}} (iBeacon/Eddystone)}: dispositivi \textit{beacons\ped{G}} Apple/Google per gestire la microlocalizzazione indoor;
\item \textbf{\textit{HTML5\ped{G}}, \textit{CSS3\ped{G}}, \textit{jQuery\ped{G}}, \textit{JavaScript\ped{G}}}: consigliati per l'implementazione dell'interfaccia mobile;
\item \textbf{Android/iOS API}: librerie che permettono di interfacciarsi ai vari dispositivi \textit{beacons\ped{G}} con i dispositivi mobile su cui verranno scaricate le applicazioni per interagire con il servizio.
\end{itemize}

  
\subsection{Criticità}
Nell'analisi del capitolato il team ha rilevato le seguenti criticità:
\begin{itemize} 
\item Viene richiesta la definizione di un'area indoor che verrà coperta dal nuovo servizio, in cui poi eseguire i test di fattibilità tecnica;
\item Viene richiesta una conoscenza approfondita dei dispositivi \textit{beacons\ped{G}} e del protocollo \textit{BLE\ped{G}}.
\end{itemize}


\subsection{Valutazione Finale}
Il gruppo ha deciso di scartare questo capitolato a causa delle tecnologie richieste, sconosciute alla totalità dei membri del gruppo. L'utilizzo di una tecnologia proprietaria (Ubiika) inoltre risulta essere vincolante e di scarsa utilità per la realizzazione di un progetto futuro. Infine si ritiene che l'interfacciamento hardware e i test di fattibilità tecnica potrebbero risultare onerosi per un gruppo con scarsa esperienza nell'ambito dei \textit{beacons\ped{G}} e nell'utilizzo della tecnologia \textit{BLE\ped{G}}.  
