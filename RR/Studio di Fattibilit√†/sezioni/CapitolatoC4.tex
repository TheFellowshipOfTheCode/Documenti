\section{Capitolato C4}
\subsection{Descrizione}
Ogni applicazione che usa una base di dati persistente necessita di un sistema per l'amministrazione e la visualizzazione dei dati.\\ Uno dei capitolati proposti nell'anno accademico 2013/2014 prevedeva la realizzazione di MaaP (MondoDB as an admin Platform), una piattaforma capace di risolvere il problema della visualizzazione dei dati contenuti in un database.\\ 
L'obbiettivo del capitolato in questione si figura nel proporre MaaP come servizio: da qui il nome MaaS (MongoDB as admin Service). Tale servizio consiste nel rendere dispobile, attraverso una piattaforma Web, l'utilizzio di MaaP a qualsiasi utente, senza che quest'ultimo si preoccupi di gestire e mantenere le risorse hardware e software. 

\subsection{Dominio Applicativo}
La tipologia di servizio proposto mira a ridurre le competenze tecniche necessarie per poter utilizzare il prodotto MaaP. Il dominio applicativo risulta quindi essere un ambiente della bussiness area che necessita di poter consultare e gestire dei dati contenuti in una base di dati senza essere a conoscenza di come tale sistema funzioni e di come sia nessario configurarlo. 

\subsection{Dominio Tecnologico}
Per la realizzazione del seguente capitolato sono necessarie le seguenti conoscenze tecniche:
\begin{itemize}
	\item \textbf{Node.js:} è un framework event-driven per il motore JavaScript V8 sviluppado da Google. Si tratta quindi di un framework relativo all'utilizzo server-side di Javascript. 
	\item \textbf{MondoDB:} è un DBMS non relazionale, orientato ai documenti, classificato come NoSQL. 
	\item \textbf{Git:} è un sistema software di controllo di versione distribuito. 
	\item \textbf{React.js:} è una libreria scritta in JavaScript open-source sviluppata da Facebook che consente di creare interfacce utente complesse tramite la loro suddivisione in componenti. 
	\item \textbf{MaaP:} è una piattaforma capace di risolvere il problema della visualizzazione dei dati contenuti in una base di dati.	
\end{itemize}

\subsection{Criticità}
Nell’analisi del capitolato, il team ha rilevato le seguenti criticità:
\begin{itemize}
	\item Nessun incentivo verso lo sviluppo innovativo ma una rivisitazione di cose già fatte. 
	\item Il livello di inglese, da parte del team, non adeguato per una stesura completa ed ottimale della documentazione. 	
	\item Scomodità nell'effettuare incontri con il proponente vista la lontananza tra Amsterdam e Padova.
\end{itemize}

\subsection{Valutazione Finale}
Il tema proposto dal capitolato non risulta essere interessante per il gruppo poiché non punta verso un fattore innovativo ma più che altro su una rivisitazione di cose già fatte. 
Per di più la mera gestione di database di tipo NoSQL non risulta essere di gradimento da parte del gruppo: non da competenze concrete e non ha tecnologie rilevanti rispetto ad altri capitolati. 

