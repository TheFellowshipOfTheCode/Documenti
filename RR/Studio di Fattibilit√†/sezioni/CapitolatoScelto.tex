\newpage
\section{Capitolato Scelto}
\subsection{Descrizione}
Il capitolato proposto da \proponente{} riguarda la realizzazione di un software per la creazione e la gestione di questionari. 
Questo dovrà offrire la possibilità di: archiviare quiz in un server suddivisi per argomento, tradurre da \textit{QML\ped{G}} a \textit{HTML\ped{G}} le domande archiviate, gestire risposte vero/falso, risposte a scelta multipla, testi e immagini tramite \textit{QML\ped{G}}, archiviare questionari contenenti domande, proporre questionari preconfezionati e valutare le risposte fornite dall'utente.
La parte destinata ai creatori di questionari dovrà essere fruibile attraverso un \textit{browser\ped{G}} desktop, abilitato all'utilizzo delle tecnologie \textit{HTML5\ped{G}}, \textit{CSS3\ped{G}} e \textit{JavaScript\ped{G}}.
La parte destinata agli esaminandi sarà utilizzabile su qualunque dispositivo: dal personal computer, ai tablet e smartphone.

\subsection{Dominio Applicativo}
Il dominio applicativo del prodotto è la verifica delle conoscenze acquisite durante il processo di apprendimento di uno specifico argomento.
Il principale ambito di utilizzo del software riguarda la didattica in tutti i suoi campi.

\subsection{Dominio Tecnologico}
Per realizzare l'oggetto del capitolato vengono richieste al gruppo conoscenze legate all'ambito \textit{web\ped{G}}:
\begin{itemize}
\item \textbf{\textit{JavaScript\ped{G}}}: per la realizzazione della parte attiva;
\item \textbf{\textit{HTML5\ped{G}}}: per la strutturazione e la presentazione; 
\item \textbf{\textit{CSS3\ped{G}}}: per l'aspetto grafico;
\item \textbf{\textit{PostgreSQL\ped{G}} o \textit{MongoDB\ped{G}}}: per l'archiviazione dei dati.
\end{itemize}

\subsection{Criticità}
Sono state individuate le seguenti criticità sulla scelta del capitolato:
\begin{itemize}
\item Viene suggerito l'utilizzo di due \textit{framework\ped{G}} in base al linguaggio scelto per la programmazione. La scelta finale presa dai Progettisti potrebbe risultare non adatta allo sviluppo del progetto;
\item Viene richiesto lo sviluppo di un \textit{linguaggio di markup} studiato per descrivere i quiz. I testi descritti in \textit{QML\ped{G}} devono poi essere tradotti in \textit{HTML5\ped{G}} per essere presentati al candidato. L'implementazione di tale linguaggio non presenta alcun vincolo suggerito dal proponente, ma viene lasciata interamente al fornitore. Questa eccessiva libertà di scelta potrebbe portare ad uno sviluppo errato del linguaggio.
\end{itemize}

\subsection{Valutazione Finale}
La scelta del seguente capitolato d'appalto è stata determinata in base all'individuazione delle seguenti caratteristiche positive:
\begin{itemize} 
\item Interesse nelle moderne tecnologie web proposte;
\item Interesse nel lavorare con un'azienda solida e presente nel Padovano;
\item Acquisizione di esperienza e conoscenze tecniche utili e spendibili nel mondo del lavoro.
\end{itemize}
