\newpage
\section{Descrizione generale}
\subsection{Contesto d'uso del prodotto}
Il prodotto dovrà essere utilizzabile su un qualsiasi \textit{browser\ped{G}}, sia da desktop che da mobile, senza alcuna limitazione in base al sistema operativo.
\subsection{Funzioni del prodotto}
Il prodotto consiste in un applicativo \textit{web\ped{G}} che presenta le seguenti funzionalità principali:
\begin{itemize}
\item Creazione di un questionario tramite il linguaggio di markup QML;
\item Modifica di un questionario che l'utente aveva creato precedentemente;
\item Salvataggio di un questionario nell'archivio interno;
\item Esecuzione di un questionario presente nell'archivio;
\item Visualizzazione della valutazione e riepilogo delle risposte date al questionario svolto.
\end{itemize}
\subsection{Caratteristiche degli utenti}
Non sono richieste competenze particolari per poter utilizzare questo prodotto, che deve risultare quindi accessibile ad un ampia categoria di utenti. L'interfaccia dovrà quindi essere il più semplice e intuitiva possibile, senza però limitare le funzionalità offerte dal software stesso. Per questo motivo verrà fornito anche un \textit{\MU} con tutte le indicazioni necessarie per consentire un utilizzo corretto ed efficace del prodotto.
\subsection{Vincoli generali}
Tutti i questionari creati vengono salvati all'interno dell'applicativo stesso e potranno essere modificati, dal proprietario, o eseguiti, da altri utenti, solo all'interno di quest'ultimo.
\subsection{Assunzione dipendenze}
Per il corretto funzionamento dell'applicazione sarà necessario l'utilizzo di un \textit{browser\ped{G}} che sia compatibile con gli standard \textit{HTML5\ped{G}} e \textit{CSS3\ped{G}}.