\section{Casi d'uso}
Vengono elencati i casi d'uso rilavati dall'analisi del capitolato C5 e durante le riunioni con Zucchetti S.P.A.. \\
Ogni caso d'uso è identificato da seguente formalismo:
\begin{center}
	UC[codice]
\end{center}
dove il codice indica il codice univoco di ogni caso d'uso che verrà riportato in forma gerarchica.

\subsection{Attori}
Di seguito è riportato un diagramma \textit{UML\ped{g}} che descrive gli attori. L'utente non autenticato, l'utente autenticato e infine l'utente autenticato pro che deriva dall'utente autenticato, ereditandone tutte le funzionalità.
\label{Attori}
\begin{figure}
	\centering
	\includegraphics[scale=0.45]{UML/attori.png}
	\caption{Attori dell'applicazione}
\end{figure}


\subsubsection{Caso d'uso UC1: Registrazione}

\newpage
\subsection{Caso d'uso UC2: Registrazione}
\begin{itemize}
\item \textbf{Attori}: Utente;
\item \textbf{Scopo e descrizione}: per poter usufruire dei servizi forniti dalla piattaforma, l'utente deve registrarsi inserendo email e password;
\item \textbf{Pre-condizione}: il sistema è avviato e mostra la pagina iniziale;
\item \textbf{Flusso principale degli eventi}:
	\begin{enumerate}
	\item L'utente inserisce il proprio nome utente (nome e cognome) [UCD2.1];
	\item L'utente inserisce l'email [UCD2.2];
	\item L'utente inserisce la password [UCD2.3];
	\item L'utente inserisce una seconda volta la password [UCD2.4];
	\item L'utente conferma i dati inseriti [UCD2.5].
	\end{enumerate}
\item \textbf{Scenario alternativo}: possono verificarsi uno o più dei seguenti scenari:
	\begin{itemize}
	\item[-] Il nome utente inserito non è valido oppure è vuoto;
	\item[-] L'email inserita non è valida oppure è vuota;
	\item[-] L'email inserita è già presente nel sistema;
	\item[-] La password inserita è vuota oppure non è di almeno 8 caratteri;
	\item[-] La password e la conferma password non coincidono.
	\end{itemize}
In tal caso il sistema ritorna allo stato precedente l'inserimento dei dati visualizzando un messaggio di errore;
\item \textbf{Estensione}: l'utente visualizza un messaggio di errore [UCD2.6].
\item \textbf{Post-condizione}: il sistema ha registrato l'utente.
\end{itemize}

\subsubsection{Caso d'uso UC2.1: Inserimento nome utente}
\begin{itemize}
\item \textbf{Attori}: Utente;
\item \textbf{Scopo e descrizione}: l'utente inserisce il proprio nome e cognome (nome utente) per potersi registrare;
\item \textbf{Pre-condizione}: il sistema presenta all'utente lo spazio destinato a questa operazione;
\item \textbf{Post-condizione}: il nome utente è stato inserito.
\end{itemize}

\subsubsection{Caso d'uso UC2.2: Inserimento email}
\begin{itemize}
\item \textbf{Attori}: Utente;
\item \textbf{Scopo e descrizione}: l'utente inserisce il proprio indirizzo email per potersi registrare;
\item \textbf{Pre-condizione}: il sistema presenta all'utente lo spazio destinato a questa operazione;
\item \textbf{Post-condizione}: l'email è stata inserita.
\end{itemize}

\subsubsection{Caso d'uso UC2.3: Inserimento password}
\begin{itemize}
\item \textbf{Attori}: Utente;
\item \textbf{Scopo e descrizione}: l'utente inserisce una password a sua scelta per potersi registrare;
\item \textbf{Pre-condizione}: il sistema presenta all'utente lo spazio destinato a questa operazione;
\item \textbf{Post-condizione}: la password è stata inserita.
\end{itemize}

\subsubsection{Caso d'uso UC2.4: Inserimento conferma password}
\begin{itemize}
\item \textbf{Attori}: Utente;
\item \textbf{Scopo e descrizione}: l'utente inserisce nuovamente la password scelta;
\item \textbf{Pre-condizione}: il sistema presenta all'utente lo spazio destinato a questa operazione;
\item \textbf{Post-condizione}: la conferma della password è stata inserita.
\end{itemize}

\subsubsection{Caso d'uso UC2.5: Conferma registrazione}
\begin{itemize}
\item \textbf{Attori}: Utente;
\item \textbf{Scopo e descrizione}: l'utente conferma i dati inseriti;
\item \textbf{Pre-condizione}: il sistema presenta all'utente lo spazio destinato a questa operazione;
\item \textbf{Post-condizione}: il sistema ha ricevuto i dati per la registrazione.
\end{itemize}

\subsubsection{Caso d'uso UC2.6: Visualizzazione errore registrazione}
\begin{itemize}
\item \textbf{Attori}: Utente;
\item \textbf{Scopo e descrizione}: l'utente visualizza un messaggio d'errore nel caso si fossero verificati uno o più scenari alternativi;
\item \textbf{Pre-condizione}: il sistema ha ricevuto dei dati errati per la registrazione;
\item \textbf{Post-condizione}: il sistema mostra un messaggio di errore.
\end{itemize}


\newpage
\subsection{Caso d'uso UC3: Login}
\label{UC3}
\begin{figure}
	\centering
	\includegraphics[scale=0.5]{UML/UC3.png}
	\caption{UC3: Login}
\end{figure}
\FloatBarrier
\begin{itemize}
	\item \textbf{Attori}: utente non autenticato;
	\item \textbf{Descrizione}: l'attore si può autenticare tramite uno dei metodi proposti;
	\item \textbf{Precondizione}: il sistema è avviato e pronto per l'utilizzo e mostra la pagina di login;
	\item \textbf{Postcondizione}: l'attore è autenticato;
	\item \textbf{Scenario principale}: l'attore sceglie uno dei seguenti modi per autenticarsi:
		\begin{enumerate}
			\item L'attore può effettuare il login con \progetto (UC3.1);
			\item L'attore può effettuare il login con Facebook (UC3.2);
			\item L'attore può effettuare il login con Twitter (UC3.3);
			\item L'attore può effettuare il login con Google+ (UC3.4);
			\item L'attore può effettuare il login con LinkedIn (UC3.5).
		\end{enumerate}
\end{itemize}

\subsubsection{Caso d'uso UC3.1: Login da \progetto}
\label{UC3.1}
\begin{figure}
	\centering
	\includegraphics[scale=0.5]{UML/UC3_1.png}
	\caption{UC3.1: Login da \progetto}
\end{figure}
\FloatBarrier
\begin{itemize}
	\item \textbf{Attori}: utente non autenticato;
	\item \textbf{Descrizione}: l'attore si può autenticare inserendo username/mail e password, con cui è registrato;
	\item \textbf{Precondizione}: il sistema è avviato e pronto per l'utilizzo e mostra la pagina di login;
	\item \textbf{Postcondizione}: il sistema ha autenticato l'attore e quindi mostra all'attore autenticato la sua area riservata;
	\item \textbf{Scenario Principale}:
	\begin{enumerate}
		\item L'attore può inserire l'username oppure la mail utilizzata al momento della registrazione (UC3.1.1);
		\item L'attore può inserire la password (UC3.1.2);
		\item L'attore può confermare il login (UC3.1.3).
	\end{enumerate}
	\item \textbf{Estensioni}: autenticazione fallita (UC3.1.4).
\end{itemize}

\subsubsection{Caso d'uso UC3.1.1: Inserimento username/mail}
\begin{itemize}
	\item \textbf{Attori}: utente non autenticato;
	\item \textbf{Descrizione}: l'attore può inserire l'username o la mail associata al proprio account;
	\item \textbf{Precondizione}: il sistema presente all'attore lo spazio destinato a questa operazione;
	\item \textbf{Postcondizione}: l'attore ha inserito username/mail;
	\item \textbf{Scenario principale}: l'attore inserisce l'username o la mail associata al proprio account. 
\end{itemize}

\subsubsection{Caso d'uso UC3.1.2: Inserimento password}
\begin{itemize}
	\item \textbf{Attori}: utente non autenticato;
	\item \textbf{Descrizione}: l'attore può inserire la password associata al proprio account;
	\item \textbf{Precondizione}: il sistema presenta all'attore lo spazio destinato a questa operazione;
	\item \textbf{Postcondizione}: l'attore inserisce la password;
	\item \textbf{Scenario principale}: l'attore inserisce la password associata al proprio account.
\end{itemize}

\subsubsection{Caso d'uso UC3.1.3: Conferma login}
\begin{itemize}
	\item \textbf{Attori}: utente non autenticato;
	\item \textbf{Descrizione}: l'attore può confermare i dati inseriti per effettuare il login;
	\item \textbf{Precondizione}: l'attore ha inserito l'username/mail e la password;
	\item \textbf{Postcondizione}: l'attore è autenticato;
	\item \textbf{Scenario principale}: l'attore conferma i dati inseriti per effettuare la login con il proprio account.
\end{itemize}

\subsubsection{Caso d'uso UC3.1.4: Autenticazione fallita}
\begin{itemize}
	\item \textbf{Attori}: utente non autenticato;
	\item \textbf{Descrizione}: l'attore può visualizzare un messaggio d'errore nel caso si fossero verificati uno o più scenari alternativi durante la fase di autenticazione;
	\item \textbf{Precondizione}: il sistema ha ricevuto dei dati errati per l'autenticazione;
	\item \textbf{Postcondizione}: il sistema avvisa l'attore dell'errore verificatosi tramite un opportuno messaggio;
	\item \textbf{Scenario principale}: l'attore visualizza un messaggio d'errore.
\end{itemize}

\subsubsection{Caso d'uso UC3.2: Login con Facebook}
\begin{itemize}
	\item \textbf{Attori}: utente non autenticato;
	\item \textbf{Descrizione}: l'attore può autenticarsi utilizzando Facebook;
	\item \textbf{Precondizione}: l'attore visualizza la pagina di login e sceglie la login con Facebook;
	\item \textbf{Postcondizione}: l'attore è autenticato;
	\item \textbf{Scenario principale}: l'attore effettua il login tramite Facebook.
\end{itemize}
\subsubsection{Caso d'uso UC3.3: Login con Twitter}
\begin{itemize}
	\item \textbf{Attori}: utente non autenticato;
	\item \textbf{Scopo e descrizione}: l'attore può autenticarsi utilizzando Twitter;
	\item \textbf{Precondizione}: l'attore visualizza la pagina di login e sceglie la login con Twitter;
	\item \textbf{Postcondizione}: l'attore è autenticato;
	\item \textbf{Scenario principale}: l'attore effettua il login tramite Twitter.
\end{itemize}
\subsubsection{Caso d'uso UC3.4: Login con Google+}
\begin{itemize}
	\item \textbf{Attori}: utente non autenticato;
	\item \textbf{Descrizione}: l'attore può autenticarsi utilizzando Google+;
	\item \textbf{Precondizione}: l'attore visualizza la pagina di login e sceglie la login con Google+;
	\item \textbf{Postcondizione}: l'attore è autenticato;
	\item \textbf{Scenario principale}: l'attore effettua il login tramite Google+.
\end{itemize}
\subsubsection{Caso d'uso UC3.5: Login con LinkedIn}
\begin{itemize}
	\item \textbf{Attori}: utente non autenticato;
	\item \textbf{Descrizione}: l'attore può autenticarsi utilizzando LinkedIn;
	\item \textbf{Precondizione}: l'attore visualizza la pagina di login e sceglie la login con LinkedIn;
	\item \textbf{Postcondizione}: l'attore è autenticato;
	\item \textbf{Scenario principale}: l'attore effettua il login tramite LinkedIn.
\end{itemize}

\subsection{Caso d'uso UC4: Logout}
	\label{UC4}
	\begin{figure}[h]
		\centering
			\includegraphics[scale=0.5,keepaspectratio]{UML/UC4.png}
		\caption{UC4: Logout}
	\end{figure}
	\FloatBarrier
	\begin{itemize}
		\item
			\textbf{Attori}: utente autenticato, utente autenticato pro;
		\item		
			\textbf{Descrizione}: l'attore termina la sua sessione, uscendo dalla sua area riservata;
		\item
			\textbf{Precondizione}: l'attore è autenticato presso il sistema;
		\item
			\textbf{Postcondizione}: l'attore non è autenticato presso il sistema;
		\item
			\textbf{Scenario principale}:
	       		\begin{itemize}
					\item
					L'attore riceve la conferma della disconnessione (UC4.1).
	 			\end{itemize}
	\end{itemize}

\subsubsection{Caso d'uso UC4.1: Conferma disconnessione}
	\begin{itemize}
		\item
			\textbf{Attori}: utente autenticato, utente autenticato pro;
		\item
			\textbf{Descrizione}: l'attore, una volta selezionato l'opzione logout, attende la conferma della disconnessione da parte del sistema;
 		\item
			\textbf{Precondizione}: l'attore ha selezionato l'opzione logout;
		\item
			\textbf{Postcondizione}: Il sistema conferma all'attore l'uscita dalla sua area riservata;
		\item
			\textbf{Scenario principale}: l'attore riceve conferma del logout.
	\end{itemize}		
	

\subsection{Caso d'uso UC5: Ricerca questionari esistenti}

\newpage
\subsection{Caso d'uso UC6: Ricerca e iscrizione questionario}
\label{UC6}
\begin{figure}[h]
\centering
\includegraphics[scale=0.5,keepaspectratio]{UML/UC6.png}
\caption{UC6: Ricerca e iscrizione questionario}
\end{figure}
\FloatBarrier
\begin{itemize}
\item\textbf{Attori}: utente autenticato, utente autenticato pro;
\item\textbf{Descrizione}: nella schermata principale qualsiasi attore che voglia iscriversi ad un questionario può ricercarlo attraverso la barra di ricerca. L'attore, per poter compilare un questionario, dovrà poi iscriversi al questionario selezionato;	
\item\textbf{Precondizione}: l'attore si trova nella pagina principale dell'applicazione;
\item\textbf{Postcondizione}: l'attore si è iscritto al questionario che vuole compilare;
\item\textbf{Scenario principale}:
\begin{enumerate}
\item L'attore può cercare un questionario tramite barra di ricerca (UC6.1);
\item L'attore può iscriversi al questionario scelto (UC6.2).
\end{enumerate}
\item\textbf{Estensioni}: ricerca questionario fallita (UC6.3);
\item\textbf{Scenari Alternativi}: l'attore ricerca un questionario che non esiste. 
\end{itemize}

\subsubsection{Caso d'uso UC6.1: Ricerca questionario tramite barra di ricerca}
\label{UC6.1}
\begin{itemize}
\item\textbf{Attori}: utente autenticato, utente autenticato pro;
\item\textbf{Descrizione}: all'interno della pagina principale dell'applicazione è presente una barra di ricerca dove l'attore può cercare i questionari;
\item\textbf{Precondizione}: l'attore si trova nella pagina principale dell'applicazione;
\item\textbf{Postcondizione}: l'attore visualizza i questionari che contengono il testo scritto nella barra di ricerca;
\item\textbf{Scenario principale}: l'attore utilizza la barra di ricerca per cercare un questionario.
\end{itemize}

\subsubsection{Caso d'uso UC6.2: Iscrizione questionario}
\label{UC6.2}
\begin{figure}[h]
\centering
\includegraphics[scale=0.5,keepaspectratio]{UML/UC6_2.png}
\caption{UC6.2: Iscrizione questionario}
\end{figure}
\FloatBarrier
\begin{itemize}
\item\textbf{Attori}: utente autenticato, utente autenticato pro;
\item\textbf{Descrizione}: l'attore dopo aver scelto un questionario, per iniziare a compilarlo, può iscriversi;
\item\textbf{Precondizione}: l'attore ha scelto il questionario che vuole svolgere;
\item\textbf{Postcondizione}: l'attore si è iscritto al questionario scelto;
\item\textbf{Scenario principale}: 
\begin{enumerate}
\item L'attore può confermare l'iscrizione al questionario (UC6.2.1).
\end{enumerate}
\item\textbf{Scenari alternativi}: l'attore cambia idea sul questionario al quale vuole iscriversi e annulla l'operazione tornando alla schermata precedente.
\end{itemize}

\subsubsection{Caso d'uso UC6.2.1: Conferma iscrizione questionario}
\label{UC6.2.1}
\begin{itemize}
\item\textbf{Attori}: utente autenticato, utente autenticato pro;
\item\textbf{Descrizione}: l'attore conferma l'iscrizione al questionario, potendo così procedere con la compilazione;
\item\textbf{Precondizione}: l'attore si è iscritto ad un questionario;
\item\textbf{Postcondizione}: l'attore ha confermato l'iscrizione al questionario;
\item\textbf{Scenario principale}: l'attore conferma di voler compilare il questionario a cui si è iscritto.
\end{itemize}

\subsubsection{Caso d'uso UC6.3: Ricerca questionario fallita}
\label{UC6.3}
\begin{itemize}
\item\textbf{Attori}: utente autenticato, utente autenticato pro;
\item\textbf{Descrizione}: l'attore ha inserito nella barra di ricerca del testo che non compare in nessuno dei questionari esistenti;
\item\textbf{Precondizione}: l'attore ha utilizzato la barra di ricerca per cercare un questionario che non esiste;
\item\textbf{Postcondizione}: il sistema avvisa l'attore dell'errore verificatosi tramite un opportuno messaggio;
\item\textbf{Scenario principale}: l'attore visualizza un messaggio che lo avvisa del mancato ritrovamento di questionari.
\end{itemize}

\newpage
\subsection{Caso d'uso UC7: Compilazione questionario}
\label{UC7}
\begin{figure}[h]
\centering
\includegraphics[scale=0.5,keepaspectratio]{UML/UC7.png}
\caption{UC7: Compilazione questionario}
\end{figure}
\FloatBarrier
\begin{itemize}
\item\textbf{Attori}: utente autenticato, utente autenticato pro;
\item\textbf{Descrizione}: l'attore che compila un questionario può rispondere alle domande che gli si presentano nell'ordine che preferisce spostandosi alla domanda successiva, a quella precedente oppure ad una a sua scelta e infine conferma le risposte date per poter visualizzare il risultato ottenuto e il riepilogo delle risposte date;
\item\textbf{Precondizione}: l'attore si è iscritto ad un questionario e l'autore del questionario ne ha abilitato la compilazione;
\item\textbf{Postcondizione}: l'attore visualizza la valutazione finale che ha ottenuto e il riepilogo delle risposte date;
\item\textbf{Scenario principale}:
\begin{itemize}
\item L'attore può passare alla domanda successiva (UC7.1);
\item L'attore può passare alla domanda precedente (UC7.2);
\item L'attore può passare ad una domanda a sua scelta (UC7.3);
\item L'attore può confermare le risposte date alle domande (UC7.4);
\item L'attore può visualizzare il risultato e il riepilogo del questionario svolto (UC7.5).
\end{itemize}
\item\textbf{Scenari alternativi}: nel caso in cui il questionario non è stato abilitato compare una schermata di attesa e l'attore può annullare l'operazione tornando alla schermata precedente.
\end{itemize}

\subsubsection{Caso d'uso UC7.1: Spostamento alla domanda successiva}
\label{UC7.1}
\begin{itemize}
\item\textbf{Attori}: utente autenticato, utente autenticato pro;
\item\textbf{Descrizione}: l'attore può passare alla domanda successiva;
\item\textbf{Precondizione}: l'attore sta compilando un questionario;
\item\textbf{Postcondizione}: il sistema visualizza la domanda successiva;
\item\textbf{Scenario principale}: l'attore seleziona il comando per passare alla domanda successiva.
\end{itemize}

\subsubsection{Caso d'uso UC7.2: Spostamento alla domanda precedente}
\label{UC7.2}
\begin{itemize}
\item\textbf{Attori}: utente autenticato, utente autenticato pro;
\item\textbf{Descrizione}: l'attore può passare alla domanda precedente;
\item\textbf{Precondizione}: l'attore sta compilando un questionario;
\item\textbf{Postcondizione}: il sistema visualizza la domanda precedente;
\item\textbf{Scenario principale}: l'attore seleziona il comando per passare alla domanda precedente.
\end{itemize}

\subsubsection{Caso d'uso UC7.3: Spostamento ad una domanda a scelta}
\label{UC7.3}
\begin{itemize}
\item\textbf{Attori}: utente autenticato, utente autenticato pro;
\item\textbf{Descrizione}: l'attore può passare ad una domanda a sua scelta selezionandola attraverso la tabella delle domande. Queste saranno identificate da un numero, in base all'ordine in cui sono poste all'interno del questionario, e da un colore, arancioni se è già stata data una risposta oppure grige in caso contrario;
\item\textbf{Precondizione}: l'attore sta compilando un questionario;
\item\textbf{Postcondizione}: il sistema visualizza la domanda selezionata dall'attore;
\item\textbf{Scenario principale}: l'attore seleziona una domanda a cui deve ancora rispondere o una alla quale vuole cambiare la risposta data.
\end{itemize}

\subsubsection{Caso d'uso UC7.4: Conferma risposte}
\label{UC7.4}
\begin{itemize}
\item\textbf{Attori}: utente autenticato, utente autenticato pro;
\item\textbf{Descrizione}: l'attore può confermare le domande a cui ha dato una risposta;
\item\textbf{Precondizione}: l'attore sta compilando un questionario;
\item\textbf{Postcondizione}: l'attore ha confermato le risposte date;
\item\textbf{Scenario principale}: l'attore che ha finito di rispondere a tutte le domande conferma le risposte date;
\item\textbf{Scenari alternativi}: l'attore conferma le risposte date anche se non le ha compilate tutte, accettando il fatto che queste verranno considerate sbagliate.
\end{itemize}

\subsubsection{Caso d'uso UC7.5: Visualizzazione risultato e riepilogo questionario compilato}
\label{UC7.5}
\begin{itemize}
\item\textbf{Attori}: utente autenticato, utente autenticato pro;
\item\textbf{Descrizione}: l'attore, dopo aver finito di compilare il questionario, può visualizzare una schermata contenente il riepilogo delle risposte date alle domande. Sarà possibile vedere quali di queste sono corrette e quali no e verrà inoltre data una valutazione complessiva;
\item\textbf{Precondizione}: l'attore ha finito di compilare il questionario;
\item\textbf{Postcondizione}: il sistema visualizza una schermata contenente il riepilogo del questionario svolto e un voto;
\item\textbf{Scenario principale}: l'attore visualizza che voto gli è stato dato e a quali domande ha risposto correttamente o meno.
\end{itemize}

\subsection{Caso d'uso UC8: Gestione delle domande}
	
\subsubsection{Caso d'uso UC8.1: Creazione nuova domanda}
	\label{UC8.1}
	\begin{figure}[h]
		\centering
			\includegraphics[scale=0.45,keepaspectratio]{UML/UC8_1.png}
		\caption{Caso d'uso UC8.1: Creazione nuova domanda}
	\end{figure}
	\FloatBarrier
	\begin{itemize}
		\item
			\textbf{Attori}: utente autenticato, utente autenticato pro;
		\item		
			\textbf{Scopo e descrizione}: L'utente autenticato crea una nuova domanda, utile all'apprendimento di un certo argomento;
		\item
			\textbf{Precondizione}: L'utente è autenticato presso il sistema. 
		\item
			\textbf{	Postcondizione}: L'utente riceve conferma dell'avvenuta creazione.		
		\item
			\textbf{Scenario principale}:
	       		\begin{enumerate}
					\item 	
					L'utente seleziona l'opzione per la creazione di una nuova domanda [UC8.1.1];
					\item
					L'utente compila i campi dati del modulo previsto per compiere questa operazione [UC8.1.2];
					\item
					L'utente conferma la creazione della domanda [UC8.1.3];
	 			\end{enumerate}
	 	\item
			\textbf{Estensione}: L'utente visualizza un messaggio d'errore di conferma [UC8.1.4].
	 	\item
	 		\textbf{Scenario alternativo}: possono verificarsi uno o più di questi scenari:
				\begin{itemize}
					\item[-] 	
						E' presente un campo obbligatorio vuoto;
					\item[-] 
    						E' presente un dato non valido;
					\item[-] 
						L'operazione di conferma non è andata a buon fine.
				\end{itemize}
			In tal caso il sistema ripresenta il modulo di creazione della domanda, presentando un messaggio d'errore.
	\end{itemize}
	\subsubsection{Caso d'uso UC8.1.1: Selezione opzione creazione domanda}
	\begin{itemize}
		\item
			\textbf{Attori}: utente autenticato, utente autenticato pro;
		\item
			\textbf{Scopo e descrizione}: l'utente autenticato seleziona l'opzione per la creazione di una domanda;
		\item		
			\textbf{Precondizione}: il sistema presenta all'utente autenticato l'opzione per compiere questa operazione;
		\item
			\textbf{Postcondizione}: il sistema presenta all'utente autenticato il modulo per creare la domanda.
	\end{itemize}	
	\subsubsection{Caso d'uso UC8.1.2: Compilazione campi dati}
	\begin{itemize}
		\item
			\textbf{Attori}: utente autenticato, utente autenticato pro;
		\item
			\textbf{Scopo e descrizione}: l'utente autenticato compila i campi dati del modulo per la creazione della domanda;
		\item		
			\textbf{Precondizione}: il sistema presenta all'utente autenticato ha selezionato l'opzione per creare la domanda;
		\item
			\textbf{Postcondizione}: i campi dati del modulo di creazione sono riempiti.
	\end{itemize}	
	\subsubsection{Caso d'uso UC8.1.3: Conferma creazione}
	\begin{itemize}
		\item
			\textbf{Attori}: utente autenticato, utente autenticato pro;
		\item
			\textbf{Scopo e descrizione}: l'utente autenticato conferma la creazione della domanda per essere inserita nel DB;
		\item		
			\textbf{Precondizione}: il sistema presenta all'utente autenticato l'opzione per compiere questa operazione;
		\item
			\textbf{Postcondizione}: il sistema ha ricevuto i dati per la registrazione.
	\end{itemize}	
	\subsubsection{Caso d'uso UC8.1.4: Visualizzazione errore creazione}
	\begin{itemize}
		\item
			\textbf{Attori}: utente autenticato, utente autenticato pro;
		\item
			\textbf{Scopo e descrizione}: l'utente visualizza un messaggio d'errore nel caso si fossero verificati uno o più scenari alternativi;
		\item		
			\textbf{Precondizione}: il sistema ha ricevuto dei dati errati per creazione;
		\item
			\textbf{Postcondizione}: il sistema mostra un messaggio d'errore.
	\end{itemize}	
	\subsubsection{Caso d'uso UC8.2: Modifica domanda esistente}
	\label{UC8.2}
	\begin{figure}[h]
		\centering
			\includegraphics[scale=0.45,keepaspectratio]{UML/UC8_2.png}
		\caption{Caso d'uso UC8.2: Modifica domanda esistente}
	\end{figure}
	\FloatBarrier
	\begin{itemize}
		\item
			\textbf{Attori}: utente autenticato, utente autenticato pro;
		\item		
			\textbf{Scopo e descrizione}: L'utente autenticato modifica una domanda da lui in precedenza creata per correggere errori ortografici o sintattici;
		\item
			\textbf{Precondizione}: L'utente è autenticato presso il sistema. 
		\item
			\textbf{	Postcondizione}: L'utente riceve conferma dell'avvenuta modifica.
		\item
			\textbf{Scenario principale}:
	       		\begin{enumerate}
					\item 	
					L'utente seleziona l'opzione per la modifica della domanda [UC8.2.1];
					\item
					L'utente modifica i campi dati, già precompilati dai dati della domanda selezionata, del modulo previsto per compiere questa operazione [UC8.2.2];
					\item
					L'utente conferma la modifica della domanda [UC8.2.3];
	 			\end{enumerate}
	 	\item
			\textbf{Estensione}: L'utente visualizza un messaggio d'errore di conferma [UC8.2.4].
	 	\item
	 		\textbf{Scenario alternativo}: possono verificarsi uno o più di questi scenari:
				\begin{itemize}
					\item[-] 	
						E' presente un campo obbligatorio vuoto;
					\item[-] 
    						E' presente  un dato non valido;
					\item[-] 
						L'operazione di conferma non è andata a buon fine.
				\end{itemize}
			In tal caso il sistema ripresenta il modulo di modifica della domanda, presentando un messaggio d'errore.
	\end{itemize}
	\subsubsection{Caso d'uso UC8.2.1: Selezione opzione modifica domanda}
	\begin{itemize}
		\item
			\textbf{Attori}: utente autenticato, utente autenticato pro;
		\item
			\textbf{Scopo e descrizione}: l'utente autenticato seleziona l'opzione per compiere la modifica della domanda;
		\item		
			\textbf{Precondizione}: il sistema presenta all'utente autenticato l'opzione per compiere questa operazione;
		\item
			\textbf{Postcondizione}: il sistema presenta all'utente autenticato il modulo per modificare la domanda.
	\end{itemize}
	\subsubsection{Caso d'uso UC8.2.2: Modifica campi dati}
	\begin{itemize}
		\item
			\textbf{Attori}: utente autenticato, utente autenticato pro;
		\item
			\textbf{Scopo e descrizione}: l'utente autenticato modifica i dati della domanda selezionata;
		\item		
			\textbf{Precondizione}: il sistema presenta all'utente autenticato il modulo per modificare la domanda selezionata;
		\item
			\textbf{Postcondizione}: i campi dati del modulo di modifica, già precompilati con i dati della domanda selezionata, sono modificati.
	\end{itemize}		
	\subsubsection{Caso d'uso UC8.2.3: Conferma modifica}
	\begin{itemize}
		\item
			\textbf{Attori}: utente autenticato, utente autenticato pro;
		\item
			\textbf{Scopo e descrizione}: l'utente autenticato conferma la modifica dei dati della domanda selezionata;
		\item		
			\textbf{Precondizione}: il sistema presenta all'utente autenticato l'opzione per compiere questa operazione;
		\item
			\textbf{Postcondizione}: il sistema ha ricevuto i dati per la registrazione.
	\end{itemize}		
	\subsubsection{Caso d'uso UC8.2.4: Visualizzazione errore modifica}
	\begin{itemize}
		\item
			\textbf{Attori}: utente autenticato, utente autenticato pro;
		\item
			\textbf{Scopo e descrizione}: l'utente visualizza un messaggio d'errore nel caso si fossero verificati uno o più scenari alternativi;
		\item		
			\textbf{Precondizione}: il sistema ha ricevuto dei dati errati per la modifica;
		\item
			\textbf{Postcondizione}: il sistema mostra un messaggio d'errore.
	\end{itemize}	
	
	\subsubsection{Caso d'uso UC8.3: Eliminazione domanda}	
	\label{UC8.3}
	\begin{figure}[h]
		\centering
			\includegraphics[scale=0.45,keepaspectratio]{UML/UC8_3.png}
		\caption{Caso d'uso UC8.3: Eliminazione domanda}
	\end{figure}
	\FloatBarrier
	\begin{itemize}
		\item
			\textbf{Attori}: utente autenticato, utente autenticato pro;
		\item		
			\textbf{Scopo e descrizione}: L'utente autenticato elimina una domanda, da lui in precedenza creata, per motivi validi e giustificati;
		\item
			\textbf{Precondizione}: L'utente è autenticato presso il sistema;
		\item
			\textbf{	Postcondizione}: L'utente riceve conferma dell'avvenuta eliminazione.
		\item
			\textbf{Scenario principale}:
	       		\begin{enumerate}
					\item 	
					L'utente seleziona l'opzione per l'eliminazione di una domanda [UC8.3.1];
					\item
					L'utente seleziona la domanda da eliminare tramite uno strumento di selezione presentato dal sistema [UC8.3.2];
					\item
					L'utente conferma l'eliminazione della domanda [UC8.3.3];
	 			\end{enumerate}
	\end{itemize}
	\subsubsection{Caso d'uso UC8.3.1: Selezione opzione eliminazione domanda}
	\begin{itemize}
		\item
			\textbf{Attori}: utente autenticato, utente autenticato pro;
		\item
			\textbf{Scopo e descrizione}: l'utente autenticato seleziona l'opzione per compiere l' eliminazione di una domanda;
		\item		
			\textbf{Precondizione}: il sistema presenta all'utente autenticato l'opzione per compiere questa operazione;
		\item
			\textbf{Postcondizione}: il sistema presenta all'utente autenticato lo strumento di selezione della domanda da eliminare.
	\end{itemize}
	\subsubsection{Caso d'uso UC8.3.2: Selezione domanda da eliminare}
	\begin{itemize}
		\item
			\textbf{Attori}: utente autenticato, utente autenticato pro;
		\item
			\textbf{Scopo e descrizione}: l'utente autenticato seleziona, tramite uno specifico strumento presentato dal sistema, una domanda da eliminare;
		\item		
			\textbf{Precondizione}: il sistema presenta all'utente autenticato il modulo per modificare la domanda selezionata;
		\item
			\textbf{Postcondizione}: i campi dati del modulo di modifica, già precompilati con i dati della domanda selezionata, sono modificati.
	\end{itemize}		
	\subsubsection{Caso d'uso UC8.3.3: Conferma eliminazione}
	\begin{itemize}
		\item
			\textbf{Attori}: utente autenticato, utente autenticato pro;
		\item
			\textbf{Scopo e descrizione}: l'utente autenticato conferma l'eliminazione della domanda dal sistema;
		\item		
			\textbf{Precondizione}: il sistema presenta all'utente autenticato l'opzione per compiere questa operazione;
		\item
			\textbf{Postcondizione}: il sistema ha eliminato la domanda dal sistema.
	\end{itemize}	

\subsection{Caso d'uso UC9: Gestione dei questionari}
\label{UC9}
\begin{figure}[h]
	\centering
	\includegraphics[scale=0.5,keepaspectratio]{UML/UC9.png}
	\caption{UC9: Gestione dei questionari}
\end{figure}
\FloatBarrier
\begin{itemize}
	\item \textbf{Attori}: \uau, \uaupro;
	\item \textbf{Descrizione}: il sistema mostra una schermata in cui l'utente può gestire i propri questionari; 
	\item \textbf{Precondizione}: l'utente accede al sito \textit{web\ped{G}} mediante ad un \textit{browser\ped{G}} supportato dal sistema;
	\item \textbf{Postcondizione}: il sistema ha eseguito le funzionalità scelte dall'utente;
	\item \textbf{Scenario principale}:
		\begin{enumerate}
			\item L'utente può visualizzare i propri questionati (UC9.1);
			\item L'utente può creare nuovi questionari (UC9.2).
		\end{enumerate}
\end{itemize}

	\subsubsection{Caso d'uso UC9.1: Visualizza questionari}
	\label{UC9.1}
	\begin{figure}[h]
		\centering
	\includegraphics[scale=0.5,keepaspectratio]{UML/UC9.png}
		\caption{UC9.1: Visualizza questionari}
	\end{figure}
	\FloatBarrier
	\begin{itemize}
		\item \textbf{Attori}: \uau, \uaupro;
		\item \textbf{Descrizione}: l'utente può visualizzare i propri questionari creati, quelli svolti e quelli salvati per essere eseguiti in un secondo momento; 
		\item \textbf{Precondizione}: l'utente ha selezionato l'opzione "Visualizza questionari" tra le possibilità proposte in UC9;
		\item \textbf{Postcondizione}: il sistema ha mostrato all'utente i questionari e ha permesso delle operazioni su di essi; 
		\item \textbf{Scenario principale}: 
			\begin{enumerate}
				\item L'utente può visualizzare l'elenco dei questionari da svolgere in un secondo momento (UC9.1.1);
				\item L'utente può visualizzare i questionari che ha creato (UC9.1.2); 
				\item L'utente può visualizzare i questionari che ha svolto (UC9.1.3); 
			\end{enumerate}
	\end{itemize}
	
			\subsubsection{Caso d'uso UC9.1.1: Visualizza questionari della lista "Fai più tardi"}
			\label{UC9.1.4}
			\begin{figure}[h]
				\centering
				\includegraphics[scale=0.5,keepaspectratio]{UML/UC9.png}
				\caption{UC9.1.4: Visualizza questionari da svolgere più tardi}
			\end{figure}
			\FloatBarrier
			\begin{itemize}
				\item \textbf{Attori}: \uau, \uaupro;
				\item \textbf{Descrizione}: l'utente può visualizzare l'elenco dei questionari da lui selezionati per essere svolti in un secondo momento;
				\item \textbf{Precondizione}: l'utente ha selezionato l'opzione "Visualizza questionari della lista Fai più tardi" tra le possibilità proposte in UC9.1;
				\item \textbf{Postcondizione}: il sistema ha eseguito le opzioni scelte dall'utente;
				\item \textbf{Scenario principale}: 
				\begin{enumerate}
					\item L'utente può compilare il questionario (UC7);
					\item L'utente può eliminare un questionario dalla lista "Fai più tardi" (UC9.1.1.1).
				\end{enumerate}
			\end{itemize}
				
				\subsubsection{Caso d'uso UC9.1.1.1: Eliminare questionario dalla lista "Fai più tardi"}
				\label{UC9.1.4.1}
				\begin{figure}[h]
					\centering
					\includegraphics[scale=0.5,keepaspectratio]{UML/UC9.png}
					\caption{UC9.1.1.1: Eliminare questionario dalla lista "Fai più tardi"}
				\end{figure}
				\FloatBarrier
				\begin{itemize}
					\item \textbf{Attori}: \uau, \uaupro;
					\item \textbf{Descrizione}: l'utente ha la possibilità di eliminare il questionario selezionato dalla lista "Fai più tardi";
					\item \textbf{Precondizione}: l'utente ha selezionato l'opzione "Eliminare questionario dalla lista Fai più tardi" tra le possibilità proposte in UC9.1.1 su un questionario;
					\item \textbf{Postcondizione}: il sistema ha eliminato, dall'elenco dei questionari da svolgere più tardi, quello selezionato;
					\item \textbf{Scenario principale}: l'utente può confermare l'eliminazione del questionario dall'elenco dei questionari da svolgere più tardi (UC9.1.1.1.1).	
				\end{itemize}
				
					\subsubsection{Caso d'uso UC9.1.1.1.1: Confermare eliminazione questionario dalla lista "Fai più tardi"}
					\label{UC9.1.1.1.1}
					\begin{figure}[h]
						\centering
						\includegraphics[scale=0.5,keepaspectratio]{UML/UC9.png}
						\caption{UC9.1.1.1.1: Confermare eliminazione questionario dalla lista "Fai più tardi"}
					\end{figure}
					\FloatBarrier
					\begin{itemize}
						\item \textbf{Attori}: \uau, \uaupro;
						\item \textbf{Descrizione}: l'utente dopo aver scelto di eliminare il questionario deve confermare tale operazione;
						\item \textbf{Precondizione}: l'utente ha scelto di eliminare il questionario;
						\item \textbf{Postcondizione}: l'utente ha eliminato il questionario;
						\item \textbf{Scenario principale}: l'utente deve confermare l'eliminazione del questionario dalla lista "Fai più tardi". Una volta fatto ciò viene mandato nella pagina che contiene la lista dei questionari "Fai più tardi" (UC9.1.1);
						\item \textbf{Scenari alternativi}: L'utente non conferma l'eliminazione del questionario dalla lista "Fai più tardi" e viene mandato alla schermata precedente.
					\end{itemize}
							
		\subsubsection{Caso d'uso UC9.1.2: Visualizza questionari creati}
		\label{UC9.1.2}
		\begin{figure}[h]
			\centering
		\includegraphics[scale=0.5,keepaspectratio]{UML/UC9.png}
			\caption{UC9.1.2: Visualizza questionari creati}
		\end{figure}
		\FloatBarrier
		\begin{itemize}
			\item \textbf{Attori}: \uau, \uaupro;
			\item \textbf{Descrizione}: l'utente può visualizzare i questionari da lui creati;
			\item \textbf{Precondizione}: l'utente ha selezionato l'opzione "Visualizza questionari creati" tra le possibilità proposte in UC9.1;
			\item \textbf{Postcondizione}: il sistema ha eseguito le opzioni scelte dall'utente;
			\item \textbf{Scenario principale}: 
				\begin{enumerate}
					\item L'utente può modificare un questionario creato (UC9.1.2.1);
					\item L'utente può eliminare un questionario creato (UC9.1.2.2);
					\item L'utente può compilare un questionario (UC7).
				\end{enumerate}
		\end{itemize}
		
			\subsubsection{Caso d'uso UC9.1.2.1: Modifica questionario}
			\label{UC9.1.2.1}
			\begin{figure}[h]
				\centering
			\includegraphics[scale=0.5,keepaspectratio]{UML/UC9.png}
				\caption{UC9.1.2.1: Modifica questionario}
			\end{figure}
			\FloatBarrier
			\begin{itemize}
				\item \textbf{Attori}: \uau, \uaupro;
				\item \textbf{Descrizione}: l'utente può modificare il questionario selezionato;
				\item \textbf{Precondizione}: l'utente ha selezionato l'opzione "Modifica questionario" tra le possibilità proposte in UC9.1.2 su un questionario;
				\item \textbf{Postcondizione}: l'utente ha modificato il questionario selezionato; 
				\item \textbf{Scenario principale}:
					\begin{enumerate}
						\item L'utente può modificare la tipologia del questionario (UC9.1.2.1.1);
						\item L'utente può modificare il nome del questionario (UC9.1.2.1.2);
						\item L'utente può modificare gli argomenti del questionario (UC9.1.2.1.3);
						\item L'utente può gestire le domande del questionario (UC9.1.2.1.4);
						\item L'utente può confermare le modifiche fatte (UC9.1.2.1.5).
					\end{enumerate}
			\end{itemize}
			
					\subsubsection{Caso d'uso UC9.1.2.1.1: Modifica tipologia questionario}
					\label{UC9.1.2.1.1}
					\begin{figure}[h]
						\centering
					\includegraphics[scale=0.5,keepaspectratio]{UML/UC9.png}
						\caption{UC9.1.2.1.1: Modifica tipologia questionario}
					\end{figure}
					\FloatBarrier
					\begin{itemize}
						\item \textbf{Attori}: \uau, \uaupro;
						\item \textbf{Descrizione}: l'utente può modificare la tipologia del questionario facendolo passare da pubblico a privato e viceversa; 
						\item \textbf{Precondizione}: l'utente ha selezionato l'opzione "Modifica tipologia questionario" tra le scelte possibili in UC9.1.2.1;
						\item \textbf{Postcondizione}: l'utente ha modificato la tipologia del questionario;
						\item \textbf{Scenario principale}: l'utente modifica la tipologia del questionario;
						\item \textbf{Scenari alternativi}: l'utente non possiede un account pro e tenta di selezionare la tipologia privata per il questionario. Viene mandato in una pagina in cui può cambiare la tipologia del proprio account (UC5.7); 
					\end{itemize}
											
					\subsubsection{Caso d'uso UC9.1.2.1.2: Modifica nome questionario}
					\label{UC9.1.2.1.2}
					\begin{figure}[h]
						\centering
					\includegraphics[scale=0.5,keepaspectratio]{UML/UC9.png}
						\caption{UC9.1.2.1.2: Modifica nome questionario}
					\end{figure}
					\FloatBarrier
					\begin{itemize}
						\item \textbf{Attori}: \uau, \uaupro;
						\item \textbf{Descrizione}: l'utente può modificare il nome del questionario; 
						\item \textbf{Precondizione}: l'utente ha selezionato l'opzione "Modifica nome questionario" tra le scelte possibili in UC9.1.2.1;
						\item \textbf{Postcondizione}: l'utente ha modificato il nome del questionario; 
						\item \textbf{Scenario principale}: l'utente modifica il nome del questionario;
					\end{itemize}
					
					\subsubsection{Caso d'uso UC9.1.2.1.3: Modifica argomenti questionario}
					\label{UC9.1.2.1.3}
					\begin{figure}[h]
						\centering
					\includegraphics[scale=0.5,keepaspectratio]{UML/UC9.png}
						\caption{UC9.1.2.1.3: Modifica argomenti questionario}
					\end{figure}
					\FloatBarrier
					\begin{itemize}
						\item \textbf{Attori}: \uau, \uaupro;
						\item \textbf{Descrizione}: l'utente può decidere di modificare gli argomenti che riassumono il questionario, aggiungendone oppure togliendone qualcuno; 
						\item \textbf{Precondizione}: l'utente ha selezionato l'opzione "Modifica argomenti questionario" tra le scelte possibili in UC9.1.2.1; 
						\item \textbf{Postcondizione}: l'utente ha modificato gli argomenti che riassumono il questionario; 
						\item \textbf{Scenario principale}:
							\begin{enumerate}
								\item L'utente può creare un nuovo argomento (UC9.1.2.1.3.1);
								\item L'utente può aggiungere un nuovo argomento all'elenco di argomenti (UC9.1.2.1.3.2);
								\item L'utente può eliminare un argomento (UC9.1.2.1.3.3);
							\end{enumerate}
						\item \textbf{Scenari alternativi}: l'utente elimina tutti gli argomenti. In questo caso deve inserirne almeno uno e viene mandato al caso d'uso UC9.1.2.1.3.2.
					\end{itemize}
					
						\subsubsection{Caso d'uso UC9.1.2.1.3.1: Crea argomento}
						\label{UC9.1.2.1.3.1}
						\begin{figure}[h]
							\centering
						\includegraphics[scale=0.5,keepaspectratio]{UML/UC9.png}
							\caption{UC9.1.2.1.3.1: Crea argomento}
						\end{figure}
						\FloatBarrier
						\begin{itemize}
							\item \textbf{Attori}: \uau, \uaupro;
							\item \textbf{Descrizione}: l'utente può creare un nuovo argomento non presente tra quelli già esistenti;
							\item \textbf{Precondizione}: l'utente ha selezionato l'opzione "Crea argomento" tra le scelte possibili;
							\item \textbf{Postcondizione}: l'utente ha creato un nuovo argomento; 
							\item \textbf{Scenario principale}:
								\begin{enumerate}
									\item L'utente deve inserire il nome del nuovo argomento (UC9.1.2.1.3.1.1);
									\item L'utente deve confermare il nuovo argomento (UC9.1.2.1.3.1.2).
								\end{enumerate}
						\end{itemize}
						
							\subsubsection{Caso d'uso UC9.1.2.1.3.1.1: Inserisci nome nuovo argomento}
							\label{UC9.1.2.1.3.1.1}
							\begin{figure}[h]
								\centering
								\includegraphics[scale=0.5,keepaspectratio]{UML/UC9.png}
								\caption{UC9.1.2.1.3.1.1: Inserisci nome nuovo argomento}
							\end{figure}
							\FloatBarrier
							\begin{itemize}
								\item \textbf{Attori}: \uau, \uaupro;
								\item \textbf{Descrizione}: l'utente inserisce il nome del nuovo argomento;
								\item \textbf{Precondizione}: l'utente ha selezionato l'opzione "Inserisci nome utente" tra le scelte possibili in UC9.1.2.1.3.1;
								\item \textbf{Postcondizione}: l'utente ha inserito il nome del nuovo argomento;
								\item \textbf{Scenario principale}: l'utente inserisce il nome del nuovo argomento;
							\end{itemize}
							
							\subsubsection{Caso d'uso UC9.1.2.1.3.1.2: Conferma creazione nuovo argomento}
							\label{UC9.1.2.1.3.1.2}
							\begin{figure}[h]
								\centering
								\includegraphics[scale=0.5,keepaspectratio]{UML/UC9.png}
								\caption{UC9.1.2.1.3.1.2: Conferma creazione nuovo argomento}
							\end{figure}
							\FloatBarrier
							\begin{itemize}
								\item \textbf{Attori}: \uau, \uaupro;
								\item \textbf{Descrizione}: l'utente conferma il nuovo argomento;
								\item \textbf{Precondizione}: l'utente ha selezionato l'opzione "Conferma creazione nuovo argomento" tra le scelte possibili in UC9.1.2.1.3.1;
								\item \textbf{Postcondizione}: l'utente ha confermato la creazione di un nuovo argomento;
								\item \textbf{Scenario principale}: l'utente conferma la creazione di un nuovo argomento;
								\item \textbf{Scenari alternativi}: 
									\begin{enumerate}
										\item L'utente seleziona un nome che già esiste. Viene riportato alla situazione del caso d'uso UC9.1.2.1.3.1.1;
										\item L'utente annulla l'operazione e viene portato alla situazione del caso d'uso UC9.1.2.1.3;
									\end{enumerate}
						
							\end{itemize}
						
						\subsubsection{Caso d'uso UC9.1.2.1.3.2: Aggiungi argomento}
						\label{UC9.1.2.1.3.2}
						\begin{figure}[h]
							\centering
							\includegraphics[scale=0.5,keepaspectratio]{UML/UC9.png}
							\caption{UC9.1.2.1.3.2: Aggiungi argomento}
						\end{figure}
						\FloatBarrier
						\begin{itemize}
							\item \textbf{Attori}: \uau, \uaupro;
							\item \textbf{Descrizione}: l'utente può aggiungere altri argomenti selezionandoli tra quelli archiviati;  
							\item \textbf{Precondizione}: l'utente ha selezionato l'opzione "Aggiungi argomento" tra le scelte possibili;
							\item \textbf{Postcondizione}: l'utente ha selezionato dei nuovi argomenti;
							\item \textbf{Scenario principale}: l'utente può ricercare argomenti tra quelli archiviati (UC9.1.2.1.3.2.1);
						\end{itemize}
						
							\subsubsection{Caso d'uso UC9.1.2.1.3.2.1: Ricerca argomenti}
							\label{UC9.1.2.1.3.2.1}
							\begin{figure}[h]
								\centering
								\includegraphics[scale=0.5,keepaspectratio]{UML/UC9.png}
								\caption{UC9.1.2.1.3.2: Aggiungi argomento}
							\end{figure}
							\FloatBarrier
							\begin{itemize}
								\item \textbf{Attori}: \uau, \uaupro;
								\item \textbf{Descrizione}: l'utente può eseguire una ricerca tra gli argomenti archiviati; 
								\item \textbf{Precondizione}: l'utente ha selezionato l'opzione "Ricerca argomenti" tra le scelte possibili in UC9.1.2.1.3.2;
								\item \textbf{Postcondizione}: l'utente ha ricercato tra gli argomenti archiviati;
								\item \textbf{Scenario principale}: l'utente può selezionare gli argomenti (UC9.1.2.1.3.2.1.1); 
								\item \textbf{Scenari alternativi}: nel caso in cui non ci sia nessun risultato dalla ricerca all'utente viene proposta la possibilità di creare un nuovo argomento (UC9.1.2.1.3.1);
							\end{itemize}
							
								\subsubsection{Caso d'uso UC9.1.2.1.3.2.1.1: Seleziona argomenti}
								\label{UC9.1.2.1.3.2.2}
								\begin{figure}[h]
									\centering
									\includegraphics[scale=0.5,keepaspectratio]{UML/UC9.png}
									\caption{UC9.1.2.1.3.2.2: Aggiungi argomento}
								\end{figure}
								\FloatBarrier
								\begin{itemize}
									\item \textbf{Attori}: \uau, \uaupro;
									\item \textbf{Descrizione}: l'utente può selezionare gli argomenti tra quelli ottenuti dalla ricerca;
									\item \textbf{Precondizione}: l'utente ha ottenuto una lista di risultati;
									\item \textbf{Postcondizione}: l'utente ha selezionato degli argomenti; 
									\item \textbf{Scenario principale}: l'utente seleziona gli argomenti;
								\end{itemize}
						
						\subsubsection{Caso d'uso UC9.1.2.1.3.3: Elimina argomento}
						\label{UC9.1.2.1.3.3}
						\begin{figure}[h]
							\centering
							\includegraphics[scale=0.5,keepaspectratio]{UML/UC9.png}
							\caption{UC9.1.2.1.3.3: Elimina argomento}
						\end{figure}
						\FloatBarrier
						\begin{itemize}
							\item \textbf{Attori}: \uau, \uaupro;
							\item \textbf{Descrizione}: l'utente può eliminare un argomento da un questionario;
							\item \textbf{Precondizione}: l'utente ha selezionato l'opzione "Elimina argomento" tra le scelte possibili;
							\item \textbf{Postcondizione}: l'utente ha eliminato un argomento;
							\item \textbf{Scenario principale}: l'utente deve confermare di voler eliminare l'argomento; 
							\item \textbf{Scenari alternativi}: l'utente ha cancellato tutti gli argomenti. Deve allora inserirne almeno uno, viene allora rimandato a UC9.1.2.1.3.2.
						\end{itemize}
						
							\subsubsection{Caso d'uso UC9.1.2.1.3.3.1: Conferma elimina argomento}
							\label{UC9.1.2.1.3.3.1}
							\begin{figure}[h]
								\centering
								\includegraphics[scale=0.5,keepaspectratio]{UML/UC9.png}
								\caption{UC9.1.2.1.3.3.1: Conferma elimina argomento}
							\end{figure}
							\FloatBarrier
							\begin{itemize}
								\item \textbf{Attori}: \uau, \uaupro;
								\item \textbf{Descrizione}: l'utente deve confermare di voler eliminare l'argomento;
								\item \textbf{Precondizione}: l'utente ha deciso di voler eliminare l'argomento;
								\item \textbf{Postcondizione}: l'utente ha eliminato un argomento;
								\item \textbf{Scenario principale}: l'utente conferma di voler eliminare l'argomento. 
								\item \textbf{Scenari alternativi}: l'utente annulla l'eliminazione dell'argomento, viene allora rimandato a UC9.1.2.1.3.
							\end{itemize}						
						
					\subsubsection{Caso d'uso UC9.1.2.1.4: Gestione domande}
					\label{UC9.1.2.1.4}
					\begin{figure}[h]
						\centering
					\includegraphics[scale=0.5,keepaspectratio]{UML/UC9.png}
						\caption{UC9.1.2.1.4: Gestione domande}
					\end{figure}
					\FloatBarrier
					\begin{itemize}
						\item \textbf{Attori}: \uau, \uaupro;
						\item \textbf{Descrizione}: l'utente può gestire le domande di un proprio questionario, aggiungendone oppure togliendone;
						\item \textbf{Precondizione}: l'utente ha selezionato l'opzione "Gestione domande" tra le scelte possibili in UC9.1.2.1.4;
						\item \textbf{Postcondizione}: l'utente ha gestito le domande di un questionario;
						\item \textbf{Scenario principale}: 
							\begin{enumerate}
								\item L'utente aggiunge altre domande (UC9.1.2.1.4.1);
								\item L'utente elimina una domanda (UC9.1.2.1.4.2).
							\end{enumerate}
					\end{itemize}
					
						\subsubsection{Caso d'uso UC9.1.2.1.4.1: Aggiungi altre domande}
						\label{UC9.1.2.1.4.1}
						\begin{figure}[h]
							\centering
						\includegraphics[scale=0.5,keepaspectratio]{UML/UC9.png}
							\caption{UC9.1.2.1.4.1: Aggiungi altre domande}
						\end{figure}
						\FloatBarrier
						\begin{itemize}
							\item \textbf{Attori}: \uau, \uaupro;
							\item \textbf{Descrizione}: l'utente può aggiungere altre domande cercando tra quelle già memorizzate nell'archivio oppure creandone una nuova; 
							\item \textbf{Precondizione}: l'utente ha selezionato l'opzione "Aggiungi altre domande" tra le scelte possibili in UC9.1.2.1.4;
							\item \textbf{Postcondizione}: l'utente ha aggiunto nuove domande;
							\item \textbf{Scenario principale}:
								\begin{enumerate}
									\item L'utente può eseguire una ricerca nelle domande archiviate (UC9.1.2.1.4.1.1);
									\item L'utente può creare una nuova domanda (UC8.1);
								\end{enumerate}
						\end{itemize}
						
							\subsubsection{Caso d'uso UC9.1.2.1.4.1.1: Ricerca domande}
							\label{UC9.1.2.1.4.1.1}
							\begin{figure}[h]
								\centering
								\includegraphics[scale=0.5,keepaspectratio]{UML/UC9.png}
								\caption{UC9.1.2.1.4.1.1: Ricerca domande}
							\end{figure}
							\FloatBarrier
							\begin{itemize}
								\item \textbf{Attori}: \uau, \uaupro;
								\item \textbf{Descrizione}: l'utente può eseguire una ricerca tra le domande archiviate; 
								\item \textbf{Precondizione}: l'utente ha selezionato l'opzione "Ricerca domande" tra le scelte possibili in UC9.1.2.1.4.1;
								\item \textbf{Postcondizione}: l'utente ha ricercato tra le domande archiviate;
								\item \textbf{Scenario principale}: l'utente può selezionare le domande (UC9.1.2.1.4.1.1.1); 
								\item \textbf{Scenari alternativi}: nel caso in cui non ci sia nessun risultato dalla ricerca all'utente viene proposta la possibilità di creare una nuova domanda (UC8.1);
							\end{itemize}
							
								\subsubsection{Caso d'uso UC9.1.2.1.4.1.1.1: Seleziona domande}
								\label{UC9.1.2.1.4.1.1.1}
								\begin{figure}[h]
									\centering
									\includegraphics[scale=0.5,keepaspectratio]{UML/UC9.png}
									\caption{UC9.1.2.1.4.1.1.1: Seleziona domande}
								\end{figure}
								\FloatBarrier
								\begin{itemize}
									\item \textbf{Attori}: \uau, \uaupro;
									\item \textbf{Descrizione}: l'utente può selezionare le domande tra quelle ottenute dalla ricerca;
									\item \textbf{Precondizione}: l'utente ha ottenuto una lista di risultati;
									\item \textbf{Postcondizione}: l'utente ha selezionato delle domande; 
									\item \textbf{Scenario principale}: l'utente seleziona le domande;
								\end{itemize}
							
						\subsubsection{Caso d'uso UC9.1.2.1.4.2: Elimina domanda}
						\label{UC9.1.2.1.4.2}
						\begin{figure}[h]
							\centering
						\includegraphics[scale=0.5,keepaspectratio]{UML/UC9.png}
							\caption{UC9.1.2.1.4.2: Elimina domande}
						\end{figure}
						\FloatBarrier
						\begin{itemize}
							\item \textbf{Attori}: \uau, \uaupro;
							\item \textbf{Descrizione}: l'utente può eliminare una domanda da un questionario;
							\item \textbf{Precondizione}: l'utente ha selezionato l'opzione "Elimina domanda" tra le scelte possibili in UC9.1.2.1.4;
							\item \textbf{Postcondizione}: l'utente ha eliminato una domanda;
							\item \textbf{Scenario principale}: l'utente deve confermare di voler eliminare la domanda; 
							\item \textbf{Scenari alternativi}: l'utente ha cancellato tutte le domande. Deve allora inserirne almeno una, viene allora rimandato a UC9.1.2.1.4.1.
						\end{itemize}
						
							\subsubsection{Caso d'uso UC9.1.2.1.4.2.1: Conferma eliminazione domanda}
							\label{UC9.1.2.1.4.2.1}
							\begin{figure}[h]
								\centering
								\includegraphics[scale=0.5,keepaspectratio]{UML/UC9.png}
								\caption{UC9.1.2.1.4.2.1: Conferma eliminazione domanda}
							\end{figure}
							\FloatBarrier
							\begin{itemize}
								\item \textbf{Attori}: \uau, \uaupro;
								\item \textbf{Descrizione}: l'utente deve confermare di voler eliminare la domanda;
								\item \textbf{Precondizione}: l'utente ha deciso di voler eliminare la domanda;
								\item \textbf{Postcondizione}: l'utente ha eliminato una domanda;
								\item \textbf{Scenario principale}: l'utente conferma di voler eliminare la domanda;
								\item \textbf{Scenari alternativi}: l'utente annulla l'eliminazione della domanda, viene allora rimandato a UC9.1.2.1.4.
							\end{itemize}	
																
					\subsubsection{Caso d'uso UC9.1.2.1.5: Conferma modifiche}
					\label{UC9.1.2.1.6}
					\begin{figure}[h]
						\centering
					\includegraphics[scale=0.5,keepaspectratio]{UML/UC9.png}
						\caption{UC9.1.2.1.6: Conferma modifiche}
					\end{figure}
					\FloatBarrier
					\begin{itemize}
						\item \textbf{Attori}: \uau, \uaupro;
						\item \textbf{Descrizione}: l'utente ha eseguito tutte le modifiche e ora deve salvare in modo che siano archiviate;
						\item \textbf{Precondizione}: l'utente ha selezionato l'opzione "Conferma modifiche" tra le scelte possibili in UC9.1.2.1;
						\item \textbf{Postcondizione}: l'utente ha confermato di volere salvare le modifiche fatte;
						\item \textbf{Scenario principale}: l'utente conferma di volere salvare le modifiche fatte;
						\item \textbf{Scenari alternativi}: l'utente non conferma e le modifiche fatte non vengono salvate. L'utente viene indirizzato alla pagina contenente la lista dei questionari da lui creati (UC9.1.2).
					\end{itemize}
										
			\subsubsection{Caso d'uso UC9.1.2.2: Elimina questionario}
			\label{UC9.1.2.2}
			\begin{figure}[h]
				\centering
			\includegraphics[scale=0.5,keepaspectratio]{UML/UC9.png}
				\caption{UC9.1.2.2: Elimina questionario}
			\end{figure}
			\FloatBarrier
			\begin{itemize}
				\item \textbf{Attori}: \uau, \uaupro;
				\item \textbf{Descrizione}: l'utente decide di voler eliminare il questionario dall'archivio di questionari;
				\item \textbf{Precondizione}: l'utente ha selezionato l'opzione "Elimina questionario" tra le scelte possibili in UC9.1.2;
				\item \textbf{Postcondizione}: l'utente ha eliminato il questionario;
				\item \textbf{Scenario principale}:l'utente deve confermare di voler eliminare il questionario (UC9.1.2.2.1);
			\end{itemize}
			
				\subsubsection{Caso d'uso UC9.1.2.2.1: Conferma eliminazione}
				\label{UC9.1.2.2.1}
				\begin{figure}[h]
					\centering
				\includegraphics[scale=0.5,keepaspectratio]{UML/UC9.png}
					\caption{UC9.1.2.2.1: Conferma eliminazione}
				\end{figure}
				\FloatBarrier
				\begin{itemize}
					\item \textbf{Attori}: \uau, \uaupro;
					\item \textbf{Descrizione}: l'utente deve confermare di voler eliminare il questionario; 
					\item \textbf{Precondizione}: l'utente ha selezionato l'opzione "Conferma eliminazione" tra le scelte possibili in UC9.1.2.2;
					\item \textbf{Postcondizione}: l'utente ha confermato di voler eliminare il questionario;
					\item \textbf{Scenario principale}: l'utente conferma di voler eliminare il questionario;
					\item \textbf{Scenari alternativi}: l'utente non conferma di voler eliminare il questionario. L'utente viene indirizzato alla pagina contenente la lista dei questionari da lui creati (UC9.1.2).
				\end{itemize}
								
		\subsubsection{Caso d'uso UC9.1.3: Visualizza cronologia questionari svolti}
		\label{UC9.1.3}
		\begin{figure}[h]
			\centering
			\includegraphics[scale=0.5,keepaspectratio]{UML/UC9.png}
			\caption{UC9.1.3: Visualizza cronologia questionari svolti}
		\end{figure}
		\FloatBarrier
		\begin{itemize}
			\item \textbf{Attori}: \uau, \uaupro;
			\item \textbf{Descrizione}: l'utente può visualizzare l'elenco dei questionari che ha svolto;
			\item \textbf{Precondizione}: l'utente ha selezionato l'opzione "Visualizza cronologia questionari svolti" tra le possibilità proposte in UC9.1;
			\item \textbf{Postcondizione}: il sistema ha eseguito le opzioni scelte dall'utente;
			\item \textbf{Scenario principale}: 
			\begin{enumerate}
				\item L'utente può compilare un questionario di nuovo (UC7);
				\item L'utente può visualizzare le statistiche di un questionario (UC9.1.3.1);
			\end{enumerate}
		\end{itemize}
		
				\subsubsection{Caso d'uso UC9.1.3.1: Visualizzare le statistiche di un questionario}
				\label{UC9.1.3.1}
				\begin{figure}[h]
					\centering
					\includegraphics[scale=0.5,keepaspectratio]{UML/UC9.png}
					\caption{UC9.1.3.1: Visualizzare le statistiche di un questionario}
				\end{figure}
				\FloatBarrier
				\begin{itemize}
					\item \textbf{Attori}: \uau, \uaupro; 
					\item \textbf{Descrizione}: l'utente visualizza le statistiche di un questionario;
					\item \textbf{Precondizione}: l'utente ha scelto l'opzione "Visualizzare le statistiche di un questionario" tra le scelte possibili in UC9.1.3;
					\item \textbf{Postcondizione}: l'utente ha visualizzato le statistiche di un questionario; 
					\item \textbf{Scenario principale}: l'utente visualizza le statistiche di un questionario;
				\end{itemize}
						

				
	\subsubsection{Caso d'uso UC9.2: Crea questionari}
	\label{UC9.2}
	\begin{figure}[h]
		\centering
	\includegraphics[scale=0.5,keepaspectratio]{UML/UC9.png}
		\caption{UC9.2: Crea questionari}
	\end{figure}
	\FloatBarrier
	\begin{itemize}
		\item \textbf{Attori}: \uau, \uaupro;
		\item \textbf{Descrizione}: l'utente può creare un nuovo questionario; 
		\item \textbf{Precondizione}: l'utente ha selezionato l'opzione "Crea questionari" tra le possibilità proposte in UC9;
		\item \textbf{Postcondizione}: l'utente ha creato un questionario;
		\item \textbf{Scenario principale}:
			\begin{enumerate}
				\item L'utente può selezionare la tipologia del questionario (UC9.2.1);
				\item L'utente può inserire il nome del questionario (UC9.2.2);
				\item L'utente può selezionare gli argomenti del questionario (UC9.2.3);
				\item L'utente può gestire le domande (UC9.2.4);
				\item L'utente può concludere il questionario (UC9.2.5);
			\end{enumerate}
	\end{itemize}
	
		\subsubsection{Caso d'uso UC9.2.1: Seleziona tipologia questionario}
		\label{UC9.2.1}
		\begin{figure}[h]
			\centering
		\includegraphics[scale=0.5,keepaspectratio]{UML/UC9.png}
			\caption{UC9.2.1: Seleziona tipologia questionario}
		\end{figure}
		\FloatBarrier
		\begin{itemize}
			\item \textbf{Attori}: \uau, \uaupro;
			\item \textbf{Descrizione}: l'utente può selezionare la tipologia del questionario; 
			\item \textbf{Precondizione}: l'utente ha selezionato l'opzione "Seleziona tipologia questionario" tra le scelte possibili in UC9.2;
			\item \textbf{Postcondizione}: l'utente ha selezionato la tipologia del questionario;
			\item \textbf{Scenario principale}: l'utente seleziona la tipologia del questionario;
			\item \textbf{Scenari alternativi}: l'utente non possiede un account pro e tenta di selezionare la tipologia privata per il questionario. Viene mandato in una pagina in cui può cambiare la tipologia del proprio account (UC5.7); 
		\end{itemize}
			
		\subsubsection{Caso d'uso UC9.2.2: Inserisci nome questionario}
		\label{UC9.2.2}
		\begin{figure}[h]
			\centering
		\includegraphics[scale=0.5,keepaspectratio]{UML/UC9.png}
			\caption{UC9.2.2: Inserisci nome questionario}
		\end{figure}
		\FloatBarrier
		\begin{itemize}
			\item \textbf{Attori}: \uau, \uaupro;
			\item \textbf{Descrizione}: l'utente può inserire il nome del questionario; 
			\item \textbf{Precondizione}: l'utente ha selezionato l'opzione "Inserisci nome questionario" tra le scelte possibili in UC9.2;
			\item \textbf{Postcondizione}: l'utente ha inserito il nome del questionario; 
			\item \textbf{Scenario principale}: l'utente inserisce il nome del questionario;
		\end{itemize}
		
		\subsubsection{Caso d'uso UC9.2.3: Seleziona argomenti questionario}
		\label{UC9.2.3}
		\begin{figure}[h]
			\centering
		\includegraphics[scale=0.5,keepaspectratio]{UML/UC9.png}
			\caption{UC9.2.3: Seleziona argomenti questionario}
		\end{figure}
		\FloatBarrier
		\begin{itemize}
			\item \textbf{Attori}: \uau, \uaupro;
			\item \textbf{Descrizione}: l'utente può selezionare gli argomenti che riassumono il questionario;
			\item \textbf{Precondizione}: l'utente ha selezionato l'opzione "Seleziona argomenti questionario" tra le scelte possibili in UC9.2; 
			\item \textbf{Postcondizione}: l'utente ha selezionato gli argomenti che riassumono il questionario; 
			\item \textbf{Scenario principale}:
			\begin{enumerate}
				\item L'utente può creare un nuovo argomento (UC9.1.2.1.3.1);
				\item L'utente può aggiungere un nuovo argomento all'elenco di argomenti (UC9.1.2.1.3.2);
				\item L'utente può eliminare un argomento (UC9.1.2.1.3.3);
			\end{enumerate}
			\item \textbf{Scenari alternativi}: l'utente elimina tutti gli argomenti. In questo caso deve inserirne almeno uno e viene mandato al caso d'uso UC9.1.2.1.3.1.
		\end{itemize}
							
		\subsubsection{Caso d'uso UC9.2.5: Gestione domande}
		\label{UC9.2.5}
		\begin{figure}[h]
			\centering
			\includegraphics[scale=0.5,keepaspectratio]{UML/UC9.png}
			\caption{UC9.2.5: Gestione domande}
		\end{figure}
		\FloatBarrier
		\begin{itemize}
			\item \textbf{Attori}: \uau, \uaupro;
			\item \textbf{Descrizione}: l'utente può gestire le domande di un proprio questionario, aggiungendone oppure togliendone;
			\item \textbf{Precondizione}: l'utente ha selezionato l'opzione "Gestione domande" tra le scelte possibili in UC9.2;
			\item \textbf{Postcondizione}: l'utente ha gestito le domande di un questionario;
			\item \textbf{Scenario principale}: 
			\begin{enumerate}
				\item L'utente aggiunge altre domande (UC9.1.2.1.4.1);
				\item L'utente elimina una domanda (UC9.1.2.1.4.2).
			\end{enumerate}
		\end{itemize}
		
		\subsubsection{Caso d'uso UC9.2.6: Conclusione questionario}
		\label{UC9.2.6}
		\begin{figure}[h]
			\centering
			\includegraphics[scale=0.5,keepaspectratio]{UML/UC9.png}
			\caption{UC9.2.6: Conclusione questionario}
		\end{figure}
		\FloatBarrier
		\begin{itemize}
			\item \textbf{Attori}: \uau, \uaupro; 
			\item \textbf{Descrizione}: l'utente decide che il questionario è concluso;
			\item \textbf{Precondizione}: l'utente ha scelto l'opzione "Conclusione questionario" tra le scelte possibili in UC9.2;
			\item \textbf{Postcondizione}: l'utente ha completato il questionario;
			\item \textbf{Scenario principale}: 
				\begin{enumerate}
					\item L'utente visualizza il riepilogo finale del questionario appena creato (UC9.2.5.1); 
					\item L'utente approva la creazione del questionario (UC9.2.5.2); 
				\end{enumerate}
		\end{itemize}
				
			\subsubsection{Caso d'uso UC9.2.5.1: Resoconto questionario}
			\label{UC9.2.5.1}
			\begin{figure}[h]
				\centering
			\includegraphics[scale=0.5,keepaspectratio]{UML/UC9.png}
				\caption{UC9.2.5.1: Resoconto questionario}
			\end{figure}
			\FloatBarrier
			\begin{itemize}
				\item \textbf{Attori}: \uau, \uaupro;
				\item \textbf{Descrizione}: l'utente visualizza le scelte fatte finora per la creazione del questionario;
				\item \textbf{Precondizione}: l'utente ha scelto l'opzione "Resoconto questionario" tra le scelte possibili in UC9.2.5;
				\item \textbf{Postcondizione}: l'utente ha visualizzato le scelte fatte finora per la creazione del questionario;
				\item \textbf{Scenario principale}: l'utente visualizza le scelte fatte finora per la creazione del questionario;
			\end{itemize}
			
			\subsubsection{Caso d'uso UC9.2.5.2: Approvazione questionario}
			\label{UC9.2.5.2}
			\begin{figure}[h]
				\centering
			\includegraphics[scale=0.5,keepaspectratio]{UML/UC9.png}
				\caption{UC9.2.5.2: Approvazione questionario}
			\end{figure}
			\FloatBarrier
			\begin{itemize}
				\item \textbf{Attori}: \uau, \uaupro;
				\item \textbf{Descrizione}: l'utende deve approvare il questionario appena creato;
				\item \textbf{Precondizione}: l'utente ha scelto l'opzione "Approvazione questionario" tra le scelte possibili in UC9.2.5; 
				\item \textbf{Postcondizione}: l'utente ha approvato il questionario;
				\item \textbf{Scenario principale}: l'utente approva il questionario;
				\item \textbf{Scenari alternativi}: l'utente non approva il questionario e quest'ultimo non viene archiviato. L'utente viene mandato alla pagina precedente.
			\end{itemize}
	 
	
	

\newpage
\subsection{Caso d'uso UC10: Modalità allenamento}
\label{UC10}
	\begin{figure}
	\centering
	\includegraphics[scale=0.5]{UML/UC10.png}
	\caption{UC10: Modalità allenamento}
	\end{figure}
\FloatBarrier
\begin{itemize}
\item\textbf{Attori}: utente non autenticato, utente autenticato, utente autenticato pro;
\item\textbf{Descrizione}: l'attore può svolgere un allenamento che consiste nel rispondere a domande che gli vengono proposte in modo dinamico. La difficoltà di ogni domanda viene calcolata in modo che essa sia sempre in linea con la competenza dell'utente sull'argomento scelto;
\item\textbf{Precondizione}: l'attore ha selezionato l'opzione di allenamento;
\item\textbf{Postcondizione}: l'attore ha finito di svolgere l'allenamento e può quindi visualizzare la valutazione finale che ha ottenuto e il riepilogo delle risposte date nonché il proprio livello sugli argomenti;
\item\textbf{Scenario principale}:
	\begin{enumerate}
		\item L'attore può scegliere l'argomento per iniziare l'allenamento (UC10.1);
		\item L'attore può scegliere delle parole chiave (UC10.2);
		\item L'attore può scegliere il numero di domande che compongono l'allenamento (UC10.3);
		\item L'attore può iniziare l'allenamento (UC10.4).
	\end{enumerate}
\item \textbf{Inclusioni}: l'attore risponde ad una domanda (UC12).
\end{itemize}

\subsubsection{Caso d'uso UC10.1: Scelta argomento per l'allenamento}
	\begin{itemize}
		\item \textbf{Attori}: utente non autenticato, utente autenticato, utente autenticato pro;
		\item \textbf{Descrizione}: l'attore può scegliere l'argomento delle domande che compariranno nell'allenamento;
		\item \textbf{Precondizione}: l'attore ha selezionato la modalità allenamento;
		\item \textbf{Postcondizione}: l'attore ha scelto un argomento per l'allenamento;
		\item \textbf{Scenario Principale}: l'attore sceglie l'argomento sul quale vuole allenarsi.
	\end{itemize}
	
\subsubsection{Caso d'uso UC10.2: Scelta parole chiave}
	\begin{itemize}
		\item \textbf{Attori}: utente non autenticato, utente autenticato, utente autenticato pro;
		\item \textbf{Descrizione}: l'attore può scegliere le parole chiave, ovvero degli argomenti più specifici, delle domande che compariranno nell'allenamento;
		\item \textbf{Precondizione}: l'attore ha selezionato la modalità allenamento;
		\item \textbf{Postcondizione}: l'attore ha scelto delle parole chiave;
		\item \textbf{Scenario principale}: l'attore sceglie delle parole chiave per filtrare in modo più dettagliato possibile l'argomento delle domande proposte per l'allenamento.
	\end{itemize}
	
\subsubsection{Caso d'uso UC10.3: Scelta numero di domande}
	\begin{itemize}
		\item \textbf{Attori}: utente non autenticato, utente autenticato, utente autenticato pro;
		\item \textbf{Descrizione}: l'attore può scegliere il numero di domande che comporranno l'allenamento, senza restrizione sul numero;
		\item \textbf{Precondizione}: l'attore ha selezionato la modalità allenamento;
		\item \textbf{Postcondizione}: l'attore ha scelto il numero di domande che comporranno l'allenamento;
		\item \textbf{Scenario principale}: l'attore sceglie il numero di domande che comporranno l'allenamento;
		\item \textbf{Scenari alternativi}: l'attore non sceglie un numero di domande che saranno quindi infinite.
	\end{itemize}
	
\subsubsection{Caso d'uso UC10.4: Inizia allenamento}
\label{UC10.4}
\begin{figure}
	\centering
	\includegraphics[scale=0.5]{UML/UC10_4.png}
	\caption{UC10.4: Inizia allenamento}
\end{figure}
\FloatBarrier
	\begin{itemize}
		\item \textbf{Attori}: utente non autenticato, utente autenticato, utente autenticato pro;
		\item \textbf{Descrizione}: l'attore può iniziare l'allenamento;
		\item \textbf{Precondizione}: l'attore ha scelto i parametri che consentiranno la creazione dell'allenamento personalizzato;
		\item \textbf{Postcondizione}: l'attore ha iniziato l'allenamento;
		\item \textbf{Scenario principale}: 
			\begin{enumerate}
				\item L'attore può confermare la risposta (UC10.4.1);
				\item L'attore può lasciare un like ad una domanda (UC10.4.2);
				\item L'attore può lasciare dei commenti alla domanda (UC10.4.3);
				\item L'attore può segnalare la domanda (UC10.4.4);
				\item L'attore può avanzare alla domanda successiva (UC10.4.5);
				\item L'utente può concludere l'allenamento (UC10.4.6).
			\end{enumerate}	
	\end{itemize}
	
\subsubsection{Caso d'uso UC10.4.1: Conferma risposta}
	\begin{itemize}
		\item \textbf{Attori}: utente non autenticato, utente autenticato, utente autenticato pro;
		\item \textbf{Descrizione}: l'attore può confermare la risposta data;
		\item \textbf{Precondizione}: l'attore ha risposto ad una domanda;
		\item \textbf{Postcondizione}: l'attore ha confermato la risposta e prosegue con l'allenamento;
		\item \textbf{Scenario principale}: l'attore conferma la risposta data alla domanda corrente.
	\end{itemize}
	
\subsubsection{Caso d'uso UC10.4.2: Like alla domanda}
	\begin{itemize}
		\item \textbf{Attori}: utente non autenticato, utente autenticato, utente autenticato pro;
		\item \textbf{Descrizione}: l'attore può lasciare un like a una domanda proposta;
		\item \textbf{Precondizione}: l'attore ha confermato la risposta ad una domanda;
		\item \textbf{Postcondizione}: l'attore ha lasciato un like alla domanda proposta;
		\item \textbf{Scenario principale}: l'attore lascia un like ad una domanda.
	\end{itemize}
	
\subsubsection{Caso d'uso UC10.4.3: Commenti alla domanda}
\label{UC10.4.3}
\begin{figure}
	\centering
	\includegraphics[scale=0.5]{UML/UC10_4_3.png}
	\caption{UC10.4.3: Commenti alla domanda}
\end{figure}
	\begin{itemize}
		\item \textbf{Attori}: utente non autenticato, utente autenticato, utente autenticato pro;
		\item \textbf{Descrizione}: l'attore può commentare una domanda;
		\item \textbf{Precondizione}: l'attore ha confermato la risposta ad una domanda;
		\item \textbf{Postcondizione}: l'attore ha commentato una domanda;
		\item \textbf{Scenario principale}:
			\begin{enumerate}
				\item L'attore scrive un commento sulla domanda (UC10.4.3.1);
				\item L'attore conferma il rilascio del commento (UC10.4.3.2).
			\end{enumerate}
	\end{itemize}
\subsubsection{Caso d'uso UC10.4.3.1: Scrivi commento}
	\begin{itemize}
		\item \textbf{Attori}: utente non autenticato, utente autenticato, utente autenticato pro;
		\item \textbf{Descrizione}: l'attore può scrivere il commento alla domanda proposta;
		\item \textbf{Precondizione}: l'attore ha confermato la risposta ad una domanda;
		\item \textbf{Postcondizione}: l'attore ha scritto un commento;
		\item \textbf{Scenario principale}: l'attore scrive un commento.
	\end{itemize}
\subsubsection{Caso d'uso UC10.4.3.2: Conferma rilascio commento}
	\begin{itemize}
		\item \textbf{Attori}: utente non autenticato, utente autenticato, utente autenticato pro;
		\item \textbf{Descrizione}: l'attore può confermare il rilascio del commento scritto;
		\item \textbf{Precondizione}: l'attore ha scritto un commento;
		\item \textbf{Postcondizione}: l'attore ha commentato una domanda;
		\item \textbf{Scenario principale}: l'attore conferma di voler inserire il commento scritto.
	\end{itemize}
\subsubsection{Caso d'uso UC10.4.4: Segnalare la domanda}
\label{UC10.4.4}
\begin{figure}
	\centering
	\includegraphics[scale=0.5]{UML/UC10_4_4.png}
	\caption{UC10.4.4: Segnalare una domanda}
\end{figure}
\FloatBarrier
	\begin{itemize}
		\item \textbf{Attori}: utente non autenticato, utente autenticato, utente autenticato pro;
		\item \textbf{Descrizione}: l'attore può segnalare una domanda se ritiene che sia sbagliata, mal formata oppure contenga contenuti non appropriati;
		\item \textbf{Precondizione}: l'attore ha confermato la risposta ad una domanda;
		\item \textbf{Postcondizione}: l'attore ha segnalato una domanda;
		\item \textbf{Scenario principale}:		
				\begin{enumerate}
					\item L'attore sceglie il tipo di segnalazione tra quelli proposti (UC10.4.4.1);
					\item L'attore scrive il contenuto della segnalazione (UC10.4.4.2).
				\end{enumerate}
	\end{itemize}
\subsubsection{Caso d'uso UC10.4.4.1: Tipo di segnalazione}
	\begin{itemize}
		\item \textbf{Attori}: utente non autenticato, utente autenticato, utente autenticato pro;
		\item \textbf{Descrizione}: l'attore può decidere il tipo di segnalazione tra quelli proposti;
		\item \textbf{Precondizione}: l'attore ha scelto di segnalare una domanda;
		\item \textbf{Postcondizione}: l'attore ha scelto un tipo di segnalazione;
		\item \textbf{Scenario principale}: l'attore sceglie un tipo di segnalazione.
	\end{itemize}
\subsubsection{Caso d'uso UC10.4.4.2: Scrivi segnalazione}
	\begin{itemize}
		\item \textbf{Attori}: utente non autenticato, utente autenticato, utente autenticato pro;
		\item \textbf{Descrizione}: l'attore può scrivere un commento sulla segnalazione per approfondire la motivazione;
		\item \textbf{Precondizione}: l'attore ha scelto di segnalare una domanda;
		\item \textbf{Postcondizione}: l'attore ha segnalato una domanda;
		\item \textbf{Scenario principale}: l'attore specifica perché vuole segnalare la domanda.
	\end{itemize}
\subsubsection{Caso d'uso UC10.4.5: Avanzamento domanda successiva}
	\begin{itemize}
		\item \textbf{Attori}: utente non autenticato, utente autenticato, utente autenticato pro;
		\item \textbf{Descrizione}: l'attore può avanzare alla domanda successiva dopo aver risposto alla domanda proposta; 
		\item \textbf{Precondizione}: l'attore ha risposto alla domanda proposta;
		\item \textbf{Postcondizione}: l'attore visualizza la domanda successiva;
		\item \textbf{Scenario principale}: l'attore seleziona il comando per passare alla domanda successiva.
	\end{itemize}
\subsubsection{Caso d'uso UC10.4.6: Conclusione allenamento}
	\begin{itemize}
		\item \textbf{Attori}: utente non autenticato, utente autenticato, utente autenticato pro;
		\item \textbf{Descrizione}: l'attore può concludere l'allenamento in qualunque punto dell'allenamento, anche se inizialmente aveva stabilito il numero di domande da affrontare;
		\item \textbf{Precondizione}: l'attore sta eseguendo un allenamento;
		\item \textbf{Postcondizione}: l'attore ha concluso l'allenamento e visualizza il risultato finale e le statistiche;
		\item \textbf{Scenario principale}: l'attore seleziona il comando per concludere l'allenamento.
	\end{itemize}

\newpage
\subsection{Caso d'uso UC11: Visualizzazione pagina utente}
\label{UC11}
\begin{figure}[h]
	\centering
	\includegraphics[scale=0.5]{UML/UC11.png}
	\caption{UC11: Gestione pagina utente}
\end{figure}

\begin{itemize}
\item\textbf{Attori}: utente autenticato, utente autenticato pro;
\item\textbf{Descrizione}: l'attore da questa pagina può: 
\begin{itemize}
	\item Visualizzare tutti i dati del loro profilo;
	\item Visualizzare le statistiche dei questionari svolti;
	\item Visualizzare la cronologia dei questionari svolti;
	\item Accedere alla parte del sistema che permette di modificare il profilo;
	\item Accedere alla parte del sistema che permette l'aggiunta di nuove domande;
	\item Accedere alla parte del sistema che permette la creazione di questionari.
\end{itemize}
\item\textbf{Precondizione}: l'attore è entrato nella pagina di visualizzazione del profilo e il sistema è pronto per mostrare i dati;
\item\textbf{Postcondizione}: l'attore ha visualizzato i suoi dati come richiesto al sistema;
\item\textbf{Scenario principale}:
\begin{itemize}
\item L'attore ha scelto di andare alla pagina di gestione del profilo (UC11.1);
\item L'attore ha scelto di andare alla pagina di gestione delle domande (UC11.2);  
\item L'attore ha scelto di andare alla pagina di gestione dei questionari (UC11.3);
\item L'attore ha scelto di andare alla pagina di visualizzazione della cronologia dei questionari svolti (UC11.4);
\item L'attore ha scelto di andare alla pagina di visualizzazione dei questionari abilitati (UC11.5);
\end{itemize}
\item\textbf{Inclusioni}: Viene incluso l'UC7 per la compilazione dei questionari.
\end{itemize}

\subsubsection{Caso d'uso UC11.1: Collegamento gestione profilo}
\begin{itemize}
\item\textbf{Attori}: utente autenticato, utente autenticato pro;
\item\textbf{Descrizione}: l'attore viene portato nella pagina di gestione del profilo dove potrà modificare liberamente tutti i suoi dati;
\item\textbf{Precondizione}: l'attore ha premuto l'apposito link di gestione del profilo;
\item\textbf{Postcondizione}: il sistema ha portato l'attore alla pagina di gestione del profilo;
\item\textbf{Scenario principale}: l'attore si trova nella pagina di gestione del profilo e potrà attuare tutte le modifiche desiderate.
\end{itemize}

\subsubsection{Caso d'uso UC11.2: Collegamento gestione delle domande}
\begin{itemize}
\item\textbf{Attori}: utente autenticato, utente autenticato pro;
\item\textbf{Descrizione}: l'attore viene portato nella pagina di gestione delle domande dove potrà inserire o modificare le sue domande;
\item\textbf{Precondizione}: l'attore ha premuto l'apposito link di gestione delle domande;
\item\textbf{Postcondizione}: il sistema ha portato l'attore alla pagina di gestione delle domande;
\item\textbf{Scenario principale}: l'attore si trova nella pagina di gestione delle domande e potrà attuare tutte le modifiche desiderate.
\end{itemize}

\subsubsection{Caso d'uso UC11.3: Collegamento gestione questionari}
\begin{itemize}
\item\textbf{Attori}: utente autenticato, utente autenticato pro;
\item\textbf{Descrizione}: l'attore viene portato nella pagina di gestione dei questionari dove potranno compiere tutte le azioni possbili sui questionari;
\item\textbf{Precondizione}: l'attore ha premuto l'apposito link di gestione dei questionari;
\item\textbf{Postcondizione}: il sistema ha portato l'attore alla pagina di gestione dei questionari;
\item\textbf{Scenario principale}: l'attore si trova nella pagina di gestione dei questionari e potrà attuare tutte le modifiche desiderate.
\end{itemize}

\subsubsection{Caso d'uso UC11.4: Visualizzazione cronologia questionari svolti}
\label{UC11.4}
\begin{figure}[h]
	\centering
	\includegraphics[scale=0.5]{UML/UC11_4.png}
	\caption{UC11.4: Visualizzazione cronologia questionari svolti}
\end{figure}
\begin{itemize}
\item\textbf{Attori}: utente autenticato, utente autenticato pro;
\item\textbf{Descrizione}: l'attore può visualizzare la cronologia dei questionari svolti;
\item\textbf{Precondizione}: l'attore ha compilato almeno un questionario;
\item\textbf{Postcondizione}: il sistema ha mostrato all'attore la cronologia dei questionari;
\item\textbf{Scenario principale}: 
\begin{itemize}
\item L'attore ha scelto di visualizzare le statistiche di un questionario (UC11.4.1).
\end{itemize}
\end{itemize}

\subsubsection{Caso d'uso UC11.4.1: Selezione e visualizzazione delle statistiche di un questionario scelto}
\begin{itemize}
\item\textbf{Attori}: utente autenticato, utente autenticato pro;
\item\textbf{Descrizione}: l'attore può selezionare e visualizzare le statistiche di un questionario da lui scelto fra quelli presenti nella sua cronologia;
\item\textbf{Precondizione}: il sistema mostra all'utente tutti i questionari da lui svolti;
\item\textbf{Postcondizione}: il sistema ha mostrato all'attore le statistiche del questionario selezionato;
\item\textbf{Scenario principale}: l'attore si trova nella pagina dedicata alla visualizzazione delle statistiche di un particolare questionario da lui scelto.
\end{itemize}

\subsubsection{Caso d'uso UC11.5: Visualizzazione questionari abilitati}
\label{UC11.5}
\begin{figure}[h]
	\centering
	\includegraphics[scale=0.5]{UML/UC11_5.png}
	\caption{UC11.5: Visualizza questionari abilitati}
\end{figure}
\begin{itemize}
\item\textbf{Attori}: utente autenticato, utente autenticato pro;
\item\textbf{Descrizione}: l'attore può visualizzare la lista dei questionari a cui è stato abilitato da parte dell'utente autenticato pro proprietario del questionario.
\item\textbf{Precondizione}: l'attore deve aver richiesto l'abilitazione ad un questionario e deve essere stato accettato dal proprietario;
\item\textbf{Postcondizione}: il sistema ha mostrato all'attore tutti i questionari a cui è stato abilitato;
\item\textbf{Scenario principale}: 
\begin{itemize}
\item L'attore ha selezionato un questionario abilitato (UC11.5.1).
\end{itemize} 
\end{itemize}

\subsubsection{Caso d'uso UC11.5.1: Seleziona questionario abilitato}
\begin{itemize}
\item\textbf{Attori}: utente autenticato, utente autenticato pro;
\item\textbf{Descrizione}: l'attore può selezionare un questionario a cui è stato abilitato ed eseguirlo;
\item\textbf{Precondizione}: l'attore deve essere stato abilitato ad almeno un questionario;
\item\textbf{Postcondizione}: l'attore ha selezionato un questionario a cui è stato abilitato e il sistema è pronto per farlo eseguire;
\item\textbf{Scenario principale}: l'attore richiede di eseguire un questionario a cui è stato abilitato.
\end{itemize}

\newpage
\subsection{Caso d'uso UC12: Risposta ad una domanda}
\label{UC12}
\begin{figure}[h]
	\centering
	\includegraphics[scale=0.5]{UML/UC12.png}
	\caption{UC12: Risposta ad una domanda}
\end{figure}

\begin{itemize}
\item \textbf{Attori}: utente non autenticato, utente autenticato, utente autenticato pro;
\item \textbf{Descrizione}: l'attore può rispondere alla domanda che viene visualizzata in modo diverso in base alla tipologia corrispondente;
\item \textbf{Precondizione}: l'attore visualizza la domanda e le possibili risposte;
\item \textbf{Postcondizione}: l'attore ha selezionato una o più risposte;
\item \textbf{Scenario principale}: si verifica uno dei seguenti scenari:
\begin{enumerate}
	\item L'attore può rispondere ad una domanda vero/falso (UC12.1);
	\item L'attore può rispondere ad una domanda con risposte multiple (UC12.2);
	\item L'attore può compilare un esercizio riempiendo gli spazi vuoti (UC12.3);
	\item L'attore associa le voci della colonna di sinistra con quelle della colonna di destra (UC12.4);
	\item L'attore può ordinare le immagini secondo il criterio richiesto dalla domanda (UC12.5);
	\item L'attore può ordinare le stringhe secondo il criterio richiesto dalla domanda (UC12.6);
	\item L'attore può rispondere alla domanda selezionando un'area cliccabile (UC12.7). 
\end{enumerate}
\end{itemize}

\subsubsection{Caso d'uso UC12.1: Risposta ad una domanda vero/falso}
\begin{itemize}
\item \textbf{Attori}: utente non autenticato, utente autenticato, utente autenticato pro;
\item \textbf{Descrizione}: l'attore può selezionare una risposta alla domanda vero/falso;
\item \textbf{Precondizione}: l'attore visualizza la domanda vero/falso;
\item \textbf{Postcondizione}: l'attore ha risposto alla domanda vero/falso;
\item \textbf{Scenario principale}: l'attore seleziona una risposta alla domanda vero/falso.
\end{itemize}

\subsubsection{Caso d'uso UC12.2: Risposta ad una domanda a risposta multipla}
\begin{itemize}
\item \textbf{Attori}: utente non autenticato, utente autenticato, utente autenticato pro;
\item \textbf{Descrizione}: l'attore può selezionare una o più risposte alla domanda a risposta multipla;
\item \textbf{Precondizione}: l'attore visualizza la domanda a risposta multipla;
\item \textbf{Postcondizione}: l'attore ha risposto alla domanda a risposta multipla;
\item \textbf{Scenario principale}: l'attore seleziona una o più risposte alla domanda a risposta multipla.
\end{itemize}

\subsubsection{Caso d'uso UC12.3: Compilazione esercizio a riempimento di spazi vuoti}
\label{UC12.3}
\begin{figure}[h]
	\centering
	\includegraphics[scale=0.5]{UML/UC12_3.png}
	\caption{UC12.3: Compilazione esercizio a riempimento di spazi vuoti}
\end{figure}
\begin{itemize}
\item \textbf{Attori}: utente non autenticato, utente autenticato, utente autenticato pro;
\item \textbf{Descrizione}: l'attore può compilare l'esercizio riempiendone gli spazi vuoti;
\item \textbf{Precondizione}: l'attore visualizza l'esercizio di riempimento degli spazi vuoti;
\item \textbf{Postcondizione}: l'attore ha compilato l'esercizio riempiendone gli spazi vuoti;
\item \textbf{Scenario principale}: l'attore può riempire uno spazio vuoto dopo averlo selezionato (UC12.3.1).
\end{itemize}

\subsubsection{Caso d'uso UC12.3.1: Riempimento di uno spazio vuoto selezionato}
\begin{itemize}
\item \textbf{Attori}: utente non autenticato, utente autenticato, utente autenticato pro;
\item \textbf{Descrizione}: l'attore può riempire o modificare lo spazio selezionato;
\item \textbf{Precondizione}: l'attore ha selezionato uno spazio;
\item \textbf{Postcondizione}: l'attore ha riempito lo spazio selezionato;
\item \textbf{Scenario principale}: l'attore riempie o modifica lo spazio selezionato. 
\end{itemize}

\subsubsection{Caso d'uso UC12.4: Risposta ad una domanda di collegamento}
\begin{figure}[h]
	\centering
	\includegraphics[scale=0.5]{UML/UC12_4.png}
	\caption{UC12.4: Risposta ad una domanda di collegamento}
\end{figure}
\begin{itemize}
\item \textbf{Attori}: utente non autenticato, utente autenticato, utente autenticato pro;
\item \textbf{Descrizione}: l'attore può collegare le voci della colonna di sinistra con le voci della colonna di destra;
\item \textbf{Precondizione}: l'attore visualizza la domanda di collegamento;
\item \textbf{Postcondizione}: l'attore ha collegato le voci della colonna di sinistra con le voci della colonna di destra;
\item \textbf{Scenario principale}: l'attore può collegare le voci della colonna di sinistra con le voci della colonna di destra (UC12.4.1).
\end{itemize}

\newpage
\subsubsection{Caso d'uso UC12.4.1: Collegamento delle voci}
\label{UC12.4.1}
\begin{figure}[h]
	\centering
	\includegraphics[scale=0.5]{UML/UC12_4_1.png}
	\caption{UC12.4.1: Collegamento delle voci}
\end{figure}
\begin{itemize}
\item \textbf{Attori}: utente non autenticato, utente autenticato, utente autenticato pro;
\item \textbf{Descrizione}: l'attore può selezionare una voce della colonna di sinistra/destra e successivamente selezionare una voce della colonna di destra/sinistra per associarle;
\item \textbf{Precondizione}: l'attore visualizza la domanda di collegamento;
\item \textbf{Postcondizione}: l'attore ha associato due voci;
\item \textbf{Scenario principale}:
\begin{enumerate}
\item L'attore può selezionare una voce della colonna di sinistra (UC12.4.1.1);
\item L'attore può selezionare una voce della colonna di destra (UC12.4.1.2).
\end{enumerate}
\end{itemize}

\subsubsection{Caso d'uso UC12.4.1.1: Selezione di una voce della colonna di sinistra}
\begin{itemize}
\item \textbf{Attori}: utente non autenticato, utente autenticato, utente autenticato pro;
\item \textbf{Descrizione}: l'attore può selezionare una voce della colonna di sinistra;
\item \textbf{Precondizione}: l'attore visualizza la domanda di collegamento;
\item \textbf{Postcondizione}: l'attore ha selezionato una voce della colonna di sinistra;
\item \textbf{Scenario principale}: l'attore seleziona una voce della colonna di sinistra. 
\end{itemize}

\subsubsection{Caso d'uso UC12.4.1.2: Selezione di una voce della colonna di destra}
\begin{itemize}
\item \textbf{Attori}: utente non autenticato, utente autenticato, utente autenticato pro;
\item \textbf{Descrizione}: l'attore può selezionare una voce della colonna di destra;
\item \textbf{Precondizione}: l'attore visualizza la domanda di collegamento;
\item \textbf{Postcondizione}: l'attore ha selezionato una voce della colonna di destra;
\item \textbf{Scenario principale}: l'attore seleziona una voce della colonna di destra. 
\end{itemize}

\subsubsection{Caso d'uso UC12.5: Ordinamento di immagini}
\label{UC12.5}
\begin{figure}[h]
	\centering
	\includegraphics[scale=0.5]{UML/UC12_5.png}
	\caption{UC12.5: Ordinamento di immagini}
\end{figure}
\begin{itemize}
\item \textbf{Attori}: utente non autenticato, utente autenticato, utente autenticato pro;
\item \textbf{Descrizione}: l'attore può ordinare le immagini secondo il criterio richiesto nella domanda inserendole negli appositi spazi;
\item \textbf{Precondizione}: l'attore visualizza la domanda ad ordinamento di immagini;
\item \textbf{Postcondizione}: l'attore ha inserito le immagini nella lista ordinata;
\item \textbf{Scenario principale}: l'attore può ordinare le immagini secondo il criterio richiesto nella domanda inserendole negli appositi spazi (UC12.5.1).
\end{itemize}

\newpage
\subsubsection{Caso d'uso UC12.5.1: Inserimento di un'immagine in uno spazio}
\label{UC12.5.1}
\begin{figure}[h]
	\centering
	\includegraphics[scale=0.5]{UML/UC12_5_1.png}
	\caption{UC12.5.1: Inserimento di un'immagine in uno spazio}
\end{figure}
\begin{itemize}
\item \textbf{Attori}: utente non autenticato, utente autenticato, utente autenticato pro;
\item \textbf{Descrizione}: l'attore può selezionare un'immagine per inserirla in uno spazio vuoto;
\item \textbf{Precondizione}: l'attore visualizza la domanda ad ordinamento di immagini;
\item \textbf{Postcondizione}: l'attore ha inserito l'immagine nello spazio selezionato;
\item \textbf{Scenario principale}: 
\begin{enumerate}
\item L'attore può selezionare un'immagine (UC12.5.1.1);
\item L'attore può selezionare uno spazio vuoto per inserire l'immagine selezionata (UC12.5.1.2).
\end{enumerate}
\item \textbf{Scenari alternativi}: se l'attore, dopo aver selezionato un'immagine, seleziona uno spazio che è già occupato da un'altra immagine, queste due si scambieranno di posto.
\end{itemize}

\subsubsection{Caso d'uso UC12.5.1.1: Selezione di un'immagine}
\begin{itemize}
\item \textbf{Attori}: utente non autenticato, utente autenticato, utente autenticato pro;
\item \textbf{Descrizione}: l'attore può selezionare un'immagine;
\item \textbf{Precondizione}: l'attore visualizza la domanda ad ordinamento di immagini;
\item \textbf{Postcondizione}: l'attore ha selezionato un'immagine;
\item \textbf{Scenario principale}: l'attore seleziona un'immagine.
\end{itemize}

\subsubsection{Caso d'uso UC12.5.1.2: Selezione di uno spazio vuoto}
\begin{itemize}
\item \textbf{Attori}: utente non autenticato, utente autenticato, utente autenticato pro;
\item \textbf{Descrizione}: l'attore può selezionare uno spazio vuoto;
\item \textbf{Precondizione}: l'attore visualizza la domanda ad ordinamento di immagini;
\item \textbf{Postcondizione}: l'attore ha selezionato uno spazio vuoto;
\item \textbf{Scenario principale}: l'attore seleziona uno spazio vuoto.
\end{itemize}

\subsubsection{Caso d'uso UC12.6: Ordinamento di stringhe}
\label{UC12.6}
\begin{figure}[h]
	\centering
	\includegraphics[scale=0.5]{UML/UC12_6.png}
	\caption{UC12.6: Ordinamento di stringhe}
\end{figure}
\begin{itemize}
\item \textbf{Attori}: utente non autenticato, utente autenticato, utente autenticato pro;
\item \textbf{Descrizione}: l'attore può ordinare le stringhe secondo il criterio richiesto nella domanda inserendole negli appositi spazi;
\item \textbf{Precondizione}: l'attore visualizza la domanda ad ordinamento di stringhe;
\item \textbf{Postcondizione}: l'attore ha inserito le stringhe nella lista ordinata;
\item \textbf{Scenario principale}: l'attore può ordinare le stringhe secondo il criterio richiesto nella domanda inserendole negli appositi spazi (UC12.6.1).
\end{itemize}

\newpage
\subsubsection{Caso d'uso UC12.6.1: Inserimento di una stringa in uno spazio}
\label{UC12.6.1}
\begin{figure}[h]
	\centering
	\includegraphics[scale=0.5]{UML/UC12_6_1.png}
	\caption{UC12.6.1: Inserimento di una stringa in uno spazio}
\end{figure}
\begin{itemize}
\item \textbf{Attori}: utente non autenticato, utente autenticato, utente autenticato pro;
\item \textbf{Descrizione}: l'attore può selezionare una stringa per inserirla in uno spazio vuoto o scambiarla con un'altra se lo spazio è già occupato;
\item \textbf{Precondizione}: l'attore visualizza la domanda ad ordinamento di stringhe;
\item \textbf{Postcondizione}: l'attore ha inserito la stringa nello spazio selezionato;
\item \textbf{Scenario principale}: 
\begin{enumerate}
\item L'attore può selezionare una stringa (UC12.6.1.1);
\item L'attore può selezionare uno spazio vuoto per inserire la stringa selezionata (UC12.6.1.2).
\end{enumerate}
\item \textbf{Scenari alternativi}: se l'attore, dopo aver selezionato una stringa, seleziona uno spazio che è già occupato da un'altra stringa, queste due si scambieranno di posto.
\end{itemize}

\subsubsection{Caso d'uso UC12.6.1.1: Selezione di una stringa}
\begin{itemize}
\item \textbf{Attori}: utente non autenticato, utente autenticato, utente autenticato pro;
\item \textbf{Descrizione}: l'attore può selezionare una stringa;
\item \textbf{Precondizione}: l'attore visualizza la domanda ad ordinamento di stringhe;
\item \textbf{Postcondizione}: l'attore ha selezionato una stringa;
\item \textbf{Scenario principale}: l'attore seleziona una stringa.
\end{itemize}

\subsubsection{Caso d'uso UC12.6.1.2: Selezione di uno spazio vuoto}
\begin{itemize}
\item \textbf{Attori}: utente non autenticato, utente autenticato, utente autenticato pro;
\item \textbf{Descrizione}: l'attore può selezionare uno spazio vuoto;
\item \textbf{Precondizione}: l'attore visualizza la domanda ad ordinamento di stringhe;
\item \textbf{Postcondizione}: l'attore ha selezionato uno spazio vuoto;
\item \textbf{Scenario principale}: l'attore seleziona uno spazio vuoto.
\end{itemize}

\subsubsection{Caso d'uso UC12.7: Risposta ad una domanda con area cliccabile nell'immagine}
\label{UC12.7}
\begin{figure}[h]
	\centering
	\includegraphics[scale=0.5]{UML/UC12_7.png}
	\caption{UC12.7: Risposta ad una domanda con area cliccabile nell'immagine}
\end{figure}
\begin{itemize}
\item \textbf{Attori}: utente non autenticato, utente autenticato, utente autenticato pro;
\item \textbf{Descrizione}: l'attore può selezionare una o più aree cliccabili;
\item \textbf{Precondizione}: l'attore visualizza la domanda con area cliccabile nell'immagine;
\item \textbf{Postcondizione}: l'attore ha risposto alla domanda con area cliccabile;
\item \textbf{Scenario principale}: l'attore può selezionare un area cliccabile (UC12.7.1).
\end{itemize}

\subsubsection{Caso d'uso UC12.7.1: Selezione di un'area cliccabile}
\begin{itemize}
\item \textbf{Attori}: utente non autenticato, utente autenticato, utente autenticato pro;
\item \textbf{Descrizione}: l'attore può selezionare un'area cliccabile;
\item \textbf{Precondizione}: l'attore visualizza la domanda con area cliccabile nell'immagine;
\item \textbf{Postcondizione}: l'attore ha selezionato un'area cliccabile;
\item \textbf{Scenario principale}: l'attore seleziona un'area cliccabile.
\end{itemize}

\newpage
\subsection{Caso d'uso UC13: Ricerca utente}
\label{UC11}
\begin{figure}[h]
	\centering
	\includegraphics[scale=0.5]{UML/UC13.png}
	\caption{UC12: Ricerca utente}
\end{figure}

\newpage
\subsection{Caso d'uso UC14: Login con Facebook}
\label{UC14}
\begin{figure}[ht]
	\centering
	\includegraphics[scale=0.48]{UML/UC14.png}
	\caption{UC14: Login da Facebook}
\end{figure}
\FloatBarrier
\begin{itemize}
	\item \textbf{Attori}: utente non autenticato, Facebook;
	\item \textbf{Descrizione}: l'attore può autenticarsi utilizzando Facebook;
	\item \textbf{Precondizione}: l'attore visualizza la pagina di login e sceglie il login con Facebook;
	\item \textbf{Postcondizione}: l'attore è autenticato;
	\item \textbf{Scenario principale}: l'attore effettua il login tramite Facebook.
\end{itemize}


\newpage
\subsection{Caso d'uso UC15: Login con Twitter}
\label{UC15}
\begin{figure}
	\centering
	\includegraphics[scale=0.48]{UML/UC15.png}
	\caption{UC15: Login da Twitter}
\end{figure}
\FloatBarrier
\begin{itemize}
	\item \textbf{Attori}: utente non autenticato, Twitter;
	\item \textbf{Scopo e descrizione}: l'attore può autenticarsi utilizzando Twitter;
	\item \textbf{Precondizione}: l'attore visualizza la pagina di login e sceglie il login con Twitter;
	\item \textbf{Postcondizione}: l'attore è autenticato;
	\item \textbf{Scenario principale}: l'attore effettua il login tramite Twitter.
\end{itemize}


\newpage
\subsection{Caso d'uso UC16: Login con Google+}
\label{UC16}
\begin{figure}
	\centering
	\includegraphics[scale=0.48]{UML/UC16.png}
	\caption{UC16: Login da Google+}
\end{figure}
\FloatBarrier
\begin{itemize}
	\item \textbf{Attori}: utente non autenticato, Google+;
	\item \textbf{Descrizione}: l'attore può autenticarsi utilizzando Google+;
	\item \textbf{Precondizione}: l'attore visualizza la pagina di login e sceglie il login con Google+;
	\item \textbf{Postcondizione}: l'attore è autenticato;
	\item \textbf{Scenario principale}: l'attore effettua il login tramite Google+.
\end{itemize}


\newpage
\subsection{Caso d'uso UC17: Login con LinkedIn}
\label{UC17}
\begin{figure}
	\centering
	\includegraphics[scale=0.48]{UML/UC17.png}
	\caption{UC17: Login da LinkedIn}
\end{figure}
\FloatBarrier
\begin{itemize}
	\item \textbf{Attori}: utente non autenticato, LinkedIn;
	\item \textbf{Descrizione}: l'attore può autenticarsi utilizzando LinkedIn;
	\item \textbf{Precondizione}: l'attore visualizza la pagina di login e sceglie il login con LinkedIn;
	\item \textbf{Postcondizione}: l'attore è autenticato;
	\item \textbf{Scenario principale}: l'attore effettua il login tramite LinkedIn.
\end{itemize}