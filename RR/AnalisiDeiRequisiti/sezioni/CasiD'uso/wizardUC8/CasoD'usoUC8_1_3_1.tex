\subsubsection{Caso d'uso UC8.1.3.1: Creazione domanda vero/falso}
	\label{UC8.1.3.1}
	\begin{figure}[h]
		\centering
			\includegraphics[scale=0.45,keepaspectratio]{UML/UC8_1_3_1.png}
		\caption{UC8.1.3.1: Creazione domanda vero/falso}
	\end{figure}
	\FloatBarrier
	\begin{itemize}
		\item
			\textbf{Attori}: utente autenticato, utente autenticato pro;
		\item		
			\textbf{Descrizione}: l'attore può utilizzare la procedura guidata per la creazione di una domanda vero/falso;
		\item
			\textbf{Precondizione}: l'attore ha selezionato la funzionalità di creazione di una domanda vero/falso; 
		\item
			\textbf{Postcondizione}: l'attore ha creato una domanda vero/falso;
		\item
			\textbf{Scenario principale}:
	       		\begin{enumerate}
	       			\item
	       			L'attore può inserire il testo della domanda (UC8.1.3.1.1);
	       			\item
	       			L'attore può inserire un'immagine relativa al testo della domanda (UC8.1.3.1.2);
					\item
					L'attore può indicare la risposta corretta tramite uno strumento di selezione (UC8.1.3.1.3).
	 			\end{enumerate}
	\end{itemize}

\subsubsection{Caso d'uso UC8.1.3.1.1: Inserimento testo della domanda}
	\begin{itemize}
		\item
			\textbf{Attori}: utente autenticato, utente autenticato pro;
		\item		
			\textbf{Descrizione}: l'attore può inserire il testo della domanda;
		\item
			\textbf{Precondizione}: l'attore ha selezionato la funzionalità di creazione di una domanda vero/falso; 
		\item
			\textbf{Postcondizione}: l'attore ha inserito il testo della domanda;
		\item
			\textbf{Scenario principale}: l'attore inserisce il testo della domanda. 
	 			
	\end{itemize}
	
\subsubsection{Caso d'uso UC8.1.3.1.2: Inserimento immagine}
	\label{UC8.1.3.1.2}
	\begin{figure}[h]
		\centering
			\includegraphics[scale=0.45,keepaspectratio]{UML/UC8_1_3_1_2.png}
		\caption{UC8.1.3.1.2: Inserimento immagine}
	\end{figure}
	\FloatBarrier
	\begin{itemize}
		\item
			\textbf{Attori}: utente autenticato, utente autenticato pro;
		\item		
			\textbf{Descrizione}: l'attore può inserire un'immagine relativa al testo della domanda;
		\item
			\textbf{Precondizione}: l'attore ha selezionato la funzionalità di creazione di una domanda vero/falso; 
		\item
			\textbf{Postcondizione}: l'attore ha inserito un'immagine relativa al testo della domanda;
		\item
			\textbf{Scenario principale}: l'attore può eliminare l'immagine inserita (UC8.1.3.1.2.1).						
	\end{itemize}
	
	\subsubsection{Caso d'uso UC8.1.3.1.2.1: Eliminazione immagine inserita}
		\begin{itemize}
		\item
			\textbf{Attori}: utente autenticato, utente autenticato pro;
		\item		
			\textbf{Descrizione}: l'attore può rimuovere l'immagine relativa al testo della domanda che era stata inserita precedentemente;
		\item
			\textbf{Precondizione}: l'attore ha inserito un'immagine relativa al testo della domanda;
		\item
			\textbf{Postcondizione}: l'attore ha eliminato l'immagine relativa alla domanda;
		\item
			\textbf{Scenario principale}: l'attore rimuove l'immagine relativa alla domanda. 
		\end{itemize}

\subsubsection{Caso d'uso UC8.1.3.1.3: Selezione risposta corretta}
	\begin{itemize}
		\item
			\textbf{Attori}: utente autenticato, utente autenticato pro;
		\item		
			\textbf{Descrizione}: l'attore può indicare la risposta corretta;
		\item
			\textbf{Precondizione}: l'attore ha selezionato la funzionalità di creazione di una domanda vero/falso; 
		\item
			\textbf{Postcondizione}: l'attore ha selezionato la risposta corretta;
		\item
			\textbf{Scenario principale}: l'attore indica la risposta corretta.  			
	\end{itemize}