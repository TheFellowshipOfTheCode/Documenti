\subsubsection{Caso d'uso UC8.1.3.1: Creazione domanda vero/falso}
	\begin{itemize}
		\item
			\textbf{Attori}: utente autenticato, utente autenticato pro;
		\item		
			\textbf{Descrizione}: lo scopo di questa funzionalità è offrire agli attori la possibilità di creare domande vero/falso;
		\item
			\textbf{Precondizione}: gli attori hanno selezionato la seguente funzionalita'; 
		\item
			\textbf{Postcondizione}: gli attori hanno creato una domanda vero/falso;
		\item
			\textbf{Scenario principale}:
	       		\begin{enumerate}
	       			\item
	       			Gli attori devono compilare il campo dati destinato alla scrittura del testo della domanda [UC8.1.3.1.1]
	       			\item
	       			Gli attori possono inserire una immagine relativa al testo della domanda [UC8.1.3.1.2]
					\item
					Gli attori devono indicare la risposta corretta tramite uno strumento di selezione [UC8.1.3.1.3].
	 			\end{enumerate}
	\end{itemize}

\subsubsection{Caso d'uso UC8.1.3.1.1: Inserimento testo della domanda}
	\begin{itemize}
		\item
			\textbf{Attori}: utente autenticato, utente autenticato pro;
		\item		
			\textbf{Descrizione}: lo scopo di questa funzionalità è offrire agli attori la possibilità di inserire il testo della domanda;
		\item
			\textbf{Precondizione}: gli attori hanno selezionato la modalità di creazione di una domanda vero/falso; 
		\item
			\textbf{Postcondizione}: gli attori hanno inserito il testo della domanda;
		\item
			\textbf{Scenario principale}: gli attori inseriscono il testo della domanda. 
	 			
	\end{itemize}
	
\subsubsection{Caso d'uso UC8.1.3.1.2: Inserimento immagine}
	\begin{itemize}
		\item
			\textbf{Attori}: utente autenticato, utente autenticato pro;
		\item		
			\textbf{Descrizione}: lo scopo di questa funzionalità è offrire agli attori la possibilità di inserire un'immagine relativa al testo della domanda;
		\item
			\textbf{Precondizione}: gli attori hanno selezionato la modalità di creazione di una domanda vero/falso; 
		\item
			\textbf{Postcondizione}: gli attori hanno inserito l'immagine;
		\item
			\textbf{Scenario principale}: gli attori inseriscono l'immagine. 	
	\end{itemize}
	

\subsubsection{Caso d'uso UC8.1.3.1.3: Selezione risposta corretta}
	\begin{itemize}
		\item
			\textbf{Attori}: utente autenticato, utente autenticato pro;
		\item		
			\textbf{Descrizione}: lo scopo di questa funzionalità è offrire agli attori la possibilità, tramite uno strumento di selezione, di indicare la risposta corretta;
		\item
			\textbf{Precondizione}: gli attori hanno selezionato la modalità di creazione di una domanda vero/falso; 
		\item
			\textbf{Postcondizione}: gli attori hanno selezionato la risposta corretta;
		\item
			\textbf{Scenario principale}: gli attori indicano la risposta corretta tramite uno strumento di selezione. 
	 			
	\end{itemize}
	

	
