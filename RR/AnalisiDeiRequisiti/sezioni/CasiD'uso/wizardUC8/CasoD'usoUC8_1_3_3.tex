\subsubsection{Caso d'uso UC8.1.3.3: Creazione di esercizi di riempimento degli spazi vuoti}
	\begin{itemize}
		\item
			\textbf{Attori}: utente autenticato, utente autenticato pro;
		\item		
			\textbf{Descrizione}: lo scopo di questa funzionalità è offrire agli attori la possibilità di creare esercizi di riempimento degli spazi vuoti;
		\item
			\textbf{Precondizione}: gli attori hanno selezionato la seguente funzionalità; 
		\item
			\textbf{Postcondizione}: gli attori hanno creato un esercizio di riempimento degli spazi vuoti;
		\item
			\textbf{Scenario principale}:
	       		\begin{enumerate}
	       			\item
	       			Gli attori devono compilare il campo dati destinato alla scrittura del testo dell'esercizio (UC8.1.3.3.1);
	       			\item
	       			Gli attori devono indicare le parole che saranno sostituite con degli spazi vuoti dal sistema (UC8.1.3.3.2).
	 			\end{enumerate}
	\end{itemize}
	
\subsubsection{Caso d'uso UC8.1.3.3.1: Scrittura testo dell'esercizio}
	\begin{itemize}
		\item
			\textbf{Attori}: utente autenticato, utente autenticato pro;
		\item		
			\textbf{Descrizione}: lo scopo di questa funzionalità è offrire agli attori la possibilità di scrivere il testo dell'esercizio di riempimento;
		\item
			\textbf{Precondizione}: gli attori hanno selezionato la seguente funzionalità; 
		\item
			\textbf{Postcondizione}: gli attori hanno compilato il campo dati dedicato alla scrittura dell'esercizio di riempimento;
		\item
			\textbf{Scenario principale}: gli attori compilano il campo dati dedicato alla scrittura dell'esercizio di riempimento.
	\end{itemize}


\subsubsection{Caso d'uso UC8.1.3.3.2: Indicazione parole da oscurare}
	\begin{itemize}
		\item
			\textbf{Attori}: utente autenticato, utente autenticato pro;
		\item		
			\textbf{Descrizione}: lo scopo di questa funzionalità è offrire agli attori la possibilità di indicare le parole che saranno sostituite con degli spazi vuoti;
		\item
			\textbf{Precondizione}: gli attori hanno inserito la scrittura del testo dell'esercizio; 
		\item
			\textbf{Postcondizione}: gli attori hanno indicato le parole che saranno sostituite con degli spazi vuoti;
		\item
			\textbf{Scenario principale}: gli attori indicano le parole che verranno oscurate dal sistema.
	\end{itemize}