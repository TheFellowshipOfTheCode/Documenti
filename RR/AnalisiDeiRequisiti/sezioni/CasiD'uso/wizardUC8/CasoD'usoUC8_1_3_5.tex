\subsubsection{Caso d’uso UC8.1.3.5: Wizard creazione ordinamento di immagini}
\begin{itemize}
	\item\textbf{Attori}: utente autenticato, utente autenticato pro;
	\item\textbf{Scopo e descrizione}: gli attori possono inserire una domanda del tipo ordinamento di immagini;
	\item\textbf{Precondizione}: il sistema mostra agli attori il form di inserimento dei campi dati per la tipologia di domanda scelta; 
	\item \textbf{Postcondizione}: gli attori hanno inserito tutti i campi dati obbligatori;
	\item\textbf{Scenario principale}: gli attori devono inserire i seguenti dati:
	\begin{itemize}
		\item Il testo della domanda che si vuole sottoporre;
		\item Tutte le immagini che devono essere ordinate;
		\item Il giusto ordinamento delle immagini.
	\end{itemize}
\end{itemize}