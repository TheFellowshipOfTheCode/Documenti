\subsubsection{Caso d'uso UC8.2.1.3: Modifica esercizi di riempimento degli spazi vuoti}
	\label{UC8.2.1.3}
	\begin{figure}[h]
		\centering
			\includegraphics[scale=0.45,keepaspectratio]{UML/UC8_2_1_3.png}
		\caption{UC8.2.1.3: Modifica esercizi di riempimento degli spazi vuoti}
	\end{figure}
	\FloatBarrier
	\begin{itemize}
		\item
			\textbf{Attori}: utente autenticato, utente autenticato pro;
		\item		
			\textbf{Descrizione}: lo scopo di questa funzionalità è offrire agli attori la possibilità di modificare esercizi di riempimento degli spazi vuoti;
		\item
			\textbf{Precondizione}: gli attori hanno selezionato la seguente funzionalità; 
		\item
			\textbf{Postcondizione}: gli attori hanno modificato un esercizio di riempimento degli spazi vuoti;
		\item
			\textbf{Scenario principale}:
	       		\begin{enumerate}
	       			\item
	       			Gli attori possono modificare il testo dell'esercizio (UC8.2.1.3.1);
	       			\item
	       			Gli attori possono modificare le parole che saranno sostituite con degli spazi vuoti dal sistema (UC8.2.1.3.2).
	 			\end{enumerate}
	\end{itemize}
	
\subsubsection{Caso d'uso UC8.2.1.3.1: Modifica testo dell'esercizio}
	\begin{itemize}
		\item
			\textbf{Attori}: utente autenticato, utente autenticato pro;
		\item		
			\textbf{Descrizione}: lo scopo di questa funzionalità è offrire agli attori la possibilità di modificare il testo dell'esercizio di riempimento;
		\item
			\textbf{Precondizione}: gli attori hanno selezionato la funzionalità di modificare un esercizio di riempimento degli spazi vuoti; 
		\item
			\textbf{Postcondizione}: gli attori hanno modificato il testo dell'esercizio di riempimento;
		\item
			\textbf{Scenario principale}: gli attori compilano il testo dell'esercizio di riempimento.
	\end{itemize}


\subsubsection{Caso d'uso UC8.2.1.3.2: Modifica parole da oscurare}
	\begin{itemize}
		\item
			\textbf{Attori}: utente autenticato, utente autenticato pro;
		\item		
			\textbf{Descrizione}: lo scopo di questa funzionalità è offrire agli attori la possibilità di modificare le parole che saranno sostituite con degli spazi vuoti;
		\item
			\textbf{Precondizione}: gli attori hanno selezionato la funzionalità di modificare un esercizio di riempimento degli spazi vuoti; 
		\item
			\textbf{Postcondizione}: gli attori hanno modificato le parole che saranno sostituite con degli spazi vuoti;
		\item
			\textbf{Scenario principale}: gli attori modificano le parole che verranno oscurate dal sistema.
	\end{itemize}