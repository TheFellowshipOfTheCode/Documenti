\newpage
\subsection{Caso d'uso UC8: Gestione delle domande}
	\label{UC8}
	\begin{figure}[h]
		\centering
			\includegraphics[scale=0.45,keepaspectratio]{UML/UC8.png}
		\caption{UC8: Gestione delle domande}
	\end{figure}
	\FloatBarrier
	\begin{itemize}
		\item
			\textbf{Attori}: utente autenticato, utente autenticato pro;
		\item		
			\textbf{Descrizione}: l'attore può creare e modificare domande;
		\item
			\textbf{Precondizione}: il sistema ha riconosciuto l'autenticazione dell'attore; 
		\item
			\textbf{Postcondizione}: l'attore ha compiuto una delle operazioni appartenenti a questa funzionalità;
		\item
			\textbf{Scenario principale}:
	       		\begin{enumerate}
					\item
					L'attore crea una nuova domanda (UC8.1);
					\item
					L'attore modifica una domanda (UC8.2).
	 			\end{enumerate}
	\end{itemize}
\subsubsection{Caso d'uso UC8.1: Creazione nuova domanda}
	\label{UC8.1}
	\begin{figure}[h]
		\centering
			\includegraphics[scale=0.45,keepaspectratio]{UML/UC8_1.png}
		\caption{UC8.1: Creazione nuova domanda}
	\end{figure}
	\FloatBarrier
	\begin{itemize}
		\item
			\textbf{Attori}: utente autenticato, utente autenticato pro;
		\item		
			\textbf{Descrizione}: l'attore può creare una nuova domanda;
		\item
			\textbf{Precondizione}: il sistema presenta la funzionalità di creazione della domanda;
		\item
			\textbf{Postcondizione}: l'attore crea una domanda;		
		\item
			\textbf{Scenario principale}:
	       		\begin{enumerate}
					\item
					L'attore può inserire l'argomento relativo alla nuova domanda (UC8.1.1);
					\item
					L'attore può inserire le parole chiave relative alla nuova domanda (UC8.1.2);
					\item
					L'attore può selezionare la tipologia di domanda (UC8.1.3);
					
					\item
					L'attore può richiamare il \textit{wizard\ped{G}} per creare una domanda vero/falso (UC8.1.4);
					\item
					L'attore può richiamare il \textit{wizard\ped{G}} per creare una domanda a risposta multipla \\(UC8.1.5);
					\item
					L'attore può richiamare il \textit{wizard\ped{G}} per creare un esercizio a riempimento di spazi vuoti (UC8.1.6);
					\item
					L'attore può richiamare il \textit{wizard\ped{G}} per creare una domanda di collegamento (UC8.1.7);
					\item
					L'attore può richiamare il \textit{wizard\ped{G}} per creare una domanda a ordinamento di immagini (UC8.1.8);
					\item
					L'attore può richiamare il \textit{wizard\ped{G}} per creare una domanda a ordinamento di stringhe (UC8.1.9);
					\item
					L'attore può richiamare il \textit{wizard\ped{G}} per creare una domanda con area cliccabile nell'immagine (UC8.1.10);
					\item
					L'attore può confermare la creazione della domanda (UC8.1.11).
	 			\end{enumerate}
	 	\item
			\textbf{Estensioni}: l'attore visualizza un messaggio d'errore relativo alla creazione della domanda (UC8.1.12).
	\end{itemize}
	\subsubsection{Caso d'uso UC8.1.1: Selezione argomento}
	\begin{itemize}
		\item
			\textbf{Attori}: utente autenticato, utente autenticato pro;
		\item
			\textbf{Descrizione}: l'attore può indicare un argomento tra quelli presenti;
		\item		
			\textbf{Precondizione}: il sistema presenta la funzionalità di selezionare un argomento;
		\item
			\textbf{Postcondizione}: l'attore ha selezionato un argomento;
		\item
			\textbf{Scenario principale}: l'attore seleziona l'argomento da assegnare alla nuova domanda.		
	\end{itemize}
		
	\subsubsection{Caso d'uso UC8.1.2: Inserimento parole chiave}
	\begin{itemize}
		\item
			\textbf{Attori}: utente autenticato, utente autenticato pro;
		\item
			\textbf{Descrizione}: l'attore può inserire le parole chiave relative alla nuova domanda per specificare più dettagliatamente l'argomento della domanda;
		\item		
			\textbf{Precondizione}: il sistema presenta lo spazio destinato all'inserimento di una parola chiave;
		\item
			\textbf{Postcondizione}: l'attore ha selezionato delle parole chiave;
		\item
			\textbf{Scenario principale}: l'attore inserisce delle parole chiave relative alla nuova domanda.	
	\end{itemize}


	\subsubsection{Caso d'uso UC8.1.3: Selezione tipologia di domanda}
	\begin{itemize}
		\item
			\textbf{Attori}: utente autenticato, utente autenticato pro;
		\item
			\textbf{Descrizione}: l'attore può scegliere la tipologia di domanda che vuole inserire;
		\item		
			\textbf{Precondizione}: il sistema presenta la funzionalità di selezionare una tipologia di una domanda;
		\item
			\textbf{Postcondizione}: l'utente ha selezionato la tipologia di domanda da inserire;
		\item 
			\textbf{Scenario principale}: l'utente seleziona la tipologia di domanda.
	\end{itemize}

	%inclusione file latex wizard
\input{sezioni/CasiD'uso/wizardUC8/CasoD'usoUC8_1_4.tex}
\input{sezioni/CasiD'uso/wizardUC8/CasoD'usoUC8_1_5.tex}
\subsubsection{Caso d'uso UC8.1.6: Creazione esercizio a riempimento di spazi vuoti}
	\label{UC8.1.6}
	\begin{figure}[h]
		\centering
			\includegraphics[scale=0.45,keepaspectratio]{UML/UC8_1_6.png}
		\caption{UC8.1.6: Creazione esercizio a riempimento di spazi vuoti}
	\end{figure}
	\FloatBarrier
	\begin{itemize}
		\item
			\textbf{Attori}: utente autenticato, utente autenticato pro;
		\item		
			\textbf{Descrizione}: l'attore può utilizzare la procedura guidata per la creazione di un esercizio a riempimento di spazi vuoti;
		\item
			\textbf{Precondizione}: il sistema presenta all'attore la procedura guidata per la creazione di un esercizio a riempimento di spazi vuoti;
		\item
			\textbf{Postcondizione}: l'attore ha creato un esercizio a riempimento di spazi vuoti;
		\item
			\textbf{Scenario principale}:
	       		\begin{enumerate}
	       			\item
	       			L'attore può inserire il testo dell'esercizio (UC8.1.6.1);
	       			\item
	       			L'attore può indicare le parole che saranno sostituite con degli spazi vuoti dal sistema (UC8.1.6.2).
	 			\end{enumerate}
	\end{itemize}
	
\subsubsection{Caso d'uso UC8.1.6.1: Scrittura testo dell'esercizio}
	\begin{itemize}
		\item
			\textbf{Attori}: utente autenticato, utente autenticato pro;
		\item		
			\textbf{Descrizione}: l'attore può inserire il testo dell'esercizio;
		\item
			\textbf{Precondizione}: il sistema presenta all'attore lo spazio destinato all'inserimento del testo dell'esercizio; 
		\item
			\textbf{Postcondizione}: l'attore ha scritto il testo dell'esercizio;
		\item
			\textbf{Scenario principale}: l'attore scrive il testo dell'esercizio.
	\end{itemize}


\subsubsection{Caso d'uso UC8.1.6.2: Indicazione parole da oscurare}
	\begin{itemize}
		\item
			\textbf{Attori}: utente autenticato, utente autenticato pro;
		\item		
			\textbf{Descrizione}: l'attore può indicare le parole che saranno sostituite con degli spazi vuoti;
		\item
			\textbf{Precondizione}: il sistema presenta all'attore lo spazio destinato all'indicazione delle parole da oscurare;  
		\item
			\textbf{Postcondizione}: l'attore ha indicato le parole che saranno sostituite con degli spazi vuoti;
		\item
			\textbf{Scenario principale}: l'attore indica le parole che verranno oscurate dal sistema.
	\end{itemize}
\subsubsection{Caso d'uso UC8.1.7: Creazione domanda a ordinamento di stringhe}
	\label{UC8.1.7}
	\begin{figure}[ht]
		\centering
			\includegraphics[scale=0.45,keepaspectratio]{UML/UC8_1_7.png}
		\caption{UC8.1.7: Creazione domanda a ordinamento di stringhe}
	\end{figure}
	\FloatBarrier
\begin{itemize}
	\item\textbf{Attori}: utente autenticato, utente autenticato pro;
	\item\textbf{Descrizione}: l'attore può utilizzare la procedura guidata per la creazione di una domanda a ordinamento di stringhe;
	\item \textbf{Precondizione}: il sistema presenta all'attore la procedura guidata per la creazione di una domanda a ordinamento di stringhe; 
	\item\textbf{Postcondizione}: l'attore ha creato una domanda a ordinamento di stringhe;
	\item\textbf{Scenario principale}:
		\begin{enumerate}
			\item L'attore può inserire il testo della domanda (UC8.1.7.1);
			\item L'attore può inserire le stringhe che compongono la risposta (UC8.1.7.2);
			\item L'attore può indicare la soluzione della sequenza di stringhe (UC8.1.7.3).
		\end{enumerate}
\end{itemize}

\subsubsection{Caso d'uso UC8.1.7.1: Inserimento testo della domanda}
	\begin{itemize}
		\item \textbf{Attori}: utente autenticato, utente autenticato pro;
		\item \textbf{Descrizione}: l'attore può inserire il testo della domanda;
		\item\textbf{Precondizione}: il sistema presenta all'attore lo spazio destinato all'inserimento del testo della domanda;
		\item \textbf{Postcondizione}: l'attore ha inserito il testo della domanda;
		\item\textbf{Scenario principale}: l'attore inserisce il testo della domanda.
	\end{itemize}
	
\subsubsection{Caso d'uso UC8.1.7.2: Inserimento stringhe di composizione sequenza}
	\begin{itemize}
		\item \textbf{Attori}: utente autenticato, utente autenticato pro;
		\item \textbf{Descrizione}: l'attore può inserire le stringhe che costituiscono la risposta alla domanda;
		\item\textbf{Precondizione}: il sistema presenta all'attore la funzionalità di inserire le stringhe che costituiscono la risposta alla domanda;
		\item \textbf{Postcondizione}: l'attore ha inserito le stringhe che costituiscono la risposta alla domanda;
		\item\textbf{Scenario principale}: l'attore inserisce le stringhe che costituiscono la risposta alla domanda.
	\end{itemize}
	
\subsubsection{Caso d'uso UC8.1.7.3: Composizione soluzione della sequenza}
	\begin{itemize}
		\item \textbf{Attori}: utente autenticato, utente autenticato pro;
		\item \textbf{Descrizione}: l'attore può indicare la soluzione della domanda mettendo nell'ordine corretto la sequenza di stringhe;
		\item\textbf{Precondizione}: il sistema presenta all'attore la funzionalità di indicare la soluzione della sequenza;
		\item \textbf{Postcondizione}: l'attore ha indicato la soluzione della sequenza;
		\item\textbf{Scenario principale}: l'attore indica la soluzione della sequenza. 
	\end{itemize}

\subsubsection{Caso d'uso UC8.1.8: Creazione domanda con area cliccabile nell'immagine}
\label{UC8.1.8}
\begin{figure}[h]
	\centering
	\includegraphics[scale=0.5,keepaspectratio]{UML/UC8_1_8.png}
	\caption{UC8.1.8: Creazione domanda con area cliccabile nell'immagine}
\end{figure}
\FloatBarrier
\begin{itemize}
	\item \textbf{Attori}: utente autenticato, utente autenticato pro;
	\item \textbf{Descrizione}: l'attore ha selezionato la funzionalità di creazione di una domanda la cui risposta è selezionabile all'interno di aree cliccabili in un'immagine;
	\item \textbf{Precondizione}: il sistema presenta all'attore la procedura guidata per la creazione di una domanda con area cliccabile nell'immagine;
	\item \textbf{Postcondizione}: l'attore ha creato una domanda con area cliccabile nell'immagine;
	\item \textbf{Scenario principale}:
		\begin{enumerate}
	       	\item L'attore può inserire il testo della domanda (UC8.1.8.1);
	        \item L'attore può inserire un'immagine relativa al testo della domanda (UC8.1.8.2);
		\item L'attore può scegliere quante aree saranno selezionabili all'interno dell'immagine \\(UC8.1.8.3);
		\item L'attore può scegliere quali aree saranno selezionabili all'interno dell'immagine \\(UC8.1.8.4).
	 	\end{enumerate}
\end{itemize}

\subsubsection{Caso d'uso UC8.1.8.1: Inserimento testo della domanda}
\begin{itemize}
	\item \textbf{Attori}: utente autenticato, utente autenticato pro;
	\item \textbf{Descrizione}: l'attore può inserire il testo della domanda che vuole creare;
	\item \textbf{Precondizione}: il sistema presenta all'attore lo spazio destinato all'inserimento del testo della domanda;
	\item \textbf{Postcondizione}: l'attore ha inserito il testo della domanda;
	\item \textbf{Scenario principale}: l'attore inserisce il testo della domanda. 
\end{itemize}

\subsubsection{Caso d'uso UC8.1.8.2: Inserimento immagine}
\begin{itemize}
	\item \textbf{Attori}: utente autenticato, utente autenticato pro;
	\item \textbf{Descrizione}: l'attore può inserire un'immagine relativa al testo della domanda;
	\item \textbf{Precondizione}: il sistema presenta all'attore la funzionalità di inserire un'immagine;
	\item \textbf{Postcondizione}: l'attore ha inserito un'immagine relativa al testo della domanda;
	\item \textbf{Scenario principale}: l'attore può eliminare l'immagine inserita (UC8.1.3.7.2.1).			
	\end{itemize}

\subsubsection{Caso d'uso UC8.1.8.3: Scelta numero aree selezionabili}
\begin{itemize}
	\item \textbf{Attori}: utente autenticato, utente autenticato pro;
	\item \textbf{Descrizione}: l'attore può scegliere il numero di aree selezionabili all'interno dell'immagine;
	\item \textbf{Precondizione}: il sistema presenta all'attore la funzionalità di scegliere il numero di aree selezionabili all'interno dell'immagine; 	
	\item \textbf{Postcondizione}: l'attore ha scelto il numero di aree selezionabili all'interno dell'immagine;
	\item \textbf{Scenario principale}: l'attore sceglie il numero di aree selezionabili all'interno dell'immagine. 	
\end{itemize}

\subsubsection{Caso d'uso UC8.1.8.4: Scelta aree selezionabili}
\begin{itemize}
	\item \textbf{Attori}: utente autenticato, utente autenticato pro;
	\item \textbf{Descrizione}: l'attore ha la possibilità di scegliere dove inserire le aree selezionabili all'interno dell'immagine;
	\item \textbf{Precondizione}: il sistema presenta all'attore la funzionalità di scegliere dove inserire le aree selezionabili all'interno dell'immagine; 	
	\item \textbf{Postcondizione}: l'attore ha scelto dove inserire le aree selezionabili all'interno dell'immagine;
	\item \textbf{Scenario principale}: l'attore sceglie dove inserire le aree selezionabili all'interno dell'immagine. 	
\end{itemize}


\input{sezioni/CasiD'uso/wizardUC8/CasoD'usoUC8_1_9.tex}
\input{sezioni/CasiD'uso/wizardUC8/CasoD'usoUC8_1_10.tex}


	\subsubsection{Caso d'uso UC8.1.11: Conferma creazione}
	\begin{itemize}
		\item
			\textbf{Attori}: utente autenticato, utente autenticato pro;
		\item
			\textbf{Descrizione}: l'attore può confermare la creazione della domanda;
		\item		
			\textbf{Precondizione}: il sistema presenta la funzionalità di confermare la creazione di una domanda tramite un \textit{wizard\ped{G}};
		\item
			\textbf{Postcondizione}: l'attore ha creato una domanda;
		\item
			\textbf{Scenario principale}: l'attore conferma la creazione della domanda;		
		\item
	 		\textbf{Scenari alternativi}: l'attore annulla la creazione della domanda.
					
	\end{itemize}	
	
	\subsubsection{Caso d'uso UC8.1.12: Visualizzazione errore creazione domanda}
	\begin{itemize}
		\item
			\textbf{Attori}: utente autenticato, utente autenticato pro;
		\item
			\textbf{Descrizione}: l'attore può visualizzare un messaggio d'errore nel caso si fossero verificati uno o più scenari alternativi durante la creazione della domanda;
		\item		
			\textbf{Precondizione}: il sistema ha ricevuto dei dati errati per la creazione della domanda;
		\item
			\textbf{Postcondizione}: il sistema mostra un messaggio d'errore;
		\item
			\textbf{Scenario principale}: l'attore visualizza un messaggio d'errore;	
	\end{itemize}	


	\subsubsection{Caso d'uso UC8.2: Modifica domanda esistente}
	\label{UC8.2}
	\begin{figure}[h]
		\centering
			\includegraphics[scale=0.41,keepaspectratio]{UML/UC8_2.png}
		\caption{UC8.2: Modifica domanda esistente}
	\end{figure}
	\FloatBarrier
	\begin{itemize}
		\item
			\textbf{Attori}: utente autenticato, utente autenticato pro;
		\item		
			\textbf{Descrizione}: l'attore può modificare una domanda che aveva creato precedentemente;
		\item
			\textbf{Precondizione}: il sistema visualizza la schermata di modifica di una domanda;
		\item
			\textbf{Postcondizione}: l'attore modifica una domanda;
		\item
			\textbf{Scenario principale}: 
			\begin{enumerate}
				\item
				L'attore può selezionare la domanda da modificare (UC8.2.1);
				\item
				L'attore può confermare le modifiche apportate (UC8.2.2);
				\item
				L'attore può richiamare il \textit{wizard\ped{G}} per modificare una domanda vero/falso (UC8.2.4);
				\item
				L'attore può richiamare il \textit{wizard\ped{G}} per modificare una domanda a risposta multipla (UC8.2.5);
				\item
				L'attore può richiamare il \textit{wizard\ped{G}} per modificare un esercizio a riempimento di spazi vuoti (UC8.2.6);
				\item
				L'attore può richiamare il \textit{wizard\ped{G}} per modificare una domanda di collegamento (UC8.2.7);
				\item
				L'attore può richiamare il \textit{wizard\ped{G}} per modificare una domanda a ordinamento di immagini (UC8.2.8);
				\item
				L'attore può richiamare il \textit{wizard\ped{G}} per modificare una domanda a ordinamento di stringhe (UC8.2.9);
				\item
				L'attore può richiamare il \textit{wizard\ped{G}} per modificare una domanda con area cliccabile nell'immagine (UC8.2.10).
			\end{enumerate}
	       		
	 	\item
			\textbf{Estensioni}: l'attore visualizza un messaggio d'errore relativo alla modifica della domanda (UC8.2.3);
	\end{itemize}
	
		\subsubsection{Caso d'uso UC8.2.1: Selezione domande da modificare}
		\begin{itemize}
			\item \textbf{Attori}: utente autenticato, utente autenticato pro;
			\item \textbf{Descrizione}: l'attore può selezionare la domanda da modificare;
			\item \textbf{Precondizione}: il sistema mostra la schermata di selezione della domanda da modificare;
			\item \textbf{Postcondizione}: il sistema ha ricevuto dall'attore la domanda da modificare; 
			\item \textbf{Scenario principale}: l'attore seleziona la domanda da modificare.
			
		\end{itemize}

	\subsubsection{Caso d'uso UC8.2.2: Conferma modifica}
	\begin{itemize}
		\item
			\textbf{Attori}: utente autenticato, utente autenticato pro;
		\item
			\textbf{Descrizione}: l'attore può confermare la modifica della domanda;
		\item		
			\textbf{Precondizione}: l'attore ha finito di modificare una domanda tramite un \textit{wizard\ped{G}};
		\item
			\textbf{Postcondizione}: l'attore ha modificato una domanda;
		\item
			\textbf{Scenario principale}: l'attore conferma la modifica della domanda;		
		\item
	 		\textbf{Scenari alternativi}: l'attore annulla la modifica della domanda.
	\end{itemize}		
	\subsubsection{Caso d'uso UC8.2.3: Visualizzazione errore modifica}
	\begin{itemize}
		\item
			\textbf{Attori}: utente autenticato, utente autenticato pro;
		\item
			\textbf{Descrizione}: l'attore può visualizzare un messaggio d'errore nel caso si fossero verificati uno o più scenari alternativi durante la modifica della domanda;
		\item		
			\textbf{Precondizione}: il sistema ha ricevuto dei dati errati per la modifica della domanda;
		\item
			\textbf{Postcondizione}: il sistema avvisa l'attore dell'errore verificatosi tramite un opportuno messaggio;
		\item
			\textbf{Scenario principale}: l'attore visualizza un messaggio d'errore.	
	\end{itemize}
	
%inclusione file latex wizard
\input{sezioni/CasiD'uso/wizardUC8/CasoD'usoUC8_2_4.tex}
\subsubsection{Caso d'uso UC8.2.5: Modifica domanda a ordinamento di immagini}
\label{UC8.2.5}
	\begin{figure}[h]
		\centering
			\includegraphics[scale=0.45,keepaspectratio]{UML/UC8_2_5.png}
		\caption{UC8.2.5: Modifica domanda a ordinamento di immagini}
	\end{figure}
\begin{itemize}
	\item\textbf{Attori}: utente autenticato, utente autenticato pro;
	\item\textbf{Descrizione}: l'attore può utilizzare la procedura guidata per la modifica di una domanda a ordinamento di immagini;
	\item\textbf{Precondizione}: il sistema ha ricevuto dall'attore la domanda da modificare; 
	\item \textbf{Postcondizione}: l'attore ha modificato una domanda a ordinamento di immagini;
	\item\textbf{Scenario principale}: 
	\begin{itemize}
		\item L'attore può modificare il testo della domanda (UC8.2.5.1);
		\item L'attore può modifica l'immagine relativa al testo della domanda (UC8.2.5.2);
		\item L'attore può modificare le immagini della sequenza che costituirà la risposta (UC8.2.5.3);
		\item L'attore può modifica l'ordine corretto delle immagini che costituiscono la risposta (UC8.2.5.4).
	\end{itemize}
\end{itemize}

\subsubsection{Caso d'uso UC8.2.5.1: Modifica testo della domanda}
\begin{itemize}
	\item\textbf{Attori}: utente autenticato, utente autenticato pro;
	\item \textbf{Descrizione}: l'attore può modificare il testo della domanda;
		\item
			\textbf{Precondizione}: il sistema mostra la funzionalità di modifica di una domanda a ordinamento di immagini; 
		\item
			\textbf{Postcondizione}: l'attore ha modificato il testo della domanda;
		\item
			\textbf{Scenario principale}: l'attore modifica il testo della domanda.	
	\end{itemize}

\subsubsection{Caso d'uso UC8.2.5.2: Modifica immagine della domanda}
\begin{itemize}
	\item\textbf{Attori}: utente autenticato, utente autenticato pro;
	\item \textbf{Descrizione}: l'attore può modificare l'immagine relativa al testo della domanda;
		\item
			\textbf{Precondizione}: il sistema mostra la funzionalità di modifica di una domanda di ordinamento immagini; 
		\item
			\textbf{Postcondizione}: l'attore ha modificato l'immagine relativa al testo della domanda;
		\item
			\textbf{Scenario principale}: l'attore modifica l'immagine relativa al testo della domanda. 	
	\end{itemize}

\subsubsection{Caso d'uso UC8.2.5.3: Modifica immagini come risposta}
\begin{itemize}
	\item\textbf{Attori}: utente autenticato, utente autenticato pro;
	\item\textbf{Descrizione}: l'attore può modificare le immagini che costituiscono la risposta alla domanda sostituendole con delle altre;
	\item\textbf{Precondizione}: il sistema mostra la funzionalità di modifica di una domanda di ordinamento immagini;  
	\item \textbf{Postcondizione}: l'attore ha inserito nuove immagini come risposta alla domanda;
	\item\textbf{Scenario principale}: l'attore sostituisce le immagini presenti come risposta alla domanda con delle altre.
\end{itemize}

\subsubsection{Caso d'uso UC8.2.5.4: Modifica ordine immagini come risposta}
\begin{itemize}
	\item\textbf{Attori}: utente autenticato, utente autenticato pro;
	\item\textbf{Descrizione}: l'attore può modificare l'ordine corretto delle immagini che rappresenta la soluzione della domanda;
	\item\textbf{Precondizione}: il sistema mostra la funzionalità di modifica di una domanda di ordinamento immagini; 
	\item \textbf{Postcondizione}: l'attore ha modificato l'ordine delle immagini che costituiscono la risposta alla domanda;
	\item\textbf{Scenario principale}: l'attore modifica l'ordine delle immagini che costituiscono la risposta alla domanda.
\end{itemize}

\subsubsection{Caso d'uso UC8.2.6: Modifica esercizio a riempimento di spazi vuoti}
	\label{UC8.2.6}
	\begin{figure}[h]
		\centering
			\includegraphics[scale=0.45,keepaspectratio]{UML/UC8_2_6.png}
		\caption{UC8.2.6: Modifica esercizio a riempimento di spazi vuoti}
	\end{figure}
	\FloatBarrier
	\begin{itemize}
		\item
			\textbf{Attori}: utente autenticato, utente autenticato pro;
		\item		
			\textbf{Descrizione}: l'attore può utilizzare la procedura guidata per la modifica di un esercizio a riempimento di spazi vuoti;
		\item
			\textbf{Precondizione}: il sistema ha ricevuto dall'attore la domanda da modificare;
		\item
			\textbf{Postcondizione}: l'attore ha modificato un esercizio a riempimento di spazi vuoti;
		\item
			\textbf{Scenario principale}:
	       		\begin{enumerate}
	       			\item
	       			L'attore può modificare il testo dell'esercizio (UC8.2.6.1);
	       			\item
	       			L'attore può modificare le parole che saranno sostituite con degli spazi vuoti dal sistema (UC8.2.6.2).
	 			\end{enumerate}
	\end{itemize}
	
\subsubsection{Caso d'uso UC8.2.6.1: Modifica testo dell'esercizio}
	\begin{itemize}
		\item
			\textbf{Attori}: utente autenticato, utente autenticato pro;
		\item		
			\textbf{Descrizione}: l'attore può modificare il testo dell'esercizio;
		\item
			\textbf{Precondizione}: il sistema mostra la funzionalità di modifica di un esercizio a riempimento di spazi;
		\item
			\textbf{Postcondizione}: l'attore ha modificato il testo dell'esercizio;
		\item
			\textbf{Scenario principale}: l'attore modifica il testo dell'esercizio.
	\end{itemize}


\subsubsection{Caso d'uso UC8.2.6.2: Modifica parole da oscurare}
	\begin{itemize}
		\item
			\textbf{Attori}: utente autenticato, utente autenticato pro;
		\item		
			\textbf{Descrizione}: l'attore può modificare le parole che saranno sostituite con degli spazi vuoti;
		\item
			\textbf{Precondizione}: il sistema mostra la funzionalità di modifica di un esercizio a riempimento di spazi;
		\item
			\textbf{Postcondizione}: l'attore ha modificato le parole che saranno sostituite con degli spazi vuoti;
		\item
			\textbf{Scenario principale}: l'attore modifica le parole che verranno oscurate dal sistema.
	\end{itemize}

\subsubsection{Caso d'uso UC8.2.7: Modifica domanda di collegamento}
\label{UC8.2.7}
\begin{figure}[h]
	\centering
	\includegraphics[scale=0.5,keepaspectratio]{UML/UC8_2_7.png}
	\caption{Caso d'uso UC8.2.7: Modifica domanda di collegamento}
\end{figure}
\FloatBarrier
\begin{itemize}
	\item \textbf{Attori}: \uau, \uaupro;
	\item \textbf{Descrizione}: l'attore può utilizzare la procedura guidata per la modifica di una domanda di collegamento; 
	\item \textbf{Precondizione}: il sistema ha ricevuto dall'attore la domanda da modificare; 
	\item \textbf{Postcondizione}: l'attore ha modificato una domanda di collegamento;
	\item \textbf{Scenario principale}: 
	\begin{enumerate}
			\item L'attore può modificare il testo della domanda (UC8.2.7.1);
			\item L'attore può inserire una coppia di elementi (UC8.2.7.2);
			\item L'attore può eliminare una coppia di elementi inserita (UC8.2.7.3);
			\item L'attore può modificare una coppia di elementi inserita (UC8.2.7.4).
		\end{enumerate}
	\item \textbf{Scenari alternativi}: se non ci sono almeno due coppie presenti nella lista delle coppie l'attore deve inserire una nuova coppia di elementi.
\end{itemize}

	\subsubsection{Caso d'uso UC8.2.7.1: Modifica testo della domanda}
	\label{UC8.2.7.1}
	\begin{itemize}
		\item
		\textbf{Attori}: \uau, \uaupro;
		\item		
		\textbf{Descrizione}: l'attore può modificare il testo della domanda;
		\item
		\textbf{Precondizione}: il sistema mostra la funzionalità di modifica di una domanda di collegamento; 
		\item
		\textbf{Postcondizione}: l'attore ha modificato il testo della domanda;
		\item
		\textbf{Scenario principale}: l'attore modifica il testo della domanda.	
	\end{itemize}

	\subsubsection{Caso d'uso UC8.2.7.2: Inserimento coppia di elementi}
	\label{UC8.2.7.2}
	\begin{figure}[h]
		\centering
		\includegraphics[scale=0.5,keepaspectratio]{UML/UC8_2_7_2.png}
		\caption{Caso d'uso UC8.2.7.2: Inserimento coppia di elementi}
	\end{figure}
	\FloatBarrier
	\begin{itemize}
		\item \textbf{Attori}: \uau, \uaupro;
		\item \textbf{Descrizione}: l'attore può inserire una coppia di elementi, sia immagini che testo o combinazioni di questi, che siano correlati tra loro in modo da indicare la soluzione della domanda; 
		\item \textbf{Precondizione}: il sistema mostra la funzionalità di modifica di una domanda di collegamento; 
		\item \textbf{Postcondizione}: l'attore ha inserito una coppia di elementi nella lista di coppie di elementi; 
		\item \textbf{Scenario principale}: 
		\begin{enumerate}
			\item L'attore inserisce come primo elemento un'immagine (UC8.2.7.2.1);
			\item L'attore inserisce come primo elemento un testo (UC8.2.7.2.2);
			\item L'attore inserisce come secondo elemento un'immagine (UC8.2.7.2.3);
			\item L'attore inserisce come secondo elemento un testo (UC8.2.7.2.4).	
		\end{enumerate}
	\end{itemize}
	
		\subsubsection{Caso d'uso UC8.2.7.2.1: Inserimento di un'immagine come primo elemento}
		\label{UC8.2.7.2.1}
		\begin{itemize}
			\item \textbf{Attori}: \uau, \uaupro;
			\item \textbf{Descrizione}: l'attore può inserire come primo elemento della coppia un'immagine;
			\item \textbf{Precondizione}: il sistema mostra la funzionalità di modifica di una domanda di collegamento; 
			\item \textbf{Postcondizione}: l'attore ha inserito come primo elemento un'immagine;
			\item \textbf{Scenario principale}: l'attore carica un'immagine come primo elemento della coppia.
		\end{itemize}
		
		\subsubsection{Caso d'uso UC8.2.7.2.2: Inserimento di un testo come primo elemento}
		\label{UC8.2.7.2.2}
		\begin{itemize}
			\item \textbf{Attori}: \uau, \uaupro;
			\item \textbf{Descrizione}: l'attore può inserire come primo elemento della coppia un testo;
			\item \textbf{Precondizione}: il sistema mostra la funzionalità di modifica di una domanda di collegamento; 
			\item \textbf{Postcondizione}: l'attore ha inserito come primo elemento un testo;
			\item \textbf{Scenario principale}: l'attore inserisce del testo come primo elemento della coppia.
		\end{itemize}
		
			\subsubsection{Caso d'uso UC8.2.7.2.3: Inserimento di un'immagine come secondo elemento}
		\label{UC8.2.7.2.3}
		\begin{itemize}
			\item \textbf{Attori}: \uau, \uaupro;
			\item \textbf{Descrizione}: l'attore può inserire come secondo elemento della coppia un'immagine;
			\item \textbf{Precondizione}: il sistema mostra la funzionalità di modifica di una domanda di collegamento; 
			\item \textbf{Postcondizione}: l'attore ha inserito come secondo elemento un'immagine;
			\item \textbf{Scenario principale}: l'attore carica un'immagine come secondo elemento della coppia.
		\end{itemize}
		
		\subsubsection{Caso d'uso UC8.2.7.2.4: Inserimento di un testo come secondo elemento}
		\label{UC8.2.7.2.4}
		\begin{itemize}
			\item \textbf{Attori}: \uau, \uaupro;
			\item \textbf{Descrizione}: l'attore può inserire come secondo elemento della coppia un testo;
			\item \textbf{Precondizione}:il sistema mostra la funzionalità di modifica di una domanda di collegamento; 
			\item \textbf{Postcondizione}: l'attore ha inserito come secondo elemento un testo;
			\item \textbf{Scenario principale}: l'attore inserisce del testo come secondo elemento della coppia.
		\end{itemize}
	
	\subsubsection{Caso d'uso UC8.2.7.3: Eliminazione coppia di elementi}
	\label{UC8.2.7.3}
	\begin{figure}[h]
		\centering
		\includegraphics[scale=0.5,keepaspectratio]{UML/UC8_2_7_3.png}
		\caption{Caso d'uso UC8.2.7.3: Eliminazione coppia di elementi}
	\end{figure}
	\FloatBarrier
	\begin{itemize}
		\item \textbf{Attori}: \uau, \uaupro;
		\item \textbf{Descrizione}: l'attore può rimuovere una coppia di elementi dalla lista di coppie di elementi;
		\item \textbf{Precondizione}: il sistema mostra la funzionalità di modifica di una domanda di collegamento; 
		\item \textbf{Postcondizione}: l'attore ha eliminato una coppia di elementi dalla lista delle coppie di elementi;
		\item \textbf{Scenario principale}: l'attore può confermare l'eliminazione di una coppia di elementi (UC8.2.7.3.1);
		\item \textbf{Scenari alternativi}: l'attore annulla l'operazione tornando alla schermata precedente.
	\end{itemize}

		\subsubsection{Caso d'uso UC8.2.7.3.1: Conferma eliminazione coppia di elementi}
		\label{UC8.2.7.3.1}
		\begin{itemize}
			\item \textbf{Attori}: \uau, \uaupro;
			\item \textbf{Descrizione}: l'attore può confermare la rimozione di una coppia di elementi;
			\item \textbf{Precondizione}: il sistema mostra la funzionalità di modifica di una domanda di collegamento; 
			\item \textbf{Postcondizione}: l'attore ha confermato l'eliminazione della coppia di elementi;
			\item \textbf{Scenario principale}: l'attore conferma la rimozione della coppia di elementi.
		\end{itemize}

	\subsubsection{Caso d'uso UC8.2.7.4: Modifica coppia di elementi}
	\label{UC8.2.7.4}
	\begin{figure}[h]
		\centering
		\includegraphics[scale=0.5,keepaspectratio]{UML/UC8_2_7_4.png}
		\caption{Caso d'uso UC8.2.7.4: Modifica coppia di elementi}
	\end{figure}
	\FloatBarrier
	\begin{itemize}
		\item \textbf{Attori}: \uau, \uaupro;
		\item \textbf{Descrizione}: l'attore può modificare una coppia di elementi presente nella lista di coppie di elementi;
		\item \textbf{Precondizione}: il sistema mostra la funzionalità di modifica di una domanda di collegamento; 
		\item \textbf{Postcondizione}: l'attore ha modificato una coppia di elementi presente nella lista di coppie di elementi; 
		\item \textbf{Scenario principale}: 
		\begin{enumerate}
			\item L'attore può modificare un elemento cambiandone il testo (UC8.2.7.4.1);
			\item L'attore può modificare un elemento cambiandone l'immagine (UC8.2.7.4.2);
			\item L'attore può modificare un elemento facendolo passare da testo ad immagine (UC8.2.7.4.3);
			\item L'attore può modificare un elemento facendolo passare da immagine a testo (UC8.2.7.4.4).	
		\end{enumerate}
	\end{itemize}
	
		\subsubsection{Caso d'uso UC8.2.7.4.1: Modifica testo di un elemento}
		\label{UC8.2.7.4.1}
		\begin{itemize}
			\item \textbf{Attori}: \uau, \uaupro;
			\item \textbf{Descrizione}: l'attore può modificare il testo di un elemento;
			\item \textbf{Precondizione}: il sistema mostra la funzionalità di modifica di una domanda di collegamento; 
			\item \textbf{Postcondizione}: l'attore ha modificato il testo di un elemento;
			\item \textbf{Scenario principale}: l'attore modifica il testo di un elemento.  
		\end{itemize}
		
		\subsubsection{Caso d'uso UC8.2.7.4.2: Modifica immagine di un elemento}
		\label{UC8.2.7.4.2}
		\begin{itemize}
			\item \textbf{Attori}: \uau, \uaupro;
			\item \textbf{Descrizione}: l'attore può caricare un'altra immagine per un elemento;
			\item \textbf{Precondizione}: il sistema mostra la funzionalità di modifica di una domanda di collegamento; 
			\item \textbf{Postcondizione}: l'attore ha inserito un'altra immagine per un elemento;
			\item \textbf{Scenario principale}: l'attore carica un'altra immagine per un elemento.
		\end{itemize}
		
		\subsubsection{Caso d'uso UC8.2.7.4.3: Cambia testo in immagine}
		\label{UC8.2.7.4.3}
		\begin{itemize}
			\item \textbf{Attori}: \uau, \uaupro;
			\item \textbf{Descrizione}: l'attore può modificare un elemento facendolo diventare un'immagine al posto di un testo;
			\item \textbf{Precondizione}: il sistema mostra la funzionalità di modifica di una domanda di collegamento; 
			\item \textbf{Postcondizione}: l'attore ha fatto diventare un'immagine un elemento che prima era un testo;
			\item \textbf{Scenario principale}: l'attore inserisce un'immagine come modifica dell'elemento.  
		\end{itemize}
		
		\subsubsection{Caso d'uso UC8.2.7.4.4: Cambia immagine in testo}
		\label{UC8.2.7.4.4}
		\begin{itemize}
			\item \textbf{Attori}: \uau, \uaupro;
			\item \textbf{Descrizione}: l'attore può modificare un elemento facendolo diventare un testo al posto di un'immagine;
			\item \textbf{Precondizione}: il sistema mostra la funzionalità di modifica di una domanda di collegamento; 
			\item \textbf{Postcondizione}: l'attore ha fatto diventare un testo un elemento che prima era un'immagine;
			\item \textbf{Scenario principale}: l'attore inserisce del testo come modifica dell'elemento.  
		\end{itemize}
\input{sezioni/CasiD'uso/wizardUC8/CasoD'usoUC8_2_8.tex}
\input{sezioni/CasiD'uso/wizardUC8/CasoD'usoUC8_2_9.tex}
\input{sezioni/CasiD'uso/wizardUC8/CasoD'usoUC8_2_10.tex}		
