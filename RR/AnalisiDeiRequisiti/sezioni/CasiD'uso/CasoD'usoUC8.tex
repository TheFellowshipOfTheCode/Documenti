\newpage
\subsection{Caso d'uso UC8: Gestione delle domande}
	\label{UC8}
	\begin{figure}[h]
		\centering
			\includegraphics[scale=0.45,keepaspectratio]{UML/UC8.png}
		\caption{UC8: Gestione delle domande}
	\end{figure}
	\FloatBarrier
	\begin{itemize}
		\item
			\textbf{Attori}: utente autenticato, utente autenticato pro;
		\item		
			\textbf{Descrizione}: l'attore può creare e modificare domande;
		\item
			\textbf{Precondizione}: l'attore è autenticato presso il sistema; 
		\item
			\textbf{Postcondizione}: l'attore ha compiuto una delle operazioni appartenenti a questa funzionalità;
		\item
			\textbf{Scenario principale}:
	       		\begin{enumerate}
					\item
					L'attore crea una nuova domanda (UC8.1);
					\item
					L'attore modifica una domanda (UC8.2).
	 			\end{enumerate}
	\end{itemize}
\subsubsection{Caso d'uso UC8.1: Creazione nuova domanda}
	\label{UC8.1}
	\begin{figure}[h]
		\centering
			\includegraphics[scale=0.45,keepaspectratio]{UML/UC8_1.png}
		\caption{UC8.1: Creazione nuova domanda}
	\end{figure}
	\FloatBarrier
	\begin{itemize}
		\item
			\textbf{Attori}: utente autenticato, utente autenticato pro;
		\item		
			\textbf{Descrizione}: l'attore può creare una nuova domanda;
		\item
			\textbf{Precondizione}: l'attore ha selezionato la funzionalità di creazione domanda;
		\item
			\textbf{Postcondizione}: l'attore crea una domanda;		
		\item
			\textbf{Scenario principale}:
	       		\begin{enumerate}
					\item
					L'attore può inserire l'argomento relativo alla nuova domanda (UC8.1.1);
					\item
					L'attore può inserire le parole chiave relative alla nuova domanda (UC8.1.2);
					\item
					L'attore può selezionare la tipologia di domanda (UC8.1.3);
					\item
					L'attore può confermare la creazione della domanda (UC8.1.4).
	 			\end{enumerate}
	 	\item
			\textbf{Estensioni}: l'attore visualizza un messaggio d'errore relativo alla creazione della domanda (UC8.1.5);
	 	\item
	 		\textbf{Scenari alternativi}:
				\begin{itemize}
					\item 	
						L'argomento non è stato inserito;
					\item
						Le parole chiave non sono state inserite;
					\item
						La tipologia di domanda non è stata selezionata.	
				\end{itemize}
	\end{itemize}
	\subsubsection{Caso d'uso UC8.1.1: Selezione argomento}
	\begin{itemize}
		\item
			\textbf{Attori}: utente autenticato, utente autenticato pro;
		\item
			\textbf{Descrizione}: l'attore può indicare un argomento tra quelli presenti;
		\item		
			\textbf{Precondizione}: l'attore ha selezionato la funzionalità di creazione domanda;
		\item
			\textbf{Postcondizione}: l'attore ha selezionato un argomento;
		\item
			\textbf{Scenario principale}: l'attore seleziona l'argomento da assegnare alla nuova domanda.		
	\end{itemize}
		
	\subsubsection{Caso d'uso UC8.1.2: Inserimento parole chiave}
	\begin{itemize}
		\item
			\textbf{Attori}: utente autenticato, utente autenticato pro;
		\item
			\textbf{Descrizione}: l'attore può inserire le parole chiave relative alla nuova domanda per specificare più dettagliatamente l'argomento della domanda;
		\item		
			\textbf{Precondizione}: l'attore ha selezionato la funzionalità di creazione domanda;
		\item
			\textbf{Postcondizione}: l'attore ha selezionato delle parole chiave;
		\item
			\textbf{Scenario principale}: l'attore inserisce delle parole chiave relative alla nuova domanda.	
	\end{itemize}


	\subsubsection{Caso d'uso UC8.1.3: Selezione tipologia di domanda}
	\label{UC8.1.3}
	\begin{figure}[h]
		\centering
			\includegraphics[scale=0.45,keepaspectratio]{UML/UC8_1_3.png}
		\caption{UC8.1.3: Selezione tipologia di domanda}
	\end{figure}
	\FloatBarrier
	\begin{itemize}
		\item
			\textbf{Attori}: utente autenticato, utente autenticato pro;
		\item
			\textbf{Descrizione}: l'attore può scegliere la tipologia di domanda che vuole inserire;
		\item		
			\textbf{Precondizione}: l'attore ha selezionato la funzionalità di creazione domanda;
		\item
			\textbf{Postcondizione}: l'utente ha selezionato la tipologia di domanda da inserire;
				\item \textbf{Scenario principale}: 
					\begin{enumerate}
					\item
					L'attore può richiamare il wizard per creare una domanda vero/falso (UC8.1.3.1);
					\item
					L'attore può richiamare il wizard per creare una domanda a risposta multipla (UC8.1.3.2);
					\item
					L'attore può richiamare il wizard per creare un esercizio a riempimento di spazi vuoti (UC8.1.3.3);
					\item
					L'attore può richiamare il wizard per creare una domanda di collegamento (UC8.1.3.4);
					\item
					L'attore può richiamare il wizard per creare una domanda a ordinamento di immagini (UC8.1.3.5);
					\item
					L'attore può richiamare il wizard per creare una domanda a ordinamento di stringhe (UC8.1.3.6);
					\item
					L'attore può richiamare il wizard per creare una domanda con area cliccabile nell'immagine (UC8.1.3.7).
	 			\end{enumerate}
	\end{itemize}

	%inclusione file latex wizard
\subsubsection{Caso d'uso UC8.1.3.1: Creazione domanda vero/falso}
	\label{UC8.1.3.1}
	\begin{figure}[h]
		\centering
			\includegraphics[scale=0.45,keepaspectratio]{UML/UC8_1_3_1.png}
		\caption{UC8.1.3.1: Creazione domanda vero/falso}
	\end{figure}
	\FloatBarrier
	\begin{itemize}
		\item
			\textbf{Attori}: utente autenticato, utente autenticato pro;
		\item		
			\textbf{Descrizione}: l'attore può utilizzare la procedura guidata per la creazione di una domanda vero/falso;
		\item
			\textbf{Precondizione}: l'attore ha selezionato la funzionalità di creazione di una domanda vero/falso; 
		\item
			\textbf{Postcondizione}: l'attore ha creato una domanda vero/falso;
		\item
			\textbf{Scenario principale}:
	       		\begin{enumerate}
	       			\item
	       			L'attore può inserire il testo della domanda (UC8.1.3.1.1);
	       			\item
	       			L'attore può inserire un'immagine relativa al testo della domanda (UC8.1.3.1.2);
					\item
					L'attore può indicare la risposta corretta tramite uno strumento di selezione (UC8.1.3.1.3).
	 			\end{enumerate}
	\end{itemize}

\subsubsection{Caso d'uso UC8.1.3.1.1: Inserimento testo della domanda}
	\begin{itemize}
		\item
			\textbf{Attori}: utente autenticato, utente autenticato pro;
		\item		
			\textbf{Descrizione}: l'attore può inserire il testo della domanda;
		\item
			\textbf{Precondizione}: l'attore ha selezionato la funzionalità di creazione di una domanda vero/falso; 
		\item
			\textbf{Postcondizione}: l'attore ha inserito il testo della domanda;
		\item
			\textbf{Scenario principale}: l'attore inserisce il testo della domanda. 
	 			
	\end{itemize}
	
\subsubsection{Caso d'uso UC8.1.3.1.2: Inserimento immagine}
	\begin{itemize}
		\item
			\textbf{Attori}: utente autenticato, utente autenticato pro;
		\item		
			\textbf{Descrizione}: l'attore può inserire un'immagine relativa al testo della domanda;
		\item
			\textbf{Precondizione}: l'attore ha selezionato la funzionalità di creazione di una domanda vero/falso; 
		\item
			\textbf{Postcondizione}: l'attore ha inserito un'immagine relativa al testo della domanda;
		\item
			\textbf{Scenario principale}: l'attore inserisce un'immagine.						
	\end{itemize}
	

\subsubsection{Caso d'uso UC8.1.3.1.3: Selezione risposta corretta}
	\begin{itemize}
		\item
			\textbf{Attori}: utente autenticato, utente autenticato pro;
		\item		
			\textbf{Descrizione}: l'attore può indicare la risposta corretta;
		\item
			\textbf{Precondizione}: l'attore ha selezionato la funzionalità di creazione di una domanda vero/falso; 
		\item
			\textbf{Postcondizione}: l'attore ha selezionato la risposta corretta;
		\item
			\textbf{Scenario principale}: l'attore indica la risposta corretta.  			
	\end{itemize}
\subsubsection{Caso d'uso UC8.1.3.2: Creazione domanda a risposta multipla}
	\label{UC8.1.3.2}
	\begin{figure}[h]
		\centering
			\includegraphics[scale=0.45,keepaspectratio]{UML/UC8_1_3_2.png}
		\caption{UC8.1.3.2: Creazione domanda a risposta multipla}
	\end{figure}
	\FloatBarrier
	\begin{itemize}
		\item
			\textbf{Attori}: utente autenticato, utente autenticato pro;
		\item		
			\textbf{Descrizione}: l'attore può utilizzare la procedura guidata per la creazione di una domanda a risposta multipla;
		\item
			\textbf{Precondizione}: il sistema presenta all'attore la procedura guidata per la creazione di una domanda a risposta multipla;
		\item
			\textbf{Postcondizione}: l'attore ha creato una domanda a risposta multipla;
		\item
			\textbf{Scenario principale}:
	       		\begin{enumerate}
	       			\item
	       			L'attore può inserire il testo della domanda (UC8.1.3.2.1);
	       			\item
	       			L'attore può inserire un'immagine relativa al testo della domanda (UC8.1.3.2.2);
	       			\item
	       			L'attore può aggiungere almeno due opzioni di risposta (UC8.1.3.2.3);
					\item
					L'attore può indicare una o più risposte corrette (UC8.1.3.2.4).
	 			\end{enumerate}
	\end{itemize}

\subsubsection{Caso d'uso UC8.1.3.2.1: Inserimento testo della domanda}
	\begin{itemize}
		\item
			\textbf{Attori}: utente autenticato, utente autenticato pro;
		\item		
			\textbf{Descrizione}: l'attore può inserire il testo della domanda;
		\item
			\textbf{Precondizione}: il sistema presenta all'attore lo spazio destinato all'inserimento del testo della domanda;
		\item
			\textbf{Postcondizione}: l'attore ha inserito il testo della domanda;
		\item
			\textbf{Scenario principale}: l'attore inserisce il testo della domanda. 
	 			
	\end{itemize}
	
\subsubsection{Caso d'uso UC8.1.3.2.2: Inserimento immagine}
	\begin{itemize}
		\item
			\textbf{Attori}: utente autenticato, utente autenticato pro;
		\item		
			\textbf{Descrizione}: l'attore può inserire un'immagine relativa al testo della domanda;
		\item
			\textbf{Precondizione}: il sistema presenta all'attore la funzionalità di inserimento di un immagine; 
		\item
			\textbf{Postcondizione}: l'attore ha inserito un'immagine relativa al testo della domanda;
		\item
			\textbf{Scenario principale}: l'attore inserisce l'immagine.						
	\end{itemize}


	
\subsubsection{Caso d'uso UC8.1.3.2.3: Aggiunta opzioni di risposta}
	\label{UC8.1.3.2.3}
	\begin{figure}[h]
		\centering
			\includegraphics[scale=0.45,keepaspectratio]{UML/UC8_1_3_2_3.png}
		\caption{UC8.1.3.2.3: Aggiunta opzioni di risposta}
	\end{figure}
	\FloatBarrier
	\begin{itemize}
		\item
			\textbf{Attori}: utente autenticato, utente autenticato pro;
		\item		
			\textbf{Descrizione}: l'attore può aggiungere almeno due opzioni di risposta;
		\item
			\textbf{Precondizione}: il sistema presenta all'attore la funzionalità di aggiungere due o più opzioni di risposta; 
		\item
			\textbf{Postcondizione}: l'attore ha aggiunto due o più opzioni di risposta;
		\item
			\textbf{Scenario principale}:
	       		\begin{enumerate}
	       			\item
	       			L'attore può aggiungere opzioni di risposta che includono testo (UC8.1.3.2.3.1);
					\item
					L'attore può aggiungere opzioni di risposta che includono immagini (UC8.1.3.2.3.2).
	 			\end{enumerate}
	\end{itemize}	

\subsubsection{Caso d'uso UC8.1.3.2.3.1: Aggiunta opzioni di risposta che includono testo}
	\begin{itemize}
		\item
			\textbf{Attori}: utente autenticato, utente autenticato pro;
		\item		
			\textbf{Descrizione}: l'attore può aggiungere almeno due opzioni di risposta che includono testo;
		\item
			\textbf{Precondizione}: il sistema presenta all'attore la funzionalità di aggiungere due o più opzioni di risposta che includono testo;
		\item
			\textbf{Postcondizione}: l'attore ha aggiunto due o più opzioni di risposta che includono testo;
		\item
			\textbf{Scenario principale}: l'attore aggiunge due o più opzioni che includono testo.				
	\end{itemize}	

\subsubsection{Caso d'uso UC8.1.3.2.3.2: Aggiunta opzioni di risposta che includono immagini}
	\begin{itemize}
		\item
			\textbf{Attori}: utente autenticato, utente autenticato pro;
		\item		
			\textbf{Descrizione}: l'attore può aggiungere almeno due opzioni di risposta che includono immagini;
		\item
			\textbf{Precondizione}: il sistema presenta all'attore la funzionalità di aggiungere due o più opzioni di risposta che includono immagini;
		\item
			\textbf{Postcondizione}: l'attore ha aggiunto due o più opzioni di risposta che includono immagini;
		\item
			\textbf{Scenario principale}: l'attore aggiunge due o più opzioni che includono immagini. 				
	\end{itemize}	
		
\subsubsection{Caso d'uso UC8.1.3.2.4: Selezione una o più risposte corrette}
	\begin{itemize}
		\item
			\textbf{Attori}: utente autenticato, utente autenticato pro;
		\item		
			\textbf{Descrizione}: l'attore può indicare una o più risposte corrette;
		\item
			\textbf{Precondizione}: il sistema presenta all'attore la funzionalità di selezionare una o più risposte corrette;
		\item
			\textbf{Postcondizione}: l'attore ha selezionato una o più risposte corrette;
		\item
			\textbf{Scenario principale}: l'attore indica una o più risposte corrette. 			
	\end{itemize}
\subsubsection{Caso d'uso UC8.1.3.3: Creazione esercizi di riempimento degli spazi vuoti}
	\label{UC8.1.3.3}
	\begin{figure}[h]
		\centering
			\includegraphics[scale=0.45,keepaspectratio]{UML/UC8_1_3_3.png}
		\caption{UC8.1.3.3: Creazione esercizi di riempimento degli spazi vuoti}
	\end{figure}
	\FloatBarrier
	\begin{itemize}
		\item
			\textbf{Attori}: utente autenticato, utente autenticato pro;
		\item		
			\textbf{Descrizione}: l'attore può creare esercizi di riempimento degli spazi vuoti;
		\item
			\textbf{Precondizione}: l'attore ha selezionato la seguente funzionalità; 
		\item
			\textbf{Postcondizione}: l'attore ha creato un esercizio di riempimento degli spazi vuoti;
		\item
			\textbf{Scenario principale}:
	       		\begin{enumerate}
	       			\item
	       			L'attore può compilare il campo dati destinato alla scrittura del testo dell'esercizio (UC8.1.3.3.1);
	       			\item
	       			L'attore può indicare le parole che saranno sostituite con degli spazi vuoti dal sistema (UC8.1.3.3.2).
	 			\end{enumerate}
	\end{itemize}
	
\subsubsection{Caso d'uso UC8.1.3.3.1: Scrittura testo dell'esercizio}
	\begin{itemize}
		\item
			\textbf{Attori}: utente autenticato, utente autenticato pro;
		\item		
			\textbf{Descrizione}: l'attore può inserire il testo dell'esercizio di riempimento;
		\item
			\textbf{Precondizione}: l'attore può creare un esercizio di riempimento degli spazi vuoti; 
		\item
			\textbf{Postcondizione}: l'attore ha compilato il campo dati dedicato alla scrittura del testo dell'esercizio di riempimento;
		\item
			\textbf{Scenario principale}: l'attore compila il campo dati dedicato alla scrittura del testo dell'esercizio di riempimento.
	\end{itemize}


\subsubsection{Caso d'uso UC8.1.3.3.2: Indicazione parole da oscurare}
	\begin{itemize}
		\item
			\textbf{Attori}: utente autenticato, utente autenticato pro;
		\item		
			\textbf{Descrizione}: l'attore può indicare le parole che saranno sostituite con degli spazi vuoti;
		\item
			\textbf{Precondizione}: l'attore ha inserito il testo dell'esercizio; 
		\item
			\textbf{Postcondizione}: l'attore ha indicato le parole che saranno sostituite con degli spazi vuoti;
		\item
			\textbf{Scenario principale}: l'attore indica le parole che verranno oscurate dal sistema.
	\end{itemize}
\subsubsection{Caso d'uso UC8.1.3.4: Creazione domanda di collegamento}
\label{UC8.1.3.4}
\begin{figure}[h]
	\centering
\includegraphics[scale=0.5,keepaspectratio]{UML/UC8_1_3_4.png}
	\caption{Caso d'uso UC8.1.3.4: Creazione domanda di collegamento}
\end{figure}
\FloatBarrier
\begin{itemize}
	\item \textbf{Attori}: \uau, \uaupro;
	\item \textbf{Descrizione}: l'attore può utilizzare la procedura guidata per la creazione di una domanda di collegamento; 
	\item \textbf{Precondizione}: il sistema presenta all'attore la procedura guidata per la creazione di una domanda di collegamento;
	\item \textbf{Postcondizione}: l'attore ha creato una domanda di collegamento;
	\item \textbf{Scenario principale}: 
		\begin{enumerate}
			\item L'attore può inserire il testo della domanda (UC8.1.3.4.1);
			\item L'attore può inserire una coppia di elementi (UC8.1.3.4.2);
			\item L'attore può eliminare una coppia di elementi inserita (UC8.1.3.4.3);
			\item L'attore può modificare una coppia di elementi inserita (UC8.1.3.4.4).
		\end{enumerate}
	\item \textbf{Scenari alternativi}: se non ci sono almeno due coppie presenti nella lista delle coppie l'attore deve inserire una nuova coppia di elementi.
\end{itemize}

	\subsubsection{Caso d'uso UC8.1.3.4.1: Inserimento testo della domanda}
	\label{UC8.1.3.4.1}
	\begin{itemize}
		\item
		\textbf{Attori}: \uau, \uaupro;
		\item		
		\textbf{Descrizione}: l'attore può inserire il testo della domanda;
		\item
		\textbf{Precondizione}: il sistema presenta all'attore lo spazio destinato all'inserimento del testo della domanda;
		\item \textbf{Postcondizione}: l'attore ha inserito il testo della domanda;
		\item \textbf{Scenario principale}: l'attore inserisce il testo della domanda. 
	\end{itemize}

	\subsubsection{Caso d'uso UC8.1.3.4.2: Inserimento coppia di elementi}
	\label{UC8.1.3.4.2}
	\begin{figure}[h]
		\centering
		\includegraphics[scale=0.5,keepaspectratio]{UML/UC8_1_3_4_2.png}
		\caption{Caso d'uso UC8.1.3.4.2: Inserimento coppia di elementi}
	\end{figure}
	\FloatBarrier
	\begin{itemize}
		\item \textbf{Attori}: \uau, \uaupro;
		\item \textbf{Descrizione}: l'attore può inserire una coppia di elementi, sia immagini che testo o combinazioni di questi, che siano correlati tra loro in modo da indicare la soluzione della domanda; 
		\item \textbf{Precondizione}: il sistema presenta all'attore la funzionalità di inserimento di una o più coppie di elementi;
		\item \textbf{Postcondizione}: l'attore ha inserito una coppia di elementi nella lista di coppie di elementi; 
		\item \textbf{Scenario principale}: 
		\begin{enumerate}
			\item L'attore inserisce come primo elemento un'immagine (UC8.1.3.4.2.1);
			\item L'attore inserisce come primo elemento un testo (UC8.1.3.4.2.2);
			\item L'attore inserisce come secondo elemento un'immagine (UC8.1.3.4.2.3);
			\item L'attore inserisce come secondo elemento un testo (UC8.1.3.4.2.4).	
		\end{enumerate}
	\end{itemize}
	
		\subsubsection{Caso d'uso UC8.1.3.4.2.1: Inserimento di un'immagine come primo elemento}
		\label{UC8.1.3.4.2.1}
		\begin{itemize}
			\item \textbf{Attori}: \uau, \uaupro;
			\item \textbf{Descrizione}: l'attore può inserire come primo elemento della coppia un'immagine;
			\item \textbf{Precondizione}: il sistema presenta all'attore lo spazio destinato all'inserimento di un'immagine come primo elemento;
			\item \textbf{Postcondizione}: l'attore ha inserito come primo elemento un'immagine;
			\item \textbf{Scenario principale}: l'attore carica un'immagine come primo elemento della coppia.
		\end{itemize}
		
		\subsubsection{Caso d'uso UC8.1.3.4.2.2: Inserimento di un testo come primo elemento}
		\label{UC8.1.3.4.2.2}
		\begin{itemize}
			\item \textbf{Attori}: \uau, \uaupro;
			\item \textbf{Descrizione}: l'attore può inserire come primo elemento della coppia un testo;
			\item \textbf{Precondizione}: il sistema presenta all'attore lo spazio destinato all'inserimento di un testo come primo elemento;
			\item \textbf{Postcondizione}: l'attore ha inserito come primo elemento un testo;
			\item \textbf{Scenario principale}: l'attore inserisce del testo come primo elemento della coppia.
		\end{itemize}
		
			\subsubsection{Caso d'uso UC8.1.3.4.2.3: Inserimento di un'immagine come secondo elemento}
		\label{UC8.1.3.4.2.3}
		\begin{itemize}
			\item \textbf{Attori}: \uau, \uaupro;
			\item \textbf{Descrizione}: l'attore può inserire come secondo elemento della coppia un'immagine;
			\item \textbf{Precondizione}: il sistema presenta all'attore lo spazio destinato all'inserimento di un'immagine come secondo elemento;
			\item \textbf{Postcondizione}: l'attore ha inserito come secondo elemento un'immagine;
			\item \textbf{Scenario principale}: l'attore carica un'immagine come secondo elemento della coppia.
		\end{itemize}
		
		\subsubsection{Caso d'uso UC8.1.3.4.2.4: Inserimento di un testo come secondo elemento}
		\label{UC8.1.3.4.2.4}
		\begin{itemize}
			\item \textbf{Attori}: \uau, \uaupro;
			\item \textbf{Descrizione}: l'attore può inserire come secondo elemento della coppia un testo;
			\item \textbf{Precondizione}: il sistema presenta all'attore lo spazio destinato all'inserimento di un testo come secondo elemento;
			\item \textbf{Postcondizione}: l'attore ha inserito come secondo elemento un testo;
			\item \textbf{Scenario principale}: l'attore inserisce del testo come secondo elemento della coppia.
		\end{itemize}
	
	\subsubsection{Caso d'uso UC8.1.3.4.3: Eliminazione coppia di elementi}
	\label{UC8.1.3.4.3}
	\begin{figure}[h]
		\centering
		\includegraphics[scale=0.5,keepaspectratio]{UML/UC8_1_3_4_3.png}
		\caption{Caso d'uso UC8.1.3.4.3: Eliminazione coppia di elementi}
	\end{figure}
	\FloatBarrier
	\begin{itemize}
		\item \textbf{Attori}: \uau, \uaupro;
		\item \textbf{Descrizione}: l'attore può rimuovere una coppia di elementi dalla lista di coppie di elementi;
		\item \textbf{Precondizione}: il sistema presenta all'attore la funzionalità di eliminare una coppia di elementi;
		\item \textbf{Postcondizione}: l'attore ha eliminato una coppia di elementi dalla lista delle coppie di elementi;
		\item \textbf{Scenario principale}: l'attore può confermare l'eliminazione di una coppia di elementi (UC8.1.3.4.3.1);	
		\item \textbf{Scenari alternativi}: l'attore annulla l'operazione tornando alla schermata precedente.
	\end{itemize}

		\subsubsection{Caso d'uso UC8.1.3.4.3.1: Conferma eliminazione coppia di elementi}
		\label{UC8.1.3.4.3.1}
		\begin{itemize}
			\item \textbf{Attori}: \uau, \uaupro;
			\item \textbf{Descrizione}: l'attore può confermare la rimozione di una coppia di elementi;
			\item \textbf{Precondizione}: il sistema presenta all'attore la funzionalità di confermare l'eliminazione di una coppia di elementi;
			\item \textbf{Postcondizione}: l'attore ha confermato l'eliminazione della coppia di elementi;
			\item \textbf{Scenario principale}: l'attore conferma la rimozione della coppia di elementi.
		\end{itemize}

	\subsubsection{Caso d'uso UC8.1.3.4.4: Modifica coppia di elementi}
	\label{UC8.1.3.4.4}
	\begin{figure}[h]
		\centering
		\includegraphics[scale=0.5,keepaspectratio]{UML/UC8_1_3_4_4.png}
		\caption{Caso d'uso UC8.1.3.4.4: Modifica coppia di elementi}
	\end{figure}
	\FloatBarrier
	\begin{itemize}
		\item \textbf{Attori}: \uau, \uaupro;
		\item \textbf{Descrizione}: l'attore può modificare una coppia di elementi presente nella lista di coppie di elementi;
		\item \textbf{Precondizione}: il sistema presenta all'attore la funzionalità di modificare una coppia di elementi;
		\item \textbf{Postcondizione}: l'attore ha modificato una coppia di elementi presente nella lista di coppie di elementi; 
		\item \textbf{Scenario principale}: 
		\begin{enumerate}
			\item L'attore può modificare un elemento cambiandone il testo (UC8.1.3.4.4.1);
			\item L'attore può modificare un elemento cambiandone l'immagine (UC8.1.3.4.4.2);
			\item L'attore può modificare un elemento facendolo passare da testo ad immagine \\(UC8.1.3.4.4.3);
			\item L'attore può modificare un elemento facendolo passare da immagine a testo \\(UC8.1.3.4.4.4).	
		\end{enumerate}
	\end{itemize}
	
		\subsubsection{Caso d'uso UC8.1.3.4.4.1: Modifica testo di un elemento}
		\label{UC8.1.3.4.4.1}
		\begin{itemize}
			\item \textbf{Attori}: \uau, \uaupro;
			\item \textbf{Descrizione}: l'attore può modificare il testo di un elemento;
			\item \textbf{Precondizione}: il sistema presenta all'attore la funzionalità di modificare il testo di un elemento;
			\item \textbf{Postcondizione}: l'attore ha modificato il testo di un elemento;
			\item \textbf{Scenario principale}: l'attore modifica il testo di un elemento.  
		\end{itemize}
		
		\subsubsection{Caso d'uso UC8.1.3.4.4.2: Modifica immagine di un elemento}
		\label{UC8.1.3.4.4.2}
		\begin{itemize}
			\item \textbf{Attori}: \uau, \uaupro;
			\item \textbf{Descrizione}: l'attore può caricare un'altra immagine per un elemento;
			\item \textbf{Precondizione}: il sistema presenta all'attore la funzionalità di modificare l'immagine di un elemento; 
			\item \textbf{Postcondizione}: l'attore ha inserito un'altra immagine per un elemento;
			\item \textbf{Scenario principale}: l'attore carica un'altra immagine per un elemento.
		\end{itemize}
		
		\subsubsection{Caso d'uso UC8.1.3.4.4.3: Cambia testo in immagine}
		\label{UC8.1.3.4.4.3}
		\begin{itemize}
			\item \textbf{Attori}: \uau, \uaupro;
			\item \textbf{Descrizione}: l'attore può modificare un elemento facendolo diventare un'immagine al posto di un testo;
			\item \textbf{Precondizione}: il sistema presenta all'attore la funzionalità di modificare un elemento facendolo diventare un'immagine al posto di un testo;
			\item \textbf{Postcondizione}: l'attore ha fatto diventare un'immagine un elemento che prima era un testo;
			\item \textbf{Scenario principale}: l'attore inserisce un'immagine come modifica dell'elemento.  
		\end{itemize}
		
		\subsubsection{Caso d'uso UC8.1.3.4.4.4: Cambia immagine in testo}
		\label{UC8.1.3.4.4.4}
		\begin{itemize}
			\item \textbf{Attori}: \uau, \uaupro;
			\item \textbf{Descrizione}: l'attore può modificare un elemento facendolo diventare un testo al posto di un'immagine;
			\item \textbf{Precondizione}: il sistema presenta all'attore la funzionalità di modificare un elemento facendolo diventare un testo al posto di un'immagine;
			\item \textbf{Postcondizione}: l'attore ha fatto diventare un testo un elemento che prima era un'immagine;
			\item \textbf{Scenario principale}: l'attore inserisce del testo come modifica dell'elemento.  
		\end{itemize}
		

\subsubsection{Caso d’uso UC8.1.3.5: Creazione domanda a ordinamento di immagini}
\label{UC8.1.3.5}
	\begin{figure}[h]
		\centering
			\includegraphics[scale=0.45,keepaspectratio]{UML/UC8_1_3_5.png}
		\caption{UC8.1.3.5: Creazione domanda a ordinamento immagini}
	\end{figure}
	\FloatBarrier
\begin{itemize}
	\item\textbf{Attori}: utente autenticato, utente autenticato pro;
	\item\textbf{Scopo e descrizione}: l'attore  possono inserire una domanda del tipo ordinamento di immagini;
	\item\textbf{Precondizione}: il sistema mostra all'attore  il form di inserimento dei campi dati per la tipologia di domanda scelta; 
	\item \textbf{Postcondizione}: l'attore ha inserito tutti i campi dati obbligatori;
	\item\textbf{Scenario principale}:
	\begin{itemize}
		\item Gli attori devono compilare il campo dati destinato alla scrittura del testo della domanda (UC8.1.3.5.1);
		\item Gli attoriinserisce un'immagine per il testo della domanda (UC8.1.3.5.2);
		\item Gli attori devono inserire le immagini nel giusto ordine (UC8.1.3.5.3).
	\end{itemize}
\end{itemize}

\subsubsection{Caso d’uso UC8.1.3.5.1: Inserimento testo domanda}
\begin{itemize}
	\item\textbf{Attori}: utente autenticato, utente autenticato pro;
	\item\textbf{Scopo e descrizione}: lo scopo di questa funzionalità è offrire all'attore  la possibilità di inserire il testo della domanda;
	\item\textbf{Precondizione}: l'attore ha scelto la modalità di creazione della domanda a ordinamento di immagini; 
	\item \textbf{Postcondizione}: l'attore ha inserito il testo della domanda;
	\item\textbf{Scenario principale}: l'attore  devono inserire il testo della domanda. 
\end{itemize}

\subsubsection{Caso d’uso UC8.1.3.5.2: Inserimento immagine per testo domanda}
\label{UC8.1.3.5.2}
\begin{figure}[h]
	\centering
	\includegraphics[scale=0.45,keepaspectratio]{UML/UC8_1_3_5_2.png}
	\caption{UC8.1.3.5.2: Inserimento immagine per testo domanda}
\end{figure}
\FloatBarrier
\begin{itemize}
	\item\textbf{Attori}: utente autenticato, utente autenticato pro;
	\item\textbf{Scopo e descrizione}: lo scopo di questa funzionalità è offrire all'attore  la possibilità di inserire un'immagine relativa domanda;
	\item\textbf{Precondizione}: l'attore ha scelto la modalità di creazione della domanda a ordinamento di immagini; 
	\item \textbf{Postcondizione}: l'attore ha inserito un'immagine relativa alla domanda;
	\item\textbf{Scenario principale}: 
		\begin{enumerate}
			\item L'attore inserisce un'immagine relativa alla domanda;
			\item L'attore può eliminare l'immagine appena inserita (UC8.1.3.5.2.1).
		\end{enumerate}

\end{itemize}

\subsubsection{Caso d’uso UC8.1.3.5.2.1: Eliminazione immagine per testo domanda}
\begin{itemize}
	\item\textbf{Attori}: utente autenticato, utente autenticato pro;
	\item\textbf{Scopo e descrizione}: lo scopo di questa funzionalità è offrire all'attore  la possibilità di rimuovere l'immagine relativa ad domanda appena inserita;
	\item\textbf{Precondizione}: l'attore ha scelto la modalità di eliminazione dell'immagine usata come testo; 
	\item \textbf{Postcondizione}: l'attore ha eliminato l'immagine relativa alla domanda;
	\item\textbf{Scenario principale}: l'attore rimuove l'immagine relativa alla domanda. 
\end{itemize}

\subsubsection{Caso d’uso UC8.1.3.5.3: Inserimento immagini della risposta}
\label{UC8.1.3.5.3}
\begin{figure}[h]
	\centering
	\includegraphics[scale=0.45,keepaspectratio]{UML/UC8_1_3_5_3.png}
	\caption{UC8.1.3.5.3: Inserimento immagini della risposta}
\end{figure}
\begin{itemize}
	\item\textbf{Attori}: utente autenticato, utente autenticato pro;
	\item\textbf{Scopo e descrizione}: lo scopo di questa funzionalità è offrire all'attore  la possibilità di inserire le immagini per la risposta della relativa domanda;
	\item\textbf{Precondizione}: l'attore ha scelto la modalità di creazione della domanda a ordinamento di immagini; 
	\item \textbf{Postcondizione}: l'attore ha inserito delle immagini in ordine corretto per la risposta relativa alla domanda;
	\item\textbf{Scenario principale}:
		\begin{enumerate}
			\item L''attore inserisce delle immagini in ordine corretto per la relativa domanda;
			\item L'attore può eliminare un'immagine appena inserita (UC8.1.3.5.3.1).
		\end{enumerate}
\end{itemize}
\subsubsection{Caso d’uso UC8.1.3.5.3.1: Eliminazione immagine di una risposta}
\begin{itemize}
	\item\textbf{Attori}: utente autenticato, utente autenticato pro;
	\item\textbf{Scopo e descrizione}: lo scopo di questa funzionalità è offrire all'attore  la possibilità di rimuovere un'immagine inserita come risposta;
	\item\textbf{Precondizione}: l'attore ha scelto la modalità di eliminazione di un'immagine usata come risposta; 
	\item \textbf{Postcondizione}: l'attore ha eliminato un'immagine relativa alla risposta;
	\item\textbf{Scenario principale}: l'attore rimuove un'immagine relativa alla risposta. 
\end{itemize}
\subsubsection{Caso d'uso UC8.1.3.6: Creazione domanda a ordinamento di stringhe}
	\label{UC8.1.3.6}
	\begin{figure}[h]
		\centering
			\includegraphics[scale=0.45,keepaspectratio]{UML/UC8_1_3_6.png}
		\caption{UC8.1.3.6: Creazione domanda a ordinamento di stringhe}
	\end{figure}
	\FloatBarrier
\begin{itemize}
	\item\textbf{Attori}: utente autenticato, utente autenticato pro;
	\item\textbf{Descrizione}: l'attore può utilizzare la procedura guidata per la creazione di una domanda a ordinamento di stringhe;
	\item \textbf{Precondizione}: l'attore ha selezionato la funzionalità di creazione di una domanda a ordinamento stringhe;
	\item\textbf{Postcondizione}: l'attore ha creato una domanda a ordinamento di stringhe;
	\item\textbf{Scenario principale}:
		\begin{enumerate}
			\item L'attore può inserire il testo della domanda (UC8.1.3.6.1);
			\item L'attore può inserire le stringhe che compongono la risposta (UC8.1.3.6.2);
			\item L'attore può indicare la soluzione della sequenza di stringhe (UC8.1.3.6.3).
		\end{enumerate}
\end{itemize}

\subsubsection{Caso d'uso UC8.1.3.6.1: Inserimento testo domanda}
	\begin{itemize}
		\item \textbf{Attori}: utente autenticato, utente autenticato pro;
		\item \textbf{Descrizione}: l'attore può inserire il testo della domanda;
		\item\textbf{Precondizione}: l'attore ha selezionato la funzionalità di creazione di una domanda a ordinamento di stringhe;
		\item \textbf{Postcondizione}: l'attore ha inserito il testo della domanda;
		\item\textbf{Scenario principale}: l'attore inserisce il testo della domanda.
	\end{itemize}
	
\subsubsection{Caso d'uso UC8.1.3.6.2: Inserimento stringhe di composizione sequenza}
	\begin{itemize}
		\item \textbf{Attori}: utente autenticato, utente autenticato pro;
		\item \textbf{Descrizione}: l'attore può inserire le stringhe che costituiscono la risposta alla domanda;
		\item\textbf{Precondizione}: l'attore ha selezionato la funzionalità di creazione di una domanda a ordinamento di stringhe;
		\item \textbf{Postcondizione}: l'attore ha inserito le stringhe che costituiscono la risposta alla domanda;
		\item\textbf{Scenario principale}: l'attore inserisce le stringhe che costituiscono la risposta alla domanda.
	\end{itemize}
	
\subsubsection{Caso d'uso UC8.1.3.6.3: Composizione soluzione della sequenza}
	\begin{itemize}
		\item \textbf{Attori}: utente autenticato, utente autenticato pro;
		\item \textbf{Descrizione}: l'attore può indicare la soluzione della domanda mettendo nell'ordine corretto la sequenza di stringhe;
		\item\textbf{Precondizione}: l'attore ha selezionato la funzionalità di creazione di una domanda a ordinamento di immagini;
		\item \textbf{Postcondizione}: l'attore ha indicato la soluzione della sequenza;
		\item\textbf{Scenario principale}: l'attore indica la soluzione della sequenza. 
	\end{itemize}

\subsubsection{Caso d'uso UC8.1.3.7: Creazione domanda con area cliccabile nell'immagine}
\label{UC8.1.3.7}
\begin{figure}[h]
	\centering
	\includegraphics[scale=0.5,keepaspectratio]{UML/UC8_1_3_7.png}
	\caption{UC8.1.3.7: Creazione domanda a risposta multipla}
\end{figure}
\FloatBarrier
\begin{itemize}
	\item \textbf{Attori}: utente autenticato, utente autenticato pro;
	\item \textbf{Descrizione}: questa funzionalità offre al'attore la possibilità di creare domande la cui risposta è selezionabile all'interno di aree cliccabili in un immagine;
	\item \textbf{Precondizione}: l'attore hanno selezionato la creazione di una domanda con area cliccabile nell'immagine; 
	\item \textbf{Postcondizione}: l'attore hanno creato una domanda con area cliccabile nell'immagine;
	\item \textbf{Scenario principale}:
		\begin{enumerate}
	       	\item L'attore può compilare il campo dati destinato alla scrittura del testo della domanda (UC8.1.3.7.1);
	        \item L'attore può inserire una immagine relativa al testo della domanda (UC8.1.3.7.2);
			\item L'attore può scegliere quante aree saranno selezionabili all'interno dell'immagine (UC8.1.3.7.3);
			\item L'attore può scegliere le aree selezionabili all'interno dell'immagine (UC8.1.3.7.4);
	 	\end{enumerate}
\end{itemize}

\subsubsection{Caso d'uso UC8.1.3.7.1: Compilazione testo domanda}
\begin{itemize}
	\item \textbf{Attori}: utente autenticato, utente autenticato pro;
	\item \textbf{Descrizione}: l'attore può inserire un testo per la domanda che vogliono creare;
	\item \textbf{Precondizione}: l'attore ha selezionato la modalità di creazione di una domanda con area cliccabile;
	\item \textbf{Postcondizione}: l'attore ha inserito il testo della domanda.
\end{itemize}

\subsubsection{Caso d'uso UC8.1.3.7.2: Inserimento immagine}
\begin{itemize}
	\item \textbf{Attori}: utente autenticato, utente autenticato pro;
	\item \textbf{Descrizione}: l'attore ha la possibilità di inserire un'immagine relativa al testo della domanda;
	\item \textbf{Precondizione}: l'attore ha selezionato la modalità di creazione di una domanda con area cliccabile; 
	\item \textbf{Postcondizione}: l'attore ha inserito l'immagine;
	\item \textbf{Scenario principale}: l'attore inserisce l'immagine. 	
\end{itemize}

\subsubsection{Caso d'uso UC8.1.3.7.3: Scelta numero aree selezionabili}
\begin{itemize}
	\item \textbf{Attori}: utente autenticato, utente autenticato pro;
	\item \textbf{Descrizione}: l'attore ha la possibilità di scegliere il numero di aree selezionabili all'interno dell'immagine;
	\item \textbf{Precondizione}: l'attore ha selezionato la modalità di creazione di una domanda con area cliccabile; 
	\item \textbf{Postcondizione}: l'attore ha scelto il numero di aree selezionabili all'interno dell'immagine;
	\item \textbf{Scenario principale}: l'attore sceglie il numero di aree selezionabili all'interno dell'immagine. 	
\end{itemize}

\subsubsection{Caso d'uso UC8.1.3.7.4: Scelta aree selezionabili}
\begin{itemize}
	\item \textbf{Attori}: utente autenticato, utente autenticato pro;
	\item \textbf{Descrizione}: l'attore ha la possibilità di scegliere le aree selezionabili all'interno dell'immagine;
	\item \textbf{Precondizione}: l'attore ha selezionato la modalità di creazione di una domanda con area cliccabile; 
	\item \textbf{Postcondizione}: l'attore ha scelto le aree selezionabili all'interno dell'immagine;
	\item \textbf{Scenario principale}: l'attore sceglie le aree selezionabili all'interno dell'immagine. 	
\end{itemize}




	\subsubsection{Caso d'uso UC8.1.4: Conferma creazione}
	\begin{itemize}
		\item
			\textbf{Attori}: utente autenticato, utente autenticato pro;
		\item
			\textbf{Descrizione}: l'attore può confermare la creazione della domanda;
		\item		
			\textbf{Precondizione}: l'attore ha finito di creare una domanda tramite un wizard;
		\item
			\textbf{Postcondizione}: l'attore ha creato una domanda;
		\item
			\textbf{Scenario principale}: l'attore conferma la creazione della domanda;		
		\item
	 		\textbf{Scenari alternativi}: l'attore annulla la creazione della domanda.
					
	\end{itemize}	
	
	\subsubsection{Caso d'uso UC8.1.5: Visualizzazione errore creazione domanda}
	\begin{itemize}
		\item
			\textbf{Attori}: utente autenticato, utente autenticato pro;
		\item
			\textbf{Descrizione}: l'attore può visualizzare un messaggio d'errore nel caso si fossero verificati uno o più scenari alternativi durante la creazione della domanda;
		\item		
			\textbf{Precondizione}: il sistema ha ricevuto dei dati errati per la creazione della domanda;
		\item
			\textbf{Postcondizione}: il sistema mostra un messaggio d'errore;
		\item
			\textbf{Scenario principale}: l'attore visualizza un messaggio d'errore;	
	\end{itemize}	


	\subsubsection{Caso d'uso UC8.2: Modifica domanda esistente}
	\label{UC8.2}
	\begin{figure}[h]
		\centering
			\includegraphics[scale=0.45,keepaspectratio]{UML/UC8_2.png}
		\caption{UC8.2: Modifica domanda esistente}
	\end{figure}
	\FloatBarrier
	\begin{itemize}
		\item
			\textbf{Attori}: utente autenticato, utente autenticato pro;
		\item		
			\textbf{Descrizione}: l'attore può modificare una domanda che aveva creato precedentemente;
		\item
			\textbf{Precondizione}: l'attore ha selezionato la funzionalità di modifica domanda;
		\item
			\textbf{Postcondizione}: l'attore modifica una domanda;
		\item
			\textbf{Scenario principale}:
				\begin{enumerate}
					\item
						L'attore può selezionare la domanda da modificare (UC8.2.1);
				\end{enumerate}
	       		
	 	\item
			\textbf{Estensioni}: l'attore visualizza un messaggio d'errore relativo alla modifica della domanda (UC8.2.3);
	 	\item
	 		\textbf{Scenari alternativi}:
				\begin{itemize}
					\item 	
						L'argomento non è stato inserito;
					\item
						Le parole chiave non sono state inserite;
					\item
						Non tutti i campi obbligatori sono stati inseriti; 
					\item
						Non è stata indicata la risposta corretta.	
				\end{itemize}
	\end{itemize}
	
		\subsubsection{Caso d'uso UC8.2.1: Selezione domande da modificare}
		\label{UC8.2.1}
		\begin{figure}[h]
			\centering
			\includegraphics[scale=0.5,keepaspectratio]{UML/UC8_2_1.png}
			\caption{Caso d'uso UC8.2.1: Selezione domande da modificare}
		\end{figure}
		\FloatBarrier
		\begin{itemize}
			\item \textbf{Attori}: utente autenticato, utente autenticato pro;
			\item \textbf{Descrizione}: l'attore può selezionare la domanda da modificare;
			\item \textbf{Precondizione}: l'attore ha selezionato la funzionalità di modifica domanda;
			\item \textbf{Postcondizione}: l'attore ha selezionato la domanda da modificare; 
			\item \textbf{Scenario principale}: 
					\begin{enumerate}
					\item
					L'attore può richiamare il wizard per modificare una domanda vero/falso (UC8.2.1.1);
					\item
					L'attore può richiamare il wizard per modificare una domanda a risposta multipla (UC8.2.1.2);
					\item
					L'attore può richiamare il wizard per modificare un esercizio a riempimento di spazi vuoti (UC8.2.1.3);
					\item
					L'attore può richiamare il wizard per modificare una domanda di collegamento (UC8.2.1.4);
					\item
					L'attore può richiamare il wizard per modificare una domanda a ordinamento di immagini (UC8.2.1.5);
					\item
					L'attore può richiamare il wizard per modificare una domanda a ordinamento di stringhe (UC8.2.1.6);
					\item
					L'attore può richiamare il wizard per modificare una domanda con area cliccabile nell'immagine (UC8.2.1.7).
	 			\end{enumerate}
			
		\end{itemize}

	%inclusione file latex wizard
\subsubsection{Caso d'uso UC8.2.1.1: Modifica domanda vero/falso}
	\label{UC8.2.1.1}
	\begin{figure}[h]
		\centering
			\includegraphics[scale=0.45,keepaspectratio]{UML/UC8_2_1_1.png}
		\caption{UC8.2.1.1: Modifica domanda vero/falso}
	\end{figure}
	\FloatBarrier
	\begin{itemize}
		\item
			\textbf{Attori}: utente autenticato, utente autenticato pro;
		\item		
			\textbf{Descrizione}: l'attore può utilizzare la procedura guidata per la modifica di una domanda vero/falso;
		\item
			\textbf{Precondizione}: l'attore ha selezionato la funzionalità di modifica di una domanda \\vero/falso; 
		\item
			\textbf{Postcondizione}: l'attore ha modificato una domanda vero/falso;
		\item
			\textbf{Scenario principale}:
	       		\begin{enumerate}
	       			\item
	       			L'attore può modificare il testo della domanda (UC8.2.1.1.1);
	       			\item
	       			L'attore può modificare l'immagine relativa al testo della domanda (UC8.2.1.1.2);
					\item
					L'attore può modificare la risposta corretta (UC8.2.1.1.3).
	 			\end{enumerate}
	\end{itemize}
	
\subsubsection{Caso d'uso UC8.2.1.1.1: Modifica testo della domanda}
	\begin{itemize}
		\item
			\textbf{Attori}: utente autenticato, utente autenticato pro;
		\item		
			\textbf{Descrizione}: l'attore può modificare il testo della domanda;
		\item
			\textbf{Precondizione}: l'attore ha selezionato la funzionalità di modifica di una domanda \\vero/falso; 
		\item
			\textbf{Postcondizione}: l'attore ha modificato il testo della domanda;
		\item
			\textbf{Scenario principale}: l'attore modifica il testo della domanda. 
	\end{itemize}
	
\subsubsection{Caso d'uso UC8.2.1.1.2: Modifica immagine}
	\begin{itemize}
		\item
			\textbf{Attori}: utente autenticato, utente autenticato pro;
		\item		
			\textbf{Descrizione}: l'attore può modificare l'immagine relativa al testo della domanda;
		\item
			\textbf{Precondizione}: l'attore ha selezionato la funzionalità di modifica di una domanda \\vero/falso; 
		\item
			\textbf{Postcondizione}: l'attore ha modificato l'immagine relativa al testo della domanda;
		\item
			\textbf{Scenario principale}: l'attore modifica l'immagine relativa al testo della domanda. 	
	\end{itemize}
	
\subsubsection{Caso d'uso UC8.2.1.1.3: Modifica risposta corretta}
	\begin{itemize}
		\item
			\textbf{Attori}: utente autenticato, utente autenticato pro;
		\item		
			\textbf{Descrizione}: l'attore può modificare la selezione della risposta corretta;
		\item
			\textbf{Precondizione}: l'attore ha selezionato la funzionalità di modifica di una domanda \\vero/falso; 
		\item
			\textbf{Postcondizione}: l'attore ha modificato la selezione della risposta corretta;
		\item
			\textbf{Scenario principale}: l'attore modifica la selezione della risposta corretta.	 			
	\end{itemize}
	

\input{sezioni/CasiD'uso/wizardUC8/CasoD'usoUC8_2_1_2.tex}
\input{sezioni/CasiD'uso/wizardUC8/CasoD'usoUC8_2_1_3.tex}
\input{sezioni/CasiD'uso/wizardUC8/CasoD'usoUC8_2_1_4.tex}
\input{sezioni/CasiD'uso/wizardUC8/CasoD'usoUC8_2_1_5.tex}
\subsubsection{Caso d’uso UC8.2.1.6: Modifica domanda a ordinamento di stringhe}
	\label{UC8.2.1.6}
	\begin{figure}[h]
		\centering
		\includegraphics[scale=0.45,keepaspectratio]{UML/UC8_2_1_6.png}
		\caption{UC8.2.1.6: Modifica domanda a ordinamento di stringhe}
	\end{figure}
	\FloatBarrier
\begin{itemize}
	\item\textbf{Attori}: utente autenticato, utente autenticato pro;
	\item\textbf{Descrizione}: l'attore può utilizzare la procedura guidata per la modifica di una domanda a ordinamento di stringhe;
	\item\textbf{Precondizione}: l'attore ha selezionato la funzionalità di modifica di una domanda a ordinamento di stringhe;
	\item \textbf{Postcondizione}: l'attore ha modificato una domanda a ordinamento di stringhe;
	\item\textbf{Scenario principale}:
		\begin{itemize}
			\item L'attore può modificare il testo della domanda (UC8.2.1.6.1);
			\item L'attore può modificare il testo e il numero delle stringhe che compongono la sequenza della domanda (UC8.2.1.6.2);
			\item L'attore può modificare l'ordine corretto delle stringhe che costituiscono la risposta (UC8.2.1.6.3).
		\end{itemize}
\end{itemize}

\subsubsection{Caso d'uso UC8.2.1.6.1: Modifica testo della domanda}
\begin{itemize}
	\item \textbf{Attori}: utente autenticato, utente autenticato pro;
	\item \textbf{Descrizione}: l'attore può modificare il testo della domanda;
	\item \textbf{Precondizione}: l'attore ha selezionato la funzionalità di modifica di una domanda a ordinamento di stringhe;
	\item \textbf{Postcondizione}: l'attore ha modificato il testo della domanda;
	\item \textbf{Scenario principale}: l'attore modifica il testo della domanda.
\end{itemize}

\subsubsection{Caso d'uso UC8.2.1.6.2: Modifica testo e numero di stringhe}
\begin{itemize}
	\item \textbf{Attori}: utente autenticato, utente autenticato pro;
	\item \textbf{Descrizione}: l'attore può modificare il testo e il numero di stringhe che costituiscono la risposta alla domanda;
	\item \textbf{Precondizione}: l'attore ha selezionato la funzionalità di modifica di una domanda a ordinamento di stringhe;
	\item \textbf{Postcondizione}: l'attore ha modificato il testo e il numero delle stringhe che costituiscono la risposta alla domanda;
	\item \textbf{Scenario principale}: l'attore modifica il testo e il numero delle stringhe che costituiscono la risposta alla domanda.
\end{itemize}

\subsubsection{Caso d'uso UC8.2.1.6.3: Modifica ordine stringhe come risposta}
\begin{itemize}
	\item \textbf{Attori}: utente autenticato, utente autenticato pro;
	\item \textbf{Descrizione}: l'attore può modificare l'ordine corretto delle stringhe che rappresenta la soluzione della domanda;
	\item \textbf{Precondizione}: l'attore ha selezionato la funzionalità di modifica di una domanda a ordinamento di stringhe;
	\item \textbf{Postcondizione}: l'attore ha modificato l'ordine delle stringhe che costituiscono la risposta alla domanda;
	\item \textbf{Scenario principale}: l'attore modifica l'ordine delle stringhe che costituiscono la risposta alla domanda.
\end{itemize}

\subsubsection{Caso d'uso UC8.2.1.7: Modifica domanda con area cliccabile nell'immagine}
\label{UC8.2.1.7}
\begin{figure}[h]
	\centering
	\includegraphics[scale=0.5,keepaspectratio]{UML/UC8_2_1_7.png}
	\caption{UC8.2.1.7: Modifica domanda con area cliccabile nell'immagine}
\end{figure}
\FloatBarrier
\begin{itemize}
	\item \textbf{Attori}: utente autenticato, utente autenticato pro;
	\item \textbf{Descrizione}: l'attore può utilizzare la procedura guidata per la modifica di una domanda la cui risposta è selezionabile all'interno di aree cliccabili in un'immagine;
	\item \textbf{Precondizione}: l'attore ha selezionato la modalità di modifica di una domanda con area cliccabile nell'immagine; 
	\item \textbf{Postcondizione}: l'attore ha modificato una domanda con area cliccabile nell'immagine;
	\item \textbf{Scenario principale}:
		\begin{enumerate}
	       	\item L'attore può modificare il testo della domanda (UC8.2.1.7.1);
	        \item L'attore può modificare l'immagine relativa al testo della domanda (UC8.1.1.7.2);
			\item L'attore può scegliere un nuovo numero di aree che saranno selezionabili all'interno dell'immagine (UC8.2.1.7.3);
			\item L'attore può scegliere nuove aree selezionabili all'interno dell'immagine (UC8.2.1.7.4);
	 	\end{enumerate}
\end{itemize}

\subsubsection{Caso d'uso UC8.2.1.7.1: Modifica testo della domanda}
\begin{itemize}
	\item \textbf{Attori}: utente autenticato, utente autenticato pro;
	\item \textbf{Descrizione}: l'attore può modificare il testo della domanda;
	\item \textbf{Precondizione}: l'attore ha selezionato la modalità di modifica di una domanda con area cliccabile nell'immagine; 
	\item \textbf{Postcondizione}: l'attore ha modificato il testo della domanda;
	\item \textbf{Scenario principale}: l'attore modifica il testo della domanda. 
\end{itemize}

\subsubsection{Caso d'uso UC8.2.1.7.2: Inserimento nuova immagine}
\begin{itemize}
	\item \textbf{Attori}: utente autenticato, utente autenticato pro;
	\item \textbf{Descrizione}: l'attore può inserire una nuova immagine relativa al testo della domanda che sostituisce quella già presente;
	\item \textbf{Precondizione}: l'attore ha selezionato la modalità di modifica di una domanda con area cliccabile nell'immagine;  
	\item \textbf{Postcondizione}: l'attore ha inserito una nuova immagine;
	\item \textbf{Scenario principale}: l'attore inserisce una nuova immagine al posto di quella che era presente. 	
\end{itemize}

\subsubsection{Caso d'uso UC8.2.1.7.3: Modifica numero aree selezionabili}
\begin{itemize}
	\item \textbf{Attori}: utente autenticato, utente autenticato pro;
	\item \textbf{Descrizione}: l'attore può scegliere un nuovo numero di aree selezionabili all'interno dell'immagine;
	\item \textbf{Precondizione}: l'attore ha selezionato la modalità di modifica di una domanda con area cliccabile nell'immagine; 
	\item \textbf{Postcondizione}: l'attore ha scelto il nuovo numero di aree selezionabili all'interno dell'immagine;
	\item \textbf{Scenario principale}: l'attore sceglie il nuovo numero di aree selezionabili all'interno dell'immagine. 	
\end{itemize}

\subsubsection{Caso d'uso UC8.2.1.7.4: Modifica aree selezionabili}
\begin{itemize}
	\item \textbf{Attori}: utente autenticato, utente autenticato pro;
	\item \textbf{Descrizione}: l'attore può scegliere dove inserire le nuove aree selezionabili o riposizionare quelle già presenti all'interno dell'immagine;
	\item \textbf{Precondizione}: l'attore ha selezionato la modalità di modifica di una domanda con area cliccabile nell'immagine; 
	\item \textbf{Postcondizione}: l'attore ha scelto dove inserire le nuove aree selezionabili o dove riposizionare quelle già presenti all'interno dell'immagine;
	\item \textbf{Scenario principale}: l'attore sceglie dove inserire le nuove aree selezionabili o riposiziona quelle già presenti all'interno dell'immagine. 	
\end{itemize}



	\subsubsection{Caso d'uso UC8.2.2: Conferma modifica}
	\begin{itemize}
		\item
			\textbf{Attori}: utente autenticato, utente autenticato pro;
		\item
			\textbf{Descrizione}: l'attore può confermare la modifica della domanda;
		\item		
			\textbf{Precondizione}: l'attore ha finito di modificare una domanda tramite un wizard;
		\item
			\textbf{Postcondizione}: l'attore ha modificato una domanda;
		\item
			\textbf{Scenario principale}: l'attore conferma la modifica della domanda;		
		\item
	 		\textbf{Scenari alternativi}: l'attore annulla la modifica della domanda.
	\end{itemize}		
	\subsubsection{Caso d'uso UC8.2.3: Visualizzazione errore modifica}
	\begin{itemize}
		\item
			\textbf{Attori}: utente autenticato, utente autenticato pro;
		\item
			\textbf{Descrizione}: l'attore può visualizzare un messaggio d'errore nel caso si fossero verificati uno o più scenari alternativi durante la modifica della domanda;
		\item		
			\textbf{Precondizione}: il sistema ha ricevuto dei dati errati per la modifica della domanda;
		\item
			\textbf{Postcondizione}: il sistema mostra un messaggio d'errore;
		\item
			\textbf{Scenario principale}: l'attore visualizza un messaggio d'errore;	
	\end{itemize}	