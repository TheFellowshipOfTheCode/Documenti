\subsection{Caso d'uso UC8: Gestione delle domande}
	\label{UC8}
	\begin{figure}[h]
		\centering
			\includegraphics[scale=0.45,keepaspectratio]{UML/UC8.png}
		\caption{UC8: Gestione delle domande}
	\end{figure}
	\FloatBarrier
	\begin{itemize}
		\item
			\textbf{Attori}: utente autenticato, utente autenticato pro;
		\item		
			\textbf{Descrizione}: lo scopo di questa funzionalità è offrire agli attori la possibilità di creare e modificare domande;
		\item
			\textbf{Precondizione}: gli attori sono autenticati presso il sistema; 
		\item
			\textbf{Postcondizione}: gli attori hanno compiuto una delle operazioni appartenenti a questa funzionalità;
		\item
			\textbf{Scenario principale}:
	       		\begin{enumerate}
					\item
					Gli attori creano una nuova domanda [UC8.1];
					\item
					Gli attori modificano una domanda [UC8.2].
	 			\end{enumerate}
	\end{itemize}
\subsubsection{Caso d'uso UC8.1: Creazione nuova domanda}
	\label{UC8.1}
	\begin{figure}[h]
		\centering
			\includegraphics[scale=0.45,keepaspectratio]{UML/UC8_1.png}
		\caption{UC8.1: Creazione nuova domanda}
	\end{figure}
	\FloatBarrier
	\begin{itemize}
		\item
			\textbf{Attori}: utente autenticato, utente autenticato pro;
		\item		
			\textbf{Scopo e descrizione}: gli attori possono creare una nuova domanda, utile all'apprendimento di un determinato argomento;
		\item
			\textbf{Precondizione}: gli attori hanno selezionato l'opzione per la creazione della domanda;
		\item
			\textbf{	Postcondizione}: gli attori ricevono conferma dell'avvenuta creazione;		
		\item
			\textbf{Scenario principale}:
	       		\begin{enumerate}
					\item
					L'utente inserisce l'argomento relativo alla nuova domanda[UC8.1.1];
					\item
					L'utente inserisce le parole chiave relative alla nuova domanda [UC8.1.2];
					\item
					L'utente inserisce il testo della domanda [UC8.1.3];
					\item
					L'utente deve selezionare la modalità di risposta ed eventualmente inserire le opzioni di risposta [UC8.1.4];
					\item
					L'utente deve indicare la risposta corretta [UC8.1.5];
					\item
					L'utente conferma la creazione della domanda [UC8.1.6].
	 			\end{enumerate}
	 	\item
			\textbf{Estensione}: L'utente visualizza un messaggio d'errore di conferma [UC8.1.8];
	 	\item
	 		\textbf{Scenario alternativo}: possono verificarsi uno o più di questi scenari:
				\begin{itemize}
					\item[-] 	
						L'argomento non è stato inserito;
					\item[-] 
						Le parole chiave non sono state inserite;
					\item[-] 
						La modalità di risposta non è stata selezionata o non è presente alcuna opzione di risposta; 
					\item[-]
						La risposta corretta non è stata indicata.	
				\end{itemize}
			In tal caso il sistema ripresenta il modulo di creazione della domanda, presentando un messaggio d'errore.
	\end{itemize}
	\subsubsection{Caso d'uso UC8.1.1: Selezione argomento}
	\begin{itemize}
		\item
			\textbf{Attori}: utente autenticato, utente autenticato pro;
		\item
			\textbf{Scopo e descrizione}: gli attori, tramite uno strumento di selezione offerto dal sistema, possono indicare un argomento tra quelli presenti;
		\item		
			\textbf{Precondizione}: il sistema presenta all'utente autenticato uno strumento di selezione per assegnare alla nuova domanda un argomento;
		\item
			\textbf{Postcondizione}: l'argomento è stato selezionato.
		\item
			\textbf{Scenario principale}:
				\begin{enumerate}
					\item 	
						Gli attori selezionano l'argomento da assegnare alla nuova domanda	
				\end{enumerate}
	\end{itemize}	
	\subsubsection{Caso d'uso UC8.1.2: Inserimento parole chiave}
	\begin{itemize}
		\item
			\textbf{Attori}: utente autenticato, utente autenticato pro;
		\item
			\textbf{Scopo e descrizione}: gli attori possono inserire le parole chiave relative alla nuova domanda per facilitare la ricerca di una domanda;
		\item		
			\textbf{Precondizione}: il sistema presenta all'utente autenticato il campo dati per l'inserimento delle parole chiave;
		\item
			\textbf{Postcondizione}: l'inserimento delle parole chiave è avvenuto;
		\item
			\textbf{Scenario principale}:
				\begin{enumerate}
					\item 	
						Gli attori inseriscono le parole chiave relative alla nuova domanda	
				\end{enumerate}
	\end{itemize}


	\subsubsection{Caso d'uso UC8.1.3: Seleziona tipologia di domanda}
	\begin{itemize}
		\item
			\textbf{Attori}: utente autenticato, utente autenticato pro;
		\item
			\textbf{Scopo e descrizione}: gli attori possono scegliere la tipologia di domanda che vogliono inserire;
		\item		
			\textbf{Precondizione}: il sistema mostra all'utente le tipologie di domande possibili;
		\item
			\textbf{Postcondizione}: l'utente ha selezionato la tipologia di domanda da inserire;
		\item
			\textbf{Scenario principale}:
				\begin{enumerate}
					\item 	
						Gli attori scelgono la tipologia di domanda che vogliono inserire nel sistema.	
				\end{enumerate}
	\end{itemize}

	%inclusione file latex wizard
\subsubsection{Caso d'uso UC8.1.3.1: Creazione domanda vero/falso}
	\label{UC8.1.3.1}
	\begin{figure}[h]
		\centering
			\includegraphics[scale=0.45,keepaspectratio]{UML/UC8_1_3_1.png}
		\caption{UC8.1.3.1: Creazione domanda vero/falso}
	\end{figure}
	\FloatBarrier
	\begin{itemize}
		\item
			\textbf{Attori}: utente autenticato, utente autenticato pro;
		\item		
			\textbf{Descrizione}: l'attore può utilizzare la procedura guidata per la creazione di una domanda vero/falso;
		\item
			\textbf{Precondizione}: l'attore ha selezionato la funzionalità di creazione di una domanda vero/falso; 
		\item
			\textbf{Postcondizione}: l'attore ha creato una domanda vero/falso;
		\item
			\textbf{Scenario principale}:
	       		\begin{enumerate}
	       			\item
	       			L'attore può inserire il testo della domanda (UC8.1.3.1.1);
	       			\item
	       			L'attore può inserire un'immagine relativa al testo della domanda (UC8.1.3.1.2);
					\item
					L'attore può indicare la risposta corretta tramite uno strumento di selezione (UC8.1.3.1.3).
	 			\end{enumerate}
	\end{itemize}

\subsubsection{Caso d'uso UC8.1.3.1.1: Inserimento testo della domanda}
	\begin{itemize}
		\item
			\textbf{Attori}: utente autenticato, utente autenticato pro;
		\item		
			\textbf{Descrizione}: l'attore può inserire il testo della domanda;
		\item
			\textbf{Precondizione}: l'attore ha selezionato la funzionalità di creazione di una domanda vero/falso; 
		\item
			\textbf{Postcondizione}: l'attore ha inserito il testo della domanda;
		\item
			\textbf{Scenario principale}: l'attore inserisce il testo della domanda. 
	 			
	\end{itemize}
	
\subsubsection{Caso d'uso UC8.1.3.1.2: Inserimento immagine}
	\begin{itemize}
		\item
			\textbf{Attori}: utente autenticato, utente autenticato pro;
		\item		
			\textbf{Descrizione}: l'attore può inserire un'immagine relativa al testo della domanda;
		\item
			\textbf{Precondizione}: l'attore ha selezionato la funzionalità di creazione di una domanda vero/falso; 
		\item
			\textbf{Postcondizione}: l'attore ha inserito un'immagine relativa al testo della domanda;
		\item
			\textbf{Scenario principale}: l'attore inserisce un'immagine.						
	\end{itemize}
	

\subsubsection{Caso d'uso UC8.1.3.1.3: Selezione risposta corretta}
	\begin{itemize}
		\item
			\textbf{Attori}: utente autenticato, utente autenticato pro;
		\item		
			\textbf{Descrizione}: l'attore può indicare la risposta corretta;
		\item
			\textbf{Precondizione}: l'attore ha selezionato la funzionalità di creazione di una domanda vero/falso; 
		\item
			\textbf{Postcondizione}: l'attore ha selezionato la risposta corretta;
		\item
			\textbf{Scenario principale}: l'attore indica la risposta corretta.  			
	\end{itemize}
\subsubsection{Caso d'uso UC8.1.3.2: Creazione domanda a risposta multipla}
	\label{UC8.1.3.2}
	\begin{figure}[h]
		\centering
			\includegraphics[scale=0.45,keepaspectratio]{UML/UC8_1_3_2.png}
		\caption{UC8.1.3.2: Creazione domanda a risposta multipla}
	\end{figure}
	\FloatBarrier
	\begin{itemize}
		\item
			\textbf{Attori}: utente autenticato, utente autenticato pro;
		\item		
			\textbf{Descrizione}: l'attore può utilizzare la procedura guidata per la creazione di una domanda a risposta multipla;
		\item
			\textbf{Precondizione}: il sistema presenta all'attore la procedura guidata per la creazione di una domanda a risposta multipla;
		\item
			\textbf{Postcondizione}: l'attore ha creato una domanda a risposta multipla;
		\item
			\textbf{Scenario principale}:
	       		\begin{enumerate}
	       			\item
	       			L'attore può inserire il testo della domanda (UC8.1.3.2.1);
	       			\item
	       			L'attore può inserire un'immagine relativa al testo della domanda (UC8.1.3.2.2);
	       			\item
	       			L'attore può aggiungere almeno due opzioni di risposta (UC8.1.3.2.3);
					\item
					L'attore può indicare una o più risposte corrette (UC8.1.3.2.4).
	 			\end{enumerate}
	\end{itemize}

\subsubsection{Caso d'uso UC8.1.3.2.1: Inserimento testo della domanda}
	\begin{itemize}
		\item
			\textbf{Attori}: utente autenticato, utente autenticato pro;
		\item		
			\textbf{Descrizione}: l'attore può inserire il testo della domanda;
		\item
			\textbf{Precondizione}: il sistema presenta all'attore lo spazio destinato all'inserimento del testo della domanda;
		\item
			\textbf{Postcondizione}: l'attore ha inserito il testo della domanda;
		\item
			\textbf{Scenario principale}: l'attore inserisce il testo della domanda. 
	 			
	\end{itemize}
	
\subsubsection{Caso d'uso UC8.1.3.2.2: Inserimento immagine}
	\begin{itemize}
		\item
			\textbf{Attori}: utente autenticato, utente autenticato pro;
		\item		
			\textbf{Descrizione}: l'attore può inserire un'immagine relativa al testo della domanda;
		\item
			\textbf{Precondizione}: il sistema presenta all'attore la funzionalità di inserimento di un immagine; 
		\item
			\textbf{Postcondizione}: l'attore ha inserito un'immagine relativa al testo della domanda;
		\item
			\textbf{Scenario principale}: l'attore inserisce l'immagine.						
	\end{itemize}


	
\subsubsection{Caso d'uso UC8.1.3.2.3: Aggiunta opzioni di risposta}
	\label{UC8.1.3.2.3}
	\begin{figure}[h]
		\centering
			\includegraphics[scale=0.45,keepaspectratio]{UML/UC8_1_3_2_3.png}
		\caption{UC8.1.3.2.3: Aggiunta opzioni di risposta}
	\end{figure}
	\FloatBarrier
	\begin{itemize}
		\item
			\textbf{Attori}: utente autenticato, utente autenticato pro;
		\item		
			\textbf{Descrizione}: l'attore può aggiungere almeno due opzioni di risposta;
		\item
			\textbf{Precondizione}: il sistema presenta all'attore la funzionalità di aggiungere due o più opzioni di risposta; 
		\item
			\textbf{Postcondizione}: l'attore ha aggiunto due o più opzioni di risposta;
		\item
			\textbf{Scenario principale}:
	       		\begin{enumerate}
	       			\item
	       			L'attore può aggiungere opzioni di risposta che includono testo (UC8.1.3.2.3.1);
					\item
					L'attore può aggiungere opzioni di risposta che includono immagini (UC8.1.3.2.3.2).
	 			\end{enumerate}
	\end{itemize}	

\subsubsection{Caso d'uso UC8.1.3.2.3.1: Aggiunta opzioni di risposta che includono testo}
	\begin{itemize}
		\item
			\textbf{Attori}: utente autenticato, utente autenticato pro;
		\item		
			\textbf{Descrizione}: l'attore può aggiungere almeno due opzioni di risposta che includono testo;
		\item
			\textbf{Precondizione}: il sistema presenta all'attore la funzionalità di aggiungere due o più opzioni di risposta che includono testo;
		\item
			\textbf{Postcondizione}: l'attore ha aggiunto due o più opzioni di risposta che includono testo;
		\item
			\textbf{Scenario principale}: l'attore aggiunge due o più opzioni che includono testo.				
	\end{itemize}	

\subsubsection{Caso d'uso UC8.1.3.2.3.2: Aggiunta opzioni di risposta che includono immagini}
	\begin{itemize}
		\item
			\textbf{Attori}: utente autenticato, utente autenticato pro;
		\item		
			\textbf{Descrizione}: l'attore può aggiungere almeno due opzioni di risposta che includono immagini;
		\item
			\textbf{Precondizione}: il sistema presenta all'attore la funzionalità di aggiungere due o più opzioni di risposta che includono immagini;
		\item
			\textbf{Postcondizione}: l'attore ha aggiunto due o più opzioni di risposta che includono immagini;
		\item
			\textbf{Scenario principale}: l'attore aggiunge due o più opzioni che includono immagini. 				
	\end{itemize}	
		
\subsubsection{Caso d'uso UC8.1.3.2.4: Selezione una o più risposte corrette}
	\begin{itemize}
		\item
			\textbf{Attori}: utente autenticato, utente autenticato pro;
		\item		
			\textbf{Descrizione}: l'attore può indicare una o più risposte corrette;
		\item
			\textbf{Precondizione}: il sistema presenta all'attore la funzionalità di selezionare una o più risposte corrette;
		\item
			\textbf{Postcondizione}: l'attore ha selezionato una o più risposte corrette;
		\item
			\textbf{Scenario principale}: l'attore indica una o più risposte corrette. 			
	\end{itemize}
\subsubsection{Caso d'uso UC8.1.3.3: Creazione esercizi di riempimento degli spazi vuoti}
	\label{UC8.1.3.3}
	\begin{figure}[h]
		\centering
			\includegraphics[scale=0.45,keepaspectratio]{UML/UC8_1_3_3.png}
		\caption{UC8.1.3.3: Creazione esercizi di riempimento degli spazi vuoti}
	\end{figure}
	\FloatBarrier
	\begin{itemize}
		\item
			\textbf{Attori}: utente autenticato, utente autenticato pro;
		\item		
			\textbf{Descrizione}: l'attore può creare esercizi di riempimento degli spazi vuoti;
		\item
			\textbf{Precondizione}: l'attore ha selezionato la seguente funzionalità; 
		\item
			\textbf{Postcondizione}: l'attore ha creato un esercizio di riempimento degli spazi vuoti;
		\item
			\textbf{Scenario principale}:
	       		\begin{enumerate}
	       			\item
	       			L'attore può compilare il campo dati destinato alla scrittura del testo dell'esercizio (UC8.1.3.3.1);
	       			\item
	       			L'attore può indicare le parole che saranno sostituite con degli spazi vuoti dal sistema (UC8.1.3.3.2).
	 			\end{enumerate}
	\end{itemize}
	
\subsubsection{Caso d'uso UC8.1.3.3.1: Scrittura testo dell'esercizio}
	\begin{itemize}
		\item
			\textbf{Attori}: utente autenticato, utente autenticato pro;
		\item		
			\textbf{Descrizione}: l'attore può inserire il testo dell'esercizio di riempimento;
		\item
			\textbf{Precondizione}: l'attore può creare un esercizio di riempimento degli spazi vuoti; 
		\item
			\textbf{Postcondizione}: l'attore ha compilato il campo dati dedicato alla scrittura del testo dell'esercizio di riempimento;
		\item
			\textbf{Scenario principale}: l'attore compila il campo dati dedicato alla scrittura del testo dell'esercizio di riempimento.
	\end{itemize}


\subsubsection{Caso d'uso UC8.1.3.3.2: Indicazione parole da oscurare}
	\begin{itemize}
		\item
			\textbf{Attori}: utente autenticato, utente autenticato pro;
		\item		
			\textbf{Descrizione}: l'attore può indicare le parole che saranno sostituite con degli spazi vuoti;
		\item
			\textbf{Precondizione}: l'attore ha inserito il testo dell'esercizio; 
		\item
			\textbf{Postcondizione}: l'attore ha indicato le parole che saranno sostituite con degli spazi vuoti;
		\item
			\textbf{Scenario principale}: l'attore indica le parole che verranno oscurate dal sistema.
	\end{itemize}
\subsubsection{Caso d'uso UC8.1.3.4: Creazione domanda di collegamento}
\label{UC8.1.3.4}
\begin{figure}[h]
	\centering
\includegraphics[scale=0.5,keepaspectratio]{UML/UC8_1_3_4.png}
	\caption{Caso d'uso UC8.1.3.4: Creazione domanda di collegamento}
\end{figure}
\FloatBarrier
\begin{itemize}
	\item \textbf{Attori}: \uau, \uaupro;
	\item \textbf{Descrizione}: l'attore può utilizzare la procedura guidata per la creazione di una domanda di collegamento; 
	\item \textbf{Precondizione}: il sistema presenta all'attore la procedura guidata per la creazione di una domanda di collegamento;
	\item \textbf{Postcondizione}: l'attore ha creato una domanda di collegamento;
	\item \textbf{Scenario principale}: 
		\begin{enumerate}
			\item L'attore può inserire il testo della domanda (UC8.1.3.4.1);
			\item L'attore può inserire una coppia di elementi (UC8.1.3.4.2);
			\item L'attore può eliminare una coppia di elementi inserita (UC8.1.3.4.3);
			\item L'attore può modificare una coppia di elementi inserita (UC8.1.3.4.4).
		\end{enumerate}
	\item \textbf{Scenari alternativi}: se non ci sono almeno due coppie presenti nella lista delle coppie l'attore deve inserire una nuova coppia di elementi.
\end{itemize}

	\subsubsection{Caso d'uso UC8.1.3.4.1: Inserimento testo della domanda}
	\label{UC8.1.3.4.1}
	\begin{itemize}
		\item
		\textbf{Attori}: \uau, \uaupro;
		\item		
		\textbf{Descrizione}: l'attore può inserire il testo della domanda;
		\item
		\textbf{Precondizione}: il sistema presenta all'attore lo spazio destinato all'inserimento del testo della domanda;
		\item \textbf{Postcondizione}: l'attore ha inserito il testo della domanda;
		\item \textbf{Scenario principale}: l'attore inserisce il testo della domanda. 
	\end{itemize}

	\subsubsection{Caso d'uso UC8.1.3.4.2: Inserimento coppia di elementi}
	\label{UC8.1.3.4.2}
	\begin{figure}[h]
		\centering
		\includegraphics[scale=0.5,keepaspectratio]{UML/UC8_1_3_4_2.png}
		\caption{Caso d'uso UC8.1.3.4.2: Inserimento coppia di elementi}
	\end{figure}
	\FloatBarrier
	\begin{itemize}
		\item \textbf{Attori}: \uau, \uaupro;
		\item \textbf{Descrizione}: l'attore può inserire una coppia di elementi, sia immagini che testo o combinazioni di questi, che siano correlati tra loro in modo da indicare la soluzione della domanda; 
		\item \textbf{Precondizione}: il sistema presenta all'attore la funzionalità di inserimento di una o più coppie di elementi;
		\item \textbf{Postcondizione}: l'attore ha inserito una coppia di elementi nella lista di coppie di elementi; 
		\item \textbf{Scenario principale}: 
		\begin{enumerate}
			\item L'attore inserisce come primo elemento un'immagine (UC8.1.3.4.2.1);
			\item L'attore inserisce come primo elemento un testo (UC8.1.3.4.2.2);
			\item L'attore inserisce come secondo elemento un'immagine (UC8.1.3.4.2.3);
			\item L'attore inserisce come secondo elemento un testo (UC8.1.3.4.2.4).	
		\end{enumerate}
	\end{itemize}
	
		\subsubsection{Caso d'uso UC8.1.3.4.2.1: Inserimento di un'immagine come primo elemento}
		\label{UC8.1.3.4.2.1}
		\begin{itemize}
			\item \textbf{Attori}: \uau, \uaupro;
			\item \textbf{Descrizione}: l'attore può inserire come primo elemento della coppia un'immagine;
			\item \textbf{Precondizione}: il sistema presenta all'attore lo spazio destinato all'inserimento di un'immagine come primo elemento;
			\item \textbf{Postcondizione}: l'attore ha inserito come primo elemento un'immagine;
			\item \textbf{Scenario principale}: l'attore carica un'immagine come primo elemento della coppia.
		\end{itemize}
		
		\subsubsection{Caso d'uso UC8.1.3.4.2.2: Inserimento di un testo come primo elemento}
		\label{UC8.1.3.4.2.2}
		\begin{itemize}
			\item \textbf{Attori}: \uau, \uaupro;
			\item \textbf{Descrizione}: l'attore può inserire come primo elemento della coppia un testo;
			\item \textbf{Precondizione}: il sistema presenta all'attore lo spazio destinato all'inserimento di un testo come primo elemento;
			\item \textbf{Postcondizione}: l'attore ha inserito come primo elemento un testo;
			\item \textbf{Scenario principale}: l'attore inserisce del testo come primo elemento della coppia.
		\end{itemize}
		
			\subsubsection{Caso d'uso UC8.1.3.4.2.3: Inserimento di un'immagine come secondo elemento}
		\label{UC8.1.3.4.2.3}
		\begin{itemize}
			\item \textbf{Attori}: \uau, \uaupro;
			\item \textbf{Descrizione}: l'attore può inserire come secondo elemento della coppia un'immagine;
			\item \textbf{Precondizione}: il sistema presenta all'attore lo spazio destinato all'inserimento di un'immagine come secondo elemento;
			\item \textbf{Postcondizione}: l'attore ha inserito come secondo elemento un'immagine;
			\item \textbf{Scenario principale}: l'attore carica un'immagine come secondo elemento della coppia.
		\end{itemize}
		
		\subsubsection{Caso d'uso UC8.1.3.4.2.4: Inserimento di un testo come secondo elemento}
		\label{UC8.1.3.4.2.4}
		\begin{itemize}
			\item \textbf{Attori}: \uau, \uaupro;
			\item \textbf{Descrizione}: l'attore può inserire come secondo elemento della coppia un testo;
			\item \textbf{Precondizione}: il sistema presenta all'attore lo spazio destinato all'inserimento di un testo come secondo elemento;
			\item \textbf{Postcondizione}: l'attore ha inserito come secondo elemento un testo;
			\item \textbf{Scenario principale}: l'attore inserisce del testo come secondo elemento della coppia.
		\end{itemize}
	
	\subsubsection{Caso d'uso UC8.1.3.4.3: Eliminazione coppia di elementi}
	\label{UC8.1.3.4.3}
	\begin{figure}[h]
		\centering
		\includegraphics[scale=0.5,keepaspectratio]{UML/UC8_1_3_4_3.png}
		\caption{Caso d'uso UC8.1.3.4.3: Eliminazione coppia di elementi}
	\end{figure}
	\FloatBarrier
	\begin{itemize}
		\item \textbf{Attori}: \uau, \uaupro;
		\item \textbf{Descrizione}: l'attore può rimuovere una coppia di elementi dalla lista di coppie di elementi;
		\item \textbf{Precondizione}: il sistema presenta all'attore la funzionalità di eliminare una coppia di elementi;
		\item \textbf{Postcondizione}: l'attore ha eliminato una coppia di elementi dalla lista delle coppie di elementi;
		\item \textbf{Scenario principale}: l'attore può confermare l'eliminazione di una coppia di elementi (UC8.1.3.4.3.1);	
		\item \textbf{Scenari alternativi}: l'attore annulla l'operazione tornando alla schermata precedente.
	\end{itemize}

		\subsubsection{Caso d'uso UC8.1.3.4.3.1: Conferma eliminazione coppia di elementi}
		\label{UC8.1.3.4.3.1}
		\begin{itemize}
			\item \textbf{Attori}: \uau, \uaupro;
			\item \textbf{Descrizione}: l'attore può confermare la rimozione di una coppia di elementi;
			\item \textbf{Precondizione}: il sistema presenta all'attore la funzionalità di confermare l'eliminazione di una coppia di elementi;
			\item \textbf{Postcondizione}: l'attore ha confermato l'eliminazione della coppia di elementi;
			\item \textbf{Scenario principale}: l'attore conferma la rimozione della coppia di elementi.
		\end{itemize}

	\subsubsection{Caso d'uso UC8.1.3.4.4: Modifica coppia di elementi}
	\label{UC8.1.3.4.4}
	\begin{figure}[h]
		\centering
		\includegraphics[scale=0.5,keepaspectratio]{UML/UC8_1_3_4_4.png}
		\caption{Caso d'uso UC8.1.3.4.4: Modifica coppia di elementi}
	\end{figure}
	\FloatBarrier
	\begin{itemize}
		\item \textbf{Attori}: \uau, \uaupro;
		\item \textbf{Descrizione}: l'attore può modificare una coppia di elementi presente nella lista di coppie di elementi;
		\item \textbf{Precondizione}: il sistema presenta all'attore la funzionalità di modificare una coppia di elementi;
		\item \textbf{Postcondizione}: l'attore ha modificato una coppia di elementi presente nella lista di coppie di elementi; 
		\item \textbf{Scenario principale}: 
		\begin{enumerate}
			\item L'attore può modificare un elemento cambiandone il testo (UC8.1.3.4.4.1);
			\item L'attore può modificare un elemento cambiandone l'immagine (UC8.1.3.4.4.2);
			\item L'attore può modificare un elemento facendolo passare da testo ad immagine \\(UC8.1.3.4.4.3);
			\item L'attore può modificare un elemento facendolo passare da immagine a testo \\(UC8.1.3.4.4.4).	
		\end{enumerate}
	\end{itemize}
	
		\subsubsection{Caso d'uso UC8.1.3.4.4.1: Modifica testo di un elemento}
		\label{UC8.1.3.4.4.1}
		\begin{itemize}
			\item \textbf{Attori}: \uau, \uaupro;
			\item \textbf{Descrizione}: l'attore può modificare il testo di un elemento;
			\item \textbf{Precondizione}: il sistema presenta all'attore la funzionalità di modificare il testo di un elemento;
			\item \textbf{Postcondizione}: l'attore ha modificato il testo di un elemento;
			\item \textbf{Scenario principale}: l'attore modifica il testo di un elemento.  
		\end{itemize}
		
		\subsubsection{Caso d'uso UC8.1.3.4.4.2: Modifica immagine di un elemento}
		\label{UC8.1.3.4.4.2}
		\begin{itemize}
			\item \textbf{Attori}: \uau, \uaupro;
			\item \textbf{Descrizione}: l'attore può caricare un'altra immagine per un elemento;
			\item \textbf{Precondizione}: il sistema presenta all'attore la funzionalità di modificare l'immagine di un elemento; 
			\item \textbf{Postcondizione}: l'attore ha inserito un'altra immagine per un elemento;
			\item \textbf{Scenario principale}: l'attore carica un'altra immagine per un elemento.
		\end{itemize}
		
		\subsubsection{Caso d'uso UC8.1.3.4.4.3: Cambia testo in immagine}
		\label{UC8.1.3.4.4.3}
		\begin{itemize}
			\item \textbf{Attori}: \uau, \uaupro;
			\item \textbf{Descrizione}: l'attore può modificare un elemento facendolo diventare un'immagine al posto di un testo;
			\item \textbf{Precondizione}: il sistema presenta all'attore la funzionalità di modificare un elemento facendolo diventare un'immagine al posto di un testo;
			\item \textbf{Postcondizione}: l'attore ha fatto diventare un'immagine un elemento che prima era un testo;
			\item \textbf{Scenario principale}: l'attore inserisce un'immagine come modifica dell'elemento.  
		\end{itemize}
		
		\subsubsection{Caso d'uso UC8.1.3.4.4.4: Cambia immagine in testo}
		\label{UC8.1.3.4.4.4}
		\begin{itemize}
			\item \textbf{Attori}: \uau, \uaupro;
			\item \textbf{Descrizione}: l'attore può modificare un elemento facendolo diventare un testo al posto di un'immagine;
			\item \textbf{Precondizione}: il sistema presenta all'attore la funzionalità di modificare un elemento facendolo diventare un testo al posto di un'immagine;
			\item \textbf{Postcondizione}: l'attore ha fatto diventare un testo un elemento che prima era un'immagine;
			\item \textbf{Scenario principale}: l'attore inserisce del testo come modifica dell'elemento.  
		\end{itemize}
		

\subsubsection{Caso d’uso UC8.1.3.5: Creazione domanda a ordinamento di immagini}
\label{UC8.1.3.5}
	\begin{figure}[h]
		\centering
			\includegraphics[scale=0.45,keepaspectratio]{UML/UC8_1_3_5.png}
		\caption{UC8.1.3.5: Creazione domanda a ordinamento immagini}
	\end{figure}
	\FloatBarrier
\begin{itemize}
	\item\textbf{Attori}: utente autenticato, utente autenticato pro;
	\item\textbf{Scopo e descrizione}: l'attore  possono inserire una domanda del tipo ordinamento di immagini;
	\item\textbf{Precondizione}: il sistema mostra all'attore  il form di inserimento dei campi dati per la tipologia di domanda scelta; 
	\item \textbf{Postcondizione}: l'attore ha inserito tutti i campi dati obbligatori;
	\item\textbf{Scenario principale}:
	\begin{itemize}
		\item Gli attori devono compilare il campo dati destinato alla scrittura del testo della domanda (UC8.1.3.5.1);
		\item Gli attoriinserisce un'immagine per il testo della domanda (UC8.1.3.5.2);
		\item Gli attori devono inserire le immagini nel giusto ordine (UC8.1.3.5.3).
	\end{itemize}
\end{itemize}

\subsubsection{Caso d’uso UC8.1.3.5.1: Inserimento testo domanda}
\begin{itemize}
	\item\textbf{Attori}: utente autenticato, utente autenticato pro;
	\item\textbf{Scopo e descrizione}: lo scopo di questa funzionalità è offrire all'attore  la possibilità di inserire il testo della domanda;
	\item\textbf{Precondizione}: l'attore ha scelto la modalità di creazione della domanda a ordinamento di immagini; 
	\item \textbf{Postcondizione}: l'attore ha inserito il testo della domanda;
	\item\textbf{Scenario principale}: l'attore  devono inserire il testo della domanda. 
\end{itemize}

\subsubsection{Caso d’uso UC8.1.3.5.2: Inserimento immagine per testo domanda}
\label{UC8.1.3.5.2}
\begin{figure}[h]
	\centering
	\includegraphics[scale=0.45,keepaspectratio]{UML/UC8_1_3_5_2.png}
	\caption{UC8.1.3.5.2: Inserimento immagine per testo domanda}
\end{figure}
\FloatBarrier
\begin{itemize}
	\item\textbf{Attori}: utente autenticato, utente autenticato pro;
	\item\textbf{Scopo e descrizione}: lo scopo di questa funzionalità è offrire all'attore  la possibilità di inserire un'immagine relativa domanda;
	\item\textbf{Precondizione}: l'attore ha scelto la modalità di creazione della domanda a ordinamento di immagini; 
	\item \textbf{Postcondizione}: l'attore ha inserito un'immagine relativa alla domanda;
	\item\textbf{Scenario principale}: 
		\begin{enumerate}
			\item L'attore inserisce un'immagine relativa alla domanda;
			\item L'attore può eliminare l'immagine appena inserita (UC8.1.3.5.2.1).
		\end{enumerate}

\end{itemize}

\subsubsection{Caso d’uso UC8.1.3.5.2.1: Eliminazione immagine per testo domanda}
\begin{itemize}
	\item\textbf{Attori}: utente autenticato, utente autenticato pro;
	\item\textbf{Scopo e descrizione}: lo scopo di questa funzionalità è offrire all'attore  la possibilità di rimuovere l'immagine relativa ad domanda appena inserita;
	\item\textbf{Precondizione}: l'attore ha scelto la modalità di eliminazione dell'immagine usata come testo; 
	\item \textbf{Postcondizione}: l'attore ha eliminato l'immagine relativa alla domanda;
	\item\textbf{Scenario principale}: l'attore rimuove l'immagine relativa alla domanda. 
\end{itemize}

\subsubsection{Caso d’uso UC8.1.3.5.3: Inserimento immagini della risposta}
\label{UC8.1.3.5.3}
\begin{figure}[h]
	\centering
	\includegraphics[scale=0.45,keepaspectratio]{UML/UC8_1_3_5_3.png}
	\caption{UC8.1.3.5.3: Inserimento immagini della risposta}
\end{figure}
\begin{itemize}
	\item\textbf{Attori}: utente autenticato, utente autenticato pro;
	\item\textbf{Scopo e descrizione}: lo scopo di questa funzionalità è offrire all'attore  la possibilità di inserire le immagini per la risposta della relativa domanda;
	\item\textbf{Precondizione}: l'attore ha scelto la modalità di creazione della domanda a ordinamento di immagini; 
	\item \textbf{Postcondizione}: l'attore ha inserito delle immagini in ordine corretto per la risposta relativa alla domanda;
	\item\textbf{Scenario principale}:
		\begin{enumerate}
			\item L''attore inserisce delle immagini in ordine corretto per la relativa domanda;
			\item L'attore può eliminare un'immagine appena inserita (UC8.1.3.5.3.1).
		\end{enumerate}
\end{itemize}
\subsubsection{Caso d’uso UC8.1.3.5.3.1: Eliminazione immagine di una risposta}
\begin{itemize}
	\item\textbf{Attori}: utente autenticato, utente autenticato pro;
	\item\textbf{Scopo e descrizione}: lo scopo di questa funzionalità è offrire all'attore  la possibilità di rimuovere un'immagine inserita come risposta;
	\item\textbf{Precondizione}: l'attore ha scelto la modalità di eliminazione di un'immagine usata come risposta; 
	\item \textbf{Postcondizione}: l'attore ha eliminato un'immagine relativa alla risposta;
	\item\textbf{Scenario principale}: l'attore rimuove un'immagine relativa alla risposta. 
\end{itemize}
\subsubsection{Caso d'uso UC8.1.3.6: Creazione domanda a ordinamento di stringhe}
	\label{UC8.1.3.6}
	\begin{figure}[h]
		\centering
			\includegraphics[scale=0.45,keepaspectratio]{UML/UC8_1_3_6.png}
		\caption{UC8.1.3.6: Creazione domanda a ordinamento di stringhe}
	\end{figure}
	\FloatBarrier
\begin{itemize}
	\item\textbf{Attori}: utente autenticato, utente autenticato pro;
	\item\textbf{Descrizione}: l'attore può utilizzare la procedura guidata per la creazione di una domanda a ordinamento di stringhe;
	\item \textbf{Precondizione}: l'attore ha selezionato la funzionalità di creazione di una domanda a ordinamento stringhe;
	\item\textbf{Postcondizione}: l'attore ha creato una domanda a ordinamento di stringhe;
	\item\textbf{Scenario principale}:
		\begin{enumerate}
			\item L'attore può inserire il testo della domanda (UC8.1.3.6.1);
			\item L'attore può inserire le stringhe che compongono la risposta (UC8.1.3.6.2);
			\item L'attore può indicare la soluzione della sequenza di stringhe (UC8.1.3.6.3).
		\end{enumerate}
\end{itemize}

\subsubsection{Caso d'uso UC8.1.3.6.1: Inserimento testo domanda}
	\begin{itemize}
		\item \textbf{Attori}: utente autenticato, utente autenticato pro;
		\item \textbf{Descrizione}: l'attore può inserire il testo della domanda;
		\item\textbf{Precondizione}: l'attore ha selezionato la funzionalità di creazione di una domanda a ordinamento di stringhe;
		\item \textbf{Postcondizione}: l'attore ha inserito il testo della domanda;
		\item\textbf{Scenario principale}: l'attore inserisce il testo della domanda.
	\end{itemize}
	
\subsubsection{Caso d'uso UC8.1.3.6.2: Inserimento stringhe di composizione sequenza}
	\begin{itemize}
		\item \textbf{Attori}: utente autenticato, utente autenticato pro;
		\item \textbf{Descrizione}: l'attore può inserire le stringhe che costituiscono la risposta alla domanda;
		\item\textbf{Precondizione}: l'attore ha selezionato la funzionalità di creazione di una domanda a ordinamento di stringhe;
		\item \textbf{Postcondizione}: l'attore ha inserito le stringhe che costituiscono la risposta alla domanda;
		\item\textbf{Scenario principale}: l'attore inserisce le stringhe che costituiscono la risposta alla domanda.
	\end{itemize}
	
\subsubsection{Caso d'uso UC8.1.3.6.3: Composizione soluzione della sequenza}
	\begin{itemize}
		\item \textbf{Attori}: utente autenticato, utente autenticato pro;
		\item \textbf{Descrizione}: l'attore può indicare la soluzione della domanda mettendo nell'ordine corretto la sequenza di stringhe;
		\item\textbf{Precondizione}: l'attore ha selezionato la funzionalità di creazione di una domanda a ordinamento di immagini;
		\item \textbf{Postcondizione}: l'attore ha indicato la soluzione della sequenza;
		\item\textbf{Scenario principale}: l'attore indica la soluzione della sequenza. 
	\end{itemize}

\subsubsection{Caso d'uso UC8.1.3.7: Creazione domanda con area cliccabile nell'immagine}
\label{UC8.1.3.7}
\begin{figure}[h]
	\centering
	\includegraphics[scale=0.5,keepaspectratio]{UML/UC8_1_3_7.png}
	\caption{UC8.1.3.7: Creazione domanda a risposta multipla}
\end{figure}
\FloatBarrier
\begin{itemize}
	\item \textbf{Attori}: utente autenticato, utente autenticato pro;
	\item \textbf{Descrizione}: questa funzionalità offre al'attore la possibilità di creare domande la cui risposta è selezionabile all'interno di aree cliccabili in un immagine;
	\item \textbf{Precondizione}: l'attore hanno selezionato la creazione di una domanda con area cliccabile nell'immagine; 
	\item \textbf{Postcondizione}: l'attore hanno creato una domanda con area cliccabile nell'immagine;
	\item \textbf{Scenario principale}:
		\begin{enumerate}
	       	\item L'attore può compilare il campo dati destinato alla scrittura del testo della domanda (UC8.1.3.7.1);
	        \item L'attore può inserire una immagine relativa al testo della domanda (UC8.1.3.7.2);
			\item L'attore può scegliere quante aree saranno selezionabili all'interno dell'immagine (UC8.1.3.7.3);
			\item L'attore può scegliere le aree selezionabili all'interno dell'immagine (UC8.1.3.7.4);
	 	\end{enumerate}
\end{itemize}

\subsubsection{Caso d'uso UC8.1.3.7.1: Compilazione testo domanda}
\begin{itemize}
	\item \textbf{Attori}: utente autenticato, utente autenticato pro;
	\item \textbf{Descrizione}: l'attore può inserire un testo per la domanda che vogliono creare;
	\item \textbf{Precondizione}: l'attore ha selezionato la modalità di creazione di una domanda con area cliccabile;
	\item \textbf{Postcondizione}: l'attore ha inserito il testo della domanda.
\end{itemize}

\subsubsection{Caso d'uso UC8.1.3.7.2: Inserimento immagine}
\begin{itemize}
	\item \textbf{Attori}: utente autenticato, utente autenticato pro;
	\item \textbf{Descrizione}: l'attore ha la possibilità di inserire un'immagine relativa al testo della domanda;
	\item \textbf{Precondizione}: l'attore ha selezionato la modalità di creazione di una domanda con area cliccabile; 
	\item \textbf{Postcondizione}: l'attore ha inserito l'immagine;
	\item \textbf{Scenario principale}: l'attore inserisce l'immagine. 	
\end{itemize}

\subsubsection{Caso d'uso UC8.1.3.7.3: Scelta numero aree selezionabili}
\begin{itemize}
	\item \textbf{Attori}: utente autenticato, utente autenticato pro;
	\item \textbf{Descrizione}: l'attore ha la possibilità di scegliere il numero di aree selezionabili all'interno dell'immagine;
	\item \textbf{Precondizione}: l'attore ha selezionato la modalità di creazione di una domanda con area cliccabile; 
	\item \textbf{Postcondizione}: l'attore ha scelto il numero di aree selezionabili all'interno dell'immagine;
	\item \textbf{Scenario principale}: l'attore sceglie il numero di aree selezionabili all'interno dell'immagine. 	
\end{itemize}

\subsubsection{Caso d'uso UC8.1.3.7.4: Scelta aree selezionabili}
\begin{itemize}
	\item \textbf{Attori}: utente autenticato, utente autenticato pro;
	\item \textbf{Descrizione}: l'attore ha la possibilità di scegliere le aree selezionabili all'interno dell'immagine;
	\item \textbf{Precondizione}: l'attore ha selezionato la modalità di creazione di una domanda con area cliccabile; 
	\item \textbf{Postcondizione}: l'attore ha scelto le aree selezionabili all'interno dell'immagine;
	\item \textbf{Scenario principale}: l'attore sceglie le aree selezionabili all'interno dell'immagine. 	
\end{itemize}



\subsubsection{Caso d'uso UC8.1.3.8: Descrizione della domanda/esercizio}
\begin{itemize}
		\item
			\textbf{Attori}: utente autenticato, utente autenticato pro;
		\item
			\textbf{Scopo e descrizione}: gli attori possono inserire una descrizione alla domanda;
		\item		
			\textbf{Precondizione}: il sistema mostra all'utente un form per l'inserimento della descrizione;
		\item
			\textbf{Postcondizione}: l'utente ha inserito una descrizione alla domanda;
		\item
			\textbf{Scenario principale}:
				\begin{enumerate}
					\item 	
						Gli attori possono inserire una descrizione alla domanda.	
				\end{enumerate}
	\end{itemize}



	\subsubsection{Caso d'uso UC8.1.4: Conferma creazione}
	\begin{itemize}
		\item
			\textbf{Attori}: utente autenticato, utente autenticato pro;
		\item
			\textbf{Scopo e descrizione}: gli attori possono confermare la creazione della domanda;
		\item		
			\textbf{Precondizione}: il sistema presenta all'utente autenticato l'opzione per compiere questa operazione;
		\item
			\textbf{Postcondizione}: il sistema ha ricevuto i dati per la creazione;
		\item
			\textbf{Scenario Principale}:  
					\begin{enumerate}
						\item
							Gli attori confermano la creazione della domanda.
					\end{enumerate}
	\end{itemize}	
	\subsubsection{Caso d'uso UC8.1.5: Visualizzazione errore creazione domanda}
	\begin{itemize}
		\item
			\textbf{Attori}: utente autenticato, utente autenticato pro;
		\item
			\textbf{Scopo e descrizione}: gli attori possono visualizzare un messaggio d'errore nel caso si fossero verificati uno o più scenari alternativi;
		\item		
			\textbf{Precondizione}: il sistema ha ricevuto dei dati errati per la creazione della domanda;
		\item
			\textbf{Postcondizione}: il sistema mostra un messaggio d'errore;
	\end{itemize}	






	\subsubsection{Caso d'uso UC8.2: Modifica domanda esistente}
	\label{UC8.2}
	\begin{figure}[h]
		\centering
			\includegraphics[scale=0.45,keepaspectratio]{UML/UC8_2.png}
		\caption{UC8.2: Modifica domanda esistente}
	\end{figure}
	\FloatBarrier
	\begin{itemize}
		\item
			\textbf{Attori}: utente autenticato, utente autenticato pro;
		\item		
			\textbf{Scopo e descrizione}: L'utente autenticato modifica una domanda da lui in precedenza creata per correggere errori ortografici o sintattici;
		\item
			\textbf{Precondizione}: L'utente autenticato ha selezionato l'opzione per modificare una domanda;
		\item
			\textbf{Postcondizione}: L'utente riceve conferma dell'avvenuta modifica.
		\item
			\textbf{Scenario principale}:
	       		\begin{enumerate}
					\item
					L'attore richiama il wizard per modificare una domanda vero/falso[UC8.2.1];
					\item
					L'attore richiama il wizard per modificare una domanda a risposta multipla [UC8.2.2];
					\item
					L'attore richiama il wizard per modificare una domanda a riempimento di spazi vuoti [UC8.2.3];
					\item
					L'attore richiama il wizard per modificare una domanda di tipo a collegamento [UC8.2.4];
					\item
					L'attore richiama il wizard per modificare una domanda ad inserimento di immagini [UC8.2.5];
					\item
					L'attore richiama il wizard per modificare una domanda ad inserimento di stringhe [UC8.2.6];
					\item
					L'attore richiama il wizard per modificare una domanda con area cliccabile nell'immagine [UC8.2.7].
					\item
					L'attore richiama il wizard per modificare la descrizione della domanda [UC8.2.8].
	 			\end{enumerate}
	 	\item
			\textbf{Estensione}: L'utente visualizza un messaggio d'errore di conferma [UC8.2.10].
	 	\item
	 		\textbf{Scenario alternativo}: possono verificarsi uno o più di questi scenari:
				\begin{itemize}
					\item[-] 	
						L'argomento non è stato inserito;
					\item[-] 
						Le parole chiave non sono state inserite;
					\item[-] 
						Non tutti i campi obblogatori sono stati inseriti; 
					\item[-]
						Non è stata indicata la risposta corretta.	
				\end{itemize}
			In tal caso il sistema ripresenta il modulo di modifica della domanda, presentando un messaggio d'errore.
	\end{itemize}
	
		\subsubsection{Caso d'uso UC8.2.1: Selezione domande da modificare}
		\label{UC8.2.1}
		\begin{figure}[h]
			\centering
			\includegraphics[scale=0.5,keepaspectratio]{UML/UC8.png}
			\caption{Caso d'uso UC8.2.1: Selezione domande da modificare}
		\end{figure}
		\FloatBarrier
		\begin{itemize}
			\item \textbf{Attori}: 
			\item \textbf{Descrizione}:
			\item \textbf{Precondizione}: 
			\item \textbf{Postcondizione}: 
			\item \textbf{Scenario principale}: 
			\begin{enumerate}
				\item
			\end{enumerate}
			\item \textbf{Inclusioni}: 
			\item \textbf{Estensioni}: 
			\item \textbf{Scenari alternativi}: 
		\end{itemize}

	%inclusione file latex wizard
\subsubsection{Caso d'uso UC8.2.3.1: Modifica domanda vero/falso}
	\begin{itemize}
		\item
			\textbf{Attori}: utente autenticato, utente autenticato pro;
		\item		
			\textbf{Descrizione}: lo scopo di questa funzionalità è offrire agli attori la possibilità di modificare domande vero/falso;
		\item
			\textbf{Precondizione}: gli attori hanno selezionato la seguente funzionalità; 
		\item
			\textbf{Postcondizione}: gli attori hanno modificato una domanda vero/falso;
		\item
			\textbf{Scenario principale}:
	       		\begin{enumerate}
	       			\item
	       			Gli attori possono modificare il testo della domanda (UC8.2.3.1.1)
	       			\item
	       			Gli attori possono modificare l'immagine relativa al testo della domanda (UC8.2.3.1.2)
					\item
					Gli attori possono modificare la risposta corretta (UC8.2.3.1.3).
	 			\end{enumerate}
	\end{itemize}
	
\subsubsection{Caso d'uso UC8.2.3.1.1: Modifica testo della domanda}
	\begin{itemize}
		\item
			\textbf{Attori}: utente autenticato, utente autenticato pro;
		\item		
			\textbf{Descrizione}: lo scopo di questa funzionalità è offrire agli attori la possibilità di modificare il testo della domanda;
		\item
			\textbf{Precondizione}: gli attori hanno selezionato la funzionalità di modifica di una domanda vero/falso; 
		\item
			\textbf{Postcondizione}: gli attori hanno modificato il testo della domanda;
		\item
			\textbf{Scenario principale}: gli attori modificano il testo della domanda. 		
	\end{itemize}
	
\subsubsection{Caso d'uso UC8.2.3.1.2: Modifica immagine}
	\begin{itemize}
		\item
			\textbf{Attori}: utente autenticato, utente autenticato pro;
		\item		
			\textbf{Descrizione}: lo scopo di questa funzionalità è offrire agli attori la possibilità di modificare l'immagine relativa al testo della domanda;
		\item
			\textbf{Precondizione}: gli attori hanno selezionato la funzionalità di modifica di una domanda vero/falso; 
		\item
			\textbf{Postcondizione}: gli attori hanno modificato l'immagine;
		\item
			\textbf{Scenario principale}: gli attori modificano l'immagine. 	
	\end{itemize}
	
\subsubsection{Caso d'uso UC8.2.3.1.3: Modifica risposta corretta}
	\begin{itemize}
		\item
			\textbf{Attori}: utente autenticato, utente autenticato pro;
		\item		
			\textbf{Descrizione}: lo scopo di questa funzionalità è offrire agli attori la possibilità di modificare la risposta corretta;
		\item
			\textbf{Precondizione}: gli attori hanno selezionato la funzionalità di modifica di una domanda vero/falso; 
		\item
			\textbf{Postcondizione}: gli attori hanno modificato la risposta corretta;
		\item
			\textbf{Scenario principale}: gli attori modificano la risposta corretta tramite uno strumento di selezione. 
	 			
	\end{itemize}
	
\subsubsection{Caso d'uso UC8.2.1.2: Modifica domanda a risposta multipla}
	\label{UC8.2.1.2}
	\begin{figure}[h]
		\centering
			\includegraphics[scale=0.45,keepaspectratio]{UML/UC8_2_1_2.png}
		\caption{UC8.2.1.2: Modifica domanda a risposta multipla}
	\end{figure}
	\FloatBarrier
	\begin{itemize}
		\item
			\textbf{Attori}: utente autenticato, utente autenticato pro;
		\item		
			\textbf{Descrizione}: lo scopo di questa funzionalità è offrire agli attori la possibilità di modificare una domanda a risposta multipla;
		\item
			\textbf{Precondizione}: gli attori hanno selezionato la seguente funzionalità; 
		\item
			\textbf{Postcondizione}: gli attori hanno modificato una domanda a risposta multipla;
		\item
			\textbf{Scenario principale}:
	       		\begin{enumerate}
	       			\item
	       			Gli attori possono modificare il testo della domanda (UC8.2.1.2.1)
	       			\item
	       			Gli attori possono modificare l'immagine (UC8.2.1.2.2)
	       			\item
	       			Gli attori possono modificare le opzioni di risposta (UC8.2.1.2.3);
					\item
					Gli attori possono modificare le risposte corrette (UC8.2.1.2.4).
	 			\end{enumerate}
	\end{itemize}

\subsubsection{Caso d'uso UC8.2.1.2.1: Modifica testo della domanda}
	\begin{itemize}
		\item
			\textbf{Attori}: utente autenticato, utente autenticato pro;
		\item		
			\textbf{Descrizione}: lo scopo di questa funzionalità è offrire agli attori la possibilità di modificare il testo della domanda;
		\item
			\textbf{Precondizione}: gli attori hanno selezionato la modalità di modificare una domanda a risposta multipla; 
		\item
			\textbf{Postcondizione}: gli attori hanno modificato il testo della domanda;
		\item
			\textbf{Scenario principale}: gli attori modificano il testo della domanda. 
	 			
	\end{itemize}
	
\subsubsection{Caso d'uso UC8.2.1.2.2: Modifica immagine}
	\begin{itemize}
		\item
			\textbf{Attori}: utente autenticato, utente autenticato pro;
		\item		
			\textbf{Descrizione}: lo scopo di questa funzionalità è offrire agli attori la possibilità di modificare l'immagine relativa al testo della domanda;
		\item
			\textbf{Precondizione}: gli attori hanno selezionato la funzionalità di modificare una domanda a risposta multipla; 
		\item
			\textbf{Postcondizione}: gli attori hanno modificato l'immagine;
		\item
			\textbf{Scenario principale}: gli attori modificano l'immagine. 	
	\end{itemize}
	
	
\subsubsection{Caso d'uso UC8.2.1.2.3: Modifica opzioni di risposta}
	\begin{itemize}
		\item
			\textbf{Attori}: utente autenticato, utente autenticato pro;
		\item		
			\textbf{Descrizione}: lo scopo di questa funzionalità è offrire agli attori la possibilità di modificare le opzioni di risposta;
		\item
			\textbf{Precondizione}: gli attori hanno selezionato la funzionalità di modificare una domanda a risposta multipla; 
		\item
			\textbf{Postcondizione}: gli attori hanno modificato le opzioni di risposta;
		\item
			\textbf{Scenario principale}:
	       		\begin{enumerate}
	       			\item
	       			Gli attori possono modificare opzioni di risposta che includono testo (UC8.2.1.2.3.1);
					\item
					Gli attori possono modificare opzioni di risposta che includono immagini (UC8.2.1.2.3.2).
	 			\end{enumerate}
	\end{itemize}	
	
\subsubsection{Caso d'uso UC8.2.1.2.3.1: Modifica opzioni di risposta che includono testo}
	\begin{itemize}
		\item
			\textbf{Attori}: utente autenticato, utente autenticato pro;
		\item		
			\textbf{Descrizione}: lo scopo di questa funzionalità è offrire agli attori la possibilità di modificare le opzioni di risposta che includono testo;
		\item
			\textbf{Precondizione}: gli attori hanno selezionato la funzionalità di modificare una domanda a risposta multipla con opzioni di risposta che includono testo; 
		\item
			\textbf{Postcondizione}: gli attori hanno modificato le opzioni di risposta che includono testo;
		\item
			\textbf{Scenario principale}: gli attori modificano le opzioni di risposta che includono testo. 			
	\end{itemize}	
	
\subsubsection{Caso d'uso UC8.2.1.2.3.2: Modifica opzioni di risposta che includono immagini}
	\begin{itemize}
		\item
			\textbf{Attori}: utente autenticato, utente autenticato pro;
		\item		
			\textbf{Descrizione}: lo scopo di questa funzionalità è offrire agli attori la possibilità di modificare le opzioni di risposta che includono immagini;
		\item
			\textbf{Precondizione}: gli attori hanno selezionato la funzionalità di modificare una domanda a risposta multipla con opzioni di risposta che includono immagini; 
		\item
			\textbf{Postcondizione}: gli attori hanno modificato le opzioni di risposta che includono immagini;
		\item
			\textbf{Scenario principale}: gli attori modificano le opzioni di risposta che includono immagini. 			
	\end{itemize}
	
\subsubsection{Caso d'uso UC8.2.1.2.4: Modifica risposte corrette}
	\begin{itemize}
		\item
			\textbf{Attori}: utente autenticato, utente autenticato pro;
		\item		
			\textbf{Descrizione}: lo scopo di questa funzionalità è offrire agli attori la possibilità di modificare la selezione delle risposte corrette;
		\item
			\textbf{Precondizione}: gli attori hanno selezionato la funzionalità di modificare domande a risposte multiple; 
		\item
			\textbf{Postcondizione}: gli attori hanno modificato la selezione delle risposte corrette;
		\item
			\textbf{Scenario principale}: gli attori modificano la selezione delle risposte corrette. 			
	\end{itemize}

	
	
	
	
	
	
	
\subsubsection{Caso d'uso UC8.2.1.3: Modifica esercizi di riempimento degli spazi vuoti}
	\label{UC8.2.1.3}
	\begin{figure}[h]
		\centering
			\includegraphics[scale=0.45,keepaspectratio]{UML/UC8_2_1_3.png}
		\caption{UC8.2.1.3: Modifica esercizi di riempimento degli spazi vuoti}
	\end{figure}
	\FloatBarrier
	\begin{itemize}
		\item
			\textbf{Attori}: utente autenticato, utente autenticato pro;
		\item		
			\textbf{Descrizione}: l'attore può utilizzare la procedura guidata per la modifica di un esercizio di riempimento degli spazi vuoti;
		\item
			\textbf{Precondizione}: l'attore ha selezionato la funzionalità di modifica di un esercizio di riempimento degli spazi vuoti; 
		\item
			\textbf{Postcondizione}: l'attore ha modificato un esercizio di riempimento degli spazi vuoti;
		\item
			\textbf{Scenario principale}:
	       		\begin{enumerate}
	       			\item
	       			L'attore può modificare il testo dell'esercizio (UC8.2.1.3.1);
	       			\item
	       			L'attore può modificare le parole che saranno sostituite con degli spazi vuoti dal sistema (UC8.2.1.3.2).
	 			\end{enumerate}
	\end{itemize}
	
\subsubsection{Caso d'uso UC8.2.1.3.1: Modifica testo dell'esercizio}
	\begin{itemize}
		\item
			\textbf{Attori}: utente autenticato, utente autenticato pro;
		\item		
			\textbf{Descrizione}: l'attore può modificare il testo dell'esercizio;
		\item
			\textbf{Precondizione}: l'attore ha selezionato la funzionalità di modifica di un esercizio di riempimento degli spazi vuoti; 
		\item
			\textbf{Postcondizione}: l'attore ha modificato il testo dell'esercizio;
		\item
			\textbf{Scenario principale}: l'attore modifica il testo dell'esercizio.
	\end{itemize}


\subsubsection{Caso d'uso UC8.2.1.3.2: Modifica parole da oscurare}
	\begin{itemize}
		\item
			\textbf{Attori}: utente autenticato, utente autenticato pro;
		\item		
			\textbf{Descrizione}: l'attore può modificare le parole che saranno sostituite con degli spazi vuoti;
		\item
			\textbf{Precondizione}: l'attore ha selezionato la funzionalità di modifica di un esercizio di riempimento degli spazi vuoti; 
		\item
			\textbf{Postcondizione}: l'attore ha modificato le parole che saranno sostituite con degli spazi vuoti;
		\item
			\textbf{Scenario principale}: l'attore modifica le parole che verranno oscurate dal sistema.
	\end{itemize}
\subsubsection{Caso d'uso UC8.2.1.4: Modifica domanda di collegamento}
\label{UC8.2.1.4}
\begin{figure}[h]
	\centering
	\includegraphics[scale=0.5,keepaspectratio]{UML/UC8_2_1_4.png}
	\caption{Caso d'uso UC8.2.1.4: Modifica domanda di collegamento}
\end{figure}
\FloatBarrier
\begin{itemize}
	\item \textbf{Attori}: \uau, \uaupro;
	\item \textbf{Descrizione}: l'attore utilizza la procedura guidata per la modifica di una domanda di tipo collegamento. 
	\item \textbf{Precondizione}: l'attore ha scelto l'opzione "Modifica domanda di collegamento" nelle scelte possibili nel caso d'uso UC8.2.1;
	\item \textbf{Postcondizione}: l'attore ha modificato tutti i campi necessari per completare la modifica di una domanda di tipo collegamento. L'attore deve inserire almeno due coppie per rispettare la postcondizione;
	\item \textbf{Scenario principale}: 
	\begin{enumerate}
		\item L'attore inserisce una coppia di elementi (UC8.2.1.4.1);
		\item L'attore può eliminare una coppia di elementi appena inserita (UC8.2.1.4.2);
		\item L'attore può modificare una coppia di elementi (UC8.2.1.4.3).
	\end{enumerate}
	\item \textbf{Scenari alternativi}: se non ci sono almeno due coppie presenti nella lista delle coppie l'utente deve inserire una nuova coppia di elementi.
\end{itemize}

	\subsubsection{Caso d'uso UC8.2.1.4.1: Inserimento coppia di elementi}
	\label{UC8.2.1.4.1}
	\begin{figure}[h]
		\centering
		\includegraphics[scale=0.5,keepaspectratio]{UML/UC8_2_1_4_1.png}
		\caption{Caso d'uso UC8.2.1.4.1: Inserimento coppia di elementi}
	\end{figure}
	\FloatBarrier
	\begin{itemize}
		\item \textbf{Attori}: \uau, \uaupro;
		\item \textbf{Descrizione}: l'attore inserisce una coppia di elementi e questi possono essere sia immagini che testo o combinazioni di questi. Un coppia una volta creata sarà già soluzione di essa stessa: il primo elemento dovrà essere collegato con il secondo o viceversa. 
		\item \textbf{Precondizione}: l'attore ha scelto l'opzione "Inserimento coppia di elementi" nelle scelte possibili nel caso d'uso UC8.2.1.4;
		\item \textbf{Postcondizione}: l'attore ha inserito una coppia di elementi nella lista di coppie di elementi; 
		\item \textbf{Scenario principale}: 
		\begin{enumerate}
			\item L'attore inserisce come primo elemento un'immagine (UC8.2.1.4.1.1);
			\item L'attore inserisce come primo elemento un testo (UC8.2.1.4.1.2);
			\item L'attore inserisce come secondo elemento un'immagine (UC8.2.1.4.1.3);
			\item L'attore inserisce come secondo elemento un testo (UC8.2.1.4.1.4).	
		\end{enumerate}
	\end{itemize}
	
		\subsubsection{Caso d'uso UC8.2.1.4.1.1: Inserimento di un'immagine come primo elemento}
		\label{UC8.2.1.4.1.1}
		\begin{itemize}
			\item \textbf{Attori}: \uau, \uaupro;
			\item \textbf{Descrizione}: l'attore decide di inserire come primo elemento della coppia un'immagine;
			\item \textbf{Precondizione}: l'attore ha scelto l'opzione "Inserimento di un'immagine come primo elemento" nelle scelte possibili nel caso d'uso UC8.2.1.4.1 e non ha già scelto l'opzione presentata dall'use case UC8.2.1.4.1.2;
			\item \textbf{Postcondizione}: l'utente ha inserito come primo elemento un'immagine;
			\item \textbf{Scenario principale}: l'attore carica un'immagine come primo elemento della coppia;  
			\item \textbf{Scenari alternativi}: l'attore non può selezionare il seguente campo se è già stata utilizzata l'opzione proposta dall'use case UC8.2.1.4.1.2. Viene così rimandato all'use case UC8.2.1.4.1;
		\end{itemize}
		
		\subsubsection{Caso d'uso UC8.2.1.4.1.2: Inserimento di un testo come primo elemento}
		\label{UC8.2.1.4.1.2}
		\begin{itemize}
			\item \textbf{Attori}: \uau, \uaupro;
			\item \textbf{Descrizione}: l'attore decide di inserire come primo elemento della coppia un testo;
			\item \textbf{Precondizione}: l'attore ha scelto l'opzione "Inserimento di un testo come primo elemento" nelle scelte possibili nel caso d'uso UC8.2.1.4.1 e non ha già scelto l'opzione presentata dall'use case UC8.2.1.4.1.1;
			\item \textbf{Postcondizione}: l'utente ha inserito come primo elemento un testo;
			\item \textbf{Scenario principale}: l'attore inserisce del testo come primo elemento della coppia ;  
			\item \textbf{Scenari alternativi}: l'attore non può selezionare il seguente campo se è già stata utilizzata l'opzione proposta dall'use case UC8.2.1.4.1.1. Viene così rimandato all'use case UC8.2.1.4.1;
		\end{itemize}
		
		\subsubsection{Caso d'uso UC8.2.1.4.1.3: Inserimento di un'immagine come secondo elemento}
		\label{UC8.2.1.4.1.3}
		\begin{itemize}
			\item \textbf{Attori}: \uau, \uaupro;
			\item \textbf{Descrizione}: l'attore decide di inserire come secondo elemento della coppia un'immagine;
			\item \textbf{Precondizione}: l'attore ha scelto l'opzione "Inserimento di un'immagine come secondo elemento" nelle scelte possibili nel caso d'uso UC8.2.1.4.1 e non ha già scelto l'opzione presentata dall'use case UC8.2.1.4.1.4;
			\item \textbf{Postcondizione}: l'utente ha inserito come secondo elemento un'immagine;
			\item \textbf{Scenario principale}: l'attore carica un'immagine come secondo elemento della coppia;  
			\item \textbf{Scenari alternativi}: l'attore non può selezionare il seguente campo se è già stata utilizzata l'opzione proposta dall'use case UC8.2.1.4.1.4. Viene così rimandato all'use case UC8.2.1.4.1;
		\end{itemize}
		
		\subsubsection{Caso d'uso UC8.2.1.4.1.4: Inserimento di un testo come secondo elemento}
		\label{UC8.2.1.4.1.4}
		\begin{itemize}
			\item \textbf{Attori}: \uau, \uaupro;
			\item \textbf{Descrizione}: l'attore decide di inserire come secondo elemento della coppia un testo;
			\item \textbf{Precondizione}: l'attore ha scelto l'opzione "Inserimento di un testo come secondo elemento" nelle scelte possibili nel caso d'uso UC8.2.1.4.1 e non ha già scelto l'opzione presentata dall'use case UC8.2.1.4.1.3;
			\item \textbf{Postcondizione}: l'utente ha inserito come secondo elemento un'immagine;
			\item \textbf{Scenario principale}: l'attore inserisce del testo come secondo elemento della coppia;  
			\item \textbf{Scenari alternativi}: l'attore non può selezionare il seguente campo se è già stata utilizzata l'opzione proposta dall'use case UC8.2.1.4.1.3. Viene così rimandato all'use case UC8.2.1.4.1;
		\end{itemize}
	
	\subsubsection{Caso d'uso UC8.2.1.4.2: Eliminazione coppia di elementi}
	\label{UC8.2.1.4.2}
	\begin{figure}[h]
		\centering
		\includegraphics[scale=0.5,keepaspectratio]{UML/UC8_2_1_4_2.png}
		\caption{Caso d'uso UC8.2.1.4.2: Eliminazione coppia di elementi}
	\end{figure}
	\FloatBarrier
	\begin{itemize}
		\item \textbf{Attori}: \uau, \uaupro;
		\item \textbf{Descrizione}: l'utente decide di eliminare una coppia di elementi dalla lista di coppie di elementi;
		\item \textbf{Precondizione}: l'attore ha scelto l'opzione "Eliminazione coppia di elementi" nelle scelte possibili nel caso d'uso UC8.2.1.4;
		\item \textbf{Postcondizione}: l'attore ha eliminato, dalla lista delle coppie di elementi, una coppia;
		\item \textbf{Scenario principale}: l'attore deve confermare di voler eliminare una coppia di elementi (UC8.2.1.4.2.1); 
	\end{itemize}
	
		\subsubsection{Caso d'uso UC8.2.1.4.2.1: Confermazione eliminazione coppia di elementi}
		\label{UC8.2.1.4.2.1}
		\begin{itemize}
			\item \textbf{Attori}: \uau, \uaupro;
			\item \textbf{Descrizione}: l'attore deve confermare di voler eliminare la coppia di elementi;
			\item \textbf{Precondizione}: l'attore ha scelto l'opzione "Confermazione eliminazione coppia di elementi" nelle scelte possibili nel caso d'uso UC8.2.1.4.2;
			\item \textbf{Postcondizione}: l'attore ha confermato di voler eliminare la coppia di elementi;
			\item \textbf{Scenario principale}: l'attore conferma di voler eliminare la coppia di elementi;
			\item \textbf{Scenari alternativi}: l'attore non conferma di voler eliminare la coppia di elementi; 
		\end{itemize}
	
	\subsubsection{Caso d'uso UC8.2.1.4.3: Modifica coppia di elementi}
	\label{UC8.2.1.4.3}
	\begin{figure}[h]
		\centering
		\includegraphics[scale=0.5,keepaspectratio]{UML/UC8_2_1_4_3.png}
		\caption{Caso d'uso UC8.2.1.4.3: Modifica coppia di elementi}
	\end{figure}
	\FloatBarrier
	\begin{itemize}
		\item \textbf{Attori}: \uau, \uaupro;
		\item \textbf{Descrizione}: l'attore modifica una coppia di elementi;
		\item \textbf{Precondizione}: l'attore ha scelto l'opzione "Modifica coppia di elementi" nelle scelte possibili nel caso d'uso UC8.2.1.4;
		\item \textbf{Postcondizione}: l'attore ha modificato una coppia di elementi nella lista di coppie di elementi; 
		\item \textbf{Scenario principale}: 
		\begin{enumerate}
			\item L'attore modifica un elemento cambiandone il testo (UC8.2.1.4.1.1);
			\item L'attore modifica un elemento cambiandone l'immagine (UC8.2.1.4.1.2);
			\item L'attore modifica un elemento facendolo passare da testo ad immagine (UC8.2.1.4.1.3);
			\item L'attore modifica un elemento facendolo passare da immagine a testo (UC8.2.1.4.1.4).	
		\end{enumerate}
	\end{itemize}
	
		\subsubsection{Caso d'uso UC8.2.1.4.3.1: Modifica testo di un elemento}
		\label{UC8.2.1.4.3.1}
		\begin{itemize}
			\item \textbf{Attori}: \uau, \uaupro;
			\item \textbf{Descrizione}: l'attore decide di modificare il testo di un elemento;
			\item \textbf{Precondizione}: l'attore ha scelto l'opzione "Modifica testo di un elemento" nelle scelte possibili nel caso d'uso UC8.2.1.4.3;
			\item \textbf{Postcondizione}: l'utente ha modificato il testo di un elemento;
			\item \textbf{Scenario principale}: l'attore modifica il testo di un elemento;  
		\end{itemize}
		
		\subsubsection{Caso d'uso UC8.2.1.4.3.2: Modifica immagine di un elemento}
		\label{UC8.2.1.4.3.2}
		\begin{itemize}
			\item \textbf{Attori}: \uau, \uaupro;
			\item \textbf{Descrizione}: l'attore decide di caricare un'altra immagine per un elemento;
			\item \textbf{Precondizione}: l'attore ha scelto l'opzione "Modifica immagine di un elemento" nelle scelte possibili nel caso d'uso UC8.2.1.4.3;
			\item \textbf{Postcondizione}: l'utente ha caricato un'altra immagine per un elemento;
			\item \textbf{Scenario principale}: l'attore carica un'altra immagine per un elemento;
		\end{itemize}
		
		\subsubsection{Caso d'uso UC8.2.1.4.3.3: Cambia testo in immagine}
		\label{UC8.2.1.4.3.3}
		\begin{itemize}
			\item \textbf{Attori}: \uau, \uaupro;
			\item \textbf{Descrizione}: l'attore decide di modificare un elemento facendolo diventare un'immagine al posto di un testo;
			\item \textbf{Precondizione}: l'attore ha scelto l'opzione "Cambia testo in immagine" nelle scelte possibili nel caso d'uso UC8.2.1.4.3;
			\item \textbf{Postcondizione}: l'utente ha fatto diventare un'immagine un elemento che prima era un testo;
			\item \textbf{Scenario principale}: l'attore inserisce un'immagine come modifica dell'elemento;  
		\end{itemize}
		
		\subsubsection{Caso d'uso UC8.2.1.4.3.4: Cambia immagine in testo}
		\label{UC8.2.1.4.3.4}
		\begin{itemize}
			\item \textbf{Attori}: \uau, \uaupro;
			\item \textbf{Descrizione}: l'attore decide di modificare un elemento facendolo diventare un testo al posto di un'immagine;
			\item \textbf{Precondizione}: l'attore ha scelto l'opzione "Cambia immagine in testo" nelle scelte possibili nel caso d'uso UC8.2.1.4.3;
			\item \textbf{Postcondizione}: l'utente ha fatto diventare un testo un elemento che prima era un'immagine;
			\item \textbf{Scenario principale}: l'attore inserisce del testo come modifica dell'elemento;  
		\end{itemize}

\subsubsection{Caso d'uso UC8.2.1.5: Modifica domanda a ordinamento di immagini}
\label{UC8.2.1.5}
	\begin{figure}[h]
		\centering
			\includegraphics[scale=0.45,keepaspectratio]{UML/UC8_2_1_5.png}
		\caption{UC8.2.1.5: Modifica domanda a ordinamento di immagini}
	\end{figure}
\begin{itemize}
	\item\textbf{Attori}: utente autenticato, utente autenticato pro;
	\item\textbf{Descrizione}: l'attore può utilizzare la procedura guidata per la modifica di una domanda a ordinamento di immagini;
	\item\textbf{Precondizione}: l'attore ha selezionato la funzionalità di modifica di una domanda a ordinamento di immagini;
	\item \textbf{Postcondizione}: l'attore ha modificato una domanda a ordinamento di immagini;
	\item\textbf{Scenario principale}: 
	\begin{itemize}
		\item L'attore può modificare il testo della domanda (UC8.2.1.5.1);
		\item L'attore può modifica l'immagine relativa al testo della domanda (UC8.2.1.5.2);
		\item L'attore può modificare le immagini della sequenza che costituirà la risposta (UC8.2.1.5.3);
		\item L'attore può modifica l'ordine corretto delle immagini che costituiscono la risposta (UC8.2.1.5.4).
	\end{itemize}
\end{itemize}

\subsubsection{Caso d'uso UC8.2.1.5.1: Modifica testo della domanda}
\begin{itemize}
	\item\textbf{Attori}: utente autenticato, utente autenticato pro;
	\textbf{Descrizione}: l'attore può modificare il testo della domanda;
		\item
			\textbf{Precondizione}: l'attore ha selezionato la funzionalità di modifica di una domanda a ordinamento di immagini; 
		\item
			\textbf{Postcondizione}: l'attore ha modificato il testo della domanda;
		\item
			\textbf{Scenario principale}: l'attore modifica il testo della domanda.	
	\end{itemize}

\subsubsection{Caso d'uso UC8.2.1.5.2: Modifica immagine della domanda}
\begin{itemize}
	\item\textbf{Attori}: utente autenticato, utente autenticato pro;
	\textbf{Descrizione}: l'attore può modificare l'immagine relativa al testo della domanda;
		\item
			\textbf{Precondizione}: l'attore ha selezionato la funzionalità di modifica di una domanda a ordinamento di immagini; 
		\item
			\textbf{Postcondizione}: l'attore ha modificato l'immagine relativa al testo della domanda;
		\item
			\textbf{Scenario principale}: l'attore modifica l'immagine relativa al testo della domanda. 	
	\end{itemize}

\subsubsection{Caso d'uso UC8.2.1.5.3: Modifica immagini come risposta}
\begin{itemize}
	\item\textbf{Attori}: utente autenticato, utente autenticato pro;
	\item\textbf{Descrizione}: l'attore può modificare le immagini presenti come risposta alla domanda sostituendole con delle altre;
	\item\textbf{Precondizione}: l'attore ha selezionato la funzionalità di modifica di una domanda a ordinamento di immagini;
	\item \textbf{Postcondizione}: l'attore ha inserito le nuove immagini come risposta alla domanda;
	\item\textbf{Scenario principale}: l'attore sostituisce le immagini presenti come risposta della domanda con delle altre.
\end{itemize}

\subsubsection{Caso d'uso UC8.2.1.5.4: Modifica ordine immagini come risposta}
\begin{itemize}
	\item\textbf{Attori}: utente autenticato, utente autenticato pro;
	\item\textbf{Descrizione}: l'attore può modificare l'ordine corretto delle immagini che costituiscono la risposta;
	\item\textbf{Precondizione}: l'attore ha selezionato la funzionalità di modifica di una domanda a ordinamento di immagini;
	\item \textbf{Postcondizione}: l'attore ha modificato l'ordine delle immagini che costituiscono la risposta;
	\item\textbf{Scenario principale}: l'attore modifica l'ordine delle immagini che costituiscono la risposta.
\end{itemize}

\subsubsection{Caso d’uso UC8.2.3.6: Modifica ordinamento di stringhe}
	\label{UC8.2.3.6}
	\begin{figure}[h]
		\centering
		\includegraphics[scale=0.45,keepaspectratio]{UML/UC8_2_3_6.png}
		\caption{UC8.2.3.6: Modifica domanda ordinamento stringhe}
	\end{figure}
	\FloatBarrier
\begin{itemize}
	\item\textbf{Attori}: utente autenticato, utente autenticato pro;
	\item\textbf{Descrizione}: gli attori possono modificare una domanda del tipo ordinamento di stringhe aggiunta da loro;
	\item\textbf{Precondizione}: il sistema mostra agli attori il form di modifica dei campi dati per la tipologia di domanda scelta; 
	\item \textbf{Postcondizione}: gli attori hanno modificato la domanda;
	\item\textbf{Scenario principale}:
		\begin{itemize}
			\item Gli attori possono modificare il testo della domanda (UC8.2.3.6.1);
			\item Gli attori possono modificare il testo e il numero delle stringhe che compongono la sequenza della domanda (UC8.2.3.6.2);
			\item Gli attori possono modificare la soluzione della domanda (UC8.2.3.6.3); 
		\end{itemize}
\end{itemize}

\subsubsection{Caso d'uso UC8.2.3.6.1: Modifica testo della domanda}
\begin{itemize}
	\item \textbf{Attori}: utente autenticato, utente autenticato pro;
	\item \textbf{Descrizione}: gli attori possono modificare il testo della domanda;
	\item \textbf{Precondizione}: gli attori hanno selezionato l'opzione di modifica di una domanda;
	\item \textbf{Postcondizione}: gli attori hanno modificato il testo della domanda.
\end{itemize}

\subsubsection{Caso d'uso UC8.2.3.6.2: Modifica del testo e del numero di stringhe}
\begin{itemize}
	\item \textbf{Attori}: utente autenticato, utente autenticato pro;
	\item \textbf{Descrizione}: gli attori possono modificare il testo e il numero di stringhe che compongono la domanda;
	\item \textbf{Precondizione}: gli attori hanno selezionato l'opzione di modifica di una domanda;
	\item \textbf{Postcondizione}: gli attori hanno modificato il testo e il numero delle stringhe che compongono la domanda.
\end{itemize}

\subsubsection{Caso d'uso UC8.2.3.6.3: Modifica soluzione domanda}
\begin{itemize}
	\item \textbf{Attori}: utente autenticato, utente autenticato pro;
	\item \textbf{Descrizione}: gli attori possono modificare la sequenza della soluzione della domanda;
	\item \textbf{Precondizione}: gli autori hanno selezionato l'opzione di modifica di una domanda;
	\item \textbf{Postcondizione}: gli autori hanno modificato la sequenza della soluzione della domanda. 
\end{itemize}
\subsubsection{Caso d'uso UC8.2.1.7: Modifica domanda con area cliccabile nell'immagine}
\label{UC8.2.1.7}
\begin{figure}[h]
	\centering
	\includegraphics[scale=0.5,keepaspectratio]{UML/UC8_2_1_7.png}
	\caption{UC8.2.1.7: Creazione domanda a risposta multipla}
\end{figure}
\FloatBarrier
\begin{itemize}
	\item \textbf{Attori}: utente autenticato, utente autenticato pro;
	\item \textbf{Descrizione}: questa funzionalità offre all'attore la possibilità di modificare una domanda la cui risposta è selezionabile all'interno di aree cliccabili in un immagine;
	\item \textbf{Precondizione}: l'attore ha selezionato la modalità di modifica di una domanda con area cliccabile; 
	\item \textbf{Postcondizione}: l'attore ha modificato una domanda con area cliccabile nell'immagine;
	\item \textbf{Scenario principale}:
		\begin{enumerate}
	       	\item L'attore può modificare il campo dati destinato alla scrittura del testo della domanda (UC8.2.1.7.1);
	        \item L'attore può inserire una nuova immagine relativa al testo della domanda (UC8.1.1.7.2);
			\item L'attore può scegliere un nuovo numero di aree che saranno selezionabili all'interno dell'immagine (UC8.2.1.7.3);
			\item L'attore può scegliere nuove aree selezionabili all'interno dell'immagine (UC8.2.1.7.4);
	 	\end{enumerate}
\end{itemize}

\subsubsection{Caso d'uso UC8.2.1.7.1: Modifica testo domanda}
\begin{itemize}
	\item \textbf{Attori}: utente autenticato, utente autenticato pro;
	\item \textbf{Descrizione}: l'attore può modificare il testo della domanda;
	\item \textbf{Precondizione}: l'attore ha selezionato la modalità di modifica del testo della domanda;
	\item \textbf{Postcondizione}: l'attore ha modificato il testo della domanda.
\end{itemize}

\subsubsection{Caso d'uso UC8.2.1.7.2: Inserimento nuova immagine}
\begin{itemize}
	\item \textbf{Attori}: utente autenticato, utente autenticato pro;
	\item \textbf{Descrizione}: l'attore ha la possibilità di inserire una nuova immagine relativa al testo della domanda;
	\item \textbf{Precondizione}: l'attore ha selezionato la modalità di modifica dell'immagine; 
	\item \textbf{Postcondizione}: l'attore ha inserito una nuova immagine;
	\item \textbf{Scenario principale}: l'attore inserisce una nuova immagine. 	
\end{itemize}

\subsubsection{Caso d'uso UC8.2.1.7.3: Modifica numero aree selezionabili}
\begin{itemize}
	\item \textbf{Attori}: utente autenticato, utente autenticato pro;
	\item \textbf{Descrizione}: l'attore ha la possibilità di scegliere un nuovo numero di aree selezionabili all'interno dell'immagine;
	\item \textbf{Precondizione}: l'attore ha selezionato la modalità di modifica del numero di aree selezionabili; 
	\item \textbf{Postcondizione}: l'attore ha scelto il nuovo numero di aree selezionabili all'interno dell'immagine;
	\item \textbf{Scenario principale}: l'attore sceglie il nuovo numero di aree selezionabili all'interno dell'immagine. 	
\end{itemize}

\subsubsection{Caso d'uso UC8.2.1.7.4: Scelta nuove aree selezionabili}
\begin{itemize}
	\item \textbf{Attori}: utente autenticato, utente autenticato pro;
	\item \textbf{Descrizione}: l'attore ha la possibilità di scegliere nuove aree selezionabili all'interno dell'immagine;
	\item \textbf{Precondizione}: l'attore ha selezionato la modalità di scelta delle nuove aree cliccabili; 
	\item \textbf{Postcondizione}: l'attore ha scelto nuove aree selezionabili all'interno dell'immagine;
	\item \textbf{Scenario principale}: l'attore sceglie nuove aree selezionabili all'interno dell'immagine. 	
\end{itemize}



		\subsubsection{Caso d'uso UC8.2.1.7: Modifica descrizione della domanda/esercizio}
		\label{UC8.2.1.7}
		\begin{figure}[h]
			\centering
			\includegraphics[scale=0.5,keepaspectratio]{UML/UC8.png}
			\caption{Caso d'uso UC8.2.1.7: Modifica descrizione della domanda/esercizio}
		\end{figure}
		\FloatBarrier
		\begin{itemize}
			\item \textbf{Attori}: 
			\item \textbf{Descrizione}:
			\item \textbf{Precondizione}: 
			\item \textbf{Postcondizione}: 
			\item \textbf{Scenario principale}: 
			\begin{enumerate}
				\item
			\end{enumerate}
			\item \textbf{Inclusioni}: 
			\item \textbf{Estensioni}: 
			\item \textbf{Scenari alternativi}: 
		\end{itemize}


	\subsubsection{Caso d'uso UC8.2.2: Conferma modifica}
	\begin{itemize}
		\item
			\textbf{Attori}: Utente autenticato, utente autenticato pro;
		\item
			\textbf{Scopo e descrizione}: L'utente autenticato conferma la modifica dei dati della domanda selezionata;
		\item		
			\textbf{Precondizione}: Il sistema presenta all'utente autenticato l'opzione per compiere questa operazione;
		\item
			\textbf{Postcondizione}: Il sistema ha ricevuto i dati per la modifica.
	\end{itemize}		
	\subsubsection{Caso d'uso UC8.2.3: Visualizzazione errore modifica}
	\begin{itemize}
		\item
			\textbf{Attori}: Utente autenticato, utente autenticato pro;
		\item
			\textbf{Scopo e descrizione}: L'utente visualizza un messaggio d'errore nel caso si fossero verificati uno o più scenari alternativi;
		\item		
			\textbf{Precondizione}: Il sistema ha ricevuto dei dati errati per la modifica;
		\item
			\textbf{Postcondizione}: Il sistema mostra un messaggio d'errore.
	\end{itemize}	
	
