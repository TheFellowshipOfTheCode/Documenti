\subsubsection{Caso d'uso UC4: Gestione profilo utente}

\begin{itemize}
	\item \textbf{Attori}: utente autenticato;
	\item \textbf{Descrizione}: l'utente autenticato può visualizzare e modificare i suoi dati personali;
	\item \textbf{Precondizione}: il sistema è predisposto per consentire all'utente autenticato di gestire i propri dati personali;
	\item \textbf{Postcondizione}: il sistema ha attuato le modifiche effettuate dall'utente autenticato ai propri dati personali;
	\item \textbf{Scenario principale}:
		\begin{enumerate}
			\item L'utente autenticato può modificare il proprio nome utente (UC4.1);
			\item L'utente autenticato può modificare la propria e-mail (UC4.2);
			\item L'utente autenticato può modificare la propria password (UC4.3);
			\item L'utente autenticato può eliminare il proprio account (UC4.4).
		\end{enumerate} 
	\item \textbf{Scenari alternativi}: se le operazioni di modifica non vengono confermate il sistema non le rende persistenti e visualizza le funzionalità di gestione del profilo utente. 
\end{itemize}

\paragraph{Caso d'uso UC4.1: Modifica nome utente}

\begin{itemize}
	\item \textbf{Attori}: utente autenticato;
	\item \textbf{Descrizione}: l'utente autenticato può modificare il proprio nome e cognome utente inserendone di nuovo;
	\item \textbf{Precondizione}: il sistema presenta la schermata dove è possibile modificare nome e cognome utente;
	\item \textbf{Postcondizione}: il sistema ha reso persistenti le modifiche al nome e cognome utente;
	\item \textbf{Scenario principale}:
		\begin{enumerate}
			\item L'utente autenticato può inserire un nuovo nome utente (UC4.1.1);
			\item L'utente autenticato può inserire un nuovo cognome utente (UC4.1.2);
			\item L'utente autenticato può confermare le modifiche al nome e cognome utente (UC4.1.3);
		\end{enumerate}
	\item \textbf{Scenari alternativi}: l'utente autenticato annulla le modifiche e il sistema riporta lo alla schermata di gestione del profilo utente.
\end{itemize}

\paragraph{Caso d'uso UC4.2: Modifica e-mail}

\begin{itemize}
	\item \textbf{Attori}: utente autenticato;
	\item \textbf{Descrizione}: l'utente autenticato può modificare il proprio indirizzo di posta elettronica inserendone uno nuovo;
	\item \textbf{Precondizione}: il sistema presenta la schermata dove è possibile modificare l'indirizzo di posta elettronica;
	\item \textbf{Postcondizione}: il sistema ha reso persistenti le modifiche all'indirizzo di posta elettronica;
	\item \textbf{Scenario principale}:
		\begin{enumerate}
			\item L'utente autenticato può inserire un nuovo indirizzo di posta elettronica (UC4.2.1);
			\item L'utente autenticato può confermare le modifiche al proprio indirizzo di posta elettronica (UC4.2.2).
		\end{enumerate}
	\item \textbf{Scenari alternativi}: l'utente autenticato annulla le modifiche e il sistema riporta lo alla schermata di gestione del profilo utente.
\end{itemize}

\paragraph{Caso d'uso UC4.3: Modifica password}

\begin{itemize}
	\item \textbf{Attori}: utente autenticato;
	\item \textbf{Descrizione}: l'utente autenticato può modificare la propria password elettronica inserendone una nuova;
	\item \textbf{Precondizione}: il sistema presenta la schermata dove è possibile modificare la propria password;
	\item \textbf{Postcondizione}: il sistema ha reso persistenti le modifiche alla propria password;
	\item \textbf{Scenario principale}:
	\begin{enumerate}
		\item L'utente autenticato può inserire la propria vecchia password (UC4.3.2);
		\item L'utente autenticato può inserire una nuova password (UC4.3.2);
		\item L'utente autenticato può confermare la modifica della password (UC4.3.3).
	\end{enumerate}
	\item \textbf{Scenari alternativi}: l'utente autenticato annulla le modifiche e il sistema riporta lo alla schermata di gestione del profilo utente.
\end{itemize}

\paragraph{Caso d'uso UC4.4: Eliminazione account}

\begin{itemize}
	\item \textbf{Attori}: utente autenticato;
	\item \textbf{Descrizione}: l'utente autenticato può eliminare il proprio account dal sistema; l'eliminazione dell'account comporta la cancellazione dei propri dati dal sistema; 
	\item \textbf{Precondizione}: il sistema visualizza l'opzione dove è possibile eliminare il proprio account;
	\item \textbf{Postcondizione}: il sistema ha eliminato in maniera persistente il proprio account e tutti i relativi dati;
	\item \textbf{Scenario principale}:
		\begin{enumerate}
			\item L'utente può confermare l'eliminazione del proprio account e dei relativi dati personali (UC4.4.1).
		\end{enumerate}
\end{itemize}

