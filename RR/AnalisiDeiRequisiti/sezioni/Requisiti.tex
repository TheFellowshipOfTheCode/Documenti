\newpage
\section{Requisiti}
In questa sezione verranno presentati i requisiti individuati dal \textit{team\ped{G}} durante l'analisi del capitolato e dei casi d'uso, discussi con il proponente durante le riunioni esterne e decisi dai componenti nelle riunioni interne. Per facilitare la consultazione, i
requisiti saranno separati su più tabelle in base alla loro categoria. I requisiti saranno classificati per tipo e importanza e utilizzeranno la seguente sintassi:
\begin{center}
	R[Importanza][Tipo][Codice]
\end{center}
dove:
\begin{itemize}
	\item \textbf{Importanza}: può assumere uno tra i seguenti valori:
	\begin{itemize}
		\item O: requisito obbligatorio;
		\item D: requisito desiderabile;
		\item F: requisito facoltativo.
	\end{itemize}
	\item \textbf{Tipo}: può assumere uno tra i seguenti valori:
	\begin{itemize}
		\item F: funzionale;
		\item P: prestazionale;
		\item Q: qualità;
		\item V: vincolo.
	\end{itemize}
	\item \textbf{Codice}: è il codice gerarchico univoco di ogni requisito espresso in numeri.
\end{itemize} 
Per ogni requisito inoltre verranno riportate:
\begin{itemize}
	\item \textbf{Descrizione}: breve testo ma completo che andrà a descrivere il requisito in esame;
	\item \textbf{Fonte}: che potrà essere una tra le seguenti:
	\begin{itemize}
		\item Capitolato C5: requisito dedotto direttamente dall'analisi del capitolato C5;
		\item Verbale 2015-12-03: requisito derivato dal suddetto verbale;
		\item Verbale 2015-12-11: requisito derivato dal suddetto verbale;
		\item Verbale 2015-12-29: requisito derivato dal suddetto verbale;
		\item Verbale 2016-01-11: requisito derivato dal suddetto verbale;
		\item Verbale 2016-01-12: requisito derivato dal suddetto verbale;
		\item Interno: requisito identificato dagli Analisti;
		\item Caso d’uso: si tratta di un requisito emerso da un caso d’uso; viene riportato l’identificativo del caso d’uso associato.
	\end{itemize} 
\end{itemize}
%\begin{itemize}
\end{itemize}
\subsection{Riepilogo Requisiti}
\normalsize
\begin{longtable}{|c|c|c|c|}
\hline 
\textbf{Tipo} & \textbf{Obbligatorio} & \textbf{Desiderabile} & \textbf{Facoltativo}\\
\hline
Funzionale & 54 & 185 & 12\\ \hline
Prestazionale & 0 & 0 & 0\\ \hline
Di Qualità & 8 & 0 & 1\\ \hline
Di Vincolo & 10 & 5 & 7\\ \hline
\caption[Riepilogo Requisiti]{Riepilogo Requisiti}
\label{tabella:riepilogorequi}
\end{longtable}
\clearpage

\subsection{Requisiti Funzionali}
\normalsize
\begin{longtable}{|c|>{\centering}m{7cm}|c|}
\hline
\textbf{Id Requisito} & \textbf{Descrizione} & \textbf{Fonti}\\
\hline
\endhead \hypertarget{{RFO1}}{{RFO1}} & L’utente non autenticato può registrarsi & \makecell{Interno\\ UC2 } \\ \hline
			 \hypertarget{{RFO1.1}}{{RFO1.1}} & L’utente non autenticato può inserire il proprio nome & \makecell{Interno\\ UC2 \\UC2.1 } \\ \hline
			 \hypertarget{{RFO1.2}}{{RFO1.2}} & L’utente non autenticato può inserire il proprio cognome & \makecell{Interno\\ UC2 \\UC2.2 } \\ \hline
			 \hypertarget{{RFO1.3}}{{RFO1.3}} & L’utente non autenticato può inserire il proprio nome utente & \makecell{Interno\\ UC2 \\UC2.3 } \\ \hline
			 \hypertarget{{RFO1.4}}{{RFO1.4}} & L’utente non autenticato può inserire la propria email & \makecell{Interno\\ UC2 \\UC2.4 } \\ \hline
			 \hypertarget{{RFO1.5}}{{RFO1.5}} & L’utente non autenticato può inserire una password & \makecell{Interno\\ UC2 \\UC2.5 } \\ \hline
			 \hypertarget{{RFO1.6}}{{RFO1.6}} & L’utente non autenticato può confermare la propria password & \makecell{Interno\\ UC2 \\UC2.6 } \\ \hline
			 \hypertarget{{RFO1.7}}{{RFO1.7}} & L’utente non autenticato può confermare la registrazione & \makecell{Interno\\ UC2 \\UC2.7 } \\ \hline
			 \hypertarget{{RFO1.8}}{{RFO1.8}} & Il sistema deve mostrare un messaggio d’errore di registrazione all’utente non registrato se i dati non sono stati immessi correttamente o sono assenti & \makecell{Interno\\ UC2.8 } \\ \hline
			 \hypertarget{{RFO2}}{{RFO2}} & L’utente non autenticato può effettuare il login & \makecell{Verbale interno\\ UC3 } \\ \hline
			 \hypertarget{{RFO2.1}}{{RFO2.1}} & L'utente non autenticato può fare il login mediante le credenziali di QuizziPedia & \makecell{Verbale interno\\ UC3 \\UC3.1 } \\ \hline
			 \hypertarget{{RFO2.1.1}}{{RFO2.1.1}} & L'utente non autenticato può inserire il nome utente o la mail usata durante la fase di registrazione & \makecell{Interno\\ UC3.1 \\UC3.1.1 } \\ \hline
			 \hypertarget{{RFO2.1.2}}{{RFO2.1.2}} & L'utente non autenticato può inserire la password associata al nome utente o alla mail usata durante la fase di registrazione & \makecell{Interno\\ UC3.1 \\UC3.1.2 } \\ \hline
			 \hypertarget{{RFO2.1.3}}{{RFO2.1.3}} & L'utente non autenticato può confermare il login & \makecell{Interno\\ UC3.1 \\UC3.1.3 } \\ \hline
			 \hypertarget{{RFO2.1.4}}{{RFO2.1.4}} & Il sistema deve visualizzare un messaggio di errore in caso di inserimento dei dati errato o assente & \makecell{Interno\\ UC3.1 \\UC3.1.4 } \\ \hline
			 \hypertarget{{RFF2.2}}{{RFF2.2}} & L'utente non autenticato può eseguire il login tramite Facebook & \makecell{Interno\\ UC3 \\UC3.2 } \\ \hline
			 \hypertarget{{RFF2.3}}{{RFF2.3}} & L'utente non autenticato può eseguire il login tramite Twitter & \makecell{Interno\\ UC3 \\UC3.3 } \\ \hline
			 \hypertarget{{RFF2.4}}{{RFF2.4}} & L'utente non autenticato può eseguire il login tramite Google+ & \makecell{Interno\\ UC3 \\UC3.4 } \\ \hline
			 \hypertarget{{RFF2.5}}{{RFF2.5}} & L'utente non autenticato può eseguire il login tramite LinkedIn & \makecell{Interno\\ UC3 \\UC3.5 } \\ \hline
			 \hypertarget{{RFO3}}{{RFO3}} & L’utente autenticato e l’utente autenticato pro possono effettuare il logout & \makecell{Interno\\ UC4 } \\ \hline
			 \hypertarget{{RFO3.1}}{{RFO3.1}} & Il sistema deve notificare all’utente autenticato o all’utente autenticato pro la disconnessione dall’area riservata & \makecell{Interno\\ UC4 \\UC4.1 } \\ \hline
			 \hypertarget{{RFD4}}{{RFD4}} & L’utente autenticato e l’utente autenticato pro possono gestire il proprio profilo & \makecell{Interno } \\ \hline
			 \hypertarget{{RFD4.1}}{{RFD4.1}} & L’utente autenticato e l’utente autenticato pro possono modificare il proprio nome  & \makecell{Interno\\ UC5 \\UC5.1 } \\ \hline
			 \hypertarget{{RFD4.1.1}}{{RFD4.1.1}} & L’utente autenticato e l’utente autenticato pro possono confermare le modifiche al proprio nome & \makecell{Interno\\ UC5.1 \\UC5.1.2 } \\ \hline
			 \hypertarget{{RFD4.1.2}}{{RFD4.1.2}} & Il sistema deve visualizzare un messaggio di errore nel caso l’utente autenticato o l’utente autenticato pro inseriscano non correttamente il nuovo nome & \makecell{Interno\\ UC5.1 \\UC5.1.3 } \\ \hline
			 \hypertarget{{RFD4.2}}{{RFD4.2}} & L’utente autenticato e l’utente autenticato pro possono modificare il proprio cognome  & \makecell{Interno\\ UC5 \\UC5.2 } \\ \hline
			 \hypertarget{{RFD4.2.1}}{{RFD4.2.1}} & L’utente autenticato e l’utente autenticato pro possono confermare le modifiche al proprio cognome & \makecell{Interno\\ UC5.2 \\UC5.2.2 } \\ \hline
			 \hypertarget{{RFD4.2.2}}{{RFD4.2.2}} & Il sistema deve visualizzare un messaggio di errore nel caso l’utente autenticato o l’utente autenticato pro inseriscano non correttamente il nuovo cognome & \makecell{Interno\\ UC5.2 \\UC5.2.3 } \\ \hline
			 \hypertarget{{RFD4.3}}{{RFD4.3}} & L’utente autenticato e l’utente autenticato pro possono modificare il proprio username & \makecell{Interno\\ UC5 \\UC5.3 } \\ \hline
			 \hypertarget{{RFD4.3.1}}{{RFD4.3.1}} &  L’utente autenticato e l’utente autenticato pro possono confermare le modifiche al proprio username & \makecell{Interno\\ UC5.3 \\UC5.3.2 } \\ \hline
			 \hypertarget{{RFD4.3.2}}{{RFD4.3.2}} & Il sistema deve visualizzare un messaggio di errore nel caso l’utente autenticato o l’utente autenticato pro inseriscano un username già presente & \makecell{Interno\\ UC5.3 \\UC5.3.3 } \\ \hline
			 \hypertarget{{RFD4.4}}{{RFD4.4}} & L’utente autenticato e l’utente autenticato pro possono inserire una foto/immagine  & \makecell{Interno\\ UC5 \\UC5.4 } \\ \hline
			 \hypertarget{{RFD4.4.1}}{{RFD4.4.1}} & L’utente autenticato e l’utente autenticato pro possono confermare l’inserimento della foto/immagine & \makecell{Interno\\ UC5.4 \\UC5.4.2 } \\ \hline
			 \hypertarget{{RFD4.4.2}}{{RFD4.4.2}} & Il sistema deve visualizzare un messaggio di errore nel caso l’utente autenticato o l’utente autenticato pro inseriscano non correttamente la nuova foto/immagine & \makecell{Interno\\ UC5.4 \\UC5.4.3 } \\ \hline
			 \hypertarget{{RFD4.5}}{{RFD4.5}} & L’utente autenticato e l’utente autenticato pro possono modificare la propria e-mail  & \makecell{Interno\\ UC5 \\UC5.5 } \\ \hline
			 \hypertarget{{RFD4.5.1}}{{RFD4.5.1}} & L’utente autenticato e l’utente autenticato pro possono confermare le modifiche alla propria e-mail  & \makecell{Interno\\ UC5.5 \\UC5.5.2 } \\ \hline
			 \hypertarget{{RFD4.5.2}}{{RFD4.5.2}} & Il sistema deve visualizzare un messaggio di errore nel caso l’utente autenticato o l’utente autenticato pro inseriscano non correttamente la nuova mail & \makecell{Interno\\ UC5.5 \\UC5.5.3 } \\ \hline
			 \hypertarget{{RFD4.6}}{{RFD4.6}} & L’utente autenticato e l’utente autenticato pro possono modificare la propria password & \makecell{Interno\\ UC5 \\UC5.6 } \\ \hline
			 \hypertarget{{RFD4.6.1}}{{RFD4.6.1}} & L’utente autenticato e l’utente autenticato pro possono confermare le modifiche alla propria password  & \makecell{Interno\\ UC5.6 \\UC5.6.4 } \\ \hline
			 \hypertarget{{RFD4.6.2}}{{RFD4.6.2}} & Il sistema deve visualizzare un messaggio di errore nel caso l’utente autenticato o l’utente autenticato pro inseriscano non correttamente la nuova password & \makecell{Interno\\ UC5.6 \\UC5.6.5 } \\ \hline
			 \hypertarget{{RFD4.7}}{{RFD4.7}} & L’utente autenticato e l’utente autenticato pro possono cambiare la propria tipologia di account & \makecell{Interno\\ UC5 \\UC5.7 } \\ \hline
			 \hypertarget{{RFD4.7.1}}{{RFD4.7.1}} & L’utente autenticato e l’utente autenticato pro possono confermare la nuova tipologia di account  & \makecell{Interno\\ UC5.7 \\UC5.7.2 } \\ \hline
			 \hypertarget{{RFD4.8}}{{RFD4.8}} & L’utente autenticato e l’utente autenticato pro possono eliminare il proprio account  & \makecell{Interno\\ UC5 \\UC5.8 } \\ \hline
			 \hypertarget{{RFD4.8.1}}{{RFD4.8.1}} & L’utente autenticato e l’utente autenticato pro possono confermare l’eliminazione del proprio account & \makecell{Interno\\ UC5.8 \\UC5.8.1 } \\ \hline
			 \hypertarget{{RFD5}}{{RFD5}} & L’utente autenticato e l’utente autenticato pro possono effettuare una ricerca per l’iscrizione ad un questionario & \makecell{Interno\\ UC6 } \\ \hline
			 \hypertarget{{RFD5.1}}{{RFD5.1}} & L’utente autenticato e l’utente autenticato pro possono cercare un questionario tramite la barra di ricerca & \makecell{Interno\\ UC6 \\UC6.1 } \\ \hline
			 \hypertarget{{RFD5.2}}{{RFD5.2}} & L’utente autenticato e l’utente autenticato pro possono iscriversi ad un questionario & \makecell{Interno\\ UC6 \\UC6.2 } \\ \hline
			 \hypertarget{{RFD5.2.1}}{{RFD5.2.1}} & L’utente autenticato e l’utente autenticato pro possono confermare l’iscrizione ad un questionario & \makecell{Interno\\ UC6.2 \\UC6.2.1 } \\ \hline
			 \hypertarget{{RFD5.3}}{{RFD5.3}} & Il sistema deve visualizzare un messaggio di errore nel caso l’utente autenticato o l’utente autenticato pro ricerchino un questionario inesistente  & \makecell{Interno\\ UC6 \\UC6.3 } \\ \hline
			 \hypertarget{{RFO6}}{{RFO6}} & L’utente autenticato e l’utente autenticato pro possono compilare un questionario & \makecell{Capitolato\\ UC7 } \\ \hline
			 \hypertarget{{RFD6.1}}{{RFD6.1}} & L’utente autenticato e l’utente autenticato pro a partire da una domanda possono scegliere di spostarsi alla domanda successiva del questionario  & \makecell{Interno\\ UC7 \\UC7.1 } \\ \hline
			 \hypertarget{{RFD6.2}}{{RFD6.2}} & L’utente autenticato e l’utente autenticato pro a partire da una domanda possono scegliere di spostarsi alla domanda precedente del questionario & \makecell{Interno\\ UC7 \\UC7.2 } \\ \hline
			 \hypertarget{{RFD6.3}}{{RFD6.3}} & L’utente autenticato e l’utente autenticato pro a partire da una domanda possono scegliere di spostarsi ad una qualsiasi altra domanda presente nel questionario & \makecell{Interno\\ UC7 \\UC7.3 } \\ \hline
			 \hypertarget{{RFD6.4}}{{RFD6.4}} & L’utente autenticato e l’utente autenticato pro possono concludere il questionario confermando le risposte date alle domande che lo compongono  & \makecell{Interno\\ UC7 \\UC7.4 } \\ \hline
			 \hypertarget{{RFO6.5}}{{RFO6.5}} & Il sistema deve valutare le risposte date dagli utilizzatori & \makecell{Capitolato } \\ \hline
			 \hypertarget{{RFO7}}{{RFO7}} & L’utente autenticato e l’utente autenticato pro possono gestire le domande che hanno creato & \makecell{Capitolato } \\ \hline
			 \hypertarget{{RFO7.1}}{{RFO7.1}} & L’utente autenticato e l’utente autenticato pro possono creare una domanda  & \makecell{Capitolato\\ UC8 \\UC8.1 } \\ \hline
			 \hypertarget{{RFD7.1.1}}{{RFD7.1.1}} & L’utente autenticato e l’utente autenticato pro possono scegliere un argomento da assegnare alla nuova domanda  & \makecell{Interno\\ UC8.1 \\UC8.1.1 } \\ \hline
			 \hypertarget{{RFD7.1.2}}{{RFD7.1.2}} & L’utente autenticato e l’utente autenticato pro possono inserire delle parole chiave relative alla nuova domanda & \makecell{Interno\\ UC8.1 \\UC8.1.2 } \\ \hline
			 \hypertarget{{RFO7.1.3}}{{RFO7.1.3}} & L’utente autenticato e l’utente autenticato pro possono scegliere la tipologia  di domanda da creare  & \makecell{Interno\\ UC8.1 \\UC8.1.3 } \\ \hline
			 \hypertarget{{RFO7.1.3.1}}{{RFO7.1.3.1}} & L’utente autenticato e l’utente autenticato pro possono scegliere di creare una domanda vero o falso & \makecell{Capitolato\\ UC8.1.3 \\UC8.1.3.1 } \\ \hline
			 \hypertarget{{RFO7.1.3.1.1}}{{RFO7.1.3.1.1}} & L’utente autenticato e l'utente autenticato pro possono aggiungere il testo della domanda vero o falso & \makecell{Capitolato\\ UC8.1.3.1 \\UC8.1.3.1.1 } \\ \hline
			 \hypertarget{{RFD7.1.3.1.2}}{{RFD7.1.3.1.2}} & L’utente autenticato e l'utente autenticato pro possono inserire un'immagine relativa al testo della domanda vero o falso & \makecell{Capitolato\\ UC8.1.3.1 \\UC8.1.3.1.2 } \\ \hline
			 \hypertarget{{RFD7.1.3.1.2.1}}{{RFD7.1.3.1.2.1}} & L’utente autenticato e l'utente autenticato pro possono eliminare l’immagine, relativa al testo della domanda vero o falso, appena inserita & \makecell{Interno\\ UC8.1.3.1.2 \\UC8.1.3.1.2.1 } \\ \hline
			 \hypertarget{{RFO7.1.3.1.3}}{{RFO7.1.3.1.3}} & L’utente autenticato e l'utente autenticato pro possono selezionare la risposta corretta  & \makecell{Capitolato\\ UC8.1.3.1 \\UC8.1.3.1.3 } \\ \hline
			 \hypertarget{{RFO7.1.3.2}}{{RFO7.1.3.2}} & L’utente autenticato e l’utente autenticato pro possono scegliere di creare una domanda a risposta multipla & \makecell{Capitolato\\ UC8.1.3 \\UC8.1.3.2 } \\ \hline
			 \hypertarget{{RFO7.1.3.2.1}}{{RFO7.1.3.2.1}} & L’utente autenticato e l'utente autenticato pro possono aggiungere il testo della domanda a risposta multipla  & \makecell{Capitolato\\ UC8.1.3.1.2 \\UC8.1.3.1.2.1 } \\ \hline
			 \hypertarget{{RFD7.1.3.2.2}}{{RFD7.1.3.2.2}} & L’utente autenticato e l'utente autenticato pro possono inserire un'immagine relativa al testo della domanda a risposta multipla  & \makecell{Capitolato\\ UC8.1.3.2 \\UC8.1.3.2.2 } \\ \hline
			 \hypertarget{{RFD7.1.3.2.2.1}}{{RFD7.1.3.2.2.1}} & L’utente autenticato e l'utente autenticato pro possono eliminare l’immagine, relativa al testo della domanda a risposta multipla, appena inserita & \makecell{Interno\\ UC8.1.3.2.2 \\UC8.1.3.2.2.1 } \\ \hline
			 \hypertarget{{RFO7.1.3.2.3}}{{RFO7.1.3.2.3}} & L’utente autenticato e l'utente autenticato pro possono aggiungere due o più opzioni di risposta & \makecell{Capitolato\\ UC8.1.3.2 \\UC8.1.3.2.3 } \\ \hline
			 \hypertarget{{RFO7.1.3.2.3.1}}{{RFO7.1.3.2.3.1}} & L’utente autenticato e l'utente autenticato pro possono aggiungere due o più opzioni di risposta che includono testo & \makecell{Capitolato\\ UC8.1.3.2.3 \\UC8.1.3.2.3.1 } \\ \hline
			 \hypertarget{{RFD7.1.3.2.3.1.1}}{{RFD7.1.3.2.3.1.1}} & L’utente autenticato e l'utente autenticato pro possono eliminare le opzioni di risposta che includono testo & \makecell{Interno\\ UC8.1.3.2.3.1 \\UC8.1.3.2.3.1.1 } \\ \hline
			 \hypertarget{{RFD7.1.3.2.3.2}}{{RFD7.1.3.2.3.2}} & L’utente autenticato e l'utente autenticato pro possono aggiungere due o più opzioni di risposta che includono immagini & \makecell{Capitolato\\ UC8.1.3.2.3 \\UC8.1.3.2.3.2 } \\ \hline
			 \hypertarget{{RFD7.1.3.2.3.2.1}}{{RFD7.1.3.2.3.2.1}} & L’utente autenticato e l'utente autenticato pro possono eliminare le opzioni di risposta che includono immagini  & \makecell{Interno\\ UC8.1.3.2.3.2 \\UC8.1.3.2.3.2.1 } \\ \hline
			 \hypertarget{{RFO7.1.3.2.4}}{{RFO7.1.3.2.4}} & L’utente autenticato e l'utente autenticato pro possono selezionare una o più risposte corrette & \makecell{Capitolato\\ UC8.1.3.2 \\UC8.1.3.2.4 } \\ \hline
			 \hypertarget{{RFD7.1.3.3}}{{RFD7.1.3.3}} & L’utente autenticato e l’utente autenticato pro possono scegliere di creare un esercizio di riempimento degli spazi vuoti  & \makecell{Capitolato\\ UC8.1.3 \\UC8.1.3.3 } \\ \hline
			 \hypertarget{{RFD7.1.3.3.1}}{{RFD7.1.3.3.1}} & L’utente autenticato e l'utente autenticato pro possono scrivere il testo dell’esercizio & \makecell{Capitolato\\ UC8.1.3.3 \\UC8.1.3.3.1 } \\ \hline
			 \hypertarget{{RFD7.1.3.3.2}}{{RFD7.1.3.3.2}} & L’utente autenticato e l'utente autenticato pro possono indicare le parole da oscurare & \makecell{Capitolato\\ UC8.1.3.3 \\UC8.1.3.3.2 } \\ \hline
			 \hypertarget{{RFD7.1.3.4}}{{RFD7.1.3.4}} & L’utente autenticato e l’utente autenticato pro possono scegliere di creare una domanda di collegamento & \makecell{Capitolato\\ UC8.1.3 \\UC8.1.3.4 } \\ \hline
			 \hypertarget{{RFD7.1.3.4.1}}{{RFD7.1.3.4.1}} & L’utente autenticato e l'utente autenticato pro possono inserire il testo della domanda & \makecell{Capitolato\\ UC8.1.3.4 \\UC8.1.3.4.1 } \\ \hline
			 \hypertarget{{RFD7.1.3.4.2}}{{RFD7.1.3.4.2}} & L’utente autenticato e l'utente autenticato pro possono inserire una o più coppie di elementi & \makecell{Capitolato\\ UC8.1.3.4 \\UC8.1.3.4.2 } \\ \hline
			 \hypertarget{{RFD7.1.3.4.2.1}}{{RFD7.1.3.4.2.1}} & L’utente autenticato e l'utente autenticato pro possono inserire un’immagine come primo elemento & \makecell{Capitolato\\ UC8.1.3.4.2 \\UC8.1.3.4.2.1 } \\ \hline
			 \hypertarget{{RFD7.1.3.4.2.2}}{{RFD7.1.3.4.2.2}} & L’utente autenticato e l'utente autenticato pro possono inserire un testo come primo elemento & \makecell{Capitolato\\ UC8.1.3.4.2 \\UC8.1.3.4.2.2 } \\ \hline
			 \hypertarget{{RFD7.1.3.4.2.3}}{{RFD7.1.3.4.2.3}} & L’utente autenticato e l'utente autenticato pro possono inserire un’immagine come secondo elemento & \makecell{Capitolato\\ UC8.1.3.4.2 \\UC8.1.3.4.2.3 } \\ \hline
			 \hypertarget{{RFD7.1.3.4.2.4}}{{RFD7.1.3.4.2.4}} & L’utente autenticato e l'utente autenticato pro possono inserire un testo come secondo elemento & \makecell{Capitolato\\ UC8.1.3.4.2 \\UC8.1.3.4.2.4 } \\ \hline
			 \hypertarget{{RFD7.1.3.4.3}}{{RFD7.1.3.4.3}} & L’utente autenticato e l'utente autenticato pro possono eliminare una o più coppie di elementi  & \makecell{Interno\\ UC8.1.3.4 \\UC8.1.3.4.3 } \\ \hline
			 \hypertarget{{RFD7.1.3.4.3.1}}{{RFD7.1.3.4.3.1}} & L’utente autenticato e l'utente autenticato pro possono confermare l'eliminazione di una coppia di elementi & \makecell{Interno\\ UC8.1.3.4.3 \\UC8.1.3.4.3.1 } \\ \hline
			 \hypertarget{{RFD7.1.3.4.4}}{{RFD7.1.3.4.4}} & L’utente autenticato e l'utente autenticato pro possono modificare una o più coppie di elementi & \makecell{Interno\\ UC8.1.3.4 \\UC8.1.3.4.4 } \\ \hline
			 \hypertarget{{RFD7.1.3.4.4.1}}{{RFD7.1.3.4.4.1}} & L’utente autenticato e l'utente autenticato pro possono modificare il testo di un elemento & \makecell{Interno\\ UC8.1.3.4.4 \\UC8.1.3.4.4.1 } \\ \hline
			 \hypertarget{{RFD7.1.3.4.4.2}}{{RFD7.1.3.4.4.2}} & L’utente autenticato e l'utente autenticato pro possono modificare l’immagine di un elemento & \makecell{Interno\\ UC8.1.3.4.4 \\UC8.1.3.4.4.2 } \\ \hline
			 \hypertarget{{RFD7.1.3.4.4.3}}{{RFD7.1.3.4.4.3}} & L’utente autenticato e l'utente autenticato pro possono cambiare il testo di un elemento di una coppia in un’immagine & \makecell{Interno\\ UC8.1.3.4.4 \\UC8.1.3.4.4.3 } \\ \hline
			 \hypertarget{{RFD7.1.3.4.4.4}}{{RFD7.1.3.4.4.4}} & L’utente autenticato e l'utente autenticato pro possono cambiare l’immagine di un elemento di una coppia in testo & \makecell{Interno\\ UC8.1.3.4.4 \\UC8.1.3.4.4.4 } \\ \hline
			 \hypertarget{{RFD7.1.3.5}}{{RFD7.1.3.5}} & L’utente autenticato e l’utente autenticato pro possono scegliere di creare una domanda a ordinamento di immagini  & \makecell{Capitolato\\ UC8.1.3 \\UC8.1.3.5 } \\ \hline
			 \hypertarget{{RFD7.1.3.5.1}}{{RFD7.1.3.5.1}} & L’utente autenticato e l’utente autenticato pro possono inserire il testo della domanda & \makecell{Capitolato\\ UC8.1.3.5 \\UC8.1.3.5.1 } \\ \hline
			 \hypertarget{{RFD7.1.3.5.2}}{{RFD7.1.3.5.2}} & L’utente autenticato e l’utente autenticato pro possono inserire un'immagine per il testo della domanda & \makecell{Capitolato\\ UC8.1.3.5 \\UC8.1.3.5.2 } \\ \hline
			 \hypertarget{{RFD7.1.3.5.2.1}}{{RFD7.1.3.5.2.1}} & L’utente autenticato e l’utente autenticato pro possono eliminare un'immagine relativa al testo della domanda  & \makecell{Interno\\ UC8.1.3.5.2 \\UC8.1.3.5.2.1 } \\ \hline
			 \hypertarget{{RFD7.1.3.5.3}}{{RFD7.1.3.5.3}} & L’utente autenticato e l’utente autenticato pro possono inserire immagini come risposta & \makecell{Capitolato\\ UC8.1.3.5 \\UC8.1.3.5.3 } \\ \hline
			 \hypertarget{{RFD7.1.3.5.3.1}}{{RFD7.1.3.5.3.1}} & L’utente autenticato e l’utente autenticato pro possono eliminare l’immagine di una risposta & \makecell{Interno\\ UC8.1.3.5.3 \\UC8.1.3.5.3.1 } \\ \hline
			 \hypertarget{{RFD7.1.3.6}}{{RFD7.1.3.6}} & L’utente autenticato e l’utente autenticato pro possono scegliere di creare una domanda a ordinamento di stringhe & \makecell{Capitolato\\ UC8.1.3 \\UC8.1.3.6 } \\ \hline
			 \hypertarget{{RFD7.1.3.6.1}}{{RFD7.1.3.6.1}} & L’utente autenticato e l’utente autenticato pro possono inserire il testo della domanda a ordinamento di stringhe & \makecell{Capitolato\\ UC8.1.3.6 \\UC8.1.3.6.1 } \\ \hline
			 \hypertarget{{RFD7.1.3.6.2}}{{RFD7.1.3.6.2}} & L’utente autenticato e l’utente autenticato pro possono inserire stringhe di composizione sequenza & \makecell{Capitolato\\ UC8.1.3.6 \\UC8.1.3.6.2 } \\ \hline
			 \hypertarget{{RFD7.1.3.6.3}}{{RFD7.1.3.6.3}} & L’utente autenticato e l’utente autenticato pro possono comporre la soluzione della sequenza & \makecell{Capitolato\\ UC8.1.3.6 \\UC8.1.3.6.3 } \\ \hline
			 \hypertarget{{RFD7.1.3.7}}{{RFD7.1.3.7}} & L’utente autenticato e l’utente autenticato pro possono creare una domanda con area cliccabile nell’immagine & \makecell{Capitolato\\ UC8.1.3 \\UC8.1.3.7 } \\ \hline
			 \hypertarget{{RFD7.1.3.7.1}}{{RFD7.1.3.7.1}} & L’utente autenticato e l’utente autenticato pro possono inserire il testo della domanda & \makecell{Capitolato\\ UC8.1.3.7 \\UC8.1.3.7.1 } \\ \hline
			 \hypertarget{{RFD7.1.3.7.2}}{{RFD7.1.3.7.2}} & L’utente autenticato e l’utente autenticato pro possono inserire un’immagine & \makecell{Capitolato\\ UC8.1.3.7 \\UC8.1.3.7.2 } \\ \hline
			 \hypertarget{{RFD7.1.3.7.3}}{{RFD7.1.3.7.3}} & L’utente autenticato e l’utente autenticato pro possono scegliere un numero di aree selezionabili & \makecell{Capitolato\\ UC8.1.3.7 \\UC8.1.3.7.3 } \\ \hline
			 \hypertarget{{RFD7.1.3.7.4}}{{RFD7.1.3.7.4}} & L’utente autenticato e l’utente autenticato pro possono scegliere le aree  selezionabili & \makecell{Capitolato\\ UC8.1.3.7 \\UC8.1.3.7.4 } \\ \hline
			 \hypertarget{{RFO7.1.4}}{{RFO7.1.4}} & L’utente autenticato e l'utente autenticato pro possono confermare la creazione della domanda & \makecell{Capitolato\\ UC8.1 \\UC8.1.4 } \\ \hline
			 \hypertarget{{RFO7.1.5}}{{RFO7.1.5}} & Il sistema deve mostrare un messaggio di errore in caso la conferma della creazione della domanda non sia andata a buon fine & \makecell{Capitolato\\ UC8.1 \\UC8.1.5 } \\ \hline
			 \hypertarget{{RFD7.2}}{{RFD7.2}} & L’utente autenticato e l’utente autenticato pro possono modificare una domanda & \makecell{Interno\\ UC8.2 } \\ \hline
			 \hypertarget{{RFD7.2.1}}{{RFD7.2.1}} & L’utente autenticato e l'utente autenticato pro possono selezionare una domanda da modificare & \makecell{Interno\\ UC8.2 \\UC8.2.1 } \\ \hline
			 \hypertarget{{RFD7.2.1.1}}{{RFD7.2.1.1}} & L’utente autenticato e l’utente autenticato pro possono scegliere di modificare una domanda vero o falso & \makecell{Interno\\ UC8.2.1 \\UC8.2.1.1 } \\ \hline
			 \hypertarget{{RFD7.2.1.1.1}}{{RFD7.2.1.1.1}} & L’utente autenticato e l'utente autenticato pro possono modificare il testo della domanda vero o falso & \makecell{Interno\\ UC8.2.1.1 \\UC8.2.1.1.1 } \\ \hline
			 \hypertarget{{RFD7.2.1.1.2}}{{RFD7.2.1.1.2}} & L’utente autenticato e l'utente autenticato pro possono modificare l’immagine della domanda vero o falso & \makecell{Interno\\ UC8.2.1.1 \\UC8.2.1.1.2 } \\ \hline
			 \hypertarget{{RFD7.2.1.1.3}}{{RFD7.2.1.1.3}} & L’utente autenticato e l'utente autenticato pro possono modificare la risposta corretta della domanda vero o falso & \makecell{Interno\\ UC8.2.1.1 \\UC8.2.1.1.3 } \\ \hline
			 \hypertarget{{RFD7.2.1.2}}{{RFD7.2.1.2}} & L’utente autenticato e l’utente autenticato pro possono scegliere di modificare una domanda a risposta multipla & \makecell{Interno\\ UC8.2.1.2 } \\ \hline
			 \hypertarget{{RFD7.2.1.2.1}}{{RFD7.2.1.2.1}} & L’utente autenticato e l'utente autenticato pro possono modificare il testo della domanda a risposta multipla & \makecell{Interno\\ UC8.2.1.2 \\UC8.2.1.2.1 } \\ \hline
			 \hypertarget{{RFD7.2.1.2.2}}{{RFD7.2.1.2.2}} & L’utente autenticato e l'utente autenticato pro possono modificare l’immagine della domanda a risposta multipla & \makecell{Interno\\ UC8.2.1.2 \\UC8.2.1.2.2 } \\ \hline
			 \hypertarget{{RFD7.2.1.2.3}}{{RFD7.2.1.2.3}} & L’utente autenticato e l'utente autenticato pro possono modificare l’opzione di risposta della domanda a risposta multipla & \makecell{Interno\\ UC8.2.1.2 \\UC8.2.1.2.3 } \\ \hline
			 \hypertarget{{RFD7.2.1.2.3.1}}{{RFD7.2.1.2.3.1}} & L’utente autenticato e l'utente autenticato pro possono modificare l’opzione di risposta che include del testo & \makecell{Interno\\ UC8.2.1.2.3 \\UC8.2.1.2.3.1 } \\ \hline
			 \hypertarget{{RFD7.2.1.2.3.2}}{{RFD7.2.1.2.3.2}} & L’utente autenticato e l'utente autenticato pro possono modificare l’opzione di risposta che include delle immagini & \makecell{Interno\\ UC8.2.1.2.3 \\UC8.2.1.2.3.1 } \\ \hline
			 \hypertarget{{RFD7.2.1.2.4}}{{RFD7.2.1.2.4}} & L’utente autenticato e l’utente autenticato pro possono modificare la risposta corretta o le risposte corrette della domanda a risposta multipla & \makecell{Interno\\ UC8.2.1.2 \\UC8.2.1.2.4 } \\ \hline
			 \hypertarget{{RFD7.2.1.3}}{{RFD7.2.1.3}} & L’utente autenticato e l’utente autenticato pro possono scegliere di modificare una domanda a riempimento di spazi vuoti & \makecell{Interno\\ UC8.2.1.3 } \\ \hline
			 \hypertarget{{RFD7.2.1.3.1}}{{RFD7.2.1.3.1}} & L’utente autenticato e l'utente autenticato pro possono modificare il testo della domanda a riempimento di spazi & \makecell{Interno\\ UC8.2.1.3 \\UC8.2.1.3.1 } \\ \hline
			 \hypertarget{{RFD7.2.1.3.2}}{{RFD7.2.1.3.2}} & L’utente autenticato e l'utente autenticato pro possono modificare le parole da oscurare della domanda a riempimento di spazi  & \makecell{Interno\\ UC8.2.1.3 \\UC8.2.1.3.2 } \\ \hline
			 \hypertarget{{RFD7.2.1.4}}{{RFD7.2.1.4}} & L’utente autenticato e l’utente autenticato pro possono scegliere di modificare una domanda di collegamento & \makecell{Interno\\ UC8.2.1 \\UC8.2.1.4 } \\ \hline
			 \hypertarget{{RFD7.2.1.4.1}}{{RFD7.2.1.4.1}} & L’utente autenticato e l'utente autenticato pro possono inserire una nuova coppia di elementi da collegare & \makecell{Interno\\ UC8.2.1.4 \\UC8.2.1.4.1 } \\ \hline
			 \hypertarget{{RFD7.2.1.4.1.1}}{{RFD7.2.1.4.1.1}} & L’utente autenticato e l'utente autenticato pro possono inserire un’immagine come primo elemento della domanda & \makecell{Interno\\ UC8.2.1.4.1 \\UC8.2.1.4.1.1 } \\ \hline
			 \hypertarget{{RFD7.2.1.4.1.2}}{{RFD7.2.1.4.1.2}} & L’utente autenticato e l'utente autenticato pro possono inserire un testo come primo elemento della domanda & \makecell{Interno\\ UC8.2.1.4.1 \\UC8.2.1.4.1.2 } \\ \hline
			 \hypertarget{{RFD7.2.1.4.1.3}}{{RFD7.2.1.4.1.3}} & L’utente autenticato e l'utente autenticato pro possono inserire un’immagine come secondo elemento della domanda  & \makecell{Interno\\ UC8.2.1.4.1 \\UC8.2.1.4.1.3 } \\ \hline
			 \hypertarget{{RFD7.2.1.4.1.4}}{{RFD7.2.1.4.1.4}} & L’utente autenticato e l'utente autenticato pro possono inserire un testo come secondo elemento della domanda  & \makecell{Interno\\ UC8.2.1.4.1 \\UC8.2.1.4.1.4 } \\ \hline
			 \hypertarget{{RFD7.2.1.4.2}}{{RFD7.2.1.4.2}} & L’utente autenticato e l'utente autenticato pro possono eliminare una coppia di elementi da collegare & \makecell{Interno\\ UC8.2.1.4 \\UC8.2.1.4.2 } \\ \hline
			 \hypertarget{{RFD7.2.1.4.2.1}}{{RFD7.2.1.4.2.1}} & L’utente autenticato e l'utente autenticato pro possono confermare di eliminare una coppia di elementi da collegare & \makecell{Interno\\ UC8.2.1.4.2 \\UC8.2.1.4.2.1 } \\ \hline
			 \hypertarget{{RFD7.2.1.4.3}}{{RFD7.2.1.4.3}} & L’utente autenticato e l'utente autenticato pro possono modificare una coppia di elementi da collegare 
 & \makecell{Interno\\ UC8.2.1.4 \\UC8.2.1.4.2 } \\ \hline
			 \hypertarget{{RFD7.2.1.4.3.1}}{{RFD7.2.1.4.3.1}} & L’utente autenticato e l'utente autenticato pro possono modificare il testo di un elemento da collegare & \makecell{Interno\\ UC8.2.1.4.3 \\UC8.2.1.4.3.1 } \\ \hline
			 \hypertarget{{RFD7.2.1.4.3.2}}{{RFD7.2.1.4.3.2}} & L’utente autenticato e l'utente autenticato pro possono modificare un’immagine di un elemento da collegare & \makecell{Interno\\ UC8.2.1.4.3 \\UC8.2.1.4.3.2 } \\ \hline
			 \hypertarget{{RFD7.2.1.4.3.3}}{{RFD7.2.1.4.3.3}} & L’utente autenticato e l'utente autenticato pro possono cambiare il testo di un elemento da collegare in un'immagine & \makecell{Interno\\ UC8.2.1.4.3 \\UC8.2.1.4.3.3 } \\ \hline
			 \hypertarget{{RFD7.2.1.4.3.4}}{{RFD7.2.1.4.3.4}} & L’utente autenticato e l'utente autenticato pro possono cambiare l’immagine di un elemento da collegare in un testo & \makecell{Interno\\ UC8.2.1.4.3 \\UC8.2.1.4.3.4 } \\ \hline
			 \hypertarget{{RFD7.2.1.4.4}}{{RFD7.2.1.4.4}} & L’utente autenticato e l'utente autenticato pro possono eliminare una coppia di elementi & \makecell{Interno\\ UC8.2.1.4 \\UC8.2.1.4.2 } \\ \hline
			 \hypertarget{{RFD7.2.1.4.5}}{{RFD7.2.1.4.5}} & L’utente autenticato e l'utente autenticato pro possono eliminare una coppia di elementi da collegare & \makecell{Interno\\ UC8.2.1.4 \\UC8.2.1.4.3 } \\ \hline
			 \hypertarget{{RFD7.2.1.4.6}}{{RFD7.2.1.4.6}} & L’utente autenticato e l'utente autenticato pro possono modificare il testo della domanda a coppia di elementi da collegare  & \makecell{Interno\\ UC8.2.1.4 \\UC8.2.1.4.4 } \\ \hline
			 \hypertarget{{RFD7.2.1.5}}{{RFD7.2.1.5}} & L’utente autenticato e l’utente autenticato pro possono scegliere di modificare una domanda di ordinamento immagini & \makecell{Interno\\ UC8.2.1 \\UC8.2.1.5 } \\ \hline
			 \hypertarget{{RFD7.2.1.5.1}}{{RFD7.2.1.5.1}} & L’utente autenticato e l'utente autenticato pro possono modificare il testo della domanda a ordinamento di immagini & \makecell{Interno\\ UC8.2.1.5 \\UC8.2.1.5.1 } \\ \hline
			 \hypertarget{{RFD7.2.1.5.2}}{{RFD7.2.1.5.2}} & L’utente autenticato e l'utente autenticato pro possono modificare l’immagine della domanda a ordinamento di immagini  & \makecell{Interno\\ UC8.2.1.5 \\UC8.2.1.5.2 } \\ \hline
			 \hypertarget{{RFD7.2.1.5.3}}{{RFD7.2.1.5.3}} & L’utente autenticato e l'utente autenticato pro possono modificare le immagini della risposta della domanda a ordinamento di immagini & \makecell{Interno\\ UC8.2.1.5 \\UC8.2.1.5.3 } \\ \hline
			 \hypertarget{{RFD7.2.1.5.4}}{{RFD7.2.1.5.4}} & L’utente autenticato e l'utente autenticato pro possono modificare l’ordine delle immagini della risposta della domanda a ordinamento di immagini & \makecell{Interno\\ UC8.2.1.5 \\UC8.2.1.5.4 } \\ \hline
			 \hypertarget{{RFD7.2.1.6}}{{RFD7.2.1.6}} & L’utente autenticato e l’utente autenticato pro possono scegliere di modificare una domanda di ordinamento stringhe & \makecell{Interno\\ UC8.2.1 \\UC8.2.1.6 } \\ \hline
			 \hypertarget{{RFD7.2.1.6.1}}{{RFD7.2.1.6.1}} & L’utente autenticato e l'utente autenticato pro possono modificare il testo della domanda a ordinamento di stringhe & \makecell{Interno\\ UC8.2.1.6 \\UC8.2.1.6.1 } \\ \hline
			 \hypertarget{{RFD7.2.1.6.2}}{{RFD7.2.1.6.2}} & L’utente autenticato e l'utente autenticato pro possono modificare il testo della risposta e il numero di stringhe della domanda a ordinamento di stringhe & \makecell{Interno\\ UC8.2.1.6 \\UC8.2.1.6.2 } \\ \hline
			 \hypertarget{{RFD7.2.1.6.3}}{{RFD7.2.1.6.3}} & L’utente autenticato e l'utente autenticato pro possono modificare la soluzione della domanda a ordinamento di stringhe & \makecell{Interno\\ UC8.2.1.6 \\UC8.2.1.6.3 } \\ \hline
			 \hypertarget{{RFD7.2.1.7}}{{RFD7.2.1.7}} & L’utente autenticato e l’utente autenticato pro possono scegliere di modificare una domanda con immagine ad aree cliccabili & \makecell{Interno\\ UC8.2.1 \\UC8.2.1.7 } \\ \hline
			 \hypertarget{{RFD7.2.1.7.1}}{{RFD7.2.1.7.1}} & L’utente autenticato e l'utente autenticato pro possono modificare il testo della domanda ad immagine con aree selezionabili & \makecell{Interno\\ UC8.2.1.7 \\UC8.2.1.7.1 } \\ \hline
			 \hypertarget{{RFD7.2.1.7.2}}{{RFD7.2.1.7.2}} & L’utente autenticato e l'utente autenticato pro possono inserire una nuova immagine della domanda ad immagine con aree selezionabili & \makecell{Interno\\ UC8.2.1.7 \\UC8.2.1.7.2 } \\ \hline
			 \hypertarget{{RFD7.2.1.7.3}}{{RFD7.2.1.7.3}} & L’utente autenticato e l'utente autenticato pro possono modificare il numero della aree selezionabili dell’immagine della domanda ad immagine con aree selezionabili & \makecell{Interno\\ UC8.2.1.7 \\UC8.2.1.7.3 } \\ \hline
			 \hypertarget{{RFD7.2.1.7.4}}{{RFD7.2.1.7.4}} & L’utente autenticato e l'utente autenticato pro possono scegliere nuove aree selezionabili dell’immagine della domanda ad immagine con aree selezionabili & \makecell{Interno\\ UC8.2.1.7 \\UC8.2.1.7.4 } \\ \hline
			 \hypertarget{{RFD7.2.1.8}}{{RFD7.2.1.8}} & L’utente autenticato e l’utente autenticato pro possono modificare una domanda con immagine ad aree cliccabili & \makecell{Interno\\ UC8.2.1 \\UC8.2.1.7 } \\ \hline
			 \hypertarget{{RFD7.2.2}}{{RFD7.2.2}} & L’utente autenticato e l'utente autenticato pro possono confermare le modifiche apportate alla domanda & \makecell{Interno\\ UC8.2 \\UC8.2.2 } \\ \hline
			 \hypertarget{{RFD7.2.3}}{{RFD7.2.3}} & Il sistema deve mostrare un messaggio di errore in caso la conferma delle modifiche non sia andata a buon fine & \makecell{Interno\\ UC8.2 \\UC8.2.3 } \\ \hline
			 \hypertarget{{RFO8}}{{RFO8}} & L’utente autenticato pro può gestire i questionari che ha creato & \makecell{Verbale 2016-01-11\\ UC9 } \\ \hline
			 \hypertarget{{RFD8.1}}{{RFD8.1}} & L’utente autenticato pro può visualizzare i questionari creati & \makecell{Interno\\ UC9 \\UC9.1 } \\ \hline
			 \hypertarget{{RFD8.1.1}}{{RFD8.1.1}} & L’utente autenticato pro può modificare un questionario che ha creato & \makecell{Interno\\ UC9.1 \\UC9.1.1 } \\ \hline
			 \hypertarget{{RFD8.1.1.1}}{{RFD8.1.1.1}} & L’utente autenticato pro può modificare il nome di un questionario & \makecell{Interno\\ UC9.1.1 \\UC9.1.1.1 } \\ \hline
			 \hypertarget{{RFD8.1.1.2}}{{RFD8.1.1.2}} & L’utente autenticato pro può confermare le modifiche che ha effettuato nel questionario & \makecell{Interno\\ UC9.1.1 \\UC9.1.1.2 } \\ \hline
			 \hypertarget{{RFD8.1.2}}{{RFD8.1.2}} & L’utente autenticato pro può eliminare un questionario che ha creato & \makecell{Interno\\ UC9.1 \\UC9.1.2 } \\ \hline
			 \hypertarget{{RFD8.1.2.1}}{{RFD8.1.2.1}} & L’utente autenticato pro può confermare se eliminare un questionario che ha creato & \makecell{Interno\\ UC9.1.2 \\UC9.1.2.1 } \\ \hline
			 \hypertarget{{RFD8.1.3}}{{RFD8.1.3}} & L’utente autenticato pro può visualizzare i risultati degli esaminandi  & \makecell{Verbale 2016-01-11\\ UC9.1 \\UC9.1.3 } \\ \hline
			 \hypertarget{{RFD8.1.4}}{{RFD8.1.4}} & L’utente autenticato pro può rendere il questionario compilabile da parte degli esaminandi & \makecell{Interno\\ UC9.1 \\UC9.1.4 } \\ \hline
			 \hypertarget{{RFO8.2}}{{RFO8.2}} & L’utente autenticato pro può creare un nuovo questionario & \makecell{Capitolato\\ UC9.2 } \\ \hline
			 \hypertarget{{RFO8.2.1}}{{RFO8.2.1}} & L’utente autenticato pro può scegliere l’argomento del questionario & \makecell{Capitolato\\ UC9.2 \\UC9.2.1 } \\ \hline
			 \hypertarget{{RFD8.2.2}}{{RFD8.2.2}} & L’utente autenticato pro può scegliere delle parole chiave che identifichino il questionario & \makecell{Interno\\ UC9.2 \\UC9.2.2 } \\ \hline
			 \hypertarget{{RFD8.2.3}}{{RFD8.2.3}} & L’utente autenticato pro può inserire il nome del questionario & \makecell{Interno\\ UC9.2 \\UC9.2.3 } \\ \hline
			 \hypertarget{{RFO8.2.4}}{{RFO8.2.4}} & L’utente autenticato pro può concludere il questionario & \makecell{Interno\\ UC9.2 \\UC9.2.4 } \\ \hline
			 \hypertarget{{RFD8.2.4.1}}{{RFD8.2.4.1}} & L’utente autenticato pro può consultare il resoconto del questionario dopo aver deciso di concluderlo & \makecell{Interno\\ UC9.2.4 \\UC9.2.4.1 } \\ \hline
			 \hypertarget{{RFO8.2.4.2}}{{RFO8.2.4.2}} & L’utente autenticato pro può approvare la conclusione del questionario & \makecell{Interno\\ UC9.2.4 \\UC9.2.4.2 } \\ \hline
			 \hypertarget{{RFO8.3}}{{RFO8.3}} & L’utente autenticato pro può gestire le domande di un questionario & \makecell{Interno\\ UC9.1.1 \\UC9.2 \\UC9.3 } \\ \hline
			 \hypertarget{{RFO8.3.1}}{{RFO8.3.1}} & L’utente autenticato pro può aggiungere delle domande all’interno del questionario & \makecell{Interno\\ UC9.1.1 \\UC9.2 \\UC9.3 \\UC9.3.1 } \\ \hline
			 \hypertarget{{RFO8.3.1.1}}{{RFO8.3.1.1}} & L’utente autenticato pro può ricercare una domanda all’interno del database di domande & \makecell{Interno\\ UC9.1.1 \\UC9.2 \\UC9.3.1 \\UC9.3.1.1 } \\ \hline
			 \hypertarget{{RFO8.3.1.1.1}}{{RFO8.3.1.1.1}} & L’utente autenticato pro può selezionare delle domande da inserire all’interno dei questionari & \makecell{Interno\\ UC9.1.1 \\UC9.2 \\UC9.3.1.1 \\UC9.3.1.1.1 } \\ \hline
			 \hypertarget{{RFD8.3.1.1.2}}{{RFD8.3.1.1.2}} & L’utente autenticato pro può applicare dei filtri per effettuare una ricerca delle domande dettagliata & \makecell{Interno\\ UC9.1.1 \\UC9.2 \\UC9.3.1.1 \\UC9.3.1.1.2 } \\ \hline
			 \hypertarget{{RFO8.3.2}}{{RFO8.3.2}} & L’utente autenticato pro può eliminare una domanda dal questionario & \makecell{Interno\\ UC9.1.1 \\UC9.2 \\UC9.3 \\UC9.3.2 } \\ \hline
			 \hypertarget{{RFO8.3.2.1}}{{RFO8.3.2.1}} & L’utente autenticato pro può confermare se eliminare una domanda dal questionario & \makecell{Interno\\ UC9.1.1 \\UC9.2 \\UC9.3.2 \\UC9.3.2.1 } \\ \hline
			 \hypertarget{{RFD8.4}}{{RFD8.4}} & L’utente autenticato pro può gestire la iscrizione degli esaminandi ai questionari & \makecell{Interno\\ UC9 \\UC9.4 } \\ \hline
			 \hypertarget{{RFD8.4.1}}{{RFD8.4.1}} & L’utente autenticato pro può selezionare il questionario del quale gestire le iscrizioni & \makecell{Interno\\ UC9.4 \\UC9.4.1 } \\ \hline
			 \hypertarget{{RFD8.4.1.1}}{{RFD8.4.1.1}} & L’utente autenticato pro può accettare le iscrizioni degli esaminandi ai questionari & \makecell{Interno\\ UC9.4.1 \\UC9.4.1.1 } \\ \hline
			 \hypertarget{{RFO9}}{{RFO9}} & L’utente non autenticato, l’utente autenticato e l’utente autenticato pro possono esercitarsi nella modalità allenamento & \makecell{Verbale 2016-01-11\\ UC10 } \\ \hline
			 \hypertarget{{RFD9.1}}{{RFD9.1}} & L'utente non autenticato, l'utente autenticato e l'utente autenticato pro possono decidere un argomento per fare un allenamento & \makecell{Interno\\ UC10 \\UC10.1 } \\ \hline
			 \hypertarget{{RFD9.2}}{{RFD9.2}} & L'utente non autenticato, l'utente autenticato e l'utente autenticato pro possono decidere delle parole chiave per filtrare maggiormente le domande poste durante l'allenamento & \makecell{Interno\\ UC10 \\UC10.2 } \\ \hline
			 \hypertarget{{RFD9.3}}{{RFD9.3}} & L'utente non autenticato, l'utente autenticato e l'utente autenticato pro possono scegliere il numero di domande che comporranno l'allenamento (potenzialmente anche infinite domande) & \makecell{Interno\\ UC10 \\UC10.3 } \\ \hline
			 \hypertarget{{RFD9.4}}{{RFD9.4}} & L'utente non autenticato, l'utente autenticato e l'utente autenticato pro possono rispondere alle domande proposte iniziando l'allenamento & \makecell{Interno\\ UC10 \\UC10.4 } \\ \hline
			 \hypertarget{{RFD9.4.1}}{{RFD9.4.1}} & L'utente non autenticato, l'utente autenticato e l'utente autenticato pro possono confermare una risposta durante un allenamento & \makecell{Interno\\ UC10.4 \\UC10.4.1 } \\ \hline
			 \hypertarget{{RFD9.4.2}}{{RFD9.4.2}} & L'utente non autenticato, l'utente autenticato e l'utente autenticato pro possono rilasciare un "Like" ad una domanda proposta durante un allenamento & \makecell{Interno\\ UC10.4 \\UC10.4.2 } \\ \hline
			 \hypertarget{{RFD9.4.3}}{{RFD9.4.3}} & L'utente non autenticato, l'utente autenticato e l'utente autenticato pro possono rilasciare commenti ad una domanda proposta durante un allenamento & \makecell{Interno\\ UC10.4 \\UC10.4.3 \\UC10.4.3.1 \\UC10.4.3.2 } \\ \hline
			 \hypertarget{{RFD9.4.4}}{{RFD9.4.4}} & L'utente non autenticato, l'utente autenticato e l'utente autenticato pro possono segnalare una domanda indicando il tipo di segnalazione e scrivendo un commento per essa & \makecell{Interno\\ UC10.4 \\UC10.4.4 \\UC10.4.4.1 \\UC10.4.4.2 } \\ \hline
			 \hypertarget{{RFD9.4.5}}{{RFD9.4.5}} & L'utente non autenticato, l'utente autenticato e l'utente autenticato pro possono avanzare alla domanda successiva (se presente) durante l'allenamento & \makecell{Interno\\ UC10.4 \\UC10.4.5 } \\ \hline
			 \hypertarget{{RFD9.4.6}}{{RFD9.4.6}} & L'utente non autenticato, l'utente autenticato e l'utente autenticato pro possono decidere di terminare l'allenamento in qualunque momento & \makecell{Interno\\ UC10.4 \\UC10.4.6 } \\ \hline
			 \hypertarget{{RFD9.4.7}}{{RFD9.4.7}} & Il sistema deve visualizzare le statistiche finali dell'allenamento svolto & \makecell{Interno\\ UC10.4 \\UC10.4.6 } \\ \hline
			 \hypertarget{{RFD9.5}}{{RFD9.5}} & Il sistema sceglie una domanda in base all'abilità dell'avversario sull'argomento scelto & \makecell{Interno } \\ \hline
			 \hypertarget{{RFD9.6}}{{RFD9.6}} & Il sistema aggiorna automaticamente i dati sull'abilità dell'utente ad ogni risposta & \makecell{Interno } \\ \hline
			 \hypertarget{{RFD9.7}}{{RFD9.7}} & Il sistema aggiorna automaticamente i dati sulla difficoltà di una domanda quando un utente risponde alla medesima & \makecell{Interno } \\ \hline
			 \hypertarget{{RFD10}}{{RFD10}} & L’utente autenticato e l’utente autenticato pro possono visualizzare il proprio profilo & \makecell{Interno\\ UC11 } \\ \hline
			 \hypertarget{{RFD10.1}}{{RFD10.1}} & L’utente autenticato e l’utente autenticato pro possono andare alla pagina di gestione del profilo mediante l’apposito link & \makecell{Interno\\ UC11 \\UC11.1 } \\ \hline
			 \hypertarget{{RFD10.2}}{{RFD10.2}} & L’utente autenticato e l’utente autenticato pro possono andare alla pagina di gestione delle domande mediante l’apposito link & \makecell{Interno\\ UC11 \\UC11.2 } \\ \hline
			 \hypertarget{{RFD10.3}}{{RFD10.3}} & L’utente autenticato pro può andare alla pagina di gestione dei questionari mediante l’apposito link & \makecell{Interno\\ UC11 \\UC11.3 } \\ \hline
			 \hypertarget{{RFD10.4}}{{RFD10.4}} & L’utente autenticato e l’utente autenticato pro possono visualizzare la cronologia di tutti i questionari che hanno svolto & \makecell{Interno\\ UC11 \\UC12.4 } \\ \hline
			 \hypertarget{{RFD10.4.1}}{{RFD10.4.1}} & L’utente autenticato e l’utente autenticato pro possono selezionare e visualizzare le statistiche di un questionario scelto dalla cronologia & \makecell{Interno\\ UC11.4.1 \\UC12.4 } \\ \hline
			 \hypertarget{{RFD10.5}}{{RFD10.5}} & L’utente autenticato e l’utente autenticato pro possono visualizzare la lista dei questionari abilitati & \makecell{Interno\\ UC11 \\UC11.5 } \\ \hline
			 \hypertarget{{RFD10.5.1}}{{RFD10.5.1}} & L’utente autenticato e l’utente autenticato pro possono selezionare un questionario abilitato & \makecell{Interno\\ UC11.5 \\UC11.5.1 } \\ \hline
			 \hypertarget{{RFD10.6}}{{RFD10.6}} & L’utente autenticato e l’utente autenticato pro possono tornare alla home page mediante l’apposito link  & \makecell{Interno\\ UC11.6 } \\ \hline
			 \hypertarget{{RFO11}}{{RFO11}} & L’utente non autenticato, l'utente autenticato e l’utente autenticato pro possono rispondere alle domande & \makecell{Capitolato\\ UC12 } \\ \hline
			 \hypertarget{{RFO11.1}}{{RFO11.1}} & L’utente non autenticato, l’utente autenticato e l’utente autenticato pro possono rispondere ad una domanda vero o falso & \makecell{Capitolato\\ UC12 \\UC12.1 } \\ \hline
			 \hypertarget{{RFO11.2}}{{RFO11.2}} & L’utente non autenticato, l’utente autenticato e l’utente autenticato pro possono rispondere ad una domanda a risposta multipla & \makecell{Capitolato\\ UC12 \\UC12.2 } \\ \hline
			 \hypertarget{{RFO11.3}}{{RFO11.3}} & L’utente non autenticato, l’utente autenticato e l’utente autenticato pro possono compilare un esercizio di riempimento di uno spazio vuoto & \makecell{Capitolato\\ UC12 \\UC12.3 } \\ \hline
			 \hypertarget{{RFO11.3.1}}{{RFO11.3.1}} & L’utente non autenticato, l’utente autenticato e l’utente autenticato pro possono riempire lo spazio vuoto selezionato & \makecell{Capitolato\\ UC12.3 \\UC12.3.1 } \\ \hline
			 \hypertarget{{RFD11.4}}{{RFD11.4}} & L’utente non autenticato, l’utente autenticato e l’utente autenticato pro possono rispondere ad una domanda di collegamento & \makecell{Capitolato\\ UC12 \\UC12.4 } \\ \hline
			 \hypertarget{{RFD11.4.1}}{{RFD11.4.1}} & L’utente non autenticato, l’utente autenticato e l’utente autenticato pro possono collegare le voci & \makecell{Capitolato\\ UC12.4 \\UC12.4.1 \\UC12.4.1.1 \\UC12.4.1.2 } \\ \hline
			 \hypertarget{{RFD11.5}}{{RFD11.5}} & L’utente non autenticato, l’utente autenticato e l’utente autenticato pro possono ordinare delle immagini & \makecell{Verbale 2016-01-11\\ UC12 \\UC12.5 } \\ \hline
			 \hypertarget{{RFD11.5.1}}{{RFD11.5.1}} & L’utente non autenticato, l’utente autenticato e l’utente autenticato pro possono inserire un’immagine in uno spazio già occupato oppure no & \makecell{Verbale 2016-01-11\\ UC12.5 \\UC12.5.1 \\UC12.5.1.1 \\UC12.5.1.2 } \\ \hline
			 \hypertarget{{RFD11.6}}{{RFD11.6}} & L’utente non autenticato, l’utente autenticato e l’utente autenticato pro possono ordinare delle stringhe & \makecell{Verbale 2016-01-11\\ UC12 \\UC12.6 } \\ \hline
			 \hypertarget{{RFD11.6.1}}{{RFD11.6.1}} & L’utente non autenticato, l’utente autenticato e l’utente autenticato pro possono inserire una stringa in uno spazio già occupato oppure no & \makecell{Verbale 2016-01-11\\ UC12.6 \\UC12.6.1 \\UC12.6.1.1 \\UC12.6.1.2 } \\ \hline
			 \hypertarget{{RFD11.7}}{{RFD11.7}} & L’utente non autenticato, l’utente autenticato e l’utente autenticato pro possono rispondere ad una domanda con area cliccabile & \makecell{Verbale 2016-01-11\\ UC12 \\UC12.7 } \\ \hline
			 \hypertarget{{RFD11.7.1}}{{RFD11.7.1}} & L’utente non autenticato, l’utente autenticato e l’utente autenticato pro possono selezionare un’area cliccabile & \makecell{Verbale 2016-01-11\\ UC12.7 \\UC12.7.1 } \\ \hline
			 \hypertarget{{RFO12}}{{RFO12}} & L’utente autenticato e l’utente autenticato pro possono ricercare un utente & \makecell{Interno } \\ \hline
			 \hypertarget{{RFD12.1}}{{RFD12.1}} & L’utente autenticato e l’utente autenticato pro possono inserire nome e cognome oppure lo username nella barra di ricerca per ricercare un utente & \makecell{Interno\\ UC13 } \\ \hline
			 \hypertarget{{RFD12.1.1}}{{RFD12.1.1}} &  L’utente autenticato e l’utente autenticato pro possono selezionare l’utente ricercato & \makecell{Interno\\ UC13 \\UC13.2 } \\ \hline
			 \hypertarget{{RFO13}}{{RFO13}} & Il sistema deve gestire un sistema per proporre dei questionari & \makecell{Capitolato } \\ \hline
			 \hypertarget{{RFO14}}{{RFO14}} & Il sistema deve proporre all’utilizzatore questionari specifici per l’argomento scelto & \makecell{Capitolato } \\ \hline
			 \hypertarget{{RFO15}}{{RFO15}} & Il sistema deve creare dei questionari partendo dalle domande archiviate & \makecell{Capitolato } \\ \hline
			 \hypertarget{{RFO16}}{{RFO16}} & Il sistema deve archiviare le domande attraverso uno specifico linguaggio chiamato \textit{QML\ped{G}} & \makecell{Capitolato } \\ \hline
			 \hypertarget{{RFO16.1}}{{RFO16.1}} & Il sistema attraverso il linguaggio \textit{QML\ped{G}} deve gestire domande vero o falso & \makecell{Capitolato } \\ \hline
			 \hypertarget{{RFO16.2}}{{RFO16.2}} & Il sistema attraverso il linguaggio \textit{QML\ped{G}} deve gestire domande a risposte a scelta multipla & \makecell{Capitolato } \\ \hline
			 \hypertarget{{RFO16.3}}{{RFO16.3}} & Il sistema attraverso il linguaggio \textit{QML\ped{G}} deve gestire esercizi con riempimento di spazi vuoti & \makecell{Capitolato } \\ \hline
			 \hypertarget{{RFO16.4}}{{RFO16.4}} & Il sistema attraverso il linguaggio \textit{QML\ped{G}} deve gestire esercizi con delle immagini & \makecell{Capitolato } \\ \hline
			 \hypertarget{{RFD16.5}}{{RFD16.5}} & Il sistema attraverso il linguaggio \textit{QML\ped{G}} deve gestire esercizi di ordinamento di scelte & \makecell{Capitolato } \\ \hline
			 \hypertarget{{RFD16.6}}{{RFD16.6}} & Il sistema attraverso il linguaggio \textit{QML\ped{G}} deve gestire esercizi a corrispondenza di scelte & \makecell{Capitolato } \\ \hline
			 \hypertarget{{RFO16.7}}{{RFO16.7}} & Il sistema deve tradurre le domande descritte in \textit{QML\ped{G}} in \textit{HTML\ped{G}} & \makecell{Capitolato } \\ \hline
			 \hypertarget{{RFO16.8}}{{RFO16.8}} & Il sistema deve archiviare le domande suddivise per argomento & \makecell{Capitolato } \\ \hline
			 \hypertarget{{RFO17}}{{RFO17}} & Il sistema deve archiviare i questionari creati con le domande & \makecell{Capitolato } \\ \hline
			 \hypertarget{{RFO18}}{{RFO18}} & Il sistema deve proporre all’utilizzatore dei questionari preconfezionati & \makecell{Capitolato } \\ \hline
			 \hypertarget{{RFO19}}{{RFO19}} & Il sistema deve permettere agli utilizzatori di poter creare domande e questionari da dispositivi desktop & \makecell{Capitolato } \\ \hline
			 \hypertarget{{RFO20}}{{RFO20}} & Il sistema deve permettere agli utilizzatori di poter rispondere alle domande e ai questionari da dispositivi desktop, tablet e smartphone & \makecell{Capitolato } \\ \hline
			 \hypertarget{{RFD21}}{{RFD21}} & Il sistema deve archiviare i risultati dei questionari & \makecell{Capitolato } \\ \hline
			 \hypertarget{{RFD22}}{{RFD22}} & Il sistema deve archiviare le statistiche delle risposte date ad ogni domanda & \makecell{Capitolato } \\ \hline
			 \hypertarget{{RFF23}}{{RFF23}} & Il sistema deve valutare il candidato rispetto agli altri candidati che hanno svolto lo stesso quiz & \makecell{Capitolato } \\ \hline
			 \hypertarget{{RFD24}}{{RFD24}} & Il sistema deve creare questionari dinamicamente (modalità allenamento) per un argomento scegliendo le domande in modo casuale & \makecell{Capitolato\\ UC10 } \\ \hline
			 \hypertarget{{RFD25}}{{RFD25}} & Il sistema deve creare questionari dinamicamente (modalità allenamento) scegliendo le domande in base alle risposte date dai questionari precedenti & \makecell{Capitolato\\ UC10 } \\ \hline
			 \hypertarget{{RFD26}}{{RFD26}} & Il sistema deve creare questionari dinamicamente (modalità allenamento) scegliendo tra le domande più difficili & \makecell{Capitolato\\ UC10 } \\ \hline
			 \hypertarget{{RFD27}}{{RFD27}} & Il sistema deve creare questionari dinamicamente (modalità allenamento) scegliendo tra le lacune dei partecipanti & \makecell{Capitolato\\ UC10 } \\ \hline
			 \hypertarget{{RFD28}}{{RFD28}} & Il sistema deve permettere agli utilizzatori di proporre nuove domande & \makecell{Capitolato } \\ \hline
			 \hypertarget{{RFF29}}{{RFF29}} & Il sistema deve permettere agli utilizzatori di segnalare positivamente una domanda & \makecell{Capitolato } \\ \hline
			 \hypertarget{{RFF30}}{{RFF30}} & Il sistema deve permettere agli utilizzatori di commentare un domanda & \makecell{Capitolato } \\ \hline
			 \hypertarget{{RFF31}}{{RFF31}} & Il sistema deve permettere agli utilizzatori di rispondere più volte ad una domanda & \makecell{Capitolato} \\ \hline
\caption[Requisiti Funzionali]{Requisiti Funzionali}
\label{tabella:req0}
\end{longtable}
\clearpage
\subsection{Requisiti di Qualità}
\normalsize
\begin{longtable}{|c|>{\centering}m{7cm}|c|}
\hline
\textbf{Id Requisito} & \textbf{Descrizione} & \textbf{Fonti}\\
\hline
\endhead \hypertarget{{RQO1}}{{RQO1}} & Deve essere fornito un manuale utente & \makecell{Capitolato } \\ \hline
			 \hypertarget{{RQO2}}{{RQO2}} & Il manuale utente deve contenere una sezione in cui viene spiegato come installare correttamente l'applicazione & \makecell{Interno } \\ \hline
			 \hypertarget{{RQO3}}{{RQO3}} & Il manuale utente deve contenere una sezione in cui viene approfonditamente spiegato come utilizzare l'applicazione & \makecell{Interno } \\ \hline
			 \hypertarget{{RQO4}}{{RQO4}} & Il manuale utente deve includere una sezione contenente un elenco di possibili errori e malfunzionamenti dell'applicazione e le loro possibili cause & \makecell{Interno } \\ \hline
			 \hypertarget{{RQO5}}{{RQO5}} & Il manuale utente deve contenere una sezione che spiega come segnalare eventuali errori e malfunzionamenti & \makecell{Interno } \\ \hline
			 \hypertarget{{RQO6}}{{RQO6}} & Deve essere fornito un manuale per gli utenti sviluppatori che intendono estendere l'applicazione & \makecell{Capitolato } \\ \hline
			 \hypertarget{{RQO7}}{{RQO7}} & Il manuale per gli utenti sviluppatori che intendono estendere l'applicazione deve contenere una sezione che spiega come segnalare eventuali errori o malfunzionamenti & \makecell{Interno } \\ \hline
			 \hypertarget{{RQF8}}{{RQF8}} & La documentazione per l'utente deve essere disponibile in lingua inglese & \makecell{Interno } \\ \hline
			 \hypertarget{{RQO9}}{{RQO9}} & La documentazione per l'utente deve essere disponibile in lingua italiana & \makecell{Interno} \\ \hline
\caption[Requisiti di Qualità]{Requisiti di Qualità}
\label{tabella:req2}
\end{longtable}
\clearpage
\subsection{Requisiti di Vincolo}
\normalsize
\begin{longtable}{|c|>{\centering}m{7cm}|c|}
\hline
\textbf{Id Requisito} & \textbf{Descrizione} & \textbf{Fonti}\\
\hline
\endhead \hypertarget{{RVO1}}{{RVO1}} & L’applicazione deve utilizzare il linguaggio \textit{Javascript\ped{G}}  & \makecell{Capitolato } \\ \hline
			 \hypertarget{{RVO2}}{{RVO2}} & L’applicazione deve utilizzare il \textit{linguaggio di markup\ped{G}} \textit{HTML5\ped{G}} & \makecell{Capitolato } \\ \hline
			 \hypertarget{{RVO3}}{{RVO3}} & L’applicazione deve utilizzare \textit{fogli di stile\ped{G}} in \textit{CSS\ped{G}} & \makecell{Capitolato } \\ \hline
			 \hypertarget{{RVD4}}{{RVD4}} & L’applicazione deve utilizzare \textit{fogli di stile\ped{G}} in \textit{CSS3\ped{G}} & \makecell{Capitolato } \\ \hline
			 \hypertarget{{RVO5}}{{RVO5}} & L’applicazione deve funzionare su \textit{Mozilla Firefox\ped{G}} versione 33.0 o superiore & \makecell{Interno } \\ \hline
			 \hypertarget{{RVO6}}{{RVO6}} & L’applicazione deve funzionare su \textit{Google Chrome\ped{G}} versione 31.0 o superiore & \makecell{Interno } \\ \hline
			 \hypertarget{{RVO7}}{{RVO7}} & L’applicazione deve funzionare su \textit{Safari\ped{G}} versione 7.1 o superiore & \makecell{Interno } \\ \hline
			 \hypertarget{{RVO8}}{{RVO8}} & L’applicazione deve funzionare su \textit{Opera\ped{G}} versione 26.0 o superiore & \makecell{Interno } \\ \hline
			 \hypertarget{{RVD9}}{{RVD9}} & L’applicazione deve funzionare su \textit{Internet Explorer\ped{G}} versione 11 o superiore & \makecell{Interno } \\ \hline
			 \hypertarget{{RVO10}}{{RVO10}} & L’applicazione deve funzionare su \textit{Microsoft Edge\ped{G}} versione 25  o superiore & \makecell{Interno } \\ \hline
			 \hypertarget{{RVD11}}{{RVD11}} & L’applicazione deve funzionare su \textit{Android Browser\ped{G}} versione 4.4 o superiore per le funzionalità che riguardano la compilazione dei questionari e delle domande & \makecell{Interno } \\ \hline
			 \hypertarget{{RVO12}}{{RVO12}} & L’applicazione deve funzionare su \textit{Safari per iOS 8\ped{G}} o versioni superiori per le funzionalità che riguardano la compilazione dei questionari e delle domande & \makecell{Interno } \\ \hline
			 \hypertarget{{RVD13}}{{RVD13}} & L’applicazione deve funzionare su \textit{Google Chrome per iOS\ped{G}} versione 39 o superiore per le funzionalità che riguardano la compilazione dei questionari e delle domande & \makecell{Interno } \\ \hline
			 \hypertarget{{RVO14}}{{RVO14}} & L’applicazione deve funzionare su \textit{Google Chrome per Android\ped{G}} versione 39 o superiore per le funzionalità che riguardano la compilazione dei questionari e delle domande & \makecell{Interno } \\ \hline
			 \hypertarget{{RVF15}}{{RVF15}} & L’applicazione deve funzionare su \textit{Mozilla Firefox per Android\ped{G}} versione 33 o superiore per le funzionalità che riguardano la compilazione dei questionari e delle domande & \makecell{Interno } \\ \hline
			 \hypertarget{{RVF16}}{{RVF16}} & L’applicazione deve funzionare su \textit{Microsoft Edge per Windows 10 mobile\ped{G}} versione 25 o superiore per le funzionalità che riguardano la compilazione dei questionari e delle domande & \makecell{Interno } \\ \hline
			 \hypertarget{{RVF17}}{{RVF17}} & L’applicazione deve funzionare su \textit{Browser Opera Mobile per Android\ped{G}} versione 34 o superiore per le funzionalità che riguardano la compilazione dei questionari e delle domande & \makecell{Interno } \\ \hline
			 \hypertarget{{RVF18}}{{RVF18}} & L’applicazione deve funzionare su \textit{Opera Mini per iOS\ped{G}} versione 12 o superiore per le funzionalità che riguardano la compilazione dei questionari e delle domande & \makecell{Interno } \\ \hline
			 \hypertarget{{RVD19}}{{RVD19}} & L’applicazione deve funzionare su \textit{Safari per iOS 8\ped{G}} o versioni superiori per le funzionalità che riguardano la creazione dei questionari e delle domande & \makecell{Interno } \\ \hline
			 \hypertarget{{RVD20}}{{RVD20}} & L’applicazione deve funzionare su \textit{Google Chrome per Android\ped{G}} versione 39 o superiore per le funzionalità che riguardano la creazione dei questionari e delle domande & \makecell{Interno } \\ \hline
			 \hypertarget{{RVF21}}{{RVF21}} & L’applicazione deve funzionare su \textit{Android Browser\ped{G}} versione 4.4 o superiore per le funzionalità che riguardano la creazione dei questionari e delle domande & \makecell{Interno } \\ \hline
			 \hypertarget{{RVF22}}{{RVF22}} & L’applicazione deve funzionare su \textit{Google Chrome per iOS\ped{G}} versione 39 o superiore per le funzionalità che riguardano la creazione dei questionari e delle domande & \makecell{Interno } \\ \hline
			 \hypertarget{{RVF23}}{{RVF23}} & L’applicazione deve funzionare su \textit{Mozilla Firefox per Android\ped{G}} versione 33 o superiore per le funzionalità che riguardano la creazione dei questionari e delle domande & \makecell{Interno } \\ \hline
			 \hypertarget{{RVF24}}{{RVF24}} & L’applicazione deve funzionare su \textit{Microsoft Edge per Windows 10 mobile\ped{G}} versione 25 o superiore per le funzionalità che riguardano la creazione dei questionari e delle domande & \makecell{Interno } \\ \hline
			 \hypertarget{{RVF25}}{{RVF25}} & L’applicazione deve funzionare su \textit{Browser Opera per Android\ped{G}} versione 34 o superiore per le funzionalità che riguardano la creazione dei questionari e delle domande & \makecell{Interno } \\ \hline
			 \hypertarget{{RVF26}}{{RVF26}} & L’applicazione deve funzionare su \textit{Opera Mini per iOS\ped{G}} versione 12 o superiore per le funzionalità che riguardano la creazione dei questionari e delle domande & \makecell{Interno} \\ \hline
\caption[Requisiti di Vincolo]{Requisiti di Vincolo}
\label{tabella:req3}
\end{longtable}
\clearpage

\subsection{Tracciamento Fonti-Requisiti}
\normalsize
\begin{longtable}{|>{\centering}m{5cm}|m{5cm}<{\centering}|}
\hline 
\textbf{Fonte} & \textbf{Id Requisiti}\\
\hline
\endhead
\hyperlink{Capitolato}{Capitolato} & \hyperlink{RFO6}{RFO6}\\
& \hyperlink{RFO6.5}{RFO6.5}\\
& \hyperlink{RFO7}{RFO7}\\
& \hyperlink{RFO7.1}{RFO7.1}\\
& \hyperlink{RFO7.1.3.1}{RFO7.1.3.1}\\
& \hyperlink{RFO7.1.3.1.1}{RFO7.1.3.1.1}\\
& \hyperlink{RFD7.1.3.1.2}{RFD7.1.3.1.2}\\
& \hyperlink{RFO7.1.3.1.3}{RFO7.1.3.1.3}\\
& \hyperlink{RFO7.1.3.2}{RFO7.1.3.2}\\
& \hyperlink{RFO7.1.3.2.1}{RFO7.1.3.2.1}\\
& \hyperlink{RFD7.1.3.2.2}{RFD7.1.3.2.2}\\
& \hyperlink{RFO7.1.3.2.3}{RFO7.1.3.2.3}\\
& \hyperlink{RFO7.1.3.2.3.1}{RFO7.1.3.2.3.1}\\
& \hyperlink{RFD7.1.3.2.3.2}{RFD7.1.3.2.3.2}\\
& \hyperlink{RFO7.1.3.2.4}{RFO7.1.3.2.4}\\
& \hyperlink{RFD7.1.3.3}{RFD7.1.3.3}\\
& \hyperlink{RFD7.1.3.3.1}{RFD7.1.3.3.1}\\
& \hyperlink{RFD7.1.3.3.2}{RFD7.1.3.3.2}\\
& \hyperlink{RFD7.1.3.4}{RFD7.1.3.4}\\
& \hyperlink{RFD7.1.3.4.1}{RFD7.1.3.4.1}\\
& \hyperlink{RFD7.1.3.4.2}{RFD7.1.3.4.2}\\
& \hyperlink{RFD7.1.3.4.2.1}{RFD7.1.3.4.2.1}\\
& \hyperlink{RFD7.1.3.4.2.2}{RFD7.1.3.4.2.2}\\
& \hyperlink{RFD7.1.3.4.2.3}{RFD7.1.3.4.2.3}\\
& \hyperlink{RFD7.1.3.4.2.4}{RFD7.1.3.4.2.4}\\
& \hyperlink{RFD7.1.3.5}{RFD7.1.3.5}\\
& \hyperlink{RFD7.1.3.5.1}{RFD7.1.3.5.1}\\
& \hyperlink{RFD7.1.3.5.2}{RFD7.1.3.5.2}\\
& \hyperlink{RFD7.1.3.5.3}{RFD7.1.3.5.3}\\
& \hyperlink{RFD7.1.3.6}{RFD7.1.3.6}\\
& \hyperlink{RFD7.1.3.6.1}{RFD7.1.3.6.1}\\
& \hyperlink{RFD7.1.3.6.2}{RFD7.1.3.6.2}\\
& \hyperlink{RFD7.1.3.6.3}{RFD7.1.3.6.3}\\
& \hyperlink{RFD7.1.3.7}{RFD7.1.3.7}\\
& \hyperlink{RFD7.1.3.7.1}{RFD7.1.3.7.1}\\
& \hyperlink{RFD7.1.3.7.2}{RFD7.1.3.7.2}\\
& \hyperlink{RFD7.1.3.7.3}{RFD7.1.3.7.3}\\
& \hyperlink{RFD7.1.3.7.4}{RFD7.1.3.7.4}\\
& \hyperlink{RFO7.1.4}{RFO7.1.4}\\
& \hyperlink{RFO7.1.5}{RFO7.1.5}\\
& \hyperlink{RFO8.2}{RFO8.2}\\
& \hyperlink{RFO8.2.1}{RFO8.2.1}\\
& \hyperlink{RFO11}{RFO11}\\
& \hyperlink{RFO11.1}{RFO11.1}\\
& \hyperlink{RFO11.2}{RFO11.2}\\
& \hyperlink{RFO11.3}{RFO11.3}\\
& \hyperlink{RFO11.3.1}{RFO11.3.1}\\
& \hyperlink{RFD11.4}{RFD11.4}\\
& \hyperlink{RFD11.4.1}{RFD11.4.1}\\
& \hyperlink{RFO13}{RFO13}\\
& \hyperlink{RFO14}{RFO14}\\
& \hyperlink{RFO15}{RFO15}\\
& \hyperlink{RFO16}{RFO16}\\
& \hyperlink{RFO16.1}{RFO16.1}\\
& \hyperlink{RFO16.2}{RFO16.2}\\
& \hyperlink{RFO16.3}{RFO16.3}\\
& \hyperlink{RFO16.4}{RFO16.4}\\
& \hyperlink{RFD16.5}{RFD16.5}\\
& \hyperlink{RFD16.6}{RFD16.6}\\
& \hyperlink{RFO16.7}{RFO16.7}\\
& \hyperlink{RFO16.8}{RFO16.8}\\
& \hyperlink{RFO17}{RFO17}\\
& \hyperlink{RFO18}{RFO18}\\
& \hyperlink{RFO19}{RFO19}\\
& \hyperlink{RFO20}{RFO20}\\
& \hyperlink{RFD21}{RFD21}\\
& \hyperlink{RFD22}{RFD22}\\
& \hyperlink{RFF23}{RFF23}\\
& \hyperlink{RFD24}{RFD24}\\
& \hyperlink{RFD25}{RFD25}\\
& \hyperlink{RFD26}{RFD26}\\
& \hyperlink{RFD27}{RFD27}\\
& \hyperlink{RFD28}{RFD28}\\
& \hyperlink{RFF29}{RFF29}\\
& \hyperlink{RFF30}{RFF30}\\
& \hyperlink{RFF31}{RFF31}\\
& \hyperlink{RQO1}{RQO1}\\
& \hyperlink{RQO6}{RQO6}\\
& \hyperlink{RVO1}{RVO1}\\
& \hyperlink{RVO2}{RVO2}\\
& \hyperlink{RVO3}{RVO3}\\
& \hyperlink{RVD4}{RVD4}\\ \hline
\hyperlink{Interno}{Interno} & \hyperlink{RFO1}{RFO1}\\
& \hyperlink{RFO1.1}{RFO1.1}\\
& \hyperlink{RFO1.2}{RFO1.2}\\
& \hyperlink{RFO1.3}{RFO1.3}\\
& \hyperlink{RFO1.4}{RFO1.4}\\
& \hyperlink{RFO1.5}{RFO1.5}\\
& \hyperlink{RFO1.6}{RFO1.6}\\
& \hyperlink{RFO1.7}{RFO1.7}\\
& \hyperlink{RFO1.8}{RFO1.8}\\
& \hyperlink{RFO2.1.1}{RFO2.1.1}\\
& \hyperlink{RFO2.1.2}{RFO2.1.2}\\
& \hyperlink{RFO2.1.3}{RFO2.1.3}\\
& \hyperlink{RFO2.1.4}{RFO2.1.4}\\
& \hyperlink{RFF2.2}{RFF2.2}\\
& \hyperlink{RFF2.3}{RFF2.3}\\
& \hyperlink{RFF2.4}{RFF2.4}\\
& \hyperlink{RFF2.5}{RFF2.5}\\
& \hyperlink{RFO3}{RFO3}\\
& \hyperlink{RFO3.1}{RFO3.1}\\
& \hyperlink{RFD4}{RFD4}\\
& \hyperlink{RFD4.1}{RFD4.1}\\
& \hyperlink{RFD4.1.1}{RFD4.1.1}\\
& \hyperlink{RFD4.1.2}{RFD4.1.2}\\
& \hyperlink{RFD4.2}{RFD4.2}\\
& \hyperlink{RFD4.2.1}{RFD4.2.1}\\
& \hyperlink{RFD4.2.2}{RFD4.2.2}\\
& \hyperlink{RFD4.3}{RFD4.3}\\
& \hyperlink{RFD4.3.1}{RFD4.3.1}\\
& \hyperlink{RFD4.3.2}{RFD4.3.2}\\
& \hyperlink{RFD4.4}{RFD4.4}\\
& \hyperlink{RFD4.4.1}{RFD4.4.1}\\
& \hyperlink{RFD4.4.2}{RFD4.4.2}\\
& \hyperlink{RFD4.5}{RFD4.5}\\
& \hyperlink{RFD4.5.1}{RFD4.5.1}\\
& \hyperlink{RFD4.5.2}{RFD4.5.2}\\
& \hyperlink{RFD4.6}{RFD4.6}\\
& \hyperlink{RFD4.6.1}{RFD4.6.1}\\
& \hyperlink{RFD4.6.2}{RFD4.6.2}\\
& \hyperlink{RFD4.7}{RFD4.7}\\
& \hyperlink{RFD4.7.1}{RFD4.7.1}\\
& \hyperlink{RFD4.8}{RFD4.8}\\
& \hyperlink{RFD4.8.1}{RFD4.8.1}\\
& \hyperlink{RFD5}{RFD5}\\
& \hyperlink{RFD5.1}{RFD5.1}\\
& \hyperlink{RFD5.2}{RFD5.2}\\
& \hyperlink{RFD5.2.1}{RFD5.2.1}\\
& \hyperlink{RFD5.3}{RFD5.3}\\
& \hyperlink{RFD6.1}{RFD6.1}\\
& \hyperlink{RFD6.2}{RFD6.2}\\
& \hyperlink{RFD6.3}{RFD6.3}\\
& \hyperlink{RFD6.4}{RFD6.4}\\
& \hyperlink{RFD7.1.1}{RFD7.1.1}\\
& \hyperlink{RFD7.1.2}{RFD7.1.2}\\
& \hyperlink{RFO7.1.3}{RFO7.1.3}\\
& \hyperlink{RFD7.1.3.1.2.1}{RFD7.1.3.1.2.1}\\
& \hyperlink{RFD7.1.3.2.2.1}{RFD7.1.3.2.2.1}\\
& \hyperlink{RFD7.1.3.2.3.1.1}{RFD7.1.3.2.3.1.1}\\
& \hyperlink{RFD7.1.3.2.3.2.1}{RFD7.1.3.2.3.2.1}\\
& \hyperlink{RFD7.1.3.4.3}{RFD7.1.3.4.3}\\
& \hyperlink{RFD7.1.3.4.3.1}{RFD7.1.3.4.3.1}\\
& \hyperlink{RFD7.1.3.4.4}{RFD7.1.3.4.4}\\
& \hyperlink{RFD7.1.3.4.4.1}{RFD7.1.3.4.4.1}\\
& \hyperlink{RFD7.1.3.4.4.2}{RFD7.1.3.4.4.2}\\
& \hyperlink{RFD7.1.3.4.4.3}{RFD7.1.3.4.4.3}\\
& \hyperlink{RFD7.1.3.4.4.4}{RFD7.1.3.4.4.4}\\
& \hyperlink{RFD7.1.3.5.2.1}{RFD7.1.3.5.2.1}\\
& \hyperlink{RFD7.1.3.5.3.1}{RFD7.1.3.5.3.1}\\
& \hyperlink{RFD7.2}{RFD7.2}\\
& \hyperlink{RFD7.2.1}{RFD7.2.1}\\
& \hyperlink{RFD7.2.1.1}{RFD7.2.1.1}\\
& \hyperlink{RFD7.2.1.1.1}{RFD7.2.1.1.1}\\
& \hyperlink{RFD7.2.1.1.2}{RFD7.2.1.1.2}\\
& \hyperlink{RFD7.2.1.1.3}{RFD7.2.1.1.3}\\
& \hyperlink{RFD7.2.1.2}{RFD7.2.1.2}\\
& \hyperlink{RFD7.2.1.2.1}{RFD7.2.1.2.1}\\
& \hyperlink{RFD7.2.1.2.2}{RFD7.2.1.2.2}\\
& \hyperlink{RFD7.2.1.2.3}{RFD7.2.1.2.3}\\
& \hyperlink{RFD7.2.1.2.3.1}{RFD7.2.1.2.3.1}\\
& \hyperlink{RFD7.2.1.2.3.2}{RFD7.2.1.2.3.2}\\
& \hyperlink{RFD7.2.1.2.4}{RFD7.2.1.2.4}\\
& \hyperlink{RFD7.2.1.3}{RFD7.2.1.3}\\
& \hyperlink{RFD7.2.1.3.1}{RFD7.2.1.3.1}\\
& \hyperlink{RFD7.2.1.3.2}{RFD7.2.1.3.2}\\
& \hyperlink{RFD7.2.1.4}{RFD7.2.1.4}\\
& \hyperlink{RFD7.2.1.4.1}{RFD7.2.1.4.1}\\
& \hyperlink{RFD7.2.1.4.1.1}{RFD7.2.1.4.1.1}\\
& \hyperlink{RFD7.2.1.4.1.2}{RFD7.2.1.4.1.2}\\
& \hyperlink{RFD7.2.1.4.1.3}{RFD7.2.1.4.1.3}\\
& \hyperlink{RFD7.2.1.4.1.4}{RFD7.2.1.4.1.4}\\
& \hyperlink{RFD7.2.1.4.2}{RFD7.2.1.4.2}\\
& \hyperlink{RFD7.2.1.4.2.1}{RFD7.2.1.4.2.1}\\
& \hyperlink{RFD7.2.1.4.3}{RFD7.2.1.4.3}\\
& \hyperlink{RFD7.2.1.4.3.1}{RFD7.2.1.4.3.1}\\
& \hyperlink{RFD7.2.1.4.3.2}{RFD7.2.1.4.3.2}\\
& \hyperlink{RFD7.2.1.4.3.3}{RFD7.2.1.4.3.3}\\
& \hyperlink{RFD7.2.1.4.3.4}{RFD7.2.1.4.3.4}\\
& \hyperlink{RFD7.2.1.4.4}{RFD7.2.1.4.4}\\
& \hyperlink{RFD7.2.1.4.5}{RFD7.2.1.4.5}\\
& \hyperlink{RFD7.2.1.4.6}{RFD7.2.1.4.6}\\
& \hyperlink{RFD7.2.1.5}{RFD7.2.1.5}\\
& \hyperlink{RFD7.2.1.5.1}{RFD7.2.1.5.1}\\
& \hyperlink{RFD7.2.1.5.2}{RFD7.2.1.5.2}\\
& \hyperlink{RFD7.2.1.5.3}{RFD7.2.1.5.3}\\
& \hyperlink{RFD7.2.1.5.4}{RFD7.2.1.5.4}\\
& \hyperlink{RFD7.2.1.6}{RFD7.2.1.6}\\
& \hyperlink{RFD7.2.1.6.1}{RFD7.2.1.6.1}\\
& \hyperlink{RFD7.2.1.6.2}{RFD7.2.1.6.2}\\
& \hyperlink{RFD7.2.1.6.3}{RFD7.2.1.6.3}\\
& \hyperlink{RFD7.2.1.7}{RFD7.2.1.7}\\
& \hyperlink{RFD7.2.1.7.1}{RFD7.2.1.7.1}\\
& \hyperlink{RFD7.2.1.7.2}{RFD7.2.1.7.2}\\
& \hyperlink{RFD7.2.1.7.3}{RFD7.2.1.7.3}\\
& \hyperlink{RFD7.2.1.7.4}{RFD7.2.1.7.4}\\
& \hyperlink{RFD7.2.1.8}{RFD7.2.1.8}\\
& \hyperlink{RFD7.2.2}{RFD7.2.2}\\
& \hyperlink{RFD7.2.3}{RFD7.2.3}\\
& \hyperlink{RFD8.1}{RFD8.1}\\
& \hyperlink{RFD8.1.1}{RFD8.1.1}\\
& \hyperlink{RFD8.1.1.1}{RFD8.1.1.1}\\
& \hyperlink{RFD8.1.1.2}{RFD8.1.1.2}\\
& \hyperlink{RFD8.1.2}{RFD8.1.2}\\
& \hyperlink{RFD8.1.2.1}{RFD8.1.2.1}\\
& \hyperlink{RFD8.1.4}{RFD8.1.4}\\
& \hyperlink{RFD8.2.2}{RFD8.2.2}\\
& \hyperlink{RFD8.2.3}{RFD8.2.3}\\
& \hyperlink{RFO8.2.4}{RFO8.2.4}\\
& \hyperlink{RFD8.2.4.1}{RFD8.2.4.1}\\
& \hyperlink{RFO8.2.4.2}{RFO8.2.4.2}\\
& \hyperlink{RFO8.3}{RFO8.3}\\
& \hyperlink{RFO8.3.1}{RFO8.3.1}\\
& \hyperlink{RFO8.3.1.1}{RFO8.3.1.1}\\
& \hyperlink{RFO8.3.1.1.1}{RFO8.3.1.1.1}\\
& \hyperlink{RFD8.3.1.1.2}{RFD8.3.1.1.2}\\
& \hyperlink{RFO8.3.2}{RFO8.3.2}\\
& \hyperlink{RFO8.3.2.1}{RFO8.3.2.1}\\
& \hyperlink{RFD8.4}{RFD8.4}\\
& \hyperlink{RFD8.4.1}{RFD8.4.1}\\
& \hyperlink{RFD8.4.1.1}{RFD8.4.1.1}\\
& \hyperlink{RFD9.1}{RFD9.1}\\
& \hyperlink{RFD9.2}{RFD9.2}\\
& \hyperlink{RFD9.3}{RFD9.3}\\
& \hyperlink{RFD9.4}{RFD9.4}\\
& \hyperlink{RFD9.4.1}{RFD9.4.1}\\
& \hyperlink{RFD9.4.2}{RFD9.4.2}\\
& \hyperlink{RFD9.4.3}{RFD9.4.3}\\
& \hyperlink{RFD9.4.4}{RFD9.4.4}\\
& \hyperlink{RFD9.4.5}{RFD9.4.5}\\
& \hyperlink{RFD9.4.6}{RFD9.4.6}\\
& \hyperlink{RFD9.4.7}{RFD9.4.7}\\
& \hyperlink{RFD9.5}{RFD9.5}\\
& \hyperlink{RFD9.6}{RFD9.6}\\
& \hyperlink{RFD9.7}{RFD9.7}\\
& \hyperlink{RFD10}{RFD10}\\
& \hyperlink{RFD10.1}{RFD10.1}\\
& \hyperlink{RFD10.2}{RFD10.2}\\
& \hyperlink{RFD10.3}{RFD10.3}\\
& \hyperlink{RFD10.4}{RFD10.4}\\
& \hyperlink{RFD10.4.1}{RFD10.4.1}\\
& \hyperlink{RFD10.5}{RFD10.5}\\
& \hyperlink{RFD10.5.1}{RFD10.5.1}\\
& \hyperlink{RFD10.6}{RFD10.6}\\
& \hyperlink{RFO12}{RFO12}\\
& \hyperlink{RFD12.1}{RFD12.1}\\
& \hyperlink{RFD12.1.1}{RFD12.1.1}\\
& \hyperlink{RQO2}{RQO2}\\
& \hyperlink{RQO3}{RQO3}\\
& \hyperlink{RQO4}{RQO4}\\
& \hyperlink{RQO5}{RQO5}\\
& \hyperlink{RQO7}{RQO7}\\
& \hyperlink{RQF8}{RQF8}\\
& \hyperlink{RQO9}{RQO9}\\
& \hyperlink{RVO5}{RVO5}\\
& \hyperlink{RVO6}{RVO6}\\
& \hyperlink{RVO7}{RVO7}\\
& \hyperlink{RVO8}{RVO8}\\
& \hyperlink{RVD9}{RVD9}\\
& \hyperlink{RVO10}{RVO10}\\
& \hyperlink{RVD11}{RVD11}\\
& \hyperlink{RVO12}{RVO12}\\
& \hyperlink{RVD13}{RVD13}\\
& \hyperlink{RVO14}{RVO14}\\
& \hyperlink{RVF15}{RVF15}\\
& \hyperlink{RVF16}{RVF16}\\
& \hyperlink{RVF17}{RVF17}\\
& \hyperlink{RVF18}{RVF18}\\
& \hyperlink{RVD19}{RVD19}\\
& \hyperlink{RVD20}{RVD20}\\
& \hyperlink{RVF21}{RVF21}\\
& \hyperlink{RVF22}{RVF22}\\
& \hyperlink{RVF23}{RVF23}\\
& \hyperlink{RVF24}{RVF24}\\
& \hyperlink{RVF25}{RVF25}\\
& \hyperlink{RVF26}{RVF26}\\ \hline
\hyperlink{Verbale 2016-01-11}{Verbale 2016-01-11} & \hyperlink{RFO8}{RFO8}\\
& \hyperlink{RFD8.1.3}{RFD8.1.3}\\
& \hyperlink{RFO9}{RFO9}\\
& \hyperlink{RFD11.5}{RFD11.5}\\
& \hyperlink{RFD11.5.1}{RFD11.5.1}\\
& \hyperlink{RFD11.6}{RFD11.6}\\
& \hyperlink{RFD11.6.1}{RFD11.6.1}\\
& \hyperlink{RFD11.7}{RFD11.7}\\
& \hyperlink{RFD11.7.1}{RFD11.7.1}\\ \hline
\hyperlink{Verbale interno}{Verbale interno} & \hyperlink{RFO2}{RFO2}\\
& \hyperlink{RFO2.1}{RFO2.1}\\ \hline
\hyperref[UC2]{UC2} & \hyperlink{RFO1}{RFO1}\\
& \hyperlink{RFO1.1}{RFO1.1}\\
& \hyperlink{RFO1.2}{RFO1.2}\\
& \hyperlink{RFO1.3}{RFO1.3}\\
& \hyperlink{RFO1.4}{RFO1.4}\\
& \hyperlink{RFO1.5}{RFO1.5}\\
& \hyperlink{RFO1.6}{RFO1.6}\\
& \hyperlink{RFO1.7}{RFO1.7}\\ \hline
\hyperref[UC2.1]{UC2.1} & \hyperlink{RFO1.1}{RFO1.1}\\ \hline
\hyperref[UC2.2]{UC2.2} & \hyperlink{RFO1.2}{RFO1.2}\\ \hline
\hyperref[UC2.3]{UC2.3} & \hyperlink{RFO1.3}{RFO1.3}\\ \hline
\hyperref[UC2.4]{UC2.4} & \hyperlink{RFO1.4}{RFO1.4}\\ \hline
\hyperref[UC2.5]{UC2.5} & \hyperlink{RFO1.5}{RFO1.5}\\ \hline
\hyperref[UC2.6]{UC2.6} & \hyperlink{RFO1.6}{RFO1.6}\\ \hline
\hyperref[UC2.7]{UC2.7} & \hyperlink{RFO1.7}{RFO1.7}\\ \hline
\hyperref[UC2.8]{UC2.8} & \hyperlink{RFO1.8}{RFO1.8}\\ \hline
\hyperref[UC3]{UC3} & \hyperlink{RFO2}{RFO2}\\
& \hyperlink{RFO2.1}{RFO2.1}\\
& \hyperlink{RFF2.2}{RFF2.2}\\
& \hyperlink{RFF2.3}{RFF2.3}\\
& \hyperlink{RFF2.4}{RFF2.4}\\
& \hyperlink{RFF2.5}{RFF2.5}\\ \hline
\hyperref[UC3.1]{UC3.1} & \hyperlink{RFO2.1}{RFO2.1}\\
& \hyperlink{RFO2.1.1}{RFO2.1.1}\\
& \hyperlink{RFO2.1.2}{RFO2.1.2}\\
& \hyperlink{RFO2.1.3}{RFO2.1.3}\\
& \hyperlink{RFO2.1.4}{RFO2.1.4}\\ \hline
\hyperref[UC3.1.1]{UC3.1.1} & \hyperlink{RFO2.1.1}{RFO2.1.1}\\ \hline
\hyperref[UC3.1.2]{UC3.1.2} & \hyperlink{RFO2.1.2}{RFO2.1.2}\\ \hline
\hyperref[UC3.1.3]{UC3.1.3} & \hyperlink{RFO2.1.3}{RFO2.1.3}\\ \hline
\hyperref[UC3.1.4]{UC3.1.4} & \hyperlink{RFO2.1.4}{RFO2.1.4}\\ \hline
\hyperref[UC3.2]{UC3.2} & \hyperlink{RFF2.2}{RFF2.2}\\ \hline
\hyperref[UC3.3]{UC3.3} & \hyperlink{RFF2.3}{RFF2.3}\\ \hline
\hyperref[UC3.4]{UC3.4} & \hyperlink{RFF2.4}{RFF2.4}\\ \hline
\hyperref[UC3.5]{UC3.5} & \hyperlink{RFF2.5}{RFF2.5}\\ \hline
\hyperref[UC4]{UC4} & \hyperlink{RFO3}{RFO3}\\
& \hyperlink{RFO3.1}{RFO3.1}\\ \hline
\hyperref[UC4.1]{UC4.1} & \hyperlink{RFO3.1}{RFO3.1}\\ \hline
\hyperref[UC5]{UC5} & \hyperlink{RFD4.1}{RFD4.1}\\
& \hyperlink{RFD4.2}{RFD4.2}\\
& \hyperlink{RFD4.3}{RFD4.3}\\
& \hyperlink{RFD4.4}{RFD4.4}\\
& \hyperlink{RFD4.5}{RFD4.5}\\
& \hyperlink{RFD4.6}{RFD4.6}\\
& \hyperlink{RFD4.7}{RFD4.7}\\
& \hyperlink{RFD4.8}{RFD4.8}\\ \hline
\hyperref[UC5.1]{UC5.1} & \hyperlink{RFD4.1}{RFD4.1}\\
& \hyperlink{RFD4.1.1}{RFD4.1.1}\\
& \hyperlink{RFD4.1.2}{RFD4.1.2}\\ \hline
\hyperref[UC5.1.2]{UC5.1.2} & \hyperlink{RFD4.1.1}{RFD4.1.1}\\ \hline
\hyperref[UC5.1.3]{UC5.1.3} & \hyperlink{RFD4.1.2}{RFD4.1.2}\\ \hline
\hyperref[UC5.2]{UC5.2} & \hyperlink{RFD4.2}{RFD4.2}\\
& \hyperlink{RFD4.2.1}{RFD4.2.1}\\
& \hyperlink{RFD4.2.2}{RFD4.2.2}\\ \hline
\hyperref[UC5.2.2]{UC5.2.2} & \hyperlink{RFD4.2.1}{RFD4.2.1}\\ \hline
\hyperref[UC5.2.3]{UC5.2.3} & \hyperlink{RFD4.2.2}{RFD4.2.2}\\ \hline
\hyperref[UC5.3]{UC5.3} & \hyperlink{RFD4.3}{RFD4.3}\\
& \hyperlink{RFD4.3.1}{RFD4.3.1}\\
& \hyperlink{RFD4.3.2}{RFD4.3.2}\\ \hline
\hyperref[UC5.3.2]{UC5.3.2} & \hyperlink{RFD4.3.1}{RFD4.3.1}\\ \hline
\hyperref[UC5.3.3]{UC5.3.3} & \hyperlink{RFD4.3.2}{RFD4.3.2}\\ \hline
\hyperref[UC5.4]{UC5.4} & \hyperlink{RFD4.4}{RFD4.4}\\
& \hyperlink{RFD4.4.1}{RFD4.4.1}\\
& \hyperlink{RFD4.4.2}{RFD4.4.2}\\ \hline
\hyperref[UC5.4.2]{UC5.4.2} & \hyperlink{RFD4.4.1}{RFD4.4.1}\\ \hline
\hyperref[UC5.4.3]{UC5.4.3} & \hyperlink{RFD4.4.2}{RFD4.4.2}\\ \hline
\hyperref[UC5.5]{UC5.5} & \hyperlink{RFD4.5}{RFD4.5}\\
& \hyperlink{RFD4.5.1}{RFD4.5.1}\\
& \hyperlink{RFD4.5.2}{RFD4.5.2}\\ \hline
\hyperref[UC5.5.2]{UC5.5.2} & \hyperlink{RFD4.5.1}{RFD4.5.1}\\ \hline
\hyperref[UC5.5.3]{UC5.5.3} & \hyperlink{RFD4.5.2}{RFD4.5.2}\\ \hline
\hyperref[UC5.6]{UC5.6} & \hyperlink{RFD4.6}{RFD4.6}\\
& \hyperlink{RFD4.6.1}{RFD4.6.1}\\
& \hyperlink{RFD4.6.2}{RFD4.6.2}\\ \hline
\hyperref[UC5.6.4]{UC5.6.4} & \hyperlink{RFD4.6.1}{RFD4.6.1}\\ \hline
\hyperref[UC5.6.5]{UC5.6.5} & \hyperlink{RFD4.6.2}{RFD4.6.2}\\ \hline
\hyperref[UC5.7]{UC5.7} & \hyperlink{RFD4.7}{RFD4.7}\\
& \hyperlink{RFD4.7.1}{RFD4.7.1}\\ \hline
\hyperref[UC5.7.2]{UC5.7.2} & \hyperlink{RFD4.7.1}{RFD4.7.1}\\ \hline
\hyperref[UC5.8]{UC5.8} & \hyperlink{RFD4.8}{RFD4.8}\\
& \hyperlink{RFD4.8.1}{RFD4.8.1}\\ \hline
\hyperref[UC5.8.1]{UC5.8.1} & \hyperlink{RFD4.8.1}{RFD4.8.1}\\ \hline
\hyperref[UC6]{UC6} & \hyperlink{RFD5}{RFD5}\\
& \hyperlink{RFD5.1}{RFD5.1}\\
& \hyperlink{RFD5.2}{RFD5.2}\\
& \hyperlink{RFD5.3}{RFD5.3}\\ \hline
\hyperref[UC6.1]{UC6.1} & \hyperlink{RFD5.1}{RFD5.1}\\ \hline
\hyperref[UC6.2]{UC6.2} & \hyperlink{RFD5.2}{RFD5.2}\\
& \hyperlink{RFD5.2.1}{RFD5.2.1}\\ \hline
\hyperref[UC6.2.1]{UC6.2.1} & \hyperlink{RFD5.2.1}{RFD5.2.1}\\ \hline
\hyperref[UC6.3]{UC6.3} & \hyperlink{RFD5.3}{RFD5.3}\\ \hline
\hyperref[UC7]{UC7} & \hyperlink{RFO6}{RFO6}\\
& \hyperlink{RFD6.1}{RFD6.1}\\
& \hyperlink{RFD6.2}{RFD6.2}\\
& \hyperlink{RFD6.3}{RFD6.3}\\
& \hyperlink{RFD6.4}{RFD6.4}\\ \hline
\hyperref[UC7.1]{UC7.1} & \hyperlink{RFD6.1}{RFD6.1}\\ \hline
\hyperref[UC7.2]{UC7.2} & \hyperlink{RFD6.2}{RFD6.2}\\ \hline
\hyperref[UC7.3]{UC7.3} & \hyperlink{RFD6.3}{RFD6.3}\\ \hline
\hyperref[UC7.4]{UC7.4} & \hyperlink{RFD6.4}{RFD6.4}\\ \hline
\hyperref[UC8]{UC8} & \hyperlink{RFO7.1}{RFO7.1}\\ \hline
\hyperref[UC8.1]{UC8.1} & \hyperlink{RFO7.1}{RFO7.1}\\
& \hyperlink{RFD7.1.1}{RFD7.1.1}\\
& \hyperlink{RFD7.1.2}{RFD7.1.2}\\
& \hyperlink{RFO7.1.3}{RFO7.1.3}\\
& \hyperlink{RFO7.1.4}{RFO7.1.4}\\
& \hyperlink{RFO7.1.5}{RFO7.1.5}\\ \hline
\hyperref[UC8.1.1]{UC8.1.1} & \hyperlink{RFD7.1.1}{RFD7.1.1}\\ \hline
\hyperref[UC8.1.2]{UC8.1.2} & \hyperlink{RFD7.1.2}{RFD7.1.2}\\ \hline
\hyperref[UC8.1.3]{UC8.1.3} & \hyperlink{RFO7.1.3}{RFO7.1.3}\\
& \hyperlink{RFO7.1.3.1}{RFO7.1.3.1}\\
& \hyperlink{RFO7.1.3.2}{RFO7.1.3.2}\\
& \hyperlink{RFD7.1.3.3}{RFD7.1.3.3}\\
& \hyperlink{RFD7.1.3.4}{RFD7.1.3.4}\\
& \hyperlink{RFD7.1.3.5}{RFD7.1.3.5}\\
& \hyperlink{RFD7.1.3.6}{RFD7.1.3.6}\\
& \hyperlink{RFD7.1.3.7}{RFD7.1.3.7}\\ \hline
\hyperref[UC8.1.3.1]{UC8.1.3.1} & \hyperlink{RFO7.1.3.1}{RFO7.1.3.1}\\
& \hyperlink{RFO7.1.3.1.1}{RFO7.1.3.1.1}\\
& \hyperlink{RFD7.1.3.1.2}{RFD7.1.3.1.2}\\
& \hyperlink{RFO7.1.3.1.3}{RFO7.1.3.1.3}\\ \hline
\hyperref[UC8.1.3.1.1]{UC8.1.3.1.1} & \hyperlink{RFO7.1.3.1.1}{RFO7.1.3.1.1}\\ \hline
\hyperref[UC8.1.3.1.2]{UC8.1.3.1.2} & \hyperlink{RFD7.1.3.1.2}{RFD7.1.3.1.2}\\
& \hyperlink{RFD7.1.3.1.2.1}{RFD7.1.3.1.2.1}\\
& \hyperlink{RFO7.1.3.2.1}{RFO7.1.3.2.1}\\ \hline
\hyperref[UC8.1.3.1.2.1]{UC8.1.3.1.2.1} & \hyperlink{RFD7.1.3.1.2.1}{RFD7.1.3.1.2.1}\\
& \hyperlink{RFO7.1.3.2.1}{RFO7.1.3.2.1}\\ \hline
\hyperref[UC8.1.3.1.3]{UC8.1.3.1.3} & \hyperlink{RFO7.1.3.1.3}{RFO7.1.3.1.3}\\ \hline
\hyperref[UC8.1.3.2]{UC8.1.3.2} & \hyperlink{RFO7.1.3.2}{RFO7.1.3.2}\\
& \hyperlink{RFD7.1.3.2.2}{RFD7.1.3.2.2}\\
& \hyperlink{RFO7.1.3.2.3}{RFO7.1.3.2.3}\\
& \hyperlink{RFO7.1.3.2.4}{RFO7.1.3.2.4}\\ \hline
\hyperref[UC8.1.3.2.2]{UC8.1.3.2.2} & \hyperlink{RFD7.1.3.2.2}{RFD7.1.3.2.2}\\
& \hyperlink{RFD7.1.3.2.2.1}{RFD7.1.3.2.2.1}\\ \hline
\hyperref[UC8.1.3.2.2.1]{UC8.1.3.2.2.1} & \hyperlink{RFD7.1.3.2.2.1}{RFD7.1.3.2.2.1}\\ \hline
\hyperref[UC8.1.3.2.3]{UC8.1.3.2.3} & \hyperlink{RFO7.1.3.2.3}{RFO7.1.3.2.3}\\
& \hyperlink{RFO7.1.3.2.3.1}{RFO7.1.3.2.3.1}\\
& \hyperlink{RFD7.1.3.2.3.2}{RFD7.1.3.2.3.2}\\ \hline
\hyperref[UC8.1.3.2.3.1]{UC8.1.3.2.3.1} & \hyperlink{RFO7.1.3.2.3.1}{RFO7.1.3.2.3.1}\\
& \hyperlink{RFD7.1.3.2.3.1.1}{RFD7.1.3.2.3.1.1}\\ \hline
\hyperref[UC8.1.3.2.3.1.1]{UC8.1.3.2.3.1.1} & \hyperlink{RFD7.1.3.2.3.1.1}{RFD7.1.3.2.3.1.1}\\ \hline
\hyperref[UC8.1.3.2.3.2]{UC8.1.3.2.3.2} & \hyperlink{RFD7.1.3.2.3.2}{RFD7.1.3.2.3.2}\\
& \hyperlink{RFD7.1.3.2.3.2.1}{RFD7.1.3.2.3.2.1}\\ \hline
\hyperref[UC8.1.3.2.3.2.1]{UC8.1.3.2.3.2.1} & \hyperlink{RFD7.1.3.2.3.2.1}{RFD7.1.3.2.3.2.1}\\ \hline
\hyperref[UC8.1.3.2.4]{UC8.1.3.2.4} & \hyperlink{RFO7.1.3.2.4}{RFO7.1.3.2.4}\\ \hline
\hyperref[UC8.1.3.3]{UC8.1.3.3} & \hyperlink{RFD7.1.3.3}{RFD7.1.3.3}\\
& \hyperlink{RFD7.1.3.3.1}{RFD7.1.3.3.1}\\
& \hyperlink{RFD7.1.3.3.2}{RFD7.1.3.3.2}\\ \hline
\hyperref[UC8.1.3.3.1]{UC8.1.3.3.1} & \hyperlink{RFD7.1.3.3.1}{RFD7.1.3.3.1}\\ \hline
\hyperref[UC8.1.3.3.2]{UC8.1.3.3.2} & \hyperlink{RFD7.1.3.3.2}{RFD7.1.3.3.2}\\ \hline
\hyperref[UC8.1.3.4]{UC8.1.3.4} & \hyperlink{RFD7.1.3.4}{RFD7.1.3.4}\\
& \hyperlink{RFD7.1.3.4.1}{RFD7.1.3.4.1}\\
& \hyperlink{RFD7.1.3.4.2}{RFD7.1.3.4.2}\\
& \hyperlink{RFD7.1.3.4.3}{RFD7.1.3.4.3}\\
& \hyperlink{RFD7.1.3.4.4}{RFD7.1.3.4.4}\\ \hline
\hyperref[UC8.1.3.4.1]{UC8.1.3.4.1} & \hyperlink{RFD7.1.3.4.1}{RFD7.1.3.4.1}\\ \hline
\hyperref[UC8.1.3.4.2]{UC8.1.3.4.2} & \hyperlink{RFD7.1.3.4.2}{RFD7.1.3.4.2}\\
& \hyperlink{RFD7.1.3.4.2.1}{RFD7.1.3.4.2.1}\\
& \hyperlink{RFD7.1.3.4.2.2}{RFD7.1.3.4.2.2}\\
& \hyperlink{RFD7.1.3.4.2.3}{RFD7.1.3.4.2.3}\\
& \hyperlink{RFD7.1.3.4.2.4}{RFD7.1.3.4.2.4}\\ \hline
\hyperref[UC8.1.3.4.2.1]{UC8.1.3.4.2.1} & \hyperlink{RFD7.1.3.4.2.1}{RFD7.1.3.4.2.1}\\ \hline
\hyperref[UC8.1.3.4.2.2]{UC8.1.3.4.2.2} & \hyperlink{RFD7.1.3.4.2.2}{RFD7.1.3.4.2.2}\\ \hline
\hyperref[UC8.1.3.4.2.3]{UC8.1.3.4.2.3} & \hyperlink{RFD7.1.3.4.2.3}{RFD7.1.3.4.2.3}\\ \hline
\hyperref[UC8.1.3.4.2.4]{UC8.1.3.4.2.4} & \hyperlink{RFD7.1.3.4.2.4}{RFD7.1.3.4.2.4}\\ \hline
\hyperref[UC8.1.3.4.3]{UC8.1.3.4.3} & \hyperlink{RFD7.1.3.4.3}{RFD7.1.3.4.3}\\
& \hyperlink{RFD7.1.3.4.3.1}{RFD7.1.3.4.3.1}\\ \hline
\hyperref[UC8.1.3.4.3.1]{UC8.1.3.4.3.1} & \hyperlink{RFD7.1.3.4.3.1}{RFD7.1.3.4.3.1}\\ \hline
\hyperref[UC8.1.3.4.4]{UC8.1.3.4.4} & \hyperlink{RFD7.1.3.4.4}{RFD7.1.3.4.4}\\
& \hyperlink{RFD7.1.3.4.4.1}{RFD7.1.3.4.4.1}\\
& \hyperlink{RFD7.1.3.4.4.2}{RFD7.1.3.4.4.2}\\
& \hyperlink{RFD7.1.3.4.4.3}{RFD7.1.3.4.4.3}\\
& \hyperlink{RFD7.1.3.4.4.4}{RFD7.1.3.4.4.4}\\ \hline
\hyperref[UC8.1.3.4.4.1]{UC8.1.3.4.4.1} & \hyperlink{RFD7.1.3.4.4.1}{RFD7.1.3.4.4.1}\\ \hline
\hyperref[UC8.1.3.4.4.2]{UC8.1.3.4.4.2} & \hyperlink{RFD7.1.3.4.4.2}{RFD7.1.3.4.4.2}\\ \hline
\hyperref[UC8.1.3.4.4.3]{UC8.1.3.4.4.3} & \hyperlink{RFD7.1.3.4.4.3}{RFD7.1.3.4.4.3}\\ \hline
\hyperref[UC8.1.3.4.4.4]{UC8.1.3.4.4.4} & \hyperlink{RFD7.1.3.4.4.4}{RFD7.1.3.4.4.4}\\ \hline
\hyperref[UC8.1.3.5]{UC8.1.3.5} & \hyperlink{RFD7.1.3.5}{RFD7.1.3.5}\\
& \hyperlink{RFD7.1.3.5.1}{RFD7.1.3.5.1}\\
& \hyperlink{RFD7.1.3.5.2}{RFD7.1.3.5.2}\\
& \hyperlink{RFD7.1.3.5.3}{RFD7.1.3.5.3}\\ \hline
\hyperref[UC8.1.3.5.1]{UC8.1.3.5.1} & \hyperlink{RFD7.1.3.5.1}{RFD7.1.3.5.1}\\ \hline
\hyperref[UC8.1.3.5.2]{UC8.1.3.5.2} & \hyperlink{RFD7.1.3.5.2}{RFD7.1.3.5.2}\\
& \hyperlink{RFD7.1.3.5.2.1}{RFD7.1.3.5.2.1}\\ \hline
\hyperref[UC8.1.3.5.2.1]{UC8.1.3.5.2.1} & \hyperlink{RFD7.1.3.5.2.1}{RFD7.1.3.5.2.1}\\ \hline
\hyperref[UC8.1.3.5.3]{UC8.1.3.5.3} & \hyperlink{RFD7.1.3.5.3}{RFD7.1.3.5.3}\\
& \hyperlink{RFD7.1.3.5.3.1}{RFD7.1.3.5.3.1}\\ \hline
\hyperref[UC8.1.3.5.3.1]{UC8.1.3.5.3.1} & \hyperlink{RFD7.1.3.5.3.1}{RFD7.1.3.5.3.1}\\ \hline
\hyperref[UC8.1.3.6]{UC8.1.3.6} & \hyperlink{RFD7.1.3.6}{RFD7.1.3.6}\\
& \hyperlink{RFD7.1.3.6.1}{RFD7.1.3.6.1}\\
& \hyperlink{RFD7.1.3.6.2}{RFD7.1.3.6.2}\\
& \hyperlink{RFD7.1.3.6.3}{RFD7.1.3.6.3}\\ \hline
\hyperref[UC8.1.3.6.1]{UC8.1.3.6.1} & \hyperlink{RFD7.1.3.6.1}{RFD7.1.3.6.1}\\ \hline
\hyperref[UC8.1.3.6.2]{UC8.1.3.6.2} & \hyperlink{RFD7.1.3.6.2}{RFD7.1.3.6.2}\\ \hline
\hyperref[UC8.1.3.6.3]{UC8.1.3.6.3} & \hyperlink{RFD7.1.3.6.3}{RFD7.1.3.6.3}\\ \hline
\hyperref[UC8.1.3.7]{UC8.1.3.7} & \hyperlink{RFD7.1.3.7}{RFD7.1.3.7}\\
& \hyperlink{RFD7.1.3.7.1}{RFD7.1.3.7.1}\\
& \hyperlink{RFD7.1.3.7.2}{RFD7.1.3.7.2}\\
& \hyperlink{RFD7.1.3.7.3}{RFD7.1.3.7.3}\\
& \hyperlink{RFD7.1.3.7.4}{RFD7.1.3.7.4}\\ \hline
\hyperref[UC8.1.3.7.1]{UC8.1.3.7.1} & \hyperlink{RFD7.1.3.7.1}{RFD7.1.3.7.1}\\ \hline
\hyperref[UC8.1.3.7.2]{UC8.1.3.7.2} & \hyperlink{RFD7.1.3.7.2}{RFD7.1.3.7.2}\\ \hline
\hyperref[UC8.1.3.7.3]{UC8.1.3.7.3} & \hyperlink{RFD7.1.3.7.3}{RFD7.1.3.7.3}\\ \hline
\hyperref[UC8.1.3.7.4]{UC8.1.3.7.4} & \hyperlink{RFD7.1.3.7.4}{RFD7.1.3.7.4}\\ \hline
\hyperref[UC8.1.4]{UC8.1.4} & \hyperlink{RFO7.1.4}{RFO7.1.4}\\ \hline
\hyperref[UC8.1.5]{UC8.1.5} & \hyperlink{RFO7.1.5}{RFO7.1.5}\\ \hline
\hyperref[UC8.2]{UC8.2} & \hyperlink{RFD7.2}{RFD7.2}\\
& \hyperlink{RFD7.2.1}{RFD7.2.1}\\
& \hyperlink{RFD7.2.2}{RFD7.2.2}\\
& \hyperlink{RFD7.2.3}{RFD7.2.3}\\ \hline
\hyperref[UC8.2.1]{UC8.2.1} & \hyperlink{RFD7.2.1}{RFD7.2.1}\\
& \hyperlink{RFD7.2.1.1}{RFD7.2.1.1}\\
& \hyperlink{RFD7.2.1.4}{RFD7.2.1.4}\\
& \hyperlink{RFD7.2.1.5}{RFD7.2.1.5}\\
& \hyperlink{RFD7.2.1.6}{RFD7.2.1.6}\\
& \hyperlink{RFD7.2.1.7}{RFD7.2.1.7}\\
& \hyperlink{RFD7.2.1.8}{RFD7.2.1.8}\\ \hline
\hyperref[UC8.2.1.1]{UC8.2.1.1} & \hyperlink{RFD7.2.1.1}{RFD7.2.1.1}\\
& \hyperlink{RFD7.2.1.1.1}{RFD7.2.1.1.1}\\
& \hyperlink{RFD7.2.1.1.2}{RFD7.2.1.1.2}\\
& \hyperlink{RFD7.2.1.1.3}{RFD7.2.1.1.3}\\ \hline
\hyperref[UC8.2.1.1.1]{UC8.2.1.1.1} & \hyperlink{RFD7.2.1.1.1}{RFD7.2.1.1.1}\\ \hline
\hyperref[UC8.2.1.1.2]{UC8.2.1.1.2} & \hyperlink{RFD7.2.1.1.2}{RFD7.2.1.1.2}\\ \hline
\hyperref[UC8.2.1.1.3]{UC8.2.1.1.3} & \hyperlink{RFD7.2.1.1.3}{RFD7.2.1.1.3}\\ \hline
\hyperref[UC8.2.1.2]{UC8.2.1.2} & \hyperlink{RFD7.2.1.2}{RFD7.2.1.2}\\
& \hyperlink{RFD7.2.1.2.1}{RFD7.2.1.2.1}\\
& \hyperlink{RFD7.2.1.2.2}{RFD7.2.1.2.2}\\
& \hyperlink{RFD7.2.1.2.3}{RFD7.2.1.2.3}\\
& \hyperlink{RFD7.2.1.2.4}{RFD7.2.1.2.4}\\ \hline
\hyperref[UC8.2.1.2.1]{UC8.2.1.2.1} & \hyperlink{RFD7.2.1.2.1}{RFD7.2.1.2.1}\\ \hline
\hyperref[UC8.2.1.2.2]{UC8.2.1.2.2} & \hyperlink{RFD7.2.1.2.2}{RFD7.2.1.2.2}\\ \hline
\hyperref[UC8.2.1.2.3]{UC8.2.1.2.3} & \hyperlink{RFD7.2.1.2.3}{RFD7.2.1.2.3}\\
& \hyperlink{RFD7.2.1.2.3.1}{RFD7.2.1.2.3.1}\\
& \hyperlink{RFD7.2.1.2.3.2}{RFD7.2.1.2.3.2}\\ \hline
\hyperref[UC8.2.1.2.3.1]{UC8.2.1.2.3.1} & \hyperlink{RFD7.2.1.2.3.1}{RFD7.2.1.2.3.1}\\
& \hyperlink{RFD7.2.1.2.3.2}{RFD7.2.1.2.3.2}\\ \hline
\hyperref[UC8.2.1.2.4]{UC8.2.1.2.4} & \hyperlink{RFD7.2.1.2.4}{RFD7.2.1.2.4}\\ \hline
\hyperref[UC8.2.1.3]{UC8.2.1.3} & \hyperlink{RFD7.2.1.3}{RFD7.2.1.3}\\
& \hyperlink{RFD7.2.1.3.1}{RFD7.2.1.3.1}\\
& \hyperlink{RFD7.2.1.3.2}{RFD7.2.1.3.2}\\ \hline
\hyperref[UC8.2.1.3.1]{UC8.2.1.3.1} & \hyperlink{RFD7.2.1.3.1}{RFD7.2.1.3.1}\\ \hline
\hyperref[UC8.2.1.3.2]{UC8.2.1.3.2} & \hyperlink{RFD7.2.1.3.2}{RFD7.2.1.3.2}\\ \hline
\hyperref[UC8.2.1.4]{UC8.2.1.4} & \hyperlink{RFD7.2.1.4}{RFD7.2.1.4}\\
& \hyperlink{RFD7.2.1.4.1}{RFD7.2.1.4.1}\\
& \hyperlink{RFD7.2.1.4.2}{RFD7.2.1.4.2}\\
& \hyperlink{RFD7.2.1.4.3}{RFD7.2.1.4.3}\\
& \hyperlink{RFD7.2.1.4.4}{RFD7.2.1.4.4}\\
& \hyperlink{RFD7.2.1.4.5}{RFD7.2.1.4.5}\\
& \hyperlink{RFD7.2.1.4.6}{RFD7.2.1.4.6}\\ \hline
\hyperref[UC8.2.1.4.1]{UC8.2.1.4.1} & \hyperlink{RFD7.2.1.4.1}{RFD7.2.1.4.1}\\
& \hyperlink{RFD7.2.1.4.1.1}{RFD7.2.1.4.1.1}\\
& \hyperlink{RFD7.2.1.4.1.2}{RFD7.2.1.4.1.2}\\
& \hyperlink{RFD7.2.1.4.1.3}{RFD7.2.1.4.1.3}\\
& \hyperlink{RFD7.2.1.4.1.4}{RFD7.2.1.4.1.4}\\ \hline
\hyperref[UC8.2.1.4.1.1]{UC8.2.1.4.1.1} & \hyperlink{RFD7.2.1.4.1.1}{RFD7.2.1.4.1.1}\\ \hline
\hyperref[UC8.2.1.4.1.2]{UC8.2.1.4.1.2} & \hyperlink{RFD7.2.1.4.1.2}{RFD7.2.1.4.1.2}\\ \hline
\hyperref[UC8.2.1.4.1.3]{UC8.2.1.4.1.3} & \hyperlink{RFD7.2.1.4.1.3}{RFD7.2.1.4.1.3}\\ \hline
\hyperref[UC8.2.1.4.1.4]{UC8.2.1.4.1.4} & \hyperlink{RFD7.2.1.4.1.4}{RFD7.2.1.4.1.4}\\ \hline
\hyperref[UC8.2.1.4.2]{UC8.2.1.4.2} & \hyperlink{RFD7.2.1.4.2}{RFD7.2.1.4.2}\\
& \hyperlink{RFD7.2.1.4.2.1}{RFD7.2.1.4.2.1}\\
& \hyperlink{RFD7.2.1.4.3}{RFD7.2.1.4.3}\\
& \hyperlink{RFD7.2.1.4.4}{RFD7.2.1.4.4}\\ \hline
\hyperref[UC8.2.1.4.2.1]{UC8.2.1.4.2.1} & \hyperlink{RFD7.2.1.4.2.1}{RFD7.2.1.4.2.1}\\ \hline
\hyperref[UC8.2.1.4.3]{UC8.2.1.4.3} & \hyperlink{RFD7.2.1.4.3.1}{RFD7.2.1.4.3.1}\\
& \hyperlink{RFD7.2.1.4.3.2}{RFD7.2.1.4.3.2}\\
& \hyperlink{RFD7.2.1.4.3.3}{RFD7.2.1.4.3.3}\\
& \hyperlink{RFD7.2.1.4.3.4}{RFD7.2.1.4.3.4}\\
& \hyperlink{RFD7.2.1.4.5}{RFD7.2.1.4.5}\\ \hline
\hyperref[UC8.2.1.4.3.1]{UC8.2.1.4.3.1} & \hyperlink{RFD7.2.1.4.3.1}{RFD7.2.1.4.3.1}\\ \hline
\hyperref[UC8.2.1.4.3.2]{UC8.2.1.4.3.2} & \hyperlink{RFD7.2.1.4.3.2}{RFD7.2.1.4.3.2}\\ \hline
\hyperref[UC8.2.1.4.3.3]{UC8.2.1.4.3.3} & \hyperlink{RFD7.2.1.4.3.3}{RFD7.2.1.4.3.3}\\ \hline
\hyperref[UC8.2.1.4.3.4]{UC8.2.1.4.3.4} & \hyperlink{RFD7.2.1.4.3.4}{RFD7.2.1.4.3.4}\\ \hline
\hyperref[UC8.2.1.4.4]{UC8.2.1.4.4} & \hyperlink{RFD7.2.1.4.6}{RFD7.2.1.4.6}\\ \hline
\hyperref[UC8.2.1.5]{UC8.2.1.5} & \hyperlink{RFD7.2.1.5}{RFD7.2.1.5}\\
& \hyperlink{RFD7.2.1.5.1}{RFD7.2.1.5.1}\\
& \hyperlink{RFD7.2.1.5.2}{RFD7.2.1.5.2}\\
& \hyperlink{RFD7.2.1.5.3}{RFD7.2.1.5.3}\\
& \hyperlink{RFD7.2.1.5.4}{RFD7.2.1.5.4}\\ \hline
\hyperref[UC8.2.1.5.1]{UC8.2.1.5.1} & \hyperlink{RFD7.2.1.5.1}{RFD7.2.1.5.1}\\ \hline
\hyperref[UC8.2.1.5.2]{UC8.2.1.5.2} & \hyperlink{RFD7.2.1.5.2}{RFD7.2.1.5.2}\\ \hline
\hyperref[UC8.2.1.5.3]{UC8.2.1.5.3} & \hyperlink{RFD7.2.1.5.3}{RFD7.2.1.5.3}\\ \hline
\hyperref[UC8.2.1.5.4]{UC8.2.1.5.4} & \hyperlink{RFD7.2.1.5.4}{RFD7.2.1.5.4}\\ \hline
\hyperref[UC8.2.1.6]{UC8.2.1.6} & \hyperlink{RFD7.2.1.6}{RFD7.2.1.6}\\
& \hyperlink{RFD7.2.1.6.1}{RFD7.2.1.6.1}\\
& \hyperlink{RFD7.2.1.6.2}{RFD7.2.1.6.2}\\
& \hyperlink{RFD7.2.1.6.3}{RFD7.2.1.6.3}\\ \hline
\hyperref[UC8.2.1.6.1]{UC8.2.1.6.1} & \hyperlink{RFD7.2.1.6.1}{RFD7.2.1.6.1}\\ \hline
\hyperref[UC8.2.1.6.2]{UC8.2.1.6.2} & \hyperlink{RFD7.2.1.6.2}{RFD7.2.1.6.2}\\ \hline
\hyperref[UC8.2.1.6.3]{UC8.2.1.6.3} & \hyperlink{RFD7.2.1.6.3}{RFD7.2.1.6.3}\\ \hline
\hyperref[UC8.2.1.7]{UC8.2.1.7} & \hyperlink{RFD7.2.1.7}{RFD7.2.1.7}\\
& \hyperlink{RFD7.2.1.7.1}{RFD7.2.1.7.1}\\
& \hyperlink{RFD7.2.1.7.2}{RFD7.2.1.7.2}\\
& \hyperlink{RFD7.2.1.7.3}{RFD7.2.1.7.3}\\
& \hyperlink{RFD7.2.1.7.4}{RFD7.2.1.7.4}\\
& \hyperlink{RFD7.2.1.8}{RFD7.2.1.8}\\ \hline
\hyperref[UC8.2.1.7.1]{UC8.2.1.7.1} & \hyperlink{RFD7.2.1.7.1}{RFD7.2.1.7.1}\\ \hline
\hyperref[UC8.2.1.7.2]{UC8.2.1.7.2} & \hyperlink{RFD7.2.1.7.2}{RFD7.2.1.7.2}\\ \hline
\hyperref[UC8.2.1.7.3]{UC8.2.1.7.3} & \hyperlink{RFD7.2.1.7.3}{RFD7.2.1.7.3}\\ \hline
\hyperref[UC8.2.1.7.4]{UC8.2.1.7.4} & \hyperlink{RFD7.2.1.7.4}{RFD7.2.1.7.4}\\ \hline
\hyperref[UC8.2.2]{UC8.2.2} & \hyperlink{RFD7.2.2}{RFD7.2.2}\\ \hline
\hyperref[UC8.2.3]{UC8.2.3} & \hyperlink{RFD7.2.3}{RFD7.2.3}\\ \hline
\hyperref[UC9]{UC9} & \hyperlink{RFO8}{RFO8}\\
& \hyperlink{RFD8.1}{RFD8.1}\\
& \hyperlink{RFD8.4}{RFD8.4}\\ \hline
\hyperref[UC9.1]{UC9.1} & \hyperlink{RFD8.1}{RFD8.1}\\
& \hyperlink{RFD8.1.1}{RFD8.1.1}\\
& \hyperlink{RFD8.1.2}{RFD8.1.2}\\
& \hyperlink{RFD8.1.3}{RFD8.1.3}\\
& \hyperlink{RFD8.1.4}{RFD8.1.4}\\ \hline
\hyperref[UC9.1.1]{UC9.1.1} & \hyperlink{RFD8.1.1}{RFD8.1.1}\\
& \hyperlink{RFD8.1.1.1}{RFD8.1.1.1}\\
& \hyperlink{RFD8.1.1.2}{RFD8.1.1.2}\\
& \hyperlink{RFO8.3}{RFO8.3}\\
& \hyperlink{RFO8.3.1}{RFO8.3.1}\\
& \hyperlink{RFO8.3.1.1}{RFO8.3.1.1}\\
& \hyperlink{RFO8.3.1.1.1}{RFO8.3.1.1.1}\\
& \hyperlink{RFD8.3.1.1.2}{RFD8.3.1.1.2}\\
& \hyperlink{RFO8.3.2}{RFO8.3.2}\\
& \hyperlink{RFO8.3.2.1}{RFO8.3.2.1}\\ \hline
\hyperref[UC9.1.1.1]{UC9.1.1.1} & \hyperlink{RFD8.1.1.1}{RFD8.1.1.1}\\ \hline
\hyperref[UC9.1.1.2]{UC9.1.1.2} & \hyperlink{RFD8.1.1.2}{RFD8.1.1.2}\\ \hline
\hyperref[UC9.1.2]{UC9.1.2} & \hyperlink{RFD8.1.2}{RFD8.1.2}\\
& \hyperlink{RFD8.1.2.1}{RFD8.1.2.1}\\ \hline
\hyperref[UC9.1.2.1]{UC9.1.2.1} & \hyperlink{RFD8.1.2.1}{RFD8.1.2.1}\\ \hline
\hyperref[UC9.1.3]{UC9.1.3} & \hyperlink{RFD8.1.3}{RFD8.1.3}\\ \hline
\hyperref[UC9.1.4]{UC9.1.4} & \hyperlink{RFD8.1.4}{RFD8.1.4}\\ \hline
\hyperref[UC9.2]{UC9.2} & \hyperlink{RFO8.2}{RFO8.2}\\
& \hyperlink{RFO8.2.1}{RFO8.2.1}\\
& \hyperlink{RFD8.2.2}{RFD8.2.2}\\
& \hyperlink{RFD8.2.3}{RFD8.2.3}\\
& \hyperlink{RFO8.2.4}{RFO8.2.4}\\
& \hyperlink{RFO8.3}{RFO8.3}\\
& \hyperlink{RFO8.3.1}{RFO8.3.1}\\
& \hyperlink{RFO8.3.1.1}{RFO8.3.1.1}\\
& \hyperlink{RFO8.3.1.1.1}{RFO8.3.1.1.1}\\
& \hyperlink{RFD8.3.1.1.2}{RFD8.3.1.1.2}\\
& \hyperlink{RFO8.3.2}{RFO8.3.2}\\
& \hyperlink{RFO8.3.2.1}{RFO8.3.2.1}\\ \hline
\hyperref[UC9.2.1]{UC9.2.1} & \hyperlink{RFO8.2.1}{RFO8.2.1}\\ \hline
\hyperref[UC9.2.2]{UC9.2.2} & \hyperlink{RFD8.2.2}{RFD8.2.2}\\ \hline
\hyperref[UC9.2.3]{UC9.2.3} & \hyperlink{RFD8.2.3}{RFD8.2.3}\\ \hline
\hyperref[UC9.2.4]{UC9.2.4} & \hyperlink{RFO8.2.4}{RFO8.2.4}\\
& \hyperlink{RFD8.2.4.1}{RFD8.2.4.1}\\
& \hyperlink{RFO8.2.4.2}{RFO8.2.4.2}\\ \hline
\hyperref[UC9.2.4.1]{UC9.2.4.1} & \hyperlink{RFD8.2.4.1}{RFD8.2.4.1}\\ \hline
\hyperref[UC9.2.4.2]{UC9.2.4.2} & \hyperlink{RFO8.2.4.2}{RFO8.2.4.2}\\ \hline
\hyperref[UC9.3]{UC9.3} & \hyperlink{RFO8.3}{RFO8.3}\\
& \hyperlink{RFO8.3.1}{RFO8.3.1}\\
& \hyperlink{RFO8.3.2}{RFO8.3.2}\\ \hline
\hyperref[UC9.3.1]{UC9.3.1} & \hyperlink{RFO8.3.1}{RFO8.3.1}\\
& \hyperlink{RFO8.3.1.1}{RFO8.3.1.1}\\ \hline
\hyperref[UC9.3.1.1]{UC9.3.1.1} & \hyperlink{RFO8.3.1.1}{RFO8.3.1.1}\\
& \hyperlink{RFO8.3.1.1.1}{RFO8.3.1.1.1}\\
& \hyperlink{RFD8.3.1.1.2}{RFD8.3.1.1.2}\\ \hline
\hyperref[UC9.3.1.1.1]{UC9.3.1.1.1} & \hyperlink{RFO8.3.1.1.1}{RFO8.3.1.1.1}\\ \hline
\hyperref[UC9.3.1.1.2]{UC9.3.1.1.2} & \hyperlink{RFD8.3.1.1.2}{RFD8.3.1.1.2}\\ \hline
\hyperref[UC9.3.2]{UC9.3.2} & \hyperlink{RFO8.3.2}{RFO8.3.2}\\
& \hyperlink{RFO8.3.2.1}{RFO8.3.2.1}\\ \hline
\hyperref[UC9.3.2.1]{UC9.3.2.1} & \hyperlink{RFO8.3.2.1}{RFO8.3.2.1}\\ \hline
\hyperref[UC9.4]{UC9.4} & \hyperlink{RFD8.4}{RFD8.4}\\
& \hyperlink{RFD8.4.1}{RFD8.4.1}\\ \hline
\hyperref[UC9.4.1]{UC9.4.1} & \hyperlink{RFD8.4.1}{RFD8.4.1}\\
& \hyperlink{RFD8.4.1.1}{RFD8.4.1.1}\\ \hline
\hyperref[UC9.4.1.1]{UC9.4.1.1} & \hyperlink{RFD8.4.1.1}{RFD8.4.1.1}\\ \hline
\hyperref[UC10]{UC10} & \hyperlink{RFO9}{RFO9}\\
& \hyperlink{RFD9.1}{RFD9.1}\\
& \hyperlink{RFD9.2}{RFD9.2}\\
& \hyperlink{RFD9.3}{RFD9.3}\\
& \hyperlink{RFD9.4}{RFD9.4}\\
& \hyperlink{RFD24}{RFD24}\\
& \hyperlink{RFD25}{RFD25}\\
& \hyperlink{RFD26}{RFD26}\\
& \hyperlink{RFD27}{RFD27}\\ \hline
\hyperref[UC10.1]{UC10.1} & \hyperlink{RFD9.1}{RFD9.1}\\ \hline
\hyperref[UC10.2]{UC10.2} & \hyperlink{RFD9.2}{RFD9.2}\\ \hline
\hyperref[UC10.3]{UC10.3} & \hyperlink{RFD9.3}{RFD9.3}\\ \hline
\hyperref[UC10.4]{UC10.4} & \hyperlink{RFD9.4}{RFD9.4}\\
& \hyperlink{RFD9.4.1}{RFD9.4.1}\\
& \hyperlink{RFD9.4.2}{RFD9.4.2}\\
& \hyperlink{RFD9.4.3}{RFD9.4.3}\\
& \hyperlink{RFD9.4.4}{RFD9.4.4}\\
& \hyperlink{RFD9.4.5}{RFD9.4.5}\\
& \hyperlink{RFD9.4.6}{RFD9.4.6}\\
& \hyperlink{RFD9.4.7}{RFD9.4.7}\\ \hline
\hyperref[UC10.4.1]{UC10.4.1} & \hyperlink{RFD9.4.1}{RFD9.4.1}\\ \hline
\hyperref[UC10.4.2]{UC10.4.2} & \hyperlink{RFD9.4.2}{RFD9.4.2}\\ \hline
\hyperref[UC10.4.3]{UC10.4.3} & \hyperlink{RFD9.4.3}{RFD9.4.3}\\ \hline
\hyperref[UC10.4.3.1]{UC10.4.3.1} & \hyperlink{RFD9.4.3}{RFD9.4.3}\\ \hline
\hyperref[UC10.4.3.2]{UC10.4.3.2} & \hyperlink{RFD9.4.3}{RFD9.4.3}\\ \hline
\hyperref[UC10.4.4]{UC10.4.4} & \hyperlink{RFD9.4.4}{RFD9.4.4}\\ \hline
\hyperref[UC10.4.4.1]{UC10.4.4.1} & \hyperlink{RFD9.4.4}{RFD9.4.4}\\ \hline
\hyperref[UC10.4.4.2]{UC10.4.4.2} & \hyperlink{RFD9.4.4}{RFD9.4.4}\\ \hline
\hyperref[UC10.4.5]{UC10.4.5} & \hyperlink{RFD9.4.5}{RFD9.4.5}\\ \hline
\hyperref[UC10.4.6]{UC10.4.6} & \hyperlink{RFD9.4.6}{RFD9.4.6}\\
& \hyperlink{RFD9.4.7}{RFD9.4.7}\\ \hline
\hyperref[UC11]{UC11} & \hyperlink{RFD10}{RFD10}\\
& \hyperlink{RFD10.1}{RFD10.1}\\
& \hyperlink{RFD10.2}{RFD10.2}\\
& \hyperlink{RFD10.3}{RFD10.3}\\
& \hyperlink{RFD10.4}{RFD10.4}\\
& \hyperlink{RFD10.5}{RFD10.5}\\ \hline
\hyperref[UC11.1]{UC11.1} & \hyperlink{RFD10.1}{RFD10.1}\\ \hline
\hyperref[UC11.2]{UC11.2} & \hyperlink{RFD10.2}{RFD10.2}\\ \hline
\hyperref[UC11.3]{UC11.3} & \hyperlink{RFD10.3}{RFD10.3}\\ \hline
\hyperref[UC11.4.1]{UC11.4.1} & \hyperlink{RFD10.4.1}{RFD10.4.1}\\ \hline
\hyperref[UC11.5]{UC11.5} & \hyperlink{RFD10.5}{RFD10.5}\\
& \hyperlink{RFD10.5.1}{RFD10.5.1}\\ \hline
\hyperref[UC11.5.1]{UC11.5.1} & \hyperlink{RFD10.5.1}{RFD10.5.1}\\ \hline
\hyperref[UC11.6]{UC11.6} & \hyperlink{RFD10.6}{RFD10.6}\\ \hline
\hyperref[UC12]{UC12} & \hyperlink{RFO11}{RFO11}\\
& \hyperlink{RFO11.1}{RFO11.1}\\
& \hyperlink{RFO11.2}{RFO11.2}\\
& \hyperlink{RFO11.3}{RFO11.3}\\
& \hyperlink{RFD11.4}{RFD11.4}\\
& \hyperlink{RFD11.5}{RFD11.5}\\
& \hyperlink{RFD11.6}{RFD11.6}\\
& \hyperlink{RFD11.7}{RFD11.7}\\ \hline
\hyperref[UC12.1]{UC12.1} & \hyperlink{RFO11.1}{RFO11.1}\\ \hline
\hyperref[UC12.2]{UC12.2} & \hyperlink{RFO11.2}{RFO11.2}\\ \hline
\hyperref[UC12.3]{UC12.3} & \hyperlink{RFO11.3}{RFO11.3}\\
& \hyperlink{RFO11.3.1}{RFO11.3.1}\\ \hline
\hyperref[UC12.3.1]{UC12.3.1} & \hyperlink{RFO11.3.1}{RFO11.3.1}\\ \hline
\hyperref[UC12.4]{UC12.4} & \hyperlink{RFD10.4}{RFD10.4}\\
& \hyperlink{RFD10.4.1}{RFD10.4.1}\\
& \hyperlink{RFD11.4}{RFD11.4}\\
& \hyperlink{RFD11.4.1}{RFD11.4.1}\\ \hline
\hyperref[UC12.4.1]{UC12.4.1} & \hyperlink{RFD11.4.1}{RFD11.4.1}\\ \hline
\hyperref[UC12.4.1.1]{UC12.4.1.1} & \hyperlink{RFD11.4.1}{RFD11.4.1}\\ \hline
\hyperref[UC12.4.1.2]{UC12.4.1.2} & \hyperlink{RFD11.4.1}{RFD11.4.1}\\ \hline
\hyperref[UC12.5]{UC12.5} & \hyperlink{RFD11.5}{RFD11.5}\\
& \hyperlink{RFD11.5.1}{RFD11.5.1}\\ \hline
\hyperref[UC12.5.1]{UC12.5.1} & \hyperlink{RFD11.5.1}{RFD11.5.1}\\ \hline
\hyperref[UC12.5.1.1]{UC12.5.1.1} & \hyperlink{RFD11.5.1}{RFD11.5.1}\\ \hline
\hyperref[UC12.5.1.2]{UC12.5.1.2} & \hyperlink{RFD11.5.1}{RFD11.5.1}\\ \hline
\hyperref[UC12.6]{UC12.6} & \hyperlink{RFD11.6}{RFD11.6}\\
& \hyperlink{RFD11.6.1}{RFD11.6.1}\\ \hline
\hyperref[UC12.6.1]{UC12.6.1} & \hyperlink{RFD11.6.1}{RFD11.6.1}\\ \hline
\hyperref[UC12.6.1.1]{UC12.6.1.1} & \hyperlink{RFD11.6.1}{RFD11.6.1}\\ \hline
\hyperref[UC12.6.1.2]{UC12.6.1.2} & \hyperlink{RFD11.6.1}{RFD11.6.1}\\ \hline
\hyperref[UC12.7]{UC12.7} & \hyperlink{RFD11.7}{RFD11.7}\\
& \hyperlink{RFD11.7.1}{RFD11.7.1}\\ \hline
\hyperref[UC12.7.1]{UC12.7.1} & \hyperlink{RFD11.7.1}{RFD11.7.1}\\ \hline
\hyperref[UC13]{UC13} & \hyperlink{RFD12.1}{RFD12.1}\\
& \hyperlink{RFD12.1.1}{RFD12.1.1}\\ \hline
\hyperref[UC13.2]{UC13.2} & \hyperlink{RFD12.1.1}{RFD12.1.1}\\ \hline
\caption[Tracciamento Fonti-Requisiti]{Tracciamento Fonti-Requisiti}
\label{tabella:fonti-requi}
\end{longtable}
\clearpage

\subsection{Tracciamento Requisiti-Fonti}
\normalsize
\begin{longtable}{|>{\centering}m{5cm}|m{5cm}<{\centering}|}
\hline 
\textbf{Id Requisito} & \textbf{Fonti}\\
\hline
\endhead
\hyperlink{RFO1}{RFO1} & \hyperlink{Verbale interno}{Verbale interno}\\
& \hyperref[UC2]{UC2}\\ \hline

\caption[Tracciamento Requisiti-Fonti]{Tracciamento Requisiti-Fonti}
\label{tabella:requi-fonti}
\end{longtable}
\clearpage
