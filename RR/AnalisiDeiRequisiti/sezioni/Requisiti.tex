\newpage
\section{Requisiti}
In questa sezione verranno presentati i requisiti individuati dal \textit{team\ped{G}} durante l'analisi del capitolato e dei casi d'uso, discussi con il proponente durante le riunioni esterne e decisi dai componenti nelle riunioni interne. Per facilitare la consultazione, i
requisiti saranno separati su più tabelle in base alla loro categoria. I requisiti saranno classificati per tipo e importanza e utilizzeranno la seguente sintassi:
\begin{center}
	R[Importanza][Tipo][Codice]
\end{center}
dove:
\begin{itemize}
	\item \textbf{Importanza}: può assumere uno tra i seguenti valori:
	\begin{itemize}
		\item O: requisito obbligatorio;
		\item D: requisito desiderabile;
		\item F: requisito facoltativo.
	\end{itemize}
	\item \textbf{Tipo}: può assumere uno tra i seguenti valori:
	\begin{itemize}
		\item F: funzionale;
		\item P: prestazionale;
		\item Q: qualità;
		\item V: vincolo.
	\end{itemize}
	\item \textbf{Codice}: è il codice gerarchico univoco di ogni requisito espresso in numeri.
\end{itemize} 
Per ogni requisito inoltre verranno riportate:
\begin{itemize}
	\item \textbf{Descrizione}: breve testo ma completo che andrà a descrivere il requisito in esame;
	\item \textbf{Fonte}: che potrà essere una tra le seguenti:
	\begin{itemize}
		\item Capitolato C5: requisito dedotto direttamente dall'analisi del capitolato C5;
		\item Verbale 2015-12-03: requisito derivato dal suddetto verbale;
		\item Verbale 2015-12-11: requisito derivato dal suddetto verbale;
		\item Verbale 2015-12-29: requisito derivato dal suddetto verbale;
		\item Verbale 2016-01-11: requisito derivato dal suddetto verbale;
		\item Verbale 2016-01-12: requisito derivato dal suddetto verbale;
		\item Interno: requisito identificato dagli Analisti;
		\item Caso d’uso: si tratta di un requisito emerso da un caso d’uso; viene riportato l’identificativo del caso d’uso associato.
	\end{itemize} 
\end{itemize}
%\begin{itemize}
\item \hyperlink{RFD4}{RFD4}: L’utente autenticato e l’utente
autenticato pro possono gestire il proprio
profilo;
\item \hyperlink{RFD4.1}{RFD4.1}: L’utente autenticato e l’utente
autenticato pro possono modificare il
proprio nome;
\item \hyperlink{RFD4.2}{RFD4.2}: L’utente autenticato e l’utente
autenticato pro possono modificare il
proprio cognome;
\item \hyperlink{RFD4.3}{RFD4.3}: L’utente autenticato e l’utente
autenticato pro possono inserire una
foto/immagine;
\item \hyperlink{RFD4.4}{RFD4.4}: L’utente autenticato e l’utente
autenticato pro possono modificare la
propria e-mail;
\item \hyperlink{RFD4.5}{RFD4.5}: L’utente autenticato e l’utente
autenticato pro possono modificare la
propria password;
\item \hyperlink{RFD4.5.1}{RFD4.5.1}: L’utente autenticato e l’utente
autenticato pro possono inserire la vecchia password;
\item \hyperlink{RFD4.5.2}{RFD4.5.2}: L’utente autenticato e l’utente
autenticato pro possono inserire la nuova password;
\item \hyperlink{RFD4.5.3}{RFD4.5.3}: L’utente autenticato e l’utente
autenticato pro possono inserire nuovamente la nuova password;
\item \hyperlink{RFD4.6}{RFD4.6}: L’utente autenticato e l’utente
autenticato pro possono confermare le
modifiche al proprio profilo utente;
\item \hyperlink{RFD4.7}{RFD4.7}: Il sistema deve visualizzare un messaggio
di errore nel caso l’utente autenticato o
l’utente autenticato pro abbiano effettuato modifiche
non permesse al proprio profilo utente;
\item \hyperlink{RFD4.8}{RFD4.8}: L’utente autenticato e l’utente
autenticato pro possono cambiare la
propria tipologia di account;
\item \hyperlink{RFD4.8.1}{RFD4.8.1}: L’utente autenticato e l’utente
autenticato pro possono selezionare la nuova tipologia di account;
\item \hyperlink{RFD4.8.2}{RFD4.8.2}: L’utente autenticato e l’utente autenticato pro possono inviare la richiesta di passaggio alla nuova tipologia di account;
\item \hyperlink{RFD4.9}{RFD4.9}: L’utente autenticato e l’utente autenticato pro possono eliminare il proprio account;
\item \hyperlink{RFD4.9.1}{RFD4.9.1}: L’utente autenticato e l’utente
autenticato pro possono confermare
l’eliminazione del proprio account;
\item \hyperlink{RFD5}{RFD5}: L’utente autenticato e l’utente
autenticato pro possono cercare un
questionario tramite la barra di ricerca;
\item \hyperlink{RFD6.1}{RFD6.1}: L’utente autenticato e l’utente
autenticato pro a partire da una
domanda possono scegliere di spostarsi
alla domanda successiva del questionario;
\item \hyperlink{RFD6.4}{RFD6.4}: L’utente autenticato e l’utente
autenticato pro possono concludere il
questionario confermando le risposte date
alle domande che lo compongono;
\item \hyperlink{RFD7.4}{RFD7.4}: L’utente autenticato e l’utente autenticato pro possono scegliere di modificare una domanda tramite editor di testo \textit{QML\ped{G}};
\item \hyperlink{RFD7.4.1}{RFD7.4.1}: L’utente autenticato e l’utente autenticato pro possono confermare la modifica di una domanda tramite editor di testo \textit{QML\ped{G}};
\item \hyperlink{RFD7.5}{RFD7.5}: L’utente autenticato e l’utente
autenticato pro possono scegliere un
argomento da assegnare alla nuova
domanda;
\item \hyperlink{RFD7.6}{RFD7.6}: L’utente autenticato e l’utente
autenticato pro possono inserire delle
parole chiave relative alla nuova domanda;
\item \hyperlink{RFD8.1}{RFD8.1}: L’utente autenticato pro può visualizzare
i questionari creati;
\item \hyperlink{RFD8.5}{RFD8.5}: L’utente autenticato pro può rendere il
questionario compilabile da parte degli
esaminandi;
\item \hyperlink{RFD8.6.3}{RFD8.6.3}: L’utente autenticato pro può inserire il
nome del questionario ;
\item \hyperlink{RFD8.6.4.1}{RFD8.6.4.1}: L’utente autenticato pro può consultare il
resoconto del questionario dopo aver
deciso di concluderlo;
\item \hyperlink{RFD8.8}{RFD8.8}: L’utente autenticato pro può gestire la
iscrizione degli esaminandi ai questionari;
\item \hyperlink{RFD8.8.1}{RFD8.8.1}: L’utente autenticato pro può selezionare
il questionario del quale gestire le
iscrizioni;
\item \hyperlink{RFD8.8.2}{RFD8.8.2}: L’utente autenticato pro può accettare le
iscrizioni degli esaminandi ai questionari;
\item \hyperlink{RFD9.1}{RFD9.1}: L’utente non autenticato, l’utente
autenticato e l’utente autenticato pro
possono decidere un argomento per fare
un allenamento;
\item \hyperlink{RFD9.2}{RFD9.2}: L’utente non autenticato, l’utente
autenticato e l’utente autenticato pro
possono decidere delle parole chiave per
filtrare maggiormente le domande poste
durante l’allenamento;
\item \hyperlink{RFD9.3}{RFD9.3}: L’utente non autenticato, l’utente
autenticato e l’utente autenticato pro
possono scegliere il numero di domande
che comporranno l’allenamento
(potenzialmente anche infinite domande);
\item \hyperlink{RFD9.4}{RFD9.4}: L’utente non autenticato, l’utente
autenticato e l’utente autenticato pro
possono rispondere alle domande
proposte iniziando l’allenamento;
\item \hyperlink{RFD9.4.1}{RFD9.4.1}: L’utente non autenticato, l’utente
autenticato e l’utente autenticato pro
possono confermare una risposta durante
un allenamento;
\item \hyperlink{RFD9.4.6}{RFD9.4.6}: L’utente non autenticato, l’utente
autenticato e l’utente autenticato pro
possono decidere di terminare
l’allenamento in qualunque momento;
\item \hyperlink{RFD9.5}{RFD9.5}: Il sistema sceglie una domanda in base
all’abilità dell’avversario sull’argomento
scelto;
\item \hyperlink{RFD9.6}{RFD9.6}: Il sistema aggiorna automaticamente i
dati sull’abilità dell’utente ad ogni
risposta;
\item \hyperlink{RFD9.7}{RFD9.7}: Il sistema aggiorna automaticamente i
dati sulla difficoltà di una domanda
quando un utente risponde alla medesima;
\item \hyperlink{RFD10}{RFD10}: L’utente autenticato e l’utente
autenticato pro possono visualizzare il
proprio profilo;
\item \hyperlink{RFD10.1}{RFD10.1}: L’utente autenticato e l’utente
autenticato pro possono andare alla
pagina di gestione del profilo mediante
l’apposito link;
\item \hyperlink{RFD10.2}{RFD10.2}: L’utente autenticato e l’utente
autenticato pro possono andare alla
pagina di gestione delle domande
mediante l’apposito link;
\item \hyperlink{RFD10.3}{RFD10.3}: L’utente autenticato pro può andare alla
pagina di gestione dei questionari
mediante l’apposito link;
\item \hyperlink{RFD10.4}{RFD10.4}: L’utente autenticato e l’utente
autenticato pro possono visualizzare la
cronologia di tutti i questionari che
hanno svolto;
\item \hyperlink{RFD10.4.1}{RFD10.4.1}: L’utente autenticato e l’utente
autenticato pro possono selezionare e
visualizzare le statistiche di un
questionario scelto dalla cronologia;
\item \hyperlink{RFD10.5}{RFD10.5}: L’utente autenticato e l’utente
autenticato pro possono visualizzare la
lista dei questionari abilitati;
\item \hyperlink{RFD10.5.1}{RFD10.5.1}: L’utente autenticato e l’utente
autenticato pro possono selezionare un
questionario abilitato;
\item \hyperlink{RFD10.6}{RFD10.6}: L’utente autenticato e l’utente
autenticato pro possono tornare alla
home page mediante l’apposito link;
\item \hyperlink{RFD10.7}{RFD10.7}: L’utente autenticato e l’utente
autenticato pro possono visualizzare il proprio username;
\item \hyperlink{RFD10.8}{RFD10.8}: L’utente autenticato e l’utente
autenticato pro possono visualizzare la propria immagine profilo;
\item \hyperlink{RFD10.9}{RFD10.9}: L’utente autenticato e l’utente
autenticato pro possono visualizzare il proprio livello attuale;
\item \hyperlink{RFD10.10}{RFD10.10}: L’utente autenticato e l’utente
autenticato pro possono visualizzare il numero di domande risposte in
modo esatto;
\item \hyperlink{RFD10.11}{RFD10.11}: L’utente autenticato e l’utente autenticato pro possono visualizzare il numero di domande risposte in totale;
\item \hyperlink{RFD11.4}{RFD11.4}: L’utente non autenticato, l’utente
autenticato e l’utente autenticato pro
possono rispondere ad una domanda di
collegamento;
\item \hyperlink{RFD11.4.1}{RFD11.4.1}: L’utente non autenticato, l’utente
autenticato e l’utente autenticato pro
possono collegare le voci;
\item \hyperlink{RFD11.5}{RFD11.5}: L’utente non autenticato, l’utente
autenticato e l’utente autenticato pro
possono ordinare delle immagini;
\item \hyperlink{RFD11.5.1}{RFD11.5.1}: L’utente non autenticato, l’utente
autenticato e l’utente autenticato pro
possono inserire un’immagine in uno
spazio già occupato oppure no;
\item \hyperlink{RFD11.6}{RFD11.6}: L’utente non autenticato, l’utente
autenticato e l’utente autenticato pro
possono ordinare delle stringhe;
\item \hyperlink{RFD11.6.1}{RFD11.6.1}: L’utente non autenticato, l’utente
autenticato e l’utente autenticato pro
possono inserire una stringa in uno spazio
già occupato oppure no;
\item \hyperlink{RFD12.1}{RFD12.1}: L’utente autenticato e l’utente
autenticato pro possono inserire nome e
cognome oppure lo username nella barra
di ricerca per ricercare un utente;
\item \hyperlink{RFD12.2}{RFD12.2}: L’utente autenticato e l’utente
autenticato pro possono selezionare
l’utente ricercato;
\item \hyperlink{RFD12.3}{RFD12.3}: L’utente autenticato e l’utente
autenticato pro possono visualizzare il profilo dell'utente ricercato;
\item \hyperlink{RFD12.3.1}{RFD12.3.1}: L’utente autenticato e l’utente autenticato pro possono visualizzare l'username dell'utente ricercato;
\item \hyperlink{RFD12.3.2}{RFD12.3.2}: L’utente autenticato e l’utente autenticato pro possono visualizzare l'immagine profilo dell'utente ricercato;
\item \hyperlink{RFD12.3.3}{RFD12.3.3}: L’utente autenticato e l’utente autenticato pro possono visualizzare il livello attuale dell'utente ricercato;
\item \hyperlink{RFD12.3.4}{RFD12.3.4}: L’utente autenticato e l’utente autenticato pro possono visualizzare il numero di domande risposte in modo corretto dall'utente ricercato;
\item \hyperlink{RFD12.3.5}{RFD12.3.5}: L’utente autenticato e l’utente autenticato pro possono visualizzare il numero di domande risposte in totale dall'utente ricercato;
\item \hyperlink{RFD12.3.6}{RFD12.3.6}: Il sistema deve mostrare un messaggio di errore in caso la ricerca degli utenti non sia andata a buon fine;
\item \hyperlink{RFD17}{RFD17}: Il sistema deve mostrare un messaggio di errore in caso la ricerca dei questionari non sia andata a buon fine;
\item \hyperlink{RFD18}{RFD18}: L’utente autenticato e l’utente
autenticato pro possono iscriversi ad un
questionario;
\item \hyperlink{RFD18.1}{RFD18.1}: L’utente autenticato e l’utente
autenticato pro possono confermare
l’iscrizione ad un questionario;
\item \hyperlink{RFD23.5}{RFD23.5}: Il sistema attraverso il linguaggio \textit{QML\ped{G}} deve gestire esercizi di ordinamento di
scelte;
\item \hyperlink{RFD23.6}{RFD23.6}: Il sistema attraverso il linguaggio \textit{QML\ped{G}} deve gestire esercizi a corrispondenza di
scelte;
\item \hyperlink{RFD28}{RFD28}: Il sistema deve archiviare i risultati dei
questionari;
\item \hyperlink{RFD29}{RFD29}: Il sistema deve archiviare le statistiche
delle risposte date ad ogni domanda;
\item \hyperlink{RFD31}{RFD31}: Il sistema deve creare questionari
dinamicamente (modalità allenamento)
per un argomento scegliendo le domande
in modo casuale;
\item \hyperlink{RFD32}{RFD32}: Il sistema deve creare questionari
dinamicamente (modalità allenamento)
scegliendo le domande in base alle
risposte date dai questionari precedenti;
\item \hyperlink{RFD33}{RFD33}: Il sistema deve creare questionari
dinamicamente (modalità allenamento)
scegliendo tra le domande più difficili;
\item \hyperlink{RFD34}{RFD34}: Il sistema deve creare questionari
dinamicamente (modalità allenamento)
scegliendo tra le lacune dei partecipanti;
\item \hyperlink{RFD35}{RFD35}: Il sistema deve permettere agli
utilizzatori di proporre nuove domande;
\item \hyperlink{RFF38}{RFF38}: Il sistema deve permettere agli
utilizzatori di rispondere più volte ad una
domanda;
\item \hyperlink{RFF40}{RFF40}: Il sistema deve permettere di visualizzare l'intera applicazione in lingue divese;
\item \hyperlink{RFD41}{RFD41}: Il sistema deve permettere all'utente di far conoscere chi ha sviluppato il sistema e di far capire che cos'è QuizziPedia;
\item \hyperlink{RVD4}{RVD4}: L’applicazione deve utilizzare \textit{fogli di stile\ped{G}} in \textit{CSS3\ped{G}};
\item \hyperlink{RVD13}{RVD13}: L’applicazione deve funzionare su \textit{Google Chrome per iOS\ped{G}} versione 39 o superiore per le funzionalità che riguardano la compilazione dei questionari e delle domande;
\item \hyperlink{RVF15}{RVF15}: L’applicazione deve funzionare su \textit{Mozilla Firefox per Android\ped{G}} versione 33 o superiore per le funzionalità che riguardano la compilazione dei questionari e delle domande;
\item \hyperlink{RVD18}{RVD18}: L’applicazione deve funzionare su \textit{Google Chrome per Android\ped{G}} versione 39 o superiore per le funzionalità che riguardano la creazione dei questionari e delle domande;
\item \hyperlink{RVF19}{RVF19}: L’applicazione deve funzionare su \textit{Mozilla Firefox per Android\ped{G}} versione 33 o superiore per le funzionalità che riguardano la creazione dei questionari e delle domande;
\item \hyperlink{RVF21}{RVF21}: L’applicazione deve funzionare su \textit{Browser Opera per Android\ped{G}} versione 34 o superiore per le funzionalità che riguardano la creazione dei questionari e delle domande;
\end{itemize}
\subsection{Riepilogo Requisiti}
\normalsize
\begin{longtable}{|c|c|c|c|}
\hline 
\textbf{Tipo} & \textbf{Obbligatorio} & \textbf{Desiderabile} & \textbf{Facoltativo}\\
\hline
Funzionale & 1 & 0 & 0\\ \hline
Prestazionale & 0 & 0 & 0\\ \hline
Di Qualità & 0 & 0 & 0\\ \hline
Di Vincolo & 0 & 0 & 0\\ \hline
\caption[Riepilogo Requisiti]{Riepilogo Requisiti}
\label{tabella:riepilogorequi}
\end{longtable}
\clearpage

\subsection{Requisiti Funzionali}
\normalsize
\begin{longtable}{|c|>{\centering}m{7cm}|c|}
\hline
\textbf{Id Requisito} & \textbf{Descrizione} & \textbf{Fonti}\\
\hline
\endhead
\hypertarget{RFO1}{RFO1} & L’utente non autenticato può registrarsi & Interno
\\ \hline

\hypertarget{RFO1.1}{RFO1.1} & L’utente non autenticato può inserire il proprio nome & Interno
\\ \hline

\hypertarget{RFO1.2}{RFO1.2} & L’utente non autenticato può inserire il proprio cognome & Interno
\\ \hline

\hypertarget{RFO1.3}{RFO1.3} & L’utente non autenticato può inserire il proprio nome utente & Interno
\\ \hline

\hypertarget{RFO1.4}{RFO1.4} & L’utente non autenticato può inserire la propria email & Interno
\\ \hline

\hypertarget{RFO1.5}{RFO1.5} & L’utente non autenticato può inserire una password & Interno
\\ \hline

\hypertarget{RFO1.6}{RFO1.6} & L’utente non autenticato può confermare la propria password & Interno
\\ \hline

\hypertarget{RFO1.7}{RFO1.7} & L’utente non autenticato può confermare la registrazione & Interno
\\ \hline

\hypertarget{RFO1.8}{RFO1.8} & Il sistema deve mostrare un messaggio d’errore di registrazione all’utente non registrato se i dati non sono stati immessi correttamente o sono assenti & Interno
\\ \hline

\hypertarget{RFO2}{RFO2} & L’utente non autenticato può effettuare il login & Verbale interno
\\ \hline

\hypertarget{RFO2.1}{RFO2.1} & L'utente non autenticato può fare il login tramite QuizziPedia & Verbale interno
\\ \hline

\hypertarget{RFO2.1.1}{RFO2.1.1} & L'utente non autenticato può inserire il nome utente o la mail usata durante la fase di registrazione & Interno
\\ \hline

\hypertarget{RFO2.1.2}{RFO2.1.2} & L'utente non autenticato può inserire la password associata al nome utente o alla mail usata durante la fase di registrazione & Interno
\\ \hline

\hypertarget{RFO2.1.3}{RFO2.1.3} & L'utente non autenticato può confermare il login & Interno
\\ \hline

\hypertarget{RFO2.1.4}{RFO2.1.4} & Il sistema deve visualizzare un messaggio di errore in caso di inserimento dei dati errato o assente & Interno
\\ \hline

\hypertarget{RFF2.2}{RFF2.2} & L'utente non autenticato può eseguire il login tramite Facebook & Interno
\\ \hline

\hypertarget{RFF2.3}{RFF2.3} & L'utente non autenticato può eseguire il login tramite Twitter & Interno
\\ \hline

\hypertarget{RFF2.4}{RFF2.4} & L'utente non autenticato può eseguire il login tramite Google+ & Interno
\\ \hline

\hypertarget{RFF2.5}{RFF2.5} & L'utente non autenticato può eseguire il login tramite Linkedin & Interno
\\ \hline

\hypertarget{RFO3}{RFO3} & L’utente autenticato e l’utente autenticato pro possono effettuare il logout & Interno
\\ \hline

\hypertarget{RFO3.1}{RFO3.1} & Il sistema deve notificare all’utente autenticato o all’utente autenticato pro la disconnessione dall’area riservata & Interno
\\ \hline

\hypertarget{RFD4}{RFD4} & L’utente autenticato e l’utente autenticato pro possono gestire il proprio profilo & Interno
\\ \hline

\hypertarget{RFD4.1}{RFD4.1} & L’utente autenticato e l’utente autenticato pro possono modificare il proprio nome  & Interno
\\ \hline

\hypertarget{RFD4.1.1}{RFD4.1.1} & L’utente autenticato e l’utente autenticato pro possono confermare le modifiche al proprio nome & Interno
\\ \hline

\hypertarget{RFD4.2}{RFD4.2} & L’utente autenticato e l’utente autenticato pro possono modificare il proprio cognome  & Interno
\\ \hline

\hypertarget{RFD4.2.1}{RFD4.2.1} & L’utente autenticato e l’utente autenticato pro possono confermare le modifiche al proprio cognome & Interno
\\ \hline

\hypertarget{RFD4.3}{RFD4.3} & L’utente autenticato e l’utente autenticato pro possono modificare il proprio username & Interno
\\ \hline

\hypertarget{RFD4.3.1}{RFD4.3.1} &  L’utente autenticato e l’utente autenticato pro possono confermare le modifiche al proprio username & Interno
\\ \hline

\hypertarget{RFD4.4}{RFD4.4} & L’utente autenticato e l’utente autenticato pro possono inserire una foto/immagine  & Interno
\\ \hline

\hypertarget{RFD4.4.1}{RFD4.4.1} & L’utente autenticato e l’utente autenticato pro possono confermare l’inserimento della foto/immagine & Interno
\\ \hline

\hypertarget{RFD4.5}{RFD4.5} & L’utente autenticato e l’utente autenticato pro possono modificare la propria e-mail  & Interno
\\ \hline

\hypertarget{RFD4.5.1}{RFD4.5.1} & L’utente autenticato e l’utente autenticato pro possono confermare le modifiche alla propria e-mail  & Interno
\\ \hline

\hypertarget{RFD4.6}{RFD4.6} & L’utente autenticato e l’utente autenticato pro possono modificare la propria password & Interno
\\ \hline

\hypertarget{RFD4.6.1}{RFD4.6.1} & L’utente autenticato e l’utente autenticato pro possono confermare le modifiche alla propria password  & Interno
\\ \hline

\hypertarget{RFD4.7}{RFD4.7} & L’utente autenticato e l’utente autenticato pro possono cambiare la propria tipologia di account & Interno
\\ \hline

\hypertarget{RFD4.7.1}{RFD4.7.1} & L’utente autenticato e l’utente autenticato pro possono confermare la nuova tipologia di account  & Interno
\\ \hline

\hypertarget{RFD4.8}{RFD4.8} & L’utente autenticato e l’utente autenticato pro possono eliminare il proprio account  & Interno
\\ \hline

\hypertarget{RFD4.8.1}{RFD4.8.1} & L’utente autenticato e l’utente autenticato pro possono confermare l’eliminazione del proprio account & Interno
\\ \hline

\hypertarget{RFD5}{RFD5} & L’utente autenticato e l’utente autenticato pro possono effettuare un ricerca per l’iscrizione ad un questionario & Interno
\\ \hline

\hypertarget{RFD5.1}{RFD5.1} & L’utente autenticato e l’utente autenticato pro possono cercare un questionario tramite la barra di ricerca & Interno
\\ \hline

\hypertarget{RFD5.2}{RFD5.2} & L’utente autenticato e l’utente autenticato pro possono iscriversi ad un questionario & Interno
\\ \hline

\hypertarget{RFD5.2.1}{RFD5.2.1} & L’utente autenticato e l’utente autenticato pro possono confermare l’iscrizione ad un questionario & Interno
\\ \hline

\hypertarget{RFD5.3}{RFD5.3} & Il sistema deve visualizzare un messaggio di errore nel caso l’’utente autenticato o l’utente autenticato pro ricerchino un questionario inesistente  & Interno
\\ \hline

\hypertarget{RFO6}{RFO6} & L’utente autenticato e l’utente autenticato pro possono compilare un questionario & Capitolato
\\ \hline

\hypertarget{RFD6.1}{RFD6.1} & L’utente autenticato e l’utente autenticato pro a partire da una domanda possono scegliere di spostarsi alla domanda successiva del questionario  & Interno
\\ \hline

\hypertarget{RFD6.2}{RFD6.2} & L’utente autenticato e l’utente autenticato pro a partire da una domanda possono scegliere di spostarsi alla domanda precedente del questionario & Interno
\\ \hline

\hypertarget{RFD6.3}{RFD6.3} & L’utente autenticato e l’utente autenticato pro a partire da una domanda possono scegliere di spostarsi ad una qualsiasi altra domanda presente nel questionario & Interno
\\ \hline

\hypertarget{RFD6.4}{RFD6.4} & L’utente autenticato e l’utente autenticato pro possono concludere il questionario confermando le risposte date alle domande che lo compongono  & Interno
\\ \hline

\hypertarget{RFO6.5}{RFO6.5} & Il sistema deve valutare le risposte date dagli utilizzatori & Capitolato
\\ \hline

\hypertarget{RFO7}{RFO7} & L’utente autenticato e l’utente autenticato pro possono gestire le domande che hanno creato & Capitolato
\\ \hline

\hypertarget{RFO7.1}{RFO7.1} & L’utente autenticato e l’utente autenticato pro possono creare una domanda  & Capitolato
\\ \hline

\hypertarget{RFD7.1.1}{RFD7.1.1} & L’utente autenticato e l’utente autenticato pro possono scegliere un argomento da assegnare alla nuova domanda  & Interno
\\ \hline

\hypertarget{RFD7.1.2}{RFD7.1.2} & L’utente autenticato e l’utente autenticato pro possono inserire delle parole chiave relative alla nuova domanda & Interno
\\ \hline

\hypertarget{RFO7.1.3}{RFO7.1.3} & L’utente autenticato e l’utente autenticato pro possono scegliere la tipologia  di domanda da creare  & Interno
\\ \hline

\hypertarget{RFO7.1.3.1}{RFO7.1.3.1} & L’’utente autenticato e l’utente autenticato pro possono scegliere di creare una domanda vero/falso & Capitolato
\\ \hline

\hypertarget{RFO7.1.3.1.1}{RFO7.1.3.1.1} & L’utente autenticato e autenticato pro possono creare il testo della domanda v/f  & Capitolato
\\ \hline

\hypertarget{RFD7.1.3.1.2}{RFD7.1.3.1.2} & L’utente autenticato e autenticato pro possono inserire un immagine relativa al testo della domanda v/f & Capitolato
\\ \hline

\hypertarget{RFD7.1.3.1.2.1}{RFD7.1.3.1.2.1} & L’utente autenticato e autenticato pro possono eliminare l’immagine relativa al testo della domanda v/f  & Interno
\\ \hline

\hypertarget{RFO7.1.3.1.3}{RFO7.1.3.1.3} & L’utente autenticato e autenticato pro possono selezionare la risposta corretta  & Capitolato
\\ \hline

\hypertarget{RFO7.1.3.2}{RFO7.1.3.2} & L’’utente autenticato e l’utente autenticato pro possono scegliere di creare una domanda a risposta multipla & Capitolato
\\ \hline

\hypertarget{RFO7.1.3.2.1}{RFO7.1.3.2.1} & L’utente autenticato e autenticato pro possono creare il testo della domanda a risposta multipla  & Capitolato
\\ \hline

\hypertarget{RFD7.1.3.2.2}{RFD7.1.3.2.2} & L’utente autenticato e autenticato pro possono inserire un immagine relativa al testo della domanda a risposta multipla  & Capitolato
\\ \hline

\hypertarget{RFD7.1.3.2.2.1}{RFD7.1.3.2.2.1} & L’utente autenticato e autenticato pro possono eliminare l’immagine relativa al testo della domanda a risposta multipla & Interno
\\ \hline

\hypertarget{RFO7.1.3.2.3}{RFO7.1.3.2.3} & L’utente autenticato e autenticato pro possono aggiungere due o più opzioni di risposta & Capitolato
\\ \hline

\hypertarget{RFO7.1.3.2.3.1}{RFO7.1.3.2.3.1} & L’utente autenticato e autenticato pro possono aggiungere due o più opzioni di risposta che includono testo & Capitolato
\\ \hline

\hypertarget{RFD7.1.3.2.3.1.1}{RFD7.1.3.2.3.1.1} & L’utente autenticato e autenticato pro possono eliminare   le opzioni di risposta che includono testo & Interno
\\ \hline

\hypertarget{RFD7.1.3.2.3.2}{RFD7.1.3.2.3.2} & L’utente autenticato e autenticato pro possono aggiungere due o più opzioni di risposta che includono immagini & Capitolato
\\ \hline

\hypertarget{RFD7.1.3.2.3.2.1}{RFD7.1.3.2.3.2.1} & L’utente autenticato e autenticato pro possono eliminare   le opzioni di risposta che includono immagini  & Interno
\\ \hline

\hypertarget{RFO7.1.3.2.4}{RFO7.1.3.2.4} & L’utente autenticato e autenticato pro possono selezionare una o più risposte corrette & Capitolato
\\ \hline

\hypertarget{RFD7.1.3.3}{RFD7.1.3.3} & L’utente autenticato e l’utente autenticato pro possono scegliere di creare un esercizio di riempimento degli spazi vuoti  & Capitolato
\\ \hline

\hypertarget{RFD7.1.3.3.1}{RFD7.1.3.3.1} & L’utente autenticato e autenticato pro possono scrivere testo dell’esercizio & Capitolato
\\ \hline

\hypertarget{RFD7.1.3.3.2}{RFD7.1.3.3.2} & L’utente autenticato e autenticato pro possono indicare parole da oscurare & Capitolato
\\ \hline

\hypertarget{RFD7.1.3.4}{RFD7.1.3.4} & L’utente autenticato e l’utente autenticato pro possono scegliere di creare una domanda di collegamento & Capitolato
\\ \hline

\hypertarget{RFD7.1.3.4.1}{RFD7.1.3.4.1} & L’utente autenticato e autenticato pro possono inserire il testo della domanda & Capitolato
\\ \hline

\hypertarget{RFD7.1.3.4.2}{RFD7.1.3.4.2} & L’utente autenticato e autenticato pro possono inserire una o più coppie di elementi & Capitolato
\\ \hline

\hypertarget{RFD7.1.3.4.2.1}{RFD7.1.3.4.2.1} & L’utente autenticato e autenticato pro possono inserire un’ immagine come primo elemento & Capitolato
\\ \hline

\hypertarget{RFD7.1.3.4.2.2}{RFD7.1.3.4.2.2} & L’utente autenticato e autenticato pro possono inserire un testo come primo elemento & Capitolato
\\ \hline

\hypertarget{RFD7.1.3.4.2.3}{RFD7.1.3.4.2.3} & L’utente autenticato e autenticato pro possono inserire un’ immagine come secondo elemento & Capitolato
\\ \hline

\hypertarget{RFD7.1.3.4.2.4}{RFD7.1.3.4.2.4} & L’utente autenticato e autenticato pro possono inserire un testo come secondo elemento & Capitolato
\\ \hline

\hypertarget{RFD7.1.3.4.3}{RFD7.1.3.4.3} & L’utente autenticato e autenticato pro possono eliminare una o più coppie di elementi  & Interno
\\ \hline

\hypertarget{RFD7.1.3.4.3.1}{RFD7.1.3.4.3.1} & L’utente autenticato e autenticato pro possono confermare eliminazione coppia di elementi & Interno
\\ \hline

\hypertarget{RFD7.1.3.4.4}{RFD7.1.3.4.4} & L’utente autenticato e autenticato pro possono modificare una o più coppie di elementi & Interno
\\ \hline

\hypertarget{RFD7.1.3.4.4.1}{RFD7.1.3.4.4.1} & L’utente autenticato e autenticato pro possono modificare il testo di un elemento & Interno
\\ \hline

\hypertarget{RFD7.1.3.4.4.2}{RFD7.1.3.4.4.2} & L’utente autenticato e autenticato pro possono modificare l’immagine di un elemento & Interno
\\ \hline

\hypertarget{RFD7.1.3.4.4.3}{RFD7.1.3.4.4.3} & L’utente autenticato e autenticato pro possono cambiare il testo in un’immagine & Interno
\\ \hline

\hypertarget{RFD7.1.3.4.4.4}{RFD7.1.3.4.4.4} & L’utente autenticato e autenticato pro possono cambiare l’immagine in un testo & Interno
\\ \hline

\hypertarget{RFD7.1.3.5}{RFD7.1.3.5} & L’utente autenticato e l’utente autenticato pro possono scegliere di creare una domanda a ordinamento di immagini  & Capitolato
\\ \hline

\hypertarget{RFD7.1.3.5.1}{RFD7.1.3.5.1} & L’utente autenticato e l’utente autenticato pro possono inserire il testo della domanda & Capitolato
\\ \hline

\hypertarget{RFD7.1.3.5.2}{RFD7.1.3.5.2} & L’utente autenticato e l’utente autenticato pro possono inserire immagine per il testo della domanda & Capitolato
\\ \hline

\hypertarget{RFD7.1.3.5.2.1}{RFD7.1.3.5.2.1} & L’utente autenticato e l’utente autenticato pro possono eliminare un immagine relativa al testo della domanda  & Interno
\\ \hline

\hypertarget{RFD7.1.3.5.3}{RFD7.1.3.5.3} & L’utente autenticato e l’utente autenticato pro possono inserire immagini come risposta & Capitolato
\\ \hline

\hypertarget{RFD7.1.3.5.3.1}{RFD7.1.3.5.3.1} & L’utente autenticato e l’utente autenticato pro possono eliminare l’immagine di una risposta & Interno
\\ \hline

\hypertarget{RFD7.1.3.6}{RFD7.1.3.6} & L’utente autenticato e l’utente autenticato pro possono scegliere di creare una domanda a ordinamento di stringhe & Capitolato
\\ \hline

\hypertarget{RFD7.1.3.6.1}{RFD7.1.3.6.1} & L’utente autenticato e l’utente autenticato pro possono inserire il testo della domanda  & Capitolato
\\ \hline

\hypertarget{RFD7.1.3.6.2}{RFD7.1.3.6.2} & L’utente autenticato e l’utente autenticato pro possono inserire stringhe di composizione sequenza & Capitolato
\\ \hline

\hypertarget{RFD7.1.3.6.3}{RFD7.1.3.6.3} & L’utente autenticato e l’utente autenticato pro possono comporre la soluzione della sequenza & Capitolato
\\ \hline

\hypertarget{RFD7.1.3.7}{RFD7.1.3.7} & L’utente autenticato e l’utente autenticato pro possono creare una domanda con area cliccabile nell’immagine & Capitolato
\\ \hline

\hypertarget{RFD7.1.3.7.1}{RFD7.1.3.7.1} & L’utente autenticato e l’utente autenticato pro possono inserire il testo della domanda & Capitolato
\\ \hline

\hypertarget{RFD7.1.3.7.2}{RFD7.1.3.7.2} & L’utente autenticato e l’utente autenticato pro possono inserire un’immagine & Capitolato
\\ \hline

\hypertarget{RFD7.1.3.7.3}{RFD7.1.3.7.3} & L’utente autenticato e l’utente autenticato pro possono scegliere un numero di aree selezionabili & Capitolato
\\ \hline

\hypertarget{RFD7.1.3.7.4}{RFD7.1.3.7.4} & L’utente autenticato e l’utente autenticato pro possono scegliere le aree  selezionabili & Capitolato
\\ \hline

\hypertarget{RFO7.1.4}{RFO7.1.4} & L’utente autenticato e autenticato pro possono confermare la creazione della domanda & Capitolato
\\ \hline

\hypertarget{RFO7.1.5}{RFO7.1.5} & Il sistema deve mostrare un messaggio di errore in caso la conferma della creazione della domanda non sia andata a buon fine & Capitolato
\\ \hline

\hypertarget{RFD7.2}{RFD7.2} & L’utente autenticato e l’utente autenticato pro possono modificare una domanda & Interno
\\ \hline

\hypertarget{RFD7.2.1}{RFD7.2.1} & L’utente può selezionare una domanda da modificare & Interno
\\ \hline

\hypertarget{RFD7.2.1.1}{RFD7.2.1.1} & l’utente autenticato e l’utente autenticato pro possono scegliere di modificare una domanda vero/falso & Interno
\\ \hline

\hypertarget{RFD7.2.1.1.1}{RFD7.2.1.1.1} & L’utente autenticato e autenticato pro possono modificare il testo della domanda vero/falso & Interno
\\ \hline

\hypertarget{RFD7.2.1.1.2}{RFD7.2.1.1.2} & L’utente autenticato e autenticato pro possono modificare l’immagine della domanda vero/falso & Interno
\\ \hline

\hypertarget{RFD7.2.1.1.3}{RFD7.2.1.1.3} & L’utente autenticato e autenticato pro possono modificare la risposta corretta della domanda vero/falso & Interno
\\ \hline

\hypertarget{RFD7.2.1.2}{RFD7.2.1.2} & L’utente autenticato e l’utente autenticato pro possono scegliere di modificare una domanda a risposta multipla & Interno
\\ \hline

\hypertarget{RFD7.2.1.2.1}{RFD7.2.1.2.1} & L’utente autenticato e autenticato pro possono modificare il testo della domanda a risposta multipla & Interno
\\ \hline

\hypertarget{RFD7.2.1.2.2}{RFD7.2.1.2.2} & L’utente autenticato e autenticato pro possono modificare l’immagine della domanda a risposta multipla & Interno
\\ \hline

\hypertarget{RFD7.2.1.2.3}{RFD7.2.1.2.3} & L’utente autenticato e autenticato pro possono modificare l’opzione di risposta della domanda a risposta multipla & Interno
\\ \hline

\hypertarget{RFD7.2.1.2.3.1}{RFD7.2.1.2.3.1} & L’utente autenticato e autenticato pro possono modificare l’opzione di risposta che include del testo & Interno
\\ \hline

\hypertarget{RFD7.2.1.2.3.2}{RFD7.2.1.2.3.2} & L’utente autenticato e autenticato pro possono modificare l’opzione di risposta che include delle immagini & Interno
\\ \hline

\hypertarget{RFD7.2.1.2.4}{RFD7.2.1.2.4} & L’utente autenticato e l’utente autenticato pro possono modificare la risposta corretta o le risposte corrette della domanda a risposta multipla & Interno
\\ \hline

\hypertarget{RFD7.2.1.3}{RFD7.2.1.3} & L’utente autenticato e l’utente autenticato pro possono scegliere di modificare un domanda a riempimento di spazi vuoti & Interno
\\ \hline

\hypertarget{RFD7.2.1.3.1}{RFD7.2.1.3.1} & L’utente autenticato e autenticato pro possono modificare il testo della domanda a riempimento di spazi & Interno
\\ \hline

\hypertarget{RFD7.2.1.3.2}{RFD7.2.1.3.2} & L’utente autenticato e autenticato pro possono modificare le parole da oscurare della domanda a riempimento di spazi  & Interno
\\ \hline

\hypertarget{RFD7.2.1.4}{RFD7.2.1.4} & L’utente autenticato e l’utente autenticato pro possono scegliere di modificare una domanda di collegamento & Interno
\\ \hline

\hypertarget{RFD7.2.1.4.1}{RFD7.2.1.4.1} & L’utente autenticato e autenticato pro possono inserire una nuova coppia di elementi da collegare & Interno
\\ \hline

\hypertarget{RFD7.2.1.4.1.1}{RFD7.2.1.4.1.1} & L’utente autenticato e autenticato pro possono inserire un’immagine come primo elemento della domanda & Interno
\\ \hline

\hypertarget{RFD7.2.1.4.1.2}{RFD7.2.1.4.1.2} & L’utente autenticato e autenticato pro possono inserire un testo come primo elemento della domanda & Interno
\\ \hline

\hypertarget{RFD7.2.1.4.1.3}{RFD7.2.1.4.1.3} & L’utente autenticato e autenticato pro possono inserire un’immagine come secondo elemento della domanda  & Interno
\\ \hline

\hypertarget{RFD7.2.1.4.1.4}{RFD7.2.1.4.1.4} & L’utente autenticato e autenticato pro possono inserire un testo come secondo elemento della domanda  & Interno
\\ \hline

\hypertarget{RFD7.2.1.4.2}{RFD7.2.1.4.2} & L’utente autenticato e autenticato pro possono eliminare una coppia di elementi da collegare & Interno
\\ \hline

\hypertarget{RFD7.2.1.4.2.1}{RFD7.2.1.4.2.1} & L’utente autenticato e autenticato pro possono eliminare una coppia di elementi da collegare & Interno
\\ \hline

\hypertarget{RFD7.2.1.4.3}{RFD7.2.1.4.3} & L’utente autenticato e l'utente autenticato pro possono eliminare una coppia di elementi da collegare & Interno
\\ \hline

\hypertarget{RFD7.2.1.4.3.1}{RFD7.2.1.4.3.1} & L’utente autenticato e autenticato pro possono modificare il testo di un elemento da collegare & Interno
\\ \hline

\hypertarget{RFD7.2.1.4.3.2}{RFD7.2.1.4.3.2} & L’utente autenticato e autenticato pro possono modificare un’immagine di un elemento da collegare & Interno
\\ \hline

\hypertarget{RFD7.2.1.4.3.3}{RFD7.2.1.4.3.3} & L’utente autenticato e autenticato pro possono cambiare il testo di un elemento da collegare in un immagine & Interno
\\ \hline

\hypertarget{RFD7.2.1.4.3.4}{RFD7.2.1.4.3.4} & L’utente autenticato e autenticato pro possono cambiare l’immagine di un elemento da collegare in un testo & Interno
\\ \hline

\hypertarget{RFD7.2.1.4.4}{RFD7.2.1.4.4} & L’utente autenticato e l'utente autenticato pro possono eliminare una coppia di elementi & Interno
\\ \hline

\hypertarget{RFD7.2.1.4.5}{RFD7.2.1.4.5} & L’utente autenticato e autenticato pro possono eliminare una coppia di elementi da collegare & Interno
\\ \hline

\hypertarget{RFD7.2.1.4.6}{RFD7.2.1.4.6} & L’utente autenticato e autenticato pro possono modificare il testo della domanda a coppia di elementi da collegare  & Interno
\\ \hline

\hypertarget{RFD7.2.1.5}{RFD7.2.1.5} & L’utente autenticato e l’utente autenticato pro possono scegliere di modificare una domanda di ordinamento immagini & Interno
\\ \hline

\hypertarget{RFD7.2.1.5.1}{RFD7.2.1.5.1} & L’utente autenticato e autenticato pro possono modificare il testo della domanda a ordinamento di immagini & Interno
\\ \hline

\hypertarget{RFD7.2.1.5.2}{RFD7.2.1.5.2} & L’utente autenticato e autenticato pro possono modificare l’immagine della domanda a ordinamento di immagini  & Interno
\\ \hline

\hypertarget{RFD7.2.1.5.3}{RFD7.2.1.5.3} & L’utente autenticato e autenticato pro possono modificare le immagini della risposta della domanda a ordinamento di immagini & Interno
\\ \hline

\hypertarget{RFD7.2.1.5.4}{RFD7.2.1.5.4} & L’utente autenticato e autenticato pro possono modificare l’ordine delle immagini della risposta della domanda a ordinamento di immagini & Interno
\\ \hline

\hypertarget{RFD7.2.1.6}{RFD7.2.1.6} & L’utente autenticato e l’utente autenticato pro possono scegliere di modificare una domanda di ordinamento stringhe & Interno
\\ \hline

\hypertarget{RFD7.2.1.6.1}{RFD7.2.1.6.1} & L’utente autenticato e autenticato pro possono modificare il testo della domanda a ordinamento di stringhe & Interno
\\ \hline

\hypertarget{RFD7.2.1.6.2}{RFD7.2.1.6.2} & L’utente autenticato e autenticato pro possono modificare il testo della risposta e il numero di stringhe della domanda a ordinamento di immagini & Interno
\\ \hline

\hypertarget{RFD7.2.1.6.3}{RFD7.2.1.6.3} & L’utente autenticato e autenticato pro possono modificare la soluzione della domanda a ordinamento di stringhe & Interno
\\ \hline

\hypertarget{RFD7.2.1.7}{RFD7.2.1.7} & L’utente autenticato e l’utente autenticato pro possono scegliere di modificare una domanda con immagine ad aree cliccabili & Interno
\\ \hline

\hypertarget{RFD7.2.1.7.1}{RFD7.2.1.7.1} & L’utente autenticato e autenticato pro possono modificare il testo della domanda a immagine con aree selezionabili & Interno
\\ \hline

\hypertarget{RFD7.2.1.7.2}{RFD7.2.1.7.2} & L’utente autenticato e autenticato pro possono inserire una nuova immagine della domanda a immagine con aree selezionabili & Interno
\\ \hline

\hypertarget{RFD7.2.1.7.3}{RFD7.2.1.7.3} & L’utente autenticato e autenticato pro possono modificare il numero della aree selezionabili dell’immagine della domanda a immagine con aree selezionabili & Interno
\\ \hline

\hypertarget{RFD7.2.1.7.4}{RFD7.2.1.7.4} & L’utente autenticato e autenticato pro possono scegliere nuove aree selezionabili dell’immagine della domanda a immagine con aree selezionabili & Interno
\\ \hline

\hypertarget{RFD7.2.1.8}{RFD7.2.1.8} & L’utente autenticato e l’utente autenticato pro posson modificare una domanda con immagine ad aree cliccabili & Interno
\\ \hline

\hypertarget{RFD7.2.2}{RFD7.2.2} & L’utente autenticato e autenticato pro possono confermare le modifiche apportate alla domanda & Interno
\\ \hline

\hypertarget{RFD7.2.3}{RFD7.2.3} & Il sistema deve mostrare un messaggio di errore in caso la conferma delle modifiche non sia andata a buon fine & Interno
\\ \hline

\hypertarget{RFO8}{RFO8} & L’utente autenticato pro può gestire i questionari che ha creato & Verbale 2016-01-11
\\ \hline

\hypertarget{RFD8.1}{RFD8.1} & L’utente autenticato pro può visualizzare i questionari creati & Interno
\\ \hline

\hypertarget{RFD8.1.1}{RFD8.1.1} & L’utente autenticato pro può modificare un questionario che ha creato & Interno
\\ \hline

\hypertarget{RFD8.1.1.1}{RFD8.1.1.1} & L’utente autenticato pro può modificare il nome di un questionario & Interno
\\ \hline

\hypertarget{RFD8.1.1.2}{RFD8.1.1.2} & L’utente autenticato pro può confermare le modifiche che ha effettuato nel questionario & Interno
\\ \hline

\hypertarget{RFD8.1.2}{RFD8.1.2} & L’utente autenticato pro può eliminare un questionario che ha creato & Interno
\\ \hline

\hypertarget{RFD8.1.2.1}{RFD8.1.2.1} & L’utente autenticato pro può confermare se eliminare un questionario che ha creato & Interno
\\ \hline

\hypertarget{RFD8.1.3}{RFD8.1.3} & L’utente autenticato pro può visualizzare i risultati degli esaminandi  & Verbale 2016-01-11
\\ \hline

\hypertarget{RFD8.1.4}{RFD8.1.4} & L’utente autenticato pro può rendere il questionario compilabile da parte degli esaminandi & Interno
\\ \hline

\hypertarget{RFO8.2}{RFO8.2} & L’utente autenticato pro può creare un nuovo questionario & Capitolato
\\ \hline

\hypertarget{RFO8.2.1}{RFO8.2.1} & L’utente autenticato pro può scegliere l’argomento del questionario & Capitolato
\\ \hline

\hypertarget{RFD8.2.2}{RFD8.2.2} & L’utente autenticato pro può scegliere delle parole chiave che identifichino il questionario & Interno
\\ \hline

\hypertarget{RFD8.2.3}{RFD8.2.3} & L’utente autenticato pro può inserire il nome del questionario & Interno
\\ \hline

\hypertarget{RFO8.2.4}{RFO8.2.4} & L’utente autenticato pro può concludere il questionario & Interno
\\ \hline

\hypertarget{RFD8.2.4.1}{RFD8.2.4.1} & L’utente autenticato pro può consultare il resoconto del questionario dopo aver deciso di concluderlo & Interno
\\ \hline

\hypertarget{RFO8.2.4.2}{RFO8.2.4.2} & L’utente autenticato pro può approvare la conclusione del questionario & Interno
\\ \hline

\hypertarget{RFO8.3}{RFO8.3} & L’utente autenticato pro può gestire le domande di un questionario & Interno
\\ \hline

\hypertarget{RFO8.3.1}{RFO8.3.1} & L’utente autenticato pro può aggiungere delle domande all’interno del questionario & Interno
\\ \hline

\hypertarget{RFO8.3.1.1}{RFO8.3.1.1} & L’utente autenticato pro può ricercare una domanda all’interno del database di domande & Interno
\\ \hline

\hypertarget{RFO8.3.1.1.1}{RFO8.3.1.1.1} & L’utente autenticato pro può selezionare delle domande da inserire all’interno dei questionari & Interno
\\ \hline

\hypertarget{RFD8.3.1.1.2}{RFD8.3.1.1.2} & L’utente autenticato pro può applicare dei filtri per effettuare una ricerca delle domande dettagliata & Interno
\\ \hline

\hypertarget{RFO8.3.2}{RFO8.3.2} & L’utente autenticato pro può eliminare una domanda dal questionario & Interno
\\ \hline

\hypertarget{RFO8.3.2.1}{RFO8.3.2.1} & L’utente autenticato pro può confermare se eliminare una domanda dal questionario & Interno
\\ \hline

\hypertarget{RFD8.4}{RFD8.4} & L’utente autenticato pro può gestire le iscrizione degli esaminandi ai questionari & Interno
\\ \hline

\hypertarget{RFD8.4.1}{RFD8.4.1} & L’utente autenticato pro può selezionare il questionario del quale gestire le iscrizioni & Interno
\\ \hline

\hypertarget{RFD8.4.1.1}{RFD8.4.1.1} & L’utente autenticato pro può accettare le iscrizioni degli esaminandi ai questionari & Interno
\\ \hline

\hypertarget{RFO9}{RFO9} & L’utente non autenticato, l’utente autenticato e l’utente autenticato pro possono esercitarsi nella modalità allenamento & Verbale 2016-01-11
\\ \hline

\hypertarget{RFD9.1}{RFD9.1} & L'utente non autenticato, l'utente autenticato e l'utente autenticato pro possono decidere un argomento per fare un allenamento & Interno
\\ \hline

\hypertarget{RFD9.2}{RFD9.2} & L'utente non autenticato, l'utente autenticato e l'utente autenticato pro possono decidere delle parole chiave per filtrare maggiormente le domande poste durante l'allenamento & Interno
\\ \hline

\hypertarget{RFD9.3}{RFD9.3} & L'utente non autenticato, l'utente autenticato e l'utente autenticato pro possono scegliere il numero di domande che comporranno l'allenamento, potenzialmente possono scegliere la generazione di infinite domande & Interno
\\ \hline

\hypertarget{RFD9.4}{RFD9.4} & L'utente non autenticato, l'utente autenticato e l'utente autenticato pro possono rispondere alle domande proposte iniziando l'allenamento & Interno
\\ \hline

\hypertarget{RFD9.4.1}{RFD9.4.1} & L'utente non autenticato, l'utente autenticato e l'utente autenticato pro possono confermare una risposta durante un allenamento & Interno
\\ \hline

\hypertarget{RFD9.4.2}{RFD9.4.2} & L'utente non autenticato, l'utente autenticato e l'utente autenticato pro possono rilasciare un Like ad una domanda proposta durante un allenamento & Interno
\\ \hline

\hypertarget{RFD9.4.3}{RFD9.4.3} & L'utente non autenticato, l'utente autenticato e l'utente autenticato pro possono rilasciare commenti ad una domanda proposta durante un allenamento & Interno
\\ \hline

\hypertarget{RFD9.4.4}{RFD9.4.4} & L'utente non autenticato, l'utente autenticato e l'utente autenticato pro possono segnalare una domanda indicando il tipo di segnalazione e scrivendo un commento per essa & Interno
\\ \hline

\hypertarget{RFD9.4.5}{RFD9.4.5} & L'utente non autenticato, l'utente autenticato e l'utente autenticato pro possono avanzare alla domanda successiva (se presente) durante l'allenamento & Interno
\\ \hline

\hypertarget{RFD9.4.6}{RFD9.4.6} & L'utente non autenticato, l'utente autenticato e l'utente autenticato pro possono decidere di terminare l'allenamento in qualunque momento & Interno
\\ \hline

\hypertarget{RFD9.4.7}{RFD9.4.7} & Il sistema deve visualizzare le statistiche finali dell'allenamento svolto & Interno
\\ \hline

\hypertarget{RFD9.5}{RFD9.5} & Il sistema sceglie una domanda in base all'abilità dell'aversario sull'argomento scelto & Interno
\\ \hline

\hypertarget{RFD9.6}{RFD9.6} & Il sistema aggiorna automaticamente i dati sull'abilità dell'utente ad ogni risposta & Interno
\\ \hline

\hypertarget{RFD9.7}{RFD9.7} & Il sistema aggiorna automaticamente i dati sulla difficoltà di una domanda quando un utente risponde alla medesima & Interno
\\ \hline

\hypertarget{RFD10}{RFD10} & L’utente autenticato e l’utente autenticato pro possono visualizzare il proprio profilo & Interno
\\ \hline

\hypertarget{RFD10.1}{RFD10.1} & L’utente autenticato e l’utente autenticato pro possono andare alla pagina di gestione del profilo mediante l’apposito link & Interno
\\ \hline

\hypertarget{RFD10.2}{RFD10.2} & L’utente autenticato e l’utente autenticato pro possono andare alla pagina di gestione delle domande mediante l’apposito link & Interno
\\ \hline

\hypertarget{RFD10.3}{RFD10.3} & L’utente autenticato pro può andare alla pagina di gestione dei questionari mediante l’apposito link & Interno
\\ \hline

\hypertarget{RFD10.4}{RFD10.4} & L’utente autenticato e l’utente autenticato pro possono visualizzare la cronologia di tutti i questionari che hanno svolto & Interno
\\ \hline

\hypertarget{RFD10.4.1}{RFD10.4.1} & L’utente autenticato e l’utente autenticato pro possono selezionare e visualizzare le statistiche di un questionario scelto dalla cronologia & Interno
\\ \hline

\hypertarget{RFD10.5}{RFD10.5} & L’utente autenticato e l’utente autenticato pro possono visualizzare la lista dei questionari abilitati & Interno
\\ \hline

\hypertarget{RFD10.5.1}{RFD10.5.1} & L’utente autenticato e l’utente autenticato pro possono selezionare un questionario abilitato & Interno
\\ \hline

\hypertarget{RFD10.6}{RFD10.6} & L’utente autenticato e l’utente autenticato pro possono tornare alla home page mediante l’apposito link  & Interno
\\ \hline

\hypertarget{RFO11}{RFO11} & L’utente non autenticato, l'utente autenticato e l’utente autenticato pro possono rispondere alle domande & Capitolato
\\ \hline

\hypertarget{RFO11.1}{RFO11.1} & L’utente non autenticato, l’utente autenticato e l’utente autenticato pro possono rispondere ad una domanda vero/falso & Capitolato
\\ \hline

\hypertarget{RFO11.2}{RFO11.2} & L’utente non autenticato, l’utente autenticato e l’utente autenticato pro possono rispondere ad una domanda a risposta multipla & Capitolato
\\ \hline

\hypertarget{RFO11.3}{RFO11.3} & L’utente non autenticato, l’utente autenticato e l’utente autenticato pro possono compilare un esercizio di riempimento di uno spazio vuoto & Capitolato
\\ \hline

\hypertarget{RFO11.3.1}{RFO11.3.1} & L’utente non autenticato, l’utente autenticato e l’utente autenticato pro possono riempire lo spazio vuoto selezionato & Capitolato
\\ \hline

\hypertarget{RFD11.4}{RFD11.4} & L’utente non autenticato, l’utente autenticato e l’utente autenticato pro possono rispondere ad una domanda di collegamento & Capitolato
\\ \hline

\hypertarget{RFD11.4.1}{RFD11.4.1} & L’utente non autenticato, l’utente autenticato e l’utente autenticato pro possono collegare le voci & Capitolato
\\ \hline

\hypertarget{RFD11.5}{RFD11.5} & L’utente non autenticato, l’utente autenticato e l’utente autenticato pro possono ordinare delle immagini & Verbale 2016-01-11
\\ \hline

\hypertarget{RFD11.5.1}{RFD11.5.1} & L’utente non autenticato, l’utente autenticato e l’utente autenticato pro possono inserire un’immagine in uno spazio già occupato oppure no & Verbale 2016-01-11
\\ \hline

\hypertarget{RFD11.6}{RFD11.6} & L’utente non autenticato, l’utente autenticato e l’utente autenticato pro possono ordinare delle stringhe & Verbale 2016-01-11
\\ \hline

\hypertarget{RFD11.6.1}{RFD11.6.1} & L’utente non autenticato, l’utente autenticato e l’utente autenticato pro possono inserire una stringa in uno spazio già occupato oppure no & Verbale 2016-01-11
\\ \hline

\hypertarget{RFD11.7}{RFD11.7} & L’utente non autenticato, l’utente autenticato e l’utente autenticato pro possono rispondere ad una domanda con area cliccabile & Verbale 2016-01-11
\\ \hline

\hypertarget{RFD11.7.1}{RFD11.7.1} & L’utente non autenticato, l’utente autenticato e l’utente autenticato pro possono selezionare un’area cliccabile & Verbale 2016-01-11
\\ \hline

\hypertarget{RFO12}{RFO12} & L’utente autenticato e l’utente autenticato pro possono ricercare un utente & Interno
\\ \hline

\hypertarget{RFD12.1}{RFD12.1} & L’utente autenticato e l’utente autenticato pro possono inserire un nome e cognome/username nella barra di ricerca per ricercare un utente & Interno
\\ \hline

\hypertarget{RFD12.1.1}{RFD12.1.1} &  L’utente autenticato e l’utente autenticato pro possono selezionare l’utente ricercato & Interno
\\ \hline

\hypertarget{RFO13}{RFO13} & Il sistema deve gestire un sistema per proporre dei questionari & Capitolato
\\ \hline

\hypertarget{RFO14}{RFO14} & Il sistema deve proporre all’utilizzatore questionari specifici per l’argomento scelto & Capitolato
\\ \hline

\hypertarget{RFO15}{RFO15} & Il sistema deve creare dei questionari partendo dalle domande archiviate & Capitolato
\\ \hline

\hypertarget{RFO16}{RFO16} & Il sistema deve archiviare le domande attraverso uno specifico linguaggio chiamato QML & Capitolato
\\ \hline

\hypertarget{RFO16.1}{RFO16.1} & Il sistema attraverso il linguaggio QML deve gestire domande vero e falso & Capitolato
\\ \hline

\hypertarget{RFO16.2}{RFO16.2} & Il sistema attraverso il linguaggio QML deve gestire domande a risposte a scelta multipla. & Capitolato
\\ \hline

\hypertarget{RFO16.3}{RFO16.3} & Il sistema attraverso il linguaggio QML deve gestire esercizi con riempimento di spazi vuoti & Capitolato
\\ \hline

\hypertarget{RFO16.4}{RFO16.4} & Il sistema attraverso il linguaggio QML deve gestire esercizi con delle immagini & Capitolato
\\ \hline

\hypertarget{RFD16.5}{RFD16.5} & Il sistema attraverso il linguaggio QML deve gestire esercizi di ordinamento di scelte & Capitolato
\\ \hline

\hypertarget{RFD16.6}{RFD16.6} & Il sistema attraverso il linguaggio QML deve esercizi a corrispondenza di scelte & Capitolato
\\ \hline

\hypertarget{RFO16.7}{RFO16.7} & Il sistema deve tradurre le domande descritte in QML in HTML & Capitolato
\\ \hline

\hypertarget{RFO16.8}{RFO16.8} & Il sistema deve archiviare le domande suddivise per argomento & Capitolato
\\ \hline

\hypertarget{RFO17}{RFO17} & Il sistema deve archiviare i questionari creati con le domande & Capitolato
\\ \hline

\hypertarget{RFO18}{RFO18} & Il sistema deve proporre all’utilizzatore dei questionari preconfezionati & Capitolato
\\ \hline

\hypertarget{RFO19}{RFO19} & Il sistema deve permettere agli utilizzatori di poter creare domande e questionari da dispositivi desktop & Capitolato
\\ \hline

\hypertarget{RFO20}{RFO20} & Il sistema deve permettere agli utilizzatori di poter rispondere alle domande e ai questionari da dispositivi desktop, Tablet e SmartPhone & Capitolato
\\ \hline

\hypertarget{RFD21}{RFD21} & Il sistema deve archiviare i risultati dei questionari & Capitolato
\\ \hline

\hypertarget{RFD22}{RFD22} & Il sistema deve archiviare le statistiche delle risposte date ad ogni domanda & Capitolato
\\ \hline

\hypertarget{RFF23}{RFF23} & Il sistema deve valutare il candidato rispetto agli altri candidati cha hanno svolto lo stesso quiz & Capitolato
\\ \hline

\hypertarget{RFD24}{RFD24} & Il sistema deve creare questionari dinamicamente per un argomento scegliendo le domande in modo casuale & Capitolato
\\ \hline

\hypertarget{RFD25}{RFD25} & Il sistema deve creare questionari dinamicamente scegliendo le domande in base alle risposte date dai questionari precedenti & Capitolato
\\ \hline

\hypertarget{RFD26}{RFD26} & Il sistema deve creare questionari dinamicamente scegliendo tra le domande più difficili & Capitolato
\\ \hline

\hypertarget{RFD27}{RFD27} & Il sistema deve creare questionari dinamicamente scegliendo tra le lacune dei partecipanti & Capitolato
\\ \hline

\hypertarget{RFD28}{RFD28} & Il sistema deve permettere agli utilizzatori di proporre nuove domande & Capitolato
\\ \hline

\hypertarget{RFF29}{RFF29} & Il sistema deve permettere agli utilizzatori di segnalare positivamente una domanda & Capitolato
\\ \hline

\hypertarget{RFF30}{RFF30} & Il sistema deve permettere agli utilizzatori di commentare un domanda & Capitolato
\\ \hline

\hypertarget{RFF31}{RFF31} & Il sistema deve permettere agli utilizzatori di rispondere più volte ad una domanda & Capitolato
\\ \hline

\caption[Requisiti Funzionali]{Requisiti Funzionali}
\label{tabella:req0}
\end{longtable}
\clearpage
\subsection{Requisiti di Qualità}
\normalsize
\begin{longtable}{|c|>{\centering}m{7cm}|c|}
\hline
\textbf{Id Requisito} & \textbf{Descrizione} & \textbf{Fonti}\\
\hline
\endhead
\hypertarget{RQO1}{RQO1} & Deve essere fornito un manuale utente & Capitolato
\\ \hline

\hypertarget{RQO2}{RQO2} & Il manuale utente deve contenere una sezione un cui viene spiegato come installare correttamente l'applicazione & Interno
\\ \hline

\hypertarget{RQO3}{RQO3} & il manuale utente deve contenere una sezione in vui viene approfonditamente spiegato come utilizzare l'applicazione & Interno
\\ \hline

\hypertarget{RQO4}{RQO4} & Il manuale utente deve includere una sezione contenente un elenco di possibili errori e malfunzionamenti dell'applicazione e le loro possibili cause & Interno
\\ \hline

\hypertarget{RQO5}{RQO5} & Il manuale utente deve contenere una sezione che spiega come segnalare eventuali errori e malfunzionamenti & Interno
\\ \hline

\hypertarget{RQO6}{RQO6} & Deve essere fornito un manuale per gli utenti sviluppatori che intendono estendere l'applicazione & Capitolato
\\ \hline

\hypertarget{RQO7}{RQO7} & Il manuale per gli utenti sviluppatori che intendono estendere l'applicazione deve contenere una sezione che spiega come segnalare eventuali errori o malfunzionamenti & Interno
\\ \hline

\hypertarget{RQF8}{RQF8} & La documentazione per l'utente deve essere disponibile in lingua inglese & Interno
\\ \hline

\hypertarget{RQO9}{RQO9} & La documentazione per l'utente deve essere disponibile in lingua italiana & Interno
\\ \hline

\caption[Requisiti di Qualità]{Requisiti di Qualità}
\label{tabella:req2}
\end{longtable}
\clearpage
\subsection{Requisiti di Vincolo}
\normalsize
\begin{longtable}{|c|>{\centering}m{7cm}|c|}
\hline
\textbf{Id Requisito} & \textbf{Descrizione} & \textbf{Fonti}\\
\hline
\endhead
\hypertarget{RVO1}{RVO1} & L’applicazione deve utilizzare il linguaggio \textit{Javascript\ped{G}}  & Capitolato
\\ \hline

\hypertarget{RVO2}{RVO2} & L’applicazione deve utilizzare il linguaggio di \textit{markup\ped{G}} \textit{HTML5\ped{G}} & Capitolato
\\ \hline

\hypertarget{RVO3}{RVO3} & L’applicazione deve utilizzare \textit{fogli di stile\ped{G}} in \textit{CSS\ped{G}} & Capitolato
\\ \hline

\hypertarget{RVD4}{RVD4} & L’applicazione deve utilizzare \textit{fogli di stile\ped{G}} in \textit{CSS3\ped{G}} & Capitolato
\\ \hline

\hypertarget{RVO5}{RVO5} & L’applicazione deve funzionare su \textit{Mozilla Firefox\ped{G}} versione 33.0 o superiore per le funzionalità che riguardano la compilazione dei questionari e delle domande & Interno
\\ \hline

\hypertarget{RVO6}{RVO6} & L’applicazione deve funzionare su \textit{Google Chrome\ped{G}} versione 31.0 o superiore per le funzionalità che riguardano la compilazione dei questionari e delle domande & Interno
\\ \hline

\hypertarget{RVO7}{RVO7} & L’applicazione deve funzionare su \textit{Safari\ped{G}} versione 7.1 o superiore per le funzionalità che riguardano la compilazione dei questionari e delle domande & Interno
\\ \hline

\hypertarget{RVO8}{RVO8} & L’applicazione deve funzionare su \textit{Opera\ped{G}} versione 26.0 o superiore per le funzionalità che riguardano la compilazione dei questionari e delle domande & Interno
\\ \hline

\hypertarget{RVD9}{RVD9} & L’applicazione deve funzionare su \textit{Internet Explorer\ped{G}} versione 11 o superiore per le funzionalità che riguardano la compilazione dei questionari e delle domande & Interno
\\ \hline

\hypertarget{RVO10}{RVO10} & L’applicazione deve funzionare su \textit{Microsoft Edge\ped{G}} versione 25  o superiore per le funzionalità che riguardano la compilazione dei questionari e delle domande & Interno
\\ \hline

\hypertarget{RVD11}{RVD11} & L’applicazione deve funzionare su \textit{Android Browser\ped{G}} versione 4.4 o superiore per le funzionalità che riguardano la compilazione dei questionari e delle domande & Interno
\\ \hline

\hypertarget{RVO12}{RVO12} & L’applicazione deve funzionare su \textit{Safari per iOS\ped{G}} versione 7.1 o superiore per le funzionalità che riguardano la compilazione dei questionari e delle domande & Interno
\\ \hline

\hypertarget{RVD13}{RVD13} & L’applicazione deve funzionare su \textit{Google Chrome per iOS\ped{G}} versione 39 o superiore per le funzionalità che riguardano la compilazione dei questionari e delle domande & Interno
\\ \hline

\hypertarget{RVO14}{RVO14} & L’applicazione deve funzionare su \textit{Google Chrome per Android\ped{G}} versione 39 o superiore per le funzionalità che riguardano la compilazione dei questionari e delle domande & Interno
\\ \hline

\hypertarget{RVF15}{RVF15} & L’applicazione deve funzionare su \textit{Mozilla Firefox per Android\ped{G}} versione 33 o superiore per le funzionalità che riguardano la compilazione dei questionari e delle domande & Interno
\\ \hline

\hypertarget{RVF16}{RVF16} & L’applicazione deve funzionare su \textit{Microsoft Edge per Windows 10 mobile\ped{G}} versione 25 o superiore per le funzionalità che riguardano la compilazione dei questionari e delle domande & Interno
\\ \hline

\hypertarget{RVF17}{RVF17} & L’applicazione deve funzionare su \textit{Browser Opera Mobile per Android\ped{G}} versione 34 o superiore per le funzionalità che riguardano la compilazione dei questionari e delle domande & Interno
\\ \hline

\hypertarget{RVF18}{RVF18} & L’applicazione deve funzionare su \textit{Opera Mini per iOS\ped{G}} versione 12 o superiore per le funzionalità che riguardano la compilazione dei questionari e delle domande & Interno
\\ \hline

\hypertarget{RVD19}{RVD19} & L’applicazione deve funzionare su \textit{Safari per iOS\ped{G}} versione 7.1 o superiore per le funzionalità che riguardano la creazione dei questionari e delle domande & Interno
\\ \hline

\hypertarget{RVD20}{RVD20} & L’applicazione deve funzionare su \textit{Google Chrome per Android\ped{G}} versione 39 o superiore per le funzionalità che riguardano la creazione dei questionari e delle domande & Interno
\\ \hline

\hypertarget{RVF21}{RVF21} & L’applicazione deve funzionare su \textit{Android Browser\ped{G}} versione 4.4 o superiore per le funzionalità che riguardano la creazione dei questionari e delle domande & Interno
\\ \hline

\hypertarget{RVF22}{RVF22} & L’applicazione deve funzionare su \textit{Google Chrome per iOS\ped{G}} versione 39 o superiore per le funzionalità che riguardano la creazione dei questionari e delle domande & Interno
\\ \hline

\hypertarget{RVF23}{RVF23} & L’applicazione deve funzionare su \textit{Mozilla Firefox per Android\ped{G}} versione 33 o superiore per le funzionalità che riguardano la creazione dei questionari e delle domande & Interno
\\ \hline

\hypertarget{RVF24}{RVF24} & L’applicazione deve funzionare su \textit{Microsoft Edge per Windows 10 mobile\ped{G}} versione 25 o superiore per le funzionalità che riguardano la creazione dei questionari e delle domande & Interno
\\ \hline

\hypertarget{RVF25}{RVF25} & L’applicazione deve funzionare su \textit{Browser Opera per Android\ped{G}} versione 34 o superiore per le funzionalità che riguardano la creazione dei questionari e delle domande & Interno
\\ \hline

\hypertarget{RVF26}{RVF26} & L’applicazione deve funzionare su \textit{Opera Mini per iOS\ped{G}} versione 12 o superiore per le funzionalità che riguardano la creazione dei questionari e delle domande & Interno
\\ \hline

\caption[Requisiti di Vincolo]{Requisiti di Vincolo}
\label{tabella:req3}
\end{longtable}
\clearpage

\subsection{Tracciamento Fonti-Requisiti}
\normalsize
\begin{longtable}{|>{\centering}m{5cm}|m{5cm}<{\centering}|}
\hline 
\textbf{Fonte} & \textbf{Id Requisiti}\\
\hline
\endhead
\hyperlink{Capitolato}{Capitolato} & \hyperlink{RFO6}{RFO6}\\
& \hyperlink{RFO6.5}{RFO6.5}\\
& \hyperlink{RFO7}{RFO7}\\
& \hyperlink{RFO7.3}{RFO7.3}\\
& \hyperlink{RFO7.3.1}{RFO7.3.1}\\
& \hyperlink{RFO8.6}{RFO8.6}\\
& \hyperlink{RFO8.6.1}{RFO8.6.1}\\
& \hyperlink{RFO11}{RFO11}\\
& \hyperlink{RFO11.1}{RFO11.1}\\
& \hyperlink{RFO11.2}{RFO11.2}\\
& \hyperlink{RFO11.3}{RFO11.3}\\
& \hyperlink{RFD11.4}{RFD11.4}\\
& \hyperlink{RFD11.4.1}{RFD11.4.1}\\
& \hyperlink{RFO20}{RFO20}\\
& \hyperlink{RFO21}{RFO21}\\
& \hyperlink{RFO22}{RFO22}\\
& \hyperlink{RFO23}{RFO23}\\
& \hyperlink{RFO23.1}{RFO23.1}\\
& \hyperlink{RFO23.2}{RFO23.2}\\
& \hyperlink{RFO23.3}{RFO23.3}\\
& \hyperlink{RFO23.4}{RFO23.4}\\
& \hyperlink{RFD23.5}{RFD23.5}\\
& \hyperlink{RFD23.6}{RFD23.6}\\
& \hyperlink{RFO23.7}{RFO23.7}\\
& \hyperlink{RFO23.8}{RFO23.8}\\
& \hyperlink{RFO24}{RFO24}\\
& \hyperlink{RFO25}{RFO25}\\
& \hyperlink{RFO26}{RFO26}\\
& \hyperlink{RFO27}{RFO27}\\
& \hyperlink{RFD28}{RFD28}\\
& \hyperlink{RFD29}{RFD29}\\
& \hyperlink{RFF30}{RFF30}\\
& \hyperlink{RFD31}{RFD31}\\
& \hyperlink{RFD32}{RFD32}\\
& \hyperlink{RFD33}{RFD33}\\
& \hyperlink{RFD35}{RFD35}\\
& \hyperlink{RFF36}{RFF36}\\
& \hyperlink{RFF37}{RFF37}\\
& \hyperlink{RFF38}{RFF38}\\
& \hyperlink{RQO1}{RQO1}\\
& \hyperlink{RQO6}{RQO6}\\
& \hyperlink{RVO1}{RVO1}\\
& \hyperlink{RVO2}{RVO2}\\
& \hyperlink{RVO3}{RVO3}\\
& \hyperlink{RVD4}{RVD4}\\ \hline
\hyperlink{Interno}{Interno} & \hyperlink{RFO1}{RFO1}\\
& \hyperlink{RFO1.1}{RFO1.1}\\
& \hyperlink{RFO1.2}{RFO1.2}\\
& \hyperlink{RFO1.3}{RFO1.3}\\
& \hyperlink{RFO1.4}{RFO1.4}\\
& \hyperlink{RFO1.5}{RFO1.5}\\
& \hyperlink{RFO1.6}{RFO1.6}\\
& \hyperlink{RFO1.7}{RFO1.7}\\
& \hyperlink{RFO1.8}{RFO1.8}\\
& \hyperlink{RFO2.1}{RFO2.1}\\
& \hyperlink{RFO2.2}{RFO2.2}\\
& \hyperlink{RFO2.3}{RFO2.3}\\
& \hyperlink{RFO2.4}{RFO2.4}\\
& \hyperlink{RFO3}{RFO3}\\
& \hyperlink{RFO3.1}{RFO3.1}\\
& \hyperlink{RFD4}{RFD4}\\
& \hyperlink{RFD4.1}{RFD4.1}\\
& \hyperlink{RFD4.2}{RFD4.2}\\
& \hyperlink{RFD4.3}{RFD4.3}\\
& \hyperlink{RFD4.4}{RFD4.4}\\
& \hyperlink{RFD4.5}{RFD4.5}\\
& \hyperlink{RFD4.5.1}{RFD4.5.1}\\
& \hyperlink{RFD4.5.2}{RFD4.5.2}\\
& \hyperlink{RFD4.5.3}{RFD4.5.3}\\
& \hyperlink{RFD4.6}{RFD4.6}\\
& \hyperlink{RFD4.7}{RFD4.7}\\
& \hyperlink{RFD4.8}{RFD4.8}\\
& \hyperlink{RFD4.8.1}{RFD4.8.1}\\
& \hyperlink{RFD4.8.2}{RFD4.8.2}\\
& \hyperlink{RFD4.9}{RFD4.9}\\
& \hyperlink{RFD4.9.1}{RFD4.9.1}\\
& \hyperlink{RFD5}{RFD5}\\
& \hyperlink{RFD6.1}{RFD6.1}\\
& \hyperlink{RFD6.2}{RFD6.2}\\
& \hyperlink{RFD6.3}{RFD6.3}\\
& \hyperlink{RFD6.4}{RFD6.4}\\
& \hyperlink{RFD7.1}{RFD7.1}\\
& \hyperlink{RFD7.1.1}{RFD7.1.1}\\
& \hyperlink{RFD7.1.2}{RFD7.1.2}\\
& \hyperlink{RFD7.1.2.1}{RFD7.1.2.1}\\
& \hyperlink{RFD7.1.2.2}{RFD7.1.2.2}\\
& \hyperlink{RFD7.1.2.3}{RFD7.1.2.3}\\
& \hyperlink{RFD7.1.3}{RFD7.1.3}\\
& \hyperlink{RFD7.1.3.1}{RFD7.1.3.1}\\
& \hyperlink{RFD7.1.3.2}{RFD7.1.3.2}\\
& \hyperlink{RFD7.1.3.3}{RFD7.1.3.3}\\
& \hyperlink{RFD7.1.3.3.1}{RFD7.1.3.3.1}\\
& \hyperlink{RFD7.1.3.3.2}{RFD7.1.3.3.2}\\
& \hyperlink{RFD7.1.3.4}{RFD7.1.3.4}\\
& \hyperlink{RFD7.1.4}{RFD7.1.4}\\
& \hyperlink{RFD7.1.4.1}{RFD7.1.4.1}\\
& \hyperlink{RFD7.1.4.2}{RFD7.1.4.2}\\
& \hyperlink{RFD7.1.5}{RFD7.1.5}\\
& \hyperlink{RFD7.1.5.1}{RFD7.1.5.1}\\
& \hyperlink{RFD7.1.5.2}{RFD7.1.5.2}\\
& \hyperlink{RFD7.1.5.2.1}{RFD7.1.5.2.1}\\
& \hyperlink{RFD7.1.5.2.2}{RFD7.1.5.2.2}\\
& \hyperlink{RFD7.1.5.2.3}{RFD7.1.5.2.3}\\
& \hyperlink{RFD7.1.5.2.4}{RFD7.1.5.2.4}\\
& \hyperlink{RFD7.1.5.3}{RFD7.1.5.3}\\
& \hyperlink{RFD7.1.5.3.1}{RFD7.1.5.3.1}\\
& \hyperlink{RFD7.1.5.4.1}{RFD7.1.5.4.1}\\
& \hyperlink{RFD7.1.5.4.2}{RFD7.1.5.4.2}\\
& \hyperlink{RFD7.1.5.4.3}{RFD7.1.5.4.3}\\
& \hyperlink{RFD7.1.5.4.4}{RFD7.1.5.4.4}\\
& \hyperlink{RFD7.1.6}{RFD7.1.6}\\
& \hyperlink{RFD7.1.6.1}{RFD7.1.6.1}\\
& \hyperlink{RFD7.1.6.2}{RFD7.1.6.2}\\
& \hyperlink{RFD7.1.6.3}{RFD7.1.6.3}\\
& \hyperlink{RFD7.1.7}{RFD7.1.7}\\
& \hyperlink{RFD7.1.7.2}{RFD7.1.7.2}\\
& \hyperlink{RFD7.1.7.3}{RFD7.1.7.3}\\
& \hyperlink{RFD7.1.8}{RFD7.1.8}\\
& \hyperlink{RFD7.1.8.1}{RFD7.1.8.1}\\
& \hyperlink{RFD7.1.8.2}{RFD7.1.8.2}\\
& \hyperlink{RFD7.1.8.3}{RFD7.1.8.3}\\
& \hyperlink{RFD7.1.8.4}{RFD7.1.8.4}\\
& \hyperlink{RFD7.1.9}{RFD7.1.9}\\
& \hyperlink{RFD7.1.10}{RFD7.1.10}\\
& \hyperlink{RFD7.2}{RFD7.2}\\
& \hyperlink{RFD7.2.1}{RFD7.2.1}\\
& \hyperlink{RFD7.2.1.1}{RFD7.2.1.1}\\
& \hyperlink{RFD7.2.1.2}{RFD7.2.1.2}\\
& \hyperlink{RFD7.2.1.3}{RFD7.2.1.3}\\
& \hyperlink{RFD7.2.2}{RFD7.2.2}\\
& \hyperlink{RFD7.2.2.1}{RFD7.2.2.1}\\
& \hyperlink{RFD7.2.2.2}{RFD7.2.2.2}\\
& \hyperlink{RFD7.2.2.3}{RFD7.2.2.3}\\
& \hyperlink{RFD7.2.2.3.1}{RFD7.2.2.3.1}\\
& \hyperlink{RFD7.2.2.3.2}{RFD7.2.2.3.2}\\
& \hyperlink{RFD7.2.2.4}{RFD7.2.2.4}\\
& \hyperlink{RFD7.2.3}{RFD7.2.3}\\
& \hyperlink{RFD7.2.3.1}{RFD7.2.3.1}\\
& \hyperlink{RFD7.2.3.2}{RFD7.2.3.2}\\
& \hyperlink{RFD7.2.4}{RFD7.2.4}\\
& \hyperlink{RFD7.2.4.1}{RFD7.2.4.1}\\
& \hyperlink{RFD7.2.4.2}{RFD7.2.4.2}\\
& \hyperlink{RFD7.2.4.2.1}{RFD7.2.4.2.1}\\
& \hyperlink{RFD7.2.4.2.2}{RFD7.2.4.2.2}\\
& \hyperlink{RFD7.2.4.2.3}{RFD7.2.4.2.3}\\
& \hyperlink{RFD7.2.4.2.4}{RFD7.2.4.2.4}\\
& \hyperlink{RFD7.2.4.3}{RFD7.2.4.3}\\
& \hyperlink{RFD7.2.4.3.1}{RFD7.2.4.3.1}\\
& \hyperlink{RFD7.2.4.4}{RFD7.2.4.4}\\
& \hyperlink{RFD7.2.4.4.1}{RFD7.2.4.4.1}\\
& \hyperlink{RFD7.2.4.4.2}{RFD7.2.4.4.2}\\
& \hyperlink{RFD7.2.4.4.3}{RFD7.2.4.4.3}\\
& \hyperlink{RFD7.2.4.4.4}{RFD7.2.4.4.4}\\
& \hyperlink{RFD7.2.5}{RFD7.2.5}\\
& \hyperlink{RFD7.2.5.1}{RFD7.2.5.1}\\
& \hyperlink{RFD7.2.5.2}{RFD7.2.5.2}\\
& \hyperlink{RFD7.2.5.3}{RFD7.2.5.3}\\
& \hyperlink{RFD7.2.5.4}{RFD7.2.5.4}\\
& \hyperlink{RFD7.2.6}{RFD7.2.6}\\
& \hyperlink{RFD7.2.6.1}{RFD7.2.6.1}\\
& \hyperlink{RFD7.2.6.2}{RFD7.2.6.2}\\
& \hyperlink{RFD7.2.6.3}{RFD7.2.6.3}\\
& \hyperlink{RFD7.2.7}{RFD7.2.7}\\
& \hyperlink{RFD7.2.7.1}{RFD7.2.7.1}\\
& \hyperlink{RFD7.2.7.2}{RFD7.2.7.2}\\
& \hyperlink{RFD7.2.7.3}{RFD7.2.7.3}\\
& \hyperlink{RFD7.2.7.4}{RFD7.2.7.4}\\
& \hyperlink{RFD7.2.8}{RFD7.2.8}\\
& \hyperlink{RFD7.2.9}{RFD7.2.9}\\
& \hyperlink{RFD7.4}{RFD7.4}\\
& \hyperlink{RFD7.4.1}{RFD7.4.1}\\
& \hyperlink{RFD7.5}{RFD7.5}\\
& \hyperlink{RFD7.6}{RFD7.6}\\
& \hyperlink{RFO7.7}{RFO7.7}\\
& \hyperlink{RFD8.1}{RFD8.1}\\
& \hyperlink{RFD8.2}{RFD8.2}\\
& \hyperlink{RFD8.2.1}{RFD8.2.1}\\
& \hyperlink{RFD8.2.2}{RFD8.2.2}\\
& \hyperlink{RFD8.3}{RFD8.3}\\
& \hyperlink{RFD8.3.1}{RFD8.3.1}\\
& \hyperlink{RFD8.5}{RFD8.5}\\
& \hyperlink{RFD8.6.2}{RFD8.6.2}\\
& \hyperlink{RFD8.6.3}{RFD8.6.3}\\
& \hyperlink{RFO8.6.4}{RFO8.6.4}\\
& \hyperlink{RFD8.6.4.1}{RFD8.6.4.1}\\
& \hyperlink{RFO8.6.4.2}{RFO8.6.4.2}\\
& \hyperlink{RFO8.7}{RFO8.7}\\
& \hyperlink{RFO8.7.1}{RFO8.7.1}\\
& \hyperlink{RFO8.7.1.1}{RFO8.7.1.1}\\
& \hyperlink{RFO8.7.1.1.1}{RFO8.7.1.1.1}\\
& \hyperlink{RFO8.7.1.1.2}{RFO8.7.1.1.2}\\
& \hyperlink{RFO8.7.2}{RFO8.7.2}\\
& \hyperlink{RFO8.7.2.1}{RFO8.7.2.1}\\
& \hyperlink{RFD8.8}{RFD8.8}\\
& \hyperlink{RFD8.8.1}{RFD8.8.1}\\
& \hyperlink{RFD8.8.2}{RFD8.8.2}\\
& \hyperlink{RFD9.1}{RFD9.1}\\
& \hyperlink{RFD9.2}{RFD9.2}\\
& \hyperlink{RFD9.3}{RFD9.3}\\
& \hyperlink{RFD9.4}{RFD9.4}\\
& \hyperlink{RFD9.4.1}{RFD9.4.1}\\
& \hyperlink{RFD9.4.2}{RFD9.4.2}\\
& \hyperlink{RFD9.4.3}{RFD9.4.3}\\
& \hyperlink{RFD9.4.4}{RFD9.4.4}\\
& \hyperlink{RFD9.4.5}{RFD9.4.5}\\
& \hyperlink{RFD9.4.6}{RFD9.4.6}\\
& \hyperlink{RFD9.5}{RFD9.5}\\
& \hyperlink{RFD9.6}{RFD9.6}\\
& \hyperlink{RFD9.7}{RFD9.7}\\
& \hyperlink{RFD10}{RFD10}\\
& \hyperlink{RFD10.1}{RFD10.1}\\
& \hyperlink{RFD10.2}{RFD10.2}\\
& \hyperlink{RFD10.3}{RFD10.3}\\
& \hyperlink{RFD10.4}{RFD10.4}\\
& \hyperlink{RFD10.4.1}{RFD10.4.1}\\
& \hyperlink{RFD10.5}{RFD10.5}\\
& \hyperlink{RFD10.5.1}{RFD10.5.1}\\
& \hyperlink{RFD10.6}{RFD10.6}\\
& \hyperlink{RFD10.7}{RFD10.7}\\
& \hyperlink{RFD10.8}{RFD10.8}\\
& \hyperlink{RFD10.9}{RFD10.9}\\
& \hyperlink{RFD10.10}{RFD10.10}\\
& \hyperlink{RFD10.11}{RFD10.11}\\
& \hyperlink{RFO12}{RFO12}\\
& \hyperlink{RFD12.1}{RFD12.1}\\
& \hyperlink{RFD12.2}{RFD12.2}\\
& \hyperlink{RFD12.3}{RFD12.3}\\
& \hyperlink{RFD12.3.1}{RFD12.3.1}\\
& \hyperlink{RFD12.3.2}{RFD12.3.2}\\
& \hyperlink{RFD12.3.3}{RFD12.3.3}\\
& \hyperlink{RFD12.3.4}{RFD12.3.4}\\
& \hyperlink{RFD12.3.5}{RFD12.3.5}\\
& \hyperlink{RFD12.3.6}{RFD12.3.6}\\
& \hyperlink{RFF13}{RFF13}\\
& \hyperlink{RFF14}{RFF14}\\
& \hyperlink{RFF15}{RFF15}\\
& \hyperlink{RFF16}{RFF16}\\
& \hyperlink{RFD17}{RFD17}\\
& \hyperlink{RFD18}{RFD18}\\
& \hyperlink{RFD18.1}{RFD18.1}\\
& \hyperlink{RFF19}{RFF19}\\
& \hyperlink{RFF19.1}{RFF19.1}\\
& \hyperlink{RFF19.2}{RFF19.2}\\
& \hyperlink{RFF19.3}{RFF19.3}\\
& \hyperlink{RFD34}{RFD34}\\
& \hyperlink{RQO2}{RQO2}\\
& \hyperlink{RQO3}{RQO3}\\
& \hyperlink{RQO4}{RQO4}\\
& \hyperlink{RQO5}{RQO5}\\
& \hyperlink{RQO7}{RQO7}\\
& \hyperlink{RQF8}{RQF8}\\
& \hyperlink{RQO9}{RQO9}\\
& \hyperlink{RVO5}{RVO5}\\
& \hyperlink{RVO6}{RVO6}\\
& \hyperlink{RVO7}{RVO7}\\
& \hyperlink{RVO8}{RVO8}\\
& \hyperlink{RVD9}{RVD9}\\
& \hyperlink{RVO10}{RVO10}\\
& \hyperlink{RVD11}{RVD11}\\
& \hyperlink{RVO12}{RVO12}\\
& \hyperlink{RVD13}{RVD13}\\
& \hyperlink{RVO14}{RVO14}\\
& \hyperlink{RVF15}{RVF15}\\
& \hyperlink{RVF16}{RVF16}\\
& \hyperlink{RVF18}{RVF18}\\
& \hyperlink{RVD20}{RVD20}\\
& \hyperlink{RVF23}{RVF23}\\
& \hyperlink{RVF24}{RVF24}\\
& \hyperlink{RVF25}{RVF25}\\
& \hyperlink{RVF26}{RVF26}\\ \hline
\hyperlink{Verbale 2016-01-11}{Verbale 2016-01-11} & \hyperlink{RFO8}{RFO8}\\
& \hyperlink{RFD8.4}{RFD8.4}\\
& \hyperlink{RFO9}{RFO9}\\
& \hyperlink{RFD11.5}{RFD11.5}\\
& \hyperlink{RFD11.5.1}{RFD11.5.1}\\
& \hyperlink{RFD11.6}{RFD11.6}\\
& \hyperlink{RFD11.6.1}{RFD11.6.1}\\
& \hyperlink{RFD11.7}{RFD11.7}\\
& \hyperlink{RFD11.7.1}{RFD11.7.1}\\ \hline
\hyperlink{Verbale interno}{Verbale interno} & \hyperlink{RFO2}{RFO2}\\
& \hyperlink{RFD7.1.5.4}{RFD7.1.5.4}\\
& \hyperlink{RFD7.1.7.1}{RFD7.1.7.1}\\ \hline
\hyperref[UC2]{UC2} & \hyperlink{RFO1}{RFO1}\\
& \hyperlink{RFO1.1}{RFO1.1}\\
& \hyperlink{RFO1.2}{RFO1.2}\\
& \hyperlink{RFO1.3}{RFO1.3}\\
& \hyperlink{RFO1.4}{RFO1.4}\\
& \hyperlink{RFO1.5}{RFO1.5}\\
& \hyperlink{RFO1.6}{RFO1.6}\\
& \hyperlink{RFO1.7}{RFO1.7}\\
& \hyperlink{RFO1.8}{RFO1.8}\\ \hline
\hyperref[UC2.1]{UC2.1} & \hyperlink{RFO1.1}{RFO1.1}\\ \hline
\hyperref[UC2.2]{UC2.2} & \hyperlink{RFO1.2}{RFO1.2}\\ \hline
\hyperref[UC2.3]{UC2.3} & \hyperlink{RFO1.3}{RFO1.3}\\ \hline
\hyperref[UC2.4]{UC2.4} & \hyperlink{RFO1.4}{RFO1.4}\\ \hline
\hyperref[UC2.5]{UC2.5} & \hyperlink{RFO1.5}{RFO1.5}\\ \hline
\hyperref[UC2.6]{UC2.6} & \hyperlink{RFO1.6}{RFO1.6}\\ \hline
\hyperref[UC2.7]{UC2.7} & \hyperlink{RFO1.7}{RFO1.7}\\ \hline
\hyperref[UC2.8]{UC2.8} & \hyperlink{RFO1.8}{RFO1.8}\\ \hline
\hyperref[UC3]{UC3} & \hyperlink{RFO2}{RFO2}\\
& \hyperlink{RFO2.1}{RFO2.1}\\
& \hyperlink{RFO2.2}{RFO2.2}\\
& \hyperlink{RFO2.3}{RFO2.3}\\
& \hyperlink{RFO2.4}{RFO2.4}\\ \hline
\hyperref[UC3.1]{UC3.1} & \hyperlink{RFO2.1}{RFO2.1}\\ \hline
\hyperref[UC3.2]{UC3.2} & \hyperlink{RFO2.2}{RFO2.2}\\ \hline
\hyperref[UC3.3]{UC3.3} & \hyperlink{RFO2.3}{RFO2.3}\\ \hline
\hyperref[UC3.4]{UC3.4} & \hyperlink{RFO2.4}{RFO2.4}\\ \hline
\hyperref[UC4]{UC4} & \hyperlink{RFO3}{RFO3}\\
& \hyperlink{RFO3.1}{RFO3.1}\\ \hline
\hyperref[UC4.1]{UC4.1} & \hyperlink{RFO3.1}{RFO3.1}\\ \hline
\hyperref[UC5]{UC5} & \hyperlink{RFD4}{RFD4}\\
& \hyperlink{RFD4.1}{RFD4.1}\\
& \hyperlink{RFD4.2}{RFD4.2}\\
& \hyperlink{RFD4.3}{RFD4.3}\\
& \hyperlink{RFD4.4}{RFD4.4}\\
& \hyperlink{RFD4.5}{RFD4.5}\\
& \hyperlink{RFD4.6}{RFD4.6}\\
& \hyperlink{RFD4.7}{RFD4.7}\\
& \hyperlink{RFD4.8}{RFD4.8}\\
& \hyperlink{RFD4.9}{RFD4.9}\\ \hline
\hyperref[UC5.1]{UC5.1} & \hyperlink{RFD4.1}{RFD4.1}\\ \hline
\hyperref[UC5.2]{UC5.2} & \hyperlink{RFD4.2}{RFD4.2}\\ \hline
\hyperref[UC5.3]{UC5.3} & \hyperlink{RFD4.3}{RFD4.3}\\ \hline
\hyperref[UC5.4]{UC5.4} & \hyperlink{RFD4.4}{RFD4.4}\\ \hline
\hyperref[UC5.5]{UC5.5} & \hyperlink{RFD4.5}{RFD4.5}\\
& \hyperlink{RFD4.5.1}{RFD4.5.1}\\
& \hyperlink{RFD4.5.2}{RFD4.5.2}\\
& \hyperlink{RFD4.5.3}{RFD4.5.3}\\ \hline
\hyperref[UC5.5.1]{UC5.5.1} & \hyperlink{RFD4.5.1}{RFD4.5.1}\\ \hline
\hyperref[UC5.5.2]{UC5.5.2} & \hyperlink{RFD4.5.2}{RFD4.5.2}\\ \hline
\hyperref[UC5.5.3]{UC5.5.3} & \hyperlink{RFD4.5.3}{RFD4.5.3}\\ \hline
\hyperref[UC5.6]{UC5.6} & \hyperlink{RFD4.6}{RFD4.6}\\ \hline
\hyperref[UC5.7]{UC5.7} & \hyperlink{RFD4.7}{RFD4.7}\\ \hline
\hyperref[UC5.8]{UC5.8} & \hyperlink{RFD4.8}{RFD4.8}\\
& \hyperlink{RFD4.8.1}{RFD4.8.1}\\
& \hyperlink{RFD4.8.2}{RFD4.8.2}\\ \hline
\hyperref[UC5.8.1]{UC5.8.1} & \hyperlink{RFD4.8.1}{RFD4.8.1}\\ \hline
\hyperref[UC5.8.2]{UC5.8.2} & \hyperlink{RFD4.8.2}{RFD4.8.2}\\ \hline
\hyperref[UC5.9]{UC5.9} & \hyperlink{RFD4.9}{RFD4.9}\\
& \hyperlink{RFD4.9.1}{RFD4.9.1}\\ \hline
\hyperref[UC5.9.1]{UC5.9.1} & \hyperlink{RFD4.9.1}{RFD4.9.1}\\ \hline
\hyperref[UC6]{UC6} & \hyperlink{RFD5}{RFD5}\\ \hline
\hyperref[UC7]{UC7} & \hyperlink{RFO6}{RFO6}\\
& \hyperlink{RFD6.1}{RFD6.1}\\
& \hyperlink{RFD6.2}{RFD6.2}\\
& \hyperlink{RFD6.3}{RFD6.3}\\
& \hyperlink{RFD6.4}{RFD6.4}\\ \hline
\hyperref[UC7.1]{UC7.1} & \hyperlink{RFD6.1}{RFD6.1}\\ \hline
\hyperref[UC7.2]{UC7.2} & \hyperlink{RFD6.2}{RFD6.2}\\ \hline
\hyperref[UC7.3]{UC7.3} & \hyperlink{RFD6.3}{RFD6.3}\\ \hline
\hyperref[UC7.4]{UC7.4} & \hyperlink{RFD6.4}{RFD6.4}\\ \hline
\hyperref[UC8]{UC8} & \hyperlink{RFO7}{RFO7}\\
& \hyperlink{RFD7.1}{RFD7.1}\\
& \hyperlink{RFD7.2}{RFD7.2}\\
& \hyperlink{RFO7.3}{RFO7.3}\\
& \hyperlink{RFD7.4}{RFD7.4}\\
& \hyperlink{RFD7.5}{RFD7.5}\\
& \hyperlink{RFD7.6}{RFD7.6}\\
& \hyperlink{RFO7.7}{RFO7.7}\\ \hline
\hyperref[UC8.1]{UC8.1} & \hyperlink{RFD7.1}{RFD7.1}\\
& \hyperlink{RFD7.1.1}{RFD7.1.1}\\
& \hyperlink{RFD7.1.2}{RFD7.1.2}\\
& \hyperlink{RFD7.1.3}{RFD7.1.3}\\
& \hyperlink{RFD7.1.4}{RFD7.1.4}\\
& \hyperlink{RFD7.1.5}{RFD7.1.5}\\
& \hyperlink{RFD7.1.6}{RFD7.1.6}\\
& \hyperlink{RFD7.1.7}{RFD7.1.7}\\
& \hyperlink{RFD7.1.8}{RFD7.1.8}\\
& \hyperlink{RFD7.1.9}{RFD7.1.9}\\
& \hyperlink{RFD7.1.10}{RFD7.1.10}\\ \hline
\hyperref[UC8.1.1]{UC8.1.1} & \hyperlink{RFD7.1.1}{RFD7.1.1}\\ \hline
\hyperref[UC8.1.2]{UC8.1.2} & \hyperlink{RFD7.1.2}{RFD7.1.2}\\
& \hyperlink{RFD7.1.2.1}{RFD7.1.2.1}\\
& \hyperlink{RFD7.1.2.2}{RFD7.1.2.2}\\
& \hyperlink{RFD7.1.2.3}{RFD7.1.2.3}\\ \hline
\hyperref[UC8.1.2.1]{UC8.1.2.1} & \hyperlink{RFD7.1.2.1}{RFD7.1.2.1}\\ \hline
\hyperref[UC8.1.2.2]{UC8.1.2.2} & \hyperlink{RFD7.1.2.2}{RFD7.1.2.2}\\ \hline
\hyperref[UC8.1.2.3]{UC8.1.2.3} & \hyperlink{RFD7.1.2.3}{RFD7.1.2.3}\\ \hline
\hyperref[UC8.1.3]{UC8.1.3} & \hyperlink{RFD7.1.3}{RFD7.1.3}\\
& \hyperlink{RFD7.1.3.1}{RFD7.1.3.1}\\
& \hyperlink{RFD7.1.3.2}{RFD7.1.3.2}\\
& \hyperlink{RFD7.1.3.3}{RFD7.1.3.3}\\
& \hyperlink{RFD7.1.3.4}{RFD7.1.3.4}\\ \hline
\hyperref[UC8.1.3.1]{UC8.1.3.1} & \hyperlink{RFD7.1.3.1}{RFD7.1.3.1}\\ \hline
\hyperref[UC8.1.3.2]{UC8.1.3.2} & \hyperlink{RFD7.1.3.2}{RFD7.1.3.2}\\ \hline
\hyperref[UC8.1.3.3]{UC8.1.3.3} & \hyperlink{RFD7.1.3.3}{RFD7.1.3.3}\\
& \hyperlink{RFD7.1.3.3.1}{RFD7.1.3.3.1}\\
& \hyperlink{RFD7.1.3.3.2}{RFD7.1.3.3.2}\\ \hline
\hyperref[UC8.1.3.3.1]{UC8.1.3.3.1} & \hyperlink{RFD7.1.3.3.1}{RFD7.1.3.3.1}\\ \hline
\hyperref[UC8.1.3.3.2]{UC8.1.3.3.2} & \hyperlink{RFD7.1.3.3.2}{RFD7.1.3.3.2}\\ \hline
\hyperref[UC8.1.3.4]{UC8.1.3.4} & \hyperlink{RFD7.1.3.4}{RFD7.1.3.4}\\ \hline
\hyperref[UC8.1.4]{UC8.1.4} & \hyperlink{RFD7.1.4}{RFD7.1.4}\\
& \hyperlink{RFD7.1.4.1}{RFD7.1.4.1}\\
& \hyperlink{RFD7.1.4.2}{RFD7.1.4.2}\\ \hline
\hyperref[UC8.1.4.1]{UC8.1.4.1} & \hyperlink{RFD7.1.4.1}{RFD7.1.4.1}\\ \hline
\hyperref[UC8.1.4.2]{UC8.1.4.2} & \hyperlink{RFD7.1.4.2}{RFD7.1.4.2}\\ \hline
\hyperref[UC8.1.5]{UC8.1.5} & \hyperlink{RFD7.1.5}{RFD7.1.5}\\
& \hyperlink{RFD7.1.5.1}{RFD7.1.5.1}\\
& \hyperlink{RFD7.1.5.2}{RFD7.1.5.2}\\
& \hyperlink{RFD7.1.5.3}{RFD7.1.5.3}\\
& \hyperlink{RFD7.1.5.4}{RFD7.1.5.4}\\ \hline
\hyperref[UC8.1.5.1]{UC8.1.5.1} & \hyperlink{RFD7.1.5.1}{RFD7.1.5.1}\\ \hline
\hyperref[UC8.1.5.2]{UC8.1.5.2} & \hyperlink{RFD7.1.5.2}{RFD7.1.5.2}\\
& \hyperlink{RFD7.1.5.2.1}{RFD7.1.5.2.1}\\
& \hyperlink{RFD7.1.5.2.2}{RFD7.1.5.2.2}\\
& \hyperlink{RFD7.1.5.2.3}{RFD7.1.5.2.3}\\
& \hyperlink{RFD7.1.5.2.4}{RFD7.1.5.2.4}\\ \hline
\hyperref[UC8.1.5.2.1]{UC8.1.5.2.1} & \hyperlink{RFD7.1.5.2.1}{RFD7.1.5.2.1}\\ \hline
\hyperref[UC8.1.5.2.2]{UC8.1.5.2.2} & \hyperlink{RFD7.1.5.2.2}{RFD7.1.5.2.2}\\ \hline
\hyperref[UC8.1.5.2.3]{UC8.1.5.2.3} & \hyperlink{RFD7.1.5.2.3}{RFD7.1.5.2.3}\\ \hline
\hyperref[UC8.1.5.2.4]{UC8.1.5.2.4} & \hyperlink{RFD7.1.5.2.4}{RFD7.1.5.2.4}\\ \hline
\hyperref[UC8.1.5.3]{UC8.1.5.3} & \hyperlink{RFD7.1.5.3}{RFD7.1.5.3}\\
& \hyperlink{RFD7.1.5.3.1}{RFD7.1.5.3.1}\\ \hline
\hyperref[UC8.1.5.3.1]{UC8.1.5.3.1} & \hyperlink{RFD7.1.5.3.1}{RFD7.1.5.3.1}\\ \hline
\hyperref[UC8.1.5.4]{UC8.1.5.4} & \hyperlink{RFD7.1.5.4}{RFD7.1.5.4}\\
& \hyperlink{RFD7.1.5.4.1}{RFD7.1.5.4.1}\\
& \hyperlink{RFD7.1.5.4.2}{RFD7.1.5.4.2}\\
& \hyperlink{RFD7.1.5.4.3}{RFD7.1.5.4.3}\\
& \hyperlink{RFD7.1.5.4.4}{RFD7.1.5.4.4}\\ \hline
\hyperref[UC8.1.5.4.1]{UC8.1.5.4.1} & \hyperlink{RFD7.1.5.4.1}{RFD7.1.5.4.1}\\ \hline
\hyperref[UC8.1.5.4.2]{UC8.1.5.4.2} & \hyperlink{RFD7.1.5.4.2}{RFD7.1.5.4.2}\\ \hline
\hyperref[UC8.1.5.4.3]{UC8.1.5.4.3} & \hyperlink{RFD7.1.5.4.3}{RFD7.1.5.4.3}\\ \hline
\hyperref[UC8.1.5.4.4]{UC8.1.5.4.4} & \hyperlink{RFD7.1.5.4.4}{RFD7.1.5.4.4}\\ \hline
\hyperref[UC8.1.6]{UC8.1.6} & \hyperlink{RFD7.1.6}{RFD7.1.6}\\
& \hyperlink{RFD7.1.6.1}{RFD7.1.6.1}\\
& \hyperlink{RFD7.1.6.2}{RFD7.1.6.2}\\
& \hyperlink{RFD7.1.6.3}{RFD7.1.6.3}\\ \hline
\hyperref[UC8.1.6.1]{UC8.1.6.1} & \hyperlink{RFD7.1.6.1}{RFD7.1.6.1}\\ \hline
\hyperref[UC8.1.6.2]{UC8.1.6.2} & \hyperlink{RFD7.1.6.2}{RFD7.1.6.2}\\ \hline
\hyperref[UC8.1.6.3]{UC8.1.6.3} & \hyperlink{RFD7.1.6.3}{RFD7.1.6.3}\\ \hline
\hyperref[UC8.1.7]{UC8.1.7} & \hyperlink{RFD7.1.7}{RFD7.1.7}\\
& \hyperlink{RFD7.1.7.1}{RFD7.1.7.1}\\
& \hyperlink{RFD7.1.7.2}{RFD7.1.7.2}\\
& \hyperlink{RFD7.1.7.3}{RFD7.1.7.3}\\ \hline
\hyperref[UC8.1.7.1]{UC8.1.7.1} & \hyperlink{RFD7.1.7.1}{RFD7.1.7.1}\\ \hline
\hyperref[UC8.1.7.2]{UC8.1.7.2} & \hyperlink{RFD7.1.7.2}{RFD7.1.7.2}\\ \hline
\hyperref[UC8.1.7.3]{UC8.1.7.3} & \hyperlink{RFD7.1.7.3}{RFD7.1.7.3}\\ \hline
\hyperref[UC8.1.8]{UC8.1.8} & \hyperlink{RFD7.1.8}{RFD7.1.8}\\
& \hyperlink{RFD7.1.8.1}{RFD7.1.8.1}\\
& \hyperlink{RFD7.1.8.2}{RFD7.1.8.2}\\
& \hyperlink{RFD7.1.8.3}{RFD7.1.8.3}\\
& \hyperlink{RFD7.1.8.4}{RFD7.1.8.4}\\ \hline
\hyperref[UC8.1.8.1]{UC8.1.8.1} & \hyperlink{RFD7.1.8.1}{RFD7.1.8.1}\\ \hline
\hyperref[UC8.1.8.2]{UC8.1.8.2} & \hyperlink{RFD7.1.8.2}{RFD7.1.8.2}\\ \hline
\hyperref[UC8.1.8.3]{UC8.1.8.3} & \hyperlink{RFD7.1.8.3}{RFD7.1.8.3}\\ \hline
\hyperref[UC8.1.8.4]{UC8.1.8.4} & \hyperlink{RFD7.1.8.4}{RFD7.1.8.4}\\ \hline
\hyperref[UC8.1.9]{UC8.1.9} & \hyperlink{RFD7.1.9}{RFD7.1.9}\\ \hline
\hyperref[UC8.1.10]{UC8.1.10} & \hyperlink{RFD7.1.10}{RFD7.1.10}\\ \hline
\hyperref[UC8.2]{UC8.2} & \hyperlink{RFD7.2}{RFD7.2}\\
& \hyperlink{RFD7.2.1}{RFD7.2.1}\\
& \hyperlink{RFD7.2.2}{RFD7.2.2}\\
& \hyperlink{RFD7.2.3}{RFD7.2.3}\\
& \hyperlink{RFD7.2.4}{RFD7.2.4}\\
& \hyperlink{RFD7.2.5}{RFD7.2.5}\\
& \hyperlink{RFD7.2.6}{RFD7.2.6}\\
& \hyperlink{RFD7.2.7}{RFD7.2.7}\\
& \hyperlink{RFD7.2.8}{RFD7.2.8}\\
& \hyperlink{RFD7.2.9}{RFD7.2.9}\\ \hline
\hyperref[UC8.2.1]{UC8.2.1} & \hyperlink{RFD7.2.1}{RFD7.2.1}\\
& \hyperlink{RFD7.2.1.1}{RFD7.2.1.1}\\
& \hyperlink{RFD7.2.1.2}{RFD7.2.1.2}\\
& \hyperlink{RFD7.2.1.3}{RFD7.2.1.3}\\ \hline
\hyperref[UC8.2.1.1]{UC8.2.1.1} & \hyperlink{RFD7.2.1.1}{RFD7.2.1.1}\\ \hline
\hyperref[UC8.2.1.2]{UC8.2.1.2} & \hyperlink{RFD7.2.1.2}{RFD7.2.1.2}\\ \hline
\hyperref[UC8.2.1.3]{UC8.2.1.3} & \hyperlink{RFD7.2.1.3}{RFD7.2.1.3}\\ \hline
\hyperref[UC8.2.2]{UC8.2.2} & \hyperlink{RFD7.2.2}{RFD7.2.2}\\
& \hyperlink{RFD7.2.2.1}{RFD7.2.2.1}\\
& \hyperlink{RFD7.2.2.2}{RFD7.2.2.2}\\
& \hyperlink{RFD7.2.2.3}{RFD7.2.2.3}\\
& \hyperlink{RFD7.2.2.4}{RFD7.2.2.4}\\ \hline
\hyperref[UC8.2.2.1]{UC8.2.2.1} & \hyperlink{RFD7.2.2.1}{RFD7.2.2.1}\\ \hline
\hyperref[UC8.2.2.2]{UC8.2.2.2} & \hyperlink{RFD7.2.2.2}{RFD7.2.2.2}\\ \hline
\hyperref[UC8.2.2.3]{UC8.2.2.3} & \hyperlink{RFD7.2.2.3}{RFD7.2.2.3}\\
& \hyperlink{RFD7.2.2.3.1}{RFD7.2.2.3.1}\\
& \hyperlink{RFD7.2.2.3.2}{RFD7.2.2.3.2}\\ \hline
\hyperref[UC8.2.2.3.1]{UC8.2.2.3.1} & \hyperlink{RFD7.2.2.3.1}{RFD7.2.2.3.1}\\ \hline
\hyperref[UC8.2.2.3.2]{UC8.2.2.3.2} & \hyperlink{RFD7.2.2.3.2}{RFD7.2.2.3.2}\\ \hline
\hyperref[UC8.2.2.4]{UC8.2.2.4} & \hyperlink{RFD7.2.2.4}{RFD7.2.2.4}\\ \hline
\hyperref[UC8.2.3]{UC8.2.3} & \hyperlink{RFD7.2.3}{RFD7.2.3}\\
& \hyperlink{RFD7.2.3.1}{RFD7.2.3.1}\\
& \hyperlink{RFD7.2.3.2}{RFD7.2.3.2}\\ \hline
\hyperref[UC8.2.3.1]{UC8.2.3.1} & \hyperlink{RFD7.2.3.1}{RFD7.2.3.1}\\ \hline
\hyperref[UC8.2.3.2]{UC8.2.3.2} & \hyperlink{RFD7.2.3.2}{RFD7.2.3.2}\\ \hline
\hyperref[UC8.2.4]{UC8.2.4} & \hyperlink{RFD7.2.4}{RFD7.2.4}\\
& \hyperlink{RFD7.2.4.1}{RFD7.2.4.1}\\
& \hyperlink{RFD7.2.4.2}{RFD7.2.4.2}\\
& \hyperlink{RFD7.2.4.3}{RFD7.2.4.3}\\
& \hyperlink{RFD7.2.4.4}{RFD7.2.4.4}\\ \hline
\hyperref[UC8.2.4.1]{UC8.2.4.1} & \hyperlink{RFD7.2.4.1}{RFD7.2.4.1}\\ \hline
\hyperref[UC8.2.4.2]{UC8.2.4.2} & \hyperlink{RFD7.2.4.2}{RFD7.2.4.2}\\
& \hyperlink{RFD7.2.4.2.1}{RFD7.2.4.2.1}\\
& \hyperlink{RFD7.2.4.2.2}{RFD7.2.4.2.2}\\
& \hyperlink{RFD7.2.4.2.3}{RFD7.2.4.2.3}\\
& \hyperlink{RFD7.2.4.2.4}{RFD7.2.4.2.4}\\ \hline
\hyperref[UC8.2.4.2.1]{UC8.2.4.2.1} & \hyperlink{RFD7.2.4.2.1}{RFD7.2.4.2.1}\\ \hline
\hyperref[UC8.2.4.2.2]{UC8.2.4.2.2} & \hyperlink{RFD7.2.4.2.2}{RFD7.2.4.2.2}\\ \hline
\hyperref[UC8.2.4.2.3]{UC8.2.4.2.3} & \hyperlink{RFD7.2.4.2.3}{RFD7.2.4.2.3}\\ \hline
\hyperref[UC8.2.4.2.4]{UC8.2.4.2.4} & \hyperlink{RFD7.2.4.2.4}{RFD7.2.4.2.4}\\ \hline
\hyperref[UC8.2.4.3]{UC8.2.4.3} & \hyperlink{RFD7.2.4.3}{RFD7.2.4.3}\\
& \hyperlink{RFD7.2.4.3.1}{RFD7.2.4.3.1}\\ \hline
\hyperref[UC8.2.4.3.1]{UC8.2.4.3.1} & \hyperlink{RFD7.2.4.3.1}{RFD7.2.4.3.1}\\ \hline
\hyperref[UC8.2.4.4]{UC8.2.4.4} & \hyperlink{RFD7.2.4.4}{RFD7.2.4.4}\\
& \hyperlink{RFD7.2.4.4.1}{RFD7.2.4.4.1}\\
& \hyperlink{RFD7.2.4.4.2}{RFD7.2.4.4.2}\\
& \hyperlink{RFD7.2.4.4.3}{RFD7.2.4.4.3}\\
& \hyperlink{RFD7.2.4.4.4}{RFD7.2.4.4.4}\\ \hline
\hyperref[UC8.2.4.4.1]{UC8.2.4.4.1} & \hyperlink{RFD7.2.4.4.1}{RFD7.2.4.4.1}\\ \hline
\hyperref[UC8.2.4.4.2]{UC8.2.4.4.2} & \hyperlink{RFD7.2.4.4.2}{RFD7.2.4.4.2}\\ \hline
\hyperref[UC8.2.4.4.3]{UC8.2.4.4.3} & \hyperlink{RFD7.2.4.4.3}{RFD7.2.4.4.3}\\ \hline
\hyperref[UC8.2.4.4.4]{UC8.2.4.4.4} & \hyperlink{RFD7.2.4.4.4}{RFD7.2.4.4.4}\\ \hline
\hyperref[UC8.2.5]{UC8.2.5} & \hyperlink{RFD7.2.5}{RFD7.2.5}\\
& \hyperlink{RFD7.2.5.1}{RFD7.2.5.1}\\
& \hyperlink{RFD7.2.5.2}{RFD7.2.5.2}\\
& \hyperlink{RFD7.2.5.3}{RFD7.2.5.3}\\
& \hyperlink{RFD7.2.5.4}{RFD7.2.5.4}\\ \hline
\hyperref[UC8.2.5.1]{UC8.2.5.1} & \hyperlink{RFD7.2.5.1}{RFD7.2.5.1}\\ \hline
\hyperref[UC8.2.5.2]{UC8.2.5.2} & \hyperlink{RFD7.2.5.2}{RFD7.2.5.2}\\ \hline
\hyperref[UC8.2.5.3]{UC8.2.5.3} & \hyperlink{RFD7.2.5.3}{RFD7.2.5.3}\\ \hline
\hyperref[UC8.2.5.4]{UC8.2.5.4} & \hyperlink{RFD7.2.5.4}{RFD7.2.5.4}\\ \hline
\hyperref[UC8.2.6]{UC8.2.6} & \hyperlink{RFD7.2.6}{RFD7.2.6}\\
& \hyperlink{RFD7.2.6.1}{RFD7.2.6.1}\\
& \hyperlink{RFD7.2.6.2}{RFD7.2.6.2}\\
& \hyperlink{RFD7.2.6.3}{RFD7.2.6.3}\\ \hline
\hyperref[UC8.2.6.1]{UC8.2.6.1} & \hyperlink{RFD7.2.6.1}{RFD7.2.6.1}\\ \hline
\hyperref[UC8.2.6.2]{UC8.2.6.2} & \hyperlink{RFD7.2.6.2}{RFD7.2.6.2}\\ \hline
\hyperref[UC8.2.6.3]{UC8.2.6.3} & \hyperlink{RFD7.2.6.3}{RFD7.2.6.3}\\ \hline
\hyperref[UC8.2.7]{UC8.2.7} & \hyperlink{RFD7.2.7}{RFD7.2.7}\\
& \hyperlink{RFD7.2.7.1}{RFD7.2.7.1}\\
& \hyperlink{RFD7.2.7.2}{RFD7.2.7.2}\\
& \hyperlink{RFD7.2.7.3}{RFD7.2.7.3}\\
& \hyperlink{RFD7.2.7.4}{RFD7.2.7.4}\\ \hline
\hyperref[UC8.2.7.1]{UC8.2.7.1} & \hyperlink{RFD7.2.7.1}{RFD7.2.7.1}\\ \hline
\hyperref[UC8.2.7.2]{UC8.2.7.2} & \hyperlink{RFD7.2.7.2}{RFD7.2.7.2}\\ \hline
\hyperref[UC8.2.7.3]{UC8.2.7.3} & \hyperlink{RFD7.2.7.3}{RFD7.2.7.3}\\ \hline
\hyperref[UC8.2.7.4]{UC8.2.7.4} & \hyperlink{RFD7.2.7.4}{RFD7.2.7.4}\\ \hline
\hyperref[UC8.2.8]{UC8.2.8} & \hyperlink{RFD7.2.8}{RFD7.2.8}\\ \hline
\hyperref[UC8.2.9]{UC8.2.9} & \hyperlink{RFD7.2.9}{RFD7.2.9}\\ \hline
\hyperref[UC8.3]{UC8.3} & \hyperlink{RFO7.3}{RFO7.3}\\
& \hyperlink{RFO7.3.1}{RFO7.3.1}\\ \hline
\hyperref[UC8.3.1]{UC8.3.1} & \hyperlink{RFO7.3.1}{RFO7.3.1}\\ \hline
\hyperref[UC8.4]{UC8.4} & \hyperlink{RFD7.4}{RFD7.4}\\
& \hyperlink{RFD7.4.1}{RFD7.4.1}\\ \hline
\hyperref[UC8.4.1]{UC8.4.1} & \hyperlink{RFD7.4.1}{RFD7.4.1}\\ \hline
\hyperref[UC8.5]{UC8.5} & \hyperlink{RFD7.5}{RFD7.5}\\ \hline
\hyperref[UC8.6]{UC8.6} & \hyperlink{RFD7.6}{RFD7.6}\\ \hline
\hyperref[UC8.7]{UC8.7} & \hyperlink{RFO7.7}{RFO7.7}\\ \hline
\hyperref[UC9]{UC9} & \hyperlink{RFO8}{RFO8}\\
& \hyperlink{RFD8.1}{RFD8.1}\\
& \hyperlink{RFD8.2}{RFD8.2}\\
& \hyperlink{RFD8.3}{RFD8.3}\\
& \hyperlink{RFD8.4}{RFD8.4}\\
& \hyperlink{RFD8.5}{RFD8.5}\\
& \hyperlink{RFO8.6}{RFO8.6}\\
& \hyperlink{RFO8.7}{RFO8.7}\\
& \hyperlink{RFD8.8}{RFD8.8}\\ \hline
\hyperref[UC9.1]{UC9.1} & \hyperlink{RFD8.1}{RFD8.1}\\ \hline
\hyperref[UC9.2]{UC9.2} & \hyperlink{RFD8.2}{RFD8.2}\\
& \hyperlink{RFD8.2.1}{RFD8.2.1}\\
& \hyperlink{RFD8.2.2}{RFD8.2.2}\\ \hline
\hyperref[UC9.2.1]{UC9.2.1} & \hyperlink{RFD8.2.1}{RFD8.2.1}\\ \hline
\hyperref[UC9.2.2]{UC9.2.2} & \hyperlink{RFD8.2.2}{RFD8.2.2}\\ \hline
\hyperref[UC9.3]{UC9.3} & \hyperlink{RFD8.3}{RFD8.3}\\
& \hyperlink{RFD8.3.1}{RFD8.3.1}\\ \hline
\hyperref[UC9.3.1]{UC9.3.1} & \hyperlink{RFD8.3.1}{RFD8.3.1}\\ \hline
\hyperref[UC9.4]{UC9.4} & \hyperlink{RFD8.4}{RFD8.4}\\ \hline
\hyperref[UC9.5]{UC9.5} & \hyperlink{RFD8.5}{RFD8.5}\\ \hline
\hyperref[UC9.6]{UC9.6} & \hyperlink{RFO8.6}{RFO8.6}\\
& \hyperlink{RFO8.6.1}{RFO8.6.1}\\
& \hyperlink{RFD8.6.2}{RFD8.6.2}\\
& \hyperlink{RFD8.6.3}{RFD8.6.3}\\
& \hyperlink{RFO8.6.4}{RFO8.6.4}\\ \hline
\hyperref[UC9.6.1]{UC9.6.1} & \hyperlink{RFO8.6.1}{RFO8.6.1}\\ \hline
\hyperref[UC9.6.2]{UC9.6.2} & \hyperlink{RFD8.6.2}{RFD8.6.2}\\ \hline
\hyperref[UC9.6.3]{UC9.6.3} & \hyperlink{RFD8.6.3}{RFD8.6.3}\\ \hline
\hyperref[UC9.6.4]{UC9.6.4} & \hyperlink{RFO8.6.4}{RFO8.6.4}\\
& \hyperlink{RFD8.6.4.1}{RFD8.6.4.1}\\
& \hyperlink{RFO8.6.4.2}{RFO8.6.4.2}\\ \hline
\hyperref[UC9.6.4.1]{UC9.6.4.1} & \hyperlink{RFD8.6.4.1}{RFD8.6.4.1}\\ \hline
\hyperref[UC9.6.4.2]{UC9.6.4.2} & \hyperlink{RFO8.6.4.2}{RFO8.6.4.2}\\ \hline
\hyperref[UC9.7]{UC9.7} & \hyperlink{RFO8.7}{RFO8.7}\\
& \hyperlink{RFO8.7.1}{RFO8.7.1}\\
& \hyperlink{RFO8.7.2}{RFO8.7.2}\\ \hline
\hyperref[UC9.7.1]{UC9.7.1} & \hyperlink{RFO8.7.1.1}{RFO8.7.1.1}\\ \hline
\hyperref[UC9.7.1.1]{UC9.7.1.1} & \hyperlink{RFO8.7.1}{RFO8.7.1}\\
& \hyperlink{RFO8.7.1.1}{RFO8.7.1.1}\\
& \hyperlink{RFO8.7.1.1.1}{RFO8.7.1.1.1}\\
& \hyperlink{RFO8.7.1.1.2}{RFO8.7.1.1.2}\\ \hline
\hyperref[UC9.7.1.1.1]{UC9.7.1.1.1} & \hyperlink{RFO8.7.1.1.1}{RFO8.7.1.1.1}\\ \hline
\hyperref[UC9.7.1.1.2]{UC9.7.1.1.2} & \hyperlink{RFO8.7.1.1.2}{RFO8.7.1.1.2}\\ \hline
\hyperref[UC9.7.2]{UC9.7.2} & \hyperlink{RFO8.7.2}{RFO8.7.2}\\
& \hyperlink{RFO8.7.2.1}{RFO8.7.2.1}\\ \hline
\hyperref[UC9.7.2.1]{UC9.7.2.1} & \hyperlink{RFO8.7.2.1}{RFO8.7.2.1}\\ \hline
\hyperref[UC9.8]{UC9.8} & \hyperlink{RFD8.8}{RFD8.8}\\
& \hyperlink{RFD8.8.1}{RFD8.8.1}\\
& \hyperlink{RFD8.8.2}{RFD8.8.2}\\ \hline
\hyperref[UC9.8.1]{UC9.8.1} & \hyperlink{RFD8.8.1}{RFD8.8.1}\\ \hline
\hyperref[UC9.8.2]{UC9.8.2} & \hyperlink{RFD8.8.2}{RFD8.8.2}\\ \hline
\hyperref[UC10]{UC10} & \hyperlink{RFO9}{RFO9}\\
& \hyperlink{RFD9.1}{RFD9.1}\\
& \hyperlink{RFD9.2}{RFD9.2}\\
& \hyperlink{RFD9.3}{RFD9.3}\\
& \hyperlink{RFD9.4}{RFD9.4}\\
& \hyperlink{RFD31}{RFD31}\\
& \hyperlink{RFD32}{RFD32}\\
& \hyperlink{RFD33}{RFD33}\\
& \hyperlink{RFD34}{RFD34}\\ \hline
\hyperref[UC10.1]{UC10.1} & \hyperlink{RFD9.1}{RFD9.1}\\ \hline
\hyperref[UC10.2]{UC10.2} & \hyperlink{RFD9.2}{RFD9.2}\\ \hline
\hyperref[UC10.3]{UC10.3} & \hyperlink{RFD9.3}{RFD9.3}\\ \hline
\hyperref[UC10.4]{UC10.4} & \hyperlink{RFD9.4}{RFD9.4}\\
& \hyperlink{RFD9.4.1}{RFD9.4.1}\\
& \hyperlink{RFD9.4.2}{RFD9.4.2}\\
& \hyperlink{RFD9.4.3}{RFD9.4.3}\\
& \hyperlink{RFD9.4.4}{RFD9.4.4}\\
& \hyperlink{RFD9.4.5}{RFD9.4.5}\\
& \hyperlink{RFD9.4.6}{RFD9.4.6}\\ \hline
\hyperref[UC10.4.1]{UC10.4.1} & \hyperlink{RFD9.4.1}{RFD9.4.1}\\ \hline
\hyperref[UC10.4.2]{UC10.4.2} & \hyperlink{RFD9.4.2}{RFD9.4.2}\\ \hline
\hyperref[UC10.4.3]{UC10.4.3} & \hyperlink{RFD9.4.3}{RFD9.4.3}\\ \hline
\hyperref[UC10.4.3.1]{UC10.4.3.1} & \hyperlink{RFD9.4.3}{RFD9.4.3}\\ \hline
\hyperref[UC10.4.3.2]{UC10.4.3.2} & \hyperlink{RFD9.4.3}{RFD9.4.3}\\ \hline
\hyperref[UC10.4.4]{UC10.4.4} & \hyperlink{RFD9.4.4}{RFD9.4.4}\\ \hline
\hyperref[UC10.4.4.1]{UC10.4.4.1} & \hyperlink{RFD9.4.4}{RFD9.4.4}\\ \hline
\hyperref[UC10.4.4.2]{UC10.4.4.2} & \hyperlink{RFD9.4.4}{RFD9.4.4}\\ \hline
\hyperref[UC10.4.5]{UC10.4.5} & \hyperlink{RFD9.4.5}{RFD9.4.5}\\ \hline
\hyperref[UC10.4.6]{UC10.4.6} & \hyperlink{RFD9.4.6}{RFD9.4.6}\\ \hline
\hyperref[UC11]{UC11} & \hyperlink{RFD10}{RFD10}\\
& \hyperlink{RFD10.1}{RFD10.1}\\
& \hyperlink{RFD10.2}{RFD10.2}\\
& \hyperlink{RFD10.3}{RFD10.3}\\
& \hyperlink{RFD10.4}{RFD10.4}\\
& \hyperlink{RFD10.5}{RFD10.5}\\
& \hyperlink{RFD10.6}{RFD10.6}\\
& \hyperlink{RFD10.7}{RFD10.7}\\
& \hyperlink{RFD10.8}{RFD10.8}\\
& \hyperlink{RFD10.9}{RFD10.9}\\
& \hyperlink{RFD10.10}{RFD10.10}\\
& \hyperlink{RFD10.11}{RFD10.11}\\ \hline
\hyperref[UC11.1]{UC11.1} & \hyperlink{RFD10.1}{RFD10.1}\\ \hline
\hyperref[UC11.2]{UC11.2} & \hyperlink{RFD10.2}{RFD10.2}\\ \hline
\hyperref[UC11.3]{UC11.3} & \hyperlink{RFD10.3}{RFD10.3}\\ \hline
\hyperref[UC11.4]{UC11.4} & \hyperlink{RFD10.4}{RFD10.4}\\
& \hyperlink{RFD10.4.1}{RFD10.4.1}\\ \hline
\hyperref[UC11.4.1]{UC11.4.1} & \hyperlink{RFD10.4.1}{RFD10.4.1}\\ \hline
\hyperref[UC11.5]{UC11.5} & \hyperlink{RFD10.5}{RFD10.5}\\
& \hyperlink{RFD10.5.1}{RFD10.5.1}\\ \hline
\hyperref[UC11.5.1]{UC11.5.1} & \hyperlink{RFD10.5.1}{RFD10.5.1}\\ \hline
\hyperref[UC11.6]{UC11.6} & \hyperlink{RFD10.6}{RFD10.6}\\ \hline
\hyperref[UC11.7]{UC11.7} & \hyperlink{RFD10.7}{RFD10.7}\\ \hline
\hyperref[UC11.9]{UC11.9} & \hyperlink{RFD10.8}{RFD10.8}\\
& \hyperlink{RFD10.9}{RFD10.9}\\ \hline
\hyperref[UC11.10]{UC11.10} & \hyperlink{RFD10.10}{RFD10.10}\\ \hline
\hyperref[UC11.11]{UC11.11} & \hyperlink{RFD10.11}{RFD10.11}\\ \hline
\hyperref[UC12]{UC12} & \hyperlink{RFO11}{RFO11}\\
& \hyperlink{RFO11.1}{RFO11.1}\\
& \hyperlink{RFO11.2}{RFO11.2}\\
& \hyperlink{RFO11.3}{RFO11.3}\\
& \hyperlink{RFD11.4}{RFD11.4}\\
& \hyperlink{RFD11.5}{RFD11.5}\\
& \hyperlink{RFD11.6}{RFD11.6}\\
& \hyperlink{RFD11.7}{RFD11.7}\\ \hline
\hyperref[UC12.1]{UC12.1} & \hyperlink{RFO11.1}{RFO11.1}\\ \hline
\hyperref[UC12.2]{UC12.2} & \hyperlink{RFO11.2}{RFO11.2}\\ \hline
\hyperref[UC12.3]{UC12.3} & \hyperlink{RFO11.3}{RFO11.3}\\ \hline
\hyperref[UC12.4]{UC12.4} & \hyperlink{RFD11.4}{RFD11.4}\\
& \hyperlink{RFD11.4.1}{RFD11.4.1}\\ \hline
\hyperref[UC12.4.1]{UC12.4.1} & \hyperlink{RFD11.4.1}{RFD11.4.1}\\ \hline
\hyperref[UC12.4.1.1]{UC12.4.1.1} & \hyperlink{RFD11.4.1}{RFD11.4.1}\\ \hline
\hyperref[UC12.4.1.2]{UC12.4.1.2} & \hyperlink{RFD11.4.1}{RFD11.4.1}\\ \hline
\hyperref[UC12.5]{UC12.5} & \hyperlink{RFD11.5}{RFD11.5}\\
& \hyperlink{RFD11.5.1}{RFD11.5.1}\\ \hline
\hyperref[UC12.5.1]{UC12.5.1} & \hyperlink{RFD11.5.1}{RFD11.5.1}\\ \hline
\hyperref[UC12.5.1.1]{UC12.5.1.1} & \hyperlink{RFD11.5.1}{RFD11.5.1}\\ \hline
\hyperref[UC12.5.1.2]{UC12.5.1.2} & \hyperlink{RFD11.5.1}{RFD11.5.1}\\ \hline
\hyperref[UC12.6]{UC12.6} & \hyperlink{RFD11.6}{RFD11.6}\\
& \hyperlink{RFD11.6.1}{RFD11.6.1}\\ \hline
\hyperref[UC12.6.1]{UC12.6.1} & \hyperlink{RFD11.6.1}{RFD11.6.1}\\ \hline
\hyperref[UC12.6.1.1]{UC12.6.1.1} & \hyperlink{RFD11.6.1}{RFD11.6.1}\\ \hline
\hyperref[UC12.6.1.2]{UC12.6.1.2} & \hyperlink{RFD11.6.1}{RFD11.6.1}\\ \hline
\hyperref[UC12.7]{UC12.7} & \hyperlink{RFD11.7}{RFD11.7}\\
& \hyperlink{RFD11.7.1}{RFD11.7.1}\\ \hline
\hyperref[UC12.7.1]{UC12.7.1} & \hyperlink{RFD11.7.1}{RFD11.7.1}\\ \hline
\hyperref[UC13]{UC13} & \hyperlink{RFO12}{RFO12}\\
& \hyperlink{RFD12.1}{RFD12.1}\\
& \hyperlink{RFD12.2}{RFD12.2}\\
& \hyperlink{RFD12.3}{RFD12.3}\\
& \hyperlink{RFD12.3.6}{RFD12.3.6}\\ \hline
\hyperref[UC13.1]{UC13.1} & \hyperlink{RFD12.1}{RFD12.1}\\ \hline
\hyperref[UC13.2]{UC13.2} & \hyperlink{RFD12.2}{RFD12.2}\\ \hline
\hyperref[UC13.3]{UC13.3} & \hyperlink{RFD12.3}{RFD12.3}\\
& \hyperlink{RFD12.3.1}{RFD12.3.1}\\
& \hyperlink{RFD12.3.2}{RFD12.3.2}\\
& \hyperlink{RFD12.3.3}{RFD12.3.3}\\
& \hyperlink{RFD12.3.4}{RFD12.3.4}\\
& \hyperlink{RFD12.3.5}{RFD12.3.5}\\ \hline
\hyperref[UC13.3.1]{UC13.3.1} & \hyperlink{RFD12.3.1}{RFD12.3.1}\\ \hline
\hyperref[UC13.3.2]{UC13.3.2} & \hyperlink{RFD12.3.2}{RFD12.3.2}\\ \hline
\hyperref[UC13.3.3]{UC13.3.3} & \hyperlink{RFD12.3.3}{RFD12.3.3}\\ \hline
\hyperref[UC13.3.4]{UC13.3.4} & \hyperlink{RFD12.3.4}{RFD12.3.4}\\ \hline
\hyperref[UC13.3.5]{UC13.3.5} & \hyperlink{RFD12.3.5}{RFD12.3.5}\\ \hline
\hyperref[UC13.4]{UC13.4} & \hyperlink{RFD12.3.6}{RFD12.3.6}\\ \hline
\hyperref[UC14]{UC14} & \hyperlink{RFF13}{RFF13}\\ \hline
\hyperref[UC15]{UC15} & \hyperlink{RFF14}{RFF14}\\ \hline
\hyperref[UC16]{UC16} & \hyperlink{RFF15}{RFF15}\\ \hline
\hyperref[UC17]{UC17} & \hyperlink{RFF16}{RFF16}\\ \hline
\hyperref[UC18]{UC18} & \hyperlink{RFD17}{RFD17}\\ \hline
\hyperref[UC19]{UC19} & \hyperlink{RFD18}{RFD18}\\
& \hyperlink{RFD18.1}{RFD18.1}\\ \hline
\hyperref[UC19.1]{UC19.1} & \hyperlink{RFD18.1}{RFD18.1}\\ \hline
\hyperref[UC20]{UC20} & \hyperlink{RFF19}{RFF19}\\
& \hyperlink{RFF19.1}{RFF19.1}\\
& \hyperlink{RFF19.2}{RFF19.2}\\
& \hyperlink{RFF19.3}{RFF19.3}\\ \hline
\hyperref[UC20.1]{UC20.1} & \hyperlink{RFF19.1}{RFF19.1}\\ \hline
\hyperref[UC20.2]{UC20.2} & \hyperlink{RFF19.2}{RFF19.2}\\ \hline
\hyperref[UC20.3]{UC20.3} & \hyperlink{RFF19.3}{RFF19.3}\\ \hline
\caption[Tracciamento Fonti-Requisiti]{Tracciamento Fonti-Requisiti}
\label{tabella:fonti-requi}
\end{longtable}
\clearpage

\subsection{Tracciamento Requisiti-Fonti}
\normalsize
\begin{longtable}{|>{\centering}m{5cm}|m{5cm}<{\centering}|}
\hline 
\textbf{Id Requisito} & \textbf{Fonti}\\
\hline
\endhead
\hyperlink{RFO1}{RFO1} & \hyperlink{Interno}{Interno}\\
& \hyperref[UC2]{UC2}\\ \hline

\hyperlink{RFO1.1}{RFO1.1} & \hyperlink{Interno}{Interno}\\
& \hyperref[UC2]{UC2}\\
& \hyperref[UC2.1]{UC2.1}\\ \hline

\hyperlink{RFO1.2}{RFO1.2} & \hyperlink{Interno}{Interno}\\
& \hyperref[UC2]{UC2}\\
& \hyperref[UC2.2]{UC2.2}\\ \hline

\hyperlink{RFO1.3}{RFO1.3} & \hyperlink{Interno}{Interno}\\
& \hyperref[UC2]{UC2}\\
& \hyperref[UC2.3]{UC2.3}\\ \hline

\hyperlink{RFO1.4}{RFO1.4} & \hyperlink{Interno}{Interno}\\
& \hyperref[UC2]{UC2}\\
& \hyperref[UC2.4]{UC2.4}\\ \hline

\hyperlink{RFO1.5}{RFO1.5} & \hyperlink{Interno}{Interno}\\
& \hyperref[UC2]{UC2}\\
& \hyperref[UC2.5]{UC2.5}\\ \hline

\hyperlink{RFO1.6}{RFO1.6} & \hyperlink{Interno}{Interno}\\
& \hyperref[UC2]{UC2}\\
& \hyperref[UC2.6]{UC2.6}\\ \hline

\hyperlink{RFO1.7}{RFO1.7} & \hyperlink{Interno}{Interno}\\
& \hyperref[UC2]{UC2}\\
& \hyperref[UC2.7]{UC2.7}\\ \hline

\hyperlink{RFO1.8}{RFO1.8} & \hyperlink{Interno}{Interno}\\
& \hyperref[UC2.8]{UC2.8}\\ \hline

\hyperlink{RFO2}{RFO2} & \hyperlink{Verbale interno}{Verbale interno}\\
& \hyperref[UC3]{UC3}\\ \hline

\hyperlink{RFO2.1}{RFO2.1} & \hyperlink{Verbale interno}{Verbale interno}\\
& \hyperref[UC3]{UC3}\\
& \hyperref[UC3.1]{UC3.1}\\ \hline

\hyperlink{RFO2.1.1}{RFO2.1.1} & \hyperlink{Interno}{Interno}\\
& \hyperref[UC3.1]{UC3.1}\\
& \hyperref[UC3.1.1]{UC3.1.1}\\ \hline

\hyperlink{RFO2.1.2}{RFO2.1.2} & \hyperlink{Interno}{Interno}\\
& \hyperref[UC3.1]{UC3.1}\\
& \hyperref[UC3.1.2]{UC3.1.2}\\ \hline

\hyperlink{RFO2.1.3}{RFO2.1.3} & \hyperlink{Interno}{Interno}\\
& \hyperref[UC3.1]{UC3.1}\\
& \hyperref[UC3.1.3]{UC3.1.3}\\ \hline

\hyperlink{RFO2.1.4}{RFO2.1.4} & \hyperlink{Interno}{Interno}\\
& \hyperref[UC3.1]{UC3.1}\\
& \hyperref[UC3.1.4]{UC3.1.4}\\ \hline

\hyperlink{RFF2.2}{RFF2.2} & \hyperlink{Interno}{Interno}\\
& \hyperref[UC3]{UC3}\\
& \hyperref[UC3.2]{UC3.2}\\ \hline

\hyperlink{RFF2.3}{RFF2.3} & \hyperlink{Interno}{Interno}\\
& \hyperref[UC3]{UC3}\\
& \hyperref[UC3.3]{UC3.3}\\ \hline

\hyperlink{RFF2.4}{RFF2.4} & \hyperlink{Interno}{Interno}\\
& \hyperref[UC3]{UC3}\\
& \hyperref[UC3.4]{UC3.4}\\ \hline

\hyperlink{RFF2.5}{RFF2.5} & \hyperlink{Interno}{Interno}\\
& \hyperref[UC3]{UC3}\\
& \hyperref[UC3.5]{UC3.5}\\ \hline

\hyperlink{RFO3}{RFO3} & \hyperlink{Interno}{Interno}\\
& \hyperref[UC4]{UC4}\\ \hline

\hyperlink{RFO3.1}{RFO3.1} & \hyperlink{Interno}{Interno}\\
& \hyperref[UC4]{UC4}\\
& \hyperref[UC4.1]{UC4.1}\\ \hline

\hyperlink{RFD4}{RFD4} & \hyperlink{Interno}{Interno}\\ \hline

\hyperlink{RFD4.1}{RFD4.1} & \hyperlink{Interno}{Interno}\\
& \hyperref[UC5]{UC5}\\
& \hyperref[UC5.1]{UC5.1}\\ \hline

\hyperlink{RFD4.1.1}{RFD4.1.1} & \hyperlink{Interno}{Interno}\\
& \hyperref[UC5.1]{UC5.1}\\
& \hyperref[UC5.1.2]{UC5.1.2}\\ \hline

\hyperlink{RFD4.1.2}{RFD4.1.2} & \hyperlink{Interno}{Interno}\\
& \hyperref[UC5.1]{UC5.1}\\
& \hyperref[UC5.1.3]{UC5.1.3}\\ \hline

\hyperlink{RFD4.2}{RFD4.2} & \hyperlink{Interno}{Interno}\\
& \hyperref[UC5]{UC5}\\
& \hyperref[UC5.2]{UC5.2}\\ \hline

\hyperlink{RFD4.2.1}{RFD4.2.1} & \hyperlink{Interno}{Interno}\\
& \hyperref[UC5.2]{UC5.2}\\
& \hyperref[UC5.2.2]{UC5.2.2}\\ \hline

\hyperlink{RFD4.2.2}{RFD4.2.2} & \hyperlink{Interno}{Interno}\\
& \hyperref[UC5.2]{UC5.2}\\
& \hyperref[UC5.2.3]{UC5.2.3}\\ \hline

\hyperlink{RFD4.3}{RFD4.3} & \hyperlink{Interno}{Interno}\\
& \hyperref[UC5]{UC5}\\
& \hyperref[UC5.3]{UC5.3}\\ \hline

\hyperlink{RFD4.3.1}{RFD4.3.1} & \hyperlink{Interno}{Interno}\\
& \hyperref[UC5.3]{UC5.3}\\
& \hyperref[UC5.3.2]{UC5.3.2}\\ \hline

\hyperlink{RFD4.3.2}{RFD4.3.2} & \hyperlink{Interno}{Interno}\\
& \hyperref[UC5.3]{UC5.3}\\
& \hyperref[UC5.3.3]{UC5.3.3}\\ \hline

\hyperlink{RFD4.4}{RFD4.4} & \hyperlink{Interno}{Interno}\\
& \hyperref[UC5]{UC5}\\
& \hyperref[UC5.4]{UC5.4}\\ \hline

\hyperlink{RFD4.4.1}{RFD4.4.1} & \hyperlink{Interno}{Interno}\\
& \hyperref[UC5.4]{UC5.4}\\
& \hyperref[UC5.4.2]{UC5.4.2}\\ \hline

\hyperlink{RFD4.4.2}{RFD4.4.2} & \hyperlink{Interno}{Interno}\\
& \hyperref[UC5.4]{UC5.4}\\
& \hyperref[UC5.4.3]{UC5.4.3}\\ \hline

\hyperlink{RFD4.5}{RFD4.5} & \hyperlink{Interno}{Interno}\\
& \hyperref[UC5]{UC5}\\
& \hyperref[UC5.5]{UC5.5}\\ \hline

\hyperlink{RFD4.5.1}{RFD4.5.1} & \hyperlink{Interno}{Interno}\\
& \hyperref[UC5.5]{UC5.5}\\
& \hyperref[UC5.5.2]{UC5.5.2}\\ \hline

\hyperlink{RFD4.5.2}{RFD4.5.2} & \hyperlink{Interno}{Interno}\\
& \hyperref[UC5.5]{UC5.5}\\
& \hyperref[UC5.5.3]{UC5.5.3}\\ \hline

\hyperlink{RFD4.6}{RFD4.6} & \hyperlink{Interno}{Interno}\\
& \hyperref[UC5]{UC5}\\
& \hyperref[UC5.6]{UC5.6}\\ \hline

\hyperlink{RFD4.6.1}{RFD4.6.1} & \hyperlink{Interno}{Interno}\\
& \hyperref[UC5.6]{UC5.6}\\
& \hyperref[UC5.6.4]{UC5.6.4}\\ \hline

\hyperlink{RFD4.6.2}{RFD4.6.2} & \hyperlink{Interno}{Interno}\\
& \hyperref[UC5.6]{UC5.6}\\
& \hyperref[UC5.6.5]{UC5.6.5}\\ \hline

\hyperlink{RFD4.7}{RFD4.7} & \hyperlink{Interno}{Interno}\\
& \hyperref[UC5]{UC5}\\
& \hyperref[UC5.7]{UC5.7}\\ \hline

\hyperlink{RFD4.7.1}{RFD4.7.1} & \hyperlink{Interno}{Interno}\\
& \hyperref[UC5.7]{UC5.7}\\
& \hyperref[UC5.7.2]{UC5.7.2}\\ \hline

\hyperlink{RFD4.8}{RFD4.8} & \hyperlink{Interno}{Interno}\\
& \hyperref[UC5]{UC5}\\
& \hyperref[UC5.8]{UC5.8}\\ \hline

\hyperlink{RFD4.8.1}{RFD4.8.1} & \hyperlink{Interno}{Interno}\\
& \hyperref[UC5.8]{UC5.8}\\
& \hyperref[UC5.8.1]{UC5.8.1}\\ \hline

\hyperlink{RFD5}{RFD5} & \hyperlink{Interno}{Interno}\\
& \hyperref[UC6]{UC6}\\ \hline

\hyperlink{RFD5.1}{RFD5.1} & \hyperlink{Interno}{Interno}\\
& \hyperref[UC6]{UC6}\\
& \hyperref[UC6.1]{UC6.1}\\ \hline

\hyperlink{RFD5.2}{RFD5.2} & \hyperlink{Interno}{Interno}\\
& \hyperref[UC6]{UC6}\\
& \hyperref[UC6.2]{UC6.2}\\ \hline

\hyperlink{RFD5.2.1}{RFD5.2.1} & \hyperlink{Interno}{Interno}\\
& \hyperref[UC6.2]{UC6.2}\\
& \hyperref[UC6.2.1]{UC6.2.1}\\ \hline

\hyperlink{RFD5.3}{RFD5.3} & \hyperlink{Interno}{Interno}\\
& \hyperref[UC6]{UC6}\\
& \hyperref[UC6.3]{UC6.3}\\ \hline

\hyperlink{RFO6}{RFO6} & \hyperlink{Capitolato}{Capitolato}\\
& \hyperref[UC7]{UC7}\\ \hline

\hyperlink{RFD6.1}{RFD6.1} & \hyperlink{Interno}{Interno}\\
& \hyperref[UC7]{UC7}\\
& \hyperref[UC7.1]{UC7.1}\\ \hline

\hyperlink{RFD6.2}{RFD6.2} & \hyperlink{Interno}{Interno}\\
& \hyperref[UC7]{UC7}\\
& \hyperref[UC7.2]{UC7.2}\\ \hline

\hyperlink{RFD6.3}{RFD6.3} & \hyperlink{Interno}{Interno}\\
& \hyperref[UC7]{UC7}\\
& \hyperref[UC7.3]{UC7.3}\\ \hline

\hyperlink{RFD6.4}{RFD6.4} & \hyperlink{Interno}{Interno}\\
& \hyperref[UC7]{UC7}\\
& \hyperref[UC7.4]{UC7.4}\\ \hline

\hyperlink{RFO6.5}{RFO6.5} & \hyperlink{Capitolato}{Capitolato}\\ \hline

\hyperlink{RFO7}{RFO7} & \hyperlink{Capitolato}{Capitolato}\\ \hline

\hyperlink{RFO7.1}{RFO7.1} & \hyperlink{Capitolato}{Capitolato}\\
& \hyperref[UC8]{UC8}\\
& \hyperref[UC8.1]{UC8.1}\\ \hline

\hyperlink{RFD7.1.1}{RFD7.1.1} & \hyperlink{Interno}{Interno}\\
& \hyperref[UC8.1]{UC8.1}\\
& \hyperref[UC8.1.1]{UC8.1.1}\\ \hline

\hyperlink{RFD7.1.2}{RFD7.1.2} & \hyperlink{Interno}{Interno}\\
& \hyperref[UC8.1]{UC8.1}\\
& \hyperref[UC8.1.2]{UC8.1.2}\\ \hline

\hyperlink{RFO7.1.3}{RFO7.1.3} & \hyperlink{Interno}{Interno}\\
& \hyperref[UC8.1]{UC8.1}\\
& \hyperref[UC8.1.3]{UC8.1.3}\\ \hline

\hyperlink{RFO7.1.3.1}{RFO7.1.3.1} & \hyperlink{Capitolato}{Capitolato}\\
& \hyperref[UC8.1.3]{UC8.1.3}\\
& \hyperref[UC8.1.3.1]{UC8.1.3.1}\\ \hline

\hyperlink{RFO7.1.3.1.1}{RFO7.1.3.1.1} & \hyperlink{Capitolato}{Capitolato}\\
& \hyperref[UC8.1.3.1]{UC8.1.3.1}\\
& \hyperref[UC8.1.3.1.1]{UC8.1.3.1.1}\\ \hline

\hyperlink{RFD7.1.3.1.2}{RFD7.1.3.1.2} & \hyperlink{Capitolato}{Capitolato}\\
& \hyperref[UC8.1.3.1]{UC8.1.3.1}\\
& \hyperref[UC8.1.3.1.2]{UC8.1.3.1.2}\\ \hline

\hyperlink{RFD7.1.3.1.2.1}{RFD7.1.3.1.2.1} & \hyperlink{Interno}{Interno}\\
& \hyperref[UC8.1.3.1.2]{UC8.1.3.1.2}\\
& \hyperref[UC8.1.3.1.2.1]{UC8.1.3.1.2.1}\\ \hline

\hyperlink{RFO7.1.3.1.3}{RFO7.1.3.1.3} & \hyperlink{Capitolato}{Capitolato}\\
& \hyperref[UC8.1.3.1]{UC8.1.3.1}\\
& \hyperref[UC8.1.3.1.3]{UC8.1.3.1.3}\\ \hline

\hyperlink{RFO7.1.3.2}{RFO7.1.3.2} & \hyperlink{Capitolato}{Capitolato}\\
& \hyperref[UC8.1.3]{UC8.1.3}\\
& \hyperref[UC8.1.3.2]{UC8.1.3.2}\\ \hline

\hyperlink{RFO7.1.3.2.1}{RFO7.1.3.2.1} & \hyperlink{Capitolato}{Capitolato}\\
& \hyperref[UC8.1.3.1.2]{UC8.1.3.1.2}\\
& \hyperref[UC8.1.3.1.2.1]{UC8.1.3.1.2.1}\\ \hline

\hyperlink{RFD7.1.3.2.2}{RFD7.1.3.2.2} & \hyperlink{Capitolato}{Capitolato}\\
& \hyperref[UC8.1.3.2]{UC8.1.3.2}\\
& \hyperref[UC8.1.3.2.2]{UC8.1.3.2.2}\\ \hline

\hyperlink{RFD7.1.3.2.2.1}{RFD7.1.3.2.2.1} & \hyperlink{Interno}{Interno}\\
& \hyperref[UC8.1.3.2.2]{UC8.1.3.2.2}\\
& \hyperref[UC8.1.3.2.2.1]{UC8.1.3.2.2.1}\\ \hline

\hyperlink{RFO7.1.3.2.3}{RFO7.1.3.2.3} & \hyperlink{Capitolato}{Capitolato}\\
& \hyperref[UC8.1.3.2]{UC8.1.3.2}\\
& \hyperref[UC8.1.3.2.3]{UC8.1.3.2.3}\\ \hline

\hyperlink{RFO7.1.3.2.3.1}{RFO7.1.3.2.3.1} & \hyperlink{Capitolato}{Capitolato}\\
& \hyperref[UC8.1.3.2.3]{UC8.1.3.2.3}\\
& \hyperref[UC8.1.3.2.3.1]{UC8.1.3.2.3.1}\\ \hline

\hyperlink{RFD7.1.3.2.3.1.1}{RFD7.1.3.2.3.1.1} & \hyperlink{Interno}{Interno}\\
& \hyperref[UC8.1.3.2.3.1]{UC8.1.3.2.3.1}\\
& \hyperref[UC8.1.3.2.3.1.1]{UC8.1.3.2.3.1.1}\\ \hline

\hyperlink{RFD7.1.3.2.3.2}{RFD7.1.3.2.3.2} & \hyperlink{Capitolato}{Capitolato}\\
& \hyperref[UC8.1.3.2.3]{UC8.1.3.2.3}\\
& \hyperref[UC8.1.3.2.3.2]{UC8.1.3.2.3.2}\\ \hline

\hyperlink{RFD7.1.3.2.3.2.1}{RFD7.1.3.2.3.2.1} & \hyperlink{Interno}{Interno}\\
& \hyperref[UC8.1.3.2.3.2]{UC8.1.3.2.3.2}\\
& \hyperref[UC8.1.3.2.3.2.1]{UC8.1.3.2.3.2.1}\\ \hline

\hyperlink{RFO7.1.3.2.4}{RFO7.1.3.2.4} & \hyperlink{Capitolato}{Capitolato}\\
& \hyperref[UC8.1.3.2]{UC8.1.3.2}\\
& \hyperref[UC8.1.3.2.4]{UC8.1.3.2.4}\\ \hline

\hyperlink{RFD7.1.3.3}{RFD7.1.3.3} & \hyperlink{Capitolato}{Capitolato}\\
& \hyperref[UC8.1.3]{UC8.1.3}\\
& \hyperref[UC8.1.3.3]{UC8.1.3.3}\\ \hline

\hyperlink{RFD7.1.3.3.1}{RFD7.1.3.3.1} & \hyperlink{Capitolato}{Capitolato}\\
& \hyperref[UC8.1.3.3]{UC8.1.3.3}\\
& \hyperref[UC8.1.3.3.1]{UC8.1.3.3.1}\\ \hline

\hyperlink{RFD7.1.3.3.2}{RFD7.1.3.3.2} & \hyperlink{Capitolato}{Capitolato}\\
& \hyperref[UC8.1.3.3]{UC8.1.3.3}\\
& \hyperref[UC8.1.3.3.2]{UC8.1.3.3.2}\\ \hline

\hyperlink{RFD7.1.3.4}{RFD7.1.3.4} & \hyperlink{Capitolato}{Capitolato}\\
& \hyperref[UC8.1.3]{UC8.1.3}\\
& \hyperref[UC8.1.3.4]{UC8.1.3.4}\\ \hline

\hyperlink{RFD7.1.3.4.1}{RFD7.1.3.4.1} & \hyperlink{Capitolato}{Capitolato}\\
& \hyperref[UC8.1.3.4]{UC8.1.3.4}\\
& \hyperref[UC8.1.3.4.1]{UC8.1.3.4.1}\\ \hline

\hyperlink{RFD7.1.3.4.2}{RFD7.1.3.4.2} & \hyperlink{Capitolato}{Capitolato}\\
& \hyperref[UC8.1.3.4]{UC8.1.3.4}\\
& \hyperref[UC8.1.3.4.2]{UC8.1.3.4.2}\\ \hline

\hyperlink{RFD7.1.3.4.2.1}{RFD7.1.3.4.2.1} & \hyperlink{Capitolato}{Capitolato}\\
& \hyperref[UC8.1.3.4.2]{UC8.1.3.4.2}\\
& \hyperref[UC8.1.3.4.2.1]{UC8.1.3.4.2.1}\\ \hline

\hyperlink{RFD7.1.3.4.2.2}{RFD7.1.3.4.2.2} & \hyperlink{Capitolato}{Capitolato}\\
& \hyperref[UC8.1.3.4.2]{UC8.1.3.4.2}\\
& \hyperref[UC8.1.3.4.2.2]{UC8.1.3.4.2.2}\\ \hline

\hyperlink{RFD7.1.3.4.2.3}{RFD7.1.3.4.2.3} & \hyperlink{Capitolato}{Capitolato}\\
& \hyperref[UC8.1.3.4.2]{UC8.1.3.4.2}\\
& \hyperref[UC8.1.3.4.2.3]{UC8.1.3.4.2.3}\\ \hline

\hyperlink{RFD7.1.3.4.2.4}{RFD7.1.3.4.2.4} & \hyperlink{Capitolato}{Capitolato}\\
& \hyperref[UC8.1.3.4.2]{UC8.1.3.4.2}\\
& \hyperref[UC8.1.3.4.2.4]{UC8.1.3.4.2.4}\\ \hline

\hyperlink{RFD7.1.3.4.3}{RFD7.1.3.4.3} & \hyperlink{Interno}{Interno}\\
& \hyperref[UC8.1.3.4]{UC8.1.3.4}\\
& \hyperref[UC8.1.3.4.3]{UC8.1.3.4.3}\\ \hline

\hyperlink{RFD7.1.3.4.3.1}{RFD7.1.3.4.3.1} & \hyperlink{Interno}{Interno}\\
& \hyperref[UC8.1.3.4.3]{UC8.1.3.4.3}\\
& \hyperref[UC8.1.3.4.3.1]{UC8.1.3.4.3.1}\\ \hline

\hyperlink{RFD7.1.3.4.4}{RFD7.1.3.4.4} & \hyperlink{Interno}{Interno}\\
& \hyperref[UC8.1.3.4]{UC8.1.3.4}\\
& \hyperref[UC8.1.3.4.4]{UC8.1.3.4.4}\\ \hline

\hyperlink{RFD7.1.3.4.4.1}{RFD7.1.3.4.4.1} & \hyperlink{Interno}{Interno}\\
& \hyperref[UC8.1.3.4.4]{UC8.1.3.4.4}\\
& \hyperref[UC8.1.3.4.4.1]{UC8.1.3.4.4.1}\\ \hline

\hyperlink{RFD7.1.3.4.4.2}{RFD7.1.3.4.4.2} & \hyperlink{Interno}{Interno}\\
& \hyperref[UC8.1.3.4.4]{UC8.1.3.4.4}\\
& \hyperref[UC8.1.3.4.4.2]{UC8.1.3.4.4.2}\\ \hline

\hyperlink{RFD7.1.3.4.4.3}{RFD7.1.3.4.4.3} & \hyperlink{Interno}{Interno}\\
& \hyperref[UC8.1.3.4.4]{UC8.1.3.4.4}\\
& \hyperref[UC8.1.3.4.4.3]{UC8.1.3.4.4.3}\\ \hline

\hyperlink{RFD7.1.3.4.4.4}{RFD7.1.3.4.4.4} & \hyperlink{Interno}{Interno}\\
& \hyperref[UC8.1.3.4.4]{UC8.1.3.4.4}\\
& \hyperref[UC8.1.3.4.4.4]{UC8.1.3.4.4.4}\\ \hline

\hyperlink{RFD7.1.3.5}{RFD7.1.3.5} & \hyperlink{Capitolato}{Capitolato}\\
& \hyperref[UC8.1.3]{UC8.1.3}\\
& \hyperref[UC8.1.3.5]{UC8.1.3.5}\\ \hline

\hyperlink{RFD7.1.3.5.1}{RFD7.1.3.5.1} & \hyperlink{Capitolato}{Capitolato}\\
& \hyperref[UC8.1.3.5]{UC8.1.3.5}\\
& \hyperref[UC8.1.3.5.1]{UC8.1.3.5.1}\\ \hline

\hyperlink{RFD7.1.3.5.2}{RFD7.1.3.5.2} & \hyperlink{Capitolato}{Capitolato}\\
& \hyperref[UC8.1.3.5]{UC8.1.3.5}\\
& \hyperref[UC8.1.3.5.2]{UC8.1.3.5.2}\\ \hline

\hyperlink{RFD7.1.3.5.2.1}{RFD7.1.3.5.2.1} & \hyperlink{Interno}{Interno}\\
& \hyperref[UC8.1.3.5.2]{UC8.1.3.5.2}\\
& \hyperref[UC8.1.3.5.2.1]{UC8.1.3.5.2.1}\\ \hline

\hyperlink{RFD7.1.3.5.3}{RFD7.1.3.5.3} & \hyperlink{Capitolato}{Capitolato}\\
& \hyperref[UC8.1.3.5]{UC8.1.3.5}\\
& \hyperref[UC8.1.3.5.3]{UC8.1.3.5.3}\\ \hline

\hyperlink{RFD7.1.3.5.3.1}{RFD7.1.3.5.3.1} & \hyperlink{Interno}{Interno}\\
& \hyperref[UC8.1.3.5.3]{UC8.1.3.5.3}\\
& \hyperref[UC8.1.3.5.3.1]{UC8.1.3.5.3.1}\\ \hline

\hyperlink{RFD7.1.3.6}{RFD7.1.3.6} & \hyperlink{Capitolato}{Capitolato}\\
& \hyperref[UC8.1.3]{UC8.1.3}\\
& \hyperref[UC8.1.3.6]{UC8.1.3.6}\\ \hline

\hyperlink{RFD7.1.3.6.1}{RFD7.1.3.6.1} & \hyperlink{Capitolato}{Capitolato}\\
& \hyperref[UC8.1.3.6]{UC8.1.3.6}\\
& \hyperref[UC8.1.3.6.1]{UC8.1.3.6.1}\\ \hline

\hyperlink{RFD7.1.3.6.2}{RFD7.1.3.6.2} & \hyperlink{Capitolato}{Capitolato}\\
& \hyperref[UC8.1.3.6]{UC8.1.3.6}\\
& \hyperref[UC8.1.3.6.2]{UC8.1.3.6.2}\\ \hline

\hyperlink{RFD7.1.3.6.3}{RFD7.1.3.6.3} & \hyperlink{Capitolato}{Capitolato}\\
& \hyperref[UC8.1.3.6]{UC8.1.3.6}\\
& \hyperref[UC8.1.3.6.3]{UC8.1.3.6.3}\\ \hline

\hyperlink{RFD7.1.3.7}{RFD7.1.3.7} & \hyperlink{Capitolato}{Capitolato}\\
& \hyperref[UC8.1.3]{UC8.1.3}\\
& \hyperref[UC8.1.3.7]{UC8.1.3.7}\\ \hline

\hyperlink{RFD7.1.3.7.1}{RFD7.1.3.7.1} & \hyperlink{Capitolato}{Capitolato}\\
& \hyperref[UC8.1.3.7]{UC8.1.3.7}\\
& \hyperref[UC8.1.3.7.1]{UC8.1.3.7.1}\\ \hline

\hyperlink{RFD7.1.3.7.2}{RFD7.1.3.7.2} & \hyperlink{Capitolato}{Capitolato}\\
& \hyperref[UC8.1.3.7]{UC8.1.3.7}\\
& \hyperref[UC8.1.3.7.2]{UC8.1.3.7.2}\\ \hline

\hyperlink{RFD7.1.3.7.3}{RFD7.1.3.7.3} & \hyperlink{Capitolato}{Capitolato}\\
& \hyperref[UC8.1.3.7]{UC8.1.3.7}\\
& \hyperref[UC8.1.3.7.3]{UC8.1.3.7.3}\\ \hline

\hyperlink{RFD7.1.3.7.4}{RFD7.1.3.7.4} & \hyperlink{Capitolato}{Capitolato}\\
& \hyperref[UC8.1.3.7]{UC8.1.3.7}\\
& \hyperref[UC8.1.3.7.4]{UC8.1.3.7.4}\\ \hline

\hyperlink{RFO7.1.4}{RFO7.1.4} & \hyperlink{Capitolato}{Capitolato}\\
& \hyperref[UC8.1]{UC8.1}\\
& \hyperref[UC8.1.4]{UC8.1.4}\\ \hline

\hyperlink{RFO7.1.5}{RFO7.1.5} & \hyperlink{Capitolato}{Capitolato}\\
& \hyperref[UC8.1]{UC8.1}\\
& \hyperref[UC8.1.5]{UC8.1.5}\\ \hline

\hyperlink{RFD7.2}{RFD7.2} & \hyperlink{Interno}{Interno}\\
& \hyperref[UC8.2]{UC8.2}\\ \hline

\hyperlink{RFD7.2.1}{RFD7.2.1} & \hyperlink{Interno}{Interno}\\
& \hyperref[UC8.2]{UC8.2}\\
& \hyperref[UC8.2.1]{UC8.2.1}\\ \hline

\hyperlink{RFD7.2.1.1}{RFD7.2.1.1} & \hyperlink{Interno}{Interno}\\
& \hyperref[UC8.2.1]{UC8.2.1}\\
& \hyperref[UC8.2.1.1]{UC8.2.1.1}\\ \hline

\hyperlink{RFD7.2.1.1.1}{RFD7.2.1.1.1} & \hyperlink{Interno}{Interno}\\
& \hyperref[UC8.2.1.1]{UC8.2.1.1}\\
& \hyperref[UC8.2.1.1.1]{UC8.2.1.1.1}\\ \hline

\hyperlink{RFD7.2.1.1.2}{RFD7.2.1.1.2} & \hyperlink{Interno}{Interno}\\
& \hyperref[UC8.2.1.1]{UC8.2.1.1}\\
& \hyperref[UC8.2.1.1.2]{UC8.2.1.1.2}\\ \hline

\hyperlink{RFD7.2.1.1.3}{RFD7.2.1.1.3} & \hyperlink{Interno}{Interno}\\
& \hyperref[UC8.2.1.1]{UC8.2.1.1}\\
& \hyperref[UC8.2.1.1.3]{UC8.2.1.1.3}\\ \hline

\hyperlink{RFD7.2.1.2}{RFD7.2.1.2} & \hyperlink{Interno}{Interno}\\
& \hyperref[UC8.2.1.2]{UC8.2.1.2}\\ \hline

\hyperlink{RFD7.2.1.2.1}{RFD7.2.1.2.1} & \hyperlink{Interno}{Interno}\\
& \hyperref[UC8.2.1.2]{UC8.2.1.2}\\
& \hyperref[UC8.2.1.2.1]{UC8.2.1.2.1}\\ \hline

\hyperlink{RFD7.2.1.2.2}{RFD7.2.1.2.2} & \hyperlink{Interno}{Interno}\\
& \hyperref[UC8.2.1.2]{UC8.2.1.2}\\
& \hyperref[UC8.2.1.2.2]{UC8.2.1.2.2}\\ \hline

\hyperlink{RFD7.2.1.2.3}{RFD7.2.1.2.3} & \hyperlink{Interno}{Interno}\\
& \hyperref[UC8.2.1.2]{UC8.2.1.2}\\
& \hyperref[UC8.2.1.2.3]{UC8.2.1.2.3}\\ \hline

\hyperlink{RFD7.2.1.2.3.1}{RFD7.2.1.2.3.1} & \hyperlink{Interno}{Interno}\\
& \hyperref[UC8.2.1.2.3]{UC8.2.1.2.3}\\
& \hyperref[UC8.2.1.2.3.1]{UC8.2.1.2.3.1}\\ \hline

\hyperlink{RFD7.2.1.2.3.2}{RFD7.2.1.2.3.2} & \hyperlink{Interno}{Interno}\\
& \hyperref[UC8.2.1.2.3]{UC8.2.1.2.3}\\
& \hyperref[UC8.2.1.2.3.1]{UC8.2.1.2.3.1}\\ \hline

\hyperlink{RFD7.2.1.2.4}{RFD7.2.1.2.4} & \hyperlink{Interno}{Interno}\\
& \hyperref[UC8.2.1.2]{UC8.2.1.2}\\
& \hyperref[UC8.2.1.2.4]{UC8.2.1.2.4}\\ \hline

\hyperlink{RFD7.2.1.3}{RFD7.2.1.3} & \hyperlink{Interno}{Interno}\\
& \hyperref[UC8.2.1.3]{UC8.2.1.3}\\ \hline

\hyperlink{RFD7.2.1.3.1}{RFD7.2.1.3.1} & \hyperlink{Interno}{Interno}\\
& \hyperref[UC8.2.1.3]{UC8.2.1.3}\\
& \hyperref[UC8.2.1.3.1]{UC8.2.1.3.1}\\ \hline

\hyperlink{RFD7.2.1.3.2}{RFD7.2.1.3.2} & \hyperlink{Interno}{Interno}\\
& \hyperref[UC8.2.1.3]{UC8.2.1.3}\\
& \hyperref[UC8.2.1.3.2]{UC8.2.1.3.2}\\ \hline

\hyperlink{RFD7.2.1.4}{RFD7.2.1.4} & \hyperlink{Interno}{Interno}\\
& \hyperref[UC8.2.1]{UC8.2.1}\\
& \hyperref[UC8.2.1.4]{UC8.2.1.4}\\ \hline

\hyperlink{RFD7.2.1.4.1}{RFD7.2.1.4.1} & \hyperlink{Interno}{Interno}\\
& \hyperref[UC8.2.1.4]{UC8.2.1.4}\\
& \hyperref[UC8.2.1.4.1]{UC8.2.1.4.1}\\ \hline

\hyperlink{RFD7.2.1.4.1.1}{RFD7.2.1.4.1.1} & \hyperlink{Interno}{Interno}\\
& \hyperref[UC8.2.1.4.1]{UC8.2.1.4.1}\\
& \hyperref[UC8.2.1.4.1.1]{UC8.2.1.4.1.1}\\ \hline

\hyperlink{RFD7.2.1.4.1.2}{RFD7.2.1.4.1.2} & \hyperlink{Interno}{Interno}\\
& \hyperref[UC8.2.1.4.1]{UC8.2.1.4.1}\\
& \hyperref[UC8.2.1.4.1.2]{UC8.2.1.4.1.2}\\ \hline

\hyperlink{RFD7.2.1.4.1.3}{RFD7.2.1.4.1.3} & \hyperlink{Interno}{Interno}\\
& \hyperref[UC8.2.1.4.1]{UC8.2.1.4.1}\\
& \hyperref[UC8.2.1.4.1.3]{UC8.2.1.4.1.3}\\ \hline

\hyperlink{RFD7.2.1.4.1.4}{RFD7.2.1.4.1.4} & \hyperlink{Interno}{Interno}\\
& \hyperref[UC8.2.1.4.1]{UC8.2.1.4.1}\\
& \hyperref[UC8.2.1.4.1.4]{UC8.2.1.4.1.4}\\ \hline

\hyperlink{RFD7.2.1.4.2}{RFD7.2.1.4.2} & \hyperlink{Interno}{Interno}\\
& \hyperref[UC8.2.1.4]{UC8.2.1.4}\\
& \hyperref[UC8.2.1.4.2]{UC8.2.1.4.2}\\ \hline

\hyperlink{RFD7.2.1.4.2.1}{RFD7.2.1.4.2.1} & \hyperlink{Interno}{Interno}\\
& \hyperref[UC8.2.1.4.2]{UC8.2.1.4.2}\\
& \hyperref[UC8.2.1.4.2.1]{UC8.2.1.4.2.1}\\ \hline

\hyperlink{RFD7.2.1.4.3}{RFD7.2.1.4.3} & \hyperlink{Interno}{Interno}\\
& \hyperref[UC8.2.1.4]{UC8.2.1.4}\\
& \hyperref[UC8.2.1.4.2]{UC8.2.1.4.2}\\ \hline

\hyperlink{RFD7.2.1.4.3.1}{RFD7.2.1.4.3.1} & \hyperlink{Interno}{Interno}\\
& \hyperref[UC8.2.1.4.3]{UC8.2.1.4.3}\\
& \hyperref[UC8.2.1.4.3.1]{UC8.2.1.4.3.1}\\ \hline

\hyperlink{RFD7.2.1.4.3.2}{RFD7.2.1.4.3.2} & \hyperlink{Interno}{Interno}\\
& \hyperref[UC8.2.1.4.3]{UC8.2.1.4.3}\\
& \hyperref[UC8.2.1.4.3.2]{UC8.2.1.4.3.2}\\ \hline

\hyperlink{RFD7.2.1.4.3.3}{RFD7.2.1.4.3.3} & \hyperlink{Interno}{Interno}\\
& \hyperref[UC8.2.1.4.3]{UC8.2.1.4.3}\\
& \hyperref[UC8.2.1.4.3.3]{UC8.2.1.4.3.3}\\ \hline

\hyperlink{RFD7.2.1.4.3.4}{RFD7.2.1.4.3.4} & \hyperlink{Interno}{Interno}\\
& \hyperref[UC8.2.1.4.3]{UC8.2.1.4.3}\\
& \hyperref[UC8.2.1.4.3.4]{UC8.2.1.4.3.4}\\ \hline

\hyperlink{RFD7.2.1.4.4}{RFD7.2.1.4.4} & \hyperlink{Interno}{Interno}\\
& \hyperref[UC8.2.1.4]{UC8.2.1.4}\\
& \hyperref[UC8.2.1.4.2]{UC8.2.1.4.2}\\ \hline

\hyperlink{RFD7.2.1.4.5}{RFD7.2.1.4.5} & \hyperlink{Interno}{Interno}\\
& \hyperref[UC8.2.1.4]{UC8.2.1.4}\\
& \hyperref[UC8.2.1.4.3]{UC8.2.1.4.3}\\ \hline

\hyperlink{RFD7.2.1.4.6}{RFD7.2.1.4.6} & \hyperlink{Interno}{Interno}\\
& \hyperref[UC8.2.1.4]{UC8.2.1.4}\\
& \hyperref[UC8.2.1.4.4]{UC8.2.1.4.4}\\ \hline

\hyperlink{RFD7.2.1.5}{RFD7.2.1.5} & \hyperlink{Interno}{Interno}\\
& \hyperref[UC8.2.1]{UC8.2.1}\\
& \hyperref[UC8.2.1.5]{UC8.2.1.5}\\ \hline

\hyperlink{RFD7.2.1.5.1}{RFD7.2.1.5.1} & \hyperlink{Interno}{Interno}\\
& \hyperref[UC8.2.1.5]{UC8.2.1.5}\\
& \hyperref[UC8.2.1.5.1]{UC8.2.1.5.1}\\ \hline

\hyperlink{RFD7.2.1.5.2}{RFD7.2.1.5.2} & \hyperlink{Interno}{Interno}\\
& \hyperref[UC8.2.1.5]{UC8.2.1.5}\\
& \hyperref[UC8.2.1.5.2]{UC8.2.1.5.2}\\ \hline

\hyperlink{RFD7.2.1.5.3}{RFD7.2.1.5.3} & \hyperlink{Interno}{Interno}\\
& \hyperref[UC8.2.1.5]{UC8.2.1.5}\\
& \hyperref[UC8.2.1.5.3]{UC8.2.1.5.3}\\ \hline

\hyperlink{RFD7.2.1.5.4}{RFD7.2.1.5.4} & \hyperlink{Interno}{Interno}\\
& \hyperref[UC8.2.1.5]{UC8.2.1.5}\\
& \hyperref[UC8.2.1.5.4]{UC8.2.1.5.4}\\ \hline

\hyperlink{RFD7.2.1.6}{RFD7.2.1.6} & \hyperlink{Interno}{Interno}\\
& \hyperref[UC8.2.1]{UC8.2.1}\\
& \hyperref[UC8.2.1.6]{UC8.2.1.6}\\ \hline

\hyperlink{RFD7.2.1.6.1}{RFD7.2.1.6.1} & \hyperlink{Interno}{Interno}\\
& \hyperref[UC8.2.1.6]{UC8.2.1.6}\\
& \hyperref[UC8.2.1.6.1]{UC8.2.1.6.1}\\ \hline

\hyperlink{RFD7.2.1.6.2}{RFD7.2.1.6.2} & \hyperlink{Interno}{Interno}\\
& \hyperref[UC8.2.1.6]{UC8.2.1.6}\\
& \hyperref[UC8.2.1.6.2]{UC8.2.1.6.2}\\ \hline

\hyperlink{RFD7.2.1.6.3}{RFD7.2.1.6.3} & \hyperlink{Interno}{Interno}\\
& \hyperref[UC8.2.1.6]{UC8.2.1.6}\\
& \hyperref[UC8.2.1.6.3]{UC8.2.1.6.3}\\ \hline

\hyperlink{RFD7.2.1.7}{RFD7.2.1.7} & \hyperlink{Interno}{Interno}\\
& \hyperref[UC8.2.1]{UC8.2.1}\\
& \hyperref[UC8.2.1.7]{UC8.2.1.7}\\ \hline

\hyperlink{RFD7.2.1.7.1}{RFD7.2.1.7.1} & \hyperlink{Interno}{Interno}\\
& \hyperref[UC8.2.1.7]{UC8.2.1.7}\\
& \hyperref[UC8.2.1.7.1]{UC8.2.1.7.1}\\ \hline

\hyperlink{RFD7.2.1.7.2}{RFD7.2.1.7.2} & \hyperlink{Interno}{Interno}\\
& \hyperref[UC8.2.1.7]{UC8.2.1.7}\\
& \hyperref[UC8.2.1.7.2]{UC8.2.1.7.2}\\ \hline

\hyperlink{RFD7.2.1.7.3}{RFD7.2.1.7.3} & \hyperlink{Interno}{Interno}\\
& \hyperref[UC8.2.1.7]{UC8.2.1.7}\\
& \hyperref[UC8.2.1.7.3]{UC8.2.1.7.3}\\ \hline

\hyperlink{RFD7.2.1.7.4}{RFD7.2.1.7.4} & \hyperlink{Interno}{Interno}\\
& \hyperref[UC8.2.1.7]{UC8.2.1.7}\\
& \hyperref[UC8.2.1.7.4]{UC8.2.1.7.4}\\ \hline

\hyperlink{RFD7.2.1.8}{RFD7.2.1.8} & \hyperlink{Interno}{Interno}\\
& \hyperref[UC8.2.1]{UC8.2.1}\\
& \hyperref[UC8.2.1.7]{UC8.2.1.7}\\ \hline

\hyperlink{RFD7.2.2}{RFD7.2.2} & \hyperlink{Interno}{Interno}\\
& \hyperref[UC8.2]{UC8.2}\\
& \hyperref[UC8.2.2]{UC8.2.2}\\ \hline

\hyperlink{RFD7.2.3}{RFD7.2.3} & \hyperlink{Interno}{Interno}\\
& \hyperref[UC8.2]{UC8.2}\\
& \hyperref[UC8.2.3]{UC8.2.3}\\ \hline

\hyperlink{RFO8}{RFO8} & \hyperlink{Verbale 2016-01-11}{Verbale 2016-01-11}\\
& \hyperref[UC9]{UC9}\\ \hline

\hyperlink{RFD8.1}{RFD8.1} & \hyperlink{Interno}{Interno}\\
& \hyperref[UC9]{UC9}\\
& \hyperref[UC9.1]{UC9.1}\\ \hline

\hyperlink{RFD8.1.1}{RFD8.1.1} & \hyperlink{Interno}{Interno}\\
& \hyperref[UC9.1]{UC9.1}\\
& \hyperref[UC9.1.1]{UC9.1.1}\\ \hline

\hyperlink{RFD8.1.1.1}{RFD8.1.1.1} & \hyperlink{Interno}{Interno}\\
& \hyperref[UC9.1.1]{UC9.1.1}\\
& \hyperref[UC9.1.1.1]{UC9.1.1.1}\\ \hline

\hyperlink{RFD8.1.1.2}{RFD8.1.1.2} & \hyperlink{Interno}{Interno}\\
& \hyperref[UC9.1.1]{UC9.1.1}\\
& \hyperref[UC9.1.1.2]{UC9.1.1.2}\\ \hline

\hyperlink{RFD8.1.2}{RFD8.1.2} & \hyperlink{Interno}{Interno}\\
& \hyperref[UC9.1]{UC9.1}\\
& \hyperref[UC9.1.2]{UC9.1.2}\\ \hline

\hyperlink{RFD8.1.2.1}{RFD8.1.2.1} & \hyperlink{Interno}{Interno}\\
& \hyperref[UC9.1.2]{UC9.1.2}\\
& \hyperref[UC9.1.2.1]{UC9.1.2.1}\\ \hline

\hyperlink{RFD8.1.3}{RFD8.1.3} & \hyperlink{Verbale 2016-01-11}{Verbale 2016-01-11}\\
& \hyperref[UC9.1]{UC9.1}\\
& \hyperref[UC9.1.3]{UC9.1.3}\\ \hline

\hyperlink{RFD8.1.4}{RFD8.1.4} & \hyperlink{Interno}{Interno}\\
& \hyperref[UC9.1]{UC9.1}\\
& \hyperref[UC9.1.4]{UC9.1.4}\\ \hline

\hyperlink{RFO8.2}{RFO8.2} & \hyperlink{Capitolato}{Capitolato}\\
& \hyperref[UC9.2]{UC9.2}\\ \hline

\hyperlink{RFO8.2.1}{RFO8.2.1} & \hyperlink{Capitolato}{Capitolato}\\
& \hyperref[UC9.2]{UC9.2}\\
& \hyperref[UC9.2.1]{UC9.2.1}\\ \hline

\hyperlink{RFD8.2.2}{RFD8.2.2} & \hyperlink{Interno}{Interno}\\
& \hyperref[UC9.2]{UC9.2}\\
& \hyperref[UC9.2.2]{UC9.2.2}\\ \hline

\hyperlink{RFD8.2.3}{RFD8.2.3} & \hyperlink{Interno}{Interno}\\
& \hyperref[UC9.2]{UC9.2}\\
& \hyperref[UC9.2.3]{UC9.2.3}\\ \hline

\hyperlink{RFO8.2.4}{RFO8.2.4} & \hyperlink{Interno}{Interno}\\
& \hyperref[UC9.2]{UC9.2}\\
& \hyperref[UC9.2.4]{UC9.2.4}\\ \hline

\hyperlink{RFD8.2.4.1}{RFD8.2.4.1} & \hyperlink{Interno}{Interno}\\
& \hyperref[UC9.2.4]{UC9.2.4}\\
& \hyperref[UC9.2.4.1]{UC9.2.4.1}\\ \hline

\hyperlink{RFO8.2.4.2}{RFO8.2.4.2} & \hyperlink{Interno}{Interno}\\
& \hyperref[UC9.2.4]{UC9.2.4}\\
& \hyperref[UC9.2.4.2]{UC9.2.4.2}\\ \hline

\hyperlink{RFO8.3}{RFO8.3} & \hyperlink{Interno}{Interno}\\
& \hyperref[UC9.1.1]{UC9.1.1}\\
& \hyperref[UC9.2]{UC9.2}\\
& \hyperref[UC9.3]{UC9.3}\\ \hline

\hyperlink{RFO8.3.1}{RFO8.3.1} & \hyperlink{Interno}{Interno}\\
& \hyperref[UC9.1.1]{UC9.1.1}\\
& \hyperref[UC9.2]{UC9.2}\\
& \hyperref[UC9.3]{UC9.3}\\
& \hyperref[UC9.3.1]{UC9.3.1}\\ \hline

\hyperlink{RFO8.3.1.1}{RFO8.3.1.1} & \hyperlink{Interno}{Interno}\\
& \hyperref[UC9.1.1]{UC9.1.1}\\
& \hyperref[UC9.2]{UC9.2}\\
& \hyperref[UC9.3.1]{UC9.3.1}\\
& \hyperref[UC9.3.1.1]{UC9.3.1.1}\\ \hline

\hyperlink{RFO8.3.1.1.1}{RFO8.3.1.1.1} & \hyperlink{Interno}{Interno}\\
& \hyperref[UC9.1.1]{UC9.1.1}\\
& \hyperref[UC9.2]{UC9.2}\\
& \hyperref[UC9.3.1.1]{UC9.3.1.1}\\
& \hyperref[UC9.3.1.1.1]{UC9.3.1.1.1}\\ \hline

\hyperlink{RFD8.3.1.1.2}{RFD8.3.1.1.2} & \hyperlink{Interno}{Interno}\\
& \hyperref[UC9.1.1]{UC9.1.1}\\
& \hyperref[UC9.2]{UC9.2}\\
& \hyperref[UC9.3.1.1]{UC9.3.1.1}\\
& \hyperref[UC9.3.1.1.2]{UC9.3.1.1.2}\\ \hline

\hyperlink{RFO8.3.2}{RFO8.3.2} & \hyperlink{Interno}{Interno}\\
& \hyperref[UC9.1.1]{UC9.1.1}\\
& \hyperref[UC9.2]{UC9.2}\\
& \hyperref[UC9.3]{UC9.3}\\
& \hyperref[UC9.3.2]{UC9.3.2}\\ \hline

\hyperlink{RFO8.3.2.1}{RFO8.3.2.1} & \hyperlink{Interno}{Interno}\\
& \hyperref[UC9.1.1]{UC9.1.1}\\
& \hyperref[UC9.2]{UC9.2}\\
& \hyperref[UC9.3.2]{UC9.3.2}\\
& \hyperref[UC9.3.2.1]{UC9.3.2.1}\\ \hline

\hyperlink{RFD8.4}{RFD8.4} & \hyperlink{Interno}{Interno}\\
& \hyperref[UC9]{UC9}\\
& \hyperref[UC9.4]{UC9.4}\\ \hline

\hyperlink{RFD8.4.1}{RFD8.4.1} & \hyperlink{Interno}{Interno}\\
& \hyperref[UC9.4]{UC9.4}\\
& \hyperref[UC9.4.1]{UC9.4.1}\\ \hline

\hyperlink{RFD8.4.1.1}{RFD8.4.1.1} & \hyperlink{Interno}{Interno}\\
& \hyperref[UC9.4.1]{UC9.4.1}\\
& \hyperref[UC9.4.1.1]{UC9.4.1.1}\\ \hline

\hyperlink{RFO9}{RFO9} & \hyperlink{Verbale 2016-01-11}{Verbale 2016-01-11}\\
& \hyperref[UC10]{UC10}\\ \hline

\hyperlink{RFD9.1}{RFD9.1} & \hyperlink{Interno}{Interno}\\
& \hyperref[UC10]{UC10}\\
& \hyperref[UC10.1]{UC10.1}\\ \hline

\hyperlink{RFD9.2}{RFD9.2} & \hyperlink{Interno}{Interno}\\
& \hyperref[UC10]{UC10}\\
& \hyperref[UC10.2]{UC10.2}\\ \hline

\hyperlink{RFD9.3}{RFD9.3} & \hyperlink{Interno}{Interno}\\
& \hyperref[UC10]{UC10}\\
& \hyperref[UC10.3]{UC10.3}\\ \hline

\hyperlink{RFD9.4}{RFD9.4} & \hyperlink{Interno}{Interno}\\
& \hyperref[UC10]{UC10}\\
& \hyperref[UC10.4]{UC10.4}\\ \hline

\hyperlink{RFD9.4.1}{RFD9.4.1} & \hyperlink{Interno}{Interno}\\
& \hyperref[UC10.4]{UC10.4}\\
& \hyperref[UC10.4.1]{UC10.4.1}\\ \hline

\hyperlink{RFD9.4.2}{RFD9.4.2} & \hyperlink{Interno}{Interno}\\
& \hyperref[UC10.4]{UC10.4}\\
& \hyperref[UC10.4.2]{UC10.4.2}\\ \hline

\hyperlink{RFD9.4.3}{RFD9.4.3} & \hyperlink{Interno}{Interno}\\
& \hyperref[UC10.4]{UC10.4}\\
& \hyperref[UC10.4.3]{UC10.4.3}\\
& \hyperref[UC10.4.3.1]{UC10.4.3.1}\\
& \hyperref[UC10.4.3.2]{UC10.4.3.2}\\ \hline

\hyperlink{RFD9.4.4}{RFD9.4.4} & \hyperlink{Interno}{Interno}\\
& \hyperref[UC10.4]{UC10.4}\\
& \hyperref[UC10.4.4]{UC10.4.4}\\
& \hyperref[UC10.4.4.1]{UC10.4.4.1}\\
& \hyperref[UC10.4.4.2]{UC10.4.4.2}\\ \hline

\hyperlink{RFD9.4.5}{RFD9.4.5} & \hyperlink{Interno}{Interno}\\
& \hyperref[UC10.4]{UC10.4}\\
& \hyperref[UC10.4.5]{UC10.4.5}\\ \hline

\hyperlink{RFD9.4.6}{RFD9.4.6} & \hyperlink{Interno}{Interno}\\
& \hyperref[UC10.4]{UC10.4}\\
& \hyperref[UC10.4.6]{UC10.4.6}\\ \hline

\hyperlink{RFD9.4.7}{RFD9.4.7} & \hyperlink{Interno}{Interno}\\
& \hyperref[UC10.4]{UC10.4}\\
& \hyperref[UC10.4.6]{UC10.4.6}\\ \hline

\hyperlink{RFD9.5}{RFD9.5} & \hyperlink{Interno}{Interno}\\ \hline

\hyperlink{RFD9.6}{RFD9.6} & \hyperlink{Interno}{Interno}\\ \hline

\hyperlink{RFD9.7}{RFD9.7} & \hyperlink{Interno}{Interno}\\ \hline

\hyperlink{RFD10}{RFD10} & \hyperlink{Interno}{Interno}\\
& \hyperref[UC11]{UC11}\\ \hline

\hyperlink{RFD10.1}{RFD10.1} & \hyperlink{Interno}{Interno}\\
& \hyperref[UC11]{UC11}\\
& \hyperref[UC11.1]{UC11.1}\\ \hline

\hyperlink{RFD10.2}{RFD10.2} & \hyperlink{Interno}{Interno}\\
& \hyperref[UC11]{UC11}\\
& \hyperref[UC11.2]{UC11.2}\\ \hline

\hyperlink{RFD10.3}{RFD10.3} & \hyperlink{Interno}{Interno}\\
& \hyperref[UC11]{UC11}\\
& \hyperref[UC11.3]{UC11.3}\\ \hline

\hyperlink{RFD10.4}{RFD10.4} & \hyperlink{Interno}{Interno}\\
& \hyperref[UC11]{UC11}\\
& \hyperref[UC12.4]{UC12.4}\\ \hline

\hyperlink{RFD10.4.1}{RFD10.4.1} & \hyperlink{Interno}{Interno}\\
& \hyperref[UC11.4.1]{UC11.4.1}\\
& \hyperref[UC12.4]{UC12.4}\\ \hline

\hyperlink{RFD10.5}{RFD10.5} & \hyperlink{Interno}{Interno}\\
& \hyperref[UC11]{UC11}\\
& \hyperref[UC11.5]{UC11.5}\\ \hline

\hyperlink{RFD10.5.1}{RFD10.5.1} & \hyperlink{Interno}{Interno}\\
& \hyperref[UC11.5]{UC11.5}\\
& \hyperref[UC11.5.1]{UC11.5.1}\\ \hline

\hyperlink{RFD10.6}{RFD10.6} & \hyperlink{Interno}{Interno}\\
& \hyperref[UC11.6]{UC11.6}\\ \hline

\hyperlink{RFO11}{RFO11} & \hyperlink{Capitolato}{Capitolato}\\
& \hyperref[UC12]{UC12}\\ \hline

\hyperlink{RFO11.1}{RFO11.1} & \hyperlink{Capitolato}{Capitolato}\\
& \hyperref[UC12]{UC12}\\
& \hyperref[UC12.1]{UC12.1}\\ \hline

\hyperlink{RFO11.2}{RFO11.2} & \hyperlink{Capitolato}{Capitolato}\\
& \hyperref[UC12]{UC12}\\
& \hyperref[UC12.2]{UC12.2}\\ \hline

\hyperlink{RFO11.3}{RFO11.3} & \hyperlink{Capitolato}{Capitolato}\\
& \hyperref[UC12]{UC12}\\
& \hyperref[UC12.3]{UC12.3}\\ \hline

\hyperlink{RFO11.3.1}{RFO11.3.1} & \hyperlink{Capitolato}{Capitolato}\\
& \hyperref[UC12.3]{UC12.3}\\
& \hyperref[UC12.3.1]{UC12.3.1}\\ \hline

\hyperlink{RFD11.4}{RFD11.4} & \hyperlink{Capitolato}{Capitolato}\\
& \hyperref[UC12]{UC12}\\
& \hyperref[UC12.4]{UC12.4}\\ \hline

\hyperlink{RFD11.4.1}{RFD11.4.1} & \hyperlink{Capitolato}{Capitolato}\\
& \hyperref[UC12.4]{UC12.4}\\
& \hyperref[UC12.4.1]{UC12.4.1}\\
& \hyperref[UC12.4.1.1]{UC12.4.1.1}\\
& \hyperref[UC12.4.1.2]{UC12.4.1.2}\\ \hline

\hyperlink{RFD11.5}{RFD11.5} & \hyperlink{Verbale 2016-01-11}{Verbale 2016-01-11}\\
& \hyperref[UC12]{UC12}\\
& \hyperref[UC12.5]{UC12.5}\\ \hline

\hyperlink{RFD11.5.1}{RFD11.5.1} & \hyperlink{Verbale 2016-01-11}{Verbale 2016-01-11}\\
& \hyperref[UC12.5]{UC12.5}\\
& \hyperref[UC12.5.1]{UC12.5.1}\\
& \hyperref[UC12.5.1.1]{UC12.5.1.1}\\
& \hyperref[UC12.5.1.2]{UC12.5.1.2}\\ \hline

\hyperlink{RFD11.6}{RFD11.6} & \hyperlink{Verbale 2016-01-11}{Verbale 2016-01-11}\\
& \hyperref[UC12]{UC12}\\
& \hyperref[UC12.6]{UC12.6}\\ \hline

\hyperlink{RFD11.6.1}{RFD11.6.1} & \hyperlink{Verbale 2016-01-11}{Verbale 2016-01-11}\\
& \hyperref[UC12.6]{UC12.6}\\
& \hyperref[UC12.6.1]{UC12.6.1}\\
& \hyperref[UC12.6.1.1]{UC12.6.1.1}\\
& \hyperref[UC12.6.1.2]{UC12.6.1.2}\\ \hline

\hyperlink{RFD11.7}{RFD11.7} & \hyperlink{Verbale 2016-01-11}{Verbale 2016-01-11}\\
& \hyperref[UC12]{UC12}\\
& \hyperref[UC12.7]{UC12.7}\\ \hline

\hyperlink{RFD11.7.1}{RFD11.7.1} & \hyperlink{Verbale 2016-01-11}{Verbale 2016-01-11}\\
& \hyperref[UC12.7]{UC12.7}\\
& \hyperref[UC12.7.1]{UC12.7.1}\\ \hline

\hyperlink{RFO12}{RFO12} & \hyperlink{Interno}{Interno}\\ \hline

\hyperlink{RFD12.1}{RFD12.1} & \hyperlink{Interno}{Interno}\\
& \hyperref[UC13]{UC13}\\ \hline

\hyperlink{RFD12.1.1}{RFD12.1.1} & \hyperlink{Interno}{Interno}\\
& \hyperref[UC13]{UC13}\\
& \hyperref[UC13.2]{UC13.2}\\ \hline

\hyperlink{RFO13}{RFO13} & \hyperlink{Capitolato}{Capitolato}\\ \hline

\hyperlink{RFO14}{RFO14} & \hyperlink{Capitolato}{Capitolato}\\ \hline

\hyperlink{RFO15}{RFO15} & \hyperlink{Capitolato}{Capitolato}\\ \hline

\hyperlink{RFO16}{RFO16} & \hyperlink{Capitolato}{Capitolato}\\ \hline

\hyperlink{RFO16.1}{RFO16.1} & \hyperlink{Capitolato}{Capitolato}\\ \hline

\hyperlink{RFO16.2}{RFO16.2} & \hyperlink{Capitolato}{Capitolato}\\ \hline

\hyperlink{RFO16.3}{RFO16.3} & \hyperlink{Capitolato}{Capitolato}\\ \hline

\hyperlink{RFO16.4}{RFO16.4} & \hyperlink{Capitolato}{Capitolato}\\ \hline

\hyperlink{RFD16.5}{RFD16.5} & \hyperlink{Capitolato}{Capitolato}\\ \hline

\hyperlink{RFD16.6}{RFD16.6} & \hyperlink{Capitolato}{Capitolato}\\ \hline

\hyperlink{RFO16.7}{RFO16.7} & \hyperlink{Capitolato}{Capitolato}\\ \hline

\hyperlink{RFO16.8}{RFO16.8} & \hyperlink{Capitolato}{Capitolato}\\ \hline

\hyperlink{RFO17}{RFO17} & \hyperlink{Capitolato}{Capitolato}\\ \hline

\hyperlink{RFO18}{RFO18} & \hyperlink{Capitolato}{Capitolato}\\ \hline

\hyperlink{RFO19}{RFO19} & \hyperlink{Capitolato}{Capitolato}\\ \hline

\hyperlink{RFO20}{RFO20} & \hyperlink{Capitolato}{Capitolato}\\ \hline

\hyperlink{RFD21}{RFD21} & \hyperlink{Capitolato}{Capitolato}\\ \hline

\hyperlink{RFD22}{RFD22} & \hyperlink{Capitolato}{Capitolato}\\ \hline

\hyperlink{RFF23}{RFF23} & \hyperlink{Capitolato}{Capitolato}\\ \hline

\hyperlink{RFD24}{RFD24} & \hyperlink{Capitolato}{Capitolato}\\
& \hyperref[UC10]{UC10}\\ \hline

\hyperlink{RFD25}{RFD25} & \hyperlink{Capitolato}{Capitolato}\\
& \hyperref[UC10]{UC10}\\ \hline

\hyperlink{RFD26}{RFD26} & \hyperlink{Capitolato}{Capitolato}\\
& \hyperref[UC10]{UC10}\\ \hline

\hyperlink{RFD27}{RFD27} & \hyperlink{Capitolato}{Capitolato}\\
& \hyperref[UC10]{UC10}\\ \hline

\hyperlink{RFD28}{RFD28} & \hyperlink{Capitolato}{Capitolato}\\ \hline

\hyperlink{RFF29}{RFF29} & \hyperlink{Capitolato}{Capitolato}\\ \hline

\hyperlink{RFF30}{RFF30} & \hyperlink{Capitolato}{Capitolato}\\ \hline

\hyperlink{RFF31}{RFF31} & \hyperlink{Capitolato}{Capitolato}\\ \hline

\hyperlink{RQO1}{RQO1} & \hyperlink{Capitolato}{Capitolato}\\ \hline

\hyperlink{RQO2}{RQO2} & \hyperlink{Interno}{Interno}\\ \hline

\hyperlink{RQO3}{RQO3} & \hyperlink{Interno}{Interno}\\ \hline

\hyperlink{RQO4}{RQO4} & \hyperlink{Interno}{Interno}\\ \hline

\hyperlink{RQO5}{RQO5} & \hyperlink{Interno}{Interno}\\ \hline

\hyperlink{RQO6}{RQO6} & \hyperlink{Capitolato}{Capitolato}\\ \hline

\hyperlink{RQO7}{RQO7} & \hyperlink{Interno}{Interno}\\ \hline

\hyperlink{RQF8}{RQF8} & \hyperlink{Interno}{Interno}\\ \hline

\hyperlink{RQO9}{RQO9} & \hyperlink{Interno}{Interno}\\ \hline

\hyperlink{RVO1}{RVO1} & \hyperlink{Capitolato}{Capitolato}\\ \hline

\hyperlink{RVO2}{RVO2} & \hyperlink{Capitolato}{Capitolato}\\ \hline

\hyperlink{RVO3}{RVO3} & \hyperlink{Capitolato}{Capitolato}\\ \hline

\hyperlink{RVD4}{RVD4} & \hyperlink{Capitolato}{Capitolato}\\ \hline

\hyperlink{RVO5}{RVO5} & \hyperlink{Interno}{Interno}\\ \hline

\hyperlink{RVO6}{RVO6} & \hyperlink{Interno}{Interno}\\ \hline

\hyperlink{RVO7}{RVO7} & \hyperlink{Interno}{Interno}\\ \hline

\hyperlink{RVO8}{RVO8} & \hyperlink{Interno}{Interno}\\ \hline

\hyperlink{RVD9}{RVD9} & \hyperlink{Interno}{Interno}\\ \hline

\hyperlink{RVO10}{RVO10} & \hyperlink{Interno}{Interno}\\ \hline

\hyperlink{RVD11}{RVD11} & \hyperlink{Interno}{Interno}\\ \hline

\hyperlink{RVO12}{RVO12} & \hyperlink{Interno}{Interno}\\ \hline

\hyperlink{RVD13}{RVD13} & \hyperlink{Interno}{Interno}\\ \hline

\hyperlink{RVO14}{RVO14} & \hyperlink{Interno}{Interno}\\ \hline

\hyperlink{RVF15}{RVF15} & \hyperlink{Interno}{Interno}\\ \hline

\hyperlink{RVF16}{RVF16} & \hyperlink{Interno}{Interno}\\ \hline

\hyperlink{RVF17}{RVF17} & \hyperlink{Interno}{Interno}\\ \hline

\hyperlink{RVF18}{RVF18} & \hyperlink{Interno}{Interno}\\ \hline

\hyperlink{RVD19}{RVD19} & \hyperlink{Interno}{Interno}\\ \hline

\hyperlink{RVD20}{RVD20} & \hyperlink{Interno}{Interno}\\ \hline

\hyperlink{RVF21}{RVF21} & \hyperlink{Interno}{Interno}\\ \hline

\hyperlink{RVF22}{RVF22} & \hyperlink{Interno}{Interno}\\ \hline

\hyperlink{RVF23}{RVF23} & \hyperlink{Interno}{Interno}\\ \hline

\hyperlink{RVF24}{RVF24} & \hyperlink{Interno}{Interno}\\ \hline

\hyperlink{RVF25}{RVF25} & \hyperlink{Interno}{Interno}\\ \hline

\hyperlink{RVF26}{RVF26} & \hyperlink{Interno}{Interno}\\ \hline

\caption[Tracciamento Requisiti-Fonti]{Tracciamento Requisiti-Fonti}
\label{tabella:requi-fonti}
\end{longtable}
\clearpage
