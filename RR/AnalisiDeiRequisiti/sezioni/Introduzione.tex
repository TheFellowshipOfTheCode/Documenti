\section{Introduzione}

\subsection{Scopo del documento}
Lo scopo del documento è definire i casi d'uso e i requisiti del prodotto emersi durante lo studio del capitolato C5 e dalle riunioni con \ZU.
\subsection{Scopo del prodotto}
Lo scopo del prodotto è di permettere la creazione e gestione di questionari in grado di identificare le lacune dei candidati prima, durante e al termine di un corso di formazione. 
\\Il sistema dovrà offrire le seguenti funzionalità:
\begin{itemize}
\item
Archiviare questionari in un server suddivisi per argomento;
\item
Somministrare all'utente, tramite un'interfaccia, questionari specifici per argomento scelto;
\item
Verificare e valutare i questionari scelti dagli utenti in base alle risposte date.
\end{itemize}
La parte destinata ai creatori di questionari dovrà essere fruibile attraverso un \textit{browser\ped{G}} desktop, abilitato all'utilizzo delle tecnologie \textit{HTML5\ped{G}}, \textit{CSS3\ped{G}} e \textit{JavaScript\ped{G}}. La parte destinata agli esaminandi sarà utilizzabile su qualunque dispositivo: dal personal computer ai tablet e smartphone.

\subsection{Glossario}
Al fine di evitare ogni ambiguità i termini tecnici del dominio del progetto, gli acronimi e le parole che necessitano di ulteriori spiegazioni saranno nei vari documenti marcate con il pedice \ped{G} e quindi presenti nel documento \textit{Glossario v1.0.0}.

\subsection{Riferimenti}
\subsubsection{Informativi}
\begin{itemize}
\item
\textbf{Capitolato d'appalto C5}: \progetto. Reperibile all'indirizzo: \\
\url{http://www.math.unipd.it/~tullio/IS-1/2015/Progetto/C5.pdf}.;
\item
\textbf{\SdF}: \textit{\SdF};
\item
\textbf{Ingegneria del Software - Ian Sommerville - Ottava edizione}:
	\begin{itemize}
		\item Capitolo 6: Requisiti del software;
		\item Capitolo 7: Processi di ingegneria dei requisiti.
	\end{itemize} 
\item
\textbf{Slide dell’insegnamento - Diagrammi dei casi d’uso:}  Reperibili all'indirizzo: \\ \url{http://www.math.unipd.it/~tullio/IS-1/2014/Dispense/E1b.pdf}.
\end{itemize}

\subsubsection{Normativi}
\begin{itemize}
\item
\textbf{\NdP}: \textit{\NdP}. 
\end{itemize}