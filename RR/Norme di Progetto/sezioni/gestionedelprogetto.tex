\section{Gestione del progetto}
Per gestire nella maniera più opportuna la divisione del lavoro, si è scelto di 
utilizzare il sistema di pianificazione delle attività \textbf{Zoho}.

\subsection{Task di pianificazione}
Vengono creati dal \textsl{\RdP} e sono assegnati a due singoli membri del gruppo, 
il primo in qualità di redattore e il secondo in qualità di verificatore.
Per una gestione più chiara di questa divisione delle attività, sono stati creati su \textbf{Zoho} 
due insieme di task per ogni documento, definite:
\begin{itemize}
  \item \textsl{Nome del documento};
  \item \textsl{Nome del documento - da verificare}.
\end{itemize}
Una volta che il proprietario ritiene il suo task completato, deve spostarlo 
nella relativa sezione \textsl{da verificare} in modo tale che il verificatore 
interessato possa controllarlo e segnarlo come completato, se risulta idoneo. 

\subsubsection{Creazione di un elenco di task}
Un elenco di task rappresenta l'insieme dei task necessari per la realizzazione di un intero documento.
Per la creazione di un nuovo insieme task bisogna seguire le seguenti istruzioni:
\begin{enumerate}
   \item Dalla HomePage di Zoho selezionare il progetto interessato (\progetto);
  \item Selezionare la voce \textsl{Compiti e Pietre miliari} dal menù laterale;
   \item Selezionare la voce \textbf{Nuovo elenco di compiti} e compilare l'elenco di task nel 
  seguente modo:
  \begin{itemize}
    \item \textbf{Elenco dei compiti:} assegnare il nome del documento che si 
    vuole rappresentare;
    \item \textbf{Pietra miliare collegata:} selezionare a quale pietra miliare si 
    vuole collegare la realizzazione di questo insieme di task. Corrispondono alla 
    versione finale del documento.
  \end{itemize}
\end{enumerate}
\subsubsection{Creazione di un task}
Per la creazione di un nuovo singolo task bisogna seguire le seguenti 
istruzioni:
\begin{enumerate}
  \item Dalla HomePage di Zoho selezionare il progetto interessato (\progetto);
  \item Selezionare la voce \textsl{Compiti e pietre miliari} dal menù laterale;
  \item Selezionare la voce \textbf{Nuovo compito} e compilare il task nel 
  seguente modo:
    \begin{itemize}
      \item \textbf{Nome task:} assegnare un nome \textsl{parlante} al task;
      \item \textbf{Aggiungi descrizione:} inserire una descrizione breve ma 
      concisa del task;
      \item \textbf{Elenco dei compiti:} selezionare uno degli elenchi di 
      task, creati precedentemente, che hanno come fine ultimo la realizzazione di un 
      documento;
      \item \textbf{Chi è il responsabile?:} inserire l'intestatario del task 
      come primo utente, nel secondo invece inserire il verificatore assegnato;
      \item \textbf{Data di conclusione:} Inserire la data ultima per il 
      completamento del task;
     \item \textbf{Priorità:} inserire, opzionalmente, una priorità al task.
    \end{itemize}
\end{enumerate}


\subsubsection{Modifica di un task}
Per modificare un task seguire le seguenti istruzioni:
\begin{itemize}
  \item Selezionare il task da modificare;
  \item Dalla pagina proposta selezionare il campo che si vuole modificare;
  \item Completata la modifica premere il pulsante \textbf{Invio}.
\end{itemize}

\subsubsection{Completamente di un task}
Dopo che il verificatore ha appurato che il task soddisfa i requisiti, e quindi è stato redatto secondo le
\textsl{\NdP}, può procedere con il suo completamento seguendo le seguenti istruzioni:
\begin{itemize}
  \item Selezionare il task da segnare come completato;
  \item Nella parte inferiore della pagina proposta selezionare la voce \textbf{Contrassegna come 
  completato}.
\end{itemize}
Dopo questa operazione, il task viene spostato automaticamente nella lista dei task completati. Da qui, in
casi eccezionali, può essere rispostato nei task \textsl{da verificare}. Questi casi eccezionali possono essere:
\begin{itemize}
  \item Il \textsl{\RdP} non ha approvato il documento e quindi deve essere 
  rivisto;
  \item Il \textsl{\Ver}, dopo un secondo controllo sui task a lui assegnati, 
  può accorgersi di errori e/o incompletezze. Deve quindi essere rivisto.
\end{itemize}