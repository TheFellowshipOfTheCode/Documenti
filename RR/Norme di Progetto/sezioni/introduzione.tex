\section{Introduzione}

\subsection{Scopo del documento}
Questo documento definisce le norme interne che i membri di TheFellowshipOfTheCode dovranno seguire nello svolgimento del capitolato d'appalto QUIZZIPEDIA. Ogni membro del gruppo ha l'obbligo di visionare il documento e seguire rigorosamente le norme in esso contenute, al fine di garantire uniformità e coesione nel materiale prodotto e assicurando efficacia ed efficienza al lavoro svolto.
\\In questo documento verranno definite norme riguardanti:
\begin{itemize}
\item
La definizione dei ruoli e l'identificazione delle relative mansioni;	
\item
Le modalità di lavoro durante le varie fasi del progetto;
\item
Le modalità di stesura dei documenti e le convenzioni utilizzate;
\item
La definizione degli ambienti di sviluppo;
\item
L'organizzazione della comunicazione e della cooperazione.
\end{itemize}

\subsection{Scopo del prodotto}
Lo scopo del prodotto è di permettere la creazione e gestione di questionari in grado di identificare le lacune dei candidati prima, durante e al termine di un corso di formazione. 
\\Il sistema dovrà offrire le seguenti funzionalità:
\begin{itemize}
\item
Archiviare questionari in un server suddivisi per argomento;
\item
Somministrare all'utente, tramite un'interfaccia, questionari specifici per argomento scelto;
\item
Verificare e valutare i questionari scelti dagli utenti in base alle risposte date.
\end{itemize}

Il prodotto sarà utilizzabile attraverso un browser.

\subsection{Glossario}
Al fine di evitare ogni ambiguità i termini tecnici del dominio del progetto, gli acronimi e le parole che necessitano di ulteriori spiegazioni saranno nei vari documenti marcate con il pedice \ped{G} e quindi presenti nel documento \textit{Glossario v1.0.0}.

\subsection{Riferimenti}
\subsubsection{Informativi}
\begin{itemize}
\item
\textbf{Piano di Progetto}: \textit{Piano di Progetto v1.0.0};
\item
\textbf{Piano di Qualifica}: \textit{Piano di Qualifica v1.0.0};
\item
\textbf{Amministrazione di progetto}: \textcolor{blue} {http://www.math.unipd.it/~tullio/IS-1/2014/Dispense/P06.pdf};
\item
\textbf{Specifiche \textit{UTF-8}}\ped{G}: \textcolor{blue}{http://www.unicode.org/versions/Unicode6.1.0/ch03.pdf}.
\end{itemize}

\subsubsection{Normativi}
\begin{itemize}
\item
\textbf{\textit{ISO}\ped{G} 31-0}: \textcolor{blue}{http://en.wikipedia.org/wiki/ISOwiki/ISO\_31};
\item
\textbf{\textit{ISO}}\ped{G} \textbf{8601}: \textcolor{blue}{http://it.wikipedia.org/wiki/ISO\_8601};
\item
\textbf{\textit{ISO}}\ped{G} \textbf{12207-1995}: \textcolor{blue} {http://www.math.unipd.it/~tullio/IS-1/2009/Approfondimenti/ISO\_12207-1995.pdf}.

\end{itemize}