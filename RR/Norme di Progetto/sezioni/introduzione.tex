\section{Introduzione}

\subsection{Scopo del documento}
Questo documento definisce le norme interne che I membri di TheFellowshipOfTheCode dovranno seguire nello svolgimento del capitolato d’appalto QUIZZIPEDIA. Ogni membro del gruppo ha l’obbligo di visionare il documento e seguire rigorosamente le norme in esso contenute, al fine di garantire uniformità e coesione nel materiale prodotto e assicurando efficacia ed efficienza al lavoro svolto.
\\In questo documento verranno definite norme riguardanti:
\begin{itemize}
\item
La definizione dei ruoli e l’identificazione delle relative mansioni;	
\item
Le modalità di lavoro durante le varie fasi del progetto.
\item
Le modalita' di stesura dei documenti e le convenzioni utilizzate;
\item
La definizione degli ambienti di sviluppo;
\item
L' organizzazione della comunicazione e della cooperazione;
\end{itemize}

\subsection{Scopo del prodotto}
Lo scopo del prodotto e' di permettere la creazione e gestione di questionari in grado di identificare le lacune dei candidati prima, durante e al termine di un corso di formazione. 
\\Il sistema dovrà offrire le seguenti funzionalità:
\begin{itemize}
\item
Archiviare questionari in un server suddivisi per argomento.
\item
Somministrare all'utente, tramite un interfaccia, questionari specifici per argomento scelto.
\item
Verificare e valutare i questionari scelti dagli utenti in base alle risposte date.
\end{itemize}

Il prodotto sarà utilizzabile attraverso un browser.

\subsection{Glossario}
Al fine di evitare ogni ambiguità i termini tecnici del dominio del progetto, gli acronimi e le parole che necessitano di ulteriori spiegazioni saranno nei vari documenti marcate con il pedice \ped{G} e quindi presenti nel documento Glossario v1.0.0.

\subsection{Informativi}
\subsubsection{Riferimenti}
\begin{itemize}
\item
Piano di Progetto: Piano di Progetto v1.2.0;
\item
Piano di Qualifica: Piano di Qualifica v1.2.0;
.....
\end{itemize}