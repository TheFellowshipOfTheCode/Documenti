\section{Processo di sviluppo}

\subsection{Scopo del processo}

\subsection{Analisi dei Requisiti}
\subsubsection{Studio di Fattibilità}
Alla pubblicazione dei capitolati d'appalto il \textsl{\RdP} dovrà fissare un 
numero di riunioni volte alla discussione e al confronto tra i membri del team. 
In seguito, gli \textsl{Analisti} dovranno redigere lo \textsl{\SdF} in base a quanto 
emerso nelle riunioni.\\
Lo \textsl{\SdF} sarà articolato nei seguenti punti:
\begin{itemize}
  \item \textbf{Descrizione:} Descrizione generale di ciò che viene richiesto 
  dal capitolato;
  \item \textbf{Dominio Applicativo:} Descrizione dell'ambito di utilizzo del 
  prodotto richiesto;
  \item \textbf{Dominio Tecnologico:} Descrizione delle tecnologie impiegate 
  nello sviluppo del progetto richiesto;
  \item \textbf{Criticità:} Elenco delle possibili problematiche che potrebbero 
  sorgere durante lo sviluppo del prodotto richiesto, individuando quindi punti 
  critici ed eventuali rischi;
  \item \textbf{Valutazione Finale:} Piccolo riassunto finale nel quale verranno 
  spiegate le motivazioni per cui è stato scelto o scartato il suddetto 
  capitolato.
\end{itemize}

\subsubsection{Analisi dei Requisiti}
Ultimato lo \textsl{\SdF} gli \textsl{Analisti} dovranno redigere l'\textsl{\AdR} 
che dovrà obbligatoriamente essere strutturati nel seguente modo:

\paragraph{Classificazione dei requisiti}
Dovrà essere redatto un elenco di requisiti, emersi durante le riunioni interne 
e/o esterne. Questo compito spetta agli \textsl{Analisti}. I requisiti dovranno 
essere classificati secondo la seguente codifica:
R[Tipo][Importante][Codice]

\begin{itemize} 
  \item \textbf{Tipo:} può assumere questi valori:
  \begin{itemize}
    \item \textbf{F:} indica un requisito funzionale;
    \item \textbf{Q:} indica un requisito di qualità;
    \item \textbf{P:} indica un requisito prestazionale;
    \item \textbf{V:} indica un requisito di vincolo.
  \end{itemize}
  \item \textbf{Importanza:} può assumere questi valori:
  \begin{itemize}
    \item \textbf{O:} indica un requisito obbligatorio;
    \item \textbf{D:} indica un requisito desiderabile;
    \item \textbf{F:} indica un requisito funzionale.
  \end{itemize}
  \item \textbf{Codice:} indica il codice identificativo del requisito, è 
  univoco e deve essere indicato in forma gerarchica.
\end{itemize}
Per ogni requisito si dovrà inoltre indicare: 
\begin{itemize}
  \item \textbf{Descrizione:} una breve descrizione, deve essere meno ambigua possibile;
  \item \textbf{Fonte:} la fonte può essere una delle seguenti:
  \begin{itemize}
    \item \textsl{Capitolato:} deriva direttamente dal testo del capitolato;
    \item \textsl{Verbale:} deriva da un incontro verbalizzato;
    \item \textsl{Interno:} deriva da discussioni interne al team;
    \item \textsl{Casi d'uso:} deriva da uno o più casi d'uso.
  \end{itemize}
\end{itemize} 
\paragraph{Classificazione dei casi d'uso}
Lo scopo del documento e quello di descrivere le motivazioni che hanno portato il gruppo alla scelta del capitolato C5.
Verranno inoltre riportate le descrizioni di tutti gli altri capitolati e le motivazioni che hanno spinto il gruppo a scartarli.

\subsubsection{Tracciamento}
Gli \textsl{Analisti} hanno il compito di inserire tutti i requisiti e i casi d'uso in \textsl{DocumentsDB},
software concesso dal gruppo \textsl{PragmaDB}, appositamente modificato da alcuni membri del 
team, che permette la gestione automatica della tracciabilità degli elementi inseriti. 
Inoltre i requisiti verranno inseriti in formato tabellare nel documento di \textsl{\AdR}.
\subsection{Progettazione}

\subsubsection{Descrizione}

\subsubsection{Diagrammi}

\subsubsection{Stili di progettazione}

\subsubsection{Tracciamento}

\subsection{Codifica}

\subsubsection{Descrizione}

\subsubsection{Intestazione}

\subsubsection{Formattazione}

\subsection{Strumenti}

\subsubsection{Tracciabilità}

\subsubsection{Sistema operativo}

\subsubsection{Strumenti per il versionamento}

\subsubsection{Documentazione sorgenti}

\subsubsection{Framework}
