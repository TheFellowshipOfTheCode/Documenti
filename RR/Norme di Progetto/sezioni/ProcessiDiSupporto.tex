\section{Processi di supporto}
\subsection{Processo di documentazione}
\subsubsection{Descrizione}
In questo capitolo verranno trattate tutte le convenzioni adottate dal gruppo 
\textit{TheFellowshipOfTheCode} riguardo la stesura, verifica e approvazione dei 
documenti.
I documenti sono classificati come:
\begin{itemize}
  \item \textit{Interni}: utilizzo interno al team;
  \item \textit{Esterni}: distribuzione esterna al committente e/o proponente.
\end{itemize}
Questa classificazione è dovuta dal tipo di distribuzione che andrà effettuata di tali documenti.
I documenti \textit{approvati} dal \textit{\RdP} devono avere un nome strutturato nel seguente modo:
\begin{itemize}
  \item La prima lettera del documento deve essere maiuscola;
  \item Il nome non deve contenere spazi;
  \item Va indicata la versione del documento nella parte finale del nome, in forma numerica, 
  corrispondente a quanto indicato nel diario delle modifiche. 
  \begin{center}
  \textit{NomeDelDocumento1.0.0}
  \end{center}
\end{itemize}
\subsubsection{Strumenti}
Per la stesura dell'intera documentazione è stato scelto di utilizzare \LaTeX. 
La scelta di questo linguaggio di markup è stata effettuata per avere uno 
standard comune e per evitare possibili conflitti e incompatibilità derivanti 
dall'utilizzo di software differente.
\subsubsection{Ciclo di vita di un documento}
 I documenti vengono possono trovare in uno dei seguenti stati:
\begin{itemize}
  \item Documenti \textbf{in lavorazione};
  \item Documenti \textbf{da verificare};
  \item Documenti \textbf{approvati}.
\end{itemize}
I documenti \textbf{in lavorazione} sono quelli in fase di stesura da parte del 
relativo redattore. Ultimata la loro realizzazione questi documenti vengono 
segnati come \textbf{da verificare} e passano quindi in mano al relativo 
\textit{\Ver}. Infine i documenti verificati vengono consegnati al \textit{\RdP} 
che avrà il compito di approvarli in via definitiva.
\subsubsection{Documenti finali}
\paragraph{\SdF}
L'intento del suddetto documento è riportare lo studio effettuato dall'intero 
team che ha portato all'accettazione dello sviluppo del progetto scelto. Questo 
documento è utilizzato internamente dal team e la lista di distribuzione comprende solo i 
committenti.
\paragraph{\NdP}
L'intento del suddetto documento è riportare tutte le convenzioni, strumenti e 
norme che il team dovrà adottare durante tutto lo sviluppo del progetto. Questo 
documento è utilizzato internamente al tram e la lista di distribuzione comprende solo i committenti. 
\paragraph{\PdP}
L'intento del suddetto documento è descrivere come il team gestisce le risorse, 
temporali e umane. Tratta inoltre la gestione dei rischi. Questo documento è 
utilizzato anche esternamente e la lista di distribuzione comprende committenti 
e proponenti.
\paragraph{\PdQ}
L'intento del suddetto documento è descrivere il modo in cui l'intero team punta 
a soddisfare gli obiettivi di qualità. Questo documento è 
utilizzato anche esternamente e la lista di distribuzione comprende committenti 
e proponenti.
\paragraph{\AdR}
L'intento di tale documento è dare una visione generale dei requisiti e dei casi 
d'uso del progetto. Il documento conterrà quindi la lista di tutti i casi d'uso, 
i diagrammi delle attività di interazione tra utente e sistema sviluppato e 
infine i servizi offerti dal prodotto finale. Questo documento è 
utilizzato anche esternamente e la lista di distribuzione comprende committenti 
e proponenti.
\paragraph{\ST}
L'intento del suddetto documento è dare una visione generale del prodotto e 
delle sue richieste. Più nello specifico verrà mostrata una progettazione ad 
alto livello, basata su diagrammi dei package per la descrizione delle 
componenti, diagrammi di sequenza per descrivere gli scenari d'uso e diagrammi 
di attività. Verranno inoltre elencati i software che saranno utilizzati per la realizzazione del progetto. 
\paragraph{\DDP}
L'intento di tale documento è fornire una progettazione dettagliata del 
prodotto. In questo documento verranno forniti tutti i dettagli implementativi 
del prodotto, fondamentali in fase di codifica, che comprenderanno diagrammi UML 
delle classi con i relativi metodi.
\paragraph{Glossario}
L'intento di tale documento è fornire una definizione dettagliata di tutti i 
termini tecnici e acronimi utilizzati nell'intera documentazione. Questo documento è 
utilizzato anche esternamente e la lista di distribuzione comprende committenti 
e proponenti.
\paragraph{\MU}
L'intento di tale documento è fornire all'utente una guida dettagliata di tutte 
le funzionalità offerte dal prodotto realizzato.
\paragraph{Verbali}
Questo documento ha lo scopo di riassumere in modo formale le discussioni effettuate e le decisioni 
prese durante le riunioni. I verbali, come le riunioni, sono classificati in: 
\textbf{interni} e \textbf{esterni}. In particolare i verbali esterni, essendo 
documenti ufficiali, devono essere redatti dai \textit{\RdP}. \\
Ogni verbale dovrà essere denominato nel seguente modo:
\begin{center}
  \textit{{Data del verbale}{Tipo di verbale}{Numero del verbale}}
\end{center}
dove:
\begin{itemize}
  \item \textbf{Data del verbale:} identifica la data nella quale si è svolta la 
  riunione corrispondente al verbale. Il formato è il seguente : 
  \begin{center}
  YYYYMMDD
  \end{center}
  \item \textbf{Tipo di verbale}: identifica se si tratta di un verbale \textbf{interno} oppure \textbf{esterno}.
  Per comodità utilizzeremo le abbreviazioni \textbf{i} per i verbali interni e \textbf{e} per i verbali 
  esterni;
  \item \textbf{Numero del verbale}: numero identificativo univoco del verbale.  
\end{itemize}
Nella parte introduttiva al verbale devono essere specificati i seguenti dati:
\begin{itemize}
  \item \textbf{Data incontro}: data in cui si è svolta la riunione;
  \item \textbf{Ora inizio incontro}: orario di inizio della riunione;
  \item \textbf{Ora termine incontro}: orario di terminazione della riunione;
  \item \textbf{Luogo incontro}: luogo in cui si è svolta la riunione;
  \item \textbf{Durata}: durata della riunione;
  \item \textbf{Oggetto}: Argomento della riunione;
  \item \textbf{Segretario}: Cognome e nome del membro incaricato a redigere il 
  verbale;
  \item \textbf{Partecipanti}: Cognome e nome di tutti i membri partecipanti 
  alla riunione.
\end{itemize}

\subsubsection{Struttura del documento}
\paragraph{Prima pagina}

Ogni documento deve avere nella prima pagina le seguenti informazione:
\begin{itemize}
  \item Nome del gruppo;
  \item Logo del progetto;
  \item Nome del progetto;
  \item Nome del documento;
  \item Versione del documento;
  \item Data di creazione del documento;
  \item Data di ultima modifica del documento;
  \item Stato del documento;
  \item Nome e cognome del redattore del documento;
  \item Nome e cognome del verificatore del documento;
  \item Nome e cognome del responsabile approvatore del documento;
  \item Uso del documento;
  \item Lista di distribuzione del documento;
  \item Destinatari del documento;
  \item Email di riferimento;
  \item Un sommario, contenente una breve descrizione del documento
\end{itemize} 
\paragraph{Diario delle modifiche}
La seconda pagina contiene il diario delle modifiche. In questa tabella vengono 
inserite tutte le modifiche effettuate dai vari redattori del documento. Ogni 
riga della tabella deve contenere le seguenti informazioni:
\begin{itemize}
  \item \textbf{Versione}: versione del documento dopo la modifica;
  \item \textbf{Descrizione}: descrizione della modifica apportata;
  \item \textbf{Autore e Ruolo}: autore della modifica e ruolo che esso ricopre;
  \item \textbf{Data}: data della modifica apportata.
\end{itemize}

\paragraph{Indice}
In ogni documento, dopo il diario delle modifiche, deve essere presente un 
indice di tutte le sezioni. In presenza di tabelle e/o immagini queste devono 
essere indicate con i relativi indici.

\paragraph{Formattazione generale delle pagine}
La formattazione della pagina, oltre al contenuto, prevede un'intestazione e un 
piè di pagina. \\
L'intestazione della pagina contiene:
\begin{itemize}
  \item Nome del progetto;
  \item Nome del documento;
  \item Logo del progetto.
\end{itemize}
Il piè di pagina contiene:
\begin{itemize}
  \item Il nome del gruppo;
  \item E-mail del gruppo;
  \item Numero della pagina corrente.
\end{itemize}

\subsubsection{Norme tipografiche}
Le seguenti norme tipografiche indicano i criteri riguardanti 
l'ortografia e la tipografia di tutti i documenti. 
\paragraph{Stili di testo}
\begin{itemize}
  \item \textbf{Grassetto}: Il grassetto deve essere utilizzato per evidenziare parole 
  particolarmente importanti, negli elenchi puntati o nelle frasi;
  \item \textbf{Corsivo}: Il corsivo deve essere utilizzato nelle seguenti 
  situazioni:
  \begin{itemize}
    \item Ruoli: ogni riferimento a ruoli di progetto va scritto in corsivo;
    \item Documenti: ogni riferimento a un documento va scritto in corsivo;
    \item Citazioni: ogni citazione va scritta in corsivo.
  \end{itemize}
\end{itemize}
\paragraph{Punteggiatura}
\begin{itemize}
  \item \textbf{Punteggiatura}: ogni simbolo di punteggiatura non può seguire un 
  carattere di spazio;
  \item \textbf{Lettere maiuscole}: le lettere maiuscole vanno utilizzate dopo il punto, il punto interrogativo, 
  il punto esclamativo e all’inizio di ogni elemento di un elenco puntato;
\end{itemize}
\paragraph{Composizione del testo}
\begin{itemize}
  \item \textbf{Elenchi puntati} ; ogni punto dell’elenco deve terminare con il punto e virgola,
   tranne l’ultimo che deve terminare con il punto. La prima parola deve avere la lettera 
   maiuscola.
   \item \textbf{Glossario}: il pedice\ped{G} verrà utilizzato in corrispondenza di vocaboli presenti nel Glossario.
\end{itemize}
\paragraph{Formati}
\begin{itemize}
   \item \textbf{Date}: le date presenti nei documenti devono seguire lo standard ISOG 8601:2004
   \begin{center}
     YYYY-MM-DD
   \end{center}
   dove:
   \begin{itemize}
     \item YYYY: rappresenta l'anno;
     \item MM: rappresenta il mese;
     \item DD: rappresenta il giorno.
   \end{itemize}
   \item \textbf{Ore}: le ore presenti nei documenti devono seguire lo standard ISOG 8601:2004 
   con il sistema a 24 ore
   \begin{center}
     hh:mm
   \end{center}
   dove:
   \begin{itemize}
     \item hh: rappresentano le ore;
     \item mm: rappresentano i minuti.
   \end{itemize}
   \item \textbf{Nome del documento}: per riferirsi al nome del documento si 
   dovrà utilizzare il comando \LaTeX   \verb|\documento{Nome del documento}| 
   garantendo in questo modo la corretta sintassi;
   \item \textbf{Nome del gruppo}: per riferirsi al nome del gruppo si dovrà 
   utilizzare il comando \LaTeX \verb|\gruppo|;
   \item \textbf{Nome del progetto}: per riferirsi al nome del gruppo si dovrà 
   utilizzare il comando \LaTeX \verb|\progetto|;
   \item \textbf{Link sito del gruppo}: per riferirsi al link del sito del gruppo si dovrà 
   utilizzare il comando \LaTeX \verb|\gruppoLink|;
   \item \textbf{Email del gruppo}: per riferirsi all'indirizzo email del gruppo si dovrà 
   utilizzare il comando \LaTeX \verb|\email|;
   \item \textbf{Nome del proponente}: per riferirsi al nome del proponente, ovvero \proponente, si dovrà 
   utilizzare il comando \LaTeX  \verb|\proponente|;
      \end{itemize}

\subsubsection{Composizione e-mail}
In questo paragrafo verranno descritte le norme da applicare durante la 
composizione delle email.
\paragraph{Destinatario}
\begin{itemize}
  \item Interno: l'indirizzo da utilizzare è \textit{\email};
  \item Esterno: l'indirizzo del destinatario varia a seconda si tratti del 
  Prof. Tullio Vardanega, Prof. Riccardo Cardin o i proponenti del progetto.
\end{itemize}
\paragraph{Mittente}
\begin{itemize}
  \item Interno: l'indirizzo è di colui che scrive e spedisce la email;
  \item Esterno: l'indirizzo da utilizzare è \textit{\email} ed è utilizzabile 
  unicamente dal \textit{\RdP}.
\end{itemize}
\paragraph{Oggetto}
L'oggetto della mail deve essere chiaro, preciso e conciso in modo da rendere 
semplice il riconoscimento di una mail tra le altre.
\paragraph{Corpo}
Il testo del corpo della mail deve essere chiaro ed esaustivo. All'interno del 
testo verrà utilizzata la sintassi ''@Ruolo'' per riferirsi ad uno specifico 
ruolo del team, mentre verrà utilizzata la sintassi ''@Destinatario'' per 
riferirsi ad una o più persona del team.
\paragraph{Allegati}
È permesso, anche se sconsigliato, l'invio di allegati tramite mail. È 
preferibile infatti condividere file all'interno del gruppo tramite strumenti più consoni, come Google 
Drive. 
\subsubsection{Componenti grafiche}
\paragraph{Tabelle}
Tutte le tabelle presenti all'interno del documento devono avere una didascalia 
ed un indice identificativo univoco per la loro tracciabilità all'interno del 
documento stesso.
\paragraph{Immagini}
Le immagini inserite nel documento sono nel formato PNG. In alternativa, 
giustificando il motivo, possono essere inserite in formato PDF.
\subsubsection{Versionamento}
Ogni documento prodotto deve essere identificato, oltre che dal nome, dal numero 
di versione nel seguente modo:
\begin{center}
  vX.Y.Z
\end{center}
dove:

\begin{itemize}
  \item \textbf{X}: indica il numero di uscite formali del documento e viene 
  incrementato in seguito all'approvazione finale da parte del \textit{\RdP}. 
  L'incremento dell'indice \textbf{X} comporta l'azzeramento degli indici 
  \textbf{Y} e \textbf{Z};
  \item \textbf{Y}: indica il numero crescente delle verifiche. L'incremento viene eseguito 
  dal \textit{\Ver} e comporta l'azzeramento dell'indice 
  \textbf{Z};
  \item \textbf{Z}: indica il numero di modifiche minori apportate al documento 
  prima della sua verifica. Viene aumentato in modo incrementale. 
  \end{itemize}
A ogni modifica del documento anche il nome del file fisico deve essere 
modificato, seguendo il seguente schema:
\begin{center}
  nomeDocumentoX.Y.Z.pdf
\end{center}

\subsection{Processo di verifica}
\subsubsection{Descrizione}
Il processo di verifica ha il compito di controllare che ogni documento prodotto 
rispecchi quanto previsto dai requisiti.
\subsubsection{Analisi}
\paragraph{Analisi statica}
L'analisi statica è una tecnica che permette di trovare eventuali anomalie nella 
documentazione o nel software prodotto. Ci sono due modi con la quale essa viene 
impiegata:
\begin{itemize}
  \item \textbf{Walkthrough}: questa tecnica di analisi statica consiste nella 
  lettura a largo spettro del documento o del codice, al fine di trovare anomalie, 
  senza avere un'idea precisa degli errori da cercare. Questa tecnica risulta 
  molto utile nella fase iniziale dello sviluppo del prodotto data la scarsa 
  esperienza dei membri del team. Questa tecnica è utilizzata dai \textit{Verificatori} 
  che avranno il compito di stilare una lista contenente gli errori rilevati più 
  spesso. Una volta raggiunta la quasi completezza di questa lista, essa dovrà 
  essere allegata a questo documento, in questo modo sarà possibile l'utilizzo 
  della tecnica di \textit{inspection};
  \item \textbf{Inspection}: questa tecnica di analisi statica consiste in una 
  lettura molto più dettagliata e più mirata dei documenti o del codice, 
  utilizzando come supporto fondamentale la lista di controllo contenente fli 
  errori più frequenti. Questa tecnica diventa sempre più efficace dato che la 
  lista verrà sempre ampliata con esperienza nella verifica.
\end{itemize}
\paragraph{Analisi dinamica}
L'analisi dinamica viene applicata solamente al software prodotto, in quanto 
essa consiste nell'esecuzione di test su di essi. Grazie a questi test è 
possibile verificare la correttezza del software.
\subsubsection{Test}
\paragraph{Test di unità}
I test di unità verificano che ogni singola componente del software funzioni 
correttamente. Effettuando questi test di riduce al minimo la presenza di errori 
di tutte le componenti di base. I test di unità sono identificati dalla seguente 
sintassi:
\begin{center}
  TU[Codice Test]
\end{center}
\paragraph{Test di integrazione}
I test di integrazione verificano che più unità, validate singolarmente, funzionino 
correttamente una volta assemblate. I test di integrazione sono identificati 
dalla seguente sintassi:
\begin{center}
  TI[Codice Test]
\end{center}
\paragraph{Test di sistema}
I test di sistema vengono eseguiti sul prodotto che si ritiene essere giunto ad 
una versione definitiva, serve a verificare che tutti i requisiti siano 
soddisfatti. I test di sistema sono identificati dalla seguente sintassi:
\begin{center}
  TS[Codice Requisito]
\end{center}
\paragraph{Test di regressione}
Il test di regressione consiste nel re-eseguire tutti i test relativi ad una 
componente del software prodotto che ha subito delle modifiche. I test di 
regressione sono identificati dalla seguente sintassi:
\begin{center}
  TR[Codice Test]
\end{center}
\paragraph{Test di validazione}
Il test di validazione coincide con il collaudo del software in presenza del 
\textit{proponente}. In caso di esito positivo si procede col rilascio del 
software. I test di validazione sono identificati dalla seguente sintassi:
\begin{center}
  TV[Codice requisito]
\end{center}

\subsection{Strumenti}
da decidere insieme23/12
\subsubsection{Strumenti per l'analisi statica}

\subsubsection{Strumenti per l'analisi dinamica}



