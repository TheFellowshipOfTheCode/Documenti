\section{Processi di supporto}
\subsection{Processo di documentazione}
\subsubsection{Descrizione}
In questo capitolo verranno trattate tutte le convenzioni adottate dal gruppo 
\textsl{TheFellowshipOfTheCode} riguardo la stesura, verifica e approvazione dei 
documenti.
I documenti sono classificati come:
\begin{itemize}
  \item \textsl{Interni}: utilizzo interno al team;
  \item \textsl{Esterni}: distribuzione esterna al committente e/o proponente.
\end{itemize}
Questa classificazione è dovuta dal tipo di distribuzione che andrà effettuata di tali documenti.
I documenti \textsl{approvati} dal \textsl{\RdP} devono avere un nome strutturato nel seguente modo:
\begin{itemize}
  \item La prima lettera del documento deve essere maiuscola;
  \item Il nome non deve contenere spazi;
  \item Va indicata la versione del documento nella parte finale del nome, in forma numerica, 
  corrispondente a quanto indicato nel diario delle modifiche. 
  \begin{center}
  \textsl{NomeDelDocumento1.0.0}
  \end{center}
\end{itemize}
\subsubsection{Strumenti}
Per la stesura dell'intera documentazione è stato scelto di utilizzare \LaTeX. 
La scelta di questo linguaggio di markup è stata effettuata per avere uno 
standard comune e per evitare possibili conflitti e incompatibilità derivanti 
dall'utilizzo di software differente.
\subsubsection{Ciclo di vita di un documento}
 I documenti vengono possono trovare in uno dei seguenti stati:
\begin{itemize}
  \item Documenti \textsl{in lavorazione};
  \item Documenti \textsl{da verificare};
  \item Documenti \textsl{approvati}.
\end{itemize}
I documenti \textsl{in lavorazione} sono quelli in fase di stesura da parte del 
relativo redattore. Ultimata la loro realizzazione questi documenti vengono 
segnati come \textsl{da verificare} e passano quindi in mano al relativo 
\textsl{\Ver}. Infine i documenti verificati vengono consegnati al \textsl{\RdP} 
che avrà il compito di approvarli in via definitiva.
\subsubsection{Documenti finali}
\paragraph{\SdF}
L'intento del suddetto documento è riportare lo studio effettuato dall'intero 
team che ha portato all'accettazione dello sviluppo del progetto scelto. Questo 
documento è utilizzato internamente dal team e la lista di distribuzione comprende solo i 
committenti.
\paragraph{\NdP}
L'intento del suddetto documento è riportare tutte le convenzioni, strumenti e 
norme che il team dovrà adottare durante tutto lo sviluppo del progetto. Questo 
documento è utilizzato internamente al tram e la lista di distribuzione comprende solo i committenti. 
\paragraph{\PdP}
L'intento del suddetto documento è descrivere come il team gestisce le risorse, 
temporali e umane. Tratta inoltre la gestione dei rischi. Questo documento è 
utilizzato anche esternamente e la lista di distribuzione comprende committenti 
e proponenti.
\paragraph{\PdQ}
L'intento del suddetto documento è descrivere il modo in cui l'intero team punta 
a soddisfare gli obiettivi di qualità. Questo documento è 
utilizzato anche esternamente e la lista di distribuzione comprende committenti 
e proponenti.
\paragraph{\AdR}
L'intento di tale documento è dare una visione generale dei requisiti e dei casi 
d'uso del progetto. Il documento conterrà quindi la lista di tutti i casi d'uso, 
i diagrammi delle attività di interazione tra utente e sistema sviluppato e 
infine i servizi offerti dal prodotto finale. Questo documento è 
utilizzato anche esternamente e la lista di distribuzione comprende committenti 
e proponenti.
\paragraph{\ST}
L'intento del suddetto documento è dare una visione generale del prodotto e 
delle sue richieste. Più nello specifico verrà mostrata una progettazione ad 
alto livello, basata su diagrammi dei package per la descrizione delle 
componenti, diagrammi di sequenza per descrivere gli scenari d'uso e diagrammi 
di attività. Verranno inoltre elencati i software che saranno utilizzati per la realizzazione del progetto. 
\paragraph{\DDP}
L'intento di tale documento è fornire una progettazione dettagliata del 
prodotto. In questo documento verranno forniti tutti i dettagli implementativi 
del prodotto, fondamentali in fase di codifica, che comprenderanno diagrammi UML 
delle classi con i relativi metodi.
\paragraph{Glossario}
L'intento di tale documento è fornire una definizione dettagliata di tutti i 
termini tecnici e acronimi utilizzati nell'intera documentazione. Questo documento è 
utilizzato anche esternamente e la lista di distribuzione comprende committenti 
e proponenti.
\paragraph{\MU}
L'intento di tale documento è fornire all'utente una guida dettagliata di tutte 
le funzionalità offerte dal prodotto realizzato.
\paragraph{Verbali}
Questo documento ha lo scopo di riassumere in modo formale le discussioni effettuate e le decisioni 
prese durante le riunioni. I verbali, come le riunioni, sono classificati in: 
\textsl{interni} e \textsl{esterni}. In particolare i verbali esterni, essendo 
documenti ufficiali, devono essere redatti dai \textsl{\RdP}. \\
Ogni verbale dovrà essere denominato nel seguente modo:
\begin{center}
  \textsl{{Data del verbale}{Tipo di verbale}{Numero del verbale}}
\end{center}
dove:
\begin{itemize}
  \item \textsl{Data del verbale:} identifica la data nella quale si è svolta la 
  riunione corrispondente al verbale. Il formato è il seguente : 
  \textsl{YYYYMMDD};
  \item \textsl{Tipo di verbale}: identifica se si tratta di un verbale \textsl{interno} oppure \textsl{esterno}.
  Per comodità utilizzeremo le abbreviazioni \textsl{i} per i verbali interni e \textsl{e} per i verbali 
  esterni;
  \item \textsl{Numero del verbale}: numero identificativo univoco del verbale.  
\end{itemize}
Nella parte introduttiva al verbale devono essere specificati i seguenti dati:
\begin{itemize}
  \item \textsl{Data incontro}: data in cui si è svolta la riunione;
  \item \textsl{Ora inizio incontro}: orario di inizio della riunione;
  \item \textsl{Ora termine incontro}: orario di terminazione della riunione;
  \item \textsl{Luogo incontro}: luogo in cui si è svolta la riunione;
  \item \textsl{Durata}: durata della riunione;
  \item \textsl{Oggetto}: Argomento della riunione;
  \item \textsl{Segretario}: Cognome e nome del membro incaricato a redigere il 
  verbale;
  \item \textsl{Partecipanti}: Cognome e nome di tutti i membri partecipanti 
  alla riunione.
\end{itemize}

\subsubsection{Struttura del documento}

\paragraph{Prima pagina}

\paragraph{Diario delle modifiche}

\paragraph{Formattazione generale delle pagine}

\subsubsection{Norme tipografiche}

\paragraph{Stili di testo}

\paragraph{Punteggiatura}

\paragraph{Composizione del testo}

\paragraph{Formati}

\paragraph{Sigle}

\subsubsection{Composizione e-mail}

\paragraph{Destinatario}

\paragraph{Mittente}

\paragraph{Oggetto}

\paragraph{Corpo}

\paragraph{Allegati}

\subsubsection{Componenti grafiche}

\paragraph{Tabelle}

\paragraph{Immagini}

\subsubsection{Versionamento}

\subsubsection{Glossario}

\subsection{Processo di verifica}

\subsubsection{Descrizione}

\subsection{Strumenti}

\subsubsection{Strumenti per l'analisi statica}

\subsubsection{Strumenti per l'analisi dinamica}



