\section{Processi organizzativi}

\subsection{Processo di coordinamento}
\subsubsection{Comunicazioni interne}
Lo scopo del documento e quello di descrivere le motivazioni che hanno portato il gruppo alla scelta del capitolato C5.
Verranno inoltre riportate le descrizioni di tutti gli altri capitolati e le motivazioni che hanno spinto il gruppo a scartarli.

\subsubsection{Comunicazione esterne}
Lo scopo del documento e quello di descrivere le motivazioni che hanno portato il gruppo alla scelta del capitolato C5.
Verranno inoltre riportate le descrizioni di tutti gli altri capitolati e le motivazioni che hanno spinto il gruppo a scartarli.

\subsubsection{Comunicazioni riunione}
Lo scopo del documento e quello di descrivere le motivazioni che hanno portato il gruppo alla scelta del capitolato C5.
Verranno inoltre riportate le descrizioni di tutti gli altri capitolati e le motivazioni che hanno spinto il gruppo a scartarli.

\subsubsection{Riunioni}

\paragraph{Svolgimento}

\paragraph{Riunioni interne}

\paragraph{Riunioni esterne}

\subsection{Processo di pianificazione}

\subsubsection{Descrizione}
Durante l'intero sviluppo del progetto didattico ogni componente del gruppo 
dovrà obbligatoriamente cimentarsi in tutti i ruoli elencati di seguito. \\
Inoltre non potrà mai accadere che un membro del gruppo risulti redattore e verificatore di un medesimo documento. 
In questo modo si tende ad evitare il conflitto di interessi che potrebbe sorgere se la responsabilità della stesura 
e della verifica di un documento fosse affidata ad un'unica persona.
Un membro può inoltre ricoprire più ruoli contemporaneamente.

\paragraph{\RdP}
Il \textsl{\RdP} è colui che detiene la responsabilità del 
lavoro svolto dall'intero team. Rappresenta inoltre colui che mantiene i 
contatti diretti presso il Fornitore e il Cliente, ovvero gli enti esterni. Più 
in dettaglio, ha responsabilità su:
\begin{itemize}
  \item Pianificazione, coordinamento e controllo generale delle attività;
  \item Gestione delle risorse;
  \item Analisi e gestione dei rischi;
  \item Gestione e approvazione della documentazione;
  \item Contatti con gli enti esterni.
\end{itemize}
Il \textsl{\RdP} redige l'organigramma, si assicura che 
tutte le attività vengano svolte seguendo rigorosamente le \textsl{\NdP}, si 
assicura che vengano rispettati i ruoli assegnati nel \textsl{\PdP} e che non si 
presentino conflitti di interesse tra redattori e verificatori. Ha inoltre 
l'incarico di creare, assegnare ad ogni membro e gestire i singoli task. Redige 
il \textsl{\PdP} e collabora alla stesura del \textsl{\PdQ}. Il \textsl{\RdP} è 
l'unica persona in grado di approvare in modo definitivo un documento.

\paragraph{Amministratore}
L'\textsl{\Amm} è il responsabile di tutto ciò che riguarda l'ambiente di 
lavoro. Più in dettaglio, egli si occupa di:
\begin{itemize}
  \item Controllo dell'ambiente di lavoro;
  \item Gestione del versionamento della documentazione tramite l'uso di 
  database;
  \item Controllo delle versioni e delle configurazioni del prodotto;
  \item Risoluzione dei problemi legati alla gestione dei processi
\end{itemize}
L'\textsl{\Amm} redige le \textsl{\NdP} e collabora alla stesura del 
\textsl{\PdP}.

\paragraph{Progettista}
Il \textsl{\Prog} è il responsabile di tutto ciò che riguarda la progettazione. 
Più in dettaglio, egli si occupa di:
\begin{itemize}
  \item Produrre una soluzione attuabile, robusta e semplice entro i limiti di 
  tempo stabiliti;
  \item Effettuare scelte progettuali volte a garantire la manutenibilità e la 
  modularità del prodotto software.
\end{itemize}
Il \textsl{\Prog} redige la \textsl{\ST}, la \textsl{\DDP} e la parte 
programmatica del \textsl{\PdQ}.

\paragraph{Analista}
L'analista si occupa di tutto ciò che riguarda l'analisi del problema da 
affrontare. Le mansioni principali sono quelle di:
\begin{itemize}
  \item Studiare a fondo e capire le problematiche del prodotto da realizzare;
  \item Produrre una specifica di progetto compresibile per il 
  \textsl{Proponente}, per il \textsl{Committente} e per il 
  \textsl{Progettista}.
\end{itemize}
L'\textsl{\Ana} redige lo \textsl{\SdF}, l'\textsl{\AdR} e parte del 
\textsl{\PdQ}.

\paragraph{Verificatore}
Il \textsl{Verificatore} è responsabile di tutto ciò che riguarda l'attività di verifica.
Effettua la verifica dei documenti utilizzando gli strumenti e i metodi proposti nel 
\textsl{\PdQ} e seguendo rigorosamente quanto descritto nelle \textsl{\NdP}.
Egli ha il compito di garantire la conformità rispetto le \textsl{\NdP} dei documenti da lui verificati. 
Si occupa di redigere la sezione del \textsl{\PdQ} che illustra l'esito delle 
verifiche effettuate sui documenti.

\paragraph{Programmatore}
Il \textsl{\Progr} si occuperà di implementare le soluzioni del \textsl{\Prog}, è quindi 
responsabile dell'attività di codifica. In dettaglio, i suoi compiti sono:
\begin{itemize}
  \item implementare le soluzioni descritte dal \textsl{\Prog} in maniera 
  rigorosa;
  \item Scrivere il codice rispettando le convenzioni prese nel presente 
  documento;
  \item Implementare i test per il codice scritto da utilizzare per l'attività 
  di verifica.
\end{itemize}
Il \textsl{\Progr} redige il \textsl{\MU} e produce la documentazione del codice.

\subsection{Strumenti}

\subsubsection{Google Drive}
Lo scopo del documento e quello di descrivere le motivazioni che hanno portato il gruppo alla scelta del capitolato C5.
Verranno inoltre riportate le descrizioni di tutti gli altri capitolati e le motivazioni che hanno spinto il gruppo a scartarli.

\subsubsection{GitHub}
Lo scopo del documento e quello di descrivere le motivazioni che hanno portato il gruppo alla scelta del capitolato C5.
Verranno inoltre riportate le descrizioni di tutti gli altri capitolati e le motivazioni che hanno spinto il gruppo a scartarli.

\subsubsection{Server database requisiti}
Lo scopo del documento e quello di descrivere le motivazioni che hanno portato il gruppo alla scelta del capitolato C5.
Verranno inoltre riportate le descrizioni di tutti gli altri capitolati e le motivazioni che hanno spinto il gruppo a scartarli.

\subsubsection{Gestione delle attività del progetto}
Per gestire nella maniera più opportuna la divisione del lavoro, si è scelto di 
utilizzare il sistema di pianificazione delle attività \textbf{Zoho}.

\subsubsection{Task management}
Vengono creati dal \textsl{\RdP} e sono assegnati a due singoli membri del gruppo, 
il primo in qualità di redattore e il secondo in qualità di verificatore.
Per una gestione più chiara di questa divisione delle attività, sono stati creati su \textbf{Zoho} 
due insieme di task per ogni documento, definite:
\begin{itemize}
  \item \textsl{Nome del documento};
  \item \textsl{Nome del documento - da verificare}.
\end{itemize}
Una volta che il proprietario ritiene il suo task completato, deve spostarlo 
nella relativa sezione \textsl{da verificare} in modo tale che il verificatore 
interessato possa controllarlo e segnarlo come completato, se risulta idoneo. 

\paragraph{Creazione di un elenco di task}
Un elenco di task rappresenta l'insieme dei task necessari per la realizzazione di un intero documento.
Per la creazione di un nuovo insieme task bisogna seguire le seguenti istruzioni:
\begin{enumerate}
   \item Dalla HomePage di Zoho selezionare il progetto interessato (\progetto);
  \item Selezionare la voce \textsl{Compiti e Pietre miliari} dal menù laterale;
   \item Selezionare la voce \textbf{Nuovo elenco di compiti} e compilare l'elenco di task nel 
  seguente modo:
  \begin{itemize}
    \item \textbf{Elenco dei compiti:} assegnare il nome del documento che si 
    vuole rappresentare;
    \item \textbf{Pietra miliare collegata:} selezionare a quale pietra miliare si 
    vuole collegare la realizzazione di questo insieme di task. Corrispondono alla 
    versione finale del documento.
  \end{itemize}
\end{enumerate}
\paragraph{Creazione di un task}
Per la creazione di un nuovo singolo task bisogna seguire le seguenti 
istruzioni:
\begin{enumerate}
  \item Dalla HomePage di Zoho selezionare il progetto interessato (\progetto);
  \item Selezionare la voce \textsl{Compiti e pietre miliari} dal menù laterale;
  \item Selezionare la voce \textbf{Nuovo compito} e compilare il task nel 
  seguente modo:
    \begin{itemize}
      \item \textbf{Nome task:} assegnare un nome identificativo al task seguito dalla milestone corrispondente;
      \item \textbf{Aggiungi descrizione:} inserire una descrizione breve ma 
      concisa del task;
      \item \textbf{Elenco dei compiti:} selezionare uno degli elenchi di 
      task, creati precedentemente, che hanno come fine ultimo la realizzazione di un 
      documento;
      \item \textbf{Chi è il responsabile?:} inserire l'intestatario del task 
      come primo utente, nel secondo invece inserire il verificatore assegnato;
      \item \textbf{Data di conclusione:} Inserire la data ultima per il 
      completamento del task;
     \item \textbf{Priorità:} inserire, opzionalmente, una priorità al task.
    \end{itemize}
\end{enumerate}


\paragraph{Modifica di un task}
Per modificare un task seguire le seguenti istruzioni:
\begin{itemize}
  \item Selezionare il task da modificare;
  \item Dalla pagina proposta selezionare il campo che si vuole modificare;
  \item Completata la modifica premere il pulsante \textbf{Invio}.
\end{itemize}

\paragraph{Completamente di un task}
Dopo che il verificatore ha appurato che il task soddisfa i requisiti, e quindi è stato redatto secondo le
\textsl{\NdP}, può procedere con il suo completamento seguendo le seguenti istruzioni:
\begin{itemize}
  \item Selezionare il task da segnare come completato;
  \item Nella parte inferiore della pagina proposta selezionare la voce \textbf{Contrassegna come 
  completato}.
\end{itemize}
Dopo questa operazione, il task viene spostato automaticamente nella lista dei task completati. Da qui, in
casi eccezionali, può essere rispostato nei task \textsl{da verificare}. Questi casi eccezionali possono essere:
\begin{itemize}
  \item Il \textsl{\RdP} non ha approvato il documento e quindi deve essere 
  rivisto;
  \item Il \textsl{\Ver}, dopo un secondo controllo sui task a lui assegnati, 
  può accorgersi di errori e/o incompletezze. Deve quindi essere rivisto.
\end{itemize}