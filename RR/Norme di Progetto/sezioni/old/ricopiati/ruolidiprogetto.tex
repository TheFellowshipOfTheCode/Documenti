\section{Ruoli di Progetto}
Durante l'intero sviluppo del progetto didattico ogni componente del gruppo 
dovrà obbligatoriamente cimentarsi in tutti i ruoli elencati di seguito. \\
Inoltre non potrà mai accadere che un membro del gruppo risulti redattore e verificatore di un medesimo documento. 
In questo modo si tende ad evitare il conflitto di interessi che potrebbe sorgere se la responsabilità della stesura 
e della verifica di un documento fosse affidata ad un'unica persona.
Un membro può inoltre ricoprire più ruoli contemporaneamente.

\subsection{\RdP}
Il \textsl{\RdP} è colui che detiene la responsabilità del 
lavoro svolto dall'intero team. Rappresenta inoltre colui che mantiene i 
contatti diretti presso il Fornitore e il Cliente, ovvero gli enti esterni. Più 
in dettaglio, ha responsabilità su:
\begin{itemize}
  \item Pianificazione, coordinamento e controllo generale delle attività;
  \item Gestione delle risorse;
  \item Analisi e gestione dei rischi;
  \item Gestione e approvazione della documentazione;
  \item Contatti con gli enti esterni.
\end{itemize}
Il \textsl{\RdP} redige l'organigramma, si assicura che 
tutte le attività vengano svolte seguendo rigorosamente le \textsl{\NdP}, si 
assicura che vengano rispettati i ruoli assegnati nel \textsl{\PdP} e che non si 
presentino conflitti di interesse tra redattori e verificatori. Ha inoltre 
l'incarico di creare, assegnare ad ogni membro e gestire i singoli task. Redige 
il \textsl{\PdP} e collabora alla stesura del \textsl{\PdQ}. Il \textsl{\RdP} è 
l'unica persona in grado di approvare in modo definitivo un documento.

\subsection{Amministratore}
L'\textsl{\Amm} è il responsabile di tutto ciò che riguarda l'ambiente di 
lavoro. Più in dettaglio, egli si occupa di:
\begin{itemize}
  \item Controllo dell'ambiente di lavoro;
  \item Gestione del versionamento della documentazione tramite l'uso di 
  database;
  \item Controllo delle versioni e delle configurazioni del prodotto;
  \item Risoluzione dei problemi legati alla gestione dei processi
\end{itemize}
L'\textsl{\Amm} redige le \textsl{\NdP} e collabora alla stesura del 
\textsl{\PdP}.

\subsection{Progettista}
Il \textsl{\Prog} è il responsabile di tutto ciò che riguarda la progettazione. 
Più in dettaglio, egli si occupa di:
\begin{itemize}
  \item Produrre una soluzione attuabile, robusta e semplice entro i limiti di 
  tempo stabiliti;
  \item Effettuare scelte progettuali volte a garantire la manutenibilità e la 
  modularità del prodotto software.
\end{itemize}
Il \textsl{\Prog} redige la \textsl{\ST}, la \textsl{\DDP} e la parte 
programmatica del \textsl{\PdQ}.

\subsection{Analista}
L'analista si occupa di tutto ciò che riguarda l'analisi del problema da 
affrontare. Le mansioni principali sono quelle di:
\begin{itemize}
  \item Studiare a fondo e capire le problematiche del prodotto da realizzare;
  \item Produrre una specifica di progetto compresibile per il 
  \textsl{Proponente}, per il \textsl{Committente} e per il 
  \textsl{Progettista}.
\end{itemize}
L'\textsl{\Ana} redige lo \textsl{\SdF}, l'\textsl{\AdR} e parte del 
\textsl{\PdQ}.

\subsection{Verificatore}

Il \textsl{Verificatore} è responsabile di tutto ciò che riguarda l'attività di verifica.
Effettua la verifica dei documenti utilizzando gli strumenti e i metodi proposti nel 
\textsl{\PdQ} e seguendo rigorosamente quanto descritto nelle \textsl{\NdP}.
Egli ha il compito di garantire la conformità rispetto le \textsl{\NdP} dei documenti da lui verificati. 
Si occupa di redigere la sezione del \textsl{\PdQ} che illustra l'esito delle 
verifiche effettuate sui documenti.

\subsection{Programmatore}
Il \textsl{\Progr} si occuperà di implementare le soluzioni del \textsl{\Prog}, è quindi 
responsabile dell'attività di codifica. In dettaglio, i suoi compiti sono:
\begin{itemize}
  \item implementare le soluzioni descritte dal \textsl{\Prog} in maniera 
  rigorosa;
  \item Scrivere il codice rispettando le convenzioni prese nel presente 
  documento;
  \item Implementare i test per il codice scritto da utilizzare per l'attività 
  di verifica.
\end{itemize}
Il \textsl{\Progr} redige il \textsl{\MU} e produce la documentazione del codice.